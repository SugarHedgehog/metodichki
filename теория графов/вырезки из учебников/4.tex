\subsection{Экстремальные графы}

Среди первых результатов в одном из направлений теории графов — теории экстремальных графов (см. Эрдёш) — можно отметить следующую известную теорему Турана . Как обычно, пусть \(|r|\) — наибольшее целое число, не превышающее действительного числа \(r\), а \(\{r\} = r - |r|\) есть наименьшее целое число, не меньшее \(r\).

\begin{quote}
    \textit{1) Доказательство существования чисел \(r(m, n)\) для любых натуральных \(m\) и \(n\) см., например, у М. Холла}

    \textit{2) Отметим, что по нашему определению бесконечный граф не является графом. Имеется обзорная статья о бесконечных графах: Нэш-Вильямс.}
\end{quote}

    \textbf{Теорема 2.3.} Наибольшее число рёбер у графов, имеющих \(r\) вершин и не содержащих треугольников, равно \(\lfloor r^2/4 \rfloor\).

\textbf{Доказательство.} Утверждение очевидно для малых значений \(r\). Доказательство по индукции можно дать отдельно для нечетных и для четных \(r\); здесь будет рассмотрен только случай четных значений \(r\). Предположим, что утверждение справедливо для всех четных значений \(r \leq 2n\). Докажем его для \(r = 2n + 2\). Итак, пусть \(G\) — граф с \(p = 2n + 2\) вершинами, не содержащий треугольников. Поскольку граф \(G\) не является вполне несвязным, то в нём существуют две смежные вершины \(u\) и \(v\). В подграфе \(G' = G - \{u, v\}\) имеется \(2n\) вершин и нет треугольников, так что по предположению индукции в графе \(G'\) самое большее \(\lfloor 4n^2/4 \rfloor = n^2\) рёбер. Сколько еще рёбер может быть в графе \(G\)? В графе \(G\) нет такой вершины \(w\), что вершины \(u\) и \(v\) одновременно смежны с \(w\), т. е. вершины \(u\), \(v\) и \(w\) образуют в графе \(G\) треугольник. Таким образом, если вершина \(w\) смежна с \(k\) вершинами графа \(G'\), то вершина \(v\) может быть смежна самое большее с \(2n - k\) вершинами графа \(G'\), и граф \(G\) не больше чем \[
n^2 + k + (2n - k) + 1 = n^2 + 2n + 1 = p^2/4 = \lfloor p^2/4 \rfloor
\]

рёбер.

Для завершения доказательства осталось установить, что для каждого чётного \(p\) существует \((p, p^2/4)\)-граф, не содержащий треугольников. Такой граф можно образовать следующим образом: возьмем два множества \(V_1\) и \(V_2\), каждое из которых имеет \(p/2\) вершин, и соединим каждую вершину из \(V_1\) с каждой вершиной из \(V_2\). Для \(p = 6\) соответствующий граф \(G_1\) приведен на рис. 2.5.

\textit{Двудольный граф} (или \textit{биграф}\footnote{В литературе встречаются и другие термины для этого понятия, например бикроматический граф, простой граф, четный граф, граф паросочетаний.}) $G$ --- это граф, множество вершин $V$ которого можно разбить на два подмножества $V_1$ и $V_2$ таким образом, что каждое ребро графа $G$ соединяет вершины из разных множеств (будем говорить, что ребра графа $G$ соединяют множества $V_1$ и $V_2$). Например, граф, представленный на рис. 2.14, а, можно нарисовать так, как показано на рис. 2.14, б, чтобы подчеркнуть, что этот граф --- двудольный.

Если граф $G$ содержит все ребра, соединяющие множества $V_1$ и $V_2$, то этот граф называется \textit{полным двудольным}. Если при этом в множестве $V_1$ имеется $m$ вершин, а в $V_2$ имеется $n$ вершин, то будем писать $G = K_{m,n}$ --- $K(m, n)$. \textit{Звездой} называется полный двудольный граф $K_{1,m}$. Понятно, что в графе $K_{m,n}$ имеется $mn$ ребер. Поэтому, если $p$ четно, то граф $K(\frac{p}{2}, \frac{p}{2})$ содержит $p^2/4$ ребер; если $p$ нечетно, то граф $K(\lfloor p/2 \rfloor, \lceil p/2 \rceil)$ содержит $\lfloor p/2 \rfloor \cdot \lceil p/2 \rceil = \lfloor p^2/4 \rfloor$ ребер. В каждом из таких графов нет треугольников, что следует из теоремы Кёнига \cite{2, стр. 170}.

\textbf{Теорема 2.4.} Граф является двудольным тогда и только тогда, когда все его простые циклы чётны.

\textit{Доказательство.} Если $G$ --- двудольный граф, то множество его вершин $V$ можно разбить на два подмножества $V_1$ и $V_2$ таким образом, что любое ребро этого графа соединяет некоторую вершину из множества $V_1$ с некоторой вершиной из $V_2$. Поэтому каждый простой цикл $v_1v_2\ldots v_nv_1$ графа содержит вершины из $V_1$, скажем, с нечётными номерами и вершины из $V_2$ с чётными, так что длина $n$ этого цикла является чётным числом.

Чтобы доказать обратное, предположим, не теряя общности, что $G$ --- связный граф (поскольку каждую компоненту графа $G$ можно рассматривать отдельно). Возьмем произвольную вершину $v_1 \in V$ и обозначим через $V_1$ множество, состоящее из $v_1$ и всех вершин, находящихся в графе $G$ на чётном расстоянии от $v_1$; пусть $V_2 = V \setminus V_1$. Так как все простые циклы графа $G$ чётны, то каждое его ребро соединяет множества $V_1$ и $V_2$. В самом деле, если существует ребро $uv$, соединяющее две вершины из множества $V_2$, то объединение геодезических, идущих из вершины $v_1$ к вершине $u$, а также из вершины $v_1$ к вершине $v$, вместе с ребром $uv$ образует цикл нечётной длины; мы пришли к противоречию.

Теорема 2.3 является первым примером решения одной из задач «теории экстремальных графов»: для данного графа $H$ найти $ex(p, H)$ --- наибольшее число рёбер, которое может быть в графе, имеющем $p$ вершин и не содержащем запрещённый подграф $H$. Таким образом, в теореме 2.3 утверждается, что $ex(p, K_3) = \left\lfloor \frac{p^2}{4} \right\rfloor$. Приведем некоторые другие подобные результаты (Эрдёш [3]):

\begin{equation}
ex(p, C_p) = \left\lfloor \frac{1}{2} + \frac{p(p-1)}{2} \right\rfloor
\end{equation}

\begin{equation}
ex(p, K_{4-x}) = \left\lfloor \frac{p^2}{4} \right\rfloor
\end{equation}

\begin{equation}
ex(p, K_{3, x} - x) = \left\lfloor \frac{p^2}{4} \right\rfloor
\end{equation}

Туран \cite{1} обобщил доказанную им теорему 2.3, определив значения функции $ex(p, K_n)$ для всех $n \leq p$:

\[
ex(p, K_n) = \frac{(n-2)(p^2 - r^2)}{2(n-1)} + \binom{r}{2},
\]

где $p \equiv r \pmod{(n-1)}$ и $0 \leq r < n-1$. Другое доказательство этого результата см. у Моцкина и Штрауса \cite{11}.

Известно также, что каждый $(2n, n^2+1)$-граф содержит $n$ треугольников, каждый $(p, 3p-5)$-граф содержит два простых цикла, не имеющих общих ребер (для $p \geq 6$), и каждый $(3n, 3n^2+1)$-граф содержит $n^2$ простых циклов длины 4.
