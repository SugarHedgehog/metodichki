\subsection{Ядро графа}
Пусть $G = (X, \Gamma)$ — конечный или бесконечный граф. Множество $S \subseteq X$ называется \textit{ядром} графа, если $S$ устойчиво как внутренне, так и внешне, т.е. если

\begin{equation}
x \in S \Rightarrow \Gamma x \cap S = \varnothing,
\end{equation}

\begin{equation}
x \notin S \Rightarrow \Gamma x \cap S \neq \varnothing.
\end{equation}

Из условия (1) следует, что ядро $S$ не содержит петель. Из условия (2) — что $S$ содержит все такие вершины $x$, для которых $\Gamma x = \varnothing$. Пустое множество $\varnothing$ не может быть ядром.

\textbf{Теорема 1} \\
Если $S$ — ядро графа $(X, \Gamma)$, то множество $S$ — максимальное в семействе $\mathfrak{S}$ внутренне устойчивых множеств, т.е.
\[
A \in \mathfrak{S}, \, A \supseteq S \Rightarrow A = S
\]

\textbf{Теорема 2} \\
В симметрическом графе без петель каждое максимальное множество семейства $\mathfrak{S}$ внутренне устойчивых множеств представляет собой ядро.

\textbf{Следствие} \\
Симметрический граф без петель обладает ядром.

\textbf{Характеристическая функция} \\
Функция $\varphi_S(x)$ множества $S$ определяется как:

\[
\varphi_S(x) = 
\begin{cases} 
1, & \text{при } x \in S \\ 
0, & \text{при } x \notin S 
\end{cases}
\]

\textbf{Теорема 3} \\
Для того чтобы множество $S$ было ядром, необходимо и достаточно чтобы для характеристической функции $\varphi_S(x)$ выполнялось соотношение

\[
\varphi_S(x) = 1 - \max_{y \in \Gamma x} \varphi_S(y)
\]

\textbf{Теорема 4} \\
Прогрессивно конечный граф обладает ядром.

\textbf{Теорема Ричардсона} \\
Конечный граф, не содержащий контуров нечетной длины, обладает ядром.
