\documentclass[twocolumn]{article}
\usepackage{dashbox}
\setlength{\columnsep}{40pt}
\usepackage[T2A]{fontenc}
\usepackage[utf8]{inputenc}
\usepackage[english,russian]{babel}
\usepackage{graphicx}
\graphicspath{{pictures/}}
\DeclareGraphicsExtensions{.pdf,.png,.jpg}

\linespread{1.15}

\usepackage{../egetask}
\usepackage{../egetask-math-11-2022}

\def\examyear{2023}
\usepackage[colorlinks,linkcolor=blue]{hyperref}

\begin{document}



\cleardoublepage
\def\examvart{Вариант 17.1}
\normalsize

\begin{center}
	\textbf{
		Единый государственный экзамен\\по МАТЕМАТИКЕ\\Профильный уровень\\ \qquad \\ Инструкция по выполнению работы
	}
\end{center}


\par \qquad Экзаменационная работа состоит из двух частей, включающих в себя 18 заданий. Часть 1 содержит 11 заданий с кратким ответом базового и повышенного уровней сложности. Часть 2 содержит 7 заданий с развёрнутым ответом повышенного и высокого уровней сложности.
\par \qquad На выполнение экзаменационной работы по математике отводится 3 часа 55 минут (235 минут).
\par \qquad Ответы к заданиям 1—11 записываются по приведённому ниже \underline {образцу} в виде целого числа или конечной десятичной дроби. Числа запишите в поля ответов в тексте работы, а затем перенесите их в бланк ответов №1.
%%\includegraphics[width=0.98\linewidth]{obrazec}
\par \qquad При выполнении заданий 12—18 требуется записать полное решение и ответ в бланке ответов №2.
\par \qquad  Все бланки ЕГЭ заполняются яркими чёрными чернилами. Допускается использование гелевой или капиллярной ручки.
\par \qquad При выполнении заданий можно пользоваться черновиком. \textbf{Записи в черновике, а также в тексте контрольных измерительных материалов не учитываются при оценивании работы.}
\par \qquad  Баллы, полученные Вами за выполненные задания, суммируются. Постарайтесь выполнить как можно больше заданий и набрать наибольшее количество баллов.
\par \qquad После завершения работы проверьте, что ответ на каждое задание в бланках ответов №1 и №2 записан под правильным номером.
\begin{center}
	\textit{\textbf{Желаем успеха!}}\\ \qquad \\\textbf{ Справочные материалы} \\
$\sin^2 \alpha + \cos^2 \alpha = 1$ \\
$\sin 2\alpha=2\sin \alpha \cdot \cos \alpha$ \\
$\cos 2\alpha=\cos^2 \alpha-\sin^2 \alpha$ \\
$\sin (\alpha+\beta)=\sin \alpha \cdot \cos \beta+\cos \alpha \cdot \sin\beta$ \\
$\cos (\alpha+\beta)=\cos \alpha \cdot \cos \beta-\sin\alpha \cdot \sin\beta$
\end{center}

\startpartone
\large




\begin{taskBN}{1}
К окружности, вписанной в треугольник $TLU$, проведены три касательные. Периметры отсеченных треугольников равны 35, 64, 69. Найдите периметр треугольника $TLU$.
\end{taskBN}

\begin{taskBN}{2}
Площадь сечения, проходящего через середины четырёх рёбер правильного тетраэдра, равна 1156. Найдите ребро правильного тетраэдра.
\end{taskBN}

\begin{taskBN}{3}
В сборнике билетов по математике всего 40 билетов, в 25 из них встречается вопрос по неравенствам. Найдите вероятность того, что в случайно выбранном на экзамене билете школьнику достанется вопрос по неравенствам.
\end{taskBN}

\begin{taskBN}{4}
Если гроссмейстер Г. играет белыми, то он выигрывает у гроссмейстера П. с вероятностью 0,32. Если Г. играет черными, то Г. выигрывает у П. с вероятностью 0,91. Гроссмейстеры Г. и П. играют две партии, причем во второй партии меняют цвет фигур. Найдите вероятность того, что Г. проиграет оба раза.
\end{taskBN}

\begin{taskBN}{5}
Найдите корень уравнения $$-\log_{19}(83-5x)=-\log_{19}78$$
\end{taskBN}

\begin{taskBN}{6}
Найдите значение выражения $$ \log_{0,5}16 $$
\end{taskBN}

\begin{taskBN}{7}
Прямая $y=3x-1$ является касательной к графику функции $y=-3x^{2}+bx-4$. Найдите $b$, зная, что оно меньше 4.
\end{taskBN}

\begin{taskBN}{8}
После дождя уровень воды в колодце может повыситься. Мальчик измеряет время $t$ падения небольших камешков в колодец и рассчитывает расстояние до воды по формуле  $h=5t^2$, где $h$ — расстояние в метрах, $t$ — время падения в секундах. До дождя время падения камешков составляло 0,5 с. На сколько должен подняться уровень воды после дождя, чтобы измеряемое время изменилось на 0,1 с? Ответ выразите в метрах.
\end{taskBN}

\begin{taskBN}{9}
Теплоход проходит по течению реки до пункта назначения и после стоянки возвращается в пункт отправления. Сколько часов длится стоянка, если в пункт отправления теплоход возвращается через 59 часов после отплытия из него, при этом скорость течения составляет 4 км/ч? Скорость теплохода в неподвижной воде равна 6 км/ч, а расстояние от пункта отправления до пункта назначения равно 75 км. 
\end{taskBN}

\begin{taskBN}{10}
\addpictoright[0.4\linewidth]{images/45861336670663n0}На рисунке изображён график функции $f(x)=a\tg x+b$. Найдите $b$.\vspace{2.5cm}
\end{taskBN}

\begin{taskBN}{11}
Определите наибольшее значение функции $y = -2x^{\frac{3}{2}}+27x$ на луче $\left[81;\infty \right)$
\end{taskBN}




\cleardoublepage
\def\examvart{Вариант 17.2}
\normalsize

\begin{center}
	\textbf{
		Единый государственный экзамен\\по МАТЕМАТИКЕ\\Профильный уровень\\ \qquad \\ Инструкция по выполнению работы
	}
\end{center}


\par \qquad Экзаменационная работа состоит из двух частей, включающих в себя 18 заданий. Часть 1 содержит 11 заданий с кратким ответом базового и повышенного уровней сложности. Часть 2 содержит 7 заданий с развёрнутым ответом повышенного и высокого уровней сложности.
\par \qquad На выполнение экзаменационной работы по математике отводится 3 часа 55 минут (235 минут).
\par \qquad Ответы к заданиям 1—11 записываются по приведённому ниже \underline {образцу} в виде целого числа или конечной десятичной дроби. Числа запишите в поля ответов в тексте работы, а затем перенесите их в бланк ответов №1.
%%\includegraphics[width=0.98\linewidth]{obrazec}
\par \qquad При выполнении заданий 12—18 требуется записать полное решение и ответ в бланке ответов №2.
\par \qquad  Все бланки ЕГЭ заполняются яркими чёрными чернилами. Допускается использование гелевой или капиллярной ручки.
\par \qquad При выполнении заданий можно пользоваться черновиком. \textbf{Записи в черновике, а также в тексте контрольных измерительных материалов не учитываются при оценивании работы.}
\par \qquad  Баллы, полученные Вами за выполненные задания, суммируются. Постарайтесь выполнить как можно больше заданий и набрать наибольшее количество баллов.
\par \qquad После завершения работы проверьте, что ответ на каждое задание в бланках ответов №1 и №2 записан под правильным номером.
\begin{center}
	\textit{\textbf{Желаем успеха!}}\\ \qquad \\\textbf{ Справочные материалы} \\
$\sin^2 \alpha + \cos^2 \alpha = 1$ \\
$\sin 2\alpha=2\sin \alpha \cdot \cos \alpha$ \\
$\cos 2\alpha=\cos^2 \alpha-\sin^2 \alpha$ \\
$\sin (\alpha+\beta)=\sin \alpha \cdot \cos \beta+\cos \alpha \cdot \sin\beta$ \\
$\cos (\alpha+\beta)=\cos \alpha \cdot \cos \beta-\sin\alpha \cdot \sin\beta$
\end{center}

\startpartone
\large




\begin{taskBN}{1}
В треугольнике $GIA$ угол $I$ равен $90^\circ$.Сколько составляет  $AI$, если $\ctg^2{G}=\frac{1}{9}$? Известно, что  $AG=\sqrt{490}$. 
\end{taskBN}

\begin{taskBN}{2}
\addpictoright[0.4\linewidth]{images/409402045151619n0}Eсли ребро куба уменьшить на 3, то площадь поверхности уменьшится на 270. Найдите площадь поверхности исходного куба.\vspace{2.5cm}
\end{taskBN}

\begin{taskBN}{3}
На вступительном испытании по английскому языку 408 участников разместили в семи аудиториях. В первых шести удалось разместить по 17 человек, оставшихся перевели в запасную аудиторию в другом корпусе. Найдите вероятность того, что случайно выбранный участник писал <b>не</b> в запасной аудитории.
\end{taskBN}

\begin{taskBN}{4}
Вероятность того, что в случайный момент времени температура тела здорового человека окажется ниже чем 36.8 °С, равна 0.9. Найдите вероятность того, что в случайный момент времени у здорового человека температура окажется 36.8 °С или выше.
\end{taskBN}

\begin{taskBN}{5}
Найдите наибольший неположительный корень уравнения $$\frac{\tg\frac{\pi(-5x -20)}{6}}{-1}=1$$
\end{taskBN}

\begin{taskBN}{6}
Найдите значение выражения $$8\cdot {3}^{\log_{3}{3}} $$
\end{taskBN}

\begin{taskBN}{7}
\addpictoright[0.4\linewidth]{images/8305079155672357n0}На рисунке изображен график функции $y = f(x)$, определенной на интервале $(0; 5)$. Найдите корень уравнения $f'(x)=0$.\vspace{2.5cm}
\end{taskBN}

\begin{taskBN}{8}
По закону Ома для полной цепи сила тока, измеряемая в амперах, равна $I=\frac{\varepsilon}{R+r}$, где $\varepsilon$ — ЭДС источника (в вольтах), $r= 3{,}2 $ Ом — его внутреннее сопротивление, $R$ — сопротивление цепи (в омах). При каком наименьшем сопротивлении цепи сила тока будет составлять не более $20\%$ от силы тока короткого замыкания $I_{\mbox{кз}}=\frac{\varepsilon}{r}$? (Ответ выразите в омах.)
\end{taskBN}

\begin{taskBN}{9}
Два грузовика одновременно отправились в 120-километровый пробег. Первый ехал со скоростью, на 8 км/ч большей, чем скорость второго, и прибыл к финишу на 4 часа раньше второго. Найти скорость грузовика, пришедшего к финишу первым. Ответ дайте в км/ч.
\end{taskBN}

\begin{taskBN}{10}
\addpictoright[0.4\linewidth]{images/658324198090342n0}На рисунке изображены графики функций $f(x)=3x^{2}-3x$ и $g(x)=ax^{2} +bx+c$, которые пересекаются в точках $A$ и $B$. Найдите ординату точки $B$.\vspace{2.5cm}
\end{taskBN}

\begin{taskBN}{11}
Вычислите наибольшее значение функции $y = 54x^{2}+x^{3}+972x-14$ на отрезке $\left[-14;15 \right]$
\end{taskBN}




\cleardoublepage
\def\examvart{Вариант 17.3}
\normalsize

\begin{center}
	\textbf{
		Единый государственный экзамен\\по МАТЕМАТИКЕ\\Профильный уровень\\ \qquad \\ Инструкция по выполнению работы
	}
\end{center}


\par \qquad Экзаменационная работа состоит из двух частей, включающих в себя 18 заданий. Часть 1 содержит 11 заданий с кратким ответом базового и повышенного уровней сложности. Часть 2 содержит 7 заданий с развёрнутым ответом повышенного и высокого уровней сложности.
\par \qquad На выполнение экзаменационной работы по математике отводится 3 часа 55 минут (235 минут).
\par \qquad Ответы к заданиям 1—11 записываются по приведённому ниже \underline {образцу} в виде целого числа или конечной десятичной дроби. Числа запишите в поля ответов в тексте работы, а затем перенесите их в бланк ответов №1.
%%\includegraphics[width=0.98\linewidth]{obrazec}
\par \qquad При выполнении заданий 12—18 требуется записать полное решение и ответ в бланке ответов №2.
\par \qquad  Все бланки ЕГЭ заполняются яркими чёрными чернилами. Допускается использование гелевой или капиллярной ручки.
\par \qquad При выполнении заданий можно пользоваться черновиком. \textbf{Записи в черновике, а также в тексте контрольных измерительных материалов не учитываются при оценивании работы.}
\par \qquad  Баллы, полученные Вами за выполненные задания, суммируются. Постарайтесь выполнить как можно больше заданий и набрать наибольшее количество баллов.
\par \qquad После завершения работы проверьте, что ответ на каждое задание в бланках ответов №1 и №2 записан под правильным номером.
\begin{center}
	\textit{\textbf{Желаем успеха!}}\\ \qquad \\\textbf{ Справочные материалы} \\
$\sin^2 \alpha + \cos^2 \alpha = 1$ \\
$\sin 2\alpha=2\sin \alpha \cdot \cos \alpha$ \\
$\cos 2\alpha=\cos^2 \alpha-\sin^2 \alpha$ \\
$\sin (\alpha+\beta)=\sin \alpha \cdot \cos \beta+\cos \alpha \cdot \sin\beta$ \\
$\cos (\alpha+\beta)=\cos \alpha \cdot \cos \beta-\sin\alpha \cdot \sin\beta$
\end{center}

\startpartone
\large




\begin{taskBN}{1}
Найдите центральный угол $KJG$, если он на 57° больше вписанного угла $KUG$, опирающегося на ту же дугу. Ответ дайте в градусах.
\end{taskBN}

\begin{taskBN}{2}
Ребро правильного тетраэдра равно 68. Найдите площадь сечения, проходящего через середины четырёх рёбер правильного тетраэдра.
\end{taskBN}

\begin{taskBN}{3}
Перед началом первого тура чемпионата по бадминтону участниц разбивают на игровые пары случайным образом с помощью жребия. Всего в чемпионате участвует 496 спортсменок, среди которых 100 участниц из Словакии, в том числе Мария. Найдите вероятность того, что в первом туре Мария будет играть с какой-либо спортсменкой не из Словакии.
\end{taskBN}

\begin{taskBN}{4}
Вероятность того, что на экзамене по математике ученик Н. верно решит больше 7 задач, равна 0.58. Вероятность того, что Н. верно решит больше 6 задач, равна 0.7. Найдите вероятность того, что Н. верно решит ровно 7 задач.
\end{taskBN}

\begin{taskBN}{5}
Найдите корень уравнения $$\frac{7}{9}x^2={38}\frac{1}{9}$$ Если корней несколько, в ответе укажите их произведение.
\end{taskBN}

\begin{taskBN}{6}
Найдите значение выражения $$\log_{0,2}{100} - \log_{0,2}{4} $$
\end{taskBN}

\begin{taskBN}{7}
\addpictoright[0.4\linewidth]{images/5336318216237479n0}На рисунке изображен график функции $y = f(x)$, определенной на интервале $(-1; 5)$. Найдите корень уравнения $f'(x)=0$.\vspace{2.5cm}
\end{taskBN}

\begin{taskBN}{8}
Зависимость объёма спроса $q$ (единиц в месяц) на продукцию предприятия-монополиста от цены $p$ (тыс. руб.) задаётся формулой $q=504-18p$. Выручка предприятия за месяц $r$ (в тыс. руб.) вычисляется по формуле $r(p)=q\cdot p$. Определите наибольшую цену $p$, при которой месячная выручка $r(p)$ составит не менее 3366 тыс. руб. Ответ приведите в тыс. руб.
\end{taskBN}

\begin{taskBN}{9}
Из пункта P в пункт Z одновременно выехали два "Запорожца". Второй проехал с постоянной скоростью весь путь. Первый проехал первую половину пути со скоростью, на 7 км/ч меньшей скорости второго, а вторую половину пути — со скоростью 42 км/ч, в результате чего прибыл в пункт Z одновременно co вторым "Запорожцем". Найдите скорость второго "Запорожца", если известно, что она больше 25,4. Ответ дайте в км/ч.
\end{taskBN}

\begin{taskBN}{10}
\addpictoright[0.4\linewidth]{images/60040435586912n0}На рисунке изображены графики функций $f(x)=a\sqrt{x}+c$ и $g(x)=kx+b$, которые пересекаются в точке $A$. Найдите абсциссу точки $A$.\vspace{2.5cm}
\end{taskBN}

\begin{taskBN}{11}
Вычислите наименьшее значение функции $y =34+3x^{2}-3x-x^{3}$ на отрезке $\left[-2;16 \right]$
\end{taskBN}
\end{document}