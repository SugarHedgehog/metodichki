\documentclass[a4paper,12pt]{article}
\usepackage[utf8]{inputenc}
\usepackage[russian]{babel}
\usepackage{graphicx}
\usepackage{hyperref}
\usepackage{enumitem}
\usepackage{geometry}
\geometry{top=10mm, bottom=10mm, left=10mm, right=10mm}
\pagestyle{empty}

\begin{document}

\begin{center}
	\hfill \break
	\large{МИНОБРНАУКИ РОССИИ}\\
	\footnotesize{ФЕДЕРАЛЬНОЕ ГОСУДАРСТВЕННОЕ БЮДЖЕТНОЕ ОБРАЗОВАТЕЛЬНОЕ УЧРЕЖДЕНИЕ}\\
	\footnotesize{ВЫСШЕГО ПРОФЕССИОНАЛЬНОГО ОБРАЗОВАНИЯ}\\
	\small{\textbf{«ВОРОНЕЖСКИЙ ГОСУДАРСТВЕННЫЙ УНИВЕРСИТЕТ»}}\\
	\hfill \break
	\normalsize{Математический факультет}\\
	\hfill \break
	\normalsize{Кафедра теории функций и геометрии}\\
	\hfill\break
	\hfill \break
	\hfill \break
	\hfill \break
	\hfill \break
	\hfill \break
	\hfill \break
	\hfill \break
	\hfill \break
	\large{Проект воспитательного мероприятияпо теме "Профилактика детского дорожно-транспортного травматизма" для 5 класса}\\
	\hfill \break
	\hfill \break
	\hfill \break
\hfill \break
\hfill \break
\hfill \break
\hfill \break
\hfill \break
\hfill \break
\normalsize{
	\begin{tabular}{ccc}
		Выполнила студентка 5 курса & \underline{\hspace{3cm}} &  Суматохина А.С. \\\\
		Руководитель пед. практики & \underline{\hspace{3cm}} & Прядиев В.Л.    \\\\
	\end{tabular}
}\\
\hfill \break
\hfill \break
\hfill \break
\hfill \break
\hfill \break
\hfill \break
\hfill \break
\hfill \break
\hfill \break
\hfill \break
\hfill \break
\hfill \break
\hfill \break
\hfill \break
\hfill \break
\hfill \break
\hfill \break
Воронеж 2024 \end{center}
\newpage

\section*{Цели и задачи мероприятия}
\begin{itemize}
    \item \textbf{Цель:} Повышение осведомленности учащихся о правилах дорожного движения и мерах безопасности на дороге.
    \item \textbf{Задачи:}
    \begin{enumerate}[label=\arabic*]
        \item Ознакомить детей с основными правилами дорожного движения.
        \item Развить навыки безопасного поведения на дороге.
        \item Формировать ответственное отношение к своему поведению в качестве пешеходов и пассажиров.
    \end{enumerate}
\end{itemize}

\section*{Описание мероприятия}
\begin{itemize}
    \item \textbf{Форма проведения:} Интерактивный урок с элементами игры и практических заданий.
    \item \textbf{Продолжительность:} 1,5 часа.
    \item \textbf{Место проведения:} Классная комната, возможно использование школьного двора для практических заданий.
\end{itemize}

\section*{Структура мероприятия}
\begin{enumerate}[label=\arabic*]
    \item \textbf{Введение (10 минут):}
    \begin{itemize}
        \item Приветствие участников.
        \item Краткое обсуждение: "Почему важно знать правила дорожного движения?" (мозговой штурм).
        \item Объяснение целей и задач занятия.
    \end{itemize}
    \item \textbf{Основная часть (60 минут):}
    \begin{enumerate}[label=\roman*]
        \item \textbf{Блок 1: Знакомство с правилами (20 минут):}
        \begin{itemize}
            \item Презентация о правилах дорожного движения (использование слайдов).
            \item Обсуждение знаков дорожного движения, их значений и важности.
        \end{itemize}
        \item \textbf{Блок 2: Интерактивная игра "Дорожные знаки" (20 минут):}
        \begin{itemize}
            \item Разделение класса на группы.
            \item Каждая группа получает карточки с изображениями дорожных знаков и их значениями.
            \item Задание: сопоставить знаки с их описаниями и объяснить их значение.
        \end{itemize}
        \item \textbf{Блок 3: Практическое занятие "Безопасный маршрут" (20 минут):}
        \begin{itemize}
            \item На школьном дворе или в классе создается макет дороги с дорожными знаками.
            \item Учащиеся в парах разыгрывают ситуации: один играет роль пешехода, другой - водителя.
            \item Обсуждение правильных действий в различных дорожных ситуациях.
        \end{itemize}
    \end{enumerate}
    \item \textbf{Заключительная часть (20 минут):}
    \begin{itemize}
        \item Обсуждение итогов мероприятия: что нового узнали, что было интересно.
        \item Проведение викторины по изученным материалам (вопросы о правилах дорожного движения и знаках).
        \item Раздача памяток с основными правилами дорожного движения.
    \end{itemize}
\end{enumerate}

\section*{Ожидаемые результаты}
\begin{itemize}
    \item Повышение уровня знаний учащихся о правилах дорожного движения.
    \item Формирование навыков безопасного поведения на дороге.
    \item Увеличение осведомленности о важности соблюдения правил как пешеходами, так и водителями.
\end{itemize}

\section*{Ресурсы}
\begin{itemize}
    \item Презентация о правилах дорожного движения.
    \item Карточки с изображениями дорожных знаков.
    \item Материалы для создания макета дороги (бумага, краски, мел).
    \item Памятки для раздачи.
\end{itemize}

\section*{Оценка эффективности}
\begin{itemize}
    \item Опрос учащихся в конце мероприятия о том, что они узнали и как изменилось их отношение к правилам дорожного движения.
    \item Наблюдение за поведением учащихся на дороге в течение следующего месяца.
\end{itemize}

\end{document}
