\documentclass[10pt]{article}
\usepackage[russian]{babel}
\usepackage[utf8]{inputenc}
\usepackage[T2A]{fontenc}
\usepackage{amsmath}
\usepackage{amsfonts}
\usepackage{amssymb}
\usepackage[version=4]{mhchem}
\usepackage{stmaryrd}
\usepackage{graphicx}
\usepackage[export]{adjustbox}
\graphicspath{ {./images/} }

\begin{document}

\section*{Конспект: Нелинейный резонанс}

\subsection*{Линейный осциллятор и резонанс}
Линейный осциллятор демонстрирует эффект резонанса, когда частота внешнего воздействия $\omega$ близка к собственной частоте $\omega_{0}$ осциллятора. В этом случае амплитуда колебаний становится большой. Уравнение вынужденных колебаний линейного осциллятора имеет вид:

\begin{equation*}
\ddot{x}+\omega_{0}^{2} x=f \cos \omega t
\end{equation*}

Решение ищется в виде $x=A \cos \omega t$, что приводит к соотношению амплитуды:

\begin{equation*}
|A|=\frac{f}{\left|\omega^{2}-\omega_{0}^{2}\right|}
\end{equation*}

При $\omega \to \omega_{0}$ амплитуда стремится к бесконечности, но с учетом диссипации она ограничена.

\subsection*{Нелинейный осциллятор}
Для нелинейного осциллятора частота свободных колебаний зависит от амплитуды. Например, для кубического осциллятора:

\begin{equation*}
\ddot{x}+\omega_{0}^{2} x+\beta x^{3}=0
\end{equation*}

Частота колебаний сдвигается на $\Delta \omega(A) \approx 3 \beta A^{2} / 8 \omega_{0}$. Для вынужденных колебаний используется модифицированная частота $\omega_{0}+\Delta \omega(A)$:

\begin{equation*}
|A|=\frac{f}{\left|\omega^{2}-\left(\omega_{0}+\Delta \omega\right)^{2}\right|} \approx \frac{f}{\left|\omega^{2}-\omega_{0}^{2}-3 \beta A^{2} \omega_{0}^{2} / 4\right|}
\end{equation*}

\subsection*{Резонансные кривые}
На рис. 14.4 показаны резонансные кривые для линейного и нелинейного осцилляторов. При $\beta>0$ верхняя часть кривой наклонена вправо, при $\beta<0$ — влево. В отличие от линейного случая, амплитуда вынужденных колебаний остается конечной даже при $\omega = \omega_{0}$ из-за неизохронности.

\subsection*{Нелинейный резонанс}
Нелинейный резонанс характеризуется изменением амплитуды и частоты вынужденных колебаний в зависимости от параметров воздействия. Зависимость амплитуды от частоты может быть неоднозначной.

\subsection*{Консервативный осциллятор}
Консервативный осциллятор сохраняет память о начальных условиях. Движение может быть периодическим или квазипериодическим в зависимости от соотношения частот. При наличии малой диссипации устанавливается режим вынужденных колебаний, описываемый резонансными кривыми.

\end{document}