\subsection{Числа Рамсея}

\noindent\textbf{Мотивационная задача:} В любой группе из 6 человек найдутся либо 3 попарно знакомых, либо 3 попарно незнакомых (переформулировка в терминах графов).

\noindent\textbf{Теорема 2.2} (о существовании треугольника): В графе $G$ с 6 вершинами либо $G$, либо $\overline{G}$ содержит треугольник.

\noindent\textbf{Доказательство:} 
Пусть $v$ -- произвольная вершина графа $G$. Среди 5 оставшихся вершин найдутся 3 вершины $u_1, u_2, u_3$, смежные с $v$ в $G$ (иначе они были бы смежны в $\overline{G}$). Если любые две из $u_1, u_2, u_3$ смежны в $G$ -- получаем треугольник с $v$. Если нет -- $u_1, u_2, u_3$ образуют треугольник в $\overline{G}$.

\noindent\textbf{Определение.} \textit{Число Рамсея} $r(m,n)$ (минимальное число вершин, гарантирующее наличие либо $K_m$, либо $K_n$):
\begin{itemize}[noitemsep,topsep=0pt]
\item Симметричность: $r(m,n) = r(n,m)$
\item Верхняя оценка (Эрдёш-Секереш): $r(m,n) \leq \binom{m + n - 2}{m - 1}$
\end{itemize}

\noindent\textbf{Теорема Рамсея} (для бесконечных графов): Каждый бесконечный граф содержит либо $\aleph_0$ попарно смежных вершин, либо $\aleph_0$ попарно несмежных вершин.

\noindent\textbf{Примечание:} Задача нахождения точных значений $r(m,n)$ остаётся открытой. Известные значения приведены в таблице 2.1.