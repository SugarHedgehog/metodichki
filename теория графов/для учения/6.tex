\subsection{Эйлеровы графы}

\noindent\textbf{Эйлеров граф} -- граф, содержащий цикл со всеми вершинами и рёбрами (имеет эйлеров цикл). Обязательно связный.

\noindent\textbf{Теорема 7.1} (критерий эйлеровости). Для связного графа $G$ эквивалентны:
\begin{enumerate}[noitemsep,topsep=0pt]
\item $G$ -- эйлеров граф
\item Все вершины имеют чётную степень
\item Рёбра можно разбить на простые циклы
\end{enumerate}

\noindent\textbf{Доказательство:}\\
(1)$\Rightarrow$(2): В эйлеровом цикле каждое прохождение вершины даёт +2 к её степени. Каждое ребро используется один раз $\Rightarrow$ степени чётны.

\noindent(2)$\Rightarrow$(3): В связном графе с чётными степенями:
\begin{itemize}[noitemsep]
\item Найдём простой цикл $Z$
\item Удалим его рёбра -- получим граф $G_1$ с чётными степенями
\item Повторяем до пустого графа $G_n$
\end{itemize}

\noindent(3)$\Rightarrow$(1): Имея разбиение на циклы:
\begin{itemize}[noitemsep]
\item Берём цикл $Z_1$
\item Находим цикл $Z_2$ с общей вершиной $v$
\item Строим замкнутую цепь из $Z_1$ и $Z_2$
\item Продолжаем до полного эйлерова цикла
\end{itemize}

\noindent\textbf{Следствие 7.1(a).} В связном графе с $2n$ вершинами нечётной степени ($n \geq 1$) рёбра можно разбить на $n$ открытых цепей.

\noindent\textbf{Следствие 7.1(б).} В связном графе с двумя вершинами нечётной степени существует открытая цепь, содержащая все рёбра (начинается и заканчивается в вершинах нечётной степени).