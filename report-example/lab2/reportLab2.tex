\documentclass[a4paper, 14pt]{extarticle}
\usepackage{ifxetex,ifluatex,ifpdf}

\ifluatex
    \usepackage{fontspec}
    \usepackage{polyglossia}
    \setmainfont{CMU Serif}
    \newfontfamily{\cyrillicfont}{CMU Serif}
    \setsansfont{CMU Sans Serif}
    \newfontfamily{\cyrillicfontsf}{CMU Sans Serif}
    \setmonofont{CMU Typewriter Text}
    \newfontfamily{\cyrillicfonttt}{CMU Typewriter Text}
    \setdefaultlanguage{russian}
    \else  
    \ifpdf
        \usepackage[english, russian]{babel}
        \usepackage[T2A]{fontenc}
        \else
        \ifxetex
        	\usepackage{fontspec}
        	\defaultfontfeatures{Ligatures={TeX}}
        	\setmainfont{Noto Serif}
        	\setsansfont{Noto Sans}
        	\setmonofont{Noto Sans Mono}
        \fi
    \fi
\fi

\usepackage[left=1cm,right=1cm,top=2cm,bottom=2cm]{geometry}
\usepackage{amsmath}


\begin{document}
\thispagestyle{empty}
\begin{center}
	\hfill \break
	\large{МИНОБРНАУКИ РОССИИ}\\
	\footnotesize{ФЕДЕРАЛЬНОЕ ГОСУДАРСТВЕННОЕ БЮДЖЕТНОЕ ОБРАЗОВАТЕЛЬНОЕ УЧРЕЖДЕНИЕ}\\
	\footnotesize{ВЫСШЕГО ПРОФЕССИОНАЛЬНОГО ОБРАЗОВАНИЯ}\\
	\small{\textbf{«ВОРОНЕЖСКИЙ ГОСУДАРСТВЕННЫЙ УНИВЕРСИТЕТ»}}\\

	\hfill \break
	\hfill \break
	\hfill \break
	\hfill \break
	\hfill \break
	\hfill \break
	\hfill \break
	\hfill \break
	\hfill \break

	\large\textbf{Лабораторная работа №2}\\
	\large\textbf{Криптоанализ шифров моноалфавитной замены}\\
\end{center}

\hfill \break
\hfill \break
\hfill \break
\hfill \break
\hfill \break
\hfill \break
\hfill \break


\begin{tabular}{p{0.8\textwidth}}
	Студента(ки) 4 курса математического факультета \newline
	\underline{Суматохина А.С.} \\
	\\
	Преподаватель \newline
	\underline{Сидельникова Софья Юрьевна} \\                 \\
\end{tabular}

\hfill \break
\hfill \break
\hfill \break
\hfill \break
\hfill \break
\hfill \break
\hfill \break
\hfill \break
\hfill \break
\hfill \break
\hfill \break
\begin{center}
	Воронеж 2023г
\end{center}
\newpage

\tableofcontents
\newpage

\section{Исходный текст}

Текст №3

ВКДЖЮТОДЧЮЫБМДЩХШЮДЖКФЭЫЧЮХШЭКУДКШФЫГТОБЮЕМШДЫШЯФОТЭОЕДЩХШЮШРБЩУОАДЩХШЮДЖКФЭЫЧЮ
ЦШДЫЯФЮЭОФШЯЫФКБЮШЯКБМГКСЫЕОБОЮШДОШКЕДКРЮЕРЦШВШКЯФОТОБОДДКЮШЯФОТЭОЕДКЮШКАБЫРЕЮШ
СШЮДЖКФЭЫЧЮКДДКЮШРЮРЕОЭОШКДЫШЮЪФЫОЕШЕОЗДЮЬОРВЩХШФКБМШДКШООШФЫРВФИЕЮОШКРКАОДДКШК
ЯЫРДКШЯКРВКБМВЩШКДКШЬФОСЫЕКШЯКБЩЬОДЮОЭШДОРЫДВЧЮКДЮФКСЫДДКЪКШТКРЕЩЯЫШВКШСРОЮШЮДЖ
КФЭЫЧЮЮШСШЕКЭШЬЮРБОШЯФОТЭОЕДКЮШТЫУОШОРБЮШЮДЖКФЭЫЧЮЦШЗФЫДЮЕРЦШСШВКЭЯМХЕОФОШЮБЮШЯ
ФОТДЫГДЫЬОДЫШТБЦШВКЭЯМХЕОФДКЪКШЮРЯКБМГКСЫДЮЦШЩЪФКГИШООШВКДЖЮТОДЧЮЫБМДКРЕЮШЭКЪЩЕ
ШДКРЮЕМШДОВКЭЯМХЕОФДИЮШЮШСККАЙОШДОЕОЗДЮЬОРВЮЮШЗЫФЫВЕОФШЭДКЪЮЭШБХТЦЭШЯФЮЗКТЮЕРЦШ
СИРЕЩЯЫЕМШСШВЫЬОРЕСОШЯКБМГКСЫЕОБОЮШДОШКТДКЮШЫШЧОБКЪКШФЦТЫШРЮРЕОЭШЮДЖКФЭЫЧЮКДДИЗ
ШРОФСЮРКСШОРБЮШТБЦШТКРЕЩЯЫШВШЕЫВЮЭШРЮРЕОЭЫЭШЮРЯКБМГЩХЕРЦШЭДКЪКФЫГКСИОШЯЫФКБЮШЮБ
ЮШЮДЫЦШВКДЖЮТОДЧЮЫБМДЫЦШЮДЖКФЭЫЧЮЦШЕКШДЫСОФДЦВЫШЛЕЮШТЫДДИОШАЩТЩЕШЗФЫДЮЕМРЦШДОШЕ
КБМВКШСШЪКБКСОШДКШЮШСШГЫЯЮРДКЮШВДЮУВОШЮБЮШДЫШБЮРЕВЫЗШАЩЭЫЪЮШВКЕКФИОШЯКБМГКСЫЕОБ
МШЬЫРЕКШКРЕЫСБЦОЕШДЫШФЫАКЬОЭШРЕКБОШЫШЕКШЮШЯКЯФКРЕЩШЕОФЦОЕШЮШТОБКШГТОРМШДОШСШДОК
ФЪЫДЮГКСЫДДКРЕЮШБХТОЮШЫШСШЮГДЫЬЫБМДКЮШДОЯФЮЪКТДКРЕЮШЯЫФКБМДКЮШРЗОЭИШДОСКГЭКУДКШ
ЯКЭДЮЕМШЭДКЪКШФЫГДИЗШЯЫФКБОЮШФОВКЭОДТЫЧЮЮШЯКШЮЗШФОЪЩБЦФДКЮШЯКШСКГЭКУДКРЕЮШЬЫРЕК
ЮШРЭОДОШЕКБМВКШЩРЩЪЩАБЦХЕШЯКБКУОДЮОШГЫРЕЫСБЦЦШЯФЮЭОДЦЕМШДОРБКУДИОШРЗОЭИШЬОФОТКС
ЫДЮЦШЮБЮШСККАЙОШРЕЫФЫЕМРЦШРСОРЕЮШТОБКШВШТСЩЭШЕФОЭШБОЪВКШГЫЯКЭЮДЫОЭИЭШЮШРЕКБМШУО
ШБОЪВКШЩЪЫТИСЫОЭИЭШЯЫФКБЦЭШКЯЮРЫДДИЮШВБЫРРШЩЦГСЮЭИЗШЭОРЕШЭКУДКШДЫГСЫЕМШФЫГЭОЙОД
ЮОЭШВКДЖЮТОДЧЮЫБМДИЗШТЫДДИЗШСШРФОТОШЪТОШЮЭШДОШКАОРЯОЬОДЫШГЫЬЫРЕЩХШЮШДОШЭКУОЕШАИ
ЕМШДОКАЗКТЮЭЫЦШГЫЙЮЕЫШЩЪФКГЫШУОШРКРЕКЮЕШСШЕКЭШЬЕКШЭДКЪЮОШДОШКЕВЫУЩЕРЦШЩГДЫЕМШРО
ВФОЕИШВКЕКФИОШРЫЭЮШЯФКРЦЕРЦШСШФЩВЮШЯКЭЮЭКШЯЫФКБОЮШЗФЫДЦЙЮЗРЦШСШГЫЯЮРДИЗШВДЮУВЫЗ
ШЯКБМГКСЫЕОБОЮШСШЛЕКЕШВБЫРРШЯКЯЫТЫОЕШЯОФОТЫЬЫШВКДЖЮТОДЧЮЫБМДИЗШТЫДДИЗШСШКЕВФИЕК
ЭШСЮТОШСШФЫГЪКСКФОШСШЯЮРМЭОШЯКШРОЕЮШВКЕКФЫЦШТОБЫОЕШСКГЭКУДИЭШЯОФОЗСЫЕШТЫДДИЗШТБ
ЦШЫЕЫВЮШЭКЪЩЕШЮРЯКБМГКСЫЕМРЦШФЫГДИОШЕОЗДЮЬОРВЮОШРФОТРЕСЫШЯКТРБЩНЮСЫДЮОШЮБЮШЯФКР
БЩНЮСЫДЮОШФЫГЪКСКФКСШЯЫРРЮСДКОШЯФКРБЩНЮСЫДЮОШРОЕЮШДКШЮТОЦШКТДЫШКРЩЙОРЕСЮЕМШТКРЕ
ЩЯШВШТЫДДИЭШСШЕКЕШЭКЭОДЕШВКЪТЫШКДЮШДЫЮЭОДООШГЫЙЮЙОДИШРБЩУОАДЫЦШЮДЖКФЭЫЧЮЦШДЫЯФЮ
ЭОФШЯЫФКБЮШЯКБМГКСЫЕОБОЮШДОШКЕДКРЮЕРЦШВШКЯФОТОБОДДКЮШЯФОТЭОЕДКЮШКАБЫРЕЮШСШЮДЖКФ
ЭЫЧЮКДДКЮШРЮРЕОЭОШКДЫШЮЪФЫОЕШЕОЗДЮЬОРВЩХШФКБМ

\subsection{Статистика для исходного текста}

Статистика составлется функцией \texttt{statistics}

\noindent
Всего символов: 2055
\\А = 14
\\Б = 73
\\В = 47
\\Г = 34
\\Д = 152
\\Е = 110
\\Ж = 14
\\З = 26
\\И = 175
\\К = 198
\\Л = 2
\\М = 39
\\Н = 3
\\О = 171
\\П = 0
\\Р = 98
\\С = 60
\\Т = 51
\\У = 16
\\Ф = 83
\\Х = 14
\\Ц = 40
\\Ч = 17
\\Ш = 284
\\Щ = 37
\\Э = 73
\\Ю = 164
\\Я = 60

\textbf{Индекс совпадений: 0.06541867035954875}

Индекс совпадений считается по формуле $\sum^{28}_{i=1} p_i^2$, где $p_i$ - через вероятности появления $i$-го символа. Расчёт производит фнукция \texttt{indexOfMatches}\\


\section{Расшифрованный текст}

конфиденциальную информацию можно разделить на предметную и служебную информация например пароли пользователеи не относится к определеннои предметнои области в информационнои системе она играет техническую роль но ее раскрытие особенно опасно поскольку оно чревато получением несанкционированного доступа ко всеи информации в том числе предметнои даже если информация хранится в компьютере или предназначена для компьютерного использования угрозы ее конфиденциальности могут носить некомпьютерныи и вообще нетехническии характер многим людям приходится выступать в качестве пользователеи не однои а целого ряда систем информационных сервисов если для доступа к таким системам используются многоразовые пароли или иная конфиденциальная информация то наверняка эти данные будут храниться не только в голове но и в записнои книжке или на листках бумаги которые пользователь часто оставляет на рабочем столе а то и попросту теряет и дело здесь не в неорганизованности людеи а в изначальнои непригодности парольнои схемы невозможно помнить много разных паролеи рекомендации по их регулярнои по возможности частои смене только усугубляют положение заставляя применять несложные схемы чередования или вообще стараться свести дело к двум трем легко запоминаемым и столь же легко угадываемым паролям описанныи класс уязвимых мест можно назвать размещением конфиденциальных данных в среде где им не обеспечена зачастую и не может быть необходимая защита угроза же состоит в том что многие не откажутся узнать секреты которые сами просятся в руки помимо паролеи хранящихся в записных книжках пользователеи в этот класс попадает передача конфиденциальных данных в открытом виде в разговоре в письме по сети которая делает возможным перехват данных для атаки могут использоваться разные технические средства подслушивание или прослушивание разговоров пассивное прослушивание сети но идея одна осуществить доступ к данным в тот момент когда они наименее защищены служебная информация например пароли пользователеи не относится к определеннои предметнои области в информационнои системе она играет техническую роль

\subsection{Процесс расшифровки текста}

\begin{enumerate}
    \item Составление статистики для текста
    \item Замена самой частой буквы "Ш"(встретилась 284 раз) на символ пробела
    \item Нахождение односимвольных и двух-символьных слов. Выявление по ним таких букв, как "И", "Н", "Е", "О".
    \item Подбор остальных букв методом подбора
\end{enumerate}

\subsection{Статистика для расшифрованного текста}

Всего символов: 2055
\\А = 136
\\Б = 14
\\В = 60
\\Г = 26
\\Д = 51
\\Е = 171
\\Ж = 16
\\З = 34
\\И = 195
\\К = 47
\\Л = 73
\\М = 73
\\Н = 152
\\О = 198
\\П = 60
\\Р = 83
\\С = 98
\\Т = 110
\\У = 37
\\Ф = 14
\\Х = 26
\\Ц = 17
\\Ч = 18
\\Ш = 3
\\Щ = 8
\\Э = 2
\\Ю = 14
\\Я = 40

\textbf{Индекс совпадений: 0.06541867035954875}
\section{Вывод}
Вывод:  текст был зашифрован с помощью метода моноалфавитной замены, так как индекс совпадения одинаков для зашифрованного и дешифрованного текста.
\end{document}