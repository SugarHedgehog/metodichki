\documentclass[10pt]{article}
\usepackage[russian]{babel}
\usepackage[utf8]{inputenc}
\usepackage[T2A]{fontenc}
\usepackage{amsmath}
\usepackage{amsfonts}
\usepackage{amssymb}
\usepackage{graphicx}
\usepackage[export]{adjustbox}
\graphicspath{ {./images/} }

\begin{document}

\title{Параметрический резонанс}
\author{}
\date{}
\maketitle

\section*{Введение}
Параметрический резонанс — это явление, возникающее в колебательных системах при периодическом изменении параметров системы. В отличие от вынужденных колебаний, где внешняя сила непосредственно воздействует на систему, параметрический резонанс связан с изменением внутренних параметров, таких как емкость или индуктивность.

\section*{Основные понятия}
\subsection*{Параметрические колебания}
Параметрические колебания возникают в системах, где параметры, такие как емкость или индуктивность, изменяются во времени. Это изменение может быть вызвано механическим воздействием, например, изменением расстояния между пластинами конденсатора.

\subsection*{Дифференциальное уравнение}
Колебания в системе с переменной емкостью описываются уравнением:
\begin{equation*}
\ddot{q}+\frac{1}{L C(t)} q=0
\end{equation*}
где $q$ — заряд, $L$ — индуктивность, $C(t)$ — емкость, зависящая от времени.

\section*{Параметрический резонанс}
\subsection*{Условия резонанса}
Для эффективного поступления энергии в систему необходимо, чтобы период изменения параметра $T$ был примерно равен половине периода собственных колебаний $T_0$:
\begin{equation*}
T \approx \frac{T_{0}}{2}
\end{equation*}
Это условие отличается от резонансного условия для вынужденных колебаний, где $T \approx T_{0}$.

\subsection*{Модель уравнения Матьё}
Основной моделью для изучения параметрических колебаний является уравнение Матьё:
\begin{equation*}
\ddot{x}+\omega_{0}^{2}(1+f \cos \omega t) x=0
\end{equation*}
Это уравнение описывает линейный осциллятор с гармоническим параметрическим возбуждением.

\section*{Зоны неустойчивости}
На плоскости параметров амплитуда-частота воздействия существуют зоны неустойчивости, которые имеют вид характерных клювов, расположенных в окрестности резонансных частот:
\begin{equation*}
\omega \approx \frac{2 \omega_{0}}{n}
\end{equation*}
где $n$ — порядок резонанса.

\section*{Влияние затухания и нелинейности}
Добавление линейного затухания не стабилизирует неустойчивость, а лишь сужает границы зон. Нелинейные эффекты играют ключевую роль в развитии неустойчивости и определении характеристик установившегося режима колебаний.

\section*{Заключение}
Параметрический резонанс является важным явлением в теории колебаний, отличающимся от вынужденных колебаний. Понимание его механизмов и условий позволяет эффективно управлять колебательными системами и предсказывать их поведение.

\end{document}