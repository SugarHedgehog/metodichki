\documentclass[4apaper]{article}
\usepackage{dashbox}
\usepackage[T2A]{fontenc}
\usepackage[utf8]{inputenc}
\usepackage[english,russian]{babel}
\usepackage{graphicx}
\DeclareGraphicsExtensions{.pdf,.png,.jpg}

\linespread{1.15}

\usepackage{../egetask_ver}

\def\examyear{2023}
\usepackage[colorlinks,linkcolor=blue]{hyperref}

\begin{document}
\begin{taskBN}{1}
\addpictoright[0.2\textwidth]{images/499533942227888n0}Прямоугольный параллелепипед описан около сферы c радиусом $4.5$. Найдите его объём. 
\end{taskBN}

\begin{taskBN}{2}
\addpictoright[0.2\textwidth]{images/957426726370402n0}Цилиндр и конус имеют общие основание и высоту. Объём цилиндра равен $486$. Найдите объём конуса.
\end{taskBN}

\begin{taskBN}{3}
\addpictoright[0.2\textwidth]{images/0287367242237904n0}Цилиндр и конус имеют общие основание и высоту. Объём конуса равен $72$. Найдите объём цилиндра.
\end{taskBN}

\begin{taskBN}{4}
\addpictoright[0.2\textwidth]{images/471459626742512n0}В основании прямой призмы лежит прямоугольный треугольник с катетами $6$ и $9$. Боковые рёбра призмы равны $12$. Найдите объём цилиндра, описанного около этой призмы, делённый на $\pi$.
\end{taskBN}

\begin{taskBN}{5}
\addpictoright[0.2\textwidth]{images/958821756296259n0} Цилиндр, полная площадь поверхности которого равна $384\pi$, описан около шара. Найдите диаметр шара.
\end{taskBN}

\begin{taskBN}{6}
\addpictoright[0.2\textwidth]{images/422844998711115n0}В основании прямой призмы лежит прямоугольный треугольник с катетами $9$ и $4$. Боковые рёбра призмы равны $\frac{3}{\pi}$. Найдите объём цилиндра, описанного около этой призмы.
\end{taskBN}

\begin{taskBN}{7}
\addpictoright[0.2\textwidth]{images/424748950884881n0}Шар вписан в цилиндр. Диаметр шара равен $16$. Найдите объём цилиндра, делённый на $\pi$.
\end{taskBN}

\begin{taskBN}{8}
\addpictoright[0.2\textwidth]{images/75582296656287n0}В основании прямой призмы лежит прямоугольный треугольник с катетами $7$ и $10$. Боковые рёбра призмы равны $6$. Найдите объём цилиндра, описанного около этой призмы, делённый на $\pi$.
\end{taskBN}

\begin{taskBN}{9}
\addpictoright[0.2\textwidth]{images/524389462350228n0}Цилиндр и конус имеют общие основание и высоту. Объём конуса равен $144$. Найдите объём цилиндра.
\end{taskBN}

\begin{taskBN}{10}
\addpictoright[0.2\textwidth]{images/260120824552322n0}Цилиндр и конус имеют общие основание и высоту. Объём конуса равен $36$. Найдите объём цилиндра.
\end{taskBN}

\begin{taskBN}{11}
\addpictoright[0.2\textwidth]{images/362433917659719n0} Цилиндр, диаметр основания которого равен $6$, описан около шара. Найдите диаметр шара.
\end{taskBN}

\begin{taskBN}{12}
\addpictoright[0.2\textwidth]{images/379876012681573n0}Шар вписан в цилиндр. Радиус шара равен $7$. Найдите диаметр основания цилиндра.
\end{taskBN}

\begin{taskBN}{13}
\addpictoright[0.2\textwidth]{images/376998086144989n0}Прямоугольный параллелепипед описан около цилиндра. Объём и сторона основания параллелепипеда равны $3600$ и $20$ соотвественно. Найдите объём цилиндра, делённый на $\pi$.
\end{taskBN}

\begin{taskBN}{14}
\addpictoright[0.2\textwidth]{images/36141631180848n0}Цилиндр и конус имеют общие основание и высоту. Объём конуса равен $288$. Найдите объём цилиндра.
\end{taskBN}

\begin{taskBN}{15}
\addpictoright[0.2\textwidth]{images/879480369514593n0}Цилиндр и конус имеют общие основание и высоту. Объём цилиндра равен $972$. Найдите объём конуса.
\end{taskBN}

\begin{taskBN}{16}
\addpictoright[0.2\textwidth]{images/836968087563067n0} Цилиндр, высота которого равна $12$, описан около шара. Найдите площадь поверхности шара, делённую на $\pi$.
\end{taskBN}

\begin{taskBN}{17}
\addpictoright[0.2\textwidth]{images/6451570622837237n0} Цилиндр, диаметр основания которого равен $18$, описан около шара. Найдите площадь поверхности шара, делённую на $\pi$.
\end{taskBN}

\begin{taskBN}{18}
\addpictoright[0.2\textwidth]{images/754695001101598n0} Цилиндр, объём которого равен $128\pi$, описан около шара. Найдите диаметр шара.
\end{taskBN}

\begin{taskBN}{19}
\addpictoright[0.2\textwidth]{images/987912201834207n0} Цилиндр, площадь боковой поверхности которого равна $64\pi$, описан около шара. Найдите радиус шара.
\end{taskBN}

\begin{taskBN}{20}
\addpictoright[0.2\textwidth]{images/2552261004624132n0}В основании прямой призмы лежит прямоугольный треугольник с катетами $11$ и $7$. Боковые рёбра призмы равны $\frac{12}{\pi}$. Найдите объём цилиндра, описанного около этой призмы.
\end{taskBN}

\begin{taskBN}{21}
\addpictoright[0.2\textwidth]{images/3424085198502225n0} Цилиндр, площадь боковой поверхности которого равна $36\pi$, описан около шара. Найдите объём шара, делённый на $\pi$.
\end{taskBN}

\begin{taskBN}{22}
\addpictoright[0.2\textwidth]{images/7944525529683544n0}В основании прямой призмы лежит прямоугольный треугольник с катетами $11$ и $8$. Боковые рёбра призмы равны $\frac{6}{\pi}$. Найдите объём цилиндра, описанного около этой призмы.
\end{taskBN}

\begin{taskBN}{23}
\addpictoright[0.2\textwidth]{images/0912955234420738n0}В основании прямой призмы лежит прямоугольный треугольник с катетами $5$ и $11$. Боковые рёбра призмы равны $3$. Найдите объём цилиндра, описанного около этой призмы, делённый на $\pi$.
\end{taskBN}

\begin{taskBN}{24}
\addpictoright[0.2\textwidth]{images/97502931327676n0}Цилиндр и конус имеют общие основание и высоту. Объём цилиндра равен $216$. Найдите объём конуса.
\end{taskBN}

\begin{taskBN}{25}
\addpictoright[0.2\textwidth]{images/205439499906203n0} Цилиндр, объём которого равен $686\pi$, описан около шара. Найдите площадь поверхности шара, делённую на $\pi$.
\end{taskBN}

\begin{taskBN}{26}
\addpictoright[0.2\textwidth]{images/663737912605293n0}Шар вписан в цилиндр. Объём шара равен $972\pi$. Найдите диаметр основания цилиндра.
\end{taskBN}

\begin{taskBN}{27}
\addpictoright[0.2\textwidth]{images/8739132275534904n0}В основании прямой призмы лежит прямоугольный треугольник с катетами $10$ и $10$. Боковые рёбра призмы равны $\frac{12}{\pi}$. Найдите объём цилиндра, описанного около этой призмы.
\end{taskBN}

\begin{taskBN}{28}
\addpictoright[0.2\textwidth]{images/667680112054083n0}В основании прямой призмы лежит прямоугольный треугольник с катетами $3$ и $5$. Боковые рёбра призмы равны $\frac{3}{\pi}$. Найдите объём цилиндра, описанного около этой призмы.
\end{taskBN}

\begin{taskBN}{29}
\addpictoright[0.2\textwidth]{images/075086936880944n0}Конус вписан в шар (см. рисунок). Центр сферы совпадает с центром основания конуса. Площадь боковой поверхности конуса равна $9\sqrt{2}\pi$. Найдите площадь поверхности шара, делённую на $\pi$.
\end{taskBN}

\begin{taskBN}{30}
\addpictoright[0.2\textwidth]{images/423284009148858n0}Шар вписан в цилиндр. Радиус шара равен $2$. Найдите объём цилиндра, делённый на $\pi$.
\end{taskBN}

\begin{taskBN}{31}
\addpictoright[0.2\textwidth]{images/262417896623822n0}Цилиндр и конус имеют общие основание и высоту. Объём цилиндра равен $972$. Найдите объём конуса.
\end{taskBN}

\begin{taskBN}{32}
\addpictoright[0.2\textwidth]{images/435881246910676n0}В куб с квадратом диагонали $12$ вписан шар. Найдите площадь поверхности этого шара, делённую на $\pi$.
\end{taskBN}

\begin{taskBN}{33}
\addpictoright[0.2\textwidth]{images/9778964597308n0}Цилиндр и конус имеют общие основание и высоту. Объём цилиндра равен $108$. Найдите объём конуса.
\end{taskBN}

\begin{taskBN}{34}
\addpictoright[0.2\textwidth]{images/220961022178319n0}Прямоугольный параллелепипед описан около цилиндра. Диагональ основания и объём параллелепипеда равны $8\sqrt{2}$ и $448$ соотвественно. Найдите высоту цилиндра.
\end{taskBN}

\begin{taskBN}{35}
\addpictoright[0.2\textwidth]{images/421243612234619n0}Цилиндр и конус имеют общие основание и высоту. Объём конуса равен $162$. Найдите объём цилиндра.
\end{taskBN}

\begin{taskBN}{36}
\addpictoright[0.2\textwidth]{images/9148215405121374n0}Цилиндр и конус имеют общие основание и высоту. Объём цилиндра равен $729$. Найдите объём конуса.
\end{taskBN}

\begin{taskBN}{37}
\addpictoright[0.2\textwidth]{images/836491744949327n0}В основании прямой призмы лежит прямоугольный треугольник с катетами $8$ и $2$. Боковые рёбра призмы равны $\frac{12}{\pi}$. Найдите объём цилиндра, описанного около этой призмы.
\end{taskBN}

\begin{taskBN}{38}
\addpictoright[0.2\textwidth]{images/952147094596325n0}Прямоугольный параллелепипед описан около цилиндра. Диагональ и диагональ основания параллелепипеда равны $2\sqrt{33}$ и $4\sqrt{2}$ соотвественно. Найдите высоту цилиндра.
\end{taskBN}

\begin{taskBN}{39}
\addpictoright[0.2\textwidth]{images/806458528232735n0}В основании прямой призмы лежит прямоугольный треугольник с катетами $7$ и $9$. Боковые рёбра призмы равны $9$. Найдите объём цилиндра, описанного около этой призмы, делённый на $\pi$.
\end{taskBN}

\begin{taskBN}{40}
\addpictoright[0.2\textwidth]{images/187425001801285n0}Шар вписан в цилиндр. Диаметр шара равен $20$. Найдите площадь боковой поверхности цилиндра, делённую на $\pi$.
\end{taskBN}

\begin{taskBN}{41}
\addpictoright[0.2\textwidth]{images/822395059458989n0} Цилиндр, площадь боковой поверхности которого равна $196\pi$, описан около шара. Найдите площадь поверхности шара, делённую на $\pi$.
\end{taskBN}

\begin{taskBN}{42}
\addpictoright[0.2\textwidth]{images/025292332336376n0}Шар вписан в цилиндр. Площадь поверхности шара равна $100\pi$. Найдите площадь боковой поверхности цилиндра, делённую на $\pi$.
\end{taskBN}

\begin{taskBN}{43}
\addpictoright[0.2\textwidth]{images/250279896897191n0}В основании прямой призмы лежит прямоугольный треугольник с катетами $6$ и $6$. Боковые рёбра призмы равны $\frac{3}{\pi}$. Найдите площадь боковой поверхности цилиндра, описанного около этой призмы. Ответ умножьте на $\sqrt{2}$.
\end{taskBN}

\begin{taskBN}{44}
\addpictoright[0.2\textwidth]{images/914857181128965n0} Цилиндр, площадь боковой поверхности которого равна $100\pi$, описан около шара. Найдите площадь поверхности шара, делённую на $\pi$.
\end{taskBN}

\begin{taskBN}{45}
\addpictoright[0.2\textwidth]{images/62244324658861205n0}Шар вписан в цилиндр. Площадь поверхности шара равна $36\pi$. Найдите диаметр основания цилиндра.
\end{taskBN}

\begin{taskBN}{46}
\addpictoright[0.2\textwidth]{images/8631507049008644n0}Прямоугольный параллелепипед описан около сферы c радиусом $4$. Найдите его ребро. 
\end{taskBN}

\begin{taskBN}{47}
\addpictoright[0.2\textwidth]{images/494497268093766n0}Шар вписан в цилиндр. Радиус шара равен $4$. Найдите диаметр основания цилиндра.
\end{taskBN}

\begin{taskBN}{48}
\addpictoright[0.2\textwidth]{images/206087644132368n0}В основании прямой призмы лежит прямоугольный треугольник с катетами $4$ и $3$. Боковые рёбра призмы равны $\frac{3}{\pi}$. Найдите площадь боковой поверхности цилиндра, описанного около этой призмы.
\end{taskBN}

\begin{taskBN}{49}
\addpictoright[0.2\textwidth]{images/961382927456963n0}Цилиндр и конус имеют общие основание и высоту. Объём конуса равен $576$. Найдите объём цилиндра.
\end{taskBN}

\begin{taskBN}{50}
\addpictoright[0.2\textwidth]{images/33338973503586n0}Цилиндр и конус имеют общие основание и высоту. Объём конуса равен $72$. Найдите объём цилиндра.
\end{taskBN}

\begin{taskBN}{51}
\addpictoright[0.2\textwidth]{images/530580744396916n0}Прямоугольный параллелепипед описан около цилиндра. Объём и сторона основания параллелепипеда равны $2304$ и $16$ соотвественно. Найдите объём цилиндра, делённый на $\pi$.
\end{taskBN}

\begin{taskBN}{52}
\addpictoright[0.2\textwidth]{images/94987214967867n0}Около конуса описана сфера (сфера содержит окружность основания конуса и его вершину).  Радиус основания конуса равен радиусу шара. Радиус основания конуса равен $9$. Найдите площадь поверхности шара, делённую на $\pi$.
\end{taskBN}

\begin{taskBN}{53}
\addpictoright[0.2\textwidth]{images/961294642618861n0}Шар вписан в цилиндр. Площадь поверхности шара равна $196\pi$. Найдите площадь боковой поверхности цилиндра, делённую на $\pi$.
\end{taskBN}

\begin{taskBN}{54}
\addpictoright[0.2\textwidth]{images/256641018574553n0}Цилиндр и конус имеют общие основание и высоту. Объём цилиндра равен $81$. Найдите объём конуса.
\end{taskBN}

\begin{taskBN}{55}
\addpictoright[0.2\textwidth]{images/0067271420104325n0}Прямоугольный параллелепипед описан около цилиндра. Объём и диагональ основания параллелепипеда равны $144$ и $4\sqrt{2}$ соотвественно. Найдите высоту цилиндра.
\end{taskBN}

\begin{taskBN}{56}
\addpictoright[0.2\textwidth]{images/882047498399538n0}В основании прямой призмы лежит прямоугольный треугольник с катетами $3$ и $3$. Боковые рёбра призмы равны $9$. Найдите объём цилиндра, описанного около этой призмы, делённый на $\pi$.
\end{taskBN}

\begin{taskBN}{57}
\addpictoright[0.2\textwidth]{images/05064424662222n0}Прямоугольный параллелепипед описан около цилиндра. Сторона основания и диагональ одной из боковых сторон параллелепипеда равны $12$ и $2\sqrt{37}$ соотвественно. Найдите высоту цилиндра.
\end{taskBN}

\begin{taskBN}{58}
\addpictoright[0.2\textwidth]{images/158033209173048n0}Шар вписан в цилиндр. Площадь поверхности шара равна $100\pi$. Найдите площадь боковой поверхности цилиндра, делённую на $\pi$.
\end{taskBN}

\begin{taskBN}{59}
\addpictoright[0.2\textwidth]{images/37992414747208n0}В основании прямой призмы лежит прямоугольный треугольник с катетами $11$ и $5$. Боковые рёбра призмы равны $\frac{6}{\pi}$. Найдите объём цилиндра, описанного около этой призмы.
\end{taskBN}

\begin{taskBN}{60}
\addpictoright[0.2\textwidth]{images/824880087629206n0}Прямоугольный параллелепипед описан около цилиндра. Диагональ основания и объём параллелепипеда равны $6\sqrt{2}$ и $288$ соотвественно. Найдите объём цилиндра, делённый на $\pi$.
\end{taskBN}

\begin{taskBN}{61}
\addpictoright[0.2\textwidth]{images/219993970338027n0}Прямоугольный параллелепипед описан около цилиндра. Диагональ и сторона основания параллелепипеда равны $12$ и $8$ соотвественно. Найдите высоту цилиндра.
\end{taskBN}

\begin{taskBN}{62}
\addpictoright[0.2\textwidth]{images/604194105841461n0}Цилиндр и конус имеют общие основание и высоту. Объём конуса равен $576$. Найдите объём цилиндра.
\end{taskBN}

\begin{taskBN}{63}
\addpictoright[0.2\textwidth]{images/5763897423988147n0}Шар вписан в цилиндр. Диаметр шара равен $6$. Найдите площадь боковой поверхности цилиндра, делённую на $\pi$.
\end{taskBN}

\begin{taskBN}{64}
\addpictoright[0.2\textwidth]{images/9367553562049078n0} Цилиндр, объём которого равен $54\pi$, описан около шара. Найдите объём шара, делённый на $\pi$.
\end{taskBN}

\begin{taskBN}{65}
\addpictoright[0.2\textwidth]{images/4879314773773153n0}Прямоугольный параллелепипед описан около цилиндра. Сторона основания и объём параллелепипеда равны $18$ и $2916$ соотвественно. Найдите объём цилиндра, делённый на $\pi$.
\end{taskBN}

\begin{taskBN}{66}
\addpictoright[0.2\textwidth]{images/65663579809921n0}Прямоугольный параллелепипед описан около цилиндра. Сторона основания и объём параллелепипеда равны $12$ и $288$ соотвественно. Найдите высоту цилиндра.
\end{taskBN}

\begin{taskBN}{67}
\addpictoright[0.2\textwidth]{images/17068820765271n0}В основании прямой призмы лежит прямоугольный треугольник с катетами $3$ и $3$. Боковые рёбра призмы равны $12$. Найдите объём цилиндра, описанного около этой призмы, делённый на $\pi$.
\end{taskBN}

\begin{taskBN}{68}
\addpictoright[0.2\textwidth]{images/248558758615302n0}В основании прямой призмы лежит прямоугольный треугольник с катетами $11$ и $4$. Боковые рёбра призмы равны $\frac{3}{\pi}$. Найдите объём цилиндра, описанного около этой призмы.
\end{taskBN}

\begin{taskBN}{69}
\addpictoright[0.2\textwidth]{images/703178256536376n0}Прямоугольный параллелепипед описан около цилиндра. Диагональ основания и диагональ одной из боковых сторон параллелепипеда равны $20\sqrt{2}$ и $5\sqrt{17}$ соотвественно. Найдите высоту цилиндра.
\end{taskBN}

\begin{taskBN}{70}
\addpictoright[0.2\textwidth]{images/046185436532824n0}Прямоугольный параллелепипед описан около сферы c объёмом $36$$\pi$. Найдите его площадь поверхности. 
\end{taskBN}

\begin{taskBN}{71}
\addpictoright[0.2\textwidth]{images/1077064875097793n0}Прямоугольный параллелепипед описан около цилиндра. Диагональ и сторона основания параллелепипеда равны $\sqrt{809}$ и $20$ соотвественно. Найдите объём цилиндра, делённый на $\pi$.
\end{taskBN}

\begin{taskBN}{72}
\addpictoright[0.2\textwidth]{images/207219470343552n0}Цилиндр и конус имеют общие основание и высоту. Объём конуса равен $72$. Найдите объём цилиндра.
\end{taskBN}

\begin{taskBN}{73}
\addpictoright[0.2\textwidth]{images/066190688131688n0}В основании прямой призмы лежит прямоугольный треугольник с катетами $11$ и $10$. Боковые рёбра призмы равны $\frac{6}{\pi}$. Найдите объём цилиндра, описанного около этой призмы.
\end{taskBN}

\begin{taskBN}{74}
\addpictoright[0.2\textwidth]{images/2209281907800282n0}В основании прямой призмы лежит прямоугольный треугольник с катетами $9$ и $10$. Боковые рёбра призмы равны $9$. Найдите объём цилиндра, описанного около этой призмы, делённый на $\pi$.
\end{taskBN}

\begin{taskBN}{75}
\addpictoright[0.2\textwidth]{images/785539973205645n0}Прямоугольный параллелепипед описан около цилиндра. Объём и диагональ основания параллелепипеда равны $588$ и $14\sqrt{2}$ соотвественно. Найдите объём цилиндра, делённый на $\pi$.
\end{taskBN}

\begin{taskBN}{76}
\addpictoright[0.2\textwidth]{images/736529308160236n0}Цилиндр и конус имеют общие основание и высоту. Объём цилиндра равен $108$. Найдите объём конуса.
\end{taskBN}

\begin{taskBN}{77}
\addpictoright[0.2\textwidth]{images/553350519843923n0}В куб с квадратом диагонали $75$ вписан шар. Найдите радиус этого шара.
\end{taskBN}

\begin{taskBN}{78}
\addpictoright[0.2\textwidth]{images/552489342508858n0} Цилиндр, высота которого равна $16$, описан около шара. Найдите площадь поверхности шара, делённую на $\pi$.
\end{taskBN}

\begin{taskBN}{79}
\addpictoright[0.2\textwidth]{images/1273959873336115n0}Около конуса описана сфера (сфера содержит окружность основания конуса и его вершину). Центр сферы совпадает с центром основания конуса. Радиус основания конуса равен $10$. Найдите площадь поверхности шара, делённую на $\pi$.
\end{taskBN}

\begin{taskBN}{80}
\addpictoright[0.2\textwidth]{images/7910532860911517n0}Цилиндр и конус имеют общие основание и высоту. Объём цилиндра равен $216$. Найдите объём конуса.
\end{taskBN}

\begin{taskBN}{81}
\addpictoright[0.2\textwidth]{images/675485646620307n0}В основании прямой призмы лежит прямоугольный треугольник с катетами $5$ и $4$. Боковые рёбра призмы равны $\frac{9}{\pi}$. Найдите объём цилиндра, описанного около этой призмы.
\end{taskBN}

\begin{taskBN}{82}
\addpictoright[0.2\textwidth]{images/341879370612013n0}В основании прямой призмы лежит прямоугольный треугольник с катетами $11$ и $8$. Боковые рёбра призмы равны $\frac{12}{\pi}$. Найдите объём цилиндра, описанного около этой призмы.
\end{taskBN}

\begin{taskBN}{83}
\addpictoright[0.2\textwidth]{images/335583670070856n0}Около конуса описана сфера (сфера содержит окружность основания конуса и его вершину). Центр сферы совпадает с центром основания конуса. Объём конуса равен $243\pi$. Найдите площадь поверхности шара, делённую на $\pi$.
\end{taskBN}

\begin{taskBN}{84}
\addpictoright[0.2\textwidth]{images/103981596453402n0}Около конуса описана сфера (сфера содержит окружность основания конуса и его вершину). Центр сферы совпадает с центром основания конуса. Радиус основания конуса равен $9$. Найдите объём шара, делённый на $\pi$.
\end{taskBN}

\begin{taskBN}{85}
\addpictoright[0.2\textwidth]{images/040105492529355n0}Цилиндр и конус имеют общие основание и высоту. Объём конуса равен $162$. Найдите объём цилиндра.
\end{taskBN}

\begin{taskBN}{86}
\addpictoright[0.2\textwidth]{images/287887411948016n0}Прямоугольный параллелепипед описан около цилиндра. Диагональ основания и объём параллелепипеда равны $10\sqrt{2}$ и $800$ соотвественно. Найдите площадь боковой поверхности цилиндра, делённую на $\pi$.
\end{taskBN}

\begin{taskBN}{87}
\addpictoright[0.2\textwidth]{images/500482737949714n0}Цилиндр и конус имеют общие основание и высоту. Объём цилиндра равен $216$. Найдите объём конуса.
\end{taskBN}

\begin{taskBN}{88}
\addpictoright[0.2\textwidth]{images/507731015771493n0}В основании прямой призмы лежит прямоугольный треугольник с катетами $8$ и $10$. Боковые рёбра призмы равны $12$. Найдите объём цилиндра, описанного около этой призмы, делённый на $\pi$.
\end{taskBN}

\begin{taskBN}{89}
\addpictoright[0.2\textwidth]{images/6933294439029936n0}В основании прямой призмы лежит прямоугольный треугольник с катетами $7$ и $5$. Боковые рёбра призмы равны $9$. Найдите объём цилиндра, описанного около этой призмы, делённый на $\pi$.
\end{taskBN}

\begin{taskBN}{90}
\addpictoright[0.2\textwidth]{images/274993345205573n0}Цилиндр и конус имеют общие основание и высоту. Объём конуса равен $81$. Найдите объём цилиндра.
\end{taskBN}

\begin{taskBN}{91}
\addpictoright[0.2\textwidth]{images/000926804704127n0}Конус вписан в шар (см. рисунок).  Радиус основания конуса равен радиусу шара. Площадь боковой поверхности конуса равна $9\sqrt{2}\pi$. Найдите площадь поверхности шара, делённую на $\pi$.
\end{taskBN}

\begin{taskBN}{92}
\addpictoright[0.2\textwidth]{images/8891318634330676n0} Цилиндр, радиус основания которого равен $2$, описан около шара. Найдите площадь поверхности шара, делённую на $\pi$.
\end{taskBN}

\begin{taskBN}{93}
\addpictoright[0.2\textwidth]{images/2993101762875912n0}Цилиндр и конус имеют общие основание и высоту. Объём цилиндра равен $729$. Найдите объём конуса.
\end{taskBN}

\begin{taskBN}{94}
\addpictoright[0.2\textwidth]{images/3573426692877626n0}Прямоугольный параллелепипед описан около цилиндра. Диагональ основания и объём параллелепипеда равны $10\sqrt{2}$ и $800$ соотвественно. Найдите высоту цилиндра.
\end{taskBN}

\begin{taskBN}{95}
\addpictoright[0.2\textwidth]{images/682299517774835n0}Прямоугольный параллелепипед описан около цилиндра. Диагональ основания и объём параллелепипеда равны $18\sqrt{2}$ и $2592$ соотвественно. Найдите высоту цилиндра.
\end{taskBN}

\begin{taskBN}{96}
\addpictoright[0.2\textwidth]{images/355113679977123n0}В основании прямой призмы лежит прямоугольный треугольник с катетами $10$ и $5$. Боковые рёбра призмы равны $\frac{6}{\pi}$. Найдите объём цилиндра, описанного около этой призмы.
\end{taskBN}

\begin{taskBN}{97}
\addpictoright[0.2\textwidth]{images/891523529284784n0}Прямоугольный параллелепипед описан около цилиндра. Диагональ основания и диагональ параллелепипеда равны $16\sqrt{2}$ и $\sqrt{537}$ соотвественно. Найдите высоту цилиндра.
\end{taskBN}

\begin{taskBN}{98}
\addpictoright[0.2\textwidth]{images/002432005983508n0}Прямоугольный параллелепипед описан около цилиндра. Диагональ основания и диагональ одной из боковых сторон параллелепипеда равны $12\sqrt{2}$ и $15$ соотвественно. Найдите высоту цилиндра.
\end{taskBN}

\begin{taskBN}{99}
\addpictoright[0.2\textwidth]{images/259523720841094n0}Прямоугольный параллелепипед описан около цилиндра. Сторона основания и диагональ одной из боковых сторон параллелепипеда равны $18$ и $6\sqrt{10}$ соотвественно. Найдите высоту цилиндра.
\end{taskBN}

\begin{taskBN}{100}
\addpictoright[0.2\textwidth]{images/260982964811337n0}В основании прямой призмы лежит прямоугольный треугольник с катетами $3$ и $10$. Боковые рёбра призмы равны $12$. Найдите объём цилиндра, описанного около этой призмы, делённый на $\pi$.
\end{taskBN}

\begin{taskBN}{101}
\addpictoright[0.2\textwidth]{images/4430137651587864n0}Прямоугольный параллелепипед описан около сферы c площадью поверхности $25$$\pi$. Найдите его диагональ. Ответ поделите на $\sqrt{3}$.
\end{taskBN}

\begin{taskBN}{102}
\addpictoright[0.2\textwidth]{images/7050891353006428n0}Шар вписан в цилиндр. Объём шара равен $972\pi$. Найдите объём цилиндра, делённый на $\pi$.
\end{taskBN}

\begin{taskBN}{103}
\addpictoright[0.2\textwidth]{images/083286918330953n0}В основании прямой призмы лежит прямоугольный треугольник с катетами $4$ и $11$. Боковые рёбра призмы равны $\frac{9}{\pi}$. Найдите объём цилиндра, описанного около этой призмы.
\end{taskBN}

\begin{taskBN}{104}
\addpictoright[0.2\textwidth]{images/491486206949154n0}Прямоугольный параллелепипед описан около цилиндра. Сторона основания и диагональ параллелепипеда равны $12$ и $2\sqrt{73}$ соотвественно. Найдите объём цилиндра, делённый на $\pi$.
\end{taskBN}

\begin{taskBN}{105}
\addpictoright[0.2\textwidth]{images/732128036372199n0}Цилиндр и конус имеют общие основание и высоту. Объём цилиндра равен $1296$. Найдите объём конуса.
\end{taskBN}

\begin{taskBN}{106}
\addpictoright[0.2\textwidth]{images/335374388764791n0}В куб с квадратом диагонали $243$ вписан шар. Найдите радиус этого шара.
\end{taskBN}

\begin{taskBN}{107}
\addpictoright[0.2\textwidth]{images/208404091469458n0}В основании прямой призмы лежит прямоугольный треугольник с катетами $10$ и $9$. Боковые рёбра призмы равны $6$. Найдите объём цилиндра, описанного около этой призмы, делённый на $\pi$.
\end{taskBN}

\begin{taskBN}{108}
\addpictoright[0.2\textwidth]{images/5999244917496687n0}Цилиндр и конус имеют общие основание и высоту. Объём цилиндра равен $108$. Найдите объём конуса.
\end{taskBN}

\begin{taskBN}{109}
\addpictoright[0.2\textwidth]{images/299125290251983n0}Около конуса описана сфера (сфера содержит окружность основания конуса и его вершину). Центр сферы совпадает с центром основания конуса. Радиус основания конуса равен $3$. Найдите объём шара, делённый на $\pi$.
\end{taskBN}

\begin{taskBN}{110}
\addpictoright[0.2\textwidth]{images/162554293358001n0}Цилиндр и конус имеют общие основание и высоту. Объём конуса равен $72$. Найдите объём цилиндра.
\end{taskBN}

\begin{taskBN}{111}
\addpictoright[0.2\textwidth]{images/84096957879551n0}Цилиндр и конус имеют общие основание и высоту. Объём цилиндра равен $108$. Найдите объём конуса.
\end{taskBN}

\begin{taskBN}{112}
\addpictoright[0.2\textwidth]{images/92265645180753n0}В основании прямой призмы лежит прямоугольный треугольник с катетами $5$ и $8$. Боковые рёбра призмы равны $\frac{12}{\pi}$. Найдите объём цилиндра, описанного около этой призмы.
\end{taskBN}

\begin{taskBN}{113}
\addpictoright[0.2\textwidth]{images/171977625791432n0}Прямоугольный параллелепипед описан около цилиндра. Площадь полной поверхности и сторона основания параллелепипеда равны $528$ и $12$ соотвественно. Найдите высоту цилиндра.
\end{taskBN}

\begin{taskBN}{114}
\addpictoright[0.2\textwidth]{images/6124649376003435n0}В основании прямой призмы лежит прямоугольный треугольник с катетами $6$ и $3$. Боковые рёбра призмы равны $\frac{9}{\pi}$. Найдите объём цилиндра, описанного около этой призмы.
\end{taskBN}

\begin{taskBN}{115}
\addpictoright[0.2\textwidth]{images/5318190675878216n0}В основании прямой призмы лежит прямоугольный треугольник с катетами $3$ и $9$. Боковые рёбра призмы равны $\frac{12}{\pi}$. Найдите объём цилиндра, описанного около этой призмы.
\end{taskBN}

\begin{taskBN}{116}
\addpictoright[0.2\textwidth]{images/475816458533397n0}В основании прямой призмы лежит прямоугольный треугольник с катетами $2$ и $9$. Боковые рёбра призмы равны $6$. Найдите объём цилиндра, описанного около этой призмы, делённый на $\pi$.
\end{taskBN}

\begin{taskBN}{117}
\addpictoright[0.2\textwidth]{images/516896087509886n0}Цилиндр и конус имеют общие основание и высоту. Объём цилиндра равен $864$. Найдите объём конуса.
\end{taskBN}

\begin{taskBN}{118}
\addpictoright[0.2\textwidth]{images/220869553671283n0}Прямоугольный параллелепипед описан около сферы c диаметром $7$. Найдите его площадь поверхности. 
\end{taskBN}

\begin{taskBN}{119}
\addpictoright[0.2\textwidth]{images/9971259428124832n0}Прямоугольный параллелепипед описан около цилиндра. Объём и диагональ основания параллелепипеда равны $324$ и $6\sqrt{2}$ соотвественно. Найдите высоту цилиндра.
\end{taskBN}

\begin{taskBN}{120}
\addpictoright[0.2\textwidth]{images/8926250859847265n0}В основании прямой призмы лежит прямоугольный треугольник с катетами $7$ и $7$. Боковые рёбра призмы равны $\frac{9}{\pi}$. Найдите площадь боковой поверхности цилиндра, описанного около этой призмы. Ответ умножьте на $\sqrt{2}$.
\end{taskBN}

\begin{taskBN}{121}
\addpictoright[0.2\textwidth]{images/016520823520185n0}Цилиндр и конус имеют общие основание и высоту. Объём конуса равен $108$. Найдите объём цилиндра.
\end{taskBN}

\begin{taskBN}{122}
\addpictoright[0.2\textwidth]{images/994569651839179n0}Цилиндр и конус имеют общие основание и высоту. Объём конуса равен $243$. Найдите объём цилиндра.
\end{taskBN}

\begin{taskBN}{123}
\addpictoright[0.2\textwidth]{images/770676024748428n0}Цилиндр и конус имеют общие основание и высоту. Объём конуса равен $144$. Найдите объём цилиндра.
\end{taskBN}

\begin{taskBN}{124}
\addpictoright[0.2\textwidth]{images/448918926388282n0}В основании прямой призмы лежит прямоугольный треугольник с катетами $2$ и $4$. Боковые рёбра призмы равны $9$. Найдите объём цилиндра, описанного около этой призмы, делённый на $\pi$.
\end{taskBN}

\begin{taskBN}{125}
\addpictoright[0.2\textwidth]{images/888463720485467n0}В основании прямой призмы лежит прямоугольный треугольник с катетами $4$ и $11$. Боковые рёбра призмы равны $9$. Найдите объём цилиндра, описанного около этой призмы, делённый на $\pi$.
\end{taskBN}

\begin{taskBN}{126}
\addpictoright[0.2\textwidth]{images/2632369052572034n0}Цилиндр и конус имеют общие основание и высоту. Объём конуса равен $324$. Найдите объём цилиндра.
\end{taskBN}

\begin{taskBN}{127}
\addpictoright[0.2\textwidth]{images/483309649819688n0}Цилиндр и конус имеют общие основание и высоту. Объём цилиндра равен $216$. Найдите объём конуса.
\end{taskBN}

\begin{taskBN}{128}
\addpictoright[0.2\textwidth]{images/35321970715692874n0} Цилиндр, площадь боковой поверхности которого равна $324\pi$, описан около шара. Найдите площадь поверхности шара, делённую на $\pi$.
\end{taskBN}

\begin{taskBN}{129}
\addpictoright[0.2\textwidth]{images/70691825515684n0}В основании прямой призмы лежит прямоугольный треугольник с катетами $6$ и $6$. Боковые рёбра призмы равны $\frac{9}{\pi}$. Найдите объём цилиндра, описанного около этой призмы.
\end{taskBN}

\begin{taskBN}{130}
\addpictoright[0.2\textwidth]{images/9732557023466155n0}Прямоугольный параллелепипед описан около цилиндра. Площадь полной поверхности и сторона основания параллелепипеда равны $672$ и $14$ соотвественно. Найдите объём цилиндра, делённый на $\pi$.
\end{taskBN}

\begin{taskBN}{131}
\addpictoright[0.2\textwidth]{images/5578593841722763n0}В куб с площадью поверхности $600$ вписан шар. Найдите площадь поверхности этого шара, делённую на $\pi$.
\end{taskBN}

\begin{taskBN}{132}
\addpictoright[0.2\textwidth]{images/151146549717146n0}Прямоугольный параллелепипед описан около цилиндра. Сторона основания и диагональ одной из боковых сторон параллелепипеда равны $12$ и $15$ соотвественно. Найдите объём цилиндра, делённый на $\pi$.
\end{taskBN}

\begin{taskBN}{133}
\addpictoright[0.2\textwidth]{images/498925630065678n0}Около конуса описана сфера (сфера содержит окружность основания конуса и его вершину). Центр сферы совпадает с центром основания конуса. Радиус основания конуса равен $4$. Найдите площадь поверхности шара, делённую на $\pi$.
\end{taskBN}

\begin{taskBN}{134}
\addpictoright[0.2\textwidth]{images/280489036237468n0}В куб с диагональю $2$ вписан шар. Найдите диаметр этого шара.
\end{taskBN}

\begin{taskBN}{135}
\addpictoright[0.2\textwidth]{images/158760641632166n0}Цилиндр и конус имеют общие основание и высоту. Объём конуса равен $18$. Найдите объём цилиндра.
\end{taskBN}

\begin{taskBN}{136}
\addpictoright[0.2\textwidth]{images/12784264051702188n0}Прямоугольный параллелепипед описан около цилиндра. Объём и сторона основания параллелепипеда равны $800$ и $20$ соотвественно. Найдите высоту цилиндра.
\end{taskBN}

\begin{taskBN}{137}
\addpictoright[0.2\textwidth]{images/861948735372765n0} Цилиндр, радиус основания которого равен $6$, описан около шара. Найдите объём шара, делённый на $\pi$.
\end{taskBN}

\begin{taskBN}{138}
\addpictoright[0.2\textwidth]{images/7463490936982231n0}Прямоугольный параллелепипед описан около цилиндра. Диагональ основания и площадь полной поверхности параллелепипеда равны $4\sqrt{2}$ и $144$ соотвественно. Найдите объём цилиндра, делённый на $\pi$.
\end{taskBN}

\begin{taskBN}{139}
\addpictoright[0.2\textwidth]{images/803490865762759n0}Около конуса описана сфера (сфера содержит окружность основания конуса и его вершину).  Радиус основания конуса равен радиусу шара. Объём конуса равен $243\pi$. Найдите объём шара, делённый на $\pi$.
\end{taskBN}

\begin{taskBN}{140}
\addpictoright[0.2\textwidth]{images/440923264414404n0}Цилиндр и конус имеют общие основание и высоту. Объём конуса равен $576$. Найдите объём цилиндра.
\end{taskBN}

\begin{taskBN}{141}
\addpictoright[0.2\textwidth]{images/965912646664711n0}Цилиндр и конус имеют общие основание и высоту. Объём конуса равен $243$. Найдите объём цилиндра.
\end{taskBN}

\begin{taskBN}{142}
\addpictoright[0.2\textwidth]{images/45915448282198n0}Конус вписан в шар (см. рисунок).  Радиус основания конуса равен радиусу шара. Площадь боковой поверхности конуса равна $81\sqrt{2}\pi$. Найдите объём шара, делённый на $\pi$.
\end{taskBN}

\begin{taskBN}{143}
\addpictoright[0.2\textwidth]{images/180895233605828n0}В основании прямой призмы лежит прямоугольный треугольник с катетами $5$ и $7$. Боковые рёбра призмы равны $\frac{6}{\pi}$. Найдите объём цилиндра, описанного около этой призмы.
\end{taskBN}

\begin{taskBN}{144}
\addpictoright[0.2\textwidth]{images/268119371159569n0}Цилиндр и конус имеют общие основание и высоту. Объём цилиндра равен $486$. Найдите объём конуса.
\end{taskBN}

\begin{taskBN}{145}
\addpictoright[0.2\textwidth]{images/62298484015247n0}Прямоугольный параллелепипед описан около цилиндра. Объём и диагональ основания параллелепипеда равны $1568$ и $14\sqrt{2}$ соотвественно. Найдите объём цилиндра, делённый на $\pi$.
\end{taskBN}

\begin{taskBN}{146}
\addpictoright[0.2\textwidth]{images/330263387102915n0}Цилиндр и конус имеют общие основание и высоту. Объём цилиндра равен $864$. Найдите объём конуса.
\end{taskBN}

\begin{taskBN}{147}
\addpictoright[0.2\textwidth]{images/206204426280507n0}Шар вписан в цилиндр. Площадь поверхности шара равна $256\pi$. Найдите полную площадь поверхности цилиндра, делённую на $\pi$.
\end{taskBN}

\begin{taskBN}{148}
\addpictoright[0.2\textwidth]{images/999227669046001n0}Около конуса описана сфера (сфера содержит окружность основания конуса и его вершину).  Радиус основания конуса равен радиусу шара. Объём конуса равен $9\pi$. Найдите объём шара, делённый на $\pi$.
\end{taskBN}

\begin{taskBN}{149}
\addpictoright[0.2\textwidth]{images/872490273768376n0}Цилиндр и конус имеют общие основание и высоту. Объём цилиндра равен $108$. Найдите объём конуса.
\end{taskBN}

\begin{taskBN}{150}
\addpictoright[0.2\textwidth]{images/7916660938863005n0} Цилиндр, объём которого равен $54\pi$, описан около шара. Найдите объём шара, делённый на $\pi$.
\end{taskBN}

\begin{taskBN}{151}
\addpictoright[0.2\textwidth]{images/840825884205685n0}В основании прямой призмы лежит прямоугольный треугольник с катетами $3$ и $10$. Боковые рёбра призмы равны $9$. Найдите объём цилиндра, описанного около этой призмы, делённый на $\pi$.
\end{taskBN}

\begin{taskBN}{152}
\addpictoright[0.2\textwidth]{images/112986209079614n0}В основании прямой призмы лежит прямоугольный треугольник с катетами $12$ и $8$. Боковые рёбра призмы равны $3$. Найдите объём цилиндра, описанного около этой призмы, делённый на $\pi$.
\end{taskBN}

\begin{taskBN}{153}
\addpictoright[0.2\textwidth]{images/314589632316125n0}Цилиндр и конус имеют общие основание и высоту. Объём конуса равен $243$. Найдите объём цилиндра.
\end{taskBN}

\begin{taskBN}{154}
\addpictoright[0.2\textwidth]{images/962910553353193n0}Шар вписан в цилиндр. Диаметр шара равен $18$. Найдите площадь боковой поверхности цилиндра, делённую на $\pi$.
\end{taskBN}

\begin{taskBN}{155}
\addpictoright[0.2\textwidth]{images/8945508413675456n0}Цилиндр и конус имеют общие основание и высоту. Объём цилиндра равен $486$. Найдите объём конуса.
\end{taskBN}

\begin{taskBN}{156}
\addpictoright[0.2\textwidth]{images/77439382141637n0}Около конуса описана сфера (сфера содержит окружность основания конуса и его вершину). Центр сферы совпадает с центром основания конуса. Радиус основания конуса равен $7$. Найдите площадь поверхности шара, делённую на $\pi$.
\end{taskBN}

\begin{taskBN}{157}
\addpictoright[0.2\textwidth]{images/791993423442116n0}Около конуса описана сфера (сфера содержит окружность основания конуса и его вершину). Центр сферы совпадает с центром основания конуса. Радиус основания конуса равен $2$. Найдите площадь поверхности шара, делённую на $\pi$.
\end{taskBN}

\begin{taskBN}{158}
\addpictoright[0.2\textwidth]{images/664628731935915n0}Конус вписан в шар (см. рисунок). Центр сферы совпадает с центром основания конуса. Площадь боковой поверхности конуса равна $36\sqrt{2}\pi$. Найдите объём шара, делённый на $\pi$.
\end{taskBN}

\begin{taskBN}{159}
\addpictoright[0.2\textwidth]{images/188064049681266n0}Шар вписан в цилиндр. Радиус шара равен $6$. Найдите объём цилиндра, делённый на $\pi$.
\end{taskBN}

\begin{taskBN}{160}
\addpictoright[0.2\textwidth]{images/36300223188291625n0}Цилиндр и конус имеют общие основание и высоту. Объём конуса равен $288$. Найдите объём цилиндра.
\end{taskBN}

\begin{taskBN}{161}
\addpictoright[0.2\textwidth]{images/5465422216339515n0}В основании прямой призмы лежит прямоугольный треугольник с катетами $8$ и $3$. Боковые рёбра призмы равны $\frac{9}{\pi}$. Найдите объём цилиндра, описанного около этой призмы.
\end{taskBN}

\begin{taskBN}{162}
\addpictoright[0.2\textwidth]{images/7936905887551804n0}Шар вписан в цилиндр. Площадь поверхности шара равна $16\pi$. Найдите объём цилиндра, делённый на $\pi$.
\end{taskBN}

\begin{taskBN}{163}
\addpictoright[0.2\textwidth]{images/0433547756525035n0}Шар вписан в цилиндр. Площадь поверхности шара равна $256\pi$. Найдите высоту цилиндра.
\end{taskBN}

\begin{taskBN}{164}
\addpictoright[0.2\textwidth]{images/1964532541481725n0}Прямоугольный параллелепипед описан около цилиндра. Диагональ основания и объём параллелепипеда равны $18\sqrt{2}$ и $2592$ соотвественно. Найдите высоту цилиндра.
\end{taskBN}

\begin{taskBN}{165}
\addpictoright[0.2\textwidth]{images/063413118385927n0}Около конуса описана сфера (сфера содержит окружность основания конуса и его вершину).  Радиус основания конуса равен радиусу шара. Площадь боковой поверхности конуса равна $36\sqrt{2}\pi$. Найдите площадь поверхности шара, делённую на $\pi$.
\end{taskBN}

\begin{taskBN}{166}
\addpictoright[0.2\textwidth]{images/539167568047797n0}В основании прямой призмы лежит прямоугольный треугольник с катетами $11$ и $4$. Боковые рёбра призмы равны $6$. Найдите объём цилиндра, описанного около этой призмы, делённый на $\pi$.
\end{taskBN}

\begin{taskBN}{167}
\addpictoright[0.2\textwidth]{images/606902967165571n0}Шар вписан в цилиндр. Площадь поверхности шара равна $196\pi$. Найдите площадь боковой поверхности цилиндра, делённую на $\pi$.
\end{taskBN}

\begin{taskBN}{168}
\addpictoright[0.2\textwidth]{images/858335879261788n0} Цилиндр, радиус основания которого равен $6$, описан около шара. Найдите площадь поверхности шара, делённую на $\pi$.
\end{taskBN}

\begin{taskBN}{169}
\addpictoright[0.2\textwidth]{images/9848708474808314n0}Цилиндр и конус имеют общие основание и высоту. Объём цилиндра равен $864$. Найдите объём конуса.
\end{taskBN}

\begin{taskBN}{170}
\addpictoright[0.2\textwidth]{images/490152858867609n0}Шар вписан в цилиндр. Диаметр шара равен $16$. Найдите объём цилиндра, делённый на $\pi$.
\end{taskBN}

\begin{taskBN}{171}
\addpictoright[0.2\textwidth]{images/1550079328959908n0}Цилиндр и конус имеют общие основание и высоту. Объём цилиндра равен $864$. Найдите объём конуса.
\end{taskBN}

\begin{taskBN}{172}
\addpictoright[0.2\textwidth]{images/774777072314689n0}Прямоугольный параллелепипед описан около цилиндра. Диагональ одной из боковых сторон и сторона основания параллелепипеда равны $5\sqrt{5}$ и $10$ соотвественно. Найдите высоту цилиндра.
\end{taskBN}

\begin{taskBN}{173}
\addpictoright[0.2\textwidth]{images/415048101158837n0}В основании прямой призмы лежит прямоугольный треугольник с катетами $11$ и $4$. Боковые рёбра призмы равны $6$. Найдите объём цилиндра, описанного около этой призмы, делённый на $\pi$.
\end{taskBN}

\begin{taskBN}{174}
\addpictoright[0.2\textwidth]{images/05026575122899n0}В основании прямой призмы лежит прямоугольный треугольник с катетами $12$ и $5$. Боковые рёбра призмы равны $\frac{9}{\pi}$. Найдите объём цилиндра, описанного около этой призмы.
\end{taskBN}

\begin{taskBN}{175}
\addpictoright[0.2\textwidth]{images/27596175724176n0}Шар вписан в цилиндр. Площадь поверхности шара равна $100\pi$. Найдите высоту цилиндра.
\end{taskBN}

\begin{taskBN}{176}
\addpictoright[0.2\textwidth]{images/4751999073825426n0}Цилиндр и конус имеют общие основание и высоту. Объём цилиндра равен $486$. Найдите объём конуса.
\end{taskBN}

\begin{taskBN}{177}
\addpictoright[0.2\textwidth]{images/26767789503344463n0}Прямоугольный параллелепипед описан около цилиндра. Площадь полной поверхности и сторона основания параллелепипеда равны $1200$ и $20$ соотвественно. Найдите объём цилиндра, делённый на $\pi$.
\end{taskBN}

\begin{taskBN}{178}
\addpictoright[0.2\textwidth]{images/384428775072644n0}Прямоугольный параллелепипед описан около цилиндра. Диагональ одной из боковых сторон и диагональ основания параллелепипеда равны $2\sqrt{97}$ и $18\sqrt{2}$ соотвественно. Найдите высоту цилиндра.
\end{taskBN}

\begin{taskBN}{179}
\addpictoright[0.2\textwidth]{images/039745664689835n0}Прямоугольный параллелепипед описан около цилиндра. Площадь полной поверхности и диагональ основания параллелепипеда равны $264$ и $6\sqrt{2}$ соотвественно. Найдите высоту цилиндра.
\end{taskBN}

\begin{taskBN}{180}
\addpictoright[0.2\textwidth]{images/5991235197351514n0}Шар вписан в цилиндр. Площадь поверхности шара равна $196\pi$. Найдите площадь боковой поверхности цилиндра, делённую на $\pi$.
\end{taskBN}

\begin{taskBN}{181}
\addpictoright[0.2\textwidth]{images/483384747164314n0}Прямоугольный параллелепипед описан около цилиндра. Площадь полной поверхности и диагональ основания параллелепипеда равны $112$ и $4\sqrt{2}$ соотвественно. Найдите объём цилиндра, делённый на $\pi$.
\end{taskBN}

\begin{taskBN}{182}
\addpictoright[0.2\textwidth]{images/94185834428697n0}Шар вписан в цилиндр. Радиус шара равен $10$. Найдите площадь боковой поверхности цилиндра, делённую на $\pi$.
\end{taskBN}

\begin{taskBN}{183}
\addpictoright[0.2\textwidth]{images/189948192302466n0}В основании прямой призмы лежит прямоугольный треугольник с катетами $10$ и $5$. Боковые рёбра призмы равны $9$. Найдите объём цилиндра, описанного около этой призмы, делённый на $\pi$.
\end{taskBN}

\begin{taskBN}{184}
\addpictoright[0.2\textwidth]{images/013528467465766n0}Цилиндр и конус имеют общие основание и высоту. Объём цилиндра равен $108$. Найдите объём конуса.
\end{taskBN}

\begin{taskBN}{185}
\addpictoright[0.2\textwidth]{images/773370102262375n0}Цилиндр и конус имеют общие основание и высоту. Объём цилиндра равен $54$. Найдите объём конуса.
\end{taskBN}

\begin{taskBN}{186}
\addpictoright[0.2\textwidth]{images/13419814185368528n0}Цилиндр и конус имеют общие основание и высоту. Объём цилиндра равен $432$. Найдите объём конуса.
\end{taskBN}

\begin{taskBN}{187}
\addpictoright[0.2\textwidth]{images/1186671745697694n0}В основании прямой призмы лежит прямоугольный треугольник с катетами $3$ и $8$. Боковые рёбра призмы равны $\frac{12}{\pi}$. Найдите объём цилиндра, описанного около этой призмы.
\end{taskBN}

\begin{taskBN}{188}
\addpictoright[0.2\textwidth]{images/8060227657825974n0}Цилиндр и конус имеют общие основание и высоту. Объём цилиндра равен $216$. Найдите объём конуса.
\end{taskBN}

\begin{taskBN}{189}
\addpictoright[0.2\textwidth]{images/6566567596079187n0} Цилиндр, полная площадь поверхности которого равна $24\pi$, описан около шара. Найдите радиус шара.
\end{taskBN}

\begin{taskBN}{190}
\addpictoright[0.2\textwidth]{images/07594881145261079n0}Прямоугольный параллелепипед описан около цилиндра. Сторона основания и диагональ одной из боковых сторон параллелепипеда равны $16$ и $\sqrt{265}$ соотвественно. Найдите высоту цилиндра.
\end{taskBN}

\begin{taskBN}{191}
\addpictoright[0.2\textwidth]{images/70931669486121n0}Конус вписан в шар (см. рисунок). Центр сферы совпадает с центром основания конуса. Площадь боковой поверхности конуса равна $25\sqrt{2}\pi$. Найдите площадь поверхности шара, делённую на $\pi$.
\end{taskBN}

\begin{taskBN}{192}
\addpictoright[0.2\textwidth]{images/523554767620716n0}Прямоугольный параллелепипед описан около цилиндра. Площадь полной поверхности и сторона основания параллелепипеда равны $80$ и $4$ соотвественно. Найдите высоту цилиндра.
\end{taskBN}

\begin{taskBN}{193}
\addpictoright[0.2\textwidth]{images/1931214616952286n0}В основании прямой призмы лежит прямоугольный треугольник с катетами $7$ и $7$. Боковые рёбра призмы равны $\frac{12}{\pi}$. Найдите объём цилиндра, описанного около этой призмы.
\end{taskBN}

\begin{taskBN}{194}
\addpictoright[0.2\textwidth]{images/6165359102832333n0}Шар вписан в цилиндр. Радиус шара равен $6$. Найдите площадь боковой поверхности цилиндра, делённую на $\pi$.
\end{taskBN}

\begin{taskBN}{195}
\addpictoright[0.2\textwidth]{images/204045434214778n0}В куб с ребром $6$ вписан шар. Найдите объём этого шара, делённый на $\pi$.
\end{taskBN}

\begin{taskBN}{196}
\addpictoright[0.2\textwidth]{images/368987899315371n0}Прямоугольный параллелепипед описан около сферы c объёмом $121.5$$\pi$. Найдите его диагональ. Ответ поделите на $\sqrt{3}$.
\end{taskBN}

\begin{taskBN}{197}
\addpictoright[0.2\textwidth]{images/1738337093506337n0}Цилиндр и конус имеют общие основание и высоту. Объём конуса равен $72$. Найдите объём цилиндра.
\end{taskBN}

\begin{taskBN}{198}
\addpictoright[0.2\textwidth]{images/375507032931361n0}Конус вписан в шар (см. рисунок).  Радиус основания конуса равен радиусу шара. Объём конуса равен $9\pi$. Найдите площадь поверхности шара, делённую на $\pi$.
\end{taskBN}

\begin{taskBN}{199}
\addpictoright[0.2\textwidth]{images/943863960744875n0}В куб с объёмом $125$ вписан шар. Найдите площадь поверхности этого шара, делённую на $\pi$.
\end{taskBN}

\begin{taskBN}{200}
\addpictoright[0.2\textwidth]{images/184963363406594n0}Цилиндр и конус имеют общие основание и высоту. Объём цилиндра равен $972$. Найдите объём конуса.
\end{taskBN}

\newpage

\begin{tabular}{*{4}l}
\begin{tabular}[t]{|l|l|l|}
\hline
1 & 1 & 729\\
\hline
1 & 2 & 162\\
\hline
1 & 3 & 216\\
\hline
1 & 4 & 351\\
\hline
1 & 5 & 16\\
\hline
1 & 6 & 72,75\\
\hline
1 & 7 & 1024\\
\hline
1 & 8 & 223,5\\
\hline
1 & 9 & 432\\
\hline
1 & 10 & 108\\
\hline
1 & 11 & 6\\
\hline
1 & 12 & 14\\
\hline
1 & 13 & 900\\
\hline
1 & 14 & 864\\
\hline
1 & 15 & 324\\
\hline
1 & 16 & 144\\
\hline
1 & 17 & 324\\
\hline
1 & 18 & 8\\
\hline
1 & 19 & 4\\
\hline
1 & 20 & 510\\
\hline
1 & 21 & 36\\
\hline
1 & 22 & 277,5\\
\hline
1 & 23 & 109,5\\
\hline
1 & 24 & 72\\
\hline
1 & 25 & 196\\
\hline
1 & 26 & 18\\
\hline
1 & 27 & 600\\
\hline
1 & 28 & 25,5\\
\hline
1 & 29 & 36\\
\hline
1 & 30 & 16\\
\hline
1 & 31 & 324\\
\hline
1 & 32 & 4\\
\hline
1 & 33 & 36\\
\hline
1 & 34 & 7\\
\hline
1 & 35 & 486\\
\hline
1 & 36 & 243\\
\hline
1 & 37 & 204\\
\hline
1 & 38 & 10\\
\hline
1 & 39 & 292,5\\
\hline
1 & 40 & 400\\
\hline
1 & 41 & 196\\
\hline
1 & 42 & 100\\
\hline
1 & 43 & 36\\
\hline
1 & 44 & 100\\
\hline
1 & 45 & 6\\
\hline
1 & 46 & 8\\
\hline
1 & 47 & 8\\
\hline
1 & 48 & 15\\
\hline
1 & 49 & 1728\\
\hline
\end{tabular}&\begin{tabular}[t]{|l|l|l|}
\hline
1 & 50 & 216\\
\hline
1 & 51 & 576\\
\hline
1 & 52 & 324\\
\hline
1 & 53 & 196\\
\hline
1 & 54 & 27\\
\hline
1 & 55 & 9\\
\hline
1 & 56 & 40,5\\
\hline
1 & 57 & 2\\
\hline
1 & 58 & 100\\
\hline
1 & 59 & 219\\
\hline
1 & 60 & 72\\
\hline
1 & 61 & 4\\
\hline
1 & 62 & 1728\\
\hline
1 & 63 & 36\\
\hline
1 & 64 & 36\\
\hline
1 & 65 & 729\\
\hline
1 & 66 & 2\\
\hline
1 & 67 & 54\\
\hline
1 & 68 & 102,75\\
\hline
1 & 69 & 5\\
\hline
1 & 70 & 216\\
\hline
1 & 71 & 300\\
\hline
1 & 72 & 216\\
\hline
1 & 73 & 331,5\\
\hline
1 & 74 & 407,25\\
\hline
1 & 75 & 147\\
\hline
1 & 76 & 36\\
\hline
1 & 77 & 2,5\\
\hline
1 & 78 & 256\\
\hline
1 & 79 & 400\\
\hline
1 & 80 & 72\\
\hline
1 & 81 & 92,25\\
\hline
1 & 82 & 555\\
\hline
1 & 83 & 324\\
\hline
1 & 84 & 972\\
\hline
1 & 85 & 486\\
\hline
1 & 86 & 80\\
\hline
1 & 87 & 72\\
\hline
1 & 88 & 492\\
\hline
1 & 89 & 166,5\\
\hline
1 & 90 & 243\\
\hline
1 & 91 & 36\\
\hline
1 & 92 & 16\\
\hline
1 & 93 & 243\\
\hline
1 & 94 & 8\\
\hline
1 & 95 & 8\\
\hline
1 & 96 & 187,5\\
\hline
1 & 97 & 5\\
\hline
1 & 98 & 9\\
\hline
1 & 99 & 6\\
\hline
\end{tabular}&\begin{tabular}[t]{|l|l|l|}
\hline
1 & 100 & 327\\
\hline
1 & 101 & 5\\
\hline
1 & 102 & 1458\\
\hline
1 & 103 & 308,25\\
\hline
1 & 104 & 72\\
\hline
1 & 105 & 432\\
\hline
1 & 106 & 4,5\\
\hline
1 & 107 & 271,5\\
\hline
1 & 108 & 36\\
\hline
1 & 109 & 36\\
\hline
1 & 110 & 216\\
\hline
1 & 111 & 36\\
\hline
1 & 112 & 267\\
\hline
1 & 113 & 5\\
\hline
1 & 114 & 101,25\\
\hline
1 & 115 & 270\\
\hline
1 & 116 & 127,5\\
\hline
1 & 117 & 288\\
\hline
1 & 118 & 294\\
\hline
1 & 119 & 9\\
\hline
1 & 120 & 126\\
\hline
1 & 121 & 324\\
\hline
1 & 122 & 729\\
\hline
1 & 123 & 432\\
\hline
1 & 124 & 45\\
\hline
1 & 125 & 308,25\\
\hline
1 & 126 & 972\\
\hline
1 & 127 & 72\\
\hline
1 & 128 & 324\\
\hline
1 & 129 & 162\\
\hline
1 & 130 & 245\\
\hline
1 & 131 & 100\\
\hline
1 & 132 & 324\\
\hline
1 & 133 & 64\\
\hline
1 & 134 & 2\\
\hline
1 & 135 & 54\\
\hline
1 & 136 & 2\\
\hline
1 & 137 & 288\\
\hline
1 & 138 & 28\\
\hline
1 & 139 & 972\\
\hline
1 & 140 & 1728\\
\hline
1 & 141 & 729\\
\hline
1 & 142 & 972\\
\hline
1 & 143 & 111\\
\hline
1 & 144 & 162\\
\hline
1 & 145 & 392\\
\hline
1 & 146 & 288\\
\hline
1 & 147 & 384\\
\hline
1 & 148 & 36\\
\hline
1 & 149 & 36\\
\hline
\end{tabular}&\begin{tabular}[t]{|l|l|l|}
\hline
1 & 150 & 36\\
\hline
1 & 151 & 245,25\\
\hline
1 & 152 & 156\\
\hline
1 & 153 & 729\\
\hline
1 & 154 & 324\\
\hline
1 & 155 & 162\\
\hline
1 & 156 & 196\\
\hline
1 & 157 & 16\\
\hline
1 & 158 & 288\\
\hline
1 & 159 & 432\\
\hline
1 & 160 & 864\\
\hline
1 & 161 & 164,25\\
\hline
1 & 162 & 16\\
\hline
1 & 163 & 16\\
\hline
1 & 164 & 8\\
\hline
1 & 165 & 144\\
\hline
1 & 166 & 205,5\\
\hline
1 & 167 & 196\\
\hline
1 & 168 & 144\\
\hline
1 & 169 & 288\\
\hline
1 & 170 & 1024\\
\hline
1 & 171 & 288\\
\hline
1 & 172 & 5\\
\hline
1 & 173 & 205,5\\
\hline
1 & 174 & 380,25\\
\hline
1 & 175 & 10\\
\hline
1 & 176 & 162\\
\hline
1 & 177 & 500\\
\hline
1 & 178 & 8\\
\hline
1 & 179 & 8\\
\hline
1 & 180 & 196\\
\hline
1 & 181 & 20\\
\hline
1 & 182 & 400\\
\hline
1 & 183 & 281,25\\
\hline
1 & 184 & 36\\
\hline
1 & 185 & 18\\
\hline
1 & 186 & 144\\
\hline
1 & 187 & 219\\
\hline
1 & 188 & 72\\
\hline
1 & 189 & 2\\
\hline
1 & 190 & 3\\
\hline
1 & 191 & 100\\
\hline
1 & 192 & 3\\
\hline
1 & 193 & 294\\
\hline
1 & 194 & 144\\
\hline
1 & 195 & 36\\
\hline
1 & 196 & 9\\
\hline
1 & 197 & 216\\
\hline
1 & 198 & 36\\
\hline
1 & 199 & 25\\
\hline
1 & 200 & 324\\
\hline
\end{tabular}\end{tabular}

\end{document}