\documentclass[a4paper]{article}
\usepackage{dashbox}
\usepackage[T2A]{fontenc}
\usepackage[utf8]{inputenc}
\usepackage[english,russian]{babel}
\usepackage{graphicx}
\DeclareGraphicsExtensions{.pdf,.png,.jpg}

\linespread{1.15}

\usepackage{../egetask_ver}

\def\examyear{2023}
\usepackage[colorlinks,linkcolor=blue]{hyperref}\usepackage{draftwatermark}
\SetWatermarkLightness{0.9}
\SetWatermarkText{https://vk.com/egemathika}
\SetWatermarkScale{ 0.4 }
\def\lfoottext{Источник \href{https://vk.com/egemathika}{https://vk.com/egemathika}}

\begin{document}
\begin{taskBN}{1}
\addpictoright[0.25\textwidth]{images/2211215142310128n0}Радиусы четырёх шаров равны $\sqrt[3]{55}$, $\sqrt[3]{4}$, $\sqrt[3]{2}$, $\sqrt[3]{3}$. Найдите радиус шара, объем которого равен сумме их объемов.
\end{taskBN}

\begin{taskBN}{2}
\addpictoright[0.25\textwidth]{images/023155397279567n0}Радиусы трёх шаров равны $2\sqrt[3]{2}$, $\sqrt[3]{10}$, $\sqrt[3]{2}$. Найдите радиус шара, объем которого равен сумме их объемов.
\end{taskBN}

\begin{taskBN}{3}
\addpictoright[0.25\textwidth]{images/9674682034805875n0}Радиусы трёх шаров равны $\sqrt[3]{3}$, $\sqrt[3]{4}$, $1$. Найдите радиус шара, объем которого равен сумме их объемов.
\end{taskBN}

\begin{taskBN}{4}
\addpictoright[0.25\textwidth]{images/4279972061688067n0}Радиусы трёх шаров равны $2\sqrt[3]{14}$, $\sqrt[3]{11}$, $\sqrt[3]{2}$. Найдите радиус шара, объем которого равен сумме их объемов.
\end{taskBN}

\begin{taskBN}{5}
\addpictoright[0.25\textwidth]{images/7298511360486069n0}Радиусы двух шаров равны $20$ и $10$. Найдите радиус шара, площадь поверхности которого равна сумме площадей поверхности двух данных шаров. Ответ умножьте на $\sqrt{5}$.
\end{taskBN}

\begin{taskBN}{6}
\addpictoright[0.25\textwidth]{images/2964023360110897n0}Площадь большого круга первого шара в 49 раз меньше, чем площадь большого круга второго шара. Во сколько раз объём первого шара меньше объёма второго шара?
\end{taskBN}

\begin{taskBN}{7}
\addpictoright[0.25\textwidth]{images/939240680650302n0}Во сколько раз объём первого шара меньше объёма второго шара, если радиус первого шара в 4 раза меньше, чем радиус второго шара?
\end{taskBN}

\begin{taskBN}{8}
\addpictoright[0.25\textwidth]{images/467857635560914n0}Радиусы четырёх шаров равны $\sqrt[3]{58}$, $\sqrt[3]{2}$, $\sqrt[3]{3}$, $1$. Найдите радиус шара, объем которого равен сумме их объемов.
\end{taskBN}

\begin{taskBN}{9}
\addpictoright[0.25\textwidth]{images/27287026356857n0}Радиусы четырёх шаров равны $\sqrt[3]{62}$, $\sqrt[3]{11}$, $\sqrt[3]{49}$, $\sqrt[3]{3}$. Найдите радиус шара, объем которого равен сумме их объемов.
\end{taskBN}

\begin{taskBN}{10}
\addpictoright[0.25\textwidth]{images/634252093994196n0}Радиусы двух шаров равны $8$ и $16$. Найдите радиус шара, площадь поверхности которого равна сумме площадей поверхностей двух данных шаров. Ответ умножьте на $\sqrt{5}$.
\end{taskBN}

\begin{taskBN}{11}
\addpictoright[0.25\textwidth]{images/377438003796044n0}Радиусы четырёх шаров равны $\sqrt[3]{423}$, $2\sqrt[3]{10}$, $2$, $\sqrt[3]{2}$. Найдите радиус шара, объем которого равен сумме их объемов.
\end{taskBN}

\begin{taskBN}{12}
\addpictoright[0.25\textwidth]{images/199510573771001n0}Радиусы двух шаров равны $\sqrt[3]{204}$, $\sqrt[3]{12}$. Найдите радиус шара, объем которого равен сумме их объемов.
\end{taskBN}

\begin{taskBN}{13}
\addpictoright[0.25\textwidth]{images/421763182462874n0}Радиусы трёх шаров равны $\sqrt[3]{21}$, $\sqrt[3]{4}$, $\sqrt[3]{2}$. Найдите радиус шара, объем которого равен сумме их объемов.
\end{taskBN}

\begin{taskBN}{14}
\addpictoright[0.25\textwidth]{images/919049275336772n0}Во сколько раз площадь поверхности первого шара больше площади поверхности второго шара, если радиус первого шара в 8 раз больше, чем радиус второго шара?
\end{taskBN}

\begin{taskBN}{15}
\addpictoright[0.25\textwidth]{images/093379882601811n0}Во сколько раз площадь поверхности первого шара больше площади поверхности второго шара, если радиус первого шара в 9 раз больше, чем радиус второго шара?
\end{taskBN}

\begin{taskBN}{16}
\addpictoright[0.25\textwidth]{images/900975533163084n0}Во сколько раз объём первого шара больше объёма второго шара, если радиус первого шара в 8 раз больше, чем радиус второго шара?
\end{taskBN}

\begin{taskBN}{17}
\addpictoright[0.25\textwidth]{images/93608781264952n0}Объём первого шара в 512 раз меньше, чем объём второго шара. Во сколько раз площадь большого круга первого шара меньше площади большого круга второго шара?
\end{taskBN}

\begin{taskBN}{18}
\addpictoright[0.25\textwidth]{images/030948896516388n0}Площадь большого круга первого шара в 64 раза больше, чем площадь большого круга второго шара. Во сколько раз радиус первого шара больше радиуса второго шара?
\end{taskBN}

\begin{taskBN}{19}
\addpictoright[0.25\textwidth]{images/144303905984777n0}Радиусы трёх шаров равны $\sqrt[3]{18}$, $2$, $1$. Найдите радиус шара, объем которого равен сумме их объемов.
\end{taskBN}

\begin{taskBN}{20}
\addpictoright[0.25\textwidth]{images/414423388928447n0}Радиусы четырёх шаров равны $\sqrt[3]{12}$, $\sqrt[3]{6}$, $\sqrt[3]{7}$, $\sqrt[3]{2}$. Найдите радиус шара, объем которого равен сумме их объемов.
\end{taskBN}

\begin{taskBN}{21}
\addpictoright[0.25\textwidth]{images/7198137028759954n0}Радиусы двух шаров равны $12$ и $5$. Найдите радиус шара, площадь большого круга которого равна сумме площадей больших кругов двух данных шаров.
\end{taskBN}

\begin{taskBN}{22}
\addpictoright[0.25\textwidth]{images/623342774402171n0}Радиусы четырёх шаров равны $\sqrt[3]{12}$, $\sqrt[3]{10}$, $\sqrt[3]{3}$, $\sqrt[3]{2}$. Найдите радиус шара, объем которого равен сумме их объемов.
\end{taskBN}

\begin{taskBN}{23}
\addpictoright[0.25\textwidth]{images/1686984589842293n0}Радиусы трёх шаров равны $\sqrt[3]{45}$, $\sqrt[3]{18}$, $\sqrt[3]{2}$. Найдите радиус шара, объем которого равен сумме их объемов.
\end{taskBN}

\begin{taskBN}{24}
\addpictoright[0.25\textwidth]{images/935002842812123n0}Во сколько раз увеличили площадь поверхности шара, если его объём увеличился в 1000 раз?
\end{taskBN}

\begin{taskBN}{25}
\addpictoright[0.25\textwidth]{images/045227631756507n0}Во сколько раз уменьшили площадь поверхности шара, если его объём уменьшился в 27 раз?
\end{taskBN}

\begin{taskBN}{26}
\addpictoright[0.25\textwidth]{images/14658048838909n0}Радиусы трёх шаров равны $\sqrt[3]{23}$, $\sqrt[3]{3}$, $\sqrt[3]{2}$. Найдите радиус шара, объем которого равен сумме их объемов.
\end{taskBN}

\begin{taskBN}{27}
\addpictoright[0.25\textwidth]{images/011320540967322n0}Радиусы трёх шаров равны $\sqrt[3]{33}$, $\sqrt[3]{37}$, $\sqrt[3]{55}$. Найдите радиус шара, объем которого равен сумме их объемов.
\end{taskBN}

\begin{taskBN}{28}
\addpictoright[0.25\textwidth]{images/7693562765147n0}Радиусы четырёх шаров равны $\sqrt[3]{233}$, $\sqrt[3]{220}$, $\sqrt[3]{267}$, $\sqrt[3]{9}$. Найдите радиус шара, объем которого равен сумме их объемов.
\end{taskBN}

\begin{taskBN}{29}
\addpictoright[0.25\textwidth]{images/719702694213048n0}Радиусы двух шаров равны $6$ и $8$. Найдите радиус шара, площадь большого круга которого равна сумме площадей больших кругов двух данных шаров.
\end{taskBN}

\begin{taskBN}{30}
\addpictoright[0.25\textwidth]{images/062886693593542n0}Радиусы двух шаров равны $14$ и $7$. Найдите радиус шара, площадь большого круга которого равна сумме площадей больших кругов двух данных шаров. Ответ умножьте на $\sqrt{5}$.
\end{taskBN}

\begin{taskBN}{31}
\addpictoright[0.25\textwidth]{images/1068866828472359n0}Радиусы четырёх шаров равны $\sqrt[3]{463}$, $\sqrt[3]{511}$, $\sqrt[3]{23}$, $\sqrt[3]{3}$. Найдите радиус шара, объем которого равен сумме их объемов.
\end{taskBN}

\begin{taskBN}{32}
\addpictoright[0.25\textwidth]{images/159173141204381n0}Во сколько раз площадь поверхности первого шара меньше площади поверхности второго шара, если объём первого шара в 343 раза меньше, чем объём второго шара?
\end{taskBN}

\begin{taskBN}{33}
\addpictoright[0.25\textwidth]{images/10649226548631874n0}Радиусы четырёх шаров равны $\sqrt[3]{767}$, $\sqrt[3]{223}$, $\sqrt[3]{6}$, $\sqrt[3]{4}$. Найдите радиус шара, объем которого равен сумме их объемов.
\end{taskBN}

\begin{taskBN}{34}
\addpictoright[0.25\textwidth]{images/434633584912498n0}Радиусы четырёх шаров равны $\sqrt[3]{111}$, $\sqrt[3]{3}$, $\sqrt[3]{10}$, $1$. Найдите радиус шара, объем которого равен сумме их объемов.
\end{taskBN}

\begin{taskBN}{35}
\addpictoright[0.25\textwidth]{images/360391644544378n0}Радиусы двух шаров равны $\sqrt[3]{3}$, $\sqrt[3]{5}$. Найдите радиус шара, объем которого равен сумме их объемов.
\end{taskBN}

\begin{taskBN}{36}
\addpictoright[0.25\textwidth]{images/0045717356875661n0}Радиусы двух шаров равны $8$ и $6$. Найдите радиус шара, площадь поверхности которого равна сумме площадей поверхностей двух данных шаров.
\end{taskBN}

\begin{taskBN}{37}
\addpictoright[0.25\textwidth]{images/624803168113278n0}Радиусы трёх шаров равны $2\sqrt[3]{6}$, $\sqrt[3]{14}$, $\sqrt[3]{2}$. Найдите радиус шара, объем которого равен сумме их объемов.
\end{taskBN}

\begin{taskBN}{38}
\addpictoright[0.25\textwidth]{images/7554295765242554n0}Радиусы трёх шаров равны $\sqrt[3]{331}$, $\sqrt[3]{11}$, $1$. Найдите радиус шара, объем которого равен сумме их объемов.
\end{taskBN}

\begin{taskBN}{39}
\addpictoright[0.25\textwidth]{images/829103458624846n0}Радиусы трёх шаров равны $\sqrt[3]{109}$, $\sqrt[3]{14}$, $\sqrt[3]{2}$. Найдите радиус шара, объем которого равен сумме их объемов.
\end{taskBN}

\begin{taskBN}{40}
\addpictoright[0.25\textwidth]{images/430713344301245n0}Радиусы четырёх шаров равны $\sqrt[3]{17}$, $\sqrt[3]{4}$, $\sqrt[3]{5}$, $\sqrt[3]{2}$. Найдите радиус шара, объем которого равен сумме их объемов.
\end{taskBN}

\begin{taskBN}{41}
\addpictoright[0.25\textwidth]{images/079244894929494n0}Радиусы двух шаров равны $8$ и $15$. Найдите радиус шара, площадь большого круга которого равна сумме площадей больших кругов двух данных шаров.
\end{taskBN}

\begin{taskBN}{42}
\addpictoright[0.25\textwidth]{images/4065490687961504n0}Радиусы двух шаров равны $10$ и $5$. Найдите радиус шара, площадь большого круга которого равна сумме площадей больших кругов двух данных шаров. Ответ умножьте на $\sqrt{5}$.
\end{taskBN}

\begin{taskBN}{43}
\addpictoright[0.25\textwidth]{images/979419341244596n0}Радиусы двух шаров равны $14$ и $7$. Найдите радиус шара, площадь поверхности которого равна сумме площадей поверхностей двух данных шаров. Ответ умножьте на $\sqrt{5}$.
\end{taskBN}

\begin{taskBN}{44}
\addpictoright[0.25\textwidth]{images/314744993755137n0}Радиусы двух шаров равны $15$ и $20$. Найдите радиус шара, площадь поверхности которого равна сумме площадей поверхностей двух данных шаров.
\end{taskBN}

\begin{taskBN}{45}
\addpictoright[0.25\textwidth]{images/876429867905179n0}Объём первого шара в 512 раз меньше, чем объём второго шара. Во сколько раз радиус первого шара меньше радиуса второго шара?
\end{taskBN}

\begin{taskBN}{46}
\addpictoright[0.25\textwidth]{images/001288448420597n0}Радиусы двух шаров равны $8$ и $15$. Найдите радиус шара, площадь большого круга которого равна сумме площадей больших кругов двух данных шаров.
\end{taskBN}

\begin{taskBN}{47}
\addpictoright[0.25\textwidth]{images/419802888957258n0}Радиусы двух шаров равны $\sqrt[3]{26}$, $\sqrt[3]{2}$. Найдите радиус шара, объем которого равен сумме их объемов.
\end{taskBN}

\begin{taskBN}{48}
\addpictoright[0.25\textwidth]{images/41574271182699n0}Радиусы двух шаров равны $12$ и $16$. Найдите радиус шара, площадь поверхности которого равна сумме площадей поверхностей двух данных шаров.
\end{taskBN}

\begin{taskBN}{49}
\addpictoright[0.25\textwidth]{images/4365150258057733n0}Радиусы двух шаров равны $4$ и $8$. Найдите радиус шара, площадь большого круга которого равна сумме площадей больших кругов двух данных шаров. Ответ разделите на $\sqrt{5}$.
\end{taskBN}

\begin{taskBN}{50}
\addpictoright[0.25\textwidth]{images/556227214639152n0}Во сколько раз объём первого шара меньше объёма второго шара, если площадь поверхности первого шара в 9 раз меньше, чем площадь поверхности второго шара?
\end{taskBN}

\begin{taskBN}{51}
\addpictoright[0.25\textwidth]{images/58039972244801175n0}Радиусы двух шаров равны $2$, $\sqrt[3]{19}$. Найдите радиус шара, объем которого равен сумме их объемов.
\end{taskBN}

\begin{taskBN}{52}
\addpictoright[0.25\textwidth]{images/6931763964756585n0}Радиусы двух шаров равны $2$ и $14$. Найдите радиус шара, площадь поверхности которого равна сумме площадей поверхностей двух данных шаров. Ответ разделите на $\sqrt{2}$.
\end{taskBN}

\begin{taskBN}{53}
\addpictoright[0.25\textwidth]{images/032966577915978n0}Радиусы двух шаров равны $4$ и $2$. Найдите радиус шара, площадь большого круга которого равна сумме площадей больших кругов двух данных шаров. Ответ разделите на $\sqrt{5}$.
\end{taskBN}

\begin{taskBN}{54}
\addpictoright[0.25\textwidth]{images/060130500544422n0}Радиусы двух шаров равны $\sqrt[3]{26}$, $1$. Найдите радиус шара, объем которого равен сумме их объемов.
\end{taskBN}

\begin{taskBN}{55}
\addpictoright[0.25\textwidth]{images/069649896151633n0}Во сколько раз увеличили площадь большого круга шара, если площадь его поверхности увеличилась в 25 раз?
\end{taskBN}

\begin{taskBN}{56}
\addpictoright[0.25\textwidth]{images/7818519420424319n0}Площадь большого круга первого шара в 4 раза больше, чем площадь большого круга второго шара. Во сколько раз объём первого шара больше объёма второго шара?
\end{taskBN}

\begin{taskBN}{57}
\addpictoright[0.25\textwidth]{images/108374163683573n0}Радиусы трёх шаров равны $\sqrt[3]{76}$, $\sqrt[3]{265}$, $\sqrt[3]{2}$. Найдите радиус шара, объем которого равен сумме их объемов.
\end{taskBN}

\begin{taskBN}{58}
\addpictoright[0.25\textwidth]{images/08924618481664n0}Радиусы двух шаров равны $2$ и $14$. Найдите радиус шара, площадь большого круга которого равна сумме площадей больших кругов двух данных шаров. Ответ разделите на $\sqrt{2}$.
\end{taskBN}

\begin{taskBN}{59}
\addpictoright[0.25\textwidth]{images/887958914521938n0}Радиусы двух шаров равны $1$ и $2$. Найдите радиус шара, площадь большого круга которого равна сумме площадей больших кругов двух данных шаров. Ответ разделите на $\sqrt{5}$.
\end{taskBN}

\begin{taskBN}{60}
\addpictoright[0.25\textwidth]{images/361358707722223n0}Радиусы четырёх шаров равны $\sqrt[3]{667}$, $2\sqrt[3]{41}$, $\sqrt[3]{2}$, $\sqrt[3]{3}$. Найдите радиус шара, объем которого равен сумме их объемов.
\end{taskBN}

\begin{taskBN}{61}
\addpictoright[0.25\textwidth]{images/410432925828808n0}Радиусы трёх шаров равны $4\sqrt[3]{11}$, $\sqrt[3]{20}$, $\sqrt[3]{5}$. Найдите радиус шара, объем которого равен сумме их объемов.
\end{taskBN}

\begin{taskBN}{62}
\addpictoright[0.25\textwidth]{images/5359357923762409n0}Радиусы двух шаров равны $\sqrt[3]{11}$, $2\sqrt[3]{2}$. Найдите радиус шара, объем которого равен сумме их объемов.
\end{taskBN}

\begin{taskBN}{63}
\addpictoright[0.25\textwidth]{images/040337971931588n0}Радиусы трёх шаров равны $\sqrt[3]{19}$, $\sqrt[3]{6}$, $\sqrt[3]{2}$. Найдите радиус шара, объем которого равен сумме их объемов.
\end{taskBN}

\begin{taskBN}{64}
\addpictoright[0.25\textwidth]{images/482161749626989n0}Радиусы двух шаров равны $20$ и $15$. Найдите радиус шара, площадь поверхности которого равна сумме площадей поверхностей двух данных шаров.
\end{taskBN}

\begin{taskBN}{65}
\addpictoright[0.25\textwidth]{images/2992530837088292n0}Во сколько раз радиус первого шара больше радиуса второго шара, если объём первого шара в 729 раз больше, чем объём второго шара?
\end{taskBN}

\begin{taskBN}{66}
\addpictoright[0.25\textwidth]{images/0498999310037704n0}Площадь большого круга первого шара в 49 раз больше, чем площадь большого круга второго шара. Во сколько раз площадь поверхности первого шара больше площади поверхности второго шара?
\end{taskBN}

\begin{taskBN}{67}
\addpictoright[0.25\textwidth]{images/455978879122188n0}Радиусы четырёх шаров равны $\sqrt[3]{83}$, $\sqrt[3]{34}$, $\sqrt[3]{3}$, $\sqrt[3]{5}$. Найдите радиус шара, объем которого равен сумме их объемов.
\end{taskBN}

\begin{taskBN}{68}
\addpictoright[0.25\textwidth]{images/793194083216381n0}Радиусы четырёх шаров равны $\sqrt[3]{15}$, $\sqrt[3]{2}$, $\sqrt[3]{7}$, $\sqrt[3]{3}$. Найдите радиус шара, объем которого равен сумме их объемов.
\end{taskBN}

\begin{taskBN}{69}
\addpictoright[0.25\textwidth]{images/539545969887082n0}Радиусы трёх шаров равны $\sqrt[3]{109}$, $\sqrt[3]{231}$, $\sqrt[3]{3}$. Найдите радиус шара, объем которого равен сумме их объемов.
\end{taskBN}

\begin{taskBN}{70}
\addpictoright[0.25\textwidth]{images/985186578254706n0}Радиусы двух шаров равны $\sqrt[3]{7}$, $1$. Найдите радиус шара, объем которого равен сумме их объемов.
\end{taskBN}

\begin{taskBN}{71}
\addpictoright[0.25\textwidth]{images/897222028918517n0}Радиусы четырёх шаров равны $\sqrt[3]{395}$, $\sqrt[3]{316}$, $\sqrt[3]{17}$, $1$. Найдите радиус шара, объем которого равен сумме их объемов.
\end{taskBN}

\begin{taskBN}{72}
\addpictoright[0.25\textwidth]{images/501143698500669n0}Радиусы двух шаров равны $4$ и $8$. Найдите радиус шара, площадь поверхности которого равна сумме площадей поверхностей двух данных шаров. Ответ разделите на $\sqrt{5}$.
\end{taskBN}

\begin{taskBN}{73}
\addpictoright[0.25\textwidth]{images/075746785563833n0}Во сколько раз объём первого шара больше объёма второго шара, если площадь поверхности первого шара в 9 раз больше, чем площадь поверхности второго шара?
\end{taskBN}

\begin{taskBN}{74}
\addpictoright[0.25\textwidth]{images/557495536667742n0}Радиусы четырёх шаров равны $\sqrt[3]{67}$, $6$, $\sqrt[3]{59}$, $\sqrt[3]{2}$. Найдите радиус шара, объем которого равен сумме их объемов.
\end{taskBN}

\begin{taskBN}{75}
\addpictoright[0.25\textwidth]{images/028400316673935n0}Радиусы двух шаров равны $8$ и $16$. Найдите радиус шара, площадь большого круга которого равна сумме площадей больших кругов двух данных шаров. Ответ разделите на $\sqrt{5}$.
\end{taskBN}

\begin{taskBN}{76}
\addpictoright[0.25\textwidth]{images/468056405861248n0}Радиусы двух шаров равны $14$ и $2$. Найдите радиус шара, площадь поверхности которого равна сумме площадей поверхностей двух данных шаров. Ответ разделите на $\sqrt{2}$.
\end{taskBN}

\begin{taskBN}{77}
\addpictoright[0.25\textwidth]{images/028368097156624n0}Радиусы двух шаров равны $2$ и $11$. Найдите радиус шара, площадь большого круга которого равна сумме площадей больших кругов двух данных шаров. Ответ умножьте на $\sqrt{5}$.
\end{taskBN}

\begin{taskBN}{78}
\addpictoright[0.25\textwidth]{images/4557914671580916n0}Радиусы двух шаров равны $2$ и $1$. Найдите радиус шара, площадь большого круга которого равна сумме площадей больших кругов двух данных шаров. Ответ разделите на $\sqrt{5}$.
\end{taskBN}

\begin{taskBN}{79}
\addpictoright[0.25\textwidth]{images/339302226278655n0}Радиус первого шара в 2 раза меньше, чем радиус второго шара. Во сколько раз площадь большого круга первого шара меньше площади большого круга второго шара?
\end{taskBN}

\begin{taskBN}{80}
\addpictoright[0.25\textwidth]{images/225447804297617n0}Радиусы двух шаров равны $8$ и $16$. Найдите радиус шара, площадь поверхности которого равна сумме площадей поверхностей двух данных шаров. Ответ умножьте на $\sqrt{5}$.
\end{taskBN}

\begin{taskBN}{81}
\addpictoright[0.25\textwidth]{images/180245228748495n0}Радиусы четырёх шаров равны $\sqrt[3]{59}$, $2\sqrt[3]{5}$, $\sqrt[3]{25}$, $1$. Найдите радиус шара, объем которого равен сумме их объемов.
\end{taskBN}

\begin{taskBN}{82}
\addpictoright[0.25\textwidth]{images/640555942531932n0}Радиусы четырёх шаров равны $3$, $\sqrt[3]{33}$, $\sqrt[3]{3}$, $\sqrt[3]{2}$. Найдите радиус шара, объем которого равен сумме их объемов.
\end{taskBN}

\begin{taskBN}{83}
\addpictoright[0.25\textwidth]{images/918061785977687n0}Площадь поверхности первого шара в 81 раз больше, чем площадь поверхности второго шара. Во сколько раз объём первого шара больше объёма второго шара?
\end{taskBN}

\begin{taskBN}{84}
\addpictoright[0.25\textwidth]{images/0538980364117576n0}Во сколько раз площадь большого круга первого шара меньше площади большого круга второго шара, если объём первого шара в 27 раз меньше, чем объём второго шара?
\end{taskBN}

\begin{taskBN}{85}
\addpictoright[0.25\textwidth]{images/5029808512919234n0}Радиусы четырёх шаров равны $\sqrt[3]{60}$, $\sqrt[3]{69}$, $\sqrt[3]{82}$, $\sqrt[3]{5}$. Найдите радиус шара, объем которого равен сумме их объемов.
\end{taskBN}

\begin{taskBN}{86}
\addpictoright[0.25\textwidth]{images/6029779515431186n0}Радиусы трёх шаров равны $3\sqrt[3]{2}$, $\sqrt[3]{7}$, $\sqrt[3]{3}$. Найдите радиус шара, объем которого равен сумме их объемов.
\end{taskBN}

\begin{taskBN}{87}
\addpictoright[0.25\textwidth]{images/851956650977754n0}Радиусы трёх шаров равны $2\sqrt[3]{14}$, $\sqrt[3]{11}$, $\sqrt[3]{2}$. Найдите радиус шара, объем которого равен сумме их объемов.
\end{taskBN}

\begin{taskBN}{88}
\addpictoright[0.25\textwidth]{images/964866881140179n0}Радиусы двух шаров равны $8$ и $4$. Найдите радиус шара, площадь поверхности которого равна сумме площадей поверхностей двух данных шаров. Ответ умножьте на $\sqrt{5}$.
\end{taskBN}

\begin{taskBN}{89}
\addpictoright[0.25\textwidth]{images/12550784995891n0}Радиусы четырёх шаров равны $\sqrt[3]{18}$, $\sqrt[3]{3}$, $\sqrt[3]{4}$, $\sqrt[3]{2}$. Найдите радиус шара, объем которого равен сумме их объемов.
\end{taskBN}

\begin{taskBN}{90}
\addpictoright[0.25\textwidth]{images/617674453192577n0}Во сколько раз увеличили радиус шара, если площадь его поверхности увеличилась в 36 раз?
\end{taskBN}

\begin{taskBN}{91}
\addpictoright[0.25\textwidth]{images/68295297853542n0}Объём первого шара в 343 раза больше, чем объём второго шара. Во сколько раз радиус первого шара больше радиуса второго шара?
\end{taskBN}

\begin{taskBN}{92}
\addpictoright[0.25\textwidth]{images/8594774842213964n0}Радиусы двух шаров равны $4$ и $8$. Найдите радиус шара, площадь поверхности которого равна сумме площадей поверхностей двух данных шаров. Ответ умножьте на $\sqrt{5}$.
\end{taskBN}

\begin{taskBN}{93}
\addpictoright[0.25\textwidth]{images/8718578261779353n0}Радиусы трёх шаров равны $\sqrt[3]{11}$, $\sqrt[3]{14}$, $\sqrt[3]{2}$. Найдите радиус шара, объем которого равен сумме их объемов.
\end{taskBN}

\begin{taskBN}{94}
\addpictoright[0.25\textwidth]{images/526568311407074n0}Радиусы трёх шаров равны $\sqrt[3]{207}$, $\sqrt[3]{4}$, $\sqrt[3]{5}$. Найдите радиус шара, объем которого равен сумме их объемов.
\end{taskBN}

\begin{taskBN}{95}
\addpictoright[0.25\textwidth]{images/021211415875704n0}Радиусы двух шаров равны $8$ и $15$. Найдите радиус шара, площадь поверхности которого равна сумме площадей поверхностей двух данных шаров.
\end{taskBN}

\begin{taskBN}{96}
\addpictoright[0.25\textwidth]{images/2824487263574664n0}Радиус первого шара в 8 раз меньше, чем радиус второго шара. Во сколько раз объём первого шара меньше объёма второго шара?
\end{taskBN}

\begin{taskBN}{97}
\addpictoright[0.25\textwidth]{images/44743777263621554n0}Радиусы трёх шаров равны $\sqrt[3]{11}$, $\sqrt[3]{14}$, $\sqrt[3]{2}$. Найдите радиус шара, объем которого равен сумме их объемов.
\end{taskBN}

\begin{taskBN}{98}
\addpictoright[0.25\textwidth]{images/7469774844336128n0}Во сколько раз увеличили объём шара, если площадь его поверхности увеличилась в 16 раз?
\end{taskBN}

\begin{taskBN}{99}
\addpictoright[0.25\textwidth]{images/012250719578259n0}Радиусы трёх шаров равны $\sqrt[3]{59}$, $\sqrt[3]{4}$, $1$. Найдите радиус шара, объем которого равен сумме их объемов.
\end{taskBN}

\begin{taskBN}{100}
\addpictoright[0.25\textwidth]{images/52831583050225n0}Радиусы четырёх шаров равны $\sqrt[3]{9}$, $\sqrt[3]{4}$, $\sqrt[3]{12}$, $\sqrt[3]{2}$. Найдите радиус шара, объем которого равен сумме их объемов.
\end{taskBN}

\begin{taskBN}{101}
\addpictoright[0.25\textwidth]{images/4103515475535464n0}Радиусы четырёх шаров равны $\sqrt[3]{107}$, $\sqrt[3]{223}$, $\sqrt[3]{11}$, $\sqrt[3]{2}$. Найдите радиус шара, объем которого равен сумме их объемов.
\end{taskBN}

\begin{taskBN}{102}
\addpictoright[0.25\textwidth]{images/038120189893523n0}Радиусы четырёх шаров равны $2\sqrt[3]{7}$, $\sqrt[3]{4}$, $\sqrt[3]{3}$, $\sqrt[3]{2}$. Найдите радиус шара, объем которого равен сумме их объемов.
\end{taskBN}

\begin{taskBN}{103}
\addpictoright[0.25\textwidth]{images/814429228707107n0}Радиусы двух шаров равны $7$ и $14$. Найдите радиус шара, площадь большого круга которого равна сумме площадей больших кругов двух данных шаров. Ответ умножьте на $\sqrt{5}$.
\end{taskBN}

\begin{taskBN}{104}
\addpictoright[0.25\textwidth]{images/8734248467977104n0}Радиусы двух шаров равны $18$ и $9$. Найдите радиус шара, площадь большого круга которого равна сумме площадей больших кругов двух данных шаров. Ответ разделите на $\sqrt{5}$.
\end{taskBN}

\begin{taskBN}{105}
\addpictoright[0.25\textwidth]{images/3282172302163826n0}Радиусы четырёх шаров равны $\sqrt[3]{55}$, $\sqrt[3]{3}$, $\sqrt[3]{5}$, $\sqrt[3]{2}$. Найдите радиус шара, объем которого равен сумме их объемов.
\end{taskBN}

\begin{taskBN}{106}
\addpictoright[0.25\textwidth]{images/494562844895161n0}Радиусы двух шаров равны $9$ и $12$. Найдите радиус шара, площадь поверхности которого равна сумме площадей поверхностей двух данных шаров.
\end{taskBN}

\begin{taskBN}{107}
\addpictoright[0.25\textwidth]{images/581578492506126n0}Радиусы двух шаров равны $14$ и $7$. Найдите радиус шара, площадь большого круга которого равна сумме площадей больших кругов двух данных шаров. Ответ разделите на $\sqrt{5}$.
\end{taskBN}

\begin{taskBN}{108}
\addpictoright[0.25\textwidth]{images/3416627422831648n0}Радиусы двух шаров равны $9$ и $12$. Найдите радиус шара, площадь поверхности которого равна сумме площадей поверхностей двух данных шаров.
\end{taskBN}

\begin{taskBN}{109}
\addpictoright[0.25\textwidth]{images/0333245246405762n0}Радиусы четырёх шаров равны $3\sqrt[3]{7}$, $\sqrt[3]{21}$, $\sqrt[3]{5}$, $1$. Найдите радиус шара, объем которого равен сумме их объемов.
\end{taskBN}

\begin{taskBN}{110}
\addpictoright[0.25\textwidth]{images/382730582122966n0}Радиусы четырёх шаров равны $\sqrt[3]{10}$, $\sqrt[3]{14}$, $\sqrt[3]{2}$, $1$. Найдите радиус шара, объем которого равен сумме их объемов.
\end{taskBN}

\begin{taskBN}{111}
\addpictoright[0.25\textwidth]{images/814461391966876n0}Во сколько раз увеличили площадь поверхности шара, если его радиус увеличился в 5 раз?
\end{taskBN}

\begin{taskBN}{112}
\addpictoright[0.25\textwidth]{images/6652036008862656n0}Радиусы двух шаров равны $\sqrt[3]{118}$, $\sqrt[3]{7}$. Найдите радиус шара, объем которого равен сумме их объемов.
\end{taskBN}

\begin{taskBN}{113}
\addpictoright[0.25\textwidth]{images/8509449521449819n0}Радиусы трёх шаров равны $\sqrt[3]{6}$, $\sqrt[3]{19}$, $\sqrt[3]{2}$. Найдите радиус шара, объем которого равен сумме их объемов.
\end{taskBN}

\begin{taskBN}{114}
\addpictoright[0.25\textwidth]{images/8131693790552115n0}Радиусы трёх шаров равны $\sqrt[3]{338}$, $\sqrt[3]{4}$, $1$. Найдите радиус шара, объем которого равен сумме их объемов.
\end{taskBN}

\begin{taskBN}{115}
\addpictoright[0.25\textwidth]{images/3091607715835283n0}Во сколько раз площадь большого круга первого шара больше площади большого круга второго шара, если площадь поверхности первого шара в 49 раз больше, чем площадь поверхности второго шара?
\end{taskBN}

\begin{taskBN}{116}
\addpictoright[0.25\textwidth]{images/1000078257102237n0}Радиусы трёх шаров равны $\sqrt[3]{5}$, $\sqrt[3]{2}$, $1$. Найдите радиус шара, объем которого равен сумме их объемов.
\end{taskBN}

\begin{taskBN}{117}
\addpictoright[0.25\textwidth]{images/3869876674316393n0}Радиусы четырёх шаров равны $\sqrt[3]{25}$, $\sqrt[3]{33}$, $\sqrt[3]{4}$, $\sqrt[3]{2}$. Найдите радиус шара, объем которого равен сумме их объемов.
\end{taskBN}

\begin{taskBN}{118}
\addpictoright[0.25\textwidth]{images/203644157547011n0}Радиусы трёх шаров равны $\sqrt[3]{4}$, $\sqrt[3]{3}$, $1$. Найдите радиус шара, объем которого равен сумме их объемов.
\end{taskBN}

\begin{taskBN}{119}
\addpictoright[0.25\textwidth]{images/547409788706403n0}Радиусы трёх шаров равны $\sqrt[3]{5}$, $\sqrt[3]{2}$, $1$. Найдите радиус шара, объем которого равен сумме их объемов.
\end{taskBN}

\begin{taskBN}{120}
\addpictoright[0.25\textwidth]{images/69887800999804n0}Радиусы четырёх шаров равны $\sqrt[3]{51}$, $\sqrt[3]{35}$, $\sqrt[3]{18}$, $\sqrt[3]{21}$. Найдите радиус шара, объем которого равен сумме их объемов.
\end{taskBN}

\begin{taskBN}{121}
\addpictoright[0.25\textwidth]{images/0282830249371406n0}Радиусы четырёх шаров равны $\sqrt[3]{38}$, $\sqrt[3]{19}$, $\sqrt[3]{5}$, $\sqrt[3]{2}$. Найдите радиус шара, объем которого равен сумме их объемов.
\end{taskBN}

\begin{taskBN}{122}
\addpictoright[0.25\textwidth]{images/461372064307171n0}Радиусы четырёх шаров равны $\sqrt[3]{35}$, $2\sqrt[3]{3}$, $\sqrt[3]{3}$, $\sqrt[3]{2}$. Найдите радиус шара, объем которого равен сумме их объемов.
\end{taskBN}

\begin{taskBN}{123}
\addpictoright[0.25\textwidth]{images/7703612962191644n0}Радиусы двух шаров равны $\sqrt[3]{7}$, $1$. Найдите радиус шара, объем которого равен сумме их объемов.
\end{taskBN}

\begin{taskBN}{124}
\addpictoright[0.25\textwidth]{images/673129822431119n0}Радиусы двух шаров равны $\sqrt[3]{6}$, $\sqrt[3]{2}$. Найдите радиус шара, объем которого равен сумме их объемов.
\end{taskBN}

\begin{taskBN}{125}
\addpictoright[0.25\textwidth]{images/790726679295683n0}Радиусы двух шаров равны $6$ и $3$. Найдите радиус шара, площадь большого круга которого равна сумме площадей больших кругов двух данных шаров. Ответ умножьте на $\sqrt{5}$.
\end{taskBN}

\begin{taskBN}{126}
\addpictoright[0.25\textwidth]{images/521700725785119n0}Радиусы двух шаров равны $15$ и $8$. Найдите радиус шара, площадь большого круга которого равна сумме площадей больших кругов двух данных шаров.
\end{taskBN}

\begin{taskBN}{127}
\addpictoright[0.25\textwidth]{images/5100194483718201n0}Радиусы двух шаров равны $\sqrt[3]{7}$, $\sqrt[3]{2}$. Найдите радиус шара, объем которого равен сумме их объемов.
\end{taskBN}

\begin{taskBN}{128}
\addpictoright[0.25\textwidth]{images/869221608191596n0}Радиусы трёх шаров равны $\sqrt[3]{30}$, $\sqrt[3]{22}$, $\sqrt[3]{12}$. Найдите радиус шара, объем которого равен сумме их объемов.
\end{taskBN}

\begin{taskBN}{129}
\addpictoright[0.25\textwidth]{images/989981186428572n0}Радиусы двух шаров равны $3$ и $4$. Найдите радиус шара, площадь поверхности которого равна сумме площадей поверхностей двух данных шаров.
\end{taskBN}

\begin{taskBN}{130}
\addpictoright[0.25\textwidth]{images/881298469867968n0}Радиусы двух шаров равны $12$ и $5$. Найдите радиус шара, площадь большого круга которого равна сумме площадей больших кругов двух данных шаров.
\end{taskBN}

\begin{taskBN}{131}
\addpictoright[0.25\textwidth]{images/3937908681004016n0}Радиусы двух шаров равны $12$ и $16$. Найдите радиус шара, площадь большого круга которого равна сумме площадей больших кругов двух данных шаров.
\end{taskBN}

\begin{taskBN}{132}
\addpictoright[0.25\textwidth]{images/377500295894323n0}Радиусы четырёх шаров равны $\sqrt[3]{332}$, $2$, $\sqrt[3]{2}$, $1$. Найдите радиус шара, объем которого равен сумме их объемов.
\end{taskBN}

\begin{taskBN}{133}
\addpictoright[0.25\textwidth]{images/473919301755066n0}Во сколько раз площадь большого круга первого шара больше площади большого круга второго шара, если объём первого шара в 343 раза больше, чем объём второго шара?
\end{taskBN}

\begin{taskBN}{134}
\addpictoright[0.25\textwidth]{images/175887342252934n0}Радиусы двух шаров равны $12$ и $16$. Найдите радиус шара, площадь поверхности которого равна сумме площадей поверхностей двух данных шаров.
\end{taskBN}

\begin{taskBN}{135}
\addpictoright[0.25\textwidth]{images/486775546462735n0}Радиусы двух шаров равны $4$ и $2$. Найдите радиус шара, площадь большого круга которого равна сумме площадей больших кругов двух данных шаров. Ответ разделите на $\sqrt{5}$.
\end{taskBN}

\begin{taskBN}{136}
\addpictoright[0.25\textwidth]{images/919376429063961n0}Радиусы четырёх шаров равны $\sqrt[3]{10}$, $\sqrt[3]{3}$, $\sqrt[3]{13}$, $\sqrt[3]{2}$. Найдите радиус шара, объем которого равен сумме их объемов.
\end{taskBN}

\begin{taskBN}{137}
\addpictoright[0.25\textwidth]{images/610295646068167n0}Радиусы трёх шаров равны $2$, $\sqrt[3]{17}$, $\sqrt[3]{2}$. Найдите радиус шара, объем которого равен сумме их объемов.
\end{taskBN}

\begin{taskBN}{138}
\addpictoright[0.25\textwidth]{images/000043719140061n0}Радиусы трёх шаров равны $5\sqrt[3]{2}$, $\sqrt[3]{91}$, $\sqrt[3]{2}$. Найдите радиус шара, объем которого равен сумме их объемов.
\end{taskBN}

\begin{taskBN}{139}
\addpictoright[0.25\textwidth]{images/194200952075365n0}Во сколько раз радиус первого шара меньше радиуса второго шара, если площадь большого круга первого шара в 36 раз меньше, чем площадь большого круга второго шара?
\end{taskBN}

\begin{taskBN}{140}
\addpictoright[0.25\textwidth]{images/119033183268155n0}Радиусы трёх шаров равны $\sqrt[3]{3}$, $\sqrt[3]{4}$, $1$. Найдите радиус шара, объем которого равен сумме их объемов.
\end{taskBN}

\begin{taskBN}{141}
\addpictoright[0.25\textwidth]{images/6689910453025363n0}Радиусы четырёх шаров равны $\sqrt[3]{20}$, $\sqrt[3]{55}$, $2\sqrt[3]{6}$, $\sqrt[3]{2}$. Найдите радиус шара, объем которого равен сумме их объемов.
\end{taskBN}

\begin{taskBN}{142}
\addpictoright[0.25\textwidth]{images/409044422237634n0}Радиусы четырёх шаров равны $\sqrt[3]{4}$, $\sqrt[3]{7}$, $\sqrt[3]{15}$, $1$. Найдите радиус шара, объем которого равен сумме их объемов.
\end{taskBN}

\begin{taskBN}{143}
\addpictoright[0.25\textwidth]{images/11629718108817n0}Радиус первого шара в 4 раза больше, чем радиус второго шара. Во сколько раз объём первого шара больше объёма второго шара?
\end{taskBN}

\begin{taskBN}{144}
\addpictoright[0.25\textwidth]{images/824656347781512n0}Радиусы четырёх шаров равны $3\sqrt[3]{6}$, $\sqrt[3]{43}$, $\sqrt[3]{10}$, $\sqrt[3]{2}$. Найдите радиус шара, объем которого равен сумме их объемов.
\end{taskBN}

\begin{taskBN}{145}
\addpictoright[0.25\textwidth]{images/614878419807374n0}Радиусы трёх шаров равны $\sqrt[3]{3}$, $\sqrt[3]{4}$, $1$. Найдите радиус шара, объем которого равен сумме их объемов.
\end{taskBN}

\begin{taskBN}{146}
\addpictoright[0.25\textwidth]{images/1705554860039022n0}Во сколько раз увеличили радиус шара, если его объём увеличился в 216 раз?
\end{taskBN}

\begin{taskBN}{147}
\addpictoright[0.25\textwidth]{images/866809224069039n0}Радиусы трёх шаров равны $\sqrt[3]{23}$, $\sqrt[3]{3}$, $1$. Найдите радиус шара, объем которого равен сумме их объемов.
\end{taskBN}

\begin{taskBN}{148}
\addpictoright[0.25\textwidth]{images/987528335727103n0}Радиусы четырёх шаров равны $\sqrt[3]{22}$, $\sqrt[3]{33}$, $\sqrt[3]{7}$, $\sqrt[3]{2}$. Найдите радиус шара, объем которого равен сумме их объемов.
\end{taskBN}

\begin{taskBN}{149}
\addpictoright[0.25\textwidth]{images/630591013938968n0}Радиусы четырёх шаров равны $6\sqrt[3]{2}$, $\sqrt[3]{26}$, $\sqrt[3]{52}$, $\sqrt[3]{2}$. Найдите радиус шара, объем которого равен сумме их объемов.
\end{taskBN}

\begin{taskBN}{150}
\addpictoright[0.25\textwidth]{images/646526254350173n0}Радиус первого шара в 5 раз меньше, чем радиус второго шара. Во сколько раз площадь поверхности первого шара меньше площади поверхности второго шара?
\end{taskBN}

\begin{taskBN}{151}
\addpictoright[0.25\textwidth]{images/637916641333806n0}Площадь большого круга первого шара в 81 раз меньше, чем площадь большого круга второго шара. Во сколько раз площадь поверхности первого шара меньше площади поверхности второго шара?
\end{taskBN}

\begin{taskBN}{152}
\addpictoright[0.25\textwidth]{images/596106148137827n0}Радиусы двух шаров равны $\sqrt[3]{6}$, $\sqrt[3]{2}$. Найдите радиус шара, объем которого равен сумме их объемов.
\end{taskBN}

\begin{taskBN}{153}
\addpictoright[0.25\textwidth]{images/41042599122120715n0}Радиусы двух шаров равны $\sqrt[3]{61}$, $\sqrt[3]{3}$. Найдите радиус шара, объем которого равен сумме их объемов.
\end{taskBN}

\begin{taskBN}{154}
\addpictoright[0.25\textwidth]{images/976107559666396n0}Радиусы двух шаров равны $16$ и $8$. Найдите радиус шара, площадь поверхности которого равна сумме площадей поверхностей двух данных шаров. Ответ разделите на $\sqrt{5}$.
\end{taskBN}

\begin{taskBN}{155}
\addpictoright[0.25\textwidth]{images/385415496948929n0}Радиусы четырёх шаров равны $\sqrt[3]{42}$, $2\sqrt[3]{2}$, $\sqrt[3]{4}$, $\sqrt[3]{2}$. Найдите радиус шара, объем которого равен сумме их объемов.
\end{taskBN}

\begin{taskBN}{156}
\addpictoright[0.25\textwidth]{images/2415416713079872n0}Радиусы двух шаров равны $14$ и $2$. Найдите радиус шара, площадь поверхности которого равна сумме площадей поверхностей двух данных шаров. Ответ умножьте на $\sqrt{2}$.
\end{taskBN}

\begin{taskBN}{157}
\addpictoright[0.25\textwidth]{images/94784831035392n0}Радиусы двух шаров равны $1$ и $2$. Найдите радиус шара, площадь поверхности которого равна сумме площадей поверхностей двух данных шаров. Ответ разделите на $\sqrt{5}$.
\end{taskBN}

\begin{taskBN}{158}
\addpictoright[0.25\textwidth]{images/2241424720972387n0}Радиусы четырёх шаров равны $\sqrt[3]{281}$, $\sqrt[3]{17}$, $\sqrt[3]{43}$, $\sqrt[3]{2}$. Найдите радиус шара, объем которого равен сумме их объемов.
\end{taskBN}

\begin{taskBN}{159}
\addpictoright[0.25\textwidth]{images/507870683913153n0}Радиусы четырёх шаров равны $\sqrt[3]{179}$, $\sqrt[3]{30}$, $\sqrt[3]{5}$, $\sqrt[3]{2}$. Найдите радиус шара, объем которого равен сумме их объемов.
\end{taskBN}

\begin{taskBN}{160}
\addpictoright[0.25\textwidth]{images/8112345121838564n0}Радиусы трёх шаров равны $\sqrt[3]{75}$, $\sqrt[3]{35}$, $\sqrt[3]{15}$. Найдите радиус шара, объем которого равен сумме их объемов.
\end{taskBN}

\begin{taskBN}{161}
\addpictoright[0.25\textwidth]{images/693180287149698n0}Радиусы двух шаров равны $4$ и $2$. Найдите радиус шара, площадь большого круга которого равна сумме площадей больших кругов двух данных шаров. Ответ разделите на $\sqrt{5}$.
\end{taskBN}

\begin{taskBN}{162}
\addpictoright[0.25\textwidth]{images/6596188040058315n0}Радиусы двух шаров равны $12$ и $16$. Найдите радиус шара, площадь большого круга которого равна сумме площадей больших кругов двух данных шаров.
\end{taskBN}

\begin{taskBN}{163}
\addpictoright[0.25\textwidth]{images/890763452398768n0}Радиусы двух шаров равны $\sqrt[3]{124}$, $\sqrt[3]{2}$. Найдите радиус шара, объем которого равен сумме их объемов.
\end{taskBN}

\begin{taskBN}{164}
\addpictoright[0.25\textwidth]{images/931090556707808n0}Радиусы двух шаров равны $9$ и $18$. Найдите радиус шара, площадь большого круга которого равна сумме площадей больших кругов двух данных шаров. Ответ умножьте на $\sqrt{5}$.
\end{taskBN}

\begin{taskBN}{165}
\addpictoright[0.25\textwidth]{images/136245256655958n0}Радиусы двух шаров равны $3$ и $4$. Найдите радиус шара, площадь большого круга которого равна сумме площадей больших кругов двух данных шаров.
\end{taskBN}

\begin{taskBN}{166}
\addpictoright[0.25\textwidth]{images/012642349069567n0}Радиусы четырёх шаров равны $\sqrt[3]{10}$, $\sqrt[3]{5}$, $\sqrt[3]{11}$, $1$. Найдите радиус шара, объем которого равен сумме их объемов.
\end{taskBN}

\begin{taskBN}{167}
\addpictoright[0.25\textwidth]{images/822439633805805n0}Площадь поверхности первого шара в 9 раз больше, чем площадь поверхности второго шара. Во сколько раз площадь большого круга первого шара больше площади большого круга второго шара?
\end{taskBN}

\begin{taskBN}{168}
\addpictoright[0.25\textwidth]{images/8691261448774785n0}Во сколько раз уменьшили объём шара, если площадь его поверхности уменьшилась в 25 раз?
\end{taskBN}

\begin{taskBN}{169}
\addpictoright[0.25\textwidth]{images/810890181098337n0}Радиусы двух шаров равны $2$ и $1$. Найдите радиус шара, площадь большого круга которого равна сумме площадей больших кругов двух данных шаров. Ответ разделите на $\sqrt{5}$.
\end{taskBN}

\begin{taskBN}{170}
\addpictoright[0.25\textwidth]{images/122987276160244n0}Объём первого шара в 512 раз больше, чем объём второго шара. Во сколько раз площадь большого круга первого шара больше площади большого круга второго шара?
\end{taskBN}

\begin{taskBN}{171}
\addpictoright[0.25\textwidth]{images/434465529377182n0}Радиусы двух шаров равны $17$ и $7$. Найдите радиус шара, площадь большого круга которого равна сумме площадей больших кругов двух данных шаров. Ответ разделите на $\sqrt{2}$.
\end{taskBN}

\begin{taskBN}{172}
\addpictoright[0.25\textwidth]{images/584298932146206n0}Радиусы двух шаров равны $2$ и $14$. Найдите радиус шара, площадь поверхности которого равна сумме площадей поверхностей двух данных шаров. Ответ разделите на $\sqrt{2}$.
\end{taskBN}

\begin{taskBN}{173}
\addpictoright[0.25\textwidth]{images/723313749748473n0}Объём первого шара в 512 раз больше, чем объём второго шара. Во сколько раз площадь большого круга первого шара больше площади большого круга второго шара?
\end{taskBN}

\begin{taskBN}{174}
\addpictoright[0.25\textwidth]{images/6653746273031784n0}Радиусы четырёх шаров равны $3\sqrt[3]{12}$, $\sqrt[3]{14}$, $\sqrt[3]{3}$, $\sqrt[3]{2}$. Найдите радиус шара, объем которого равен сумме их объемов.
\end{taskBN}

\begin{taskBN}{175}
\addpictoright[0.25\textwidth]{images/3822752491054278n0}Радиусы двух шаров равны $9$ и $18$. Найдите радиус шара, площадь поверхности которого равна сумме площадей поверхностей двух данных шаров. Ответ разделите на $\sqrt{5}$.
\end{taskBN}

\begin{taskBN}{176}
\addpictoright[0.25\textwidth]{images/56246050992713n0}Радиусы двух шаров равны $17$ и $7$. Найдите радиус шара, площадь поверхности которого равна сумме площадей поверхностей двух данных шаров. Ответ разделите на $\sqrt{2}$.
\end{taskBN}

\begin{taskBN}{177}
\addpictoright[0.25\textwidth]{images/200475472347627n0}Радиусы трёх шаров равны $\sqrt[3]{109}$, $\sqrt[3]{15}$, $1$. Найдите радиус шара, объем которого равен сумме их объемов.
\end{taskBN}

\begin{taskBN}{178}
\addpictoright[0.25\textwidth]{images/964960902901283n0}Радиусы двух шаров равны $4$ и $8$. Найдите радиус шара, площадь большого круга которого равна сумме площадей больших кругов двух данных шаров. Ответ умножьте на $\sqrt{5}$.
\end{taskBN}

\begin{taskBN}{179}
\addpictoright[0.25\textwidth]{images/9140091012539395n0}Радиусы трёх шаров равны $\sqrt[3]{262}$, $\sqrt[3]{79}$, $\sqrt[3]{2}$. Найдите радиус шара, объем которого равен сумме их объемов.
\end{taskBN}

\begin{taskBN}{180}
\addpictoright[0.25\textwidth]{images/699134596456426n0}Радиусы четырёх шаров равны $\sqrt[3]{318}$, $\sqrt[3]{4}$, $\sqrt[3]{19}$, $\sqrt[3]{2}$. Найдите радиус шара, объем которого равен сумме их объемов.
\end{taskBN}

\begin{taskBN}{181}
\addpictoright[0.25\textwidth]{images/042592326561371n0}Радиусы двух шаров равны $17$ и $7$. Найдите радиус шара, площадь поверхности которого равна сумме площадей поверхностей двух данных шаров. Ответ разделите на $\sqrt{2}$.
\end{taskBN}

\begin{taskBN}{182}
\addpictoright[0.25\textwidth]{images/827022626348104n0}Радиусы двух шаров равны $5$ и $10$. Найдите радиус шара, площадь поверхности которого равна сумме площадей поверхностей двух данных шаров. Ответ умножьте на $\sqrt{5}$.
\end{taskBN}

\begin{taskBN}{183}
\addpictoright[0.25\textwidth]{images/802135229631171n0}Радиусы двух шаров равны $\sqrt[3]{6}$, $\sqrt[3]{2}$. Найдите радиус шара, объем которого равен сумме их объемов.
\end{taskBN}

\begin{taskBN}{184}
\addpictoright[0.25\textwidth]{images/157501501115177n0}Радиусы двух шаров равны $12$ и $16$. Найдите радиус шара, площадь большого круга которого равна сумме площадей больших кругов двух данных шаров.
\end{taskBN}

\begin{taskBN}{185}
\addpictoright[0.25\textwidth]{images/92029148183725n0}Радиусы четырёх шаров равны $\sqrt[3]{276}$, $\sqrt[3]{35}$, $\sqrt[3]{31}$, $\sqrt[3]{2}$. Найдите радиус шара, объем которого равен сумме их объемов.
\end{taskBN}

\begin{taskBN}{186}
\addpictoright[0.25\textwidth]{images/885244849140552n0}Во сколько раз уменьшили радиус шара, если его площадь большого круга уменьшилась в 64 раза?
\end{taskBN}

\begin{taskBN}{187}
\addpictoright[0.25\textwidth]{images/992928062295427n0}Радиусы двух шаров равны $\sqrt[3]{6}$, $\sqrt[3]{2}$. Найдите радиус шара, объем которого равен сумме их объемов.
\end{taskBN}

\begin{taskBN}{188}
\addpictoright[0.25\textwidth]{images/9619425955992955n0}Радиусы двух шаров равны $18$ и $9$. Найдите радиус шара, площадь большого круга которого равна сумме площадей больших кругов двух данных шаров. Ответ разделите на $\sqrt{5}$.
\end{taskBN}

\begin{taskBN}{189}
\addpictoright[0.25\textwidth]{images/7293716335680296n0}Радиусы четырёх шаров равны $\sqrt[3]{18}$, $\sqrt[3]{4}$, $\sqrt[3]{2}$, $\sqrt[3]{3}$. Найдите радиус шара, объем которого равен сумме их объемов.
\end{taskBN}

\begin{taskBN}{190}
\addpictoright[0.25\textwidth]{images/5907986815880837n0}Радиусы двух шаров равны $\sqrt[3]{26}$, $1$. Найдите радиус шара, объем которого равен сумме их объемов.
\end{taskBN}

\begin{taskBN}{191}
\addpictoright[0.25\textwidth]{images/8480095246680344n0}Радиусы трёх шаров равны $2\sqrt[3]{7}$, $2\sqrt[3]{18}$, $2\sqrt[3]{2}$. Найдите радиус шара, объем которого равен сумме их объемов.
\end{taskBN}

\begin{taskBN}{192}
\addpictoright[0.25\textwidth]{images/317153442693288n0}Радиусы четырёх шаров равны $\sqrt[3]{10}$, $\sqrt[3]{6}$, $\sqrt[3]{9}$, $\sqrt[3]{2}$. Найдите радиус шара, объем которого равен сумме их объемов.
\end{taskBN}

\begin{taskBN}{193}
\addpictoright[0.25\textwidth]{images/42229980933192n0}Радиусы двух шаров равны $\sqrt[3]{212}$, $\sqrt[3]{4}$. Найдите радиус шара, объем которого равен сумме их объемов.
\end{taskBN}

\begin{taskBN}{194}
\addpictoright[0.25\textwidth]{images/607736932339276n0}Объём первого шара в 343 раза меньше, чем объём второго шара. Во сколько раз площадь большого круга первого шара меньше площади большого круга второго шара?
\end{taskBN}

\begin{taskBN}{195}
\addpictoright[0.25\textwidth]{images/283662520667451n0}Во сколько раз уменьшили объём шара, если его радиус уменьшился в 7 раз?
\end{taskBN}

\begin{taskBN}{196}
\addpictoright[0.25\textwidth]{images/622481696045852n0}Площадь большого круга первого шара в 64 раза больше, чем площадь большого круга второго шара. Во сколько раз радиус первого шара больше радиуса второго шара?
\end{taskBN}

\begin{taskBN}{197}
\addpictoright[0.25\textwidth]{images/430136036347715n0}Радиусы четырёх шаров равны $\sqrt[3]{107}$, $\sqrt[3]{105}$, $\sqrt[3]{3}$, $\sqrt[3]{2}$. Найдите радиус шара, объем которого равен сумме их объемов.
\end{taskBN}

\begin{taskBN}{198}
\addpictoright[0.25\textwidth]{images/0426421829590657n0}Радиусы двух шаров равны $2$ и $4$. Найдите радиус шара, площадь большого круга которого равна сумме площадей больших кругов двух данных шаров. Ответ умножьте на $\sqrt{5}$.
\end{taskBN}

\begin{taskBN}{199}
\addpictoright[0.25\textwidth]{images/739433001147228n0}Радиусы двух шаров равны $\sqrt[3]{5}$, $\sqrt[3]{3}$. Найдите радиус шара, объем которого равен сумме их объемов.
\end{taskBN}

\begin{taskBN}{200}
\addpictoright[0.25\textwidth]{images/605565033194895n0}Радиусы четырёх шаров равны $2$, $\sqrt[3]{9}$, $\sqrt[3]{7}$, $\sqrt[3]{3}$. Найдите радиус шара, объем которого равен сумме их объемов.
\end{taskBN}
\end{document}