\subsection{Неориентированные графы, степени, изоморфизм}

\begin{itemize}

\item \textbf{Граф} (математическая структура для представления связей между объектами):

\begin{itemize}

\item Обозначается как $G = (X, \Gamma)$.

\item Состоит из:

\begin{enumerate}

\item[\(1^\circ\)] Непустое множество $X$ (множество всех вершин графа).

\item[\(2^\circ\)] Отображение $\Gamma$ множества $X$ в $X$ (правило, определяющее связи между вершинами).

\end{enumerate}

\end{itemize}

\item \textbf{Элементы графа}:

\begin{itemize}

\item \textbf{Вершина} (точка, узел графа): Каждый элемент множества $X$.

\item \textbf{Дуга} (направленное ребро): Пара элементов $(x, y)$, где $y \in \Gamma x$ (показывает направленную связь от $x$ к $y$).

\end{itemize}

\item \textbf{Множество дуг} (все связи в графе):

\begin{itemize}

\item Обозначается через $U$ (полный набор всех связей).

\item Дуги обозначаются буквами $\alpha$, $\beta$, $\omega$ (при необходимости с индексами).

\end{itemize}

\end{itemize}

\noindent\textbf{Определение.} Степень вершины $v_i$ (обозн. $d_i$ или $\deg v_i$) -- число рёбер, инцидентных $v_i$ (количество связей, примыкающих к вершине).

\noindent\textbf{Теорема 2.1 (Эйлера)} (фундаментальное свойство графов). Сумма степеней вершин графа равна удвоенному числу рёбер:

$\sum_i \deg v_i = 2q$

\noindent\textbf{Следствие 2.1(a).} Число вершин с нечётными степенями всегда чётно (важно для существования эйлеровых путей).

\noindent\textbf{Ограничения степеней:}

В $(p,q)$-графе (где $p$ -- число вершин, $q$ -- число рёбер): $0 \leq \deg v \leq p-1$ для любой вершины $v$

\noindent\textbf{Обозначения:}

\begin{itemize}

\item[$\bullet$] $\delta(G) = \min \deg G$ -- минимальная степень (наименьшее число связей у вершины)

\item[$\bullet$] $\Delta(G) = \max \deg G$ -- максимальная степень (наибольшее число связей у вершины)

\end{itemize}

\noindent\textbf{Определение.} Регулярный (однородный) граф (все вершины имеют одинаковое число связей): $\delta(G) = \Delta(G) = r = \deg G$

\noindent\textbf{Классификация регулярных графов} (по количеству связей у каждой вершины):

\begin{itemize}

\item[$\bullet$] Степень 0: граф без рёбер (изолированные точки)

\item[$\bullet$] Степень 1: компоненты -- одиночные рёбра (пары связанных вершин)

\item[$\bullet$] Степень 2: компоненты -- циклы (каждая вершина связана ровно с двумя другими)

\item[$\bullet$] Степень 3: кубические графы (каждая вершина имеет ровно три связи)

\end{itemize}

\noindent\textbf{Следствие 2.1(б).} Каждый кубический граф имеет чётное число вершин (следует из теоремы Эйлера).

\noindent\textbf{Специальные вершины:}

\begin{itemize}

\item[$\bullet$] Изолированная: $\deg v = 0$ (вершина без связей)

\item[$\bullet$] Концевая (висячая): $\deg v = 1$ (вершина с единственной связью)

\end{itemize}
