\subsection{Деревья}

Граф называется \textit{ациклическим}, если в нём нет циклов. \textit{Дерево} — это связный ациклический граф. Каждый граф, не содержащий циклов, называется \textit{лесом}. Таким образом, компонентами леса являются деревья. Существуют 23 различных дерева \( ^2 \) с восемью вершинами; они показаны на рис. 4.1. Известны и другие определения дерева. В теореме 4.1 отражены некоторые из них.

\textbf{Теорема 4.1.} Для графа \( G \) следующие утверждения эквивалентны:
\begin{enumerate}
    \item \( G \) — дерево;
    \item любые две вершины в \( G \) соединены единственной простой цепью;
    \item \( G \) — связный граф и \( p = q + 1 \);
    \item \( G \) — ациклический граф и \( p = q + 1 \);
    \item \( G \) — ациклический граф, и если любую пару несмежных вершин соединить ребром \( x \), то в графе \( G + x \) будет точно один простой цикл;
    \item \( G \) — связный граф, отличный от \( K_p \) для \( p \geq 3 \), и если любую пару несмежных вершин соединить ребром \( x \), то в графе \( G + x \) будет точно один простой цикл;
    \item \( G \) — граф, отличный от \( K_3 \cup K_1 \) и \( K_3 \cup K_2 \), \( p = q + 1 \), и если любую пару несмежных вершин соединить ребром \( x \), то в графе \( G + x \) будет точно один простой цикл.
\end{enumerate}

\(^1\) Джойс Килмер (1886—1918) — американский поэт. — \textit{Прим. перев.}

\(^2\) Можно предложить читателю нарисовать деревья с восемью вершинами. Как правило, одни деревья забывают рисовать, а другие рисуют несколько раз.

\textbf{Доказательство.} (1) \(\Rightarrow\) (2). Поскольку \( G \) — связный граф, то любые две его вершины соединены простой цепью. Пусть \( P_1 \) и \( P_2 \) — две различные простые цепи, соединяющие вершины \( u \) и \( v \), и пусть \( w \) — первая вершина, принадлежащая \( P_1 \) (при переходе по \( P_1 \) из \( u \) в \( v \)), такая, что \( w \) принадлежит и \( P_1 \), и \( P_2 \), но вершина, предшествующая ей в \( P_1 \), не принадлежит \( P_2 \). Если \( w' \) — следующая за \( w \) вершина в \( P_1 \), принадлежащая также \( P_2 \), то сегменты (части) цепей \( P_1 \) и \( P_2 \), находящиеся между вершинами \( w \) и \( w' \), образуют простой цикл в графе \( G \). Поэтому, если \( G \) — ациклический граф, то между любыми двумя его вершинами существует самое большее одна простая цепь.

(2) \(\Rightarrow\) (3). Ясно, что граф \( G \) — связный. Соотношение \( p = q + 1 \) докажем по индукции. Утверждение очевидно для связных графов с одной и двумя вершинами. Предположим, что оно верно для графов, имеющих меньше \( p \) вершин. Если же граф \( G \) имеет \( p \) вершин, то удаление из него любого ребра делает граф \( G \) несвязным графом, в точности две компоненты. По предположению индукции в каждой компоненте число вершин на единицу больше числа ребер. Таким образом, общее число ребер в графе \( G \) должно равняться \( p - 1 \).

(3) \textit{влечет} (4). Предположим, что в графе \( G \) есть простой цикл длины \( n \). Этот цикл содержит \( n \) вершин и \( n \) рёбер, а для любой из \( p - n \) вершин, не принадлежащих циклу, существует инцидентное ей ребро, которое лежит на геодезической, идущей от некоторой вершины цикла. Все такие рёбра попарно различны; отсюда \( q \geq p \), т. е. пришли к противоречию.

(4) \textit{влечет} (5). Так как \( G \) — ациклический граф, то каждая его компонента является деревом. Если всего \( k \) компонент, то, поскольку в каждой из них число вершин на единицу больше числа рёбер, имеем \( p = q + k \). В нашем случае должно быть \( k = 1 \), так что \( G \) — связный граф. Таким образом, \( G \) — дерево и любые две его вершины соединяет единственная простая цепь. Если к дереву \( G \) добавить ребро \( uv \), то ребро вместе с единственной простой цепью, соединяющей вершины \( u \) и \( v \), образует простой цикл, который также единственен в силу единственности простой цепи.

(5) \textit{влечет} (6). Поскольку каждый полный граф \( K_p \) для \( p \geq 3 \) содержит простой цикл, граф \( G \) не может быть одним из этих графов. Граф \( G \) должен быть связным, так как в ином случае можно было бы добавить ребро \( x \), соединяющее две вершины из разных компонент графа \( G \), и граф \( G + x \) был бы ациклическим.

(6) \textit{влечет} (7). Докажем, что любые две вершины графа \( G \) соединены единственной простой цепью, а тогда, поскольку (2) влечет (3), получим \( p = q + 1 \). Ясно, что в графе \( G \) любые две вершины соединены простой цепью. Если какая-то пара вершин графа \( G \) соединена двумя простыми цепями, то из доказательства того, что (1) влечет (2), следует наличие у графа \( G \) простого цикла \( Z \). В \( Z \) не может быть более трех вершин, так как иначе, соединив ребром \( x \) две несмежные вершины в \( Z \), получим граф \( G + x \), имеющий более одного простого цикла (если же в \( Z \) нет несмежных вершин, то в графе \( G \) более одного цикла). Таким образом, цикл \( Z \) есть \( K_3 \), и он должен быть собственным подграфом графа \( G \), поскольку по предположению \( G \) не является полным графом \( K_p \) с \( p \geq 3 \). Так как \( G \) — связный граф, то можно предположить, что в \( G \) есть другая вершина, смежная с некоторой вершиной подграфа \( K_3 \). Тогда ясно, что если к графу \( G \) добавлять ребро, то его можно добавить так, чтобы в графе \( G + x \) образовались по крайней мере два простых цикла. Если больше нельзя добавлять новых ребер, не нарушая для графа \( G \) второго условия из (6), то \( G \) есть \( K_p \) с \( p \geq 3 \) вопреки предположению.

(7) \textit{влечет} (1). Если граф \( G \) имеет простой цикл, то этот цикл должен быть треугольником, являющимся компонентой графа \( G \), что было показано в предыдущем абзаце. В этой компоненте соответственно две вершины графа \( G \) соединены. Все остальные компоненты графа \( G \) должны быть деревьями, но для того, чтобы выполнялось соотношение \( p = q + 1 \), должно быть не более одной компоненты, отличной от указанного треугольника. Это дерево содержит

Простую цепь длины 2, то к графу \( G \) можно так добавить ребро \( x \), чтобы образовать в графе \( G + x \) два простых цикла. Следовательно, этим деревом может быть или \( K_1 \), или \( K_2 \). Значит, граф \( G \) — или \( K_3 \) или \( K_1 \), или \( K_3 \cup K_2 \), а эти графы мы исключили из рассмотрения. Таким образом, \( G \) — ациклический граф. Но если \( G \) — ациклический граф и \( p = q + 1 \), то \( G \) связан, поскольку (4) влечет (5), а (5) влечет (6). Итак, \( G \) — дерево, и теорема доказана.

Так как для нетривиального дерева \(\sum d_i = 2q = 2(p - 1)\), то в дереве должно быть по крайней мере две вершины со степенями, меньшими 2.

\textbf{Следствие 4.1 (а).} В любом нетривиальном дереве имеется по крайней мере две висячие вершины.

Этот результат также следует из теоремы 3.4.
