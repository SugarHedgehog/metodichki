\documentclass[a4paper, 12pt]{extarticle}
\documentclass[a4paper, 12pt]{extarticle}
\usepackage{fontspec}
\usepackage{polyglossia}
\setmainfont{CMU Serif}
\newfontfamily{\cyrillicfont}{CMU Serif}
\setsansfont{CMU Sans Serif}
\newfontfamily{\cyrillicfontsf}{CMU Sans Serif}
\setmonofont{CMU Typewriter Text}
\newfontfamily{\cyrillicfonttt}{CMU Typewriter Text}
\setdefaultlanguage{russian}
\usepackage[left=1cm,right=1cm,
top=2cm,bottom=2cm]{geometry}
%%% Дополнительная работа с математикой
\usepackage{amsfonts,amssymb,amsthm,mathtools} % AMS
\usepackage{amsmath}
\usepackage{icomma} % "Умная" запятая: $0,2$ --- число, $0, 2$ --- перечисление

%% Шрифты
\usepackage{euscript} % Шрифт Евклид
\usepackage{mathrsfs} % Красивый матшрифт

%% Свои команды
\DeclareMathOperator{\sgn}{\mathop{sgn}}


%% Перенос знаков в формулах (по Львовскому)
\newcommand*{\hm}[1]{#1\nobreak\discretionary{}
	{\hbox{$\mathsurround=0pt #1$}}{}}

%%% Работа с картинками
\usepackage{graphicx}  % Для вставки рисунков
\graphicspath{{Изображения/}{image}}  % папки с картинками
\setlength\fboxsep{3pt} % Отступ рамки \fbox{} от рисунка
\setlength\fboxrule{1pt} % Толщина линий рамки \fbox{}
\usepackage{wrapfig} % Обтекание рисунков и таблиц текстом

%%% Работа с таблицами
\usepackage{array,tabularx,tabulary,booktabs} % Дополнительная работа с таблицами
\usepackage{longtable}  % Длинные таблицы
\usepackage{multirow} % Слияние строк в таблице
\usepackage{tcolorbox}
\newcommand{\D}[2]{\frac{\partial #1}{\partial #2}}
\newcommand{\DQ}[2]{\frac{\partial^2 #1}{\partial #2^2}}
\newcommand{\DM}[3]{\frac{\partial^2 #1}{\partial #2 \partial #3}}

\begin{document}
    \section*{Линейные уравнения}
    \subsection*{Приведение к каноническом виду по шагам}
    Исходное уравнение:
    \begin{tcolorbox}[height=1.5cm,valign=center, width=8cm, bottom=0.6cm]
        $$ \DQ{u}{x}+2\DM{u}{x}{y}+x\DQ{u}{y}+3x\D{u}{x}-4y\D{u}{y}=0.$$
    \end{tcolorbox}
    Его коэффициенты равны: $a=1$, $b=1$, $c=x$, $d=3x$, $e=-4y$
    \begin{enumerate}[leftmargin=*]
            
        \item Определим тип уравнения:
        \begin{tcolorbox}
        \begin{tabular}{l  l}
            \begin{minipage}{0.4\textwidth}\begin{itemize}[leftmargin=*]
                \item Гиперболическое если: $b^2-ac>0$
                \item Эллиптическое если: $b^2-ac<0$
                \item Параболлическое если: $b^2-ac=0$
            \end{itemize}\end{minipage} &
            \begin{minipage}{0.5\textwidth}
            Найдём это значение для нашего уравнения:
            
            $y^2-x^2y^2=y^2(1-x^2)$. Предположим,что $y\neq 0$.

            $\begin{cases}
                x<1, \to \text{Гиперболическое}\\ 
                x>1, \to \text{Эллиптическое} \\
                x=1. \to \text{Параболлическое}
            \end{cases}$
            \end{minipage}\\
            \end{tabular}
        \end{tcolorbox}
    

    \item Теперь решение будет зависеть от выбранных условий, для каждого своё 
    характеристическое уравнение.
    
    \begin{tabularx}{\textwidth} { 
         >{\centering}X 
         >{\centering}X 
         >{\centering}X  }
            \begin{tcolorbox}[height=1cm]
                Гиперболическое
            \end{tcolorbox}      
            \begin{tcolorbox}
            $\begin{cases}
            dy=(b+\sqrt{b^2-ac})dx,\\
            dy=(b-\sqrt{b^2-ac})dx,
            \end{cases}$
            $\begin{cases}
                dy=(1+\sqrt{1-x})dx,\\
                dy=(1-\sqrt{1-x})dx,
            \end{cases}$
            $\begin{cases}
                \int dy=\int(1+\sqrt{1-x})dx,\\
                \int dy=\int(1-\sqrt{1-x})dx,
            \end{cases}$
            $\begin{cases}
                y=x-\frac{2}{3}(1-x)^{\frac{3}{2}}+C_1,\\
                y=x+\frac{2}{3}(1-x)^{\frac{3}{2}}+c_2.
            \end{cases}$
            \end{tcolorbox}
            &
            \begin{tcolorbox}[height=1cm]
                Эллиптическое
            \end{tcolorbox}

            \begin{tcolorbox}
                $\begin{cases}
                ady=(b+i\sqrt{b^2-ac})dx,\\
                ady=(b-i\sqrt{b^2-ac})dx.
            \end{cases}$

            \vspace*{2mm}
                Нужно выбрать одно для решения
            
            $dy=(1+\sqrt{1-x})dx\\$
            $\int dy=\int(1+i\sqrt{1-x})dx$
            $y=x-\frac{2}{3}i(1-x)^{\frac{3}{2}}+C$

            \end{tcolorbox}
            &
           \begin{tcolorbox}[height=1cm]
            Параболлическое 
            \end{tcolorbox}
            \begin{tcolorbox}
                $ady=bdx$

                $dy=dx$

                $\int dy=\int dx$

                $y=x+C$
            \end{tcolorbox}

    \end{tabularx}

    Проводим замену переменных

    \begin{tabularx}{\textwidth} { 
        >{\centering}X 
        >{\centering}X 
        >{\centering}X  }
        \begin{tcolorbox}[height=3cm]
            $C_1=\xi(x,y)=y-x+\frac{2}{3}(1-x)^{\frac{3}{2}}$

            $C_2=\eta(x,y)=y+x-\frac{2}{3}(1-x)^{\frac{3}{2}}$
        \end{tcolorbox}&
        \begin{tcolorbox}[height=3cm]
            $ReC=\xi(x,y)=y-x$

            $ImC=\eta(x,y)=-\frac{2}{3}(1-x)^{\frac{3}{2}}$
        \end{tcolorbox}&
        \begin{tcolorbox}[height=3cm]
            $C=\xi(x,y)=y-x$

            $\eta(x,y)=y+x$ - выбирается произвольно,
            чтобы выполнялось условие 
            из следующего пункта.
        \end{tcolorbox}

    \end{tabularx}
    Нахождение производных от новых переменных по формулам

    \begin{tabular}{ll}
        \begin{tcolorbox}[width=4.5cm,height=1.5cm, top=-0.3cm]
            $$\D{u}{x}=\D{\xi}{x}\D{v}{\xi}+\D{\eta}{x}\D{v}{\eta}$$
        \end{tcolorbox}&
        \begin{tcolorbox}[width=4.5cm,height=1.5cm, top=-0.3cm]
            $$\D{u}{y}=\D{\xi}{y}\D{v}{\xi}+\D{\eta}{y}\D{v}{\eta}$$
        \end{tcolorbox}
    \end{tabular}
    
    \begin{tcolorbox}[width=11cm,height=1.5cm, top=-0.4cm]
        $$\DQ{u}{x}=\left(\D{\xi}{x}\right)^2\DQ{v}{\xi}+2\left(\D{\xi}{x}\D{\eta}{x}\right)\DM{v}{\xi}{\eta}+\left(\D{\eta}{x}\right)^2\DQ{v}{\eta}$$
    \end{tcolorbox}
    \begin{tcolorbox}[width=13cm,height=1.5cm, top=-0.4cm]
        $$\DM{u}{x}{y}=\left(\D{\xi}{x}\D{\xi}{y}\right)\DQ{v}{\xi}+2\left(\D{\xi}{x}\D{\eta}{y}+\D{\xi}{y}\D{\eta}{x}\right)\DM{v}{\xi}{\eta}+\left(\D{\xi}{y}\D{\eta}{x}\right)^2\DQ{v}{\eta}$$
    \end{tcolorbox}
    \begin{tcolorbox}[width=11cm,height=1.5cm, top=-0.4cm]
        $$\DQ{u}{y}=\left(\D{\xi}{y}\right)^2\DQ{v}{\xi}+
        2\left(\D{\xi}{y}\D{\eta}{y}\right)\DM{v}{\xi}{\eta}+\left(\D{\eta}{y}\right)^2\DQ{v}{\eta}$$
    \end{tcolorbox}
    \item Посчитаем матрицу 

        $\delta=\begin{vmatrix}
            \D{\xi}{x}&\D{\xi}{y}\\
            \D{\eta}{x}&\D{\eta}{y}
        \end{vmatrix}\neq 0 $

    \begin{tabularx}{\textwidth} { 
        >{\centering}l 
        >{\centering}l 
        >{\centering}l}
        \begin{tcolorbox}[width=5.3cm]
            $\begin{vmatrix}
                -1+(1-x)^{\frac{1}{2}}&1\\
                1-(1-x)^{\frac{1}{2}}&1
            \end{vmatrix}\neq 0$   
        \end{tcolorbox}
        &
        \begin{tcolorbox}[width=4.4cm]
            $\begin{vmatrix}
                -1 & 1 \\
                (1-x)^{\frac{1}{2}}& 0
            \end{vmatrix}\neq 0$
        \end{tcolorbox}
        &
        \begin{tcolorbox}[width=3.3cm]
            $\begin{vmatrix}
                -1 & 1\\
                1 &  1
            \end{vmatrix}\neq 0$
        \end{tcolorbox}
        
    \end{tabularx}

    \end{enumerate}
    
\end{document}