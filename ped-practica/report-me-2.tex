
\documentclass[a4paper,12pt]{article}
\usepackage[utf8]{inputenc}
\usepackage[russian]{babel}
\usepackage{geometry}
\geometry{top=10mm, bottom=10mm, left=10mm, right=10mm}
\pagestyle{empty}

\begin{document}

\begin{center}
\textbf{ОТЧЕТ}\\
о прохождении производственной педагогической практики\\
студентки 5-го курса математического факультета\\
Суматохиной Александры Сергеевны
\end{center}

\subsection*{1. Введение}

В период с 30 сентября по 9 ноября студентка 5-го курса математического факультета Суматохина Александра Сергеевна проходила педагогическую практику в МБОУ «Средняя общеобразовательная школа № 10». В течение практики было проведено 10 уроков математики в 5А классе. Для подготовки уроков использовался учебник Виленкин Н.Я. и др. — «Математика. 5 класс. Учебник (комплект из 2-х частей)».

\subsection*{2. Описание проведенных занятий}

В ходе практики были охвачены следующие темы:

\begin{itemize}
    \item Основные понятия геометрии: точка, прямая, плоскость.
    \item Углы: острые, прямые, тупые; измерение углов с помощью транспортира.
    \item Треугольники: виды по сторонам (равносторонний, равнобедренный, разносторонний) и по углам (остроугольный, тупоугольный, прямоугольный); свойства треугольников.
    \item Четырехугольники: прямоугольник, квадрат, ромб, параллелограмм, трапеция; их свойства.
    \item Площадь и периметр: вычисление площади и периметра прямоугольника и квадрата.
    \item Работа с дробями: введение в дроби, сложение и вычитание дробей с одинаковыми знаменателями, упрощение дробей.
    \item Геометрические построения: построение треугольников и четырехугольников по заданным условиям, измерение углов.
    \item Задачи на нахождение периметра и площади различных фигур.
\end{itemize}

В рамках практики были проведены две проверочные работы и одна самостоятельная работа. Проверочные работы были направлены на оценку усвоения тем по углам и треугольникам, а также по четырёхугольникам и дробям. Самостоятельная работа была посвящена задачам на нахождение периметра и площади геометрических фигур.

\subsection*{3. Оценка результатов}

Ученики проявили активность и интерес к изучаемым темам. Они успешно справились с заданиями из учебника и практическими упражнениями, направленными на развитие навыков работы с геометрическими фигурами и углами. Результаты проверочных и самостоятельной работ показали хорошие результаты и уровень понимания материала.

\subsection*{4. Личный вклад и выводы}

Прохождение практики позволило студентке значительно развить профессиональные и личные качества. Была освоена эффективная организация учебного процесса, делающего его увлекательным и доступным для учеников. Опыт работы в классе дал ценные навыки, которые будут использованы в дальнейшей профессиональной деятельности.

\noindent
\begin{tabbing}
\hspace{2cm} \= \kill
Подпись \> \underline{\hspace{4cm}} Суматохина А.С.
\end{tabbing}

\vspace{5mm}

\noindent
\textbf{Предмет: Математика}\\
\textbf{Класс: 5А}\\

\end{document}