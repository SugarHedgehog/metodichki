\documentclass[10pt]{article}
\usepackage[russian]{babel}
\usepackage[utf8]{inputenc}
\usepackage[T2A]{fontenc}
\usepackage{amsmath}
\usepackage{amsfonts}
\usepackage{amssymb}
\usepackage[version=4]{mhchem}
\usepackage{stmaryrd}
\usepackage{graphicx}
\usepackage[export]{adjustbox}
\graphicspath{ {./images/} }

\begin{document}

\title{Конспект: Ангармоничность колебаний и генерация гармоник}
\author{}
\date{}
\maketitle

\section*{Введение}
В данном документе рассматривается ангармоничность колебаний и генерация гармоник в нелинейных системах. Основное внимание уделяется спектральному представлению колебаний и влиянию нелинейных элементов на спектр.

\section{Линейные и нелинейные колебания}
\subsection{Линейный осциллятор}
Линейный консервативный осциллятор описывается уравнением:
\begin{equation}
\ddot{x} + \omega_0^2 x = 0, \tag{3.3}
\end{equation}
где динамическая переменная изменяется по гармоническому закону:
\begin{equation*}
x=A \sin \left(\omega_{0} t+\varphi\right). \tag{3.4}
\end{equation*}

\subsection{Нелинейные колебания}
В нелинейных системах колебания отличаются от синусоидальных и называются ангармоническими. При малых амплитудах колебания близки к гармоническим, но при больших амплитудах форма колебаний значительно отличается.

\section{Спектральное представление колебаний}
Любую периодическую функцию можно разложить в ряд Фурье:
\begin{equation*}
x(t)=\sum_{m=-\infty}^{\infty} c_{m} e^{2 \pi i m t / T}, \tag{3.5}
\end{equation*}
где коэффициенты $c_m$ определяются как:
\begin{equation*}
c_{m}=\frac{1}{T} \int_{0}^{T} x(t) e^{-2 \pi i m t / T} d t. \tag{3.6}
\end{equation*}

\section{Гармоники и ангармоничность}
Разложение Фурье можно записать в виде:
\begin{equation*}
x(t)=A_{0}+\sum_{m=1}^{\infty} A_{m} \cos \left(m \omega t+\varphi_{m}\right), \tag{3.8}
\end{equation*}
где $A_{m}$ и $\varphi_{m}$ определяются через $c_m$.

\section{Нелинейные искажения}
Коэффициент нелинейных искажений определяется как:
\begin{equation*}
\chi=\frac{\sqrt{A_{2}^{2}+A_{3}^{2}+A_{4}^{2}+A_{5}^{2}+\ldots}}{A_{1}}. \tag{3.11}
\end{equation*}

\section{Генерация высших гармоник}
Нелинейные элементы преобразуют спектр входного сигнала, что приводит к появлению высших гармоник. Рассмотрим элемент с нелинейной характеристикой:
\begin{equation*}
y=a_{1} x+a_{2} x^{2}+a_{3} x^{3}+\ldots \tag{3.12}
\end{equation*}

\section{Примеры нелинейного преобразования}
Примеры включают оптические эксперименты с генерацией второй гармоники и акустические эффекты, где на большом расстоянии от источника звука появляются высшие гармоники.

\section*{Заключение}
Ангармоничность колебаний и генерация гармоник являются важными аспектами в изучении нелинейных систем. Понимание этих процессов позволяет объяснить многие физические явления и их применение в технике.

\end{document}