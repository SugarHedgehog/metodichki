\textbf{Хроматическое число графа}

 — это минимальное количество цветов, необходимых для раскраски графа так, чтобы никакие две смежные вершины не имели одинакового цвета. Граф $G$ называется $n$-раскрашиваемым, если $\chi(G) \leq n$, и $n$-хроматическим, если $\chi(G) = n$.

\textbf{Известные результаты}
\begin{itemize}
    \item $\chi(K_p) = p$
    \item $\chi(K_p - x) = p - 1$
    \item $\chi(K'_p) = 1$
    \item $\chi(K_{m,n}) = 2$
    \item $\chi(C_{2n}) = 2$
    \item $\chi(C_{2n+1}) = 3$
    \item $\chi(T) = 2$ для любого нетривиального дерева $T$
\end{itemize}


\noindent\textbf{Теорема 12.1:} Граф двуцветен тогда и только тогда, когда он не содержит нечётных простых циклов.

\noindent\textbf{Теорема 12.2:} Для любого графа $G$, $\chi(G) \leq 1 + \max \delta(G')$, где максимум берется по всем порожденным подграфам $G'$ графа $G$.

\noindent\textbf{Следствие 12.2 (a):} Для любого графа $G$, $\chi \leq 1 + \Delta$.

\noindent\textbf{Теорема 12.3 (Брукс):} Если $\Delta(G) = n$, то граф $G$ всегда $n$-раскрашиваем, за исключением следующих двух случаев:
\begin{enumerate}
    \item $n = 2$ и $G$ имеет компоненту, являющуюся нечетным циклом;
    \item $n \geq 2$ и $K_{n+1}$ — компонента графа $G$.
\end{enumerate}

\noindent\textbf{Теорема 12.5:} Для любых двух положительных целых чисел $t$ и $n$ существует $n$-хроматический граф, обхват которого превосходит $t$.

\noindent\textbf{Теорема 12.6:} Для любого графа $G$ сумма и произведение чисел $\chi$ и $\bar{\chi}$ удовлетворяют неравенствам:
\begin{equation}
2\sqrt{p} \leq \chi + \bar{\chi} \leq p + 1,
\end{equation}
\begin{equation}
p \leq \chi \bar{\chi} \leq \left( \frac{p + 1}{2} \right)^2.
\end{equation}

\textbf{Заключение}
Представленные теоремы и оценки дают представление о сложности задачи нахождения хроматического числа графа и показывают, что даже для простых графов эта задача может быть нетривиальной.
