\documentclass[a4paper,12pt]{article}
\usepackage[utf8]{inputenc}
\usepackage[russian]{babel}
\usepackage{geometry}
\geometry{top=10mm, bottom=10mm, left=10mm, right=10mm}
\pagestyle{empty}

\begin{document}

\begin{center}
	\hfill \break
	\large{МИНОБРНАУКИ РОССИИ}\\
	\footnotesize{ФЕДЕРАЛЬНОЕ ГОСУДАРСТВЕННОЕ БЮДЖЕТНОЕ ОБРАЗОВАТЕЛЬНОЕ УЧРЕЖДЕНИЕ}\\
	\footnotesize{ВЫСШЕГО ПРОФЕССИОНАЛЬНОГО ОБРАЗОВАНИЯ}\\
	\small{\textbf{«ВОРОНЕЖСКИЙ ГОСУДАРСТВЕННЫЙ УНИВЕРСИТЕТ»}}\\
	\hfill \break
	\normalsize{Математический факультет}\\
	\hfill \break
	\normalsize{Кафедра теории функций и геометрии}\\
	\hfill\break
	\hfill \break
	\hfill \break
	\hfill \break
	\hfill \break
	\hfill \break
	\hfill \break
	\hfill \break
	\hfill \break
	\large{Психолого-педагогическая характеристика обучающегося
	}\\
	\hfill \break
	\hfill \break
	\hfill \break
\hfill \break
\hfill \break
\hfill \break
\hfill \break
\hfill \break
\hfill \break
\normalsize{
	\begin{tabular}{cccc}
		Выполнила студентка 5 курса & \underline{\hspace{3cm}} &                       &  Суматохина А.С. \\\\
		Руководитель пед. практики & \underline{\hspace{3cm}} & проф. & Прядиев В.Л.    \\\\
	\end{tabular}
}\\
\hfill \break
\hfill \break
\hfill \break
\hfill \break
\hfill \break
\hfill \break
\hfill \break
\hfill \break
\hfill \break
\hfill \break
\hfill \break
\hfill \break
\hfill \break
\hfill \break
\hfill \break
\hfill \break
\hfill \break
Воронеж 2024 \end{center}
\newpage

Боронин Дмитрий Вячеславович обучается в МБОУ «СОШ №10» с 2020 года. За период обучения в школе проявил себя как честный, целеустремлённый и уверенный в себе ученик с ярко выраженными творческими способностями и высокой ответственностью.

Дмитрий активно участвует в учебной и внеклассной жизни школы, демонстрируя стабильный интерес к литературе и гуманитарным дисциплинам. Он является участником школьной олимпиады по литературе, регулярно принимает участие в конкурсе юных чтецов и проявил себя в конкурсе сочинений «Пиши ещё!». В 2022 году он участвовал в конкурсе сочинений по сказкам Бажова. Дмитрий успешно применяет свои знания на практике, анализирует литературные произведения и высказывает своё мнение, обоснованное и уверенное.

Ученик активно вовлечён во внеурочную деятельность, посещает курсы «Русская словесность» и «Разговоры о важном», где показывает высокий интерес к обсуждению актуальных тем и глубокое понимание изучаемого материала. С 2022 года Дмитрий занимается в школьном театральном кружке, что позволило ему раскрыть свои артистические способности и уверенно выступать перед аудиторией. В театральных постановках он проявляет инициативу и осознанный подход, берясь за роли с полной самоотдачей и добросовестностью.

Дмитрий зарекомендовал себя как честный и ответственный ученик, пользующийся уважением среди одноклассников и педагогов. Он отличается целеустремлённостью в достижении своих целей, всегда доводит начатое дело до конца, проявляя дисциплинированность и внимание к деталям. Ученик поддерживает доброжелательные отношения с одноклассниками, всегда готов прийти на помощь, внимательно относится к мнению других, но при этом умеет отстаивать свою позицию, что говорит о его уверенности в себе и сформированной самооценке.

Рекомендации: Для поддержания интереса к учебной деятельности и творческому развитию Дмитрия рекомендуется давать ему задачи повышенного уровня сложности, направленные на развитие аналитического мышления. Следует продолжать поддерживать его творческую инициативу и стремление к самовыражению через театральные и литературные проекты, что поспособствует его всестороннему развитию и уверенности в своих способностях.
\end{document}