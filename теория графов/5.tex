\subsection{Числа Рамсея}
Широко известна следующая головоломка.

\textit{Доказать, что среди любых шести человек найдутся либо трое попарно знакомых, либо трое попарно незнакомых.}

\begin{enumerate}
    \item Напоминаем читателю (см. введение), что в тексте не все теоремы доказываются.
    \item По своим структурным свойствам. — \textit{Прим. перев.}
\end{enumerate}

В этих терминах головоломку можно сформулировать так:

\textbf{Теорема 2.2.} Если \( G \) — граф с шестью вершинами, то либо \( G \), либо \( \overline{G} \) содержит треугольник.

\textbf{Доказательство.} Пусть \( v \) — произвольная вершина графа \( G \), имеющего шесть вершин. Так как вершина \( v \) с любой из остальных пяти вершин смежна или в \( G \), или в \( \overline{G} \), то, не теряя общности, можно предположить, что вершины \( u_1, u_2, u_3 \) смежны с \( v \) в \( G \). Если какие-либо две из вершин \( u_1, u_2, u_3 \) смежны в \( G \), то вместе с \( v \) они образуют треугольник. Если никакие две из них не смежны в \( G \), то в графе \( \overline{G} \) вершины \( u_1, u_2, u_3 \) образуют треугольник.

Обобщая теорему 2.2, естественно поставить вопрос: каково наименьшее целое число \( r(m, n) \), для которого каждый граф с \( r(m, n) \) вершинами содержит \( K_m \) или \( K_n \)?

Числа \( r(m, n) \) называются \textit{числами Рамсея} \(^1\). Ясно, что \( r(m, n) = r(n, m) \). Задача, связанная с нахождением чисел Рамсея, остается нерешенной, хотя известна простая верхняя оценка, полученная Эрдёшем и Секерешем \(^1\):

\begin{equation}
r(m, n) \leq \binom{m + n - 2}{m - 1}.
\end{equation}

Постановка этой задачи вытекает из теоремы Рамсея. Бесконечный граф \(^2\) имеет бесконечное множество вершин и не содержит кратных ребер и петель. Рамсей \(^1\) доказал (на языке теории множеств), что каждый бесконечный граф содержит \( \aleph_0 \) попарно смежных вершин или \( \aleph_0 \) попарно несмежных вершин.

Все известные числа Рамсея приведены в табл. 2.1 (взята из обзорной статьи Гравера и Якелл \(^1\)).