\documentclass[a4paper, 14pt]{extarticle}
\usepackage{ifxetex,ifluatex,ifpdf}

\ifluatex
    \usepackage{fontspec}
    \usepackage{polyglossia}
    \setmainfont{CMU Serif}
    \newfontfamily{\cyrillicfont}{CMU Serif}
    \setsansfont{CMU Sans Serif}
    \newfontfamily{\cyrillicfontsf}{CMU Sans Serif}
    \setmonofont{CMU Typewriter Text}
    \newfontfamily{\cyrillicfonttt}{CMU Typewriter Text}
    \setdefaultlanguage{russian}
    \else  
    \ifpdf
        \usepackage[english, russian]{babel}
        \usepackage[T2A]{fontenc}
        \else
        \ifxetex
        	\usepackage{fontspec}
        	\defaultfontfeatures{Ligatures={TeX}}
        	\setmainfont{Noto Serif}
        	\setsansfont{Noto Sans}
        	\setmonofont{Noto Sans Mono}
        \fi
    \fi
\fi

\usepackage[left=1cm,right=1cm,top=2cm,bottom=2cm]{geometry}
\usepackage{amsmath}
\usepackage{hyperref}

\begin{document}
\input{titleForLab4.tex}

\tableofcontents
\newpage

\section{Зашифрованный текст}

АЕДЭАЪПОСЬЩНПЮИУЪАПИЬУДБСАЫЦЪЖЯТЯЬАДАВЬУНФЬИЧЩЕЦЭАЗЦДДЧСБУГЦШЛОДШУЯРХТИЕСАБЪЛШПТРЧЮЩАЕЯЩААЭОЦУКЖЪОБЪЪХЩЯБИИУУЛПЬЩБЦНРДЪАЧЯЬУКБУОТЫЗБДЕАЮЕЦЧТХВДБЩОУЦАЖЪОЯНЫБЕЕЫЭЕЫЧСЩАИАЫИФУЧЕФЛЛЭЕДАИЫОЖГЧМХЪЫЮОСЪХЫФЬИПЦЯЧЫТЪРВЩЬИПЬЩЩЯПХДЪАЧЯШЮЖБЩЬЧЬЭФЬИПДЪЧЫЛШПЕЦШЛОДШУЭЕБЬЫБЩОУЦПЩЯКШАЖБЮЯФЬПЩЬНМУЖГЫЦХЮИРАВЪЪПЩЯТТОИЯЫТЯЦКЩГНШЧШЬЬАЖУДЫОПЪЭЧЕЧЕБУНАЧКШЬНЦПТМРШЩАТХВДЫЕЕАЧАЩЧЗФУВЫОРРЭЪЩЬЕААСЩЯТТЬЭФСШШУЭВЮИЯЬЫЩЧПЯЬОЩЯСМЦНЫЦГЪЯЕЦЩЕЫЦЧЛФЛЪРЪЪЫМФШЧБЯУИУИЕСЛХЭАУШАЩЬАЮЧБЪТЪУАЕЭИДБЯТШРКБЭЧШЮВЩЭАЗЦДРЭЕДОДЫЦММЬЩБЮУФЬЭФЬИХОЖВПРРЯФЫЬСБЪЛЯФНБКЖГЧБЪЪФЫАДРЯШЪХЕАЦИЕФММРЯФЧМЪЮЭУЦАЫЭФИАЕДЭАЛФСЩЦНЖЯТЯЬАДАВТДШДАНЪЮКЫПГЯУЬФАЫВЮКФЬОТЧАЫЯТЯЬАЕФЛЛЭФЩЯОЪЪЛЭФНШНКЩГНШЧШЯЫЖХЯАЯФТЛЫЗБЧЗТЬЫДАВХЭДБФПЯЬГРЖЛХЭДБФАУЪЕВЮОЬКРЮФНЫЬЪЫЩИЫУЖГЫИЧРЕШЯТТУДАЫЕЫОЯАПЧХЭАЩЪОАШЪШЬЕХРБЮНЧРУКЫЯПЪШУЬЫВРЭАЩАЕДЭАЪЧВЫОЛЪФБМЯЛБРРРХЕЦПНШЦБЖЩЬБАЗЩЭЕФЦОЫЬЕТЬЪАЬОЬТЪЮФОАРЕЩЬИШЧЕДЭОАОАЦУРВСАИЫБЭОИЕОХАЯЕЛШИЧЪЪАЧЯВЫЗФСЛХЭАУЪРЪГЪДЯАЬЦКЩГНШЧШУСЛПУКДОСЯУЫДАВЪЩЗЩПЛШХШКЧИЧОЫФЕИФЬИЕЧЖХЭАУДЕЭУАВЮОЕУИДПТХВДЫШАШЮЖБЩЬЧАЪЕЯЯТЬЬГЫМЫЬГГПЗЫЬЕХЮАЧЦАВЮОЕУИДЫВТЧВТЕАПЯЪИЬОЭЬЬЫЕЕАЧАЩЪРЪГЪДЯЫШЫЗБДЕАЮФВЮОЬКРЮФНЫЬЬБЧСХШУДШОДЬЯУЧСБРЪАЬОУЬЖГЫИЧРЕШЯТТОАЬЭЕЯУДЫОКЪЭКГЫЛПЦЛВЮАТШЪАЧЯЮУЗЩСОЧЧАЦФДХЭАУРОХРФИУЕШЮКЦЧИЪПЛЛФНШНИВЫРБОЕЕУЫДОЗФЦВЭУПЩЬИШЦГАЫГШВЫГБГШВЖГЫЦХЮИБСТХВДЫШАЯОЯГПБРЯФЦПЕБЮЧЫЯОТУЗМФНАЯЭЖФТАНЭГФЗВШУЕПТХЦДЭФНХЪДБЧДХНКЩЩЬЫЬИЕЧОАЬЩЩЬНЪЮКЫШОЫЮКГБИЯЬЭФЬИПЦАЬТОБЬЭЮФНШНКЩГНШДЪДШИДАИЕЮОШЮКЦЦАТЦИУАОБРАШПТХВДЫЕЕАЧЕЧЫУАЯЗБЧСБРШЕЮЕСЬЭФЬИШХШЪПЗЖЦБФШЕУЬКЩГНШДЪДШИЬВШГПКБУЗЫЯТШЧШЯЪРЪЦЯЦЫДШЯЪЮКНЪЮКЫЬАФУЮАЫСБЦТЪЫНЪЩАЛЬОАЯАШЫЛУЬЭЩЕНЪЮКЫЧТФЧШЛФСБРЛДАОШЩЕДАИБУНАЫЛЪСАЫЧЗУЬКБСЛХЭАУПТРЧЮЩЫТГЦДФЬСЪРФИЧТХВДЫЕЕАЧАИСОЧЩЕЭЬОАЯЪЫЪРЪЦЯЦЫДШЯЪЮОТРЧКЩГНШДЪДШОХЦЯШФЛШУАЮЧАУЪЪЧПТЬЬЬЖАБМЯУЫЦГЪЯЕЦЩЕЫКЖГЫММЕВЩЬНМЩАЮЧКВЮКФЮНМЩИВЫСЪПЕЯСТЪРЗЩЭЯЩОБЖЯТРЭЕЦШИЩОБВЮАТЦВБЯОСЦЗФНТШХББЭПЪЭЪААОТЫЕЯФСБАТЪЯПЭАШЕПЦШЦЛДААЫЬЭЪЧПЯЦТЕЫМЪЯЫЩЩЬЫКЪЪЫМЮЬДЩЬТМАИЕПНЪРБЫЧНФЦЭЫУУРШУАЛЕШХЫЩЩИПОЬГФГРЯФЫБЗЭКГБТУБЦГЩАЬТКИБШУОХШЦЫДАЧЛТТОБЬЭАЫСБИАЯЫДВШУАЫСБИПЕЫПЪХЭБЩЯХЯЯАПЧШЯЪЮКНЪЮДЫЦИБИЯФАРРЯФЕЮУФОАЦЮЕЬУДЫЬАШВАААЕУЪШКЧЮТАИЕПНЪРБЖЧЗРЩЪАБВАШЛЛПЕЫУАДЪРРРДБЯТШЬЬГЫМЫАЦГЫЛЛРКЩГНШЧЪЫТРРУКЦЦАШЩЕЬПМХЭЧЩЭОАЯУЪЫТЪЪШУЯНШФШЩАЗРЯЗФАЫШЬЩЮФГЖОЪЕШОЫЮКГБИЯЬЭФЬИХЦЯЧЫТЪРВЩЬИХЗБДЪЛВОКФДИОЬЩДЩУЦЦЭФЬИХЦЗЩЭОЫЯКЩГНШДЪДШИДАИЕЮОШЮКЦЯОТЪЪЯФНЫОЧЕФХЫЦБФОВЭНЪЕЯЯЮЪЕШБКБЬГАПУЖЭЕЕФХЫЦПЩЯКЪЦЗЩСОЭМОЫЧАВЪЕЦФНЛЪШЬСИБЦЧЕФХЫЦБЫОВЭНЪЕЯЯЮЬБФЦАБУВЩЭНРАПАЫТХВДЫЕЕАЧЕЧЫРРХЭЫАИПЬЩНФСБРШЦБСЭЬЭЫОХУШЕХПЛШХШКЧИЬЦЗБСОШЗББЬОЬЦБЫЪЕЯУЫБСАПЯЪИЬИЩОЩРЯТЯЬЗФЯПЯЬИЕЮАЫНЪЕЯЯЮЬГЫЮУТЩЪДАЕАЯЪЯФЕШЮЖБЩЬЧЬЭФЬИХРЕЕУЕЭИДБСЗПЯЕЫЯТЯОДЩЧЛШУЪЛПСБЦЯФСИАЦКБАМЫЬЮЩЯТТОМФШТЪЪЕЦСЛШНЦНЧХЫОЫБЯТВЫДБЯТЛЯЪИЬИЩЦАОВФХЧКЫСНЪЮКПФЕЮЪШЪАИЖУИЪЫГЪЫЗЫЭЕЫУДЫОНРЫЗЫЭЕЯАЗБСЕЫИТЪЫНЪЩАЛФСЩЬЬБЮАЧРАЕЧЯЯКДЪПКЯУЫЫАНЪБААПНАЬЭБЧСШЮКЩЭЫЫОВЫЕИХЦЫЩФСЮЬИБРНЪЮКПЧНГЪШДАРВЧКЖЮЫЮЬБЖЪАБУВПЯКРНИВЫСЪПДБЯТЛЧЭФЩИГЦБФДИПЫЕЮКЗЪРШЕФЛХЦКЩГНШЧАЦРОЭУЪМЧРЪЧЕЯЯММЮВЩЪОФЯЪИЬИЩЬАВЫНШЩШТАСЪРЕЪБПЫЬИЕКТХВДЫЕЕАЧАИЯРХТИЕСИБУНАЫЛЪСАЫЦНРЭАЫЧДХНКЩЩЬЫЬИЕЧВЩЬКБЮЫДХШШФИАЯЭБСАЫККЩГНШДЪДШИХЮЗЩУСБРШ

Наиболее вероятная длина ключа 6

Индекс совпадения: 0.03976814394396812

Разбиваем зашифрованный текст на 6 частей:
1: АПППСЯАЬЭЧШЯСПАЭЪЩУЦЧУДЧЩЪЕЧЫФАЧОЬЫЬЯЧЩЬЫШЭЩЯЮЬЫАЯЫГЬОЧЧПАЕЧОЬЯСЮЧЯЦЩФЫЯСШЧАЯЭЭЭЦЮЬПЬФЧАХФЧЦАФЯААПАЬЯФЯФГЫФЧАФЖФЮФЩЫЯЫПЪЬНЯЫАЧФРПЩЭЬЬФЬЭУЫОШЧСЪЯГСОАПЧЕЧДЮПШЩЯЫПЮЮЫЕЬЕЪЯДЮФЧШЧЬЫЯЭОЫЮЧСФРУЧФЫУЦЬЫБЫСШППЯФФФПФЧЩЧЬШБЬТФГШЮЦАПЕЫЧЮЬПШГШПЯЪЫКЬЫЫЬЫЕЧФААЫЧСПЫЬЧЕСЬЪЫОГШФЧПАЦЩЫЬЧЮЫСЭЯШЮЯНЭАФЯПАЧЫЩЫЬПЧУЛЩФБТАШЫТЫЫЫЫЩПКЦАЮЮЬАЧПЧБПЪЯЫЫГТЦПЭЫЯААФШБЬЫЬЪДЩЬЭГШЮЯФФОЯБПФЯСЧФСФОЯЦЭЫЕЫАФБОПЧСЬЪСЬЯЯЮЯЮАФЩЬУСЯЧПСАЯШСЧЯЯЬВСФАЫЭОЭСЫФЮЧПАПЧЭЕФРЧАЮЪЯЫЯЩДКФГРЧЯЪЬЫАБКЕЯСЫЦЧЩЧЮФСГШУ

Всего символов: 456
А = 33
0.07236842105263158
Б = 8
0.017543859649122806
В = 1
0.0021929824561403508
Г = 10
0.021929824561403508
Д = 5
0.010964912280701754
Е = 11
0.02412280701754386
Ж = 1
0.0021929824561403508
З = 0
0
И = 0
0
К = 4
0.008771929824561403
Л = 1
0.0021929824561403508
М = 0
0
Н = 2
0.0043859649122807015
О = 11
0.02412280701754386
П = 31
0.06798245614035088
Р = 4
0.008771929824561403
С = 22
0.04824561403508772
Т = 4
0.008771929824561403
У = 8
0.017543859649122806
Ф = 38
0.08333333333333333
Х = 1
0.0021929824561403508
Ц = 11
0.02412280701754386
Ч = 44
0.09649122807017543
Ш = 18
0.039473684210526314
Щ = 17
0.03728070175438596
Э = 17
0.03728070175438596
Ь = 35
0.07675438596491228
Ы = 46
0.10087719298245613
Ъ = 12
0.02631578947368421
Ю = 21
0.046052631578947366
Я = 40
0.08771929824561403

Индекс совпадений: 0.0625865650969529
2: ЕОЮИАТВИАСЛРАТЕООЯЛНЯОЕТООЕСИЛИМСИТИПЯЬИЛЛЕОКЯНЦВТТНАПЕКТТЕЗРЕТШИПСГЕЛМУЛАБЕТЧАЕМУИРСНБДЕММАЕСТВНГЫОТЛОННЖТЗВПЛАОНИИТЕЧОЕЧПВЕВБРНЬЕЕООИОРБХИЯЛРАНЛСВЛИИЖЕОТАЬЯМЗАОВАОЕРЫЕОНСОСОИТЕКЛАЯОДОЕИНРЫВИГГЦТАБЕОНТЗТНДЬОНОИИОННИОАОТЕУСЕИЗЕНИКТРДНАСНОЛНТСОИЛЗЛТТСТЕООРДТНОЛАТБГЕМНКНСТЯТИАОТПОСПЦАПМЬМТННУЕИГЗУЬУДОСДСПЯЧНИРУЕАЕЮНЗВЕРТМЛНРАМОТНЗЫГОИИТИЛИУИОНИООНХВЯКУХКОАНИХВЯАНТЕРИССХЛИООЕАИТПАЯУЕЕЬИЕЗТЛСИМТТЛХТТИФНЕИГЕНЕЕНСАЯКННСЫИСННРЫАКСТИИЗЛНОРМОИНСПТЕРИЛНДЬВЫИАНИС

Всего символов: 456
А = 28
0.06140350877192982
Б = 6
0.013157894736842105
В = 13
0.02850877192982456
Г = 8
0.017543859649122806
Д = 8
0.017543859649122806
Е = 43
0.09429824561403509
Ж = 2
0.0043859649122807015
З = 11
0.02412280701754386
И = 48
0.10526315789473684
К = 9
0.019736842105263157
Л = 24
0.05263157894736842
М = 13
0.02850877192982456
Н = 45
0.09868421052631579
О = 46
0.10087719298245613
П = 11
0.02412280701754386
Р = 17
0.03728070175438596
С = 26
0.05701754385964912
Т = 43
0.09429824561403509
У = 10
0.021929824561403508
Ф = 1
0.0021929824561403508
Х = 6
0.013157894736842105
Ц = 3
0.006578947368421052
Ч = 4
0.008771929824561403
Ш = 1
0.0021929824561403508
Щ = 0
0
Э = 0
0
Ь = 8
0.017543859649122806
Ы = 7
0.015350877192982455
Ъ = 0
0
Ю = 2
0.0043859649122807015
Я = 13
0.02850877192982456

Индекс совпадений: 0.0661068790397045
3: ДСИЬЫЯЬЧЗБОХБРЯЦББПРЬТАХУЯЫЩФЛЫХЪПЪПХШЧПШОБУШФМХЪТЯШЖЪБШМХАФРАТШЯЯМЪЫЪФИХЩЪЭШШЗДМФХРББЪРАМЪЫДЩЯТЪЯВТЯЛЪШШХЛТХЯХУЬЫЫЧТЫХАХРЪРДЫМРШБФТЬАШАВЭАЧВХЪЬШПЯЪШЧФХЭЕХШЧТЫЫЧЕТПЭАЪШАЬЫХДБУЧТЯЪПТЮЧХХШЪШБДЭШШШХХЯРБТААВХХХЫАЪЫЯПБШШДШТБХААБСШЖУШЬБШЪШЪФБЪАУЪФБШБЪУХРГЪХАЧАЪШРШХШУЬМЪЫММВМЪЪЩРЩТСШЪТБЭШЫЯЪЫЮМЪФРШПРЭБТОАББВБЪХШЪБРФЬШУТЪРАЫРШЫЛШРШХАЪШРШЖЫЯХЪХВОЦХЫШДШТЫЫЭЮБЖЫЪЭВЛБЫЭЮБРХАРПБЭУШЬШЬЯПЩЯЯЫЮТАШЧХЭПЯШБАЫТЪШЫВЛЩХЪЮЖЪЫРЯЫЪЩЧЯЯЪАШЫХЮЪГВЮБРЪЛГПЪХШЭЪМФЩШЪЫХАХБЪРХЫЩДАЫШХБ

Всего символов: 456
А = 27
0.05921052631578947
Б = 33
0.07236842105263158
В = 10
0.021929824561403508
Г = 3
0.006578947368421052
Д = 9
0.019736842105263157
Е = 2
0.0043859649122807015
Ж = 5
0.010964912280701754
З = 2
0.0043859649122807015
И = 2
0.0043859649122807015
К = 0
0
Л = 7
0.015350877192982455
М = 12
0.02631578947368421
Н = 0
0
О = 4
0.008771929824561403
П = 13
0.02850877192982456
Р = 24
0.05263157894736842
С = 3
0.006578947368421052
Т = 21
0.046052631578947366
У = 10
0.021929824561403508
Ф = 12
0.02631578947368421
Х = 42
0.09210526315789473
Ц = 2
0.0043859649122807015
Ч = 12
0.02631578947368421
Ш = 53
0.1162280701754386
Щ = 10
0.021929824561403508
Э = 13
0.02850877192982456
Ь = 12
0.02631578947368421
Ы = 34
0.07456140350877193
Ъ = 48
0.10526315789473684
Ю = 8
0.017543859649122806
Я = 23
0.05043859649122807

Индекс совпадений: 0.06194213604185903
4: ЭЬУУЦЬУЩЦУДТЪЧЩУЪИЬДУЫЮВЦНЭАУЭОЪХЦРЬДЮЬДПДЬЦАЬУЮЪОЦЧУЭУЬРВЧУЭАЬУЬЬЦЯЦРШУЭЬТИРЮЦОЬЬОЯЪКЪЯЦРЮЭЭЦЬДЮУЮЧЬЭЪНЧЯЫЬЭЬЭЪКЬУРУОЭШРУШЭЭОЯХЦАЦЬТРЧОСОЯЪЫЭГЦЧУУЩХОЬЭУУВЮАЬЬЬЦУЧЯЬЧГЫЮКЬШЬРЬРОУЭЦШУЧЭРЮПНООУЦВВЮВОЯЮУЯНШЦЪНЬЬЮЮЬЦЬНДАЮЦРВЧЯРЬХЦЬДВУЧЦЯЮУЦЩЯЬЮЧРЩУСЬЭЧЦРВЧЩЯЦЯЧДЦУЪЬЯЯКЕЩЮЩПРОЭОЦЦХЭЫААЦЬЦЯКЬАРЦШХОЯКЦКХЧЬИШИХЯЯЮИЯОУВЪАРЩШУРЬАРЧУЩЭЯЪФЯЬОЮЬЦРЗОЬЦЦЯДАЮЪОЦНЪЬЭЦЦМЪЪЦЦНЬУАВЧХЬРЬШХЦЗЦУЯОЬЬНЬЩЯЮЬРИЯОУЦЦЬОЪНОЫЯЦЧЮЪУЫУЫАИЩЬРКУБЬЮОЦЬЮЪЧЬУНПЧЦЫРЦЧУЧЮЯЬЩРЬВЧТУСЭНЬЬХЯКДЮР

Всего символов: 456
А = 14
0.03070175438596491
Б = 1
0.0021929824561403508
В = 12
0.02631578947368421
Г = 2
0.0043859649122807015
Д = 11
0.02412280701754386
Е = 1
0.0021929824561403508
Ж = 0
0
З = 2
0.0043859649122807015
И = 7
0.015350877192982455
К = 9
0.019736842105263157
Л = 0
0
М = 1
0.0021929824561403508
Н = 12
0.02631578947368421
О = 25
0.05482456140350877
П = 4
0.008771929824561403
Р = 27
0.05921052631578947
С = 3
0.006578947368421052
Т = 4
0.008771929824561403
У = 42
0.09210526315789473
Ф = 1
0.0021929824561403508
Х = 11
0.02412280701754386
Ц = 48
0.10526315789473684
Ч = 25
0.05482456140350877
Ш = 10
0.021929824561403508
Щ = 13
0.02850877192982456
Э = 24
0.05263157894736842
Ь = 61
0.1337719298245614
Ы = 9
0.019736842105263157
Ъ = 20
0.043859649122807015
Ю = 27
0.05921052631578947
Я = 30
0.06578947368421052

Индекс совпадений: 0.06555863342566944
5: АЩЪДЪАНЕДГШИЛЮАКЪИЩЪКЗЕДАЫЕИЧЕЖЫЫЯВЩЪЖЭЪЕШЫПЖПЖИПИКШДЧННШДАВЪСЭЭЫОНЕЧЪЧИААЪДКВДДЩЭЖФЛЖФШИЯЭФАНАШКЬКААФЛКШАЗЫДГДЕРЪЖЕДЯАЪБКУААЛЛЕБЗОЪЪЕЕААИЕЪЗАЪКШКЫЗШЫИААИДЖЪЬГЕАИВЪЬАЪЗФРЬУЯЪЖЕАДКЛЪЗААФКЛИЕЗПГЫЖИДЯФЧЗЭЭУДДКИЩККЭАЭКЪИКИАДЕЗШЭШБКЪШЗШЯЪКЮТААЭКШЛЕНАКАЮДФДАЕЪЯЪКЪЯАЪЬУЕЖВАКИЕЗБЕБВЗБЪЕТШЛЭТЫЪДИБЭУЫЬФГГИШЛЭАУПЭЯЪДЯФАДАШИБЪЛАДЬЦКЪКЕЧУШШЗЩЪКЭЯВБКЩЭЗКЪИКЪЧБЪЕГЕПЗОЕШЧБЪБВПДЕЭЩШЭЕШЗББЫЪЩЗИЪГЪЪЖЭЕДЕДЪЯКЮМЕЦЫДЪАККШИЗДЗЗТАЬАДЫАЭКВЫИКШКБВИДЭБЕШКАЪЕВЪАШЕИДАИНААКИКШЭКЪЗШ

Всего символов: 456
А = 47
0.10307017543859649
Б = 16
0.03508771929824561
В = 11
0.02412280701754386
Г = 8
0.017543859649122806
Д = 31
0.06798245614035088
Е = 34
0.07456140350877193
Ж = 12
0.02631578947368421
З = 22
0.04824561403508772
И = 28
0.06140350877192982
К = 39
0.08552631578947369
Л = 11
0.02412280701754386
М = 1
0.0021929824561403508
Н = 7
0.015350877192982455
О = 3
0.006578947368421052
П = 7
0.015350877192982455
Р = 2
0.0043859649122807015
С = 1
0.0021929824561403508
Т = 4
0.008771929824561403
У = 7
0.015350877192982455
Ф = 10
0.021929824561403508
Х = 0
0
Ц = 2
0.0043859649122807015
Ч = 8
0.017543859649122806
Ш = 28
0.06140350877192982
Щ = 9
0.019736842105263157
Э = 23
0.05043859649122807
Ь = 8
0.017543859649122806
Ы = 15
0.03289473684210526
Ъ = 46
0.10087719298245613
Ю = 4
0.008771929824561403
Я = 12
0.02631578947368421

Индекс совпадений: 0.05831602031394276
6: ЪНАБЖДФЦДЦУЕШЩАЖХУБАББЦБЖБЫАЕДГЮФЧЩЩАБФЧЦУБЩБЩГРЩЯЩЬЫЕАЦЩЫЩЫЩЩФВЩЩЫЦЛЪБЕУЮУББЩРЫБФВЫЯГЫЪЕФУИЛЖДДЫФФЫЕЩЭЩЯЯБДБРБВЮЫГШААЩШЮЫЬЩЪЪБЦЖЩЫАЮЩДЦИЕЛАФУДЩУДДЩКФЕУВДЫБЕГГХВДТИЫЩДБВЮБДУАГШЬЫГВАЩЦУИЦЛВЕФЩАГГБЫГЦЫМЖГЕЭБЩЕЩЫГФЬЮЩДЕЦУШЫЧБЕФЪФЩДГЫЯЦЮЫАЪЛШЩЫЛДДАЫБУЩФИЫИЭЫЦЮЩДШЮЧЖЫЦГЩЮФВЯЩЖЦВБФБАЯЪЕДЪЕЩЪЩЕЫЫАЩГЫБЩБЦТАЯАЕБАЮЫФЕЦЫАКЕЖАЛДБГГЩЫЦЬЩЪУЩФЮЕГФЧЩДФДФЩЩДЕЦЯЕФЕШАЕЩЩЫЦЬЕЫЕФЩАЫЧЫНЦЫХКББЫБИРФЕЕЫДЯБФЕБЫЩЛФБЩФЦНББИОЫПЪЪЫЫЫБЪЛБЕЪЫАБЩЫЩБПДЖЖПВБФФЮЕЩЦМЯЩИВТЪЕЫИЕАЫЫЩЕБШБЩДЩ

Всего символов: 455
А = 26
0.05714285714285714
Б = 45
0.0989010989010989
В = 12
0.026373626373626374
Г = 19
0.041758241758241756
Д = 27
0.05934065934065934
Е = 35
0.07692307692307693
Ж = 11
0.024175824175824177
З = 0
0
И = 10
0.02197802197802198
К = 3
0.006593406593406593
Л = 9
0.01978021978021978
М = 2
0.004395604395604396
Н = 3
0.006593406593406593
О = 1
0.002197802197802198
П = 3
0.006593406593406593
Р = 4
0.008791208791208791
С = 0
0
Т = 3
0.006593406593406593
У = 14
0.03076923076923077
Ф = 30
0.06593406593406594
Х = 3
0.006593406593406593
Ц = 24
0.05274725274725275
Ч = 6
0.013186813186813187
Ш = 9
0.01978021978021978
Щ = 56
0.12307692307692308
Э = 3
0.006593406593406593
Ь = 6
0.013186813186813187
Ы = 50
0.10989010989010989
Ъ = 16
0.035164835164835165
Ю = 14
0.03076923076923077
Я = 11
0.024175824175824177

Индекс совпадений: 0.06555246950851346

\section{Процесс расшифровки}

Таким образом мы получили 6 текстов с моноалфавитным шифрованием. Не составит труда расшифровать их методом частотного анализа текста. 
Не исключая погрешностей, автор свёл расшифрованные тексты воедино. Увидив вырисовывающийся текст, автор заменил неправильно определённые буквы в каждом из 6 подтекстов. И заново свёл его. 


По получившемуся расшифрованном тексту автор вывел ключ, которым был зашифрован текст. 

Ключ: ПАРОШФ

ИНФОРМАЦИОННАЯБЕЗОПАСНОСТДОЛЖНАДОСТИГАТСЯЭКОНОМИЧЕСКИОПРАВДАННИМИМЕРАМИВРЕФЕРАТЕОПИСИВАЕТСЯМЕТОДИКАПОЗВОЛЯЮЩАЯСОПОСТАВИТВОЗМОЖНИЕПОТЕРИОТНАРУШЕНИИИБСОСТОИМОСТЮЗАЩИТНИХСРЕДСТВОСНОВНИЕПОНЯТИЯУПРАВЛЕНИЕРИСКАМИРАССМАТРИВАЕТСЯНАМИНААДМИНИСТРАТИВНОМУРОВНЕИБПОСКОЛКУТОЛКОРУКОВОДСТВООРГАНИЗАЦИИСПОСОБНОВИДЕЛИТНЕОБХОДИМИЕРЕСУРСИИНИЦИИРОВАТИКОНТРОЛИРОВАТВИПОЛНЕНИЕСООТВЕТСТВУЮЩИХПРОГРАММВООБЩЕГОВОРЯУПРАВЛЕНИЕРИСКАМИРАВНОКАКИВИРАБОТКАСОБСТВЕННОИПОЛИТИКИБЕЗОПАСНОСТИАКТУАЛНОТОЛКОДЛЯТЕХОРГАНИЗАЦИИИНФОРМАЦИОННИЕСИСТЕМИКОТОРИХИ

Вывод: выполнена программная реализация дешифрования текста полиалфавитной замены (Шифр Виженера) до получения текста с моноалфавитной заменой. Полная расшифровка текста не может быть окончательно освобождена от ошибок из-за отклонения фактических значений частотности букв русского языка в каждом конкретном тексте от глобальных средних значений.

Код реализации: \url{https://github.com/SugarHedgehog/work/tree/main/lab_3}
\end{document}