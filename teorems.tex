\documentclass[a4paper, 12pt]{extarticle}
\usepackage{fontspec}
\usepackage{polyglossia}
\setmainfont{CMU Serif}
\newfontfamily{\cyrillicfont}{CMU Serif}
\setsansfont{CMU Sans Serif}
\newfontfamily{\cyrillicfontsf}{CMU Sans Serif}
\setmonofont{CMU Typewriter Text}
\newfontfamily{\cyrillicfonttt}{CMU Typewriter Text}
\setdefaultlanguage{russian}
\usepackage[left=1cm,right=1cm,
top=2cm,bottom=2cm]{geometry}
%%% Дополнительная работа с математикой
\usepackage{amsfonts,amssymb,amsthm,mathtools} % AMS
\usepackage{amsmath}
\usepackage{icomma} % "Умная" запятая: $0,2$ --- число, $0, 2$ --- перечисление

%% Шрифты
\usepackage{euscript} % Шрифт Евклид
\usepackage{mathrsfs} % Красивый матшрифт

%% Свои команды
\DeclareMathOperator{\sgn}{\mathop{sgn}}


%% Перенос знаков в формулах (по Львовскому)
\newcommand*{\hm}[1]{#1\nobreak\discretionary{}
	{\hbox{$\mathsurround=0pt #1$}}{}}

%%% Работа с картинками
\usepackage{graphicx}  % Для вставки рисунков
\graphicspath{{Изображения/}{image}}  % папки с картинками
\setlength\fboxsep{3pt} % Отступ рамки \fbox{} от рисунка
\setlength\fboxrule{1pt} % Толщина линий рамки \fbox{}
\usepackage{wrapfig} % Обтекание рисунков и таблиц текстом

%%% Работа с таблицами
\usepackage{array,tabularx,tabulary,booktabs} % Дополнительная работа с таблицами
\usepackage{longtable}  % Длинные таблицы
\usepackage{multirow} % Слияние строк в таблице
\newcommand{\br}{\geqslant}
\newcommand{\mr }{\leqslant}
\newcommand{\mes}{\{X, \rho\}}
\newcommand{\MS}{Пусть $\{X, \rho\}$~--метрическое пространство}
\newcommand{\tttk}{тогда и только тогда, когда }
\newcommand{\Rnp}{\mathbb{R}^n_p}
\newcommand{\crk}{^{\circ}}
\linespread{1.15}
\usepackage{xhfill}
\begin{document}
\pagestyle{empty}
\section*{Без  доков}

Примеры метрических пространств

{\setlength{\extrarowheight}{5pt}
\begin{tabularx}{\textwidth}{||l|l|X||}

    \hline
    Название               & Метрика                                                                    & Какое множество или пространство                                                                                         \\
    \hline

    Дискретная             & $
        \rho (x, y) =
    \begin{cases}
            1 & x=y      \\
            0 & x \neq y \\
        \end{cases}$        & $X$ произвольное непустое множество

    \\
    \hline

    $\Rnp$                 & $\rho_p (x, y)=(\sum_{k = 1}^{n}|x_k-y_k|^p)^\frac{1}{p}$                  & \multirow{2}{6cm}{$\mathbb{R}^n$ множество n-мерных векторов $x=(x_1, x_2, \dots, x_n)$  }                               \\
    \cline{1-2}
    $\mathbb{R}^n_\infty $ & $\rho_\infty  (x, y)=\underset{1\leq k\leq n}{\max}|x_k-y_k|$              &                                                                                                                          \\
    \hline
    $C[a, b]$              & $\rho (x, y)=\underset{a \leq t \leq b}{\max}|x(t)-y(t)|$                  & \multirow{2}{6cm}{Пространство числовых функций, непрерывных на $[a, b]$}                                                \\
    \cline{1-2}
    $C_1[a, b]$            & $\rho (x, y)=\int_{b}^{a}|x(t)-y(t)|dx $                                   &                                                                                                                          \\[5pt]
    \hline
    $M[a, b]$              & $\rho (x, y)=\underset{a \leq t \leq b}{\sup}|x(t)-y(t)|$                  & Пространство числовых функций, определённых и ограниченных на $[a, b]$                                                   \\
    \hline
    $l_p$                  & $\rho_p (x, y)=(\sum_{k = 1}^{\infty }|x_k-y_k|^p)^\frac{1}{p}$            & Пространство числовых последовательностей $x=~(x_1, x_2, \dots, x_k, \dots)$, суммируемых с $p$-той степенью             \\
    \hline
    $m$                    & $\rho (x, y)=\underset{k \in \mathbb{N} }{\sup}|x_k-y_k|$                  & Пространство произвольных числовых последовательностей $x=~(x_1, x_2, \dots, x_k, \dots)$, таких что $\sup|x_k|<\infty $ \\
    \hline
    $s$                    & $\rho_p (x, y)=\sum_{k = 1}^{\infty } \cfrac{|x_k-y_k|}{2^k(1+|x_k-y_k|)}$ & Пространство произвольных числовых последовательностей $x=~(x_1, x_2, \dots, x_k, \dots)$                                \\
    \hline
\end{tabularx}}
\vspace{\baselineskip}

\textbf{Лемма 1}

Пусть на $[a, b]$ функция $\varphi (t) \br 0$, непрерывна на $\int_{b}^{a}\varphi
    (t)\, dx=0$, тогда $\varphi\equiv 0$ на $[a, b]$.

    \noindent\dotfill

\textbf{Теорема 1 (свойства замыкания) }

\MS. $\forall(M, M_1, M_2\subset X)[(M\subset \overline{M})
        \wedge \newline (\overline{\overline{M}}\subset \overline{M})
        \wedge (M_1 \subset M_2 \rightarrow \overline{M_1} \subset \overline{M_2})
        \wedge(\overline{M_1\cup M_2}=\overline M_1 \cup \overline M_2)]$

    \noindent\dotfill
    
\textbf{Теорема 2 (критерий точки прикосновения)}

\MS и $M \subset X$. Точка $x \in \overline M $ \tttk
$\exists( \{ x_n \} \subset M)[\underset{n \rightarrow \infty  }{x_n \rightarrow x}]$

\textbf{Примеры замкнутых множеств}

\begin{tabular}{c c c}
    1. $\forall B[x_0, r]\subset \mes $ &
    2. $\varnothing \subset \mes $      &
    3. $X \subset \mes$
\end{tabular}

\noindent\dotfill
    
\textbf{Теорема 3}

В метрическом пространстве объединение конечного числа и пересечение любого числа
замкнутых множеств ~--замкнутое множество.

\noindent\dotfill
    
\textbf{Свойства сходящихся последовательностей  }
\begin{enumerate}
    \item Пусть в метрическом пространстве $\{x_n\}$ сходится. $\forall(x_{n_k} \subset
              \{x_n \})$ сходится к тому же пределу, что и $\{x_n\}$
    \item Последовательность в метрическом пространстве может иметь только один
          предел
    \item Любая сходящаяся последов в метрическом пространстве ограничена.
    \item Пусть в метрическом пространстве даны две последовательности

          $\underset{n \rightarrow \infty  }{x_n \rightarrow x \text{ и } y_n \rightarrow y} $. Тогда
          $\underset{n \rightarrow \infty  }{\rho(x_n, y_n) \rightarrow \rho(x, y)}$
\end{enumerate}

\noindent\dotfill
    
\textbf{Сходимость в пространстве $\Rnp$   }

Пусть $1 \mr  p < \infty $, ${x^m}\subset \Rnp $ и $\underset{m \rightarrow \infty  }{x^m \rightarrow x}$

$\rho(x^m, x)= (\sum_{k = 1}^{n}|x^m_k-x_k|^p)^{1/p} \rightarrow 0$ (покоординатная сходимость)

\noindent\dotfill
    
\textbf{Сходимость в пространстве $C[a, b]$}

Пусть $\{ x_n \}\subset C[a, b]$ сходится по метрике к функции $x \in C[a, b]$.

$\underset{n \rightarrow \infty  }{\rho(x_n, x)}=\underset{a \mr  t \mr  b}{\max}|x_n(t)-x(t)| \rightarrow 0$

\hspace*{30mm}$\Updownarrow$

$\forall(\varepsilon>0)\exists(N\in \mathbb{N})\forall(n\br N)\forall(t\in[a, b])
    [|x_n(t)-x(t)|<\varepsilon]$(равномерная сходимость)

    \noindent\dotfill
    
\textbf{Теорема 4 (свойства внутренности)}

\MS.

$\forall(M, M_1, M_2\subset X)[(M\crk \subset M)
        \wedge((M\crk)\crk=M)
        \wedge(M_1 \subset M_2 \rightarrow M_1\crk \subset M_2\crk)
        \wedge \\((M_1 \cap M_2)\crk=M_1\crk \cap  M_2\crk)]$

        \noindent\dotfill
    
\textbf{Теорема 5 (связь внутренности и замыкания)}

\MS и $M \subset X$. Тогда:

$X\backslash M\crk=\overline{X\backslash M}$ $X
    \backslash \overline M = (X \backslash M)\crk$

    \noindent\dotfill
    
\textbf{Примеры открытых множеств  }

\begin{tabular}{c c c}
    1. $\forall B(x_0, r)\subset \mes $ &
    2. $\varnothing \subset \mes $      &
    3. $X \subset \mes$
\end{tabular}

\noindent\dotfill
    
\textbf{Теорема 6 (о дополнении)  }

В метрическом пространстве дополнение открытого множество замкнуто,
а дополнение замкнутого множества открыто

\noindent\dotfill
    
\textbf{Теорема 7  }

В метрическом пространстве объединение любого числа и пересечения конечного числа
открытых множеств~-- открытое множество.

\noindent\dotfill
    
\textbf{Теорема 10 (о полноте подпространства) }

\MS и $M \subset X$. Для того чтобы $ \{M, \rho\}$ было полным подпространством
$\mes$, необходимо и достаточно, а в случае полноты $\mes$ и достаточно, чтобы
множество $M$ было замкнуто в $\mes$.

\noindent\dotfill
    
\textbf{Теорема 11 (о вложенных шарах) }

\MS и в нём задана последовательность $ \{ B[a_n, r_n] \} $ замкнутых шаров таких,
что $\underset{n \rightarrow \infty }{r_n \rightarrow 0}$ и $\forall(n \in \mathbb{N})[B[a_{n+1},
                r_{n+1}]\subset B[a_n, r_n]]$

Тогда в $X$ существует единственная точка, принадлежащая всем этим шарам.

\noindent\dotfill
    
\textbf{Теорема 14 (об эквивалентности определений)   }

Непрерывность по Гейне в точке $x_0 \in M$:

$\forall( \{ x_n \}\subset M)[(\underset{n \rightarrow \infty }{\lim} x_n = x_0 )\rightarrow
        (\underset{n \rightarrow \infty }{\lim}f(x_n)=f(x_0))]$

\hspace*{30mm}$\Updownarrow$

Непрерывность по Коши в точке $x_0 \in M$:

$\forall(\varepsilon>0)\exists(\delta>0)\forall(x \in M)[\rho_X(x, x_0)<\delta\rightarrow
        \rho(f(x), f(x_0))<\varepsilon]$

        \noindent\dotfill
    
\textbf{Следствие из теоремы 14}

Для непрерывности функции $f:M \rightarrow Y$ в точке $x_0 \in M$ достаточно, чтобы для
$\forall(x_n \rightarrow x \in X)$ последовательность ${f(x_n)}$ сходилась в $Y$.

\noindent\dotfill
    
\textbf{Теорема 15 (о прообразах открытых множеств) }

Пусть $\{X, \rho_X \}$, $\{Y \rho_Y \}$ метрические пространства. Функция $f:
    X \rightarrow Y$ непрерывна на $X$ \tttk для любого открытого множества $A \subset Y$
прообраз $f^{-1}[A]$~--открытое множество в $X$.

\noindent\dotfill
    
\textbf{Теорема 16 (о прообразах замкнутых множеств)}

Пусть $\{X, \rho_X \}$, $\{Y \rho_Y \}$ метрические пространства. Функция $f:
    X \rightarrow Y$ непрерывна на $X$ \tttk для любого замкнутого множества $A \subset Y$
прообраз $f^{-1}[A]$~--замкнутое множество в $X$.

\noindent\dotfill
    
\textbf{Теорема 17 (о неподвижной точке) }

Пусть \mes~--полное метрическое пространство, множество $M= \overline M \subset X
    \text{ и } f:M \rightarrow M$~--сжимающее отображение. Тогда $f$ имеет в $M$
единственную неподвижную точку $x^*=f(x^*)$.Кроме того, для любого $x_0
    \in M $ последовательность $\underset{n \in \mathbb{N} }{x_n=f(x_{n-1})}$ сходится
к $x^*$ и справедлива оценка погрешности :

$\rho(x_n, x^*)\br \cfrac{q^n}{1-q}\rho(f(x_0), x_0)$

\noindent\dotfill
    
\textbf{Теорема 18   }

В метрическом пространстве любое относите компактное множество ограниченно.

\noindent\dotfill
    
\textbf{Теорема 19 (Вейерштрасса)}

Пусть $M$ компактное множество в метрическом пространстве $\mes$ и функция
$f:M \rightarrow \mathbb{R}^1$ непрерывна на $M$. Тогда множество $f[M]$ отграничено в
$\mathbb{R}^1$ и $\exists(a \in M)[f(a)=\underset{x \in M}{\\inf} f(x)]\wedge
    \exists(b\in M)[f(b)=
        \underset{x \in M}{\sup} f(x)] $

        \noindent\dotfill
    
\textbf{Теорема 20}

Вполне ограниченное множество в метрическом пространстве ограниченно.

\noindent\dotfill
    
\textbf{Следствие из теоремы Хаусдорфа}

\MS. Для того чтобы множество $M \subset X$ было относительно компактным
необходимо, а в случае полноты пространства $X$ и достаточно, чтобы множество $M$
было вполне ограниченным.

\noindent\dotfill
    
\textbf{Теорема 22 (относительная компактность в $ \Rnp $) }

Множество $M \subset \Rnp$ относительно компактно \tttk это множество в $\Rnp$
ограниченно.

%%%%%%%%%%%%%%%%%%%%%%%%%%%%%%%%%%%%%%%%%%%%%%%%%%

\section*{С  доками}
    
\textbf{Примеры полных метрических пространств  }

\begin{enumerate}
    \item Дискретное пространство полно. Всякая фундаментальная последовательность
          в нём, начиная с некоторого номера, стабилизируется.
    \item Пространство $\Rnp$, где $1\br p \br \infty $. Пусть $\{x^m\}\subset\Rnp$~---
          фундаментальная последовательность. Здесь $x^m=(x^m_1, x^m_2, \dots, x^m_n)$.
          Тогда

          $\forall(\varepsilon>0)\exists(N)\forall(m\mr 0)[\rho(x^{m+q},x^m)=
              (\sum_{k = 1}^{n}|x^{m+q}_k-x^m_k|^p)^(1/p)<\varepsilon]$.

          Откуда получим

          $\forall(\varepsilon>0)\exists(N)\forall(m\mr  N)\forall(q\br 0)\forall(k=
              \overline{1,n})[|x^{m+q}_k-x^m_k|<\varepsilon]$

          Следовательно при $k=const$ $\{x^m_k\}^\infty _{m=1}$ фундаментальная
          в $\mathbb{R}^1\Longrightarrow$  сходится.

          $\forall(k=\overline{1,n})\exists(x_k\in\mathbb{R}^1)[x_k=\underset{m \to
                      \infty }{\lim} x^m_k]$.

          Тогда $\{x^m \} \longrightarrow x=(x_1,x_2,\dots,x_n)\in\mathbb{R}^1$
          (сходится покоординатно к $x$) $\Longleftrightarrow $ сходимости по метрике

          $\underset{m\rightarrow\infty }{\rho(x^m,x)\rightarrow\sim0}$. $\mathbb{R}^n_p$ полное.

    \item Пространство $C[a, b]$Пусть $\{x_n\} \subset C[a, b]$ — фундаментальная
          последовательность, то есть

          $\forall (\varepsilon > 0)\exists (N)\forall (n \mr  N )\forall (p \mr  0)[\rho(x_{n+p}, x_n) =  \underset{a\mr  t \mr  b}{\max}|x_{n+p}(t) x_n(t)|
              <\varepsilon]$.

          \hspace*{30mm}$\Updownarrow$

          $\forall(\varepsilon > 0)\exists(N )\forall( n \mr  N )\forall( p \mr  0)\forall( t \in
              [a, b])[|x_{n+p}(t) x_n(t)|<\varepsilon]$.

          \Rightarrow при $t=const \in [a, b]$ ${x_n(t)}$ фундаментальна в $\mathbb{R}^1$.
          Тогда определена $x(t)$, которая является
          поточечным пределом $\{x_n(t)\}$. Получим при $p \rightarrow \infty $

          $\forall( \varepsilon > 0)\exists(N )\forall( n \mr  N )\forall( t \in  [a, b])[
                  |x(t) x_n(t)| \mr  \varepsilon]$.
          $\implies  x(t)$ является равномерным пределом $\{x_n(t)\}$.

          Покажем, что функция $x(t)$ непрерывна. Пусть $t_0 \in  [a, b]$.
          При $\varepsilon > 0$ выберем $m \in  N$ такое, что

          $\forall( t \in  [a, b]) [ |x_m(t) x(t)| < \varepsilon/3]$.

          Выберем также $\delta = \delta(m) > 0$ такое, что

          $\forall( t \in  [a, b]) [ (|t t_0| < \delta) \rightarrow (|x_m(t_0) x_m(t)| < \varepsilon/3)]$.
          Тогда для $t$ таких, что $|t t_0| < \delta$ выполнена оценка

          $|x(t) -x(t_0)| \mr  |x(t)- x_m(t)| + |x_m(t) -x_m(t_0)| + |x_m(t_0)- x(t_0)| <
              \varepsilon.$

          Итак, $x \in  C[a, b]$. Осталось вспомнить, что в $C[a, b]$ сходимость по метрике
          равносильна равномерной сходимости последовательности функций, то есть
          $(x_n, x) \rightarrow 0 \text{ при } n \rightarrow \infty $.
          %%%%%%%%%%%%%%
    \item Пространство $l_p$ $(1 \mr  p < \infty )$.
          Пусть $\{x_n \}= (x_n, x_n, \dots , x_n, \dots)$
          — фундаментальная последовательность в $l_p.$

          Тогда $\forall( \varepsilon > 0)\exists(N )\forall( n \br  N )
              \forall( q \br  0)[\rho(x^{n+q}, x_n) =  (\sum^\infty_{k=1} |x^{n+q}
              x^n_k|^p )^{1/p}<\varepsilon]$.


          Отсюда следует, что

          $\forall( \varepsilon > 0)\exists(N )\forall( n \br  N )\forall( q \br  0)
              \forall( k)[|x^{n+q}_k - x^n_k| < \varepsilon]$.

          $\forall(k=const) \in  N$ $\{x_n \}$ сходится.

          Пусть $x_n \rightarrow x_k$ при $n \rightarrow
              \infty $. Обозначим $x = (x_1, x_2, \dots , x_k, \dots)$.

          Нужно показать:
          а) $x \in  l_p$, то есть	$\sum^\infty_{k=1}|x_k|^p<\infty$;
          б) $\rho(x , x) \rightarrow 0$.

          а) Зафиксируем $m \in  N$. Тогда

          $\forall( n \br  N )\forall( q \br  0)[(\sum^m_{k=1}|x^{n+q}_k-x^n_k|^p)
              ^{1/p}]<\varepsilon$

          Перейдем к пределу при $q \rightarrow \infty $. Получим

          $\forall(n\br N)[(\sum^m_{k=1}|x_k-x^n_k|^p)^{1/p}\mr\varepsilon]$

          Далее, фиксируя $n \br  N$ , установим оценку

          из которой следует, что ряд $\sum^{\infty }_{k=1} |x_k|^p$  сходится
          и $x = (x_1, x_2, \dots , x_k, \dots) \in  l_p$.

          б) Из оценки, справедливой для всех $m \in  N$ и $n \br  N$ , при $m \rightarrow \infty $
          получим

          $(\sum^\infty_{k=1}|x_k— x_n|^p)^{1/p} \mr  \varepsilon.$

          Итак, $\rho(x_n, x) \rightarrow 0$ при $n \rightarrow \infty $ и полнота
          пространства $l_p $доказана.

\end{enumerate}

\noindent\dotfill
    
\textbf{Примеры метрических пространств, не являющихся полными}
Пространство $C_1[a, b]$  не является полным. Для простоты считаем, что $[a, b] =
    [-1, 1]$
. В пространстве $C_1[-1, 1]$ рассмотрим последовательность функций

$x_n(t=)\begin{cases}
        0,  & -1\mr t \mr 0           \\
        nt, & 0 \mr t \mr \frac{1}{n} \\
        1,  & \frac{1}{n}\mr t \mr 1.
    \end{cases}$

Фундаментальность последовательности $\{x_n\}$ в $C_1[-1, 1]$ следует из того,
что для всех $m \br n$ при $n \rightarrow \infty $

Последовательность функций $\{x_n\}$ на $[-1, 1]$ поточечно сходится к разрывной
функции
$y(t) =	\begin{cases}0, & -1 \mr  t \mr  0 \\
             1, & 0 < t \mr  1 .
    \end{cases}$

Предположим, что  $\exists z \in C_1[-1, 1]$ и $\rho(x_n, z) \rightarrow 0$ при
$n \rightarrow \infty $. Тогда

$$\int_{-1}^{1} |y(t)- z(t)| \,dt \mr\int_{-1}^{1} |y(t)- x_n(t)|\,dt +
    \int_{-1}^{1}|x_n(t)- z(t)| \,dt=\frac{1}{2n}+\rho(x_n,z)\rightarrow 0$$ при $n \rightarrow \infty$

Получили, что	$\int_{-1}^{1}|y(t) z(t)| \,dt = 0.$ Из равенства

$$0 =\int_{-1}^{1}|y(t)- z(t)| \,dt +
    \int_{0}^\delta|y(t)- z(t)| \,dt +\int_\delta^1|y(t) -z(t)| \,dt$$
справедливого для произвольного сколь угодно малого $\delta > 0$, следует
$$\int_{-1}^0 |y(t) z(t)| \,dt =\int^{1}_\delta |y(t) -z(t)| \,dt = 0.$$

$y(t) -z(t) на [-1, 0] \cup [\delta, 1]$, получим $z(0) = y(0) = 0$ и $z(\delta)
    = y(\delta) = 1.$(По лемме 1 и непрерывности функции)
Тогда $z(+0) = \underset{\delta \to \infty }{\lim}z(\delta)= 1$. Итак, $z(0)\neq z(+0)$, то
есть функция $z(t)$ разрывна в точке $t = 0$. Это противоречие доказывает, что
фундаментальная последовательность $x_n$ не сходится в пространстве $C_1[-1, 1]$.

\noindent\dotfill
    
\textbf{Теорема 8 (об открытом множестве на прямой) }

Всякое непустое ограниченное открытое множество на прямой есть объединение
конечного или счетного числа попарно непересекающихся интервалов.

Пусть $A \subset \mathbb{R}^1$ – ограниченное и открытое множество.
Возьмем $x \in  A$ и рассмотрим множество $F = [x, +\infty ) \cap CA$.
Множество $F$ замкнуто как пересечение двух замкнутых множеств [x, +\infty ) и CA.
Множество $F$ непустое, так как $A$ ограничено. Кроме того, множество $F$
ограничено снизу точкой $x$. Пусть $b = \inf F$ . Очевидно, $x \mr  b$.
Заметим, что $x \in  A$, а $b \in  F = F \subset CA \Rightarrow x < b$.

Рассмотрим полуинтервал $[x, b)$ и покажем, что [$x, b) \subset
    A$. Предположим, что [$x, b) \not\subset A$.
Тогда $\exists(y  \in  [x, b)) [ y  \in /  A]$. Получили $y  \in  F$
такой, что $y  < b$,
но это невозможно, так как $b = \inf F$ . Таким образом, $[x, b) \subset A$. Заметим также,
                что $b \notin  A$, так как $b \in  F \Rightarrow b \in  CA$.

                Если для того же$ x \in  A$ рассмотреть множество $\Phi = (-\infty , x] \cap CA$, то,
рассуждая подобным образом, получим для $a = sup \Phi$,
что $(a, x] \subset A$ и $a \notin  A$.

Таким образом, если $x \in  A$, то существует интервал $(a, b) \ni x$ такой,
что  $(a, b)  \subset A$  и $ a, b  \notin  A$.  Такие  интервалы,  из  которых  состоит
открытое
множество A, называются составляющими интервалами множества $A$.

Покажем, что составляющие интервалы попарно не пересекаются. Пусть даны два
составляющих интервала $(a, b)$ и $(a', b')$, которые пересекаются, например,
с условием $a < a' < b < b'$. Тогда $a' \in  (a, b) \Rightarrow a' \in  A$,
так  как  $(a, b)  \subset A$.  Но  $a' \notin   A$,  как  конец  составляющего  интервала
$(a', b')$.

Полученное противоречие означает, что $(a, b) \cap (a', b') = \varnothing$.
Осталось заметить, что на прямой любое множество попарно непересекающихся интервалов
не более, чем счетно.

\noindent\dotfill
    
\textbf{Теорема 9 (о замкнутом множестве на прямой)}

Всякое непустое ограниченное замкнутое множество на прямой
получается из наименьшего отрезка, его содержащего, путем выбрасывания конечного или
счетного числа попарно непересекающихся интервалов.

Пусть непустое множество $F \subset \mathbb{R}^1$ ограничено и замкнуто. Пусть
$[\alpha, \beta ]$ – наименьший отрезок, содержащий это множество.
Заметим, что $\alpha, \beta
    \in  F$ .  Представим  $F$  в  виде  $F  =  [\alpha, \beta ] \backslash
    ( [\alpha, \beta ] \cap CF )  = [\alpha, \beta ] \backslash ( (\alpha, \beta )
    \cap CF )$.
Множество $(\alpha, \beta ) \cap CF$ открыто, поэтому (теорема 8)
$ (\alpha, \beta )	CF  =\underset{n}{\bigcup}(a_n , b_n )$,
где интервалы$ (a_n, b_n )$ попарно не пересекаются
\Rightarrow $F  = [\alpha , \beta ] \backslash \underset{n}{\bigcup}(a_n , b_n )$.

\noindent\dotfill
    
\textbf{Теорема 13 (Бэра)}
Всякое полное метрическое пространство является множеством второй категории.

\MS. Предположим, что множество X не является множеством второй
категории. Тогда $X  = \stackrel{\infty}{\underset{n=1}{\bigcup}}  M_n$($M_n$ – нигде не плотные).
Возьмем произвольный замкнутый шар $B[a, 1] \subset X$. Так как $B(a, 1) \subset B[a, 1]$,
а множество $M_1$ нигде не плотно, то
$\exists(B(x_1, r) \subset B(a, 1)) [ B(x_1, r) \cap M_1 =\varnothing]$.
Возьмем шар $B[x_1, r_1] \subset B(x_1, r)$, для которого также
$B[x_1, \mathbb{R}^1] \cap M_1 = \varnothing$. Можно считать, что $r_1 < 1/2$.

Точно так же можно построить $B[x_2, r_2]  \subset B[x_1, r_1]$  такой,
что$ B[x_2, r_2]$$\cap M2 = \varnothing$ и $r_2 < 1/(2^n)$.
    Поступая подобным образом, получим последовательность замкнутых шаров
${B[x_n, r_n]}$ таких, что $r_n < 1/(2^n)\rightarrow 0$
    при $n\rightarrow \infty$  и $\forall(n \in  N) [ B[x_{n+1}, r_{n+1}] \subset B[x_n, r_n]]$.

    Воспользуемся теоремой 11 о вложенных шарах. Тогда в $X$ существует
    элемент  $\overline{x}  \in  \stackrel{\infty}{\underset{n=1}{\bigcap}} B[x_n, r_n]$.
    Так  как    $\overline{x}\in  B[x_1, \mathbb{R}^1]$,  то
$\overline{x} \notin M_1$.  Аналогично
$x \notin   M_n$  для любого $n$. Это означает, что $\overline{x} \notin X$. Получили противоречие.

\noindent\dotfill
    
    \textbf{Примеры сепарабельных пространств}
    \begin{enumerate}
        \item Пространство $\Rnp$ для  $1\mr  p < \infty$ . Возьмем множество $Q^n \subset \Rnp$.
              Как известно, $Q^n \sim N$. Покажем, что $Q^n = Rn$.
              Пусть элемент $x = (x_1, x_2, \dots, x_n) \in \Rnp$ и задано $\varepsilon > 0$.
              По свойству плотности в $\mathbb{R}^1$ множества рациональных чисел
              $Q$ найдется $\alpha = (\alpha_1, \alpha_2, \dots, \alpha_n) \in  Q^n$  такой,
              что для всех $k = \overline{1, n}$ выполняется $|x_k \alpha_k| < \varepsilon /(n^{1/p})$.
              В таком случае,
              $\rho(x, \alpha) =(\sum^n_{k=1}|x_k \alpha_k|^p)^{1/p}< \varepsilon $.

              \Rightarrow$x$ – точка прикосновения множества $Q^n$. Учитывая произвольность $x \in  \Rnp$,
              доказали, что множество $Q^n$ всюду плотно в $\Rnp$. ♥
        \item Пространство  $C[a, b]$.

              Рассмотрим множество $M$ всевозможных многочленов на $[a, b]$ с
              рациональными коэффициентами.
              Заметим, что это множество  $M = \stackrel{\infty}{\underset{n=0}{\bigcup}}M_n$
              ($M_n $ –  множество  многочленов с  рациональными коэффициентами, степень
              которых не превышает $n$).
              Так как каждый многочлен однозначно определяется своими коэффициентами,
              то $M_n \sim Q^{n+1}$. Но $Q^{n+1} \sim N$, поэтому и
              $M_n \sim N \Rightarrow M \sim N$, то есть является счетным множеством.

              Покажем, что $\overline{M} = C[a, b]$.
              Возьмем элемент $x \in  C[a, b]$ и $\varepsilon  > 0$. В силу теоремы
              Стоуна-Вейерштрасса найдется многочлен
              $$P (t) = \alpha_0 + \alpha_1 t + \alpha_2 t^2 + \dots + \alpha_n t^n$$
              с вещественными коэффициентами $\alpha_i \in  R  (i  =  \overline{1, n})$ такой, что для всех
              $t \in  [a, b]$ выполняется $|x(t) -P(t)| < \varepsilon /2$. Построим многочлен
              $ Q \in  M$
              $$ Q(t) = \beta_0 + \beta_1 t + \beta_2 t^2 + \dots + \beta_n t^n$$
              с коэффициентами $\beta_i  \in  Q  (i  =  \overline{1, n})$  и такой, что
              $|\alpha_i-\beta_i| < \varepsilon /[2(n + 1)K^i]$,
              где $K = \max{|a|, |b|}$. В таком случае для всех $t \in  [a, b]$ получим
              $$|P (t)- Q(t)| \mr |\alpha_0- \beta_0| + |\alpha_1- \beta_1|K + \dots
                  + |\alpha_n- \beta_n|K^n < \varepsilon /2.$$
              $$\Downarrow $$
              $$\rho(x, Q) \mr  \underset{a\mr t\mr b}{\max }|x(t)- P (t)|
                  + \underset{a\mr t\mr b}{\max} |P (t)- Q(t)| < \varepsilon .$$

              Установили, что $x \in  \overline{M}$ , то есть $\overline{M} = C[a, b]$.
              Таким образом, M счетное всюду плотное в $C[a, b]$ множество,
              а пространство $C[a, b]$ сепарабельно.
        \item	Пространство $s$.

              Рассмотрим множество $M$ последовательностей с рациональными
              координатами, у которых число ненулевых координат конечно.
              Заметим,  что
              $M  =  \stackrel{\infty}{\underset{n=1}{\bigcup}} M_n$(  $M_n$  –
              множество  последовательностей с рациональными
              координатами, у которых ненулевыми могут быть только
              первые $n$ координат). Очевидно, что $M_n \sim Q^n \sim N$.
              Поэтому и $M \sim N$, то есть
              множество M является счетным.

              Покажем, что $\overline{M}  = s$.
              Пусть $x = (x_1, x_2, \dots, x_k, \dots) \in s$ и $\varepsilon  > 0$.
              Выберем
              $m \in  N$ такое, что	$\stackrel{\infty}{\underset{k=m+1}{\sum}}1/2^k < \varepsilon /2$.
              Найдем $\alpha_i \in  Q (i = \overline{1, m})$ такие, что
              $|x_k -\alpha_k| < \varepsilon /m$. Тогда

              $$\stackrel{m}{\underset{k=1}{\sum}}\frac{|x_k-\alpha_k|}{2^k(1+|x_k-\alpha_k|)}<
                  \frac{1}{2}\stackrel{m}{\underset{k=1}{\sum}}|x_k-\alpha_k|<\frac{\varepsilon}{2}$$
              В результате для $\alpha_= (\alpha_1, \alpha_2, \dots, \alpha_m, 0, 0, \dots) \in  M$ получим

              $$\rho(x,\alpha)=\stackrel{m}{\underset{k=1}{\sum}}
                  \frac{|x_k-\alpha_k|}{2^k(1+|x_k-\alpha_k|)} +
                  \stackrel{\infty}{\underset{k=m+1}{\sum}}
                  \frac{|x_k|}{1+|x_k|}<\frac{\varepsilon}{2} +
                  \stackrel{\infty}{\underset{k=m+1}{\sum}}\frac{1}{2^k}<\varepsilon$$

              Установили, что $x \in  \overline{M}$ , то есть $\overline{M} = s$.
              Таким образом, $M$ счетное всюду
              плотное в $s$ множество, но тогда пространство $s$ сепарабельно.
        \item	Пространство $l_p$ для $1 \mr  p < \infty$ .

              Вновь рассмотрим счетное множество
              $M$ из предыдущего примера. Очевидно, что $M \subset l_p$.
              Покажем, что $\overline{M} = l_p$.
              Возьмем $x = (x_1, x_2, \dots, x_k, \dots) \in  l_p$ и $\varepsilon  > 0$.
              Так как $\stackrel{\infty}{\underset{k=1}{\sum}}|x_k|^p < \infty$ ,
              то найдется $m \in  N$ такое,
              что	$\stackrel{\infty}{\underset{k=m+1}{\sum}}|x_k|^p  < \varepsilon^p   /2$.
              Найдем $\alpha_i \in  Q (i = \overline{1, m})$ такие, что $|x_k-\alpha_k| < \varepsilon /(2m)^{1/p}$.
              Тогда для элемента $\alpha_= (\alpha_1, \alpha_2, \dots, \alpha_m, 0, 0, \dots) \in  M $получим

              $$\rho^p(x, \alpha) = \stackrel{m}{\underset{k=1}{\sum}}|x_k-\alpha_k|^p+
                  \stackrel{\infty}{\underset{k=m+1}{\sum}}|x_k|^p<
                  \varepsilon^p$$

              $\Rightarrow\rho(x, \alpha) < \varepsilon $. Таким образом, $x \in  M$ ,
              то есть $\overline{M} = l_p$.
              Показали, что M счетное всюду плотное в $l_p$ множество, но тогда пространство $l_p$ сепарабельно.
    \end{enumerate}

    \noindent\dotfill
    
    \textbf{Примеры пространств, не являющихся сепарабельными}

    \begin{enumerate}
        \item     Пространство $M [a, b]$. Пусть параметр $\tau \in  [a, b]$. Определим на функцию
              $$x_\tau (t) =
                  \begin{cases}
                      1, & t = \tau, \\
                      0, & t= \tau.
                  \end{cases}	$$

              Очевидно, функция $x_\tau \in  M [a, b]$. Рассмотрим в $M [a, b]$
              множество функций
              $K = \{x_\tau | \tau \in  [a, b] \}$.
              Заметим, что $K \sim [a, b]$, то есть является
              множеством мощности континуум, в частности K несчетное.
              Пусть теперь $\tau_1 \neq \tau_2,$ тогда


              $$\rho(x_{\tau_1} , x_{\tau_2} ) = \underset{a\mr  t \mr  b}{\sup}
                  |x_{\tau_1} (t)- x_{\tau_2} (t)| = 1.$$

              Предположим, что существует некоторое множество $M \subset M [a, b]$ такое,
              что
              $\overline{M}   =  M [a, b]$.
              Тогда
              $\forall(x_\tau   \in   K)\exists(y_\tau   \in   M )[ \rho(x_\tau , y_\tau )
                      <  1/3 ]$.  При  этом


              разным функциям $x_{\tau_1} , x_{\tau_2}   \in  K  (\tau_1\neq\tau_2)$
              соответствуют разные функции
              $y_{\tau_1} , y_{\tau_2} \in  M$ .
              Иначе, если $y_{\tau_1} = y_{\tau_2} = y \in  M $, получим противоречие
              $$1 = \rho(x_{\tau_1} , x_{\tau_2} ) \mr  \rho(x_{\tau_1} , y) +
                  \rho(y, x_{\tau_2} ) < 1/3 + 1/3 = 2/3.$$
              Таким образом, элементов в множестве $M$ не меньше, чем в множестве $K$,
              которое несчетно.
              $\Rightarrow$ множество $M$ несчетно, а пространство M $[a, b]$
              несепарабельно.
        \item	Пространство $m$. Рассмотрим в $m$ множество $K$ всевозможных
              последовательностей из 0 и 1. Из теории множеств известно, что $K$
              является множеством
              мощности континуум. Пусть $x_1, x_2 \in  K$ – две разные
              последовательности. Очевидно,
              что $\rho(x_1, x_2) = 1$.

              Дальнейшие рассуждения аналогичны тем, что проделаны выше в
              пространстве $M [a, b]$.
              Действительно, предположим, что существует некоторое множество
              последовательностей
              $M \subset m$ такое, что $\overline{M} = m$. В таком случае,
              $\forall(x  \in  K)\exists(y  \in  M )[ \rho(x, y)  <  1/3 ]$.
              При  этом  разным
              последовательностям
              $x_1, x_2 \in  K$ соответствуют разные $y1, y2 \in  M$ .
              В противном случае, если
              соответствующие $y1 = y2$, получим $1 = \rho(x_1, x_2) < 2/3$.
              Как и выше отсюда следует,
              что множество $M$ несчетно и пространство $m$ несепарабельно.
    \end{enumerate}


    \noindent\dotfill
    
    \textbf{Теорема 21 (Хаусдорфа) }

    \MS. Для того,
    чтобы множество $M \subset X$ было относительно компактным необходимо,
    а в случае
    полноты пространства $X$ и достаточно, чтобы множество M
    было вполне ограниченным.

    Необходимость. Пусть $M $относительно компактно и не пусто.
    Возьмем
    произвольное $\varepsilon  > 0$ и $x_1 \in  M$ . Может оказаться, что
$M  \subset B(x_1, \varepsilon )$,
    тогда
$\varepsilon$ -сетью для $M$ будет множество $\{x_1\}.$
    Если же $M  \not\subset B(x_1, \varepsilon )$, то
$\exists(x_2 \in  M )[ \rho(x_2, x_1) \br \varepsilon  ].$
    Может оказаться, что $M$
$\subset B(x_1, \varepsilon ) \cup B(x_2, \varepsilon )$, тогда $\varepsilon$ -сетью для
$M$ будет множество
$\{x_1, x_2\}$. Если же это не выполнено, то
$\exists(x_3 \in  M )[ (\rho(x_3, x_1) \br \varepsilon ) \wedge  (\rho(x_3, x_2) \br \varepsilon ) ]$
    , и т.д.

    Продолжая этот процесс, возможно, что на каком-то $n$-ом шаге построим

$x_1, x_2, \dots, x_n \in  M$  такие, что множество $\{x_k\}^n_{k=1}$	будет конечной
$\varepsilon$ -сетью
    для $M$ , что завершит доказательство необходимости.
    Иначе это построение элементов
$x_n$ будет продолжаться бесконечно.
    В этом случае построим последовательность $\{x_n\} \subset M$  такую,
    что $\rho(x_i, x_j) \br \varepsilon  (i\neq	j)$. Но тогда из
$\{x_n\}$
    нельзя выделить сходящуюся подпоследовательность, что противоречит
    относительной
    компактности множества $M$ .

    Достаточность. Пусть $X$ – полное пространство и множество $M$  $\subset X$
    вполне ограничено. Возьмем последовательность $\varepsilon_n \searrow  0$ при
$n \rightarrow \infty$ .
    Для каждого
$ \varepsilon_n$ построим для $M$  конечную $\varepsilon_n$-сеть $S_n = {z^n_1, z^n_2, \dots, z^n_{m_n} }$. Пусть
    дана последовательность $\{x_n\} \subset M$ . Покажем, что из этой последовательно-
    сти можно выделить сходящуюся подпоследовательность.

    Заметим, что $\{x_n\} \subset M \subset \stackrel{m_1}{\underset{i=1}{\bigcup}} B(z^1_i, \varepsilon_1)$.
    Так как шаров конечное число,

    то существует шар $B(z^1_i, \varepsilon_1)$, который содержит бесконечно
    много членов последовательности $\{x_n\}$, то есть
$\exists(\{x^1_n \} \subset \{x^1_n\})[ \{x^1_n \} \subset B(z^1_i, \varepsilon_1) ]$.
    Повторяя
    рассуждения для $S_2$, получим существование шара $B(z^2_i, \varepsilon_2)$, содержащего
    подпоследовательность $\{x^2_n \} \subset {x^1_n }$, то есть ${x^2_n } \subset B(z^2_1, \varepsilon_2)$.
Продолжая    этот процесс получим,
$$\forall( m \in  N)\exists (B(z^m_i, \varepsilon_m))\exists (\{x^m\} \subset \{x^{m-1}\})
    [\{x^m_n \} \subset B(z^m_i, \varepsilon_m)]$$  ,
где последовательность ${x^0_n } = \{x_n\}$. Составим диагональную последовательность
${x_k^k} \subset \{x_n\}$. Покажем, что диагональная последовательность фундаментальна.
Действительно, $x_k^k, x_{k+p}^{k+p} \in  B(z^k_i, \varepsilon_k)$, где $p \br 0$. Поэтому
$$\rho(x_{k+p}^{k+p}, x_k^k) \mr  \rho(x_{k+p}^{k+p}, z^k_i) + \rho(z^k_i, x_k^k) < 2\varepsilon_k \rightarrow 0
    (k \rightarrow \infty ).$$
Осталось заметить, что фундаментальная последовательность ${x^k_k} \subset \{x_n\}$ в
силу полноты пространства $X$ сходится.

\noindent\dotfill
    
\textbf{Теорема 23 (Арцела) }

Множество $M \subset C[a, b]$ относительно компактно
тогда и только тогда, когда оно ограничено и равностепенно непрерывно.

Пусть множество $M$ относительно компактно. Тогда, по теореме 18,
оно ограничено. Покажем равностепенную непрерывность $M$ .
Возьмем произвольное $\varepsilon   >  0$. Выберем в $C[a, b]$  множество $\{\varphi\}^k_{i=1}$	,
являющееся  конечной  $\varepsilon /3$-сетью  для  $M$ .  Каждая  функция  $\varphi_i(t)$,
где $ i  =  \overline{1,k}$,
непрерывна на $[a, b] \Rightarrow$ равномерно непрерывна. Тогда
$$\exists(\delta_i > 0)\forall(t_1, t_2 \in  [a, b]) [(|t_1-t_2| < \delta_i) \rightarrow
        (|\varphi_i (t_1)- \varphi_i(t_2)| < \varepsilon /3)].$$
Возьмем $\delta = \underset{1 \mr  i \mr  k}{\min}  \delta_i > 0$ и функцию 
$x \in  M$ , для которой найдем
функцию $\varphi_i$ из $\varepsilon /3$-сети такую, что $\rho(x, \varphi_i) < \varepsilon /3$. Тогда при
условии $|t_1-t_2| < \delta$ получим
$$|x(t_1)- x(t_2)| \mr  |x(t_1) -\varphi_i(t_1)| + |\varphi_i(t_1) -\varphi_i(t_2)|+
    |\varphi_i(t_2)- x(t_2)| < 2\rho(x, \varphi_i) + \varepsilon /3 < \varepsilon .$$
Таким образом, множество $M$ равностепенно непрерывно.

Пусть теперь множество $M$ ограничено и равностепенно непрерывно. Покажем, что $M$
относительно компактно. Пусть дана последовательность функций $x_n \in  M $. Выделим
из $\{x_n\}$ сходящуюся подпоследовательность.


Рассмотрим множество $[a, b] \cap \mathbb{Q}  =  {r_1, r_2, \dots, r_k, \dots}$ рациональных  чисел.
Числовая последовательность ${x_n(r_1)}$ ограничена. Тогда существует сходящаяся
подпоследовательность ${x^1_n (r_1)} \subset {x_n(r_1)}$. 
Далее рассмотрим последовательность
$\{x^1_n (r_2)\}$, которая также ограничена. 
Тогда существует сходящаяся  подпоследовательность
${x^2_n (r_2)} \subset {x^1_n (r_2)}$.  Повторяя  эти  рассуждения для точек
$r3, r4, \dots, r_k, \dots$, получим вложенные последовательности
$\{x_n\} \supset  {x^1_n } \supset {x^2_n } \supset \dots \supset {x^k_n} \supset \dots$
такие, что для всех$ k \in  \mathbb{N} $ сходятся последовательности $\{x^k_n(r_k)\}^\infty_{n=1}$    . Выберем диагональную последовательность $\{x_n\}$.
Очевидно, что $\{x^n_n\} \subset \{x_n\}$ и последовательность ${x^n_n(t)}$
сходится в любой рациональной точке $t = r_k \in  [a, b]$. Объясняется это тем, что
$\{x^n_n(r_k)\}^\infty_{n=k}	\subset \{x^k_n(r_k)\}^\infty_{n=1}.$  

Возьмем произвольное $\varepsilon  > 0$ и из условия равностепенной непрерывности множества
функций $M$ выберем по $\varepsilon /3$ соответствующее $\delta > 0$. Построим разбиение
$a = t_0 < t_1 < t_2 < \dots < t_m = b$ отрезка $[a, b]$ на промежутки длины меньше $\delta$.
В каждом отрезке $[t_{i-1}, t_i]$ возьмем по одной рациональной точке

$r_i  \in  (t_{i-1}, t_i)$,  где  $i  =  \overline{1, m}$.  
Последовательность  $\{x^n_n(t)\}$ сходится  в  точках

$r_1, r_2, \dots, r_m$, тогда
$$\exists(N  \in  \mathbb{N})\forall(n \br N )\forall(p \br 0)\forall(i = \overline{1,m}) [|x^{n+p}_{n+p}(r_i)
        x^n_n(r_i)| < \varepsilon /3 ].$$

Возьмем точку $t \in  [a, b]$. Пусть эта точка $t \in  [t_{i-1}, t_i]$. В таком случае при
$n \br$ N и $p \br 0$ получим из и свойства равностепенной непрерывности
$|x^{n+p}_{n+p}(t)- x^n_n(t)| \mr  |x^{n+p}_{n+p}(t) x^{n+p}_{n+p}(r_i)|+|x^{n+p}_{n+p}(ri) -x_n^n(r_i)|+|x^n_n(r_i) x^n_n(t)| < \varepsilon.$ 
Таким образом, показали
$$\forall(\varepsilon  > 0)\exists(N  \in  \mathbb{N})\forall(n \br N )\forall(p \br 0)\forall(t \in  [a, b])
    [ |x^{n+p}_{n+p}(t) x^n_n(t)| < \varepsilon  ]$$,
что означает фундаментальность последовательности $\{x_n\}$ в полном пространстве $C[a, b]$.
\Rightarrow последовательность $\{x_n\}$ сходится, а множество $M$ относительно компактно.

\end{document}