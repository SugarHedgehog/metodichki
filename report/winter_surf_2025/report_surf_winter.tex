\documentclass[a4paper, 12pt]{extarticle}
\usepackage{fontspec}
\usepackage{polyglossia}
\setmainfont{CMU Serif}
\newfontfamily{\cyrillicfont}{CMU Serif}
\setsansfont{CMU Sans Serif}
\newfontfamily{\cyrillicfontsf}{CMU Sans Serif}
\setmonofont{CMU Typewriter Text}
\newfontfamily{\cyrillicfonttt}{CMU Typewriter Text}
\setdefaultlanguage{russian}

\usepackage{graphicx}
\graphicspath{{examples/image/}}

\usepackage{multicol}
\setlength{\columnsep}{2cm}

\usepackage[left=1cm,right=1cm,top=2cm,bottom=2cm]{geometry}
\usepackage{enumitem}

\begin{document}
Срок прохождения практики 11.02.2025г. по 9.04.2025г.
    
\subsection*{Введение}

В период с 11 февраля по 9 апреля студентка 5-го курса математического факультета Суматохина Александра Сергеевна проходила производственную практику в компании Surf (ООО “СёрфСтудио”) в качестве Flutter разработчика.

Surf – одна из ведущих студий мобильной разработки и дизайна в России, обладающая богатым опытом работы с крупными сервисами. Студия специализируется на электронной коммерции и мобильном банкинге, а процессы создания продуктов тесно связаны с пониманием задач бренда. Ключевые направления работы включают создание качественных приложений с привлекательным дизайном, а также их всестороннюю поддержку. Это предполагает активное взаимодействие с клиентами на стадии разработки и полноценное сопровождение продукта после его выпуска.

\section{Изучение основ языка Dart и фреймворка Flutter и написание приложения}
\subsection{Изучение языка программирования Dart и фреймворка Flutter}
Во время практики были изучены основы языка программирования Dart и фреймворк Flutter. Flutter — это фреймворк от Google для создания кроссплатформенных мобильных приложений.

\subsection{Разработка приложения «Галерея»}
\textbf{Обязанности:}
\begin{itemize}
    \item Изучение синтаксиса Dart и основных концепций Flutter
    \item Освоение инструментов разработки для Flutter (Visual Studio Code)
    \item Проектирование пользовательского интерфейса приложения
    \item Реализация функционала для управления изображениями
\end{itemize}

\textbf{Решённые задачи:}
\begin{itemize}
    \item Создание главного экрана с галереей изображений
    \item Обеспечение плавного и интерактивного пользовательского опыта
\end{itemize}

\section*{Заключение}
\addcontentsline{toc}{section}{Заключение}
В ходе производственной практики были приобретены ценные навыки в разработке мобильных приложений, изучив языка программирования и их экосистему. Практика позволила улучшить знания и умения в области программирования и разработки пользовательских интерфейсов.

\end{document}