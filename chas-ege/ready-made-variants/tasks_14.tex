



\cleardoublepage
\def\examvart{Вариант 14.1}
\normalsize

\begin{center}
	\textbf{
		Единый государственный экзамен\\по МАТЕМАТИКЕ\\Профильный уровень\\ \qquad \\ Инструкция по выполнению работы
	}
\end{center}


\par \qquad Экзаменационная работа состоит из двух частей, включающих в себя 18 заданий. Часть 1 содержит 11 заданий с кратким ответом базового и повышенного уровней сложности. Часть 2 содержит 7 заданий с развёрнутым ответом повышенного и высокого уровней сложности.
\par \qquad На выполнение экзаменационной работы по математике отводится 3 часа 55 минут (235 минут).
\par \qquad Ответы к заданиям 1—11 записываются по приведённому ниже \underline {образцу} в виде целого числа или конечной десятичной дроби. Числа запишите в поля ответов в тексте работы, а затем перенесите их в бланк ответов №1.
%%\includegraphics[width=0.98\linewidth]{obrazec}
\par \qquad При выполнении заданий 12—18 требуется записать полное решение и ответ в бланке ответов №2.
\par \qquad  Все бланки ЕГЭ заполняются яркими чёрными чернилами. Допускается использование гелевой или капиллярной ручки.
\par \qquad При выполнении заданий можно пользоваться черновиком. \textbf{Записи в черновике, а также в тексте контрольных измерительных материалов не учитываются при оценивании работы.}
\par \qquad  Баллы, полученные Вами за выполненные задания, суммируются. Постарайтесь выполнить как можно больше заданий и набрать наибольшее количество баллов.
\par \qquad После завершения работы проверьте, что ответ на каждое задание в бланках ответов №1 и №2 записан под правильным номером.
\begin{center}
	\textit{\textbf{Желаем успеха!}}\\ \qquad \\\textbf{ Справочные материалы} \\
$\sin^2 \alpha + \cos^2 \alpha = 1$ \\
$\sin 2\alpha=2\sin \alpha \cdot \cos \alpha$ \\
$\cos 2\alpha=\cos^2 \alpha-\sin^2 \alpha$ \\
$\sin (\alpha+\beta)=\sin \alpha \cdot \cos \beta+\cos \alpha \cdot \sin\beta$ \\
$\cos (\alpha+\beta)=\cos \alpha \cdot \cos \beta-\sin\alpha \cdot \sin\beta$
\end{center}

\startpartone
\large




\begin{taskBN}{1}
Основания равнобедренной трапеции равны 34 и 52. Боковые стороны равны 41. Найдите тангенс острого угла трапеции. Ответ округлите до сотых.
\end{taskBN}

\begin{taskBN}{2}
В прямоугольном параллелепипеде  $AUQXA_{1}U_{1}Q_{1}X_{1}$  известно, что  $Q_{1}X_{1} = 3$, $QU = 4$, $QQ_{1} = 9$. Найдите площадь сечения, проходящего через вершины $X_{1}$, $X$ и $U$.
\end{taskBN}

\begin{taskBN}{3}
В чемпионате по лёгкой атлетике участвуют 40 спортсменок, среди которых 10 из Австралии и 18 из Словакии. Порядок, в котором выступают спортсменки, определяется жребием. Найдите вероятность того, что спортсменка, выступающая предпоследней, окажется из Словакии.
\end{taskBN}

\begin{taskBN}{4}
При изготовлении гвоздей радиусом 52 мм вероятность того, что радиус будет отличаться от заданного не более, чем на 0,011 мм, равна 0,974. Найдите вероятность того, что случайный гвоздь будет иметь радиус менее чем 51,989 мм или более чем 52,011 мм.
\end{taskBN}

\begin{taskBN}{5}
Найдите корень уравнения $$\log_{69}{(5+x)}=\log_{69}{(11-2x)}$$
\end{taskBN}

\begin{taskBN}{6}
Найдите значение выражения $$ \log_{0,5}16 $$
\end{taskBN}

\begin{taskBN}{7}
Прямая $y=50x+115$ является касательной к графику функции $y=8x^{2}+bx+403$. Найдите $b$, зная, что оно больше 111.
\end{taskBN}

\begin{taskBN}{8}
На верфи инженеры проектируют новый аппарат для погружения на небольшие глубины. Конструкция имеет форму сферы, а значит, действующая на аппарат выталкивающая (архимедова) сила, выражаемая в ньютонах, будет определяться по формуле:  $F_{\rm{A}}  = \alpha \rho gr^3$, где $\alpha  = 4,2$ — постоянная, $r$ — радиус аппарата в метрах, $\rho  = 1000~\mbox{кг}/\mbox{м}^3$ — плотность воды, а $g$ — ускорение свободного падения (считайте $g = 10$ Н/кг). Каков может быть минимальный радиус аппарата, чтобы выталкивающая сила при погружении была не меньше, чем $2688000$ Н? Ответ выразите в метрах.
\end{taskBN}

\begin{taskBN}{9}
Из пункта Z в пункт B одновременно выехали два грузовика. Второй проехал с постоянной скоростью весь путь. Первый проехал первую половину пути со скоростью, на 90 км/ч большей скорости второго, а вторую половину пути — со скоростью 21 км/ч, в результате чего прибыл в пункт B одновременно co вторым грузовиком. Найдите скорость второго грузовика. Ответ дайте в км/ч.
\end{taskBN}

\begin{taskBN}{10}
\addpictocenter[]{images/858870914573974n0}На рисунке изображён график функции $f(x)=k\sqrt{x}$. Найдите значение $x$, при котором $f(x)=-20$. 
\end{taskBN}

\begin{taskBN}{11}
Определите точку минимума функции $y = -27x+2x \sqrt{x}$ на луче $\left[49;\infty \right)$
\end{taskBN}

\begin{taskBN}{12}
В июне 2011 года планируется взять кредит в банке на сумму S миллионов рублей на 4 года, где S – целое число. Условия его возврата таковы:<br>– каждый январь долг возрастает на 6% по сравнению с концом предыдущего года;<br>– с февраля по май каждого года необходимо выплатить одним платежом часть долга.<br>– в июне каждого года долг должен составлять некоторую сумму, выраженную в миллионах рублей, в соответствии со следующей таблицей:<br><center><br/><table border="1" cellspacing="0" cellpadding="12"><tr><td>2011</td><td>2012</td><td>2013</td><td>2014</td><td>2015</td></tr><tr><td>$S$</td><td>$\frac{1}{2}S$</td><td>$\frac{2}{5}S$</td><td>$\frac{1}{5}S$</td><td>$0$</td></tr></table></center><br/>Известно, что общая сумма выплат составила 18,016 миллионов рублей. Какая сумма в миллионах рублей была взята в кредит? 
\end{taskBN}

%\newpage
% Ответы
%
%\begin{table}[h]\begin{tabular}{|l|l|}
%\hline
%1 & 4,44
%\\
%\hline
%2 & 45
%\\
%\hline
%3 & 0,45
%\\
%\hline
%4 & 0,026
%\\
%\hline
%5 & 2
%\\
%\hline
%6 & -4
%\\
%\hline
%7 & 146
%\\
%\hline
%8 & 4
%\\
%\hline
%9 & 36
%\\
%\hline
%10 & 250
%\\
%\hline
%11 & 81
%\\
%\hline
%12 & 64885
%\\
%\hline
%\end{tabular}\end{table}
%


%\newpage




\cleardoublepage
\def\examvart{Вариант 14.2}
\normalsize

\begin{center}
	\textbf{
		Единый государственный экзамен\\по МАТЕМАТИКЕ\\Профильный уровень\\ \qquad \\ Инструкция по выполнению работы
	}
\end{center}


\par \qquad Экзаменационная работа состоит из двух частей, включающих в себя 18 заданий. Часть 1 содержит 11 заданий с кратким ответом базового и повышенного уровней сложности. Часть 2 содержит 7 заданий с развёрнутым ответом повышенного и высокого уровней сложности.
\par \qquad На выполнение экзаменационной работы по математике отводится 3 часа 55 минут (235 минут).
\par \qquad Ответы к заданиям 1—11 записываются по приведённому ниже \underline {образцу} в виде целого числа или конечной десятичной дроби. Числа запишите в поля ответов в тексте работы, а затем перенесите их в бланк ответов №1.
%%\includegraphics[width=0.98\linewidth]{obrazec}
\par \qquad При выполнении заданий 12—18 требуется записать полное решение и ответ в бланке ответов №2.
\par \qquad  Все бланки ЕГЭ заполняются яркими чёрными чернилами. Допускается использование гелевой или капиллярной ручки.
\par \qquad При выполнении заданий можно пользоваться черновиком. \textbf{Записи в черновике, а также в тексте контрольных измерительных материалов не учитываются при оценивании работы.}
\par \qquad  Баллы, полученные Вами за выполненные задания, суммируются. Постарайтесь выполнить как можно больше заданий и набрать наибольшее количество баллов.
\par \qquad После завершения работы проверьте, что ответ на каждое задание в бланках ответов №1 и №2 записан под правильным номером.
\begin{center}
	\textit{\textbf{Желаем успеха!}}\\ \qquad \\\textbf{ Справочные материалы} \\
$\sin^2 \alpha + \cos^2 \alpha = 1$ \\
$\sin 2\alpha=2\sin \alpha \cdot \cos \alpha$ \\
$\cos 2\alpha=\cos^2 \alpha-\sin^2 \alpha$ \\
$\sin (\alpha+\beta)=\sin \alpha \cdot \cos \beta+\cos \alpha \cdot \sin\beta$ \\
$\cos (\alpha+\beta)=\cos \alpha \cdot \cos \beta-\sin\alpha \cdot \sin\beta$
\end{center}

\startpartone
\large




\begin{taskBN}{1}
Четырехугольник ABCD вписан в окружность. Угол ABC равен 36°, угол CAD равен 31°. Найдите угол ABD. Ответ дайте в градусах.
\end{taskBN}

\begin{taskBN}{2}
Два ребра прямоугольного параллелепипеда, выходящие из одной вершины, равны 7 и 6. Известно, что диагональ составляет 11. Найдите третье выходящее из той же вершины ребро параллелепипеда.
\end{taskBN}

\begin{taskBN}{3}
В случайном эксперименте симметричную монету бросают 3 раза. Какова вероятность того, что орёл выпадет трижды?
\end{taskBN}

\begin{taskBN}{4}
Из деревни в районный центр каждый день ходит электричка. Вероятность того, что в субботу в электричке окажется меньше 27 пассажиров, равна 0.86. Вероятность того, что окажется меньше 19 пассажиров, равна 0.46. Найдите вероятность того, что число пассажиров будет от 19 до 26.
\end{taskBN}

\begin{taskBN}{5}
Найдите корень уравнения $$\sqrt{8x+3}-4x=0$$ Если корней несколько, в ответе укажите меньший из них.
\end{taskBN}

\begin{taskBN}{6}
Найдите значение выражения $$\frac{\log_{17}{2}}{\log_{17}{11}}+\log_{11}{0.5}$$
\end{taskBN}

\begin{taskBN}{7}
\addpictoright[0.8\linewidth]{images/819454185429666n0}На рисунке изображен график функции $y = f(x)$, определенной на интервале $(-4; 5)$. Найдите количество точек, в которых производная функции $f(x)$ равна $0$.
\end{taskBN}

\begin{taskBN}{8}
Зависимость объёма спроса $q$ (единиц в месяц) на продукцию предприятия-монополиста от цены $p$ (тыс. руб.) задаётся формулой $q=80-5p$. Выручка предприятия за месяц $r$ (в тыс. руб.) вычисляется по формуле $r(p)=q\cdot p$. Определите наибольшую цену $p$, при которой месячная выручка $r(p)$ составит не менее 315 тыс. руб. Ответ приведите в тыс. руб.
\end{taskBN}

\begin{taskBN}{9}
Моторная лодка прошла против течения реки и вернулась в пункт отправления. Чему равно суммарное пройденное лодкой расстояние, выраженное в км, если скорость лодки в неподвижной воде равна 8 км/ч, а лодка затратила на обратный путь на 31 час меньше? Скорость течения равна 4 км/ч. 
\end{taskBN}

\begin{taskBN}{10}
\addpictocenter[]{images/985844913435732n0}На рисунке изображён график функции $f(x)=\frac{k}{x+a}$. Найдите значение $x$, при котором $f(x)=-12$.
\end{taskBN}

\begin{taskBN}{11}
Определите наименьшее значение функции $y = -5\sqrt{3}\cos x-15-\frac{5\sqrt{3}}{2}x+\frac{5\sqrt{3}\pi}{12}$ на интервале $\left(-\frac{\pi}{6};\frac{\pi}{3} \right)$
\end{taskBN}

\begin{taskBN}{12}
17-го декабря планируется взять кредит в банке на сумму $S$ тыс. рублей на 30 месяцев. Условия его возврата таковы:<br/>— 1-го числа каждого месяца долг возрастает на 3\% по сравнению с концом предыдущего месяца;<br/>— со 2-го по 16-е число каждого месяца необходимо выплатить часть долга;<br/>— 17-го числа каждого месяца долг должен быть на одну и ту же величину меньше долга на 17-е число предыдущего месяца.<br/>Общая сумма выплат составит 922,95 тыс. рублей. Какую сумму планируется взять в кредит?
\end{taskBN}

%\newpage
% Ответы
%
%\begin{table}[h]\begin{tabular}{|l|l|}
%\hline
%1 & 5 
%\\
%\hline
%2 & 6
%\\
%\hline
%3 & 0,125
%\\
%\hline
%4 & 0,4
%\\
%\hline
%5 & 0,75
%\\
%\hline
%6 & 0
%\\
%\hline
%7 & 3
%\\
%\hline
%8 & 9
%\\
%\hline
%9 & 372
%\\
%\hline
%10 & 2,5
%\\
%\hline
%11 & -22,5
%\\
%\hline
%12 & 630
%\\
%\hline
%\end{tabular}\end{table}
%
%
%
%\newpage




\cleardoublepage
\def\examvart{Вариант 14.3}
\normalsize

\begin{center}
	\textbf{
		Единый государственный экзамен\\по МАТЕМАТИКЕ\\Профильный уровень\\ \qquad \\ Инструкция по выполнению работы
	}
\end{center}


\par \qquad Экзаменационная работа состоит из двух частей, включающих в себя 18 заданий. Часть 1 содержит 11 заданий с кратким ответом базового и повышенного уровней сложности. Часть 2 содержит 7 заданий с развёрнутым ответом повышенного и высокого уровней сложности.
\par \qquad На выполнение экзаменационной работы по математике отводится 3 часа 55 минут (235 минут).
\par \qquad Ответы к заданиям 1—11 записываются по приведённому ниже \underline {образцу} в виде целого числа или конечной десятичной дроби. Числа запишите в поля ответов в тексте работы, а затем перенесите их в бланк ответов №1.
%%\includegraphics[width=0.98\linewidth]{obrazec}
\par \qquad При выполнении заданий 12—18 требуется записать полное решение и ответ в бланке ответов №2.
\par \qquad  Все бланки ЕГЭ заполняются яркими чёрными чернилами. Допускается использование гелевой или капиллярной ручки.
\par \qquad При выполнении заданий можно пользоваться черновиком. \textbf{Записи в черновике, а также в тексте контрольных измерительных материалов не учитываются при оценивании работы.}
\par \qquad  Баллы, полученные Вами за выполненные задания, суммируются. Постарайтесь выполнить как можно больше заданий и набрать наибольшее количество баллов.
\par \qquad После завершения работы проверьте, что ответ на каждое задание в бланках ответов №1 и №2 записан под правильным номером.
\begin{center}
	\textit{\textbf{Желаем успеха!}}\\ \qquad \\\textbf{ Справочные материалы} \\
$\sin^2 \alpha + \cos^2 \alpha = 1$ \\
$\sin 2\alpha=2\sin \alpha \cdot \cos \alpha$ \\
$\cos 2\alpha=\cos^2 \alpha-\sin^2 \alpha$ \\
$\sin (\alpha+\beta)=\sin \alpha \cdot \cos \beta+\cos \alpha \cdot \sin\beta$ \\
$\cos (\alpha+\beta)=\cos \alpha \cdot \cos \beta-\sin\alpha \cdot \sin\beta$
\end{center}

\startpartone
\large




\begin{taskBN}{1}
В треугольнике $VTM$ угол $V$ равен $90^\circ$. Чему равна  $MV$, если $\tg{T}=\frac{24}{7}$?  $TV=28$. 
\end{taskBN}

\begin{taskBN}{2}
Eсли ребро куба увеличить на 4, то квадрат диагонали увеличится на 120. Найдите объём куба.
\end{taskBN}

\begin{taskBN}{3}
В случайном эксперименте симметричную монету бросают 2 раза. Какова вероятность того, что решка выпадет дважды?
\end{taskBN}

\begin{taskBN}{4}
Вероятность того, что на контрольной работе по математике учащийся Г. верно решит больше 11 задач, равна 0.61. Вероятность того, что Г. верно решит больше 10 задач, равна 0.79. Найдите вероятность того, что Г. верно решит ровно 11 задач.
\end{taskBN}

\begin{taskBN}{5}
Найдите корень уравнения $$\log_{21}54=\log_{21}(110-8x)$$
\end{taskBN}

\begin{taskBN}{6}
Найдите значение выражения $$\frac{\sqrt [4]{3}\cdot \sqrt [4]{16}}{\sqrt [4]{3}}$$
\end{taskBN}

\begin{taskBN}{7}
\addpictoright[0.8\linewidth]{images/182290030319229n0}На рисунке изображен график функции $y = f(x)$, определенной на интервале $(-3; 3)$. Найдите корень уравнения $f'(x)=0$.
\end{taskBN}

\begin{taskBN}{8}
В ходе распада радиоактивного изотопа его масса уменьшается по закону $m(t) = m_02^{-t/T}$, где $m_0$ — начальная масса изотопа, $t$ (ч) — прошедшее от начального момента время, $T$ — период полураспада в часах. В лаборатории получили вещество, содержащее в начальный момент времени $m (t) = 400$ г изотопа Z, период полураспада которого $T = 78$ ч. В течение скольких часов масса изотопа будет больше 50 г?
\end{taskBN}

\begin{taskBN}{9}
Половину времени, затраченного на поездку, "Москвич" двигался со скоростью 36 км/ч, а оставшуюся часть времени – со скоростью 41 км/ч. Найдите среднюю скорость "Москвича" на протяжении всего маршрута. Ответ дайте в км/ч.
\end{taskBN}

\begin{taskBN}{10}
\addpictocenter[]{images/771431586316975n0}На рисунке изображён график функции $f(x)=k\sqrt{x}$. Найдите $f(49)$. 
\end{taskBN}

\begin{taskBN}{11}
Вычислите точку минимума функции $y = -(x+16)^{2}e^{x+46}-46$
\end{taskBN}

\begin{taskBN}{12}
В сентябре 2022 года планируется взять кредит в банке на сумму 314450 рублей. Условия его возврата таковы:<br/>– каждый январь долг возрастает на 10% по сравнению с концом предыдущего года;<br/>– с февраля по август каждого года необходимо выплатить одним платежом часть долга; <br/>– кредит будет полностью погашен 3 равными платежами, то есть за 3 года. <br/>Сколько рублей составит переплата?
\end{taskBN}

%\newpage
% Ответы
%
%\begin{table}[h]\begin{tabular}{|l|l|}
%\hline
%1 & 96
%\\
%\hline
%2 & 27
%\\
%\hline
%3 & 0,25
%\\
%\hline
%4 & 0,18
%\\
%\hline
%5 & 7
%\\
%\hline
%6 & 2
%\\
%\hline
%7 & 0
%\\
%\hline
%8 & 234
%\\
%\hline
%9 & 38,5
%\\
%\hline
%10 & -42
%\\
%\hline
%11 & -18
%\\
%\hline
%12 & 64885
%\\
%\hline
%\end{tabular}\end{table}



\newpage
