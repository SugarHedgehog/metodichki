\subsection{Маршруты, связность, метрика графа}

\noindent\textbf{Определение.} \textit{Маршрут} в графе $G$ (последовательность переходов по вершинам и рёбрам) -- чередующаяся последовательность вершин и рёбер $v_0, x_1, v_1, \ldots, x_n, v_n$, где:
\begin{itemize}[noitemsep,topsep=0pt]
\item Начинается и заканчивается вершиной (точкой графа)
\item Каждое ребро инцидентно (напрямую соединяет) предшествующей и следующей вершинам
\end{itemize}

\noindent\textbf{Обозначение:} $(v_0-v_n)$-маршрут (путь от вершины $v_0$ до $v_n$) записывается как $v_0 v_1 v_2 \ldots v_n$

\noindent\textbf{Классификация маршрутов:}
\begin{itemize}[noitemsep,topsep=0pt]
\item \textit{Замкнутый}: $v_0 = v_n$ (начальная и конечная вершины совпадают)
\item \textit{Открытый}: $v_0 \neq v_n$ (начальная и конечная вершины различны)
\item \textit{Цепь} (trail): все рёбра различны (по каждому ребру проходим не более одного раза)
\item \textit{Простая цепь} (path): все вершины и рёбра различны (нигде не повторяемся)
\item \textit{Цикл}: замкнутая цепь (маршрут возвращается в начальную точку)
\item \textit{Простой цикл}: замкнутый маршрут с $n \geq 3$ различными вершинами (замкнутый путь без повторений вершин, кроме начальной/конечной)
\end{itemize}

\noindent\textbf{Длина маршрута} $v_0 v_1 \ldots v_n$ = $n$ (количество пройденных рёбер)

\noindent\textbf{Важные метрики:}
\begin{itemize}[noitemsep,topsep=0pt]
\item \textit{Обхват графа} $g(G)$: длина кратчайшего простого цикла (минимальное количество рёбер в замкнутом пути без повторений)
\item \textit{Окружение графа} $c(G)$: длина длиннейшего простого цикла (максимальное количество рёбер в замкнутом пути без повторений)
\end{itemize}

\noindent\textbf{Примечание:} $g(G)$ и $c(G)$ не определены для графов без циклов (для деревьев и лесов).