\documentclass[4apaper]{article}
\usepackage{dashbox}
\usepackage[T2A]{fontenc}
\usepackage[utf8]{inputenc}
\usepackage[english,russian]{babel}
\usepackage{graphicx}
\DeclareGraphicsExtensions{.pdf,.png,.jpg}

\linespread{1.15}

\usepackage{../egetask_ver}

\def\examyear{2023}
\usepackage[colorlinks,linkcolor=blue]{hyperref}\usepackage{draftwatermark}
\SetWatermarkLightness{0.9}
\SetWatermarkText{https://vk.com/egemathika}
\SetWatermarkScale{ 0.4 }
\def\lfoottext{Источник \href{https://vk.com/egemathika}{https://vk.com/egemathika}}

\begin{document}
\begin{taskBN}{1}
\addpictoright[0.3\textwidth]{images/7346342941200337n0} Первая цилиндрическая кружка в 6 раз шире второй, а вторая в 8 раз выше первой. Найдите отношение объёма первой кружки к объёму второй.
\end{taskBN}

\begin{taskBN}{2}
\addpictoright[0.3\textwidth]{images/644607319520861n0}Площадь полной поверхности цилиндра равна $1140\pi$, площадь боковой поверхности равна $940\pi$. Найдите длину окружности основания цилиндра. Ответ сократите на $\pi$.
\end{taskBN}

\begin{taskBN}{3}
\addpictoright[0.3\textwidth]{images/122281657905517n0}В цилиндрическом сосуде уровень жидкости достигает 49 см. На какой высоте будет находиться уровень жидкости, если её перелить во второй цилиндрический сосуд, диаметр которого в 7 раз больше диаметра первого? Ответ выразите в сантиметрах.
\end{taskBN}

\begin{taskBN}{4}
\addpictoright[0.3\textwidth]{images/8627031513238681n0}В цилиндрическом сосуде уровень жидкости достигает 96 см. На какой высоте будет находиться уровень жидкости, если её перелить во второй цилиндрический сосуд, диаметр которого в 4 раза меньше диаметра первого? Ответ выразите в сантиметрах.
\end{taskBN}

\begin{taskBN}{5}
\addpictoright[0.3\textwidth]{images/652669053135838n0}В цилиндрическом сосуде уровень жидкости достигает 32 см. На какой высоте будет находиться уровень жидкости, если её перелить во второй цилиндрический сосуд, диаметр которого в 4 раза меньше диаметра первого? Ответ выразите в сантиметрах.
\end{taskBN}

\begin{taskBN}{6}
\addpictoright[0.3\textwidth]{images/038326182866423n0} Первая цилиндрическая кружка в 6 раз шире второй, а вторая в 2 раза выше первой. Найдите отношение объёма первой кружки к объёму второй.
\end{taskBN}

\begin{taskBN}{7}
\addpictoright[0.3\textwidth]{images/220655515280736n0}В цилиндрическом сосуде уровень жидкости достигает 96 см. На какой высоте будет находиться уровень жидкости, если её перелить во второй цилиндрический сосуд, радиус которого в 4 раза больше радиуса первого? Ответ выразите в сантиметрах.
\end{taskBN}

\begin{taskBN}{8}
\addpictoright[0.3\textwidth]{images/366158590882638n0}В цилиндрическом сосуде уровень жидкости достигает 81 см. На какой высоте будет находиться уровень жидкости, если её перелить во второй цилиндрический сосуд, радиус которого в 9 раз меньше радиуса первого? Ответ выразите в сантиметрах.
\end{taskBN}

\begin{taskBN}{9}
\addpictoright[0.3\textwidth]{images/1156286025086459n0}В цилиндрический сосуд налили $1600\mbox{см}^3$ воды. Уровень воды при этом достигает высоты $5$см. В жидкость полностью погрузили деталь. При этом уровень жидкости в сосуде поднялся на $2$см. Чему равен объем детали? Ответ выразите в $\mbox{см}^3$.
\end{taskBN}

\begin{taskBN}{10}
\addpictoright[0.3\textwidth]{images/166420757742594n0}В цилиндрическом сосуде уровень жидкости достигает 98 см. На какой высоте будет находиться уровень жидкости, если её перелить во второй цилиндрический сосуд, радиус которого в 7 раз меньше радиуса первого? Ответ выразите в сантиметрах.
\end{taskBN}

\begin{taskBN}{11}
\addpictoright[0.3\textwidth]{images/960092538975555n0}В цилиндрическом сосуде уровень жидкости достигает 56 см. На какой высоте будет находиться уровень жидкости, если её перелить во второй цилиндрический сосуд, радиус которого в 2 раза больше радиуса первого? Ответ выразите в сантиметрах.
\end{taskBN}

\begin{taskBN}{12}
\addpictoright[0.3\textwidth]{images/29668781086253n0}Площадь полной поверхности цилиндра равна $816\pi$, площадь боковой поверхности равна $238\pi$. Найдите объём цилиндра. Ответ сократите на $\pi$.
\end{taskBN}

\begin{taskBN}{13}
\addpictoright[0.3\textwidth]{images/6121681885056276n0}Площадь боковой поверхности цилиндра равна $3354\pi$, длина окружности основания равна $78\pi$. Найдите площадь полной поверхности цилиндра. Ответ сократите на $\pi$.
\end{taskBN}

\begin{taskBN}{14}
\addpictoright[0.3\textwidth]{images/306844105305256n0}В цилиндрическом сосуде уровень жидкости достигает 98 см. На какой высоте будет находиться уровень жидкости, если её перелить во второй цилиндрический сосуд, радиус которого в 7 раз меньше радиуса первого? Ответ выразите в сантиметрах.
\end{taskBN}

\begin{taskBN}{15}
\addpictoright[0.3\textwidth]{images/508613199750648n0}В цилиндрическом сосуде уровень жидкости достигает 56 см. На какой высоте будет находиться уровень жидкости, если её перелить во второй цилиндрический сосуд, радиус которого в 2 раза больше радиуса первого? Ответ выразите в сантиметрах.
\end{taskBN}

\begin{taskBN}{16}
\addpictoright[0.3\textwidth]{images/2448455110849344n0}В цилиндрическом сосуде уровень жидкости достигает 76 см. На какой высоте будет находиться уровень жидкости, если её перелить во второй цилиндрический сосуд, диаметр которого в 2 раза больше диаметра первого? Ответ выразите в сантиметрах.
\end{taskBN}

\begin{taskBN}{17}
\addpictoright[0.3\textwidth]{images/36882251057103n0}Площадь боковой поверхности цилиндра равна $2160\pi$, высота равна $40$. Найдите объём цилиндра. Ответ сократите на $\pi$.
\end{taskBN}

\begin{taskBN}{18}
\addpictoright[0.3\textwidth]{images/210157271758252n0}В цилиндрическом сосуде уровень жидкости достигает 81 см. На какой высоте будет находиться уровень жидкости, если её перелить во второй цилиндрический сосуд, диаметр которого в 9 раз меньше диаметра первого? Ответ выразите в сантиметрах.
\end{taskBN}

\begin{taskBN}{19}
\addpictoright[0.3\textwidth]{images/7815653838197767n0}В цилиндрическом сосуде уровень жидкости достигает 75 см. На какой высоте будет находиться уровень жидкости, если её перелить во второй цилиндрический сосуд, радиус которого в 5 раз больше радиуса первого? Ответ выразите в сантиметрах.
\end{taskBN}

\begin{taskBN}{20}
\addpictoright[0.3\textwidth]{images/6358033224056148n0} Первая цилиндрическая кружка в 10 раз шире второй, а вторая в 4 раза выше первой. Найдите отношение объёма второй кружки к объёму первой.
\end{taskBN}

\begin{taskBN}{21}
\addpictoright[0.3\textwidth]{images/706617772729927n0}Объём цилиндра равен $68850\pi$, площадь полной поверхности равна $7110\pi$. Найдите площадь боковой поверхности цилиндра. Ответ сократите на $\pi$.
\end{taskBN}

\begin{taskBN}{22}
\addpictoright[0.3\textwidth]{images/2903218410469253n0}Высота цилиндра равна $35$, площадь полной поверхности равна $1908\pi$. Найдите площадь боковой поверхности цилиндра. Ответ сократите на $\pi$.
\end{taskBN}

\begin{taskBN}{23}
\addpictoright[0.3\textwidth]{images/8976562665538106n0} Первая цилиндрическая кружка в 2,5 раза шире второй, а вторая в 6,5 раз выше первой. Найдите отношение объёма второй кружки к объёму первой.
\end{taskBN}

\begin{taskBN}{24}
\addpictoright[0.3\textwidth]{images/6940363153704343n0}В цилиндрическом сосуде уровень жидкости достигает 32 см. На какой высоте будет находиться уровень жидкости, если её перелить во второй цилиндрический сосуд, диаметр которого в 4 раза больше диаметра первого? Ответ выразите в сантиметрах.
\end{taskBN}

\begin{taskBN}{25}
\addpictoright[0.3\textwidth]{images/9672771305211356n0}В цилиндрическом сосуде уровень жидкости достигает 68 см. На какой высоте будет находиться уровень жидкости, если её перелить во второй цилиндрический сосуд, диаметр которого в 2 раза больше диаметра первого? Ответ выразите в сантиметрах.
\end{taskBN}

\begin{taskBN}{26}
\addpictoright[0.3\textwidth]{images/904566356720066n0}В цилиндрическом сосуде уровень жидкости достигает 48 см. На какой высоте будет находиться уровень жидкости, если её перелить во второй цилиндрический сосуд, диаметр которого в 4 раза больше диаметра первого? Ответ выразите в сантиметрах.
\end{taskBN}

\begin{taskBN}{27}
\addpictoright[0.3\textwidth]{images/932497424745488n0}Объём цилиндра равен $15210\pi$, высота равна $10$. Найдите площадь окружности основания цилиндра. Ответ сократите на $\pi$.
\end{taskBN}

\begin{taskBN}{28}
\addpictoright[0.3\textwidth]{images/0338483671809107n0}Объём цилиндра равен $171\pi$, радиус основания равен $3$. Найдите площадь полной поверхности цилиндра. Ответ сократите на $\pi$.
\end{taskBN}

\begin{taskBN}{29}
\addpictoright[0.3\textwidth]{images/789421599304454n0} Первая цилиндрическая кружка в 5 раз шире второй, а вторая в 8 раз выше первой. Найдите отношение объёма первой кружки к объёму второй.
\end{taskBN}

\begin{taskBN}{30}
\addpictoright[0.3\textwidth]{images/000533801161557n0}Высота цилиндра равна $24$, площадь окружности основания равна $2401\pi$. Найдите площадь полной поверхности цилиндра. Ответ сократите на $\pi$.
\end{taskBN}

\begin{taskBN}{31}
\addpictoright[0.3\textwidth]{images/679837627946356n0}В цилиндрический сосуд налили $4600\mbox{дм}^3$ воды. Уровень воды при этом достигает высоты $46$дм. В жидкость полностью погрузили деталь. При этом уровень жидкости в сосуде поднялся в $\frac{41}{23}$ раза. Чему равен объем детали? Ответ выразите в $\mbox{дм}^3$.
\end{taskBN}

\begin{taskBN}{32}
\addpictoright[0.3\textwidth]{images/509068388793693n0} Первая цилиндрическая кружка в 2,5 раза выше второй, а вторая в 8,5 раз шире первой. Найдите отношение объёма второй кружки к объёму первой.
\end{taskBN}

\begin{taskBN}{33}
\addpictoright[0.3\textwidth]{images/266557334449457n0}В цилиндрическом сосуде уровень жидкости достигает 9 см. На какой высоте будет находиться уровень жидкости, если её перелить во второй цилиндрический сосуд, радиус которого в 3 раза меньше радиуса первого? Ответ выразите в сантиметрах.
\end{taskBN}

\begin{taskBN}{34}
\addpictoright[0.3\textwidth]{images/3757545814489083n0}Объём цилиндра равен $52020\pi$, длина окружности основания равна $68\pi$. Найдите площадь боковой поверхности цилиндра. Ответ сократите на $\pi$.
\end{taskBN}

\begin{taskBN}{35}
\addpictoright[0.3\textwidth]{images/524641834376089n0}Площадь полной поверхности цилиндра равна $3168\pi$, высота равна $50$. Найдите длину окружности основания цилиндра. Ответ сократите на $\pi$.
\end{taskBN}

\begin{taskBN}{36}
\addpictoright[0.3\textwidth]{images/450016420886221n0} Первая цилиндрическая кружка в 3 раза шире второй, а вторая в 4 раза выше первой. Найдите отношение объёма первой кружки к объёму второй.
\end{taskBN}

\begin{taskBN}{37}
\addpictoright[0.3\textwidth]{images/781788713340713n0}В цилиндрическом сосуде уровень жидкости достигает 8 см. На какой высоте будет находиться уровень жидкости, если её перелить во второй цилиндрический сосуд, радиус которого в 2 раза больше радиуса первого? Ответ выразите в сантиметрах.
\end{taskBN}

\begin{taskBN}{38}
\addpictoright[0.3\textwidth]{images/258900665758801n0}Высота цилиндра равна $13$, площадь полной поверхности равна $2028\pi$. Найдите объём цилиндра. Ответ сократите на $\pi$.
\end{taskBN}

\begin{taskBN}{39}
\addpictoright[0.3\textwidth]{images/8403694912113788n0}В цилиндрический сосуд налили $400\mbox{м}^3$ воды. Уровень воды при этом достигает высоты $80$м. В жидкость полностью погрузили деталь. При этом уровень жидкости в сосуде поднялся на $13$м. Чему равен объем детали? Ответ выразите в $\mbox{м}^3$.
\end{taskBN}

\begin{taskBN}{40}
\addpictoright[0.3\textwidth]{images/2324356293874208n0}В цилиндрическом сосуде уровень жидкости достигает 16 см. На какой высоте будет находиться уровень жидкости, если её перелить во второй цилиндрический сосуд, диаметр которого в 4 раза меньше диаметра первого? Ответ выразите в сантиметрах.
\end{taskBN}

\begin{taskBN}{41}
\addpictoright[0.3\textwidth]{images/1213402428978811n0}Высота цилиндра равна $11$, объём равен $5819\pi$. Найдите площадь полной поверхности цилиндра. Ответ сократите на $\pi$.
\end{taskBN}

\begin{taskBN}{42}
\addpictoright[0.3\textwidth]{images/8490133624518137n0}В цилиндрическом сосуде уровень жидкости достигает 28 см. На какой высоте будет находиться уровень жидкости, если её перелить во второй цилиндрический сосуд, радиус которого в 2 раза больше радиуса первого? Ответ выразите в сантиметрах.
\end{taskBN}

\begin{taskBN}{43}
\addpictoright[0.3\textwidth]{images/305478563913472n0}Площадь полной поверхности цилиндра равна $1798\pi$, объём равен $1682\pi$. Найдите площадь боковой поверхности цилиндра. Ответ сократите на $\pi$.
\end{taskBN}

\begin{taskBN}{44}
\addpictoright[0.3\textwidth]{images/564908456931848n0}В цилиндрическом сосуде уровень жидкости достигает 36 см. На какой высоте будет находиться уровень жидкости, если её перелить во второй цилиндрический сосуд, радиус которого в 6 раз больше радиуса первого? Ответ выразите в сантиметрах.
\end{taskBN}

\begin{taskBN}{45}
\addpictoright[0.3\textwidth]{images/057303017899n0}В цилиндрический сосуд налили $4000\mbox{дм}^3$ воды. Уровень воды при этом достигает высоты $50$дм. В жидкость полностью погрузили деталь. При этом уровень жидкости в сосуде поднялся на $33$дм. Чему равен объем детали? Ответ выразите в $\mbox{дм}^3$.
\end{taskBN}

\begin{taskBN}{46}
\addpictoright[0.3\textwidth]{images/028108732340887n0}В цилиндрическом сосуде уровень жидкости достигает 54 см. На какой высоте будет находиться уровень жидкости, если её перелить во второй цилиндрический сосуд, радиус которого в 3 раза больше радиуса первого? Ответ выразите в сантиметрах.
\end{taskBN}

\begin{taskBN}{47}
\addpictoright[0.3\textwidth]{images/525865002949094n0}В цилиндрическом сосуде уровень жидкости достигает 28 см. На какой высоте будет находиться уровень жидкости, если её перелить во второй цилиндрический сосуд, радиус которого в 2 раза меньше радиуса первого? Ответ выразите в сантиметрах.
\end{taskBN}

\begin{taskBN}{48}
\addpictoright[0.3\textwidth]{images/498815649580759n0} Первая цилиндрическая кружка в 8 раз выше второй, а вторая в 10 раз шире первой. Найдите отношение объёма первой кружки к объёму второй.
\end{taskBN}

\begin{taskBN}{49}
\addpictoright[0.3\textwidth]{images/54641625072808n0}В цилиндрический сосуд налили $4500\mbox{дм}^3$ воды. Уровень воды при этом достигает высоты $1500$дм. В жидкость полностью погрузили деталь. При этом уровень жидкости в сосуде поднялся в ${1}\frac{493}{1500}$ раза. Чему равен объем детали? Ответ выразите в $\mbox{дм}^3$.
\end{taskBN}

\begin{taskBN}{50}
\addpictoright[0.3\textwidth]{images/40074942677233n0}Площадь боковой поверхности цилиндра равна $2784\pi$, длина окружности основания равна $96\pi$. Найдите объём цилиндра. Ответ сократите на $\pi$.
\end{taskBN}

\begin{taskBN}{51}
\addpictoright[0.3\textwidth]{images/91974934455559n0}В цилиндрическом сосуде уровень жидкости достигает 8 см. На какой высоте будет находиться уровень жидкости, если её перелить во второй цилиндрический сосуд, радиус которого в 2 раза меньше радиуса первого? Ответ выразите в сантиметрах.
\end{taskBN}

\begin{taskBN}{52}
\addpictoright[0.3\textwidth]{images/1972360224169565n0} Первая цилиндрическая кружка в 2 раза выше второй, а вторая в 8,5 раз шире первой. Найдите отношение объёма второй кружки к объёму первой.
\end{taskBN}

\begin{taskBN}{53}
\addpictoright[0.3\textwidth]{images/200071454847866n0} Первая цилиндрическая кружка в 6,5 раз выше второй, а вторая в 2,5 раза шире первой. Найдите отношение объёма первой кружки к объёму второй.
\end{taskBN}

\begin{taskBN}{54}
\addpictoright[0.3\textwidth]{images/2962460818401493n0}В цилиндрическом сосуде уровень жидкости достигает 27 см. На какой высоте будет находиться уровень жидкости, если её перелить во второй цилиндрический сосуд, диаметр которого в 3 раза больше диаметра первого? Ответ выразите в сантиметрах.
\end{taskBN}

\begin{taskBN}{55}
\addpictoright[0.3\textwidth]{images/7213607559715316n0} Первая цилиндрическая кружка в 4,5 раз выше второй, а вторая в 9 раз шире первой. Найдите отношение объёма второй кружки к объёму первой.
\end{taskBN}

\begin{taskBN}{56}
\addpictoright[0.3\textwidth]{images/466463352096782n0} Первая цилиндрическая кружка в 8 раз выше второй, а вторая в 2 раза шире первой. Найдите отношение объёма второй кружки к объёму первой.
\end{taskBN}

\begin{taskBN}{57}
\addpictoright[0.3\textwidth]{images/7867968594105024n0}В цилиндрическом сосуде уровень жидкости достигает 92 см. На какой высоте будет находиться уровень жидкости, если её перелить во второй цилиндрический сосуд, диаметр которого в 2 раза больше диаметра первого? Ответ выразите в сантиметрах.
\end{taskBN}

\begin{taskBN}{58}
\addpictoright[0.3\textwidth]{images/07547271379829n0}В цилиндрическом сосуде уровень жидкости достигает 25 см. На какой высоте будет находиться уровень жидкости, если её перелить во второй цилиндрический сосуд, радиус которого в 5 раз меньше радиуса первого? Ответ выразите в сантиметрах.
\end{taskBN}

\begin{taskBN}{59}
\addpictoright[0.3\textwidth]{images/0027955535644966n0}Высота цилиндра равна $8$, площадь полной поверхности равна $2146\pi$. Найдите объём цилиндра. Ответ сократите на $\pi$.
\end{taskBN}

\begin{taskBN}{60}
\addpictoright[0.3\textwidth]{images/464836200396997n0}В цилиндрическом сосуде уровень жидкости достигает 24 см. На какой высоте будет находиться уровень жидкости, если её перелить во второй цилиндрический сосуд, радиус которого в 2 раза меньше радиуса первого? Ответ выразите в сантиметрах.
\end{taskBN}

\begin{taskBN}{61}
\addpictoright[0.3\textwidth]{images/502548627015849n0}Высота цилиндра равна $19$, площадь окружности основания равна $2025\pi$. Найдите площадь боковой поверхности цилиндра. Ответ сократите на $\pi$.
\end{taskBN}

\begin{taskBN}{62}
\addpictoright[0.3\textwidth]{images/655868772308049n0}В цилиндрический сосуд налили $4700\mbox{дм}^3$ воды. Уровень воды при этом достигает высоты $20$дм. В жидкость полностью погрузили деталь. При этом уровень жидкости в сосуде поднялся в $ 1{,}8 $ раза. Чему равен объем детали? Ответ выразите в $\mbox{дм}^3$.
\end{taskBN}

\begin{taskBN}{63}
\addpictoright[0.3\textwidth]{images/433550211624588n0} Первая цилиндрическая кружка в 2,5 раза выше второй, а вторая в 7 раз шире первой. Найдите отношение объёма второй кружки к объёму первой.
\end{taskBN}

\begin{taskBN}{64}
\addpictoright[0.3\textwidth]{images/435695619059849n0}Площадь боковой поверхности цилиндра равна $248\pi$, объём равен $3844\pi$. Найдите длину окружности основания цилиндра. Ответ сократите на $\pi$.
\end{taskBN}

\begin{taskBN}{65}
\addpictoright[0.3\textwidth]{images/49521973077806n0}Диаметр основания цилиндра равен $22$, высота равна $11$. Найдите объём цилиндра. Ответ сократите на $\pi$.
\end{taskBN}

\begin{taskBN}{66}
\addpictoright[0.3\textwidth]{images/209877210428611n0}Площадь боковой поверхности цилиндра равна $864\pi$, площадь полной поверхности равна $5472\pi$. Найдите площадь окружности основания цилиндра. Ответ сократите на $\pi$.
\end{taskBN}

\begin{taskBN}{67}
\addpictoright[0.3\textwidth]{images/8666841991686891n0} Первая цилиндрическая кружка в 2 раза выше второй, а вторая в 2,5 раза шире первой. Найдите отношение объёма первой кружки к объёму второй.
\end{taskBN}

\begin{taskBN}{68}
\addpictoright[0.3\textwidth]{images/8755116082367764n0}В цилиндрическом сосуде уровень жидкости достигает 12 см. На какой высоте будет находиться уровень жидкости, если её перелить во второй цилиндрический сосуд, радиус которого в 2 раза меньше радиуса первого? Ответ выразите в сантиметрах.
\end{taskBN}

\begin{taskBN}{69}
\addpictoright[0.3\textwidth]{images/042744697546195n0}Площадь окружности основания цилиндра равна $484\pi$, площадь боковой поверхности равна $1584\pi$. Найдите объём цилиндра. Ответ сократите на $\pi$.
\end{taskBN}

\begin{taskBN}{70}
\addpictoright[0.3\textwidth]{images/836577015268618n0} Первая цилиндрическая кружка в 3 раза шире второй, а вторая в 9 раз выше первой. Найдите отношение объёма второй кружки к объёму первой.
\end{taskBN}

\begin{taskBN}{71}
\addpictoright[0.3\textwidth]{images/9635241186149774n0}В цилиндрический сосуд налили $1100\mbox{дм}^3$ воды. Уровень воды при этом достигает высоты $1100$дм. В жидкость полностью погрузили деталь. При этом уровень жидкости в сосуде поднялся в ${1}\frac{151}{220}$ раза. Чему равен объем детали? Ответ выразите в $\mbox{дм}^3$.
\end{taskBN}

\begin{taskBN}{72}
\addpictoright[0.3\textwidth]{images/505612322603179n0}В цилиндрический сосуд налили $2700\mbox{дм}^3$ воды. Уровень воды при этом достигает высоты $100$дм. В жидкость полностью погрузили деталь. При этом уровень жидкости в сосуде поднялся на $75$дм. Чему равен объем детали? Ответ выразите в $\mbox{дм}^3$.
\end{taskBN}

\begin{taskBN}{73}
\addpictoright[0.3\textwidth]{images/7550062109067n0}В цилиндрический сосуд налили $500\mbox{см}^3$ воды. Уровень воды при этом достигает высоты $5$см. В жидкость полностью погрузили деталь. При этом уровень жидкости в сосуде поднялся на $1$см. Чему равен объем детали? Ответ выразите в $\mbox{см}^3$.
\end{taskBN}

\begin{taskBN}{74}
\addpictoright[0.3\textwidth]{images/81059307444444n0}Объём цилиндра равен $34992\pi$, высота равна $27$. Найдите длину окружности основания цилиндра. Ответ сократите на $\pi$.
\end{taskBN}

\begin{taskBN}{75}
\addpictoright[0.3\textwidth]{images/362399510877589n0}В цилиндрическом сосуде уровень жидкости достигает 48 см. На какой высоте будет находиться уровень жидкости, если её перелить во второй цилиндрический сосуд, диаметр которого в 4 раза меньше диаметра первого? Ответ выразите в сантиметрах.
\end{taskBN}

\begin{taskBN}{76}
\addpictoright[0.3\textwidth]{images/6481148081614418n0}В цилиндрический сосуд налили $600\mbox{дм}^3$ воды. Уровень воды при этом достигает высоты $40$дм. В жидкость полностью погрузили деталь. При этом уровень жидкости в сосуде поднялся на $13$дм. Чему равен объем детали? Ответ выразите в $\mbox{дм}^3$.
\end{taskBN}

\begin{taskBN}{77}
\addpictoright[0.3\textwidth]{images/5727652442041642n0} Первая цилиндрическая кружка в 4,5 раз шире второй, а вторая в 3 раза выше первой. Найдите отношение объёма первой кружки к объёму второй.
\end{taskBN}

\begin{taskBN}{78}
\addpictoright[0.3\textwidth]{images/7222155667549979n0}В цилиндрическом сосуде уровень жидкости достигает 72 см. На какой высоте будет находиться уровень жидкости, если её перелить во второй цилиндрический сосуд, диаметр которого в 6 раз меньше диаметра первого? Ответ выразите в сантиметрах.
\end{taskBN}

\begin{taskBN}{79}
\addpictoright[0.3\textwidth]{images/4294333174121254n0}В цилиндрическом сосуде уровень жидкости достигает 25 см. На какой высоте будет находиться уровень жидкости, если её перелить во второй цилиндрический сосуд, радиус которого в 5 раз меньше радиуса первого? Ответ выразите в сантиметрах.
\end{taskBN}

\begin{taskBN}{80}
\addpictoright[0.3\textwidth]{images/7675194292123877n0}В цилиндрический сосуд налили $2100\mbox{см}^3$ воды. Уровень воды при этом достигает высоты $3$см. В жидкость полностью погрузили деталь. При этом уровень жидкости в сосуде поднялся на $2$см. Чему равен объем детали? Ответ выразите в $\mbox{см}^3$.
\end{taskBN}

\begin{taskBN}{81}
\addpictoright[0.3\textwidth]{images/9930588972198384n0}В цилиндрическом сосуде уровень жидкости достигает 25 см. На какой высоте будет находиться уровень жидкости, если её перелить во второй цилиндрический сосуд, диаметр которого в 5 раз больше диаметра первого? Ответ выразите в сантиметрах.
\end{taskBN}

\begin{taskBN}{82}
\addpictoright[0.3\textwidth]{images/9744626770672657n0} Первая цилиндрическая кружка в 9,5 раз выше второй, а вторая в 10 раз шире первой. Найдите отношение объёма первой кружки к объёму второй.
\end{taskBN}

\begin{taskBN}{83}
\addpictoright[0.3\textwidth]{images/31016642897619n0}В цилиндрическом сосуде уровень жидкости достигает 64 см. На какой высоте будет находиться уровень жидкости, если её перелить во второй цилиндрический сосуд, диаметр которого в 8 раз больше диаметра первого? Ответ выразите в сантиметрах.
\end{taskBN}

\begin{taskBN}{84}
\addpictoright[0.3\textwidth]{images/7917966348342567n0}В цилиндрическом сосуде уровень жидкости достигает 96 см. На какой высоте будет находиться уровень жидкости, если её перелить во второй цилиндрический сосуд, радиус которого в 4 раза больше радиуса первого? Ответ выразите в сантиметрах.
\end{taskBN}

\begin{taskBN}{85}
\addpictoright[0.3\textwidth]{images/151100482752964n0} Первая цилиндрическая кружка в 5 раз шире второй, а вторая в 9,5 раз выше первой. Найдите отношение объёма второй кружки к объёму первой.
\end{taskBN}

\begin{taskBN}{86}
\addpictoright[0.3\textwidth]{images/719776533595682n0} Первая цилиндрическая кружка в 2 раза выше второй, а вторая в 10 раз шире первой. Найдите отношение объёма первой кружки к объёму второй.
\end{taskBN}

\begin{taskBN}{87}
\addpictoright[0.3\textwidth]{images/117647339313737n0}В цилиндрическом сосуде уровень жидкости достигает 50 см. На какой высоте будет находиться уровень жидкости, если её перелить во второй цилиндрический сосуд, радиус которого в 5 раз больше радиуса первого? Ответ выразите в сантиметрах.
\end{taskBN}

\begin{taskBN}{88}
\addpictoright[0.3\textwidth]{images/26183008972795n0} Первая цилиндрическая кружка в 8 раз выше второй, а вторая в 4 раза шире первой. Найдите отношение объёма второй кружки к объёму первой.
\end{taskBN}

\begin{taskBN}{89}
\addpictoright[0.3\textwidth]{images/249359367502004n0}Площадь полной поверхности цилиндра равна $4864\pi$, объём равен $37544\pi$. Найдите площадь боковой поверхности цилиндра. Ответ сократите на $\pi$.
\end{taskBN}

\begin{taskBN}{90}
\addpictoright[0.3\textwidth]{images/142230313396699n0} Первая цилиндрическая кружка в 4,5 раз выше второй, а вторая в 6 раз шире первой. Найдите отношение объёма второй кружки к объёму первой.
\end{taskBN}

\begin{taskBN}{91}
\addpictoright[0.3\textwidth]{images/78931925303772n0}Площадь полной поверхности цилиндра равна $1598\pi$, высота равна $30$. Найдите длину окружности основания цилиндра. Ответ сократите на $\pi$.
\end{taskBN}

\begin{taskBN}{92}
\addpictoright[0.3\textwidth]{images/205486764655902n0} Первая цилиндрическая кружка в 4 раза выше второй, а вторая в 7 раз шире первой. Найдите отношение объёма второй кружки к объёму первой.
\end{taskBN}

\begin{taskBN}{93}
\addpictoright[0.3\textwidth]{images/913567219145964n0}Объём цилиндра равен $96800\pi$, радиус основания равен $44$. Найдите площадь боковой поверхности цилиндра. Ответ сократите на $\pi$.
\end{taskBN}

\begin{taskBN}{94}
\addpictoright[0.3\textwidth]{images/016079664786058n0}В цилиндрический сосуд налили $5000\mbox{дм}^3$ воды. Уровень воды при этом достигает высоты $200$дм. В жидкость полностью погрузили деталь. При этом уровень жидкости в сосуде поднялся на $199$дм. Чему равен объем детали? Ответ выразите в $\mbox{дм}^3$.
\end{taskBN}

\begin{taskBN}{95}
\addpictoright[0.3\textwidth]{images/036771174700658n0}В цилиндрическом сосуде уровень жидкости достигает 98 см. На какой высоте будет находиться уровень жидкости, если её перелить во второй цилиндрический сосуд, радиус которого в 7 раз меньше радиуса первого? Ответ выразите в сантиметрах.
\end{taskBN}

\begin{taskBN}{96}
\addpictoright[0.3\textwidth]{images/4991737754309928n0}Высота цилиндра равна $8$, площадь полной поверхности равна $1426\pi$. Найдите объём цилиндра. Ответ сократите на $\pi$.
\end{taskBN}

\begin{taskBN}{97}
\addpictoright[0.3\textwidth]{images/683792860378174n0}В цилиндрическом сосуде уровень жидкости достигает 8 см. На какой высоте будет находиться уровень жидкости, если её перелить во второй цилиндрический сосуд, радиус которого в 2 раза меньше радиуса первого? Ответ выразите в сантиметрах.
\end{taskBN}

\begin{taskBN}{98}
\addpictoright[0.3\textwidth]{images/215232603498447n0}В цилиндрическом сосуде уровень жидкости достигает 44 см. На какой высоте будет находиться уровень жидкости, если её перелить во второй цилиндрический сосуд, диаметр которого в 2 раза больше диаметра первого? Ответ выразите в сантиметрах.
\end{taskBN}

\begin{taskBN}{99}
\addpictoright[0.3\textwidth]{images/809795485157979n0}В цилиндрический сосуд налили $200\mbox{дм}^3$ воды. Уровень воды при этом достигает высоты $2$дм. В жидкость полностью погрузили деталь. При этом уровень жидкости в сосуде поднялся на $1$дм. Чему равен объем детали? Ответ выразите в $\mbox{дм}^3$.
\end{taskBN}

\begin{taskBN}{100}
\addpictoright[0.3\textwidth]{images/750359299145865n0} Первая цилиндрическая кружка в 5 раз выше второй, а вторая в 6,5 раз шире первой. Найдите отношение объёма второй кружки к объёму первой.
\end{taskBN}

\begin{taskBN}{101}
\addpictoright[0.3\textwidth]{images/878415926790874n0}В цилиндрическом сосуде уровень жидкости достигает 76 см. На какой высоте будет находиться уровень жидкости, если её перелить во второй цилиндрический сосуд, радиус которого в 2 раза меньше радиуса первого? Ответ выразите в сантиметрах.
\end{taskBN}

\begin{taskBN}{102}
\addpictoright[0.3\textwidth]{images/087854312407211n0} Первая цилиндрическая кружка в 5 раз выше второй, а вторая в 10 раз шире первой. Найдите отношение объёма первой кружки к объёму второй.
\end{taskBN}

\begin{taskBN}{103}
\addpictoright[0.3\textwidth]{images/861224092664567n0}В цилиндрический сосуд налили $1800\mbox{см}^3$ воды. Уровень воды при этом достигает высоты $120$см. В жидкость полностью погрузили деталь. При этом уровень жидкости в сосуде поднялся на $65$см. Чему равен объем детали? Ответ выразите в $\mbox{см}^3$.
\end{taskBN}

\begin{taskBN}{104}
\addpictoright[0.3\textwidth]{images/458832661954307n0} Первая цилиндрическая кружка в 10 раз шире второй, а вторая в 2,5 раза выше первой. Найдите отношение объёма первой кружки к объёму второй.
\end{taskBN}

\begin{taskBN}{105}
\addpictoright[0.3\textwidth]{images/446290960812477n0}В цилиндрическом сосуде уровень жидкости достигает 50 см. На какой высоте будет находиться уровень жидкости, если её перелить во второй цилиндрический сосуд, диаметр которого в 5 раз меньше диаметра первого? Ответ выразите в сантиметрах.
\end{taskBN}

\begin{taskBN}{106}
\addpictoright[0.3\textwidth]{images/9015738377676907n0}Площадь полной поверхности цилиндра равна $4320\pi$, объём равен $6075\pi$. Найдите площадь окружности основания цилиндра. Ответ сократите на $\pi$.
\end{taskBN}

\begin{taskBN}{107}
\addpictoright[0.3\textwidth]{images/760950617145208n0}Высота цилиндра равна $38$, площадь полной поверхности равна $4080\pi$. Найдите объём цилиндра. Ответ сократите на $\pi$.
\end{taskBN}

\begin{taskBN}{108}
\addpictoright[0.3\textwidth]{images/2133932467167046n0} Первая цилиндрическая кружка в 10 раз выше второй, а вторая в 3,5 раза шире первой. Найдите отношение объёма второй кружки к объёму первой.
\end{taskBN}

\begin{taskBN}{109}
\addpictoright[0.3\textwidth]{images/1291829686651131n0}В цилиндрическом сосуде уровень жидкости достигает 4 см. На какой высоте будет находиться уровень жидкости, если её перелить во второй цилиндрический сосуд, радиус которого в 2 раза больше радиуса первого? Ответ выразите в сантиметрах.
\end{taskBN}

\begin{taskBN}{110}
\addpictoright[0.3\textwidth]{images/1411738768109112n0}Объём цилиндра равен $72200\pi$, площадь окружности основания равна $1444\pi$. Найдите площадь боковой поверхности цилиндра. Ответ сократите на $\pi$.
\end{taskBN}

\begin{taskBN}{111}
\addpictoright[0.3\textwidth]{images/7667011656304n0}В цилиндрическом сосуде уровень жидкости достигает 56 см. На какой высоте будет находиться уровень жидкости, если её перелить во второй цилиндрический сосуд, радиус которого в 2 раза больше радиуса первого? Ответ выразите в сантиметрах.
\end{taskBN}

\begin{taskBN}{112}
\addpictoright[0.3\textwidth]{images/246777665640004n0} Первая цилиндрическая кружка в 6 раз выше второй, а вторая в 4 раза шире первой. Найдите отношение объёма первой кружки к объёму второй.
\end{taskBN}

\begin{taskBN}{113}
\addpictoright[0.3\textwidth]{images/797951419149246n0}Площадь боковой поверхности цилиндра равна $960\pi$, площадь окружности основания равна $2304\pi$. Найдите объём цилиндра. Ответ сократите на $\pi$.
\end{taskBN}

\begin{taskBN}{114}
\addpictoright[0.3\textwidth]{images/450089751218764n0}В цилиндрический сосуд налили $3700\mbox{см}^3$ воды. Уровень воды при этом достигает высоты $5$см. В жидкость полностью погрузили деталь. При этом уровень жидкости в сосуде поднялся в ${1}\frac{2}{5}$ раза. Чему равен объем детали? Ответ выразите в $\mbox{см}^3$.
\end{taskBN}

\begin{taskBN}{115}
\addpictoright[0.3\textwidth]{images/2647441286564145n0}В цилиндрическом сосуде уровень жидкости достигает 25 см. На какой высоте будет находиться уровень жидкости, если её перелить во второй цилиндрический сосуд, диаметр которого в 5 раз больше диаметра первого? Ответ выразите в сантиметрах.
\end{taskBN}

\begin{taskBN}{116}
\addpictoright[0.3\textwidth]{images/047173441157795n0} Первая цилиндрическая кружка в 2 раза шире второй, а вторая в 5,5 раз выше первой. Найдите отношение объёма второй кружки к объёму первой.
\end{taskBN}

\begin{taskBN}{117}
\addpictoright[0.3\textwidth]{images/4437303868595595n0}В цилиндрическом сосуде уровень жидкости достигает 100 см. На какой высоте будет находиться уровень жидкости, если её перелить во второй цилиндрический сосуд, радиус которого в 10 раз больше радиуса первого? Ответ выразите в сантиметрах.
\end{taskBN}

\begin{taskBN}{118}
\addpictoright[0.3\textwidth]{images/0423077473876925n0}В цилиндрическом сосуде уровень жидкости достигает 27 см. На какой высоте будет находиться уровень жидкости, если её перелить во второй цилиндрический сосуд, диаметр которого в 3 раза меньше диаметра первого? Ответ выразите в сантиметрах.
\end{taskBN}

\begin{taskBN}{119}
\addpictoright[0.3\textwidth]{images/039991034133367576n0}В цилиндрическом сосуде уровень жидкости достигает 76 см. На какой высоте будет находиться уровень жидкости, если её перелить во второй цилиндрический сосуд, радиус которого в 2 раза больше радиуса первого? Ответ выразите в сантиметрах.
\end{taskBN}

\begin{taskBN}{120}
\addpictoright[0.3\textwidth]{images/385097544561675n0}В цилиндрическом сосуде уровень жидкости достигает 75 см. На какой высоте будет находиться уровень жидкости, если её перелить во второй цилиндрический сосуд, радиус которого в 5 раз меньше радиуса первого? Ответ выразите в сантиметрах.
\end{taskBN}

\begin{taskBN}{121}
\addpictoright[0.3\textwidth]{images/675943853312873n0} Первая цилиндрическая кружка в 4 раза выше второй, а вторая в 2,5 раза шире первой. Найдите отношение объёма первой кружки к объёму второй.
\end{taskBN}

\begin{taskBN}{122}
\addpictoright[0.3\textwidth]{images/864099784718965n0}В цилиндрическом сосуде уровень жидкости достигает 99 см. На какой высоте будет находиться уровень жидкости, если её перелить во второй цилиндрический сосуд, радиус которого в 3 раза меньше радиуса первого? Ответ выразите в сантиметрах.
\end{taskBN}

\begin{taskBN}{123}
\addpictoright[0.3\textwidth]{images/106548073171755n0}Площадь боковой поверхности цилиндра равна $2214\pi$, высота равна $27$. Найдите площадь окружности основания цилиндра. Ответ сократите на $\pi$.
\end{taskBN}

\begin{taskBN}{124}
\addpictoright[0.3\textwidth]{images/1102422901872895n0}В цилиндрическом сосуде уровень жидкости достигает 72 см. На какой высоте будет находиться уровень жидкости, если её перелить во второй цилиндрический сосуд, диаметр которого в 6 раз меньше диаметра первого? Ответ выразите в сантиметрах.
\end{taskBN}

\begin{taskBN}{125}
\addpictoright[0.3\textwidth]{images/2043667710259387n0}В цилиндрическом сосуде уровень жидкости достигает 45 см. На какой высоте будет находиться уровень жидкости, если её перелить во второй цилиндрический сосуд, радиус которого в 3 раза больше радиуса первого? Ответ выразите в сантиметрах.
\end{taskBN}

\begin{taskBN}{126}
\addpictoright[0.3\textwidth]{images/239606852438534n0}Площадь боковой поверхности цилиндра равна $792\pi$, площадь полной поверхности равна $1080\pi$. Найдите объём цилиндра. Ответ сократите на $\pi$.
\end{taskBN}

\begin{taskBN}{127}
\addpictoright[0.3\textwidth]{images/0784822980375106n0}Площадь полной поверхности цилиндра равна $2940\pi$, длина окружности основания равна $42\pi$. Найдите площадь боковой поверхности цилиндра. Ответ сократите на $\pi$.
\end{taskBN}

\begin{taskBN}{128}
\addpictoright[0.3\textwidth]{images/254118996506163n0}В цилиндрическом сосуде уровень жидкости достигает 16 см. На какой высоте будет находиться уровень жидкости, если её перелить во второй цилиндрический сосуд, радиус которого в 4 раза больше радиуса первого? Ответ выразите в сантиметрах.
\end{taskBN}

\begin{taskBN}{129}
\addpictoright[0.3\textwidth]{images/11151150918693n0}В цилиндрическом сосуде уровень жидкости достигает 99 см. На какой высоте будет находиться уровень жидкости, если её перелить во второй цилиндрический сосуд, радиус которого в 3 раза больше радиуса первого? Ответ выразите в сантиметрах.
\end{taskBN}

\begin{taskBN}{130}
\addpictoright[0.3\textwidth]{images/3561510915949786n0}Площадь полной поверхности цилиндра равна $3290\pi$, длина окружности основания равна $70\pi$. Найдите площадь боковой поверхности цилиндра. Ответ сократите на $\pi$.
\end{taskBN}

\begin{taskBN}{131}
\addpictoright[0.3\textwidth]{images/479607106230332n0} Первая цилиндрическая кружка в 4,5 раз шире второй, а вторая в 9 раз выше первой. Найдите отношение объёма первой кружки к объёму второй.
\end{taskBN}

\begin{taskBN}{132}
\addpictoright[0.3\textwidth]{images/010130863752746n0}Радиус основания цилиндра равен $48$, высота равна $21$. Найдите площадь боковой поверхности цилиндра. Ответ сократите на $\pi$.
\end{taskBN}

\begin{taskBN}{133}
\addpictoright[0.3\textwidth]{images/8824602249779736n0}Площадь полной поверхности цилиндра равна $4288\pi$, площадь боковой поверхности равна $2240\pi$. Найдите объём цилиндра. Ответ сократите на $\pi$.
\end{taskBN}

\begin{taskBN}{134}
\addpictoright[0.3\textwidth]{images/442962814546466n0}В цилиндрическом сосуде уровень жидкости достигает 50 см. На какой высоте будет находиться уровень жидкости, если её перелить во второй цилиндрический сосуд, радиус которого в 5 раз меньше радиуса первого? Ответ выразите в сантиметрах.
\end{taskBN}

\begin{taskBN}{135}
\addpictoright[0.3\textwidth]{images/980161853077915n0}В цилиндрический сосуд налили $1800\mbox{см}^3$ воды. Уровень воды при этом достигает высоты $360$см. В жидкость полностью погрузили деталь. При этом уровень жидкости в сосуде поднялся на $187$см. Чему равен объем детали? Ответ выразите в $\mbox{см}^3$.
\end{taskBN}

\begin{taskBN}{136}
\addpictoright[0.3\textwidth]{images/396094536906231n0} Первая цилиндрическая кружка в 8 раз выше второй, а вторая в 2 раза шире первой. Найдите отношение объёма первой кружки к объёму второй.
\end{taskBN}

\begin{taskBN}{137}
\addpictoright[0.3\textwidth]{images/641040202916042n0}Высота цилиндра равна $14$, объём равен $3150\pi$. Найдите площадь окружности основания цилиндра. Ответ сократите на $\pi$.
\end{taskBN}

\begin{taskBN}{138}
\addpictoright[0.3\textwidth]{images/413495736316533n0}В цилиндрический сосуд налили $3400\mbox{дм}^3$ воды. Уровень воды при этом достигает высоты $136$дм. В жидкость полностью погрузили деталь. При этом уровень жидкости в сосуде поднялся в ${1}\frac{13}{136}$ раза. Чему равен объем детали? Ответ выразите в $\mbox{дм}^3$.
\end{taskBN}

\begin{taskBN}{139}
\addpictoright[0.3\textwidth]{images/620320161072376n0} Первая цилиндрическая кружка в 6 раз шире второй, а вторая в 4,5 раз выше первой. Найдите отношение объёма второй кружки к объёму первой.
\end{taskBN}

\begin{taskBN}{140}
\addpictoright[0.3\textwidth]{images/629179091536294n0}В цилиндрическом сосуде уровень жидкости достигает 68 см. На какой высоте будет находиться уровень жидкости, если её перелить во второй цилиндрический сосуд, радиус которого в 2 раза больше радиуса первого? Ответ выразите в сантиметрах.
\end{taskBN}

\begin{taskBN}{141}
\addpictoright[0.3\textwidth]{images/804584229450237n0}В цилиндрическом сосуде уровень жидкости достигает 45 см. На какой высоте будет находиться уровень жидкости, если её перелить во второй цилиндрический сосуд, диаметр которого в 3 раза больше диаметра первого? Ответ выразите в сантиметрах.
\end{taskBN}

\begin{taskBN}{142}
\addpictoright[0.3\textwidth]{images/072648378979978n0}В цилиндрическом сосуде уровень жидкости достигает 96 см. На какой высоте будет находиться уровень жидкости, если её перелить во второй цилиндрический сосуд, диаметр которого в 4 раза больше диаметра первого? Ответ выразите в сантиметрах.
\end{taskBN}

\begin{taskBN}{143}
\addpictoright[0.3\textwidth]{images/957044508788041n0}Площадь боковой поверхности цилиндра равна $1120\pi$, длина окружности основания равна $56\pi$. Найдите площадь полной поверхности цилиндра. Ответ сократите на $\pi$.
\end{taskBN}

\begin{taskBN}{144}
\addpictoright[0.3\textwidth]{images/99106870740075n0} Первая цилиндрическая кружка в 5,5 раз выше второй, а вторая в 2 раза шире первой. Найдите отношение объёма первой кружки к объёму второй.
\end{taskBN}

\begin{taskBN}{145}
\addpictoright[0.3\textwidth]{images/727732554260723n0}Объём цилиндра равен $29375\pi$, площадь боковой поверхности равна $2350\pi$. Найдите площадь окружности основания цилиндра. Ответ сократите на $\pi$.
\end{taskBN}

\begin{taskBN}{146}
\addpictoright[0.3\textwidth]{images/979863943322885n0}В цилиндрическом сосуде уровень жидкости достигает 68 см. На какой высоте будет находиться уровень жидкости, если её перелить во второй цилиндрический сосуд, радиус которого в 2 раза больше радиуса первого? Ответ выразите в сантиметрах.
\end{taskBN}

\begin{taskBN}{147}
\addpictoright[0.3\textwidth]{images/26650884457863n0}В цилиндрическом сосуде уровень жидкости достигает 50 см. На какой высоте будет находиться уровень жидкости, если её перелить во второй цилиндрический сосуд, радиус которого в 5 раз меньше радиуса первого? Ответ выразите в сантиметрах.
\end{taskBN}

\begin{taskBN}{148}
\addpictoright[0.3\textwidth]{images/895680198118933n0} Первая цилиндрическая кружка в 5 раз шире второй, а вторая в 4 раза выше первой. Найдите отношение объёма второй кружки к объёму первой.
\end{taskBN}

\begin{taskBN}{149}
\addpictoright[0.3\textwidth]{images/557716050087917n0}В цилиндрическом сосуде уровень жидкости достигает 45 см. На какой высоте будет находиться уровень жидкости, если её перелить во второй цилиндрический сосуд, радиус которого в 3 раза больше радиуса первого? Ответ выразите в сантиметрах.
\end{taskBN}

\begin{taskBN}{150}
\addpictoright[0.3\textwidth]{images/5192652557312576n0}В цилиндрическом сосуде уровень жидкости достигает 12 см. На какой высоте будет находиться уровень жидкости, если её перелить во второй цилиндрический сосуд, радиус которого в 2 раза меньше радиуса первого? Ответ выразите в сантиметрах.
\end{taskBN}

\begin{taskBN}{151}
\addpictoright[0.3\textwidth]{images/161211924159726n0} Первая цилиндрическая кружка в 7,5 раз шире второй, а вторая в 4,5 раз выше первой. Найдите отношение объёма второй кружки к объёму первой.
\end{taskBN}

\begin{taskBN}{152}
\addpictoright[0.3\textwidth]{images/7769825825040166n0}В цилиндрическом сосуде уровень жидкости достигает 99 см. На какой высоте будет находиться уровень жидкости, если её перелить во второй цилиндрический сосуд, диаметр которого в 3 раза меньше диаметра первого? Ответ выразите в сантиметрах.
\end{taskBN}

\begin{taskBN}{153}
\addpictoright[0.3\textwidth]{images/537542114624584n0}В цилиндрическом сосуде уровень жидкости достигает 49 см. На какой высоте будет находиться уровень жидкости, если её перелить во второй цилиндрический сосуд, диаметр которого в 7 раз меньше диаметра первого? Ответ выразите в сантиметрах.
\end{taskBN}

\begin{taskBN}{154}
\addpictoright[0.3\textwidth]{images/6626343930798029n0}В цилиндрический сосуд налили $700\mbox{дм}^3$ воды. Уровень воды при этом достигает высоты $14$дм. В жидкость полностью погрузили деталь. При этом уровень жидкости в сосуде поднялся на $11$дм. Чему равен объем детали? Ответ выразите в $\mbox{дм}^3$.
\end{taskBN}

\begin{taskBN}{155}
\addpictoright[0.3\textwidth]{images/3258351822829035n0}Объём цилиндра равен $336\pi$, площадь окружности основания равна $16\pi$. Найдите площадь боковой поверхности цилиндра. Ответ сократите на $\pi$.
\end{taskBN}

\begin{taskBN}{156}
\addpictoright[0.3\textwidth]{images/92671991519495n0}В цилиндрическом сосуде уровень жидкости достигает 52 см. На какой высоте будет находиться уровень жидкости, если её перелить во второй цилиндрический сосуд, радиус которого в 2 раза больше радиуса первого? Ответ выразите в сантиметрах.
\end{taskBN}

\begin{taskBN}{157}
\addpictoright[0.3\textwidth]{images/942930109706857n0}Площадь полной поверхности цилиндра равна $2480\pi$, объём равен $16800\pi$. Найдите площадь боковой поверхности цилиндра. Ответ сократите на $\pi$.
\end{taskBN}

\begin{taskBN}{158}
\addpictoright[0.3\textwidth]{images/294032806692835n0}В цилиндрическом сосуде уровень жидкости достигает 50 см. На какой высоте будет находиться уровень жидкости, если её перелить во второй цилиндрический сосуд, радиус которого в 5 раз больше радиуса первого? Ответ выразите в сантиметрах.
\end{taskBN}

\begin{taskBN}{159}
\addpictoright[0.3\textwidth]{images/8254066694075566n0}В цилиндрическом сосуде уровень жидкости достигает 54 см. На какой высоте будет находиться уровень жидкости, если её перелить во второй цилиндрический сосуд, диаметр которого в 3 раза меньше диаметра первого? Ответ выразите в сантиметрах.
\end{taskBN}

\begin{taskBN}{160}
\addpictoright[0.3\textwidth]{images/565835403744296n0}В цилиндрическом сосуде уровень жидкости достигает 56 см. На какой высоте будет находиться уровень жидкости, если её перелить во второй цилиндрический сосуд, диаметр которого в 2 раза меньше диаметра первого? Ответ выразите в сантиметрах.
\end{taskBN}

\begin{taskBN}{161}
\addpictoright[0.3\textwidth]{images/03639319813928n0} Первая цилиндрическая кружка в 9 раз шире второй, а вторая в 7,5 раз выше первой. Найдите отношение объёма первой кружки к объёму второй.
\end{taskBN}

\begin{taskBN}{162}
\addpictoright[0.3\textwidth]{images/5578266039495294n0}Площадь окружности основания цилиндра равна $1521\pi$, высота равна $11$. Найдите объём цилиндра. Ответ сократите на $\pi$.
\end{taskBN}

\begin{taskBN}{163}
\addpictoright[0.3\textwidth]{images/371992858037879n0}В цилиндрическом сосуде уровень жидкости достигает 54 см. На какой высоте будет находиться уровень жидкости, если её перелить во второй цилиндрический сосуд, радиус которого в 3 раза больше радиуса первого? Ответ выразите в сантиметрах.
\end{taskBN}

\begin{taskBN}{164}
\addpictoright[0.3\textwidth]{images/0724811660605147n0}Площадь окружности основания цилиндра равна $144\pi$, площадь полной поверхности равна $984\pi$. Найдите площадь боковой поверхности цилиндра. Ответ сократите на $\pi$.
\end{taskBN}

\begin{taskBN}{165}
\addpictoright[0.3\textwidth]{images/497649878770863n0}В цилиндрическом сосуде уровень жидкости достигает 99 см. На какой высоте будет находиться уровень жидкости, если её перелить во второй цилиндрический сосуд, диаметр которого в 3 раза меньше диаметра первого? Ответ выразите в сантиметрах.
\end{taskBN}

\begin{taskBN}{166}
\addpictoright[0.3\textwidth]{images/959935370825933n0}Высота цилиндра равна $11$, площадь боковой поверхности равна $1056\pi$. Найдите объём цилиндра. Ответ сократите на $\pi$.
\end{taskBN}

\begin{taskBN}{167}
\addpictoright[0.3\textwidth]{images/4934250828946052n0} Первая цилиндрическая кружка в 6 раз выше второй, а вторая в 4,5 раз шире первой. Найдите отношение объёма второй кружки к объёму первой.
\end{taskBN}

\begin{taskBN}{168}
\addpictoright[0.3\textwidth]{images/75528885652772n0}Площадь боковой поверхности цилиндра равна $432\pi$, длина окружности основания равна $48\pi$. Найдите площадь полной поверхности цилиндра. Ответ сократите на $\pi$.
\end{taskBN}

\begin{taskBN}{169}
\addpictoright[0.3\textwidth]{images/978616037439936n0}В цилиндрическом сосуде уровень жидкости достигает 72 см. На какой высоте будет находиться уровень жидкости, если её перелить во второй цилиндрический сосуд, диаметр которого в 6 раз меньше диаметра первого? Ответ выразите в сантиметрах.
\end{taskBN}

\begin{taskBN}{170}
\addpictoright[0.3\textwidth]{images/2874922270425135n0}Площадь боковой поверхности цилиндра равна $760\pi$, длина окружности основания равна $40\pi$. Найдите площадь полной поверхности цилиндра. Ответ сократите на $\pi$.
\end{taskBN}

\begin{taskBN}{171}
\addpictoright[0.3\textwidth]{images/7938727674703805n0}В цилиндрический сосуд налили $4700\mbox{дм}^3$ воды. Уровень воды при этом достигает высоты $2$дм. В жидкость полностью погрузили деталь. При этом уровень жидкости в сосуде поднялся на $2$дм. Чему равен объем детали? Ответ выразите в $\mbox{дм}^3$.
\end{taskBN}

\begin{taskBN}{172}
\addpictoright[0.3\textwidth]{images/921511475130971n0} Первая цилиндрическая кружка в 9,5 раз выше второй, а вторая в 5 раз шире первой. Найдите отношение объёма первой кружки к объёму второй.
\end{taskBN}

\begin{taskBN}{173}
\addpictoright[0.3\textwidth]{images/270682001510516n0}В цилиндрический сосуд налили $3200\mbox{см}^3$ воды. Уровень воды при этом достигает высоты $64$см. В жидкость полностью погрузили деталь. При этом уровень жидкости в сосуде поднялся в $\frac{87}{64}$ раза. Чему равен объем детали? Ответ выразите в $\mbox{см}^3$.
\end{taskBN}

\begin{taskBN}{174}
\addpictoright[0.3\textwidth]{images/525181601893306n0}Площадь боковой поверхности цилиндра равна $60\pi$, высота равна $2$. Найдите площадь окружности основания цилиндра. Ответ сократите на $\pi$.
\end{taskBN}

\begin{taskBN}{175}
\addpictoright[0.3\textwidth]{images/818338861368789n0}Радиус основания цилиндра равен $46$, площадь полной поверхности равна $5888\pi$. Найдите площадь боковой поверхности цилиндра. Ответ сократите на $\pi$.
\end{taskBN}

\begin{taskBN}{176}
\addpictoright[0.3\textwidth]{images/79498760710791n0} Первая цилиндрическая кружка в 4 раза выше второй, а вторая в 10 раз шире первой. Найдите отношение объёма первой кружки к объёму второй.
\end{taskBN}

\begin{taskBN}{177}
\addpictoright[0.3\textwidth]{images/670217311822708n0}Высота цилиндра равна $29$, площадь боковой поверхности равна $2030\pi$. Найдите площадь полной поверхности цилиндра. Ответ сократите на $\pi$.
\end{taskBN}

\begin{taskBN}{178}
\addpictoright[0.3\textwidth]{images/50209030264244n0}В цилиндрическом сосуде уровень жидкости достигает 72 см. На какой высоте будет находиться уровень жидкости, если её перелить во второй цилиндрический сосуд, радиус которого в 6 раз меньше радиуса первого? Ответ выразите в сантиметрах.
\end{taskBN}

\begin{taskBN}{179}
\addpictoright[0.3\textwidth]{images/749931673013168n0}Высота цилиндра равна $17$, площадь полной поверхности равна $5368\pi$. Найдите площадь боковой поверхности цилиндра. Ответ сократите на $\pi$.
\end{taskBN}

\begin{taskBN}{180}
\addpictoright[0.3\textwidth]{images/687544112483226n0}Площадь окружности основания цилиндра равна $576\pi$, площадь полной поверхности равна $1392\pi$. Найдите объём цилиндра. Ответ сократите на $\pi$.
\end{taskBN}

\begin{taskBN}{181}
\addpictoright[0.3\textwidth]{images/788304823025224n0} Первая цилиндрическая кружка в 5 раз шире второй, а вторая в 9 раз выше первой. Найдите отношение объёма второй кружки к объёму первой.
\end{taskBN}

\begin{taskBN}{182}
\addpictoright[0.3\textwidth]{images/62132244264237n0}Площадь полной поверхности цилиндра равна $6636\pi$, высота равна $37$. Найдите площадь боковой поверхности цилиндра. Ответ сократите на $\pi$.
\end{taskBN}

\begin{taskBN}{183}
\addpictoright[0.3\textwidth]{images/215083561748618n0}Высота цилиндра равна $44$, площадь полной поверхности равна $2074\pi$. Найдите объём цилиндра. Ответ сократите на $\pi$.
\end{taskBN}

\begin{taskBN}{184}
\addpictoright[0.3\textwidth]{images/798133960603883n0}Высота цилиндра равна $49$, площадь полной поверхности равна $312\pi$. Найдите площадь боковой поверхности цилиндра. Ответ сократите на $\pi$.
\end{taskBN}

\begin{taskBN}{185}
\addpictoright[0.3\textwidth]{images/227356011652512n0}В цилиндрическом сосуде уровень жидкости достигает 32 см. На какой высоте будет находиться уровень жидкости, если её перелить во второй цилиндрический сосуд, диаметр которого в 4 раза больше диаметра первого? Ответ выразите в сантиметрах.
\end{taskBN}

\begin{taskBN}{186}
\addpictoright[0.3\textwidth]{images/275693835791491n0} Первая цилиндрическая кружка в 4,5 раз шире второй, а вторая в 5 раз выше первой. Найдите отношение объёма первой кружки к объёму второй.
\end{taskBN}

\begin{taskBN}{187}
\addpictoright[0.3\textwidth]{images/3401661154520792n0}В цилиндрическом сосуде уровень жидкости достигает 48 см. На какой высоте будет находиться уровень жидкости, если её перелить во второй цилиндрический сосуд, радиус которого в 4 раза меньше радиуса первого? Ответ выразите в сантиметрах.
\end{taskBN}

\begin{taskBN}{188}
\addpictoright[0.3\textwidth]{images/8500808991682065n0}Площадь полной поверхности цилиндра равна $44\pi$, высота равна $9$. Найдите длину окружности основания цилиндра. Ответ сократите на $\pi$.
\end{taskBN}

\begin{taskBN}{189}
\addpictoright[0.3\textwidth]{images/874570530057232n0}В цилиндрическом сосуде уровень жидкости достигает 68 см. На какой высоте будет находиться уровень жидкости, если её перелить во второй цилиндрический сосуд, радиус которого в 2 раза меньше радиуса первого? Ответ выразите в сантиметрах.
\end{taskBN}

\begin{taskBN}{190}
\addpictoright[0.3\textwidth]{images/5684997560848357n0}В цилиндрическом сосуде уровень жидкости достигает 50 см. На какой высоте будет находиться уровень жидкости, если её перелить во второй цилиндрический сосуд, радиус которого в 5 раз меньше радиуса первого? Ответ выразите в сантиметрах.
\end{taskBN}

\begin{taskBN}{191}
\addpictoright[0.3\textwidth]{images/381604030263221n0}В цилиндрическом сосуде уровень жидкости достигает 96 см. На какой высоте будет находиться уровень жидкости, если её перелить во второй цилиндрический сосуд, радиус которого в 4 раза больше радиуса первого? Ответ выразите в сантиметрах.
\end{taskBN}

\begin{taskBN}{192}
\addpictoright[0.3\textwidth]{images/693200301192177n0}Площадь боковой поверхности цилиндра равна $1050\pi$, высота равна $15$. Найдите площадь полной поверхности цилиндра. Ответ сократите на $\pi$.
\end{taskBN}

\begin{taskBN}{193}
\addpictoright[0.3\textwidth]{images/2234129906806397n0}Площадь полной поверхности цилиндра равна $3996\pi$, высота равна $47$. Найдите площадь боковой поверхности цилиндра. Ответ сократите на $\pi$.
\end{taskBN}

\begin{taskBN}{194}
\addpictoright[0.3\textwidth]{images/6720189464488224n0}В цилиндрическом сосуде уровень жидкости достигает 16 см. На какой высоте будет находиться уровень жидкости, если её перелить во второй цилиндрический сосуд, радиус которого в 4 раза меньше радиуса первого? Ответ выразите в сантиметрах.
\end{taskBN}

\begin{taskBN}{195}
\addpictoright[0.3\textwidth]{images/991919182672759n0}В цилиндрическом сосуде уровень жидкости достигает 27 см. На какой высоте будет находиться уровень жидкости, если её перелить во второй цилиндрический сосуд, радиус которого в 3 раза больше радиуса первого? Ответ выразите в сантиметрах.
\end{taskBN}

\begin{taskBN}{196}
\addpictoright[0.3\textwidth]{images/46339479836324915n0}Высота цилиндра равна $22$, площадь боковой поверхности равна $484\pi$. Найдите площадь полной поверхности цилиндра. Ответ сократите на $\pi$.
\end{taskBN}

\begin{taskBN}{197}
\addpictoright[0.3\textwidth]{images/3314060462947899n0}В цилиндрический сосуд налили $1600\mbox{дм}^3$ воды. Уровень воды при этом достигает высоты $8$дм. В жидкость полностью погрузили деталь. При этом уровень жидкости в сосуде поднялся в ${1}\frac{3}{4}$ раза. Чему равен объем детали? Ответ выразите в $\mbox{дм}^3$.
\end{taskBN}

\begin{taskBN}{198}
\addpictoright[0.3\textwidth]{images/0129295792616988n0}В цилиндрическом сосуде уровень жидкости достигает 60 см. На какой высоте будет находиться уровень жидкости, если её перелить во второй цилиндрический сосуд, радиус которого в 2 раза меньше радиуса первого? Ответ выразите в сантиметрах.
\end{taskBN}

\begin{taskBN}{199}
\addpictoright[0.3\textwidth]{images/175415023860405n0}Высота цилиндра равна $45$, площадь боковой поверхности равна $1620\pi$. Найдите объём цилиндра. Ответ сократите на $\pi$.
\end{taskBN}

\begin{taskBN}{200}
\addpictoright[0.3\textwidth]{images/4509306837736806n0}В цилиндрическом сосуде уровень жидкости достигает 32 см. На какой высоте будет находиться уровень жидкости, если её перелить во второй цилиндрический сосуд, диаметр которого в 4 раза меньше диаметра первого? Ответ выразите в сантиметрах.
\end{taskBN}

\begin{taskBN}{201}
\addpictoright[0.3\textwidth]{images/556068610596369n0} Первая цилиндрическая кружка в 4 раза шире второй, а вторая в 2,5 раза выше первой. Найдите отношение объёма первой кружки к объёму второй.
\end{taskBN}

\begin{taskBN}{202}
\addpictoright[0.3\textwidth]{images/284039435647622n0}В цилиндрическом сосуде уровень жидкости достигает 9 см. На какой высоте будет находиться уровень жидкости, если её перелить во второй цилиндрический сосуд, радиус которого в 3 раза меньше радиуса первого? Ответ выразите в сантиметрах.
\end{taskBN}

\begin{taskBN}{203}
\addpictoright[0.3\textwidth]{images/829124719289607n0} Первая цилиндрическая кружка в 7,5 раз выше второй, а вторая в 6 раз шире первой. Найдите отношение объёма второй кружки к объёму первой.
\end{taskBN}

\begin{taskBN}{204}
\addpictoright[0.3\textwidth]{images/543399512525398n0}Площадь боковой поверхности цилиндра равна $2080\pi$, высота равна $26$. Найдите объём цилиндра. Ответ сократите на $\pi$.
\end{taskBN}

\begin{taskBN}{205}
\addpictoright[0.3\textwidth]{images/999692674049138n0}В цилиндрическом сосуде уровень жидкости достигает 24 см. На какой высоте будет находиться уровень жидкости, если её перелить во второй цилиндрический сосуд, диаметр которого в 2 раза больше диаметра первого? Ответ выразите в сантиметрах.
\end{taskBN}

\begin{taskBN}{206}
\addpictoright[0.3\textwidth]{images/7103131557981746n0}Объём цилиндра равен $14872\pi$, площадь боковой поверхности равна $1144\pi$. Найдите площадь полной поверхности цилиндра. Ответ сократите на $\pi$.
\end{taskBN}

\begin{taskBN}{207}
\addpictoright[0.3\textwidth]{images/6548538863613587n0}В цилиндрическом сосуде уровень жидкости достигает 40 см. На какой высоте будет находиться уровень жидкости, если её перелить во второй цилиндрический сосуд, радиус которого в 2 раза меньше радиуса первого? Ответ выразите в сантиметрах.
\end{taskBN}

\begin{taskBN}{208}
\addpictoright[0.3\textwidth]{images/995389259790185n0} Первая цилиндрическая кружка в 3,5 раза шире второй, а вторая в 2,5 раза выше первой. Найдите отношение объёма первой кружки к объёму второй.
\end{taskBN}

\begin{taskBN}{209}
\addpictoright[0.3\textwidth]{images/14033065317885285n0}Высота цилиндра равна $20$, длина окружности основания равна $88\pi$. Найдите площадь боковой поверхности цилиндра. Ответ сократите на $\pi$.
\end{taskBN}

\begin{taskBN}{210}
\addpictoright[0.3\textwidth]{images/320046553795502n0} Первая цилиндрическая кружка в 2 раза шире второй, а вторая в 3 раза выше первой. Найдите отношение объёма второй кружки к объёму первой.
\end{taskBN}

\begin{taskBN}{211}
\addpictoright[0.3\textwidth]{images/2134724391537126n0} Первая цилиндрическая кружка в 7,5 раз шире второй, а вторая в 5 раз выше первой. Найдите отношение объёма первой кружки к объёму второй.
\end{taskBN}

\begin{taskBN}{212}
\addpictoright[0.3\textwidth]{images/78893024369795n0}В цилиндрический сосуд налили $3900\mbox{см}^3$ воды. Уровень воды при этом достигает высоты $150$см. В жидкость полностью погрузили деталь. При этом уровень жидкости в сосуде поднялся на $118$см. Чему равен объем детали? Ответ выразите в $\mbox{см}^3$.
\end{taskBN}

\begin{taskBN}{213}
\addpictoright[0.3\textwidth]{images/650627500176382n0}Высота цилиндра равна $31$, площадь полной поверхности равна $3132\pi$. Найдите длину окружности основания цилиндра. Ответ сократите на $\pi$.
\end{taskBN}

\begin{taskBN}{214}
\addpictoright[0.3\textwidth]{images/952037784531085n0}Длина окружности основания цилиндра равна $22\pi$, высота равна $27$. Найдите площадь полной поверхности цилиндра. Ответ сократите на $\pi$.
\end{taskBN}

\begin{taskBN}{215}
\addpictoright[0.3\textwidth]{images/614732963450331n0}В цилиндрический сосуд налили $2500\mbox{дм}^3$ воды. Уровень воды при этом достигает высоты $20$дм. В жидкость полностью погрузили деталь. При этом уровень жидкости в сосуде поднялся в $\frac{7}{4}$ раза. Чему равен объем детали? Ответ выразите в $\mbox{дм}^3$.
\end{taskBN}

\begin{taskBN}{216}
\addpictoright[0.3\textwidth]{images/161946099061439n0}В цилиндрическом сосуде уровень жидкости достигает 72 см. На какой высоте будет находиться уровень жидкости, если её перелить во второй цилиндрический сосуд, радиус которого в 6 раз меньше радиуса первого? Ответ выразите в сантиметрах.
\end{taskBN}

\begin{taskBN}{217}
\addpictoright[0.3\textwidth]{images/005134843131009n0} Первая цилиндрическая кружка в 5 раз шире второй, а вторая в 3,5 раза выше первой. Найдите отношение объёма второй кружки к объёму первой.
\end{taskBN}

\begin{taskBN}{218}
\addpictoright[0.3\textwidth]{images/71808128621871n0}Объём цилиндра равен $84672\pi$, площадь полной поверхности равна $7560\pi$. Найдите площадь окружности основания цилиндра. Ответ сократите на $\pi$.
\end{taskBN}

\begin{taskBN}{219}
\addpictoright[0.3\textwidth]{images/326997260769977n0}В цилиндрический сосуд налили $1500\mbox{см}^3$ воды. Уровень воды при этом достигает высоты $375$см. В жидкость полностью погрузили деталь. При этом уровень жидкости в сосуде поднялся на $166$см. Чему равен объем детали? Ответ выразите в $\mbox{см}^3$.
\end{taskBN}

\begin{taskBN}{220}
\addpictoright[0.3\textwidth]{images/4905317806300897n0}В цилиндрический сосуд налили $2500\mbox{см}^3$ воды. Уровень воды при этом достигает высоты $100$см. В жидкость полностью погрузили деталь. При этом уровень жидкости в сосуде поднялся на $85$см. Чему равен объем детали? Ответ выразите в $\mbox{см}^3$.
\end{taskBN}

\begin{taskBN}{221}
\addpictoright[0.3\textwidth]{images/27926794601132454n0}Площадь окружности основания цилиндра равна $1369\pi$, высота равна $22$. Найдите площадь боковой поверхности цилиндра. Ответ сократите на $\pi$.
\end{taskBN}

\begin{taskBN}{222}
\addpictoright[0.3\textwidth]{images/640670353555756n0}Диаметр основания цилиндра равен $72$, площадь полной поверхности равна $5040\pi$. Найдите объём цилиндра. Ответ сократите на $\pi$.
\end{taskBN}

\begin{taskBN}{223}
\addpictoright[0.3\textwidth]{images/837989334190173n0} Первая цилиндрическая кружка в 5 раз выше второй, а вторая в 4 раза шире первой. Найдите отношение объёма второй кружки к объёму первой.
\end{taskBN}

\begin{taskBN}{224}
\addpictoright[0.3\textwidth]{images/006547359514164n0}Высота цилиндра равна $10$, площадь полной поверхности равна $918\pi$. Найдите площадь боковой поверхности цилиндра. Ответ сократите на $\pi$.
\end{taskBN}

\begin{taskBN}{225}
\addpictoright[0.3\textwidth]{images/643746821101412n0}Площадь боковой поверхности цилиндра равна $256\pi$, площадь полной поверхности равна $768\pi$. Найдите площадь окружности основания цилиндра. Ответ сократите на $\pi$.
\end{taskBN}

\begin{taskBN}{226}
\addpictoright[0.3\textwidth]{images/1900495153017343n0}В цилиндрическом сосуде уровень жидкости достигает 81 см. На какой высоте будет находиться уровень жидкости, если её перелить во второй цилиндрический сосуд, радиус которого в 9 раз больше радиуса первого? Ответ выразите в сантиметрах.
\end{taskBN}

\begin{taskBN}{227}
\addpictoright[0.3\textwidth]{images/908544571480115n0}Объём цилиндра равен $7840\pi$, площадь окружности основания равна $784\pi$. Найдите площадь полной поверхности цилиндра. Ответ сократите на $\pi$.
\end{taskBN}

\begin{taskBN}{228}
\addpictoright[0.3\textwidth]{images/10032459078453n0}В цилиндрический сосуд налили $3200\mbox{см}^3$ воды. Уровень воды при этом достигает высоты $200$см. В жидкость полностью погрузили деталь. При этом уровень жидкости в сосуде поднялся на $3$см. Чему равен объем детали? Ответ выразите в $\mbox{см}^3$.
\end{taskBN}

\begin{taskBN}{229}
\addpictoright[0.3\textwidth]{images/1546974288468059n0}В цилиндрический сосуд налили $4800\mbox{дм}^3$ воды. Уровень воды при этом достигает высоты $15$дм. В жидкость полностью погрузили деталь. При этом уровень жидкости в сосуде поднялся в ${1}\frac{1}{3}$ раза. Чему равен объем детали? Ответ выразите в $\mbox{дм}^3$.
\end{taskBN}

\begin{taskBN}{230}
\addpictoright[0.3\textwidth]{images/982199697812884n0} Первая цилиндрическая кружка в 5,5 раз выше второй, а вторая в 5 раз шире первой. Найдите отношение объёма первой кружки к объёму второй.
\end{taskBN}

\begin{taskBN}{231}
\addpictoright[0.3\textwidth]{images/253485562491609n0} Первая цилиндрическая кружка в 7 раз шире второй, а вторая в 4 раза выше первой. Найдите отношение объёма первой кружки к объёму второй.
\end{taskBN}

\begin{taskBN}{232}
\addpictoright[0.3\textwidth]{images/377148051017366n0}В цилиндрическом сосуде уровень жидкости достигает 8 см. На какой высоте будет находиться уровень жидкости, если её перелить во второй цилиндрический сосуд, радиус которого в 2 раза больше радиуса первого? Ответ выразите в сантиметрах.
\end{taskBN}

\begin{taskBN}{233}
\addpictoright[0.3\textwidth]{images/0230271877329433n0}В цилиндрическом сосуде уровень жидкости достигает 80 см. На какой высоте будет находиться уровень жидкости, если её перелить во второй цилиндрический сосуд, радиус которого в 4 раза больше радиуса первого? Ответ выразите в сантиметрах.
\end{taskBN}

\begin{taskBN}{234}
\addpictoright[0.3\textwidth]{images/2794185812423495n0}В цилиндрическом сосуде уровень жидкости достигает 88 см. На какой высоте будет находиться уровень жидкости, если её перелить во второй цилиндрический сосуд, диаметр которого в 2 раза больше диаметра первого? Ответ выразите в сантиметрах.
\end{taskBN}

\begin{taskBN}{235}
\addpictoright[0.3\textwidth]{images/538363639342026n0} Первая цилиндрическая кружка в 2,5 раза шире второй, а вторая в 4 раза выше первой. Найдите отношение объёма второй кружки к объёму первой.
\end{taskBN}

\begin{taskBN}{236}
\addpictoright[0.3\textwidth]{images/9690033102310767n0}Объём цилиндра равен $475\pi$, высота равна $19$. Найдите площадь боковой поверхности цилиндра. Ответ сократите на $\pi$.
\end{taskBN}

\begin{taskBN}{237}
\addpictoright[0.3\textwidth]{images/401886486427063n0}Площадь боковой поверхности цилиндра равна $1254\pi$, площадь полной поверхности равна $3432\pi$. Найдите объём цилиндра. Ответ сократите на $\pi$.
\end{taskBN}

\begin{taskBN}{238}
\addpictoright[0.3\textwidth]{images/3276769717701975n0}Площадь боковой поверхности цилиндра равна $1410\pi$, площадь полной поверхности равна $5828\pi$. Найдите длину окружности основания цилиндра. Ответ сократите на $\pi$.
\end{taskBN}

\begin{taskBN}{239}
\addpictoright[0.3\textwidth]{images/650105191652116n0}Высота цилиндра равна $22$, площадь боковой поверхности равна $1496\pi$. Найдите площадь полной поверхности цилиндра. Ответ сократите на $\pi$.
\end{taskBN}

\begin{taskBN}{240}
\addpictoright[0.3\textwidth]{images/7060241806687493n0}В цилиндрический сосуд налили $1800\mbox{дм}^3$ воды. Уровень воды при этом достигает высоты $3$дм. В жидкость полностью погрузили деталь. При этом уровень жидкости в сосуде поднялся в ${1}\frac{1}{3}$ раза. Чему равен объем детали? Ответ выразите в $\mbox{дм}^3$.
\end{taskBN}

\begin{taskBN}{241}
\addpictoright[0.3\textwidth]{images/7272988860660474n0}Объём цилиндра равен $1872\pi$, площадь полной поверхности равна $600\pi$. Найдите площадь боковой поверхности цилиндра. Ответ сократите на $\pi$.
\end{taskBN}

\begin{taskBN}{242}
\addpictoright[0.3\textwidth]{images/77666251807637n0}Высота цилиндра равна $27$, площадь полной поверхности равна $4148\pi$. Найдите площадь боковой поверхности цилиндра. Ответ сократите на $\pi$.
\end{taskBN}

\begin{taskBN}{243}
\addpictoright[0.3\textwidth]{images/7036447784364885n0}В цилиндрический сосуд налили $1700\mbox{дм}^3$ воды. Уровень воды при этом достигает высоты $50$дм. В жидкость полностью погрузили деталь. При этом уровень жидкости в сосуде поднялся в $\frac{49}{25}$ раза. Чему равен объем детали? Ответ выразите в $\mbox{дм}^3$.
\end{taskBN}

\begin{taskBN}{244}
\addpictoright[0.3\textwidth]{images/5784446727599704n0}В цилиндрическом сосуде уровень жидкости достигает 25 см. На какой высоте будет находиться уровень жидкости, если её перелить во второй цилиндрический сосуд, диаметр которого в 5 раз меньше диаметра первого? Ответ выразите в сантиметрах.
\end{taskBN}

\begin{taskBN}{245}
\addpictoright[0.3\textwidth]{images/760553508637213n0}В цилиндрическом сосуде уровень жидкости достигает 25 см. На какой высоте будет находиться уровень жидкости, если её перелить во второй цилиндрический сосуд, диаметр которого в 5 раз больше диаметра первого? Ответ выразите в сантиметрах.
\end{taskBN}

\begin{taskBN}{246}
\addpictoright[0.3\textwidth]{images/680896602407643n0}В цилиндрический сосуд налили $4100\mbox{дм}^3$ воды. Уровень воды при этом достигает высоты $410$дм. В жидкость полностью погрузили деталь. При этом уровень жидкости в сосуде поднялся на $278$дм. Чему равен объем детали? Ответ выразите в $\mbox{дм}^3$.
\end{taskBN}

\begin{taskBN}{247}
\addpictoright[0.3\textwidth]{images/600114010181425n0}В цилиндрическом сосуде уровень жидкости достигает 56 см. На какой высоте будет находиться уровень жидкости, если её перелить во второй цилиндрический сосуд, диаметр которого в 2 раза больше диаметра первого? Ответ выразите в сантиметрах.
\end{taskBN}

\begin{taskBN}{248}
\addpictoright[0.3\textwidth]{images/3124244804054108n0}Объём цилиндра равен $1116\pi$, площадь полной поверхности равна $444\pi$. Найдите площадь окружности основания цилиндра. Ответ сократите на $\pi$.
\end{taskBN}

\begin{taskBN}{249}
\addpictoright[0.3\textwidth]{images/6096962946557807n0}Высота цилиндра равна $16$, площадь полной поверхности равна $5922\pi$. Найдите площадь боковой поверхности цилиндра. Ответ сократите на $\pi$.
\end{taskBN}

\begin{taskBN}{250}
\addpictoright[0.3\textwidth]{images/82536619275473n0} Первая цилиндрическая кружка в 8 раз выше второй, а вторая в 5 раз шире первой. Найдите отношение объёма второй кружки к объёму первой.
\end{taskBN}

\begin{taskBN}{251}
\addpictoright[0.3\textwidth]{images/568127675548842n0} Первая цилиндрическая кружка в 10 раз выше второй, а вторая в 6 раз шире первой. Найдите отношение объёма второй кружки к объёму первой.
\end{taskBN}

\begin{taskBN}{252}
\addpictoright[0.3\textwidth]{images/514521848607338n0}Объём цилиндра равен $22500\pi$, площадь боковой поверхности равна $1800\pi$. Найдите площадь полной поверхности цилиндра. Ответ сократите на $\pi$.
\end{taskBN}

\begin{taskBN}{253}
\addpictoright[0.3\textwidth]{images/206298166777453n0}Площадь боковой поверхности цилиндра равна $410\pi$, объём равен $1025\pi$. Найдите площадь полной поверхности цилиндра. Ответ сократите на $\pi$.
\end{taskBN}

\begin{taskBN}{254}
\addpictoright[0.3\textwidth]{images/017962334741851n0}В цилиндрический сосуд налили $3000\mbox{м}^3$ воды. Уровень воды при этом достигает высоты $6$м. В жидкость полностью погрузили деталь. При этом уровень жидкости в сосуде поднялся в $ 1{,}5 $ раза. Чему равен объем детали? Ответ выразите в $\mbox{м}^3$.
\end{taskBN}

\begin{taskBN}{255}
\addpictoright[0.3\textwidth]{images/781557221235533n0}В цилиндрическом сосуде уровень жидкости достигает 75 см. На какой высоте будет находиться уровень жидкости, если её перелить во второй цилиндрический сосуд, диаметр которого в 5 раз меньше диаметра первого? Ответ выразите в сантиметрах.
\end{taskBN}

\begin{taskBN}{256}
\addpictoright[0.3\textwidth]{images/74155540961687n0} Первая цилиндрическая кружка в 5 раз выше второй, а вторая в 6,5 раз шире первой. Найдите отношение объёма второй кружки к объёму первой.
\end{taskBN}

\begin{taskBN}{257}
\addpictoright[0.3\textwidth]{images/41497180492156n0}Площадь полной поверхности цилиндра равна $3796\pi$, объём равен $31772\pi$. Найдите площадь боковой поверхности цилиндра. Ответ сократите на $\pi$.
\end{taskBN}

\begin{taskBN}{258}
\addpictoright[0.3\textwidth]{images/283318585834197n0}В цилиндрический сосуд налили $2200\mbox{см}^3$ воды. Уровень воды при этом достигает высоты $20$см. В жидкость полностью погрузили деталь. При этом уровень жидкости в сосуде поднялся на $10$см. Чему равен объем детали? Ответ выразите в $\mbox{см}^3$.
\end{taskBN}

\begin{taskBN}{259}
\addpictoright[0.3\textwidth]{images/4282789361564099n0}В цилиндрический сосуд налили $300\mbox{см}^3$ воды. Уровень воды при этом достигает высоты $10$см. В жидкость полностью погрузили деталь. При этом уровень жидкости в сосуде поднялся в $\frac{19}{10}$ раза. Чему равен объем детали? Ответ выразите в $\mbox{см}^3$.
\end{taskBN}

\begin{taskBN}{260}
\addpictoright[0.3\textwidth]{images/458250449814751n0}В цилиндрическом сосуде уровень жидкости достигает 96 см. На какой высоте будет находиться уровень жидкости, если её перелить во второй цилиндрический сосуд, радиус которого в 4 раза меньше радиуса первого? Ответ выразите в сантиметрах.
\end{taskBN}

\begin{taskBN}{261}
\addpictoright[0.3\textwidth]{images/465578111699542n0} Первая цилиндрическая кружка в 4,5 раз выше второй, а вторая в 6 раз шире первой. Найдите отношение объёма первой кружки к объёму второй.
\end{taskBN}

\begin{taskBN}{262}
\addpictoright[0.3\textwidth]{images/071761822634273n0}Высота цилиндра равна $27$, площадь боковой поверхности равна $2376\pi$. Найдите площадь полной поверхности цилиндра. Ответ сократите на $\pi$.
\end{taskBN}

\begin{taskBN}{263}
\addpictoright[0.3\textwidth]{images/195982790205725n0}Площадь боковой поверхности цилиндра равна $98\pi$, высота равна $7$. Найдите площадь полной поверхности цилиндра. Ответ сократите на $\pi$.
\end{taskBN}

\begin{taskBN}{264}
\addpictoright[0.3\textwidth]{images/594504739858184n0}В цилиндрическом сосуде уровень жидкости достигает 90 см. На какой высоте будет находиться уровень жидкости, если её перелить во второй цилиндрический сосуд, диаметр которого в 3 раза больше диаметра первого? Ответ выразите в сантиметрах.
\end{taskBN}

\begin{taskBN}{265}
\addpictoright[0.3\textwidth]{images/945332089700053n0} Первая цилиндрическая кружка в 6 раз шире второй, а вторая в 7,5 раз выше первой. Найдите отношение объёма первой кружки к объёму второй.
\end{taskBN}

\begin{taskBN}{266}
\addpictoright[0.3\textwidth]{images/917399531612712n0}В цилиндрическом сосуде уровень жидкости достигает 48 см. На какой высоте будет находиться уровень жидкости, если её перелить во второй цилиндрический сосуд, радиус которого в 4 раза больше радиуса первого? Ответ выразите в сантиметрах.
\end{taskBN}

\begin{taskBN}{267}
\addpictoright[0.3\textwidth]{images/160037163913654n0}В цилиндрическом сосуде уровень жидкости достигает 36 см. На какой высоте будет находиться уровень жидкости, если её перелить во второй цилиндрический сосуд, радиус которого в 6 раз больше радиуса первого? Ответ выразите в сантиметрах.
\end{taskBN}

\begin{taskBN}{268}
\addpictoright[0.3\textwidth]{images/028446023853292n0}В цилиндрическом сосуде уровень жидкости достигает 96 см. На какой высоте будет находиться уровень жидкости, если её перелить во второй цилиндрический сосуд, радиус которого в 4 раза меньше радиуса первого? Ответ выразите в сантиметрах.
\end{taskBN}

\begin{taskBN}{269}
\addpictoright[0.3\textwidth]{images/3828712832684376n0}В цилиндрическом сосуде уровень жидкости достигает 40 см. На какой высоте будет находиться уровень жидкости, если её перелить во второй цилиндрический сосуд, радиус которого в 2 раза меньше радиуса первого? Ответ выразите в сантиметрах.
\end{taskBN}

\begin{taskBN}{270}
\addpictoright[0.3\textwidth]{images/890610264193592n0}В цилиндрическом сосуде уровень жидкости достигает 63 см. На какой высоте будет находиться уровень жидкости, если её перелить во второй цилиндрический сосуд, радиус которого в 3 раза меньше радиуса первого? Ответ выразите в сантиметрах.
\end{taskBN}

\begin{taskBN}{271}
\addpictoright[0.3\textwidth]{images/016558662049692696n0}В цилиндрическом сосуде уровень жидкости достигает 32 см. На какой высоте будет находиться уровень жидкости, если её перелить во второй цилиндрический сосуд, радиус которого в 4 раза меньше радиуса первого? Ответ выразите в сантиметрах.
\end{taskBN}

\begin{taskBN}{272}
\addpictoright[0.3\textwidth]{images/006039988112379n0}В цилиндрическом сосуде уровень жидкости достигает 16 см. На какой высоте будет находиться уровень жидкости, если её перелить во второй цилиндрический сосуд, диаметр которого в 4 раза больше диаметра первого? Ответ выразите в сантиметрах.
\end{taskBN}

\begin{taskBN}{273}
\addpictoright[0.3\textwidth]{images/236341015535522n0}В цилиндрическом сосуде уровень жидкости достигает 32 см. На какой высоте будет находиться уровень жидкости, если её перелить во второй цилиндрический сосуд, диаметр которого в 4 раза больше диаметра первого? Ответ выразите в сантиметрах.
\end{taskBN}

\begin{taskBN}{274}
\addpictoright[0.3\textwidth]{images/83919302674219n0}В цилиндрический сосуд налили $2500\mbox{дм}^3$ воды. Уровень воды при этом достигает высоты $2$дм. В жидкость полностью погрузили деталь. При этом уровень жидкости в сосуде поднялся в $ 1{,}5 $ раза. Чему равен объем детали? Ответ выразите в $\mbox{дм}^3$.
\end{taskBN}

\begin{taskBN}{275}
\addpictoright[0.3\textwidth]{images/368950770967803n0}В цилиндрическом сосуде уровень жидкости достигает 75 см. На какой высоте будет находиться уровень жидкости, если её перелить во второй цилиндрический сосуд, радиус которого в 5 раз больше радиуса первого? Ответ выразите в сантиметрах.
\end{taskBN}

\begin{taskBN}{276}
\addpictoright[0.3\textwidth]{images/61813481779984n0} Первая цилиндрическая кружка в 3 раза выше второй, а вторая в 7,5 раз шире первой. Найдите отношение объёма второй кружки к объёму первой.
\end{taskBN}

\begin{taskBN}{277}
\addpictoright[0.3\textwidth]{images/241344957882427n0}В цилиндрическом сосуде уровень жидкости достигает 63 см. На какой высоте будет находиться уровень жидкости, если её перелить во второй цилиндрический сосуд, диаметр которого в 3 раза меньше диаметра первого? Ответ выразите в сантиметрах.
\end{taskBN}

\begin{taskBN}{278}
\addpictoright[0.3\textwidth]{images/101053047244612n0}В цилиндрическом сосуде уровень жидкости достигает 75 см. На какой высоте будет находиться уровень жидкости, если её перелить во второй цилиндрический сосуд, радиус которого в 5 раз больше радиуса первого? Ответ выразите в сантиметрах.
\end{taskBN}

\begin{taskBN}{279}
\addpictoright[0.3\textwidth]{images/323448904612304n0}В цилиндрическом сосуде уровень жидкости достигает 99 см. На какой высоте будет находиться уровень жидкости, если её перелить во второй цилиндрический сосуд, диаметр которого в 3 раза меньше диаметра первого? Ответ выразите в сантиметрах.
\end{taskBN}

\begin{taskBN}{280}
\addpictoright[0.3\textwidth]{images/37577546975395726n0} Первая цилиндрическая кружка в 3 раза шире второй, а вторая в 7,5 раз выше первой. Найдите отношение объёма первой кружки к объёму второй.
\end{taskBN}

\begin{taskBN}{281}
\addpictoright[0.3\textwidth]{images/791030922566295n0}Объём цилиндра равен $1250\pi$, длина окружности основания равна $50\pi$. Найдите площадь боковой поверхности цилиндра. Ответ сократите на $\pi$.
\end{taskBN}

\begin{taskBN}{282}
\addpictoright[0.3\textwidth]{images/638055919615933n0}В цилиндрическом сосуде уровень жидкости достигает 36 см. На какой высоте будет находиться уровень жидкости, если её перелить во второй цилиндрический сосуд, диаметр которого в 6 раз больше диаметра первого? Ответ выразите в сантиметрах.
\end{taskBN}

\begin{taskBN}{283}
\addpictoright[0.3\textwidth]{images/838687166086141n0}Площадь боковой поверхности цилиндра равна $1274\pi$, высота равна $13$. Найдите длину окружности основания цилиндра. Ответ сократите на $\pi$.
\end{taskBN}

\begin{taskBN}{284}
\addpictoright[0.3\textwidth]{images/558790595940128n0}Площадь полной поверхности цилиндра равна $4550\pi$, объём равен $36750\pi$. Найдите площадь боковой поверхности цилиндра. Ответ сократите на $\pi$.
\end{taskBN}

\begin{taskBN}{285}
\addpictoright[0.3\textwidth]{images/4966212169637194n0}Длина окружности основания цилиндра равна $14\pi$, высота равна $37$. Найдите объём цилиндра. Ответ сократите на $\pi$.
\end{taskBN}

\begin{taskBN}{286}
\addpictoright[0.3\textwidth]{images/077708960199415n0} Первая цилиндрическая кружка в 4,5 раз шире второй, а вторая в 7,5 раз выше первой. Найдите отношение объёма первой кружки к объёму второй.
\end{taskBN}

\begin{taskBN}{287}
\addpictoright[0.3\textwidth]{images/81551981157469n0}В цилиндрическом сосуде уровень жидкости достигает 99 см. На какой высоте будет находиться уровень жидкости, если её перелить во второй цилиндрический сосуд, диаметр которого в 3 раза меньше диаметра первого? Ответ выразите в сантиметрах.
\end{taskBN}

\begin{taskBN}{288}
\addpictoright[0.3\textwidth]{images/32197600895389256n0}Высота цилиндра равна $27$, площадь боковой поверхности равна $1242\pi$. Найдите объём цилиндра. Ответ сократите на $\pi$.
\end{taskBN}

\begin{taskBN}{289}
\addpictoright[0.3\textwidth]{images/875898376181771n0}В цилиндрическом сосуде уровень жидкости достигает 80 см. На какой высоте будет находиться уровень жидкости, если её перелить во второй цилиндрический сосуд, радиус которого в 4 раза больше радиуса первого? Ответ выразите в сантиметрах.
\end{taskBN}

\begin{taskBN}{290}
\addpictoright[0.3\textwidth]{images/153078538108026n0} Первая цилиндрическая кружка в 3 раза выше второй, а вторая в 6 раз шире первой. Найдите отношение объёма второй кружки к объёму первой.
\end{taskBN}

\begin{taskBN}{291}
\addpictoright[0.3\textwidth]{images/01870312033710353n0}Площадь боковой поверхности цилиндра равна $240\pi$, объём равен $4800\pi$. Найдите длину окружности основания цилиндра. Ответ сократите на $\pi$.
\end{taskBN}

\begin{taskBN}{292}
\addpictoright[0.3\textwidth]{images/36474174145612n0}В цилиндрическом сосуде уровень жидкости достигает 16 см. На какой высоте будет находиться уровень жидкости, если её перелить во второй цилиндрический сосуд, радиус которого в 4 раза меньше радиуса первого? Ответ выразите в сантиметрах.
\end{taskBN}

\begin{taskBN}{293}
\addpictoright[0.3\textwidth]{images/621518486847437n0}В цилиндрическом сосуде уровень жидкости достигает 96 см. На какой высоте будет находиться уровень жидкости, если её перелить во второй цилиндрический сосуд, радиус которого в 4 раза больше радиуса первого? Ответ выразите в сантиметрах.
\end{taskBN}

\begin{taskBN}{294}
\addpictoright[0.3\textwidth]{images/7597045940108504n0} Первая цилиндрическая кружка в 5 раз шире второй, а вторая в 2,5 раза выше первой. Найдите отношение объёма второй кружки к объёму первой.
\end{taskBN}

\begin{taskBN}{295}
\addpictoright[0.3\textwidth]{images/281901466172132n0} Первая цилиндрическая кружка в 5 раз шире второй, а вторая в 4 раза выше первой. Найдите отношение объёма первой кружки к объёму второй.
\end{taskBN}

\begin{taskBN}{296}
\addpictoright[0.3\textwidth]{images/339315863021182n0}Площадь боковой поверхности цилиндра равна $2160\pi$, длина окружности основания равна $72\pi$. Найдите площадь полной поверхности цилиндра. Ответ сократите на $\pi$.
\end{taskBN}

\begin{taskBN}{297}
\addpictoright[0.3\textwidth]{images/723592352228099n0}В цилиндрический сосуд налили $3800\mbox{дм}^3$ воды. Уровень воды при этом достигает высоты $200$дм. В жидкость полностью погрузили деталь. При этом уровень жидкости в сосуде поднялся в $\frac{69}{40}$ раза. Чему равен объем детали? Ответ выразите в $\mbox{дм}^3$.
\end{taskBN}

\begin{taskBN}{298}
\addpictoright[0.3\textwidth]{images/847300349937838n0}Объём цилиндра равен $5082\pi$, площадь полной поверхности равна $1166\pi$. Найдите длину окружности основания цилиндра. Ответ сократите на $\pi$.
\end{taskBN}

\begin{taskBN}{299}
\addpictoright[0.3\textwidth]{images/1456723957750525n0}В цилиндрический сосуд налили $500\mbox{дм}^3$ воды. Уровень воды при этом достигает высоты $20$дм. В жидкость полностью погрузили деталь. При этом уровень жидкости в сосуде поднялся в $\frac{6}{5}$ раза. Чему равен объем детали? Ответ выразите в $\mbox{дм}^3$.
\end{taskBN}

\begin{taskBN}{300}
\addpictoright[0.3\textwidth]{images/951039323522802n0}В цилиндрическом сосуде уровень жидкости достигает 12 см. На какой высоте будет находиться уровень жидкости, если её перелить во второй цилиндрический сосуд, диаметр которого в 2 раза меньше диаметра первого? Ответ выразите в сантиметрах.
\end{taskBN}

\end{document}