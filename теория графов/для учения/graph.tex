\documentclass[a4paper,12pt,twocolumn]{extarticle} \usepackage{etoolbox,fontspec,polyglossia,pdflscape,enumitem,amsfonts,amssymb,amsthm,mathtools,amsmath,icomma,euscript,mathrsfs,graphicx,wrapfig,array,tabularx,tabulary,booktabs,longtable,multirow,multicol,tikz,pdfpages}

% Настройка языка и шрифтов 
\setdefaultlanguage{russian} \setmainfont{CMU Serif} \setsansfont{CMU Sans Serif} \setmonofont{CMU Typewriter Text} \newfontfamily{\cyrillicfont}{CMU Serif} \newfontfamily{\cyrillicfontsf}{CMU Sans Serif} \newfontfamily{\cyrillicfonttt}{CMU Typewriter Text}

% Уменьшение межстрочного интервала 
\linespread{0.9}

% Уменьшение отступов абзацев 
\setlength{\parindent}{0.8em} \setlength{\parskip}{0.1em}

% Уменьшение отступов в списках 
\setlist{noitemsep,topsep=0pt,parsep=0pt,partopsep=0pt,leftmargin=0.8em}

% Уменьшение отступов вокруг формул 
\setlength{\abovedisplayskip}{3pt} \setlength{\belowdisplayskip}{3pt} \setlength{\abovedisplayshortskip}{1pt} \setlength{\belowdisplayshortskip}{1pt}

% Уменьшение отступов вокруг секций 
\usepackage{titlesec} \titlespacing{\section}{0pt}{2pt}{2pt} \titlespacing{\subsection}{0pt}{2pt}{2pt}

% Компактная геометрия страницы 
\usepackage[a4paper,landscape,left=1cm,right=1cm,top=0.8cm,bottom=0.8cm,columnsep=0.8cm,headsep=0.2cm]{geometry}

% Базовые команды и настройки 
\DeclareMathOperator{\sgn}{\mathop{sgn}} \newcommand*{\hm}[1]{#1\nobreak\discretionary{}{\hbox{$\mathsurround=0pt #1$}}{}} \graphicspath{{Изображения/}{image}} \setlength\fboxsep{2pt} \setlength\fboxrule{0.8pt}

% Настройка теорем 
\renewcommand{\thesubsection}{\arabic{subsection}} \newtheorem*{theorem}{Теорема} \newtheorem{corollary}{Следствие}
%%% Отключение колонтитулов и нумерации
\pagestyle{empty}

\begin{document}
% Неориентированные графы, cтепени, изоморфизм.
\subsection{Неориентированные графы, степени, изоморфизм}

\begin{itemize}
	\item \textbf{Граф}:
	\begin{itemize}
		\item Обозначается как $G = (X, \Gamma)$.
		\item Состоит из:
		\begin{enumerate}
			\item[\(1^\circ\)] Непустое множество $X$.
			\item[\(2^\circ\)] Отображение $\Gamma$ множества $X$ в $X$.
		\end{enumerate}
	\end{itemize}

	\item \textbf{Элементы графа}:
	\begin{itemize}
		\item \textbf{Вершина}: Каждый элемент множества $X$ называется точкой или вершиной графа.
		\item \textbf{Дуга}: Пара элементов $(x, y)$, где $y \in \Gamma x$, называется дугой графа.
	\end{itemize}

	\item \textbf{Изображение графа}:
	\begin{itemize}
		\item Элементы $X$ изображаются точками на плоскости.
		\item Пары точек $x$ и $y$, где $y \in \Gamma x$, соединяются непрерывной линией со стрелкой от $x$ к $y$.
	\end{itemize}

	\item \textbf{Множество дуг}:
	\begin{itemize}
		\item Обозначается через $U$.
		\item Дуги обозначаются буквами $\alpha$, $\beta$, $\omega$ (при необходимости с индексами).
	\end{itemize}
\end{itemize}

\textit{Степенью} вершины \(v_i\) в графе \(G\) — обозначается \(d_i\) или \(\deg v_i\) — называется число рёбер, инцидентных \(v_i\) (то есть рёбер, которые соединены с \(v_i\)). Поскольку каждое ребро инцидентно двум вершинам, в сумму степеней вершин графа каждое ребро вносит двойку. Таким образом, мы приходим к утверждению, которое установлено Эйлером и является исторически первой теоремой теории графов.

\subsubsection*{Теорема 2.1}
Сумма степеней вершин графа \(G\) равна удвоенному числу его рёбер:
\[
\sum_i \deg v_i = 2q.
\]

\subsubsection*{Следствие 2.1 (a)}
В любом графе число вершин с нечётными степенями чётно.

В \((p, q)\)-графе \(0 \leq \deg v \leq p-1\) для любой вершины \(v\). Минимальная степень вершин графа \(G\) обозначается через \(\min \deg G\) или \(\delta(G)\), максимальная — через \(\max \deg G = \Delta(G)\). Если \(\delta(G) = \Delta(G) = r\), то все вершины имеют одинаковую степень и такой граф \(G\) называется \textit{регулярным} (или \textit{однородным}) степени \(r\). В этом случае говорят о степени графа и пишут \(\deg G = r\).

Регулярный граф степени 0 совсем не имеет рёбер. Если \(G\) — регулярный граф степени 1, то каждая его компонента содержит точно одно ребро; в регулярном графе степени 2 каждая компонента — цикл, и, конечно, обратно. Первые интересные\(^2\) регулярные графы имеют степень 3; такие графы называются \textit{кубическими}. На рис. 2.11 показаны два регулярных графа с 6 вершинами. Второй из них изоморфен каждому из трёх графов, изображённых на рис. 2.5.

\subsection*{Следствие 2.1 (б)}
Каждый кубический граф имеет чётное число вершин.

Полезно дать названия вершинам с малыми степенями. Вершина \(v\) называется \textit{изолированной}, если \(\deg v = 0\), и \textit{концевой} (или \textit{висячей}), если \(\deg v = 1\).

Два графа $G$ и $H$ изоморфны (записывается $G \cong H$ или иногда $G = H$), если между их множествами вершин существует взаимно однозначное соответствие, сохраняющее смежность. Например, графы $G_1$ и $G_2$ на рис.~2.5 изоморфны при соответствии $v_i \leftrightarrow u_i$, и чисто случайно оказалось, что граф $G_1$ изоморфен каждому из них. Совершенно очевидно, что изоморфизм есть отношение эквивалентности на графах.

% Маршруты, связность, метрика графа.
\newpage\textbf{Маршруты, связность, метрика графа}

\noindent\textbf{Определение.} \textit{Маршрут} в графе $G$ (последовательность переходов по вершинам и рёбрам) -- чередующаяся последовательность вершин и рёбер $v_0, x_1, v_1, \ldots, x_n, v_n$, где:
\begin{itemize}
\item Начинается и заканчивается вершиной (точкой графа)
\item Каждое ребро инцидентно (напрямую соединяет) предшествующей и следующей вершинам
\end{itemize}

\noindent\textbf{Обозначение:} $(v_0-v_n)$-маршрут (путь от вершины $v_0$ до $v_n$) записывается как $v_0 v_1 v_2 \ldots v_n$

\noindent\textbf{Классификация маршрутов:}
\begin{itemize}
\item \textit{Замкнутый}: $v_0 = v_n$ (начальная и конечная вершины совпадают)
\item \textit{Открытый}: $v_0 \neq v_n$ (начальная и конечная вершины различны)
\item \textit{Цепь} (trail): все рёбра различны (по каждому ребру проходим не более одного раза)
\item \textit{Простая цепь} (path): все вершины и рёбра различны (нигде не повторяемся)
\item \textit{Цикл}: замкнутая цепь (маршрут возвращается в начальную точку)
\item \textit{Простой цикл}: замкнутый маршрут с $n \geq 3$ различными вершинами (замкнутый путь без повторений вершин, кроме начальной/конечной)
\end{itemize}

\noindent\textbf{Длина маршрута} $v_0 v_1 \ldots v_n$ = $n$ (количество пройденных рёбер)

\noindent\textbf{Важные метрики:}
\begin{itemize}
\item \textit{Обхват графа} $g(G)$: длина кратчайшего простого цикла (минимальное количество рёбер в замкнутом пути без повторений)
\item \textit{Окружение графа} $c(G)$: длина длиннейшего простого цикла (максимальное количество рёбер в замкнутом пути без повторений)
\end{itemize}

\noindent\textbf{Примечание:} $g(G)$ и $c(G)$ не определены для графов без циклов (для деревьев и лесов).

%Самодополнительные графы
\newpage\subsection{Самодополнительные графы}

\begin{wrapfigure}{r}{0.35\textwidth}
    \begin{tikzpicture}[scale=1]
    % Первый граф (G)
    \begin{scope}[xshift=0cm]
        \foreach \angle [count=\i] in {90,150,...,450} {
            \node[circle, fill=black, inner sep=1.5pt] (v\i) at (\angle:1) {};
        }
        \foreach \i in {1,...,6} {
            \pgfmathtruncatemacro{\next}{mod(\i,6)+1}
            \draw (v\i) -- (v\next);
        }
        \draw (v1) -- (v4);
        \draw (v2) -- (v5);
        \draw (v3) -- (v6);
        \node at (0,-1.5) {$G$};
    \end{scope}
    
    % Второй граф (G с чертой)
    \begin{scope}[xshift=3.5cm]
        \foreach \angle [count=\i] in {30,90,...,389} {
            \node[circle, fill=black, inner sep=1.5pt] (w\i) at (\angle:1) {};
        }
        \foreach \i/\j in {1/3,3/5,5/1,2/4,4/6,6/2} {
            \draw (w\i) -- (w\j);
        }
        \node at (0,-1.5) {$\overline{G}$};
    \end{scope}
    \node at (1.5,-2) {Рис. 2.12. Граф и его дополнение.};
    \end{tikzpicture}
\end{wrapfigure}

\noindent\textbf{Определение.} \textit{Дополнение графа} $\overline{G}$ (граф с теми же вершинами, но противоположными связями):
\begin{itemize}[noitemsep,topsep=0pt]
\item Множество вершин: $V(\overline{G}) = V(G)$
\item Две вершины смежны в $\overline{G}$ $\Leftrightarrow$ несмежны в $G$
\end{itemize}

\noindent\textbf{Определение.} \textit{Самодополнительный граф} -- граф, изоморфный своему дополнению (структура графа совпадает со структурой его дополнения).

\noindent\textbf{Полный граф} $K_p$ (все вершины попарно соединены):
\begin{itemize}[noitemsep,topsep=0pt]
\item Содержит $p$ вершин
\item Имеет $\binom{p}{2}$ рёбер
\item Является регулярным степени $p-1$
\item Частный случай: $K_3$ -- треугольник
\end{itemize}

\noindent\textbf{Вполне несвязный граф} $\overline{K_p}$ -- дополнение полного графа (регулярный граф степени 0).

% Экстремальные графы
\newpage\subsection{Экстремальные графы}

\noindent\textbf{Теорема 2.3 (Турана)} (о максимальном числе рёбер в графе без треугольников):\\
Наибольшее число рёбер у графов с $r$ вершин без треугольников равно $\lfloor r^2/4 \rfloor$.

\noindent\textbf{Доказательство} (по индукции для чётных $r$):
\begin{enumerate}[noitemsep,topsep=0pt]
\item База: очевидна для малых $r$
\item Шаг: для $r = 2n + 2$, где утверждение верно для всех чётных $r \leq 2n$:
   \begin{itemize}[noitemsep]
   \item Пусть $G$ -- граф с $p = 2n + 2$ вершинами без треугольников
   \item Существуют смежные вершины $u$, $v$ (граф не вполне несвязный)
   \item В подграфе $G' = G - \{u, v\}$ максимум $n^2$ рёбер
   \item Нет вершины $w$, смежной с $u$ и $v$ одновременно
   \item Если $w$ смежна с $k$ вершинами $G'$, то $v$ смежна максимум с $(2n - k)$ вершинами
   \item Всего рёбер: $n^2 + k + (2n - k) + 1 = n^2 + 2n + 1 = p^2/4$
   \end{itemize}
\end{enumerate}

\noindent\textbf{Конструктивное доказательство существования:}\\
Для чётного $p$ $(p, p^2/4)$-граф без треугольников строится так:
\begin{itemize}[noitemsep]
\item Берём два множества $V_1$ и $V_2$ по $p/2$ вершин
\item Соединяем каждую вершину из $V_1$ с каждой из $V_2$
\end{itemize}

\noindent\textbf{Примечания:}
\begin{itemize}[noitemsep]
\item Доказательство существования чисел $r(m, n)$ см. у М. Холла
\item По определению бесконечный граф не является графом
\item Обзор бесконечных графов: см. Нэш-Вильямс
\end{itemize}

% Числа Рамсея
\newpage\subsection{Числа Рамсея}
Широко известна следующая головоломка.

\textit{Доказать, что среди любых шести человек найдутся либо трое попарно знакомых, либо трое попарно незнакомых.}

\begin{enumerate}
    \item Напоминаем читателю (см. введение), что в тексте не все теоремы доказываются.
    \item По своим структурным свойствам. — \textit{Прим. перев.}
\end{enumerate}

В этих терминах головоломку можно сформулировать так:

\textbf{Теорема 2.2.} Если \( G \) — граф с шестью вершинами, то либо \( G \), либо \( \overline{G} \) содержит треугольник.

\textbf{Доказательство.} Пусть \( v \) — произвольная вершина графа \( G \), имеющего шесть вершин. Так как вершина \( v \) с любой из остальных пяти вершин смежна или в \( G \), или в \( \overline{G} \), то, не теряя общности, можно предположить, что вершины \( u_1, u_2, u_3 \) смежны с \( v \) в \( G \). Если какие-либо две из вершин \( u_1, u_2, u_3 \) смежны в \( G \), то вместе с \( v \) они образуют треугольник. Если никакие две из них не смежны в \( G \), то в графе \( \overline{G} \) вершины \( u_1, u_2, u_3 \) образуют треугольник.

Обобщая теорему 2.2, естественно поставить вопрос: каково наименьшее целое число \( r(m, n) \), для которого каждый граф с \( r(m, n) \) вершинами содержит \( K_m \) или \( K_n \)?

Числа \( r(m, n) \) называются \textit{числами Рамсея} \(^1\). Ясно, что \( r(m, n) = r(n, m) \). Задача, связанная с нахождением чисел Рамсея, остается нерешенной, хотя известна простая верхняя оценка, полученная Эрдёшем и Секерешем \(^1\):

\begin{equation}
r(m, n) \leq \binom{m + n - 2}{m - 1}.
\end{equation}

Постановка этой задачи вытекает из теоремы Рамсея. Бесконечный граф \(^2\) имеет бесконечное множество вершин и не содержит кратных ребер и петель. Рамсей \(^1\) доказал (на языке теории множеств), что каждый бесконечный граф содержит \( \aleph_0 \) попарно смежных вершин или \( \aleph_0 \) попарно несмежных вершин.

Все известные числа Рамсея приведены в табл. 2.1 (взята из обзорной статьи Гравера и Якелл \(^1\)).

% Эйлеровы графы
\newpage\subsection{Эйлеровы графы}

\noindent\textbf{Определение.} \textit{Эйлеров граф} -- граф, содержащий цикл со всеми вершинами и рёбрами (имеет эйлеров цикл). Обязательно связный.

\noindent\textbf{Теорема 7.1} (критерий эйлеровости). Для связного графа $G$ эквивалентны:
\begin{enumerate}[noitemsep,topsep=0pt]
\item $G$ -- эйлеров граф
\item Все вершины имеют чётную степень
\item Рёбра можно разбить на простые циклы
\end{enumerate}

\noindent\textbf{Доказательство:}\\
(1)$\Rightarrow$(2): В эйлеровом цикле каждое прохождение вершины даёт +2 к её степени. Каждое ребро используется один раз $\Rightarrow$ степени чётны.

\noindent(2)$\Rightarrow$(3): В связном графе с чётными степенями:
\begin{itemize}[noitemsep]
\item Найдём простой цикл $Z$
\item Удалим его рёбра -- получим граф $G_1$ с чётными степенями
\item Повторяем до пустого графа $G_n$
\end{itemize}

\noindent(3)$\Rightarrow$(1): Имея разбиение на циклы:
\begin{itemize}[noitemsep]
\item Берём цикл $Z_1$
\item Находим цикл $Z_2$ с общей вершиной $v$
\item Строим замкнутую цепь из $Z_1$ и $Z_2$
\item Продолжаем до полного эйлерова цикла
\end{itemize}

\noindent\textbf{Следствие 7.1(a).} В связном графе с $2n$ вершинами нечётной степени ($n \geq 1$) рёбра можно разбить на $n$ открытых цепей.

\noindent\textbf{Следствие 7.1(б).} В связном графе с двумя вершинами нечётной степени существует открытая цепь, содержащая все рёбра (начинается и заканчивается в вершинах нечётной степени).

%Деревья
\newpage\subsection{Деревья}

\noindent\textbf{Основные определения:}
\noindent\textbf{Ациклический граф} -- граф без циклов.
\noindent\textbf{Дерево} -- связный ациклический граф.
\noindent\textbf{Лес} -- граф без циклов (компоненты -- деревья).

\noindent\textbf{Теорема 4.1.} Для графа $G$ эквивалентны:
\noindent 1) $G$ -- дерево
\noindent 2) любые две вершины соединены единственной простой цепью
\noindent 3) $G$ связен и $p = q + 1$
\noindent 4) $G$ ациклический и $p = q + 1$
\noindent 5) $G$ ациклический, и добавление любого ребра создаёт ровно один цикл
\noindent 6) $G$ связный, не $K_p$ при $p \geq 3$, добавление ребра создаёт один цикл
\noindent 7) $G$ не $K_3 \cup K_1$ и не $K_3 \cup K_2$, $p = q + 1$, добавление ребра создаёт один цикл

\noindent\textbf{Доказательство} (схема):
\noindent 1$\Rightarrow$2: От противного: две цепи образуют цикл
\noindent 2$\Rightarrow$3: Индукция по числу вершин
\noindent 3$\Rightarrow$4: От противного: цикл длины $n$ требует $q \geq p$
\noindent 4$\Rightarrow$5: Единственность компоненты из $p = q + k$
\noindent 5$\Rightarrow$6: $K_p$ при $p \geq 3$ содержит цикл
\noindent 6$\Rightarrow$7: Анализ возможных циклов
\noindent 7$\Rightarrow$1: Исключение случаев с циклами

\noindent\textbf{Следствие 4.1(а).} В нетривиальном дереве есть минимум две висячие вершины.
\noindent\textit{Доказательство:} Из $\sum d_i = 2(p-1)$ в дереве.

%Диаметр и радиус графа
\newpage\subsection{Диаметр и радиус графа}

\noindent\textbf{Определение.} \textit{Расстояние} $d(u,v)$ между вершинами (длина кратчайшей простой цепи):
\[ d(u,v) = \begin{cases} 
\text{длина кратчайшей }(u\text{-}v)\text{-цепи}, & \text{если вершины соединены} \\
\infty, & \text{если вершины не соединены}
\end{cases} \]

\noindent\textbf{Свойства метрики} (для связного графа):

\noindent 1) $d(u,v) \geq 0$; $d(u,v) = 0 \Leftrightarrow u = v$ (неотрицательность)

\noindent 2) $d(u,v) = d(v,u)$ (симметричность)

\noindent 3) $d(u,v) + d(v,w) \geq d(u,w)$ (неравенство треугольника)

\noindent\textbf{Термины:}

\noindent • \textit{Геодезическая} -- кратчайшая простая $(u\text{-}v)$-цепь

\noindent • \textit{Диаметр графа} $d(G)$ -- длина самой длинной геодезической

\noindent\textbf{Степени графа:} 
\noindent Для графа $G$ определяется $G^k$ ($k$-я степень):

\noindent • $V(G^k) = V(G)$ (те же вершины)

\noindent • Вершины $u,v$ смежны в $G^k \Leftrightarrow d(u,v) \leq k$ в $G$
\noindent\textit{Примеры:} $C_5^2 = K_5$, $P_4^2 = K_1 + K_3$

%Хроматическое число графа
\newpage\subsection{Хроматическое число графа}

\noindent\textbf{Определение.} \textit{p-хроматический граф} -- граф, вершины которого можно раскрасить в p цветов так, чтобы смежные вершины имели разные цвета.

\noindent\textbf{Хроматическое число} $\chi(G)$ -- минимальное p, при котором граф p-хроматический.

\noindent\textbf{Хроматический класс} -- минимальное число цветов q для раскраски рёбер без одинаковых смежных рёбер.

\noindent\textbf{Теорема о двудольных графах.} Граф двудольный $(χ(G)=2) \Longleftrightarrow$ не содержит циклов нечётной длины.

\noindent\textbf{Доказательство:}\\
($\Rightarrow$) Алгоритм раскраски в 2 цвета:
1) Выбираем вершину a, красим в синий
2) Смежные с синими красим в красный, с красными -- в синий
3) Отсутствие нечётных циклов гарантирует корректность
 
($\Leftarrow$) От противного: в двудольном графе нельзя раскрасить нечётный цикл в 2 цвета.

\noindent\textbf{Теорема 4.} Для симметрического графа G эквивалентны:
1) G является p-хроматическим
2) Существует функция Гранди g(x) с $\max g(x) \leq p-1$

\noindent\textbf{Теорема 5.} Для графов G (p+1-хром.) и H (q+1-хром.):
$\chi(G \times H) = r+1$, где $r = max{p'+q': p'\leq p, q'\leq q}$

\noindent\textbf{Теорема 6.} Для графов G и H с χ(G)=p, χ(H)=q:
$\chi(G \times H) = \min\{p,q\}$

\noindent\textbf{Важное свойство:} Для плоских графов χ(G)$leq$5 (достаточно 5 цветов для раскраски карты).

%Цикломатическое число графа
\newpage\subsection{Цикломатическое число графа}

\noindent\textbf{Мультиграф} $(X,U)$ -- пара из множества вершин $X$ и множества рёбер $U$, где пара вершин может соединяться несколькими рёбрами.

\noindent\textbf{Важные числовые характеристики:}
\begin{itemize}
    \item Для мультиграфа $G$ с $n$ вершинами, $m$ рёбрами, $p$ компонентами:
    \begin{itemize}
        \item Ранг графа: $\rho(G) = n - p$
        \item Цикломатическое число: $\nu(G) = m - n + p = m - \rho(G)$
    \end{itemize}
\end{itemize}

\noindent\textbf{Теорема 1.} При добавлении ребра между $a$ и $b$:
\begin{itemize}
    \item Если $a,b$ соединены цепью или совпадают:
    \begin{itemize}
        \item $\rho(G') = \rho(\bar{G})$
        \item $\nu(G') = \nu(\bar{G}) + 1$
    \end{itemize}
    \item Иначе:
    \begin{itemize}
        \item $\rho(G) = \rho(\bar{G}) + 1$
        \item $\nu(G') = \nu(\bar{G})$
    \end{itemize}
\end{itemize}

\noindent\textbf{Векторное представление циклов:}
\begin{itemize}
    \item Каждому ребру присваивается ориентация
    \item Для цикла $\mu$: $c^k = r_k - s_k$, где $r_k,s_k$ -- число проходов по/против ориентации
    \item Цикл представляется вектором $(c^1,\ldots,c^m)$
    \item Циклы независимы $\Leftrightarrow$ их векторы линейно независимы
\end{itemize}

\noindent\textbf{Теорема 2.} Цикломатическое число $\nu(G)$ равно максимальному количеству независимых циклов.

\noindent\textbf{Следствия:}
\begin{enumerate}
    \item $\nu(G)=0$ $\Leftrightarrow$ граф без циклов
    \item $\nu(G)=1$ $\Leftrightarrow$ граф содержит ровно один цикл
\end{enumerate}

\noindent\textbf{Теорема 3.} В сильно связном графе цикломатическое число равно максимальному количеству независимых контуров.

%Плоские графы, формула Эйлера
\newpage\subsection{Плоские графы и формула Эйлера}

\textbf{Определения}
\begin{itemize}
    \item \textbf{Планарный граф}: Граф, который можно нарисовать без пересечения рёбер.
    \item \textbf{Плоский граф}: Граф, нарисованный на плоскости.
    \item \textbf{Грани}: Области, определяемые плоским графом; внешняя грань — неограниченная.
    \item \textbf{Цикл}: Путь, начинающийся и заканчивающийся в одной вершине без повторений.
\end{itemize}

\textbf{Формула Эйлера}
Для полиэдров: \( V - E + F = 2 \), где \( V \) — вершины, \( E \) — рёбра, \( F \) — грани.

\textbf{Графовая версия}
Для связного плоского графа: \( p - q + r = 2 \).

\textbf{Следствия и теоремы}
\begin{itemize}
    \item \textbf{Следствие 11.1 (а)}: Если каждая грань — \( n \)-цикл, то \( q = \frac{n(p-2)}{n-2} \).
    \item \textbf{Максимальный планарный граф}: Граф, который перестаёт быть планарным при добавлении ребра.
    \item \textbf{Следствие 11.1 (б)}: Для максимального плоского графа \( q = 3p - 6 \).
    \item \textbf{Условие планарности}: Для \( p \geq 3 \), \( q \leq 3p - 6 \).
    \item \textbf{Непланарные графы}: \( K_5 \) и \( K_{3,3} \).
    \item \textbf{Теорема Уитни}: Граф планарен, если каждый его блок планарен.
    \item \textbf{Теорема 11.3}: Для любой грани \( f \) двусвязного плоского графа \( G \) найдётся изоморфный плоский граф с внешней гранью \( f \).
\end{itemize}

\textbf{Дополнительные концепции}
\begin{itemize}
    \item \textbf{Выпуклый многогранник}: Многогранник, содержащий любые соединяющие его точки отрезки.
    \item \textbf{Теорема Штейница и Радемахера}: Граф — 1-скелет выпуклого многогранника, если он планарен и трёхсвязен.
    \item \textbf{Теорема 11.7}: Любой планарный граф изоморфен плоскому графу с прямыми рёбрами.
\end{itemize}

%Линейно независимые циклы
\newpage\textbf{Линейно независимые циклы}

\noindent\textbf{Линейно независимые циклы}
\begin{itemize}
    \item \textbf{Пространство циклов} и \textbf{пространство коциклов} определяются над полем $F_2 = \{0, 1\}$.
    \item \textbf{0-цепь} — линейная комбинация вершин $\Sigma e_i v_i$.
    \item \textbf{1-цепь} — линейная комбинация рёбер $\Sigma e_i x_i$.
    \item \textbf{Граничный оператор} $\partial$: переводит 1-цепи в 0-цепи.
    \begin{itemize}
        \item $\partial$ — линейный оператор.
        \item Если $x = uv$, то $\partial x = u + v$.
    \end{itemize}
    \item \textbf{Кограничный оператор} $\delta$: переводит 0-цепи в 1-цепи.
    \begin{itemize}
        \item $\delta$ — линейный оператор.
        \item $\delta v = \Sigma e_i x_i$, где $e_i = 1$, если ребро $x_i$ инцидентно $v$.
    \end{itemize}
\end{itemize}

\noindent\textbf{Циклы и Коциклы}
\begin{itemize}
    \item \textbf{Циклический вектор} — 1-цепь с границей 0 (набор простых циклов без общих рёбер).
    \item \textbf{Пространство циклов} — векторное пространство всех циклических векторов.
    \item \textbf{Базис циклов} — максимальный набор независимых простых циклов.
    \item \textbf{Коцикл} — минимальный разрез графа.
    \item \textbf{Пространство коциклов} — множество всех кограниц графа.
    \item \textbf{Базис коциклов} — базис пространства коциклов, состоящий из коциклов.
\end{itemize}

\noindent\textbf{Циклический ранг}
\begin{itemize}
    \item \textbf{Теорема 4.5}: Циклический ранг $m(G)$ равен числу хорд любого остова в $G$.
    \item \textbf{Следствие 4.5 (а)}: $m(G) = q - p + 1$ для связного $(p, q)$-графа.
    \item \textbf{Следствие 4.5 (б)}: $m(G) = q - p + k$ для $(p, q)$-графа с $k$ компонентами.
\end{itemize}

\noindent\textbf{Коциклический ранг}
\begin{itemize}
    \item \textbf{Теорема 4.6}: Коциклический ранг $t^*(G)$ равен числу рёбер любого остова.
    \item \textbf{Следствие 4.6 (а)}: $t^*(G) = p - 1$ для связного $(p, q)$-графа.
    \item \textbf{Следствие 4.6 (б)}: $t^*(G) = p - k$ для $(p, q)$-графа с $k$ компонентами.
\end{itemize}

\noindent\textbf{Замечания}
\begin{itemize}
    \item Уравнение Эйлера — Пуанкаре: $p - q = k - m(G)$.
    \item Графы как симплициальные комплексы: вершины — 0-симплексы, рёбра — 1-симплексы.
\end{itemize}

%Хроматическое число плоского графа
\newpage\subsection{Хроматическое число плоского графа}
Хер знает где это искать

%Примеры неплоских графов
\newpage\subsection{Примеры неплоских графов}
Ещё одна хуета

%Ориентированные графы, порядковая функция
\newpage\textbf{Порядковые числа в графах}

\noindent\textbf{Пример использования порядковых чисел}

\noindent\textbf{Порядковое число} графа $G$ — минимальное число цветов, необходимых для раскраски вершин графа так, чтобы никакие две смежные вершины не имели одинаковый цвет.

\noindent\textbf{Теорема 5.1}: Для любого графа $G$ его порядковое число $\chi(G)$ удовлетворяет неравенству:
\[
\chi(G) \leq \Delta(G) + 1
\]
где $\Delta(G)$ — максимальная степень вершины в графе $G$.

\noindent\textbf{Пример}: Рассмотрим граф $K_4$ (полный граф с четырьмя вершинами).
\begin{itemize}
    \item Все вершины соединены друг с другом.
    \item $\Delta(K_4) = 3$.
    \item $\chi(K_4) = 4$, так как каждая вершина должна иметь уникальный цвет.
\end{itemize}

\noindent\textbf{Алгоритм раскраски графа}:
\begin{enumerate}
    \item Выберите вершину $v$ с максимальной степенью.
    \item Назначьте $v$ минимально возможный цвет, не совпадающий с цветами её соседей.
    \item Повторите для всех вершин графа.
\end{enumerate}

\noindent\textbf{Замечание}: Порядковое число графа может быть равно $\Delta(G)$, если граф является двудольным.

\noindent\textbf{Следствие 5.1(a)}: Если граф $G$ планарен, то $\chi(G) \leq 4$ (теорема о четырёх красках).

\noindent\textbf{Применение}: Раскраска графов используется в задачах планирования, таких как распределение частот в беспроводных сетях и составление расписаний.

%Функция Гранди
\newpage\textbf{Функция Гранди}

\begin{itemize}
    \item \textbf{Функция Гранди:} Для конечного графа $(X, \Gamma)$ функция $g(x)$ — это наименьшее неотрицательное целое число, не принадлежащее множеству $g(\Gamma x) = \{g(y) \mid y \in \Gamma x\}$.
    \item \textbf{Пример 1:} Граф на рис. 3-3 допускает две функции Гранди. Если $\varGamma x = \{y_1, y_2, \ldots\}$, то $g(x)$ — наименьшее число, отличное от $g(y_1), g(y_2)$.
    \item \textbf{Пример 2:} Граф на рис. 3-2 допускает единственную функцию Гранди $g(x)$, где $g(x) = o(x)$ для $x \neq a$, а в $a$ принимает значение $\omega$ (трансфинитное число).
    \item \textbf{Теорема 5:} Прогрессивно конечный граф допускает одну функцию Гранди $g(x)$, и $g(x) \leq o(x)$.
    \item \textbf{Доказательство:} Индукция по множествам:
    \begin{align*}
        X(0) &= \{x \mid \varGamma x = \emptyset\}, \\
        X(1) &= \{x \mid \varGamma x \subseteq X(0)\}, \\
        X(2) &= \{x \mid \varGamma x \subseteq X(1)\}.
    \end{align*}
    \item \textbf{Теорема 6:} Если $|X| < \infty$, то $g(x) \leq \Gamma$. Если $g(x) = n$, то $g$ принимает в $\varGamma x$ все значения $0, 1, 2, \ldots, n-1$, следовательно, $|X| \geq n - g(x)$.
    \item \textbf{Заключение:} Для $\Gamma$-конечного или прогрессивно ограниченного графа значения $g(x)$ остаются конечными.
\end{itemize}

%Внутреннее устойчивое множество
\newpage\subsection{Внутреннее устойчивое множество}

Рассмотрим граф $G = (X, \Gamma)$, множество $S \subseteq V$ называется \textit{внутренне устойчивым}, если никакие две вершины из $S$ не смежны, другими словами если

\[
\Gamma S \cap S = \emptyset
\]

Обозначим через $\mathfrak{S}$ семейство всех внутренне устойчивых множеств графа; имеем

\[
S \in \mathfrak{S} \land S \subseteq A \Rightarrow A \in \mathfrak{S}
\]

По определению \textit{число внутренней устойчивости} графа $G$ есть

\[
\alpha(G) = \max_{S \in \mathfrak{S}} |S|
\]

\textbf{Замечание 1} Хроматическое число $\gamma(G)$ и число внутренней устойчивости $\alpha(G)$ связаны неравенством
\[
\alpha(G) \gamma(G) \geq |X|.
\]

В самом деле, можно разбить $X$ на $\gamma(G)$ внутренних устойчивых множеств, образованных вершинами одинакового цвета и содержащих соответственно $m_1, m_2, \ldots, m_q$ вершин. Поэтому
\[
|X| = m_1 + m_2 + \ldots + m_q \leq \gamma(G) \cdot \alpha(G) + \ldots + \alpha(G) = \gamma(G) \cdot \alpha(G).
\]

\textbf{Замечание 2} Можно поставить вопрос, не являются ли связи между обоими понятиями более тесными и нельзя ли найти хроматическое число, окрашивая сначала в цвет (1) наибольшее внутреннее устойчивое множество $S_1$, затем в цвет (2) наибольшее внутреннее устойчивое множество $S_2$ подграфа, порожденного вершинами $X \setminus S_1$, далее в цвет (3) наибольшее внутреннее устойчивое множество оставшегося подграфа и т.д. Оказывается, это не так, что видно на примере графа, изображенного на рис. 4--6 (его хроматическое число очевидно, равно 4), вершины, изображенные белыми кружками, образуют единственное наибольшее внутреннее устойчивое множество, но если их окрасить в один цвет, то остальные вершины $a \, b \, c \, d$ надо было бы окрасить с помощью только трех цветов, а это, очевидно, невозможно.

Иногда для двух графов $G$ и $H$ возникает вопрос о нахождении числа внутренней устойчивости произведения $G \times H$.

\begin{figure}[h]
    \centering
    \includegraphics[width=0.5\textwidth]{example-imageю.png} % Replace with actual image path
    \caption{Рис. 4--6}
\end{figure}


\textbf{Лемма 1} Для двух графов $G$ и $H$
\[
\alpha(G \times H) \geq \alpha(G) \cdot \alpha(H)
\]

Действительно, если $S$ и $T$ — наибольшие внутренне устойчивые множества соответственно для $G$ и $H$, то декартово произведение $S \times T$ является внутренне устойчивым в графе $G \times H$, откуда
\[
\alpha(G \times H) \geq |S \times T| = |S| \cdot |T| = \alpha(G) \cdot \alpha(H)
\]

Эта лемма подсказывает нам следующее определение, назовем \emph{емкость} графа $G$ число
\[
\theta(G) = \sup_{n} \sqrt[n]{\alpha(G^n)}
\]

Имеем $\theta(G) \geq \alpha(G)$, мы собираемся показать, что почти всегда
\[
\theta(G) = \alpha(G)
\]

Между прочим, Шеннон установил, что граф $G$, изображенный на рис. 4-8, является единственным графом с числом вершин менее шести, для которого $\theta(G) \neq \alpha(G)$, фактически его емкость $\theta(G)$ не удалось определить, и известно лишь, что
\[
\forall \, 5 \leq \theta(G) \leq \frac{5}{2}
\]

Рассмотрим однозначное отображение $\sigma$ множества $X$ в себя, такое отображение называется \emph{сохраняющим}, если
\[
y \neq x, \, y \notin \Gamma(x) \Rightarrow \sigma(y) \neq \sigma(x), \, \sigma(y) \notin \Gamma(\sigma(x))
\]

Это отображение сохраняет свойство пары вершин "быть несмежными и различными".

\textbf{Лемма 2} Сохраняющее отображение $\sigma$ переводит внутренне устойчивое множество $S$ во внутренне устойчивое множество $\sigma(S)$, и при этом $|\sigma(S)| = |S|$.

В самом деле, ввиду однозначности отображения $\sigma$ имеем $|\sigma(S)| \leq |S|$, а так как $\sigma$ сохраняюще, то $|\sigma(S)| = |S|$.

\textbf{Лемма 3} Если множество $\sigma(X)$ внутренне устойчиво, то число внутренне устойчивости графа $G$ есть $\theta(G) = \alpha(G)$.

Действительно, раз $\sigma(X)$ внутренне устойчиво, то
\[
|\sigma(X)| \leq \max_{S} |S| = \alpha(G)
\]

С другой стороны, если $S_0$ — наибольшее внутренне устойчивое множество, то в силу леммы 2,

\[
|\sigma(X)| \geq |\sigma(S_0)| = |S_0| = \alpha(G),
\]

Отсюда

\[
\alpha(G) = |\sigma(X)|.
\]

\textbf{Теорема 7 (Шеннон)} \textit{Если хотя бы для одного из графов $G$ и $H$ существует сохранное отображение $\sigma$, переводящее множество вершин этого графа во внутренне устойчивое множество, то}

\[
\alpha(G \times H) = \alpha(G)\alpha(H).
\]

Достаточно показать, что $\alpha(G \times H) \leq \alpha(G)\alpha(H)$, пусть $\sigma$ — сохранное отображение для $G$, при котором $\sigma(X)$ внутренне устойчиво, и пусть $\sigma_0$ — отображение множества вершин графа $G \times H$ в себя, определенное следующим образом.

\[
\sigma_0(x, y) = (\sigma(x), y).
\]

Отображение $\sigma_0$ переводит две несмежные различные вершины $ξ = (x, y)$ и $ξ' = (x', y')$ в две несмежные различные вершины $(\sigma x, y)$ и $(\sigma x', y')$ и поэтому сохранно.

Если $S_0$ — наибольшее внутренне устойчивое множество графа $G \times H$, то $\alpha(G \times H) = |S_0| = |\sigma_0(S_0)|$ в силу леммы 2, распределим элементы $\sigma_0(S_0)$ по различным классам в зависимости от первой буквы каждого слова. Согласно лемме 3 получим: $|\sigma(X)| = \alpha(G)$ различных классов. Поскольку никакие два элемента из $\sigma_0(S_0)$ не смежны, каждый класс имеет самое большее $\alpha(H)$ элементов, значит

\[
\alpha(G \times H) = |\sigma_0(S_0)| \leq \alpha(G)\alpha(H).
\]

\textbf{Следствие} \textit{Если множество вершин графа $G$ при помощи сохранного отображения $\sigma$ можно перевести во внутренне устойчивое множество, то емкость этого графа совпадает с числом внутренней устойчивости.}

В самом деле,

\[
\alpha(G) \leq (\alpha(G))^2
\]

и

\[
\alpha(G \times G) \leq (\alpha(G))^3
\]

и т.д.,

отсюда

\[
\alpha(G) = \sup_n \sqrt[n]{\alpha(G^n)} = \alpha(G).
\]

%Внешнее устойчивое множество
\newpage\subsection{Внешнее устойчивое множество}

Пусть дан граф $G = (X, \Gamma)$, говорят, что множество $T \subseteq X$ внешне устойчиво, если для каждой вершины $x \notin T$ имеем $\Gamma_x \cap T \neq \varnothing$, иначе говорят, если $\Gamma' T \supseteq X \setminus T$,

Если $\mathcal{T}$ — семейство всех внешне устойчивых множеств графа, то $X \in \mathcal{T}$

$T \in \mathcal{T} \quad A \supseteq T \Rightarrow A \in \mathcal{T}$

По определению, \textit{число внешней устойчивости} графа $G$ есть

\[
\beta(G) = \min_{T \in \mathcal{T}} |T|
\]

Задача, которая нас сейчас интересует, заключается в построении внешне устойчивого множества с наименьшим числом элементов.

\subsubsection*{Алгоритм для нахождения наименьшего внешне устойчивого множества}

Рассмотрим для примера граф $G = (X, \Gamma)$ изображенный на рис. 4--11 и определим отображение $\Delta$ множества $X = \{a, b\}$ в новое множество

\[
\overline{X} = \left\{
\begin{array}{c}
\overline{a} \\
\overline{b}
\end{array}
\right\}
\]

следующим образом

\[
\overline{y} \in \Delta x \iff y = x \text{ или } y \in \Gamma^{-1} x
\]

Тем самым построен так называемый простой граф\footnote{Простой граф --- это граф без петель и кратных рёбер.}, который мы обозначим через $(X, \overline{X})$ (рис. 4--12), если $T$ --- внешне устойчивое множество графа $G$, то $\Delta T = \overline{X}$. Наоборот, если $\Delta T = \overline{X}$, то множество $T$ внешне устойчиво в $G$. Задача свелась таким образом, к определению наименьшего множества $T \subseteq X$, для которого $\Delta T = \overline{X}$.

1\textdegree{} Удаляем из простого графа каждую такую вершину $x$, что $\Delta x \subseteq \Delta y$ для некоторой вершины $y \neq x$ (в самом деле, с точки зрения нашей задачи вершина $y$ будет полностью заменять вершину $x$). В нашем примере мы удаляем, таким образом, вершины $c, d, f$.

2\textdegree{} Если в простом графе имеется висячее ребро $(x, y)$, то очевидно, $x \in T$. В данном примере множеству $T$ заведомо принадлежит вершина $a$.

3\textdegree{} Исключим из простого графа вершину $a$, уже входящую в $T$, и множество $\Delta a = \{\overline{a}, \overline{b}, \overline{c}\}$, в результате получается граф, изображенный на рис. 4--13.

4\textdegree{} Снова пытаемся удалить некоторую вершину, как в 1\textdegree{} или исключив вершину, заведомо принадлежащую $T$, как в 2\textdegree{}, если упростить граф уже нельзя (как в данном примере), то назовем его неприводимым. Временно отнесем в $T$ произвольную вершину, скажем $b$.

5\textdegree{} Исключим, как в 3\textdegree{}, вершину $b$ и множество $\Delta b = \{a, e, f\}$.

6\textdegree{} Продолжаем упрощение, как выше: из полученного графа можно исключить вершину $g$, так как $\Delta g \subseteq \Delta e = \{g\}$. Включая в $T$ последнюю вершину $e$, получаем решение $T = \{a, b, e\}$.


%Ядро графа
\newpage\subsection{Ядро графа}
Пусть $G = (X, \Gamma)$ — конечный или бесконечный граф. Множество $S \subseteq X$ называется \textit{ядром} графа, если $S$ устойчиво как внутренне, так и внешне, т.е. если

\begin{equation}
x \in S \Rightarrow \Gamma x \cap S = \varnothing,
\end{equation}

\begin{equation}
x \notin S \Rightarrow \Gamma x \cap S \neq \varnothing.
\end{equation}

Из условия (1) следует, что ядро $S$ не содержит петель. Из условия (2) — что $S$ содержит все такие вершины $x$, для которых $\Gamma x = \varnothing$. Пустое множество $\varnothing$ не может быть ядром.

\textbf{Теорема 1} \\
Если $S$ — ядро графа $(X, \Gamma)$, то множество $S$ — максимальное в семействе $\mathfrak{S}$ внутренне устойчивых множеств, т.е.
\[
A \in \mathfrak{S}, \, A \supseteq S \Rightarrow A = S
\]

\textbf{Теорема 2} \\
В симметрическом графе без петель каждое максимальное множество семейства $\mathfrak{S}$ внутренне устойчивых множеств представляет собой ядро.

\textbf{Следствие} \\
Симметрический граф без петель обладает ядром.

\textbf{Характеристическая функция} \\
Функция $\varphi_S(x)$ множества $S$ определяется как:

\[
\varphi_S(x) = 
\begin{cases} 
1, & \text{при } x \in S \\ 
0, & \text{при } x \notin S 
\end{cases}
\]

\textbf{Теорема 3} \\
Для того чтобы множество $S$ было ядром, необходимо и достаточно чтобы для характеристической функции $\varphi_S(x)$ выполнялось соотношение

\[
\varphi_S(x) = 1 - \max_{y \in \Gamma x} \varphi_S(y)
\]

\textbf{Теорема 4} \\
Прогрессивно конечный граф обладает ядром.

\textbf{Теорема Ричардсона} \\
Конечный граф, не содержащий контуров нечетной длины, обладает ядром.


%Игры на графе, игра НИМ
\newpage\subsection{Игры на графе, игра НИМ}

Граф $(X, \Gamma)$ дает возможность определить некоторую игру двух игроков, которых мы назовем $(A)$ и $(B)$. Положениями этой игры служат вершины графа, начальная вершина $x_0$ выбирается жребием, и противники играют поочередно: сперва игрок $(A)$ выбирает вершину $x_1$ в множестве $\Gamma x_0$, затем $(B)$ выбирает вершину $x_2$ в множестве $\Gamma x_1$, после этого $(A)$ опять выбирает вершину $x_3$ в $\Gamma x_2$, и т.д. Если один из игроков выбрал вершину $x_n$, для которой $\Gamma x_n = \emptyset$, то партия оканчивается, игрок, выбравший вершину последним, выиграл, а его противник проиграл. Ясно, что если граф не является прогрессивно конечным, то партия может никогда не окончиться.

В честь известного развлечения, которое здесь обобщено, будем описанную только что игру называть \textit{игрой Ним}, а определяющий ее граф обозначать через $(X, \Gamma)$; сейчас наша задача состоит в том, чтобы охарактеризовать выигрышные положения, т.е. те вершины графа, выбор которых обеспечивает выигрыш партии независимо от ответов противника. Главным результатом является следующая

\textbf{Теорема 1.} \textit{Если граф имеет ядро $S$ и если один из игроков выбрал вершину в ядре, то этот выбор обеспечивает ему выигрыш или ничью.}

Действительно, если игрок $(A)$ выбрал вершину $x_1 \in S$, то либо $\Gamma x_1 = \emptyset$, и тогда он уже выиграл партию, либо его противник $(B)$ вынужден выбрать вершину $x_2 \in X \setminus S$, а значит, следующим ходом игрок $(A)$ может выбрать $x_3$ опять в $S$ и продолжать в том же духе. Если в какой-либо определенный момент один из игроков выбрал вершину $x_n$, для которой $\Gamma x_n = \emptyset$, то $x_n \in S$, и выигравшим партнером необходимо является $(A)$.

сновной метод для хорошего игрока состоит следовательно, в вычислении какой-либо функции Гранди, если она существует, с помощью этой функции \( g(x) \) получаем ядро
\[
S = \{ x | g(x) = 0 \}
\]

рассматриваемого графа. Если начальная вершина \( x_0 \) такова, что \( g(x_0) = 0 \), то игрок (A) находится в критическом положении, ибо его противник может обеспечить себе выигрыш или ничью. Напротив, если \( g(x_0) \neq 0 \), то игрок (A) сам обеспечивает себе выигрыш или ничью, выбирая такую вершину \( x_1 \), что \( g(x_1) = 0 \).

\textbf{Следствие.} Если граф прогрессивно конечен, то существует одна и только одна функция Гранди \( g(x) \), каждый выбор такой вершины \( y \), для которой \( g(y) = 0 \), является выигрышным, а каждый выбор такой вершины \( z \), что \( g(z) \neq 0 \), — проигрышным. (Непосредственно)



%\subsection{Транспортные сети}
%\subsection{Теорема Кёнига-Холла}
%\subsection{Приложения к матрицам}
%\subsection{Бистохастические матрицы}
%\subsection{Теорема Биркгофа - фон Неймана}

\end{document}