\subsection*{Тезисы из статьи "ПРОБЛЕМЫ ИСПОЛЬЗОВАНИЯ ИНТЕРАКТИВНЫХ МЕТОДОВ ПРИ  БУЧЕНИИ ШКОЛЬНИКОВ МАТЕМАТИКЕ"}

Активность учащихся важна для успешного обучения.\\

Обучение бывает активным и пассивным в зависимости от уровня познавательной активности учеников. Пассивное обучение организуется в форме «учитель -> ученик», а активное - в форме диалога между учителем и учениками. Такую классификацию методов обучения предложил В. А. Беловолов.\\

Педагоги и психологи используют интерактивные методы обучения, чтобы создавать ситуации для активного взаимодействия учащихся друг с другом (работа в группах). Это помогает школьникам получать знания и умения.\\

Цель исследования "Интерактивные методы обучения математике в базовой школе" - изучить теоретические основы разработанных интерактивных методов обучения, выявить проблемы и обосновать оптимальные пути использования интерактивных методов при обучении математике, а также разработать и апробировать авторские методы.\\

Анализ литературы показал, что интерактивные методы, предложенные для обучения педагогике и психологии (С.С. Кашлев), могут быть эффективно применены в обучении математике.\\

Исследование выявило основные проблемы использования интерактивных методов при обучении математике.\\
\begin{enumerate}
    \item В школе преподается математика с использованием объяснительных и иллюстративных методов, которые направлены на развитие умений использования математических знаний при решении типовых задач. Эти методы уже стали устоявшимися и применяются регулярно.
    \item Учителя математики слабо знают теоретические основы интерактивных методов обучения (это показало проведенное нами анкетирование).
    \item Учителя часто не готовы использовать интерактивные методики в математике на уроках и в дополнительной работе из-за того, что они не дают быстрых результатов. Однако такие методы могут быть эффективны для формирования качественных знаний и личностных качеств учеников, что имеет значение для их будущей профессиональной деятельности.
    \item Подготовка к интерактивным урокам требует учителям больших усилий и умения руководить учащимися в нестандартных ситуациях.
    \item Школы имеют ограниченные финансовые возможности, не всегда могут использовать интерактивные доски и проводить уроки с использованием компьютеров.

\end{enumerate}

Несмотря на перечисленные проблемы, в учителя всё же используют интерактивные методы, но не так часто, как хотелось бы.
\newpage

\subsection*{Практическое задание}
1. Предложите перечень вопросов для учащихся с целью активизации их.
деятельности по заданной теме.
\subsubsection*{Тема "Комбинаторика"}

\begin{enumerate}
    \item Давайте здороваться, т.е. все пожмем друг другу руки. Рядом сидящим пожмем руку, а с остальными
    будем здороваться мысленным рукопожатием.\\
    - В классе нас сколько?\\
    Вопрос: Сколько было всего рукопожатий?\\
    - Итак, какие будут ответы?\\
    Допустим нас 25.\\
    Каждый из 25-и человек пожал руки 24-м. Однако произведение $25 * 24 = 600$ дает удвоенное число
    рукопожатий (так как в этом расчете учтено, что первый пожал руку второму, а затем второй первому, на самом же деле было одно рукопожатие). Итак, число рукопожатий равно: $(25 * 24): 2 = 300$.\\
    \item Давайте предствим, что мы переходим по мосту, по которому может перейти только один человек за раз.\\
    -***, кто пойдёт первый? \\
    -Сколько вариантов у тебя есть?\\
    20\\
    -Сколько осталось одноклассников на той стороне моста?\\
    19.\\
    Тогда мы все можем перейти по мосту $20\cdot19\cdot18\cdot \ldots \cdot1$ вариантами. При этом каждый пройдёт по мосту ТОЛЬКО ОДИН РАЗ.\\
    \item В коробке сейчас находятся карандаши, мы положили туда : три красных, 6 зелёных, и 1 чёрный.\\
    -Как вы думаете, какой карандаш я достану из коробки, не глядя?\\
    -Почему зелёный?
    -Какова вероятность, что это действительно будет зелёный?\\
    -Предположим, я достала красный карандаш. Изменится ли вероятность того, что следующим будет зелёный?\\
    Поэтому мы можем посчитать эту вероятность, как



\end{enumerate}

