\documentclass[a4paper, 12pt,fleqn]{extarticle}
\usepackage{fontspec}
\usepackage{polyglossia}
\setmainfont{CMU Serif}
\newfontfamily{\cyrillicfont}{CMU Serif}
\setsansfont{CMU Sans Serif}
\newfontfamily{\cyrillicfontsf}{CMU Sans Serif}
\setmonofont{CMU Typewriter Text}
\newfontfamily{\cyrillicfonttt}{CMU Typewriter Text}
\setdefaultlanguage{russian}
\usepackage[left=1cm,bottom=2cm,top=1cm]{geometry}
%%% Дополнительная работа с математикой
\usepackage{amsfonts,amssymb,amsthm,mathtools} % AMS
\usepackage{amsmath}
\usepackage{icomma}

%% Шрифты
\usepackage{euscript} % Шрифт Евклид
\usepackage{mathrsfs} % Красивый матшрифт

%% Свои команды
\DeclareMathOperator{\sgn}{\mathop{sgn}}


%% Перенос знаков в формулах по Львовскому)
\newcommand*{\hm}[1]{#1\nobreak\discretionary{}
	{\hbox{$\mathsurround=0pt #1$}}{}}
\newcommand{\D}[2]{\frac{\partial #1}{\partial #2}}
\newcommand{\DQ}[2]{\frac{\partial^2 #1}{\partial #2^2}}
\newcommand{\DM}[3]{\frac{\partial^2 #1}{\partial #2 \partial #3}}
\begin{document}
\begin{center}
    \hfill \break
    \large{МИНОБРНАУКИ РОССИИ}\\
    \footnotesize{ФЕДЕРАЛЬНОЕ ГОСУДАРСТВЕННОЕ БЮДЖЕТНОЕ ОБРАЗОВАТЕЛЬНОЕ УЧРЕЖДЕНИЕ}\\ 
    \footnotesize{ВЫСШЕГО ПРОФЕССИОНАЛЬНОГО ОБРАЗОВАНИЯ}\\
    \small{\textbf{«ВОРОНЕЖСКИЙ ГОСУДАРСТВЕННЫЙ УНИВЕРСИТЕТ»}}\\
    \hfill \break
    \normalsize{Математический факультет}\\
     \hfill \break
    \normalsize{Кафедра теории функций и геометрии}\\
    \hfill\break
    \hfill \break
    \hfill \break
    \hfill \break
    \large{Начально-краевая задача для волнового уравнения на геометрическом графе-звезде}\\
    \hfill \break
    \hfill \break
    \hfill \break\
    \hfill \break
    \hfill \break
    \normalsize{Домашняя контрольная работа\\
    \hfill \break
    Направление  010501 Фундаментальные математика и механика\\

    \hfill \break
    }\\
    \hfill \break
    \hfill \break
    \end{center}
    \hfill \break
     
    \normalsize{ 
    \begin{tabular}{cccc}
    Зав.кафедрой & \underline{\hspace{3cm}} &  д.физ.-мат.н.,  проф. & Е.М. Семёнов \\\\
    Обучающийся & \underline{\hspace{3cm}} & &А.С. Суматохина \\\\
    Руководитель & \underline{\hspace{3cm}}&  д.физ.-мат.н.,  проф. & В.Л. Прядиев \\\\
    \end{tabular}
    }\\
    \hfill \break
    \hfill \break
    \hfill \break
    \hfill \break
    \hfill \break
    \hfill \break
    \hfill \break
    \hfill \break
    \hfill \break
    \hfill \break
    \begin{center} Воронеж 2022 \end{center}
    \thispagestyle{empty} % выключаем отображение номера для этой страницы
     
    % КОНЕЦ ТИТУЛЬНОГО ЛИСТА
Условия: $n_1, d_2, d_3$.
$\alpha_1=1,\ \alpha_2=2,\ \alpha_3=2$
\begin{gather}
    \DQ{u_j}{x}(x,t)=\DQ{u_j}{t},\ x\in[0,1],\ t\geq 0,\ j=\overline{1,3}\\
    u_2(1,t)=0, \ t\geq 0 \label{eq:2} \\ 
    u_3(1,t)=0,  \ t\geq 0  \\
    \D{u_1}{x}(0,t)=0, \ t\geq 0\\
    \D{u_1}{x}(0,t)+2\D{u_2}{x}(0,t)+2\D{u_3}{x}(0,t)=0, \ t\geq 0\\
    u_1(0,t)=u_2(0,t)=u_3(0,t), \ t\geq 0 \label{eq:6}\\
    u_j(x,0)=\varphi_j(x), j=\overline{1,3}, \ t\geq 0\\
    \D{u_j}{t}(x,0)=0, \ t\geq 0.
\end{gather}
Решение ищется в виде:
    \begin{equation}\label{eq:9}
        u_j(x,t)=0.5(\tilde{\varphi}_j(x+t)+\tilde{\varphi}_j(x-t)), j=\overline{1,3}.
    \end{equation}
Подставляем в (\ref{eq:9}) уравнения (\ref{eq:2})-(\ref{eq:6})
\[\begin{cases}
    \tilde{\varphi}_2(1+t)+\tilde{\varphi}_2(1-t)=0 \\
    \tilde{\varphi}_3(1+t)+\tilde{\varphi}_3(1-t)=0 \\
    \tilde{\varphi}'_1(1+t)+\tilde{\varphi}'_1(1-t)=0 \\
    \tilde{\varphi}'_1(t)+\tilde{\varphi}'_1(-t)+2(\tilde{\varphi}'_2(t)+\tilde{\varphi}'_2(-t))+2(\tilde{\varphi}'_3(t)+\tilde{\varphi}'_3(-t))=0\\
    \tilde{\varphi}_1(t)+\tilde{\varphi}_1(-t)=\tilde{\varphi}_2(t)+\tilde{\varphi}_2(-t)\\
    \tilde{\varphi}_2(t)+\tilde{\varphi}_2(-t)=\tilde{\varphi}_3(t)+\tilde{\varphi}_3(-t)\\
\end{cases}\]
Проводим замены $\tilde{\varphi}_j(-t)=\psi(1+t)$ и $\tilde{\varphi}_j(1-t)=\psi(t)$, получаем систему:
\[\begin{cases}
    \tilde{\varphi}_2(1+t)+\psi_2(t)=0\\
    \tilde{\varphi}_3(1+t)+\psi_3(t)=0\\
    \tilde{\varphi}'_1(1+t)+\tilde{\varphi}'_1(1-t)=0\\
    \tilde{\varphi}'_1(t)+\tilde{\varphi}'_1(-t)+2(\tilde{\varphi}'_2(t)+\tilde{\varphi}'_2(-t))+2(\tilde{\varphi}'_3(t)+\tilde{\varphi}'_3(-t))=0\\
    \tilde{\varphi}_1(t)+\psi_1(1+t)=\tilde{\varphi}_2(t)+\psi_2(1+t)\\
    \tilde{\varphi}_2(t)+\psi_2(1+t)=\tilde{\varphi}_3(t)+\psi_3(1+t)\\
\end{cases}\]
Интегрируем третью и четвёртую строку:
\[\int_0^t \tilde{\varphi}'_1(1+s)+\tilde{\varphi}'_1(1-s)ds=
\tilde{\varphi}_1(1+t)-\tilde{\varphi}_1(1-t)\]
Отсюда: $\tilde{\varphi}_1(1+t)-\psi_1(t)=0$
\[\int_0^t \tilde{\varphi}'_1(s)+\tilde{\varphi}'_1(-s)+2(\tilde{\varphi}'_2(s)+\tilde{\varphi}'_2(-s))+2(\tilde{\varphi}'_3(s)+\tilde{\varphi}'_3(-s))ds=\]
\[\tilde{\varphi}_1(t)-\tilde{\varphi}_1(-t)+2(\tilde{\varphi}_2(t)-\tilde{\varphi}_2(-t))+2(\tilde{\varphi}_3(t)-\tilde{\varphi}_3(-t))\]
Получаем: $\tilde{\varphi}_1(t)-\psi_1(1+t)+2\tilde{\varphi}_2(t)-2\psi_2(1+t)+2\tilde{\varphi}_3(t)-2\psi_3(1+t)=0$

После всех преобразований систему будет выглядеть:
\[\begin{cases}
    \tilde{\varphi}_2(1+t)+\psi_2(t)=0\\
    \tilde{\varphi}_3(1+t)+\psi_3(t)=0\\
    \tilde{\varphi}_1(1+t)-\psi_1(t)=0\\
    \tilde{\varphi}_1(t)-\psi_1(1+t)+2\tilde{\varphi}_2(t)-2\psi_2(1+t)+2\tilde{\varphi}_3(t)-2\psi_3(1+t)=0\\
    \tilde{\varphi}_1(t)+\psi_1(1+t)=\tilde{\varphi}_2(t)+\psi_2(1+t)\\
    \tilde{\varphi}_2(t)+\psi_2(1+t)=\tilde{\varphi}_3(t)+\psi_3(1+t)\\
\end{cases}\]
Рассмотрим функцию $\eta(t)=(\tilde{\varphi}_1(t),\tilde{\varphi}_2(t),\tilde{\varphi}_3(t),\psi_1(t),\psi_2(t),\psi_3(t))^T$, тогда система принимает вид:
$A\eta(t+1)=B\eta(t)$
\[A=\begin{bmatrix}
    0 & 1 & 0 & 0 & 0 & 0\\
0 & 0 & 1 & 0 & 0 & 0\\
1 & 0 & 0 & 0 & 0 & 0\\
0 & 0 & 0 & -1 & -2 & -2\\
0 & 0 & 0 & 1 & -1 & 0\\
0 & 0 & 0 & 0 & 1 & -1
\end{bmatrix},\ B=\begin{bmatrix}
0 & 0 & 0 & 0 & 1 & 0\\
0 & 0 & 0 & 0 & 0 & 1\\
0 & 0 & 0 & -1 & 0 & 0\\
1 & 2 & 2 & 0 & 0 & 0\\
1 & -1 & 0 & 0 & 0 & 0\\
0 & 1 & -1 & 0 & 0 & 0
\end{bmatrix}\]
Выразим в явной форме $\eta(t+1)$:
\begin{gather*}
    A\eta(t+1)=B\eta(t)\\
    A^{-1}A\eta(t+1)=A^{-1}B\eta(t) \\
    \eta(t+1)=A^{-1}B\eta(t)
\end{gather*}
\[A^{-1}=\begin{bmatrix}
0 & 0 & 1 & 0 & 0 & 0\\
1 & 0 & 0 & 0 & 0 & 0\\
0 & 1 & 0 & 0 & 0 & 0\\
0 & 0 & 0 & -0.2 & 0.8 & 0.4\\
0 & 0 & 0 & -0.2 & -0.2 & 0.4\\
0 & 0 & 0 & -0.2 & -0.2 & -0.6
\end{bmatrix}\]
Пусть $A^{-1}B\eta(t)=C=\eta(t+1)$
\[C=\begin{bmatrix}
0& 0& 0& -1& 0& 0\\ 
0& 0& 0& 0& 1& 0\\ 
0& 0& 0& 0& 0& 1\\ 
0.6& -0.8& -0.8& 0& 0& 0\\ 
-0.4& 0.2& -0.8& 0& 0& 0\\ 
-0.4& -0.8& 0.2& 0& 0& 0
\end{bmatrix}\]
Найдём собственные значения $C$.
\begin{gather*}
    \lambda_1 = -\frac{1 - 2 i}{\sqrt{5}}=\cos(\pi - \arctg(2)) + i \sin(\pi - \arctg(2)),\\
    \lambda_2 = -\frac{1 + 2 i}{\sqrt{5}}=\cos(-\pi + \arctg(2)) + i \sin(-\pi + \arctg(2)),\\
    \lambda_3=-1,\\
    \lambda_4=1,\\
    \lambda_5 = \frac{1 + 2 i}{\sqrt{5}}=\cos(\arctg(2)) + i \sin(\arctg(2)),\\
    \lambda_1 = \frac{1 - 2 i}{\sqrt{5}}=\cos(-\arctg(2)) + i \sin(-\arctg(2)).
\end{gather*}
Собственные значения не имеют периода, поэтому матрица $C$ не будет периодичной. Ранее мы выяснили, что $\eta(t+n)=C^n \eta(t)$, найдём все матрицы $C$ до шестой степени.
\[C^2=
\begin{bmatrix}
    -0.6 & 0.8 & 0.8 & 0 & 0 & 0\\
-0.4 & 0.2 & -0.8 & 0 & 0 & 0\\
-0.4 & -0.8 & 0.2 & 0 & 0 & 0\\
0 & 0 & 0 & -0.6 & -0.8 & -0.8\\
0 & 0 & 0 & 0.4 & 0.2 & -0.8\\
0 & 0 & 0 & 0.4 & -0.8 & 0.2
\end{bmatrix}\]
\[C^3=
\begin{bmatrix}
    0 & 0 & 0 & 0.6 & 0.8 & 0.8\\
    0 & 0 & 0 & 0.4 & 0.2 & -0.8\\
    0 & 0 & 0 & 0.4 & -0.8 & 0.2\\
    0.28 & 0.96 & 0.96 & 0 & 0 & 0\\
    0.48 & 0.36 & -0.64 & 0 & 0 & 0\\
    0.48 & -0.64 & 0.36 & 0 & 0 & 0  
\end{bmatrix}\]
\[C^4=
\begin{bmatrix}
    -0.28 & -0.96 & -0.96 & 0 & 0 & 0\\
    0.48 & 0.36 & -0.64 & 0 & 0 & 0\\
    0.48 & -0.64 & 0.36 & 0 & 0 & 0\\
    0 & 0 & 0 & -0.28 & 0.96 & 0.96\\
    0 & 0 & 0 & -0.48 & 0.36 & -0.64\\
    0 & 0 & 0 & -0.48 & -0.64 & 0.36  
\end{bmatrix}\]
\[C^5=\begin{bmatrix}
    0 & 0 & 0 & 0.28 & -0.96 & -0.96\\
    0 & 0 & 0 & -0.48 & 0.36 & -0.64\\
    0 & 0 & 0 & -0.48 & -0.64 & 0.36\\
    -0.936 & -0.352 & -0.352 & 0 & 0 & 0\\
    -0.176 & 0.968 & -0.032 & 0 & 0 & 0\\
    -0.176 & -0.032 & 0.968 & 0 & 0 & 0 
\end{bmatrix}\]
\[C^6=
\begin{bmatrix}
    0.936 & 0.352 & 0.352 & 0 & 0 & 0\\
-0.176 & 0.968 & -0.032 & 0 & 0 & 0\\
-0.176 & -0.032 & 0.968 & 0 & 0 & 0\\
0 & 0 & 0 & 0.936 & -0.352 & -0.352\\
0 & 0 & 0 & 0.176 & 0.968 & -0.032\\
0 & 0 & 0 & 0.176 & -0.032 & 0.968
\end{bmatrix}\]

Выпишем формулы для $\eta$. Если $x \in [0,1]$, то:
\[\eta(x)=
\begin{bmatrix}
    \tilde{\varphi_1}(x)\\
    \tilde{\varphi_2}(x)\\
    \tilde{\varphi_3}(x)\\
    \tilde{\varphi_1}(1-x)\\
    \tilde{\varphi_2}(1-x)\\
    \tilde{\varphi_3}(1-x)\\
\end{bmatrix}
    \]
    Если $x \in [1,2]$:
    \[C\cdot \eta(x-1)=\begin{bmatrix}
        0& 0& 0& -1& 0& 0\\ 
        0& 0& 0& 0& 1& 0\\ 
        0& 0& 0& 0& 0& 1\\ 
        0.6& -0.8& -0.8& 0& 0& 0\\ 
        -0.4& 0.2& -0.8& 0& 0& 0\\ 
        -0.4& -0.8& 0.2& 0& 0& 0
        \end{bmatrix}\cdot
        \begin{bmatrix}
            \tilde{\varphi_1}(x-1)\\%x
            \tilde{\varphi_2}(x-1)\\%y
            \tilde{\varphi_3}(x-1)\\%z
            \tilde{\varphi_1}(2-x)\\%q
            \tilde{\varphi_2}(2-x)\\%w
            \tilde{\varphi_3}(2-x)\\%r
        \end{bmatrix}
        =\begin{bmatrix}
            -\tilde{\varphi_1}(2-x)\\
            \tilde{\varphi_2}(2-x)\\
            \tilde{\varphi_3}(2-x)\\
            0.6\tilde{\varphi_1}(x-1)-0.8\tilde{\varphi_2}(x-1)-0.8\tilde{\varphi_3}(x-1)\\
            -0.4\tilde{\varphi_1}(x-1)+0.2\tilde{\varphi_2}(x-1)-0.8\tilde{\varphi_3}(x-1)\\
            -0.4\tilde{\varphi_1}(x-1)-0.8\tilde{\varphi_2}(x-1)+0.2\tilde{\varphi_3}(x-1)
        \end{bmatrix}\]
        Если $x \in [2,3]$:
        \[C^2\cdot\eta(x-2)=
        \begin{bmatrix}
            -0.6 & 0.8 & 0.8 & 0 & 0 & 0\\
        -0.4 & 0.2 & -0.8 & 0 & 0 & 0\\
        -0.4 & -0.8 & 0.2 & 0 & 0 & 0\\
        0 & 0 & 0 & -0.6 & -0.8 & -0.8\\
        0 & 0 & 0 & 0.4 & 0.2 & -0.8\\
        0 & 0 & 0 & 0.4 & -0.8 & 0.2
        \end{bmatrix}\cdot 
        \begin{bmatrix}
            \tilde{\varphi_1}(x-2)\\%x
            \tilde{\varphi_2}(x-2)\\%y
            \tilde{\varphi_3}(x-2)\\%z
            \tilde{\varphi_1}(3-x)\\%q
            \tilde{\varphi_2}(3-x)\\%w
            \tilde{\varphi_3}(3-x)\\%r
        \end{bmatrix}=\]
        \[\begin{bmatrix}
            -0.6 \tilde{\varphi_1}(x-2) + 0.8 \tilde{\varphi_2}(x-2) + 0.8 \tilde{\varphi_3}(x-2)\\
            -0.4 \tilde{\varphi_1}(x-2) + 0.2 \tilde{\varphi_2}(x-2) - 0.8 \tilde{\varphi_3}(x-2)\\
            -0.4 \tilde{\varphi_1}(x-2) - 0.8 \tilde{\varphi_2}(x-2) + 0.2 \tilde{\varphi_3}(x-2)\\
            -0.6 \tilde{\varphi_1}(3-x)  - 0.8 \tilde{\varphi_2}(3-x)- 0.8 \tilde{\varphi_3}(3-x)\\
            0.4 \tilde{\varphi_1}(3-x) + 0.2 \tilde{\varphi_2}(3-x) - 0.8 \tilde{\varphi_3}(3-x)\\
            0.4 \tilde{\varphi_1}(3-x)  - 0.8 \tilde{\varphi_2}(3-x)+ 0.2 \tilde{\varphi_3}(3-x)
        \end{bmatrix}
        \]
        Если $x \in [3,4]$:
        \[C^3\cdot\eta(x-3)=
        \begin{bmatrix}
            0 & 0 & 0 & 0.6 & 0.8 & 0.8\\
            0 & 0 & 0 & 0.4 & 0.2 & -0.8\\
            0 & 0 & 0 & 0.4 & -0.8 & 0.2\\
            0.28 & 0.96 & 0.96 & 0 & 0 & 0\\
            0.48 & 0.36 & -0.64 & 0 & 0 & 0\\
            0.48 & -0.64 & 0.36 & 0 & 0 & 0  
        \end{bmatrix}
        \begin{bmatrix}
            \tilde{\varphi_1}(x-3)\\%x
            \tilde{\varphi_2}(x-3)\\%y
            \tilde{\varphi_3}(x-3)\\%z
            \tilde{\varphi_1}(4-x)\\%q
            \tilde{\varphi_2}(4-x)\\%w
            \tilde{\varphi_3}(4-x)\\%r
        \end{bmatrix}=\]
        \[
            \begin{bmatrix}
                0.6 \tilde{\varphi_1}(4-x)  + 0.8 \tilde{\varphi_2}(4-x)+ 0.8 \tilde{\varphi_3}(4-x) \\
                0.4 \tilde{\varphi_1}(4-x) + 0.2 \tilde{\varphi_2}(4-x) - 0.8 \tilde{\varphi_3}(4-x) \\
                0.4 \tilde{\varphi_1}(4-x) - 0.8 \tilde{\varphi_2}(4-x) + 0.2 \tilde{\varphi_3}(4-x) \\
                0.28 \tilde{\varphi_1}(x-3) + 0.96 \tilde{\varphi_2}(x-3) + 0.96 \tilde{\varphi_3}(x-3) \\
                0.48 \tilde{\varphi_1}(x-3) + 0.36 \tilde{\varphi_2}(x-3) - 0.64 \tilde{\varphi_3}(x-3) \\
                0.48 \tilde{\varphi_1}(x-3) - 0.64 \tilde{\varphi_2}(x-3) + 0.36 \tilde{\varphi_3}(x-3) \\
            \end{bmatrix}
            \]
            Если $x \in [4,5]$:
            \[C^4\cdot\eta(x+3)=
            \begin{bmatrix}
                -0.28 & -0.96 & -0.96 & 0 & 0 & 0\\
                0.48 & 0.36 & -0.64 & 0 & 0 & 0\\
                0.48 & -0.64 & 0.36 & 0 & 0 & 0\\
                0 & 0 & 0 & -0.28 & 0.96 & 0.96\\
                0 & 0 & 0 & -0.48 & 0.36 & -0.64\\
                0 & 0 & 0 & -0.48 & -0.64 & 0.36  
            \end{bmatrix}
            \begin{bmatrix}
                \tilde{\varphi_1}(x+3)\\%x
                \tilde{\varphi_2}(x+3)\\%y
                \tilde{\varphi_3}(x+3)\\%z
                \tilde{\varphi_1}(5-x)\\%q
                \tilde{\varphi_2}(5-x)\\%w
                \tilde{\varphi_3}(5-x)\\%r
            \end{bmatrix}=
            \]
        \[
            \begin{bmatrix}
            -0.28 \tilde{\varphi_1}(x+3) - 0.96 \tilde{\varphi_2}(x+3) - 0.96 \tilde{\varphi_3}(x+3) \\
            0.48 \tilde{\varphi_1}(x+3) + 0.36 \tilde{\varphi_2}(x+3) - 0.64 \tilde{\varphi_3}(x+3) \\
            0.48 \tilde{\varphi_1}(x+3) - 0.64 \tilde{\varphi_2}(x+3) + 0.36 \tilde{\varphi_3}(x+3) \\
            -0.28 \tilde{\varphi_1}(5-x) + 0.96 \tilde{\varphi_3}(5-x) + 0.96 \tilde{\varphi_2}(5-x) \\
            -0.48 \tilde{\varphi_1}(5-x) - 0.64 \tilde{\varphi_3}(5-x) + 0.36 \tilde{\varphi_2}(5-x) \\
            -0.48 \tilde{\varphi_1}(5-x) + 0.36 \tilde{\varphi_3}(5-x) - 0.64 \tilde{\varphi_2}(5-x) \\
            \end{bmatrix}
            \]
            Если $x \in [5,6]$:
            \[C^5\cdot\eta(x-5)=
            \begin{bmatrix}
                    0 & 0 & 0 & 0.28 & -0.96 & -0.96\\
                    0 & 0 & 0 & -0.48 & 0.36 & -0.64\\
                    0 & 0 & 0 & -0.48 & -0.64 & 0.36\\
                    -0.936 & -0.352 & -0.352 & 0 & 0 & 0\\
                    -0.176 & 0.968 & -0.032 & 0 & 0 & 0\\
                    -0.176 & -0.032 & 0.968 & 0 & 0 & 0  
            \end{bmatrix}
            \begin{bmatrix}
                \tilde{\varphi_1}(x-5)\\%x
                \tilde{\varphi_2}(x-5)\\%y
                \tilde{\varphi_3}(x-5)\\%z
                \tilde{\varphi_1}(x+4)\\%q
                \tilde{\varphi_2}(x+4)\\%w
                \tilde{\varphi_3}(x+4)\\%r
            \end{bmatrix}=
            \]
            
            \[
            \begin{bmatrix}
                0.28 \tilde{\varphi_1}(x+4) - 0.96 \tilde{\varphi_3}(x+4) - 0.96 \tilde{\varphi_2}(x+4) \\
                -0.48 \tilde{\varphi_1}(x+4) - 0.64 \tilde{\varphi_3}(x+4) + 0.36 \tilde{\varphi_2}(x+4) \\
                -0.48 \tilde{\varphi_1}(x+4) + 0.36 \tilde{\varphi_3}(x+4) - 0.64 \tilde{\varphi_2}(x+4) \\
                -0.936 \tilde{\varphi_1}(x-5) - 0.352 \tilde{\varphi_2}(x-5) - 0.352 \tilde{\varphi_3}(x-5) \\
                -0.176 \tilde{\varphi_1}(x-5) + 0.968 \tilde{\varphi_2}(x-5) - 0.032 \tilde{\varphi_3}(x-5) \\
                -0.176 \tilde{\varphi_1}(x-5) - 0.032 \tilde{\varphi_2}(x-5) + 0.968 \tilde{\varphi_3}(x-5) +\\
            \end{bmatrix}
            \]
            Если $x \in [6;7]$:
            \[C^6\eta(x-6)=
\begin{bmatrix}
    0.936 & 0.352 & 0.352 & 0 & 0 & 0\\
-0.176 & 0.968 & -0.032 & 0 & 0 & 0\\
-0.176 & -0.032 & 0.968 & 0 & 0 & 0\\
0 & 0 & 0 & 0.936 & -0.352 & -0.352\\
0 & 0 & 0 & 0.176 & 0.968 & -0.032\\
0 & 0 & 0 & 0.176 & -0.032 & 0.968
\end{bmatrix}
\begin{bmatrix}
    \tilde{\varphi_1}(x-6)\\%x
    \tilde{\varphi_2}(x-6)\\%y
    \tilde{\varphi_3}(x-6)\\%z
    \tilde{\varphi_1}(7-x)\\%q
    \tilde{\varphi_2}(7-x)\\%w
    \tilde{\varphi_3}(7-x)\\%r
\end{bmatrix}=\]
\[
\begin{bmatrix}
    0.936 \tilde{\varphi_1}(x-6) + 0.352 \tilde{\varphi_2}(x-6) + 0.352 \tilde{\varphi_3}(x-6)\\
    -0.176 \tilde{\varphi_1}(x-6) + 0.968 \tilde{\varphi_2}(x-6) - 0.032 \tilde{\varphi_3}(x-6) \\
    -0.176 \tilde{\varphi_1}(x-6) - 0.032 \tilde{\varphi_2}(x-6) + 0.968 \tilde{\varphi_3}(x-6) \\
    0.936 \tilde{\varphi_1}(7-x) - 0.352 \tilde{\varphi_2}(7-x) - 0.352 \tilde{\varphi_3}(7-x) \\
    0.176 \tilde{\varphi_1}(7-x) + 0.968 \tilde{\varphi_2}(7-x) - 0.032 \tilde{\varphi_3}(7-x) \\
0.176 \tilde{\varphi_1}(7-x) - 0.032 \tilde{\varphi_2}(7-x) + 0.968 \tilde{\varphi_3}(7-x) \\
\end{bmatrix}\]

Если $x \in [7;8]$:
\[C^7\eta(x-7)=
\begin{bmatrix}
    \tilde{\varphi_1}(x-7)\\%x
    \tilde{\varphi_2}(x-7)\\%y
    \tilde{\varphi_3}(x-7)\\%z
    \tilde{\varphi_1}(8-x)\\%q
    \tilde{\varphi_2}(8-x)\\%w
    \tilde{\varphi_3}(8-x)\\%r
\end{bmatrix}
\begin{bmatrix}
0 & 0 & 0 & -0.936 & 0.352 & 0.352\\
0 & 0 & 0 & 0.176 & 0.968 & -0.032\\
0 & 0 & 0 & 0.176 & -0.032 & 0.968\\
0.8432 & -0.5376 & -0.5376 & 0 & 0 & 0\\
-0.2688 & 0.0784 & -0.9216 & 0 & 0 & 0\\
-0.2688 & -0.9216 & 0.0784 & 0 & 0 & 0\\
\end{bmatrix}=\]
\[
    \begin{bmatrix}
        -0.936 \tilde{\varphi_1}(8-x) + 0.352 \tilde{\varphi_2}(8-x) + 0.352 \tilde{\varphi_3}(8-x)\\
        0.176 \tilde{\varphi_1}(8-x) + 0.968 \tilde{\varphi_2}(8-x) - 0.032 \tilde{\varphi_3}(8-x)\\
        0.176 \tilde{\varphi_1}(8-x) - 0.032 \tilde{\varphi_2}(8-x) + 0.968 \tilde{\varphi_3}(8-x)\\
        0.8432 \tilde{\varphi_1}(x-7) - 0.5376 \tilde{\varphi_2}(x-7) - 0.5376 \tilde{\varphi_3}(x-7)\\
        -0.2688 \tilde{\varphi_1}(x-7) + 0.0784 \tilde{\varphi_2}(x-7) - 0.9216 \tilde{\varphi_3}(x-7)\\
        -0.2688 \tilde{\varphi_1}(x-7) - 0.9216 \tilde{\varphi_2}(x-7) + 0.0784 \tilde{\varphi_3}(x-7)\\
        \end{bmatrix}\]
Выписываем формулы для $\tilde{\varphi}=(\tilde{\varphi_1}, \tilde{\varphi_2}, \tilde{\varphi_3})^T$
\[
    \tilde{\varphi_1}=
    \begin{cases}
        -0.936 \tilde{\varphi_1}(x+6) + 0.352 \tilde{\varphi_2}(x+6) + 0.352 \tilde{\varphi_3}(x+6)&x\in[-6;-5]\\
        0.936 \tilde{\varphi_1}(x+5) + 0.352 \tilde{\varphi_2}(x+5) + 0.352 \tilde{\varphi_3}(x+5)&x\in[-5;-4]\\
        0.28 \tilde{\varphi_1}(x+4) - 0.96 \tilde{\varphi_3}(x+4) - 0.96 \tilde{\varphi_2}(x+4)&x\in[-4;-3] \\
        -0.28 \tilde{\varphi_1}(x+3) - 0.96 \tilde{\varphi_2}(x+3) - 0.96 \tilde{\varphi_3}(x+3)&x\in[-3;-2] \\
        0.6 \tilde{\varphi_1}(x+2)  + 0.8 \tilde{\varphi_2}(x+2)+ 0.8 \tilde{\varphi_3}(x+2)&x\in[-2;-1]\\
        -0.6 \tilde{\varphi_1}(x+1) + 0.8 \tilde{\varphi_2}(x+1) + 0.8 \tilde{\varphi_3}(x+1)&x\in[-1;0]\\
        \tilde{\varphi_1}(x)&x\in[0,1]\\
        -\tilde{\varphi_1}(2-x)&x\in[1,2]\\
        -0.6 \tilde{\varphi_1}(x-2) + 0.8 \tilde{\varphi_2}(x-2) + 0.8 \tilde{\varphi_3}(x-2)&x\in[2,3]\\
        0.6 \tilde{\varphi_1}(4-x)  + 0.8 \tilde{\varphi_2}(4-x) + 0.8 \tilde{\varphi_3}(4-x)&x\in[3,4]\\
        -0.28 \tilde{\varphi_1}(x-4) - 0.96 \tilde{\varphi_2}(x-4) - 0.96 \tilde{\varphi_3}(x-4)&x\in[4,5] \\
        0.28 \tilde{\varphi_1}(6-x) - 0.96 \tilde{\varphi_3}(6-x) - 0.96 \tilde{\varphi_2}(6-x)&x\in[5,6] \\
        0.936 \tilde{\varphi_1}(x-6) + 0.352 \tilde{\varphi_2}(x-6) + 0.352 \tilde{\varphi_3}(x-6)&x\in[6,7]\\
        -0.936 \tilde{\varphi_1}(8-x) + 0.352 \tilde{\varphi_2}(8-x) + 0.352 \tilde{\varphi_3}(8-x)&x\in[7,8]
    \end{cases}\]
    \[
        \tilde{\varphi_2}=
        \begin{cases}
           -0.176 \tilde{\varphi_1}(x+6) - 0.968 \tilde{\varphi_2}(x+6) + 0.032 \tilde{\varphi_3}(x+6)&x\in[-6;-5]\\
           0.176 \tilde{\varphi_1}(x+5) - 0.968 \tilde{\varphi_2}(x+5) + 0.032 \tilde{\varphi_3}(x+5)&x\in[-5;-4]\\
           0.48 \tilde{\varphi_1}(x+4) + 0.64 \tilde{\varphi_3}(x+4) - 0.36 \tilde{\varphi_2}(x+4)&x\in[-4;-3] \\
           -0.48 \tilde{\varphi_1}(x+3) - 0.36 \tilde{\varphi_2}(x+3) + 0.64 \tilde{\varphi_3}(x+3)&x\in[-3;-2] \\
           -0.4 \tilde{\varphi_1}(x+2) - 0.2 \tilde{\varphi_2}(x+2) + 0.8 \tilde{\varphi_3}(x+2)&x\in[-2;-1]\\
           0.4 \tilde{\varphi_1}(x+1) - 0.2 \tilde{\varphi_2}(x+1) + 0.8 \tilde{\varphi_3}(x+1)&x\in[-1;0]\\
           \tilde{\varphi_2}(x)&x\in[0,1]\\
           \tilde{\varphi_2}(2-x)&x\in[1,2]\\
           -0.4 \tilde{\varphi_1}(x-2) + 0.2 \tilde{\varphi_2}(x-2) - 0.8 \tilde{\varphi_3}(x-2)&x\in[2,3]\\
           0.4 \tilde{\varphi_1}(4-x) + 0.2 \tilde{\varphi_2}(4-x) - 0.8 \tilde{\varphi_3}(4-x) &x\in[3,4]\\
           0.48 \tilde{\varphi_1}(x-4) + 0.36 \tilde{\varphi_2}(x-4) - 0.64 \tilde{\varphi_3}(x-4)&x\in[4,5] \\
           -0.48 \tilde{\varphi_1}(6-x) - 0.64 \tilde{\varphi_3}(6-x) + 0.36 \tilde{\varphi_2}(6-x)&x\in[5,6] \\
           -0.176 \tilde{\varphi_1}(x-6) + 0.968 \tilde{\varphi_2}(x-6) - 0.032 \tilde{\varphi_3}(x-6)&x\in[6,7]\\
           0.176 \tilde{\varphi_1}(8-x) + 0.968 \tilde{\varphi_2}(8-x) - 0.032 \tilde{\varphi_3}(8-x)&x\in[7,8]\\
        \end{cases}\]
        \[
            \tilde{\varphi_3}=
            \begin{cases}
               -0.176 \tilde{\varphi_1}(x+6) + 0.032 \tilde{\varphi_2}(x+6) - 0.968 \tilde{\varphi_3}(x+6)&x\in[-6;-5]\\
               0.176 \tilde{\varphi_1}(x+5) + 0.032 \tilde{\varphi_2}(x+) - 0.968 \tilde{\varphi_3}(x+)&x\in[-5;-4]\\
               0.48 \tilde{\varphi_1}(x+4) - 0.36 \tilde{\varphi_3}(x+4) + 0.64 \tilde{\varphi_2}(x+4)&x\in[-4;-3] \\
               -0.48 \tilde{\varphi_1}(x+3) + 0.64 \tilde{\varphi_2}(x+3) - 0.36 \tilde{\varphi_3}(x+3)&x\in[-3;-2] \\
               -0.4 \tilde{\varphi_1}(x+2) + 0.8 \tilde{\varphi_2}(x+2) - 0.2 \tilde{\varphi_3}(x+2)&x\in[-2;-1]\\
               0.4 \tilde{\varphi_1}(x+1) + 0.8 \tilde{\varphi_2}(x+1) - 0.2 \tilde{\varphi_3}(x+1)&x\in[-1;0]\\
               \tilde{\varphi_3}(x)&x\in[0,1]\\
               \tilde{\varphi_3}(2-x)&x\in[1,2]\\
               -0.4 \tilde{\varphi_1}(x-2) - 0.8 \tilde{\varphi_2}(x-2) + 0.2 \tilde{\varphi_3}(x-2)&x\in[2,3]\\
               0.4 \tilde{\varphi_1}(4-x) - 0.8 \tilde{\varphi_2}(4-x) + 0.2 \tilde{\varphi_3}(4-x)&x\in[3,4]\\
               0.48 \tilde{\varphi_1}(x-4) - 0.64 \tilde{\varphi_2}(x-4) + 0.36 \tilde{\varphi_3}(x-4)&x\in[4,5] \\
               -0.48 \tilde{\varphi_1}(6-x) + 0.36 \tilde{\varphi_3}(6-x) - 0.64 \tilde{\varphi_2}(6-x)&x\in[5,6] \\
               -0.176 \tilde{\varphi_1}(x-6) - 0.032 \tilde{\varphi_2}(x-6) + 0.968 \tilde{\varphi_3}(x-6)&x\in[6,7]\\
               0.176 \tilde{\varphi_1}(8-x) - 0.032 \tilde{\varphi_2}(8-x) + 0.968 \tilde{\varphi_3}(8-x)&x\in[7,8]\\
            \end{cases}\]
Подставим в общие формулы данные нам $\varphi_j,\ j=\overline{1, 3}$:
\[
    \tilde{\varphi_1}=
    \begin{cases}
        0.936 \tilde{\varphi_1}(x+6) &x\in[-6;-5]\\
        -0.936 \tilde{\varphi_1}(x+5)&x\in[-5;-4]\\
        -0.28 \tilde{\varphi_1}(x+4)&x\in[-4;-3] \\
        0.28 \tilde{\varphi_1}(x+3)&x\in[-3;-2] \\
        -0.6 \tilde{\varphi_1}(x+2)&x\in[-2;-1]\\
        0.6 \tilde{\varphi_1}(x+1)&x\in[-1;0]\\
        \tilde{\varphi_1}(x)&x\in[0,1]\\
        -\tilde{\varphi_1}(2-x)&x\in[1,2]\\
        -0.6 \tilde{\varphi_1}(x-2)&x\in[2,3]\\
        0.6 \tilde{\varphi_1}(4-x)&x\in[3,4]\\
        -0.28 \tilde{\varphi_1}(x-4)&x\in[4,5] \\
        0.28 \tilde{\varphi_1}(6-x)&x\in[5,6] \\
        0.936 \tilde{\varphi_1}(x-6)&x\in[6,7]\\
        -0.936 \tilde{\varphi_1}(8-x)&x\in[7,8]
    \end{cases}\]
    \[
        =
        \begin{cases}
           -0.176 \tilde{\varphi_1}(x+6) &x\in[-6;-5]\\
           0.176 \tilde{\varphi_1}(x+5)&x\in[-5;-4]\\
           0.48 \tilde{\varphi_1}(x+4) &x\in[-4;-3] \\
           -0.48 \tilde{\varphi_1}(x+3)&x\in[-3;-2] \\
           -0.4 \tilde{\varphi_1}(x+2) &x\in[-2;-1]\\
           0.4 \tilde{\varphi_1}(x+1) &x\in[-1;0]\\
           0&x\in[0,1]\\
           0&x\in[1,2]\\
           -0.4 \tilde{\varphi_1}(x-2)&x\in[2,3]\\
           0.4 \tilde{\varphi_1}(4-x)&x\in[3,4]\\
           0.48 \tilde{\varphi_1}(x-4)&x\in[4,5] \\
           -0.48 \tilde{\varphi_1}(6-x)&x\in[5,6] \\
           -0.176 \tilde{\varphi_1}(x-6)&x\in[6,7]\\
           0.176 \tilde{\varphi_1}(8-x)&x\in[7,8]\\
        \end{cases}\]
        \[
            =
            \begin{cases}
               -0.176 \tilde{\varphi_1}(x+6)&x\in[-6;-5]\\
               0.176 \tilde{\varphi_1}(x+5)&x\in[-5;-4]\\
               0.48 \tilde{\varphi_1}(x+4)&x\in[-4;-3] \\
               -0.48 \tilde{\varphi_1}(x+3)&x\in[-3;-2] \\
               -0.4 \tilde{\varphi_1}(x+2)&x\in[-2;-1]\\
               0.4 \tilde{\varphi_1}(x+1)&x\in[-1;0]\\
               0&x\in[0,1]\\
               0&x\in[1,2]\\
               -0.4 \tilde{\varphi_1}(x-2)&x\in[2,3]\\
               0.4 \tilde{\varphi_1}(4-x)&x\in[3,4]\\
               0.48 \tilde{\varphi_1}(x-4)&x\in[4,5] \\
               -0.48 \tilde{\varphi_1}(6-x)&x\in[5,6] \\
               -0.176 \tilde{\varphi_1}(x-6)&x\in[6,7]\\
               0.176 \tilde{\varphi_1}(8-x)&x\in[7,8]\\
            \end{cases}\]

        После подстановки функции $\tilde{\varphi_2}$ и $\tilde{\varphi_3}$ совпали.
        \newpage    

        \begin{figure}
            \caption{        График $\tilde{\varphi_1}(x)$ на отрезке $x\in [-6;8]$   }
            \includegraphics[width=\linewidth]{pictures/u_1.pdf}
        \end{figure}
        \begin{figure}
            \caption{        График $\tilde{\varphi_2}(x)$ на отрезке $x\in [-6;8]$   }
            \includegraphics[width=\linewidth]{pictures/u_2(x).pdf}
        \end{figure}
        \begin{figure}
            \caption{График $u_1=(\tilde{\varphi_1}(x+t)+\tilde{\varphi_1}(x-t))$ при $t=0.5$}
            \includegraphics[width=\linewidth]{pictures/ДКР №2(fi1,t=0.5).pdf}
        \end{figure}
        \begin{figure}
            \caption{График $u_1=(\tilde{\varphi_1}(x+t)+\tilde{\varphi_1}(x-t))$ при $t=1$}
            \includegraphics[width=\linewidth]{pictures/ДКР №2(fi1,t=1).pdf}
        \end{figure}
        \begin{figure}
            \caption{График $u_1=(\tilde{\varphi_1}(x+t)+\tilde{\varphi_1}(x-t))$ при $t=2$}
            \includegraphics[width=\linewidth,height=12cm]{pictures/ДКР №2(fi1,t=2).pdf}
        \end{figure}
        \begin{figure}
            \caption{График $u_1=(\tilde{\varphi_1}(x+t)+\tilde{\varphi_1}(x-t))$ при $t=3$}
            \includegraphics[width=\linewidth,height=12cm]{pictures/ДКР №2(fi1,t=3).pdf}
        \end{figure}
        \begin{figure}
            \caption{График $u_1=(\tilde{\varphi_1}(x+t)+\tilde{\varphi_1}(x-t))$ при $t=5$}
            \includegraphics[width=\linewidth,height=12cm]{pictures/ДКР №2(fi1,t=5).pdf}
        \end{figure}
        \begin{figure}
            \caption{График $u_1=(\tilde{\varphi_1}(x+t)+\tilde{\varphi_1}(x-t))$ при $t=6$}
            \includegraphics[width=\linewidth]{pictures/ДКР №2(fi1,t=6).pdf}
        \end{figure}

        \begin{figure}
            \caption{График $u_2=(\tilde{\varphi_2}(x+t)+\tilde{\varphi_2}(x-t))$ при $t=0.5$}
            \includegraphics[width=\linewidth]{pictures/ДКР №2(fi2,t=0.5).pdf}
        \end{figure}
        \begin{figure}
            \caption{График $u_2=(\tilde{\varphi_2}(x+t)+\tilde{\varphi_2}(x-t))$ при $t=1$}
            \includegraphics[width=\linewidth]{pictures/ДКР №2(fi2,t=1).pdf}
        \end{figure}
        \begin{figure}
            \caption{График $u_2=(\tilde{\varphi_2}(x+t)+\tilde{\varphi_2}(x-t))$ при $t=2$}
            \includegraphics[width=\linewidth,height=12cm]{pictures/ДКР №2(fi2,t=2).pdf}
        \end{figure}
        \begin{figure}
            \caption{График $u_2=(\tilde{\varphi_2}(x+t)+\tilde{\varphi_2}(x-t))$ при $t=3$}
            \includegraphics[width=\linewidth,height=12cm]{pictures/ДКР №2(fi2,t=3).pdf}
        \end{figure}
        \begin{figure}
            \caption{График $u_2=(\tilde{\varphi_2}(x+t)+\tilde{\varphi_2}(x-t))$ при $t=5$}
            \includegraphics[width=\linewidth,height=12cm]{pictures/ДКР №2(fi2,t=5).pdf}
        \end{figure}
        \begin{figure}
            \caption{График $u_2=(\tilde{\varphi_2}(x+t)+\tilde{\varphi_2}(x-t))$ при $t=6$}
            \includegraphics[width=\linewidth]{pictures/ДКР №2(fi2,t=6).pdf}
        \end{figure}
        %%Если(-6<x<-5,-0.176q(x+6),Если(-5<x<-4,0.176q(x+5),Если(-4<x<-3,0.48q(x+4),Если(-3<x<-2,-0.48q(x+3),Если(-2<x<-1,-0.4q(x+2),Если(-1<x<0,0.4q(x+1),Если(0<x<2,0,Если(2<x<3,-0.4q(x-2),Если(3<x<4,0.4q(4-x),Если(4<x<5,0.48q(x-4),Если(5<x<6,-0.48q(6-x),Если(6<x<7,-0.176q(x-6),Если(7<x<8,0.176q(8-x))))))))))))))
\end{document}