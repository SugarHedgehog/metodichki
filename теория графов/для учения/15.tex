\subsection{Ориентированные графы, порядковая функция}

\begin{itemize}
    \item \textbf{Ориентированный граф (орграф)} $D$: состоит из множества вершин $V$ и набора упорядоченных пар $X$ (дуг).
    \item \textbf{Направленный граф}: орграф без симметричных пар дуг (например, $(u, v)$ и $(v, u)$).
    \item \textbf{Помеченный граф}: граф, где вершины имеют уникальные метки (например, $v_1, v_2, \ldots$).
\end{itemize}

\textbf{Порядковая функция}
\begin{itemize}
    \item \textbf{Порядковое число}: обобщение целого числа, используется для обозначения позиций в упорядоченных множествах.
    \item \textbf{Порядковое число первого рода}: например, $1, 2, \ldots, \omega + 3$.
    \item \textbf{Порядковое число второго рода}: например, $\omega$ (предельное число).
    \item \textbf{Порядковая функция графа}: функция $o(x)$, определяющая порядок вершины $x$.
\end{itemize}

\textbf{Теоремы и Примеры}

\textbf{Пример использования порядковых чисел}
\begin{itemize}
    \item Для графа $G = (X, \Gamma)$ определяются множества $X(0), X(1), \ldots, X(\alpha)$, где $\alpha$ — порядковое число.
    \item \textbf{Порядок вершины} $x$: наименьшее $\alpha$, для которого $x \in X(\alpha)$.
\end{itemize}

\textbf{Теорема о порядковой функции}
\begin{itemize}
    \item \textbf{Теорема 4}: Порядковая функция определена на всем $X$ тогда и только тогда, когда граф прогрессивно конечен.
    \item \textbf{Доказательство}: 
    \begin{enumerate}
        \item Если $\theta(x)$ существует для всех $x \in X$, граф прогрессивно конечен.
        \item Если существует вершина без порядка, граф не прогрессивно конечен.
    \end{enumerate}
\end{itemize}