\subsection{Диаметр и радиус графа}

\noindent\textbf{Расстояние} $d(u,v)$ между вершинами (длина кратчайшей простой цепи):
\[
d(u,v) = \begin{cases} 
\text{длина кратчайшей }(u\text{-}v)\text{-цепи}, & \text{если вершины соединены} \\
\infty, & \text{если вершины не соединены}
\end{cases}
\]

\noindent\textbf{Свойства метрики} (для связного графа):
\begin{enumerate}[noitemsep,topsep=0pt]
    \item $d(u,v) \geq 0$; $d(u,v) = 0 \Leftrightarrow u = v$ (неотрицательность)
    \item $d(u,v) = d(v,u)$ (симметричность)
    \item $d(u,v) + d(v,w) \geq d(u,w)$ (неравенство треугольника)
\end{enumerate}

\noindent\textbf{Термины:}
\begin{itemize}[noitemsep,topsep=0pt]
    \item \textit{Геодезическая} -- кратчайшая простая $(u\text{-}v)$-цепь
    \item \textit{Диаметр графа} $d(G)$ -- длина самой длинной геодезической
\end{itemize}

\noindent\textbf{Степени графа:} 
Для графа $G$ определяется $G^k$ ($k$-я степень):
\begin{itemize}[noitemsep,topsep=0pt]
    \item $V(G^k) = V(G)$ (те же вершины)
    \item Вершины $u,v$ смежны в $G^k \Leftrightarrow d(u,v) \leq k$ в $G$
\end{itemize}

\noindent\textit{Примеры:} $C_5^2 = K_5$, $P_4^2 = K_1 + K_3$