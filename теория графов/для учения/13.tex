\textbf{Хроматическое число плоского графа}

\noindent\textbf{Основные утверждения:}
\begin{itemize}
    \item \(\chi(H) \leq \chi(G) + 1\) и \(\overline{\chi}(H) \leq \overline{\chi}(G) + 1\).
    \item Если \(\chi(H) < \chi(G) + 1\) или \(\overline{\chi}(H) < \overline{\chi}(G) + 1\), то \(\chi(H) + \overline{\chi}(H) \leq p + 1\).
    \item Всегда \(\chi(H) + \overline{\chi}(H) \leq p + 1\).
    \item \(\overline{\chi}\chi \leq \left(\frac{p + 1}{2}\right)^2\).
\end{itemize}

\noindent\textbf{Теорема о пяти красках:}
\begin{theorem}
Каждый планарный граф 5-раскрашиваем.
\end{theorem}

\noindent\textbf{Доказательство:} Индукция по числу $p$ вершин.
\begin{itemize}
    \item База: для $p \leq 5$ граф $p$-раскрашиваем.
    \item Шаг: для графа $G$ с $p+1$ вершинами, найдется вершина $v$ степени 5 или менее. Граф $G - v$ 5-раскрашиваем.
    \item Если все пять цветов используются, переставляем цвета, чтобы получить 5-раскраску.
\end{itemize}

\noindent\textbf{Гипотеза четырёх красок:}
\begin{itemize}
    \item Каждая плоская карта 4-раскрашиваема.
    \item Эквивалентно: каждый планарный граф 4-раскрашиваем.
\end{itemize}

\noindent\textbf{Теорема 12.8:}
\begin{theorem}
Каждый планарный граф, имеющий меньше четырех треугольников, 3-раскрашиваем.
\end{theorem}

\noindent\textbf{Следствие 12.8 (a):}
\begin{itemize}
    \item Каждый планарный граф, не содержащий треугольников, 3-раскрашиваем.
\end{itemize}

\noindent\textbf{Теорема 12.9:}
\begin{theorem}
Гипотеза четырех красок справедлива тогда и только тогда, когда каждая кубическая плоская карта, не имеющая мостов, 4-раскрашиваема.
\end{theorem}

\noindent\textbf{Доказательство:}
\begin{itemize}
    \item Любая плоская карта 4-раскрашиваема тогда и только тогда, когда справедлива гипотеза четырех красок.
    \item Если 4-раскрашиваем всякая плоская карта, не содержащая мостов, то и всякая кубическая плоская карта, не содержащая мостов, также 4-раскрашиваема.
\end{itemize}