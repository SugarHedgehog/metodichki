\subsection{Диаметр и радиус графа}
Расстоянием $d(u, v)$ между двумя вершинами $u$ и $v$ графа $G$ называется длина кратчайшей простой цепи, соединяющей их; если $u$ и $v$ не соединены, то полагаем $d(u, v) = \infty$. В связном графе расстояние является метрикой, т. е. удовлетворяет следующим аксиомам (аксиомам метрики): для любых трёх вершин $u$, $v$ и $w$

\begin{enumerate}
    \item \(d(u, v) \geq 0\) и \(d(u, v) = 0\) тогда и только тогда, когда \(u = v\);
    \item \(d(u, v) = d(v, u)\);
    \item \(d(u, v) + d(v, w) \geq d(u, w)\).
\end{enumerate}

Кратчайшая простая $(u-v)$-цепь часто называется геодезической.

Диаметр $d(G)$ связного графа $G$ — это длина самой длинной геодезической. Граф $G$ на рис. 2.9 имеет обхват $g = 3$, окружение $c = 4$ и диаметр $d = 2$.

Квадрат $G^2$ графа $G$ имеет то же множество вершин, что и граф $G$, т. е. $V(G^2) = V(G)$, и две вершины $u$ и $v$ в $G^2$ смежны тогда и только тогда, когда $d(u, v) \leq 2$ в $G$. Степени $G^3$, $G^4$, \ldots графа $G$ определяются аналогично. Например, $C_5^2 = K_5$ и $P_4^2 = K_1 + K_3$.