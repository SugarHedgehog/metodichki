\documentclass[a4paper,12pt]{article}
\usepackage[utf8]{inputenc}
\usepackage[russian]{babel}
\usepackage{geometry}
\geometry{top=10mm, bottom=10mm, left=10mm, right=10mm}
\pagestyle{empty}

\begin{document}
\begin{center}
    \textbf{Муниципальное бюджетное общеобразовательное учреждение}\\
    \textbf{«Средняя общеобразовательная школа № 10» г. Нижняя Салда Свердловской области}\\
    \end{center}
\begin{center}
    \textbf{\large ОТЗЫВ ОБ УЧЕБНО-ВОСПИТАТЕЛЬНОЙ РАБОТЕ}\\
    \textbf{студентки-практикантки 5-го курса математического факультета Воронежского государственного университета }\\
    \textbf{Суматохиной Александры Сергеевны за период с 1 по 28 сентября}
    \end{center}
\vspace{10mm}

\noindent
\textbf{Предмет: Математика}\\
\textbf{Класс: 5А}\\

В период с 2 сентября по 28 сентября 2024 года в нашей школе проходила практика студентки-практикантки 5-го курса математического факультета Суматохиной Александры Сергеевны. Александра работала с учениками 5А класса, занимаясь как учебной, так и воспитательной работой.

Александра Сергеевна показала себя грамотным и инициативным учителем. Она провела несколько уроков математики, на которых учащиеся были активно вовлечены в учебный процесс. В частности, урок по теме "Углы: острые, прямые, тупые; измерение углов с помощью транспортира" был проведен на высоком уровне, что было отмечено учащимися и коллегами.

Одним из значимых мероприятий, организованных Александрой, стал классный час по профилактике детского дорожно-транспортного травматизма. Программа включала презентацию, интерактивные игры и практические задания. Подход Александры к планированию и проведению мероприятия был продуманным и эффективным, что позволило достигнуть главной цели — повысить осведомленность детей о правилах дорожного движения и развить навыки безопасного поведения.

Александра умело использовала различные методы и технологии для создания интересной и доступной для детей учебной среды. В частности, макеты дорог и интерактивные игры помогли сделать материал более наглядным и привлекательным.

Александра Сергеевна проявила себя как дисциплинированный и ответственный специалист, легко находящий контакт с учениками. Ее уроки и мероприятия проходили в атмосфере сотрудничества и взаимопонимания, что способствовало успешному обучению и развитию детей.

\vspace{5mm}

\textbf{Результаты практики:} За время практики Александра Сергеевна продемонстрировала высокий уровень профессионализма и творческих способностей. Ее активная позиция и стремление к улучшению учебно-воспитательного процесса позволяют рекомендовать Александру Суматохину на оценку за практику \textbf{«отлично»}.

\vspace{10mm}

\noindent
Дата \underline{\hspace{5cm}}
\begin{tabbing}
\hspace{5cm} \= \kill
Классный руководитель \> \underline{\hspace{5cm}} Федорова Елена Андреевна
\end{tabbing}

\vspace{5mm}

\end{document}
