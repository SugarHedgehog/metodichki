\documentclass[10pt]{article}
\usepackage[russian]{babel}
\usepackage[utf8]{inputenc}
\usepackage[T2A]{fontenc}
\usepackage{amsmath}
\usepackage{amsfonts}
\usepackage{amssymb}
\usepackage[version=4]{mhchem}
\usepackage{stmaryrd}

\begin{document}
Лекция 9

\section*{Асимптотические методы теории нелинейных колебаний}
Разложение в ряд по параметру нелинейности.\\
Осциллятор с квадратичной нелинейностью\\
Случаи, когда удается найти точные решения в явной аналитической форме, которым была посвящена предыдущая лекция, представляют, скорее, исключение из правил. Поэтому в теории колебаний разработан богатый арсенал приближенных или асимптотических методов. Основные идеи наиболее важных из них будут рассмотрены в настоящей главе.

Начнем с осциллятора с квадратичной нелинейностью


\begin{equation*}
\ddot{x}+\omega_{0}^{2} x+\alpha x^{2}=0 \tag{9.1}
\end{equation*}


Как было показано в лекции 8, это уравнение можно привести к универсальному виду, не содержащему параметров. Однако здесь для наших целей больше подходит несколько иная нормировка переменных. Пусть известен некоторый характерный масштаб колебаний $A$. Введем безразмерные время и координату следующим образом:


\begin{equation*}
t^{\prime}=\omega_{0} t, x^{\prime}=x / A . \tag{9.2}
\end{equation*}


Уравнение (9.1) примет вид (штрихи у безразмерных переменных опускаем)


\begin{equation*}
\ddot{x}+x+\varepsilon x^{2}=0, \tag{9.3}
\end{equation*}


где $\varepsilon=\alpha A / \omega_{0}^{2}$. Рассмотрим случай слабой нелинейности, когда $\varepsilon \ll 1$, т.е. уравнение (9.3) содержит малый параметр. Вообще, следует отметить, что условием применимости любого асимптотического метода является присутствие в уравнении малого (или большого) параметра.

Уравнение (9.3) близко к уравнению линейного консервативного осциллятора, оно отличается от него малым слагаемым порядка $\varepsilon$. Поэтому интуитивно ясно, что решение будет иметь вид квазигармонических (т.е. почти гармонических, близких к гармоническим) колебаний. Попробуем построить приближенное решение уравнения (9.3). Наиболее простой способ, очевидно, состоит в том, чтобы искать решение в виде ряда по степеням малого параметра $\varepsilon$ :


\begin{equation*}
x(t)=x_{1}(t)+\varepsilon x_{2}(t)+\varepsilon^{2} x_{3}(t)+\ldots \tag{9.4}
\end{equation*}


считая $x_{1,2, . .}$ величинами порядка единицы. В литературе подобный прием называют методом разложения по малому параметру или прямым разложением. Подставив ряд (9.4) в уравнение (9.3), получим


\begin{equation*}
\ddot{x}_{1}+\varepsilon \ddot{x}_{2}+\varepsilon^{2} \ddot{x}_{3}+\ldots+x_{1}+\varepsilon x_{2}+\varepsilon^{2} x_{3}+\ldots+\varepsilon x_{1}^{2}+2 \varepsilon^{2} x_{1} x_{2}+\ldots=0 . \tag{9.5}
\end{equation*}


Приравнивая в (9.5) к нулю члены при одинаковых степенях $\varepsilon$, приходим к системе «зацепляющихся» уравнений


\begin{align*}
& \varepsilon^{0}: \ddot{x}_{1}+x_{1}=0,  \tag{9.6}\\
& \varepsilon^{1}: \ddot{x}_{2}+x_{2}+x_{1}^{2}=0,  \tag{9.7}\\
& \varepsilon^{2}: \ddot{x}_{3}+x_{3}+2 x_{1} x_{2}=0, \tag{9.8}
\end{align*}


Уравнение (9.6) есть уравнение гармонического осциллятора, решение которого имеет вид


\begin{equation*}
x_{1}=a \cos (t+\varphi), \tag{9.9}
\end{equation*}


где амплитуда $a$ и начальная фаза $\varphi$ — постоянные, определяемые из начальных условий. Далее подставим решение (9.9) в уравнение (9.7), чтобы найти $x_{2}$ :


\begin{equation*}
\ddot{x}_{2}+x_{2}=-x_{1}^{2}=-\frac{a^{2}}{2}-\frac{a^{2}}{2} \cos 2(t+\varphi) \tag{9.10}
\end{equation*}


Это уравнение формально совпадает с уравнением линейного консервативного осциллятора под внешним воздействием. Его решение следует искать в виде


\begin{equation*}
x_{2}=x_{2}^{(o)}+x_{2}^{(h)}, \tag{9.11}
\end{equation*}


где


\begin{equation*}
x_{2}^{(o)}=a_{1} \cos \left(t+\varphi_{1}\right) \tag{9.12}
\end{equation*}


\begin{itemize}
  \item решение однородного уравнения, описывающее собственные колебания осциллятора. Его амплитуда $a_{1}$ и начальная фаза $\varphi_{1}$ по-прежнему определяются из начальных условий. Второе слагаемое $x_{2}^{(h)}$ есть частное решение неоднородного уравнения. Оно
\end{itemize}

представляет собой вынужденные колебания осциллятора, т.е. отклик на внешнее воздействие. Как мы знаем из теории линейных колебаний, в спектре вынужденных колебаний будут содержаться те частоты, которые присутствуют в спектре вынуждающей силы. В данном случае это нулевая (постоянная составляющая) и вторая гармоники. Нетрудно найти, что


\begin{equation*}
x_{2}^{(t)}=-\frac{a^{2}}{2}+\frac{a^{2}}{6} \cos 2(t+\varphi) . \tag{9.13}
\end{equation*}


Итак


\begin{equation*}
x_{2}=a_{1} \cos \left(t+\varphi_{1}\right)-\frac{a^{2}}{2}+\frac{a^{2}}{6} \cos 2(t+\varphi) \tag{9.14}
\end{equation*}


Отметим, что полученное нами решение содержит четыре независимых постоянных $\left(a, \varphi, a_{1}, \varphi_{1}\right)$, для определения которых имеются только два начальных условия. Поэтому можно две из этих постоянных выбрать произвольным образом. Наиболее удобно положить $a_{1}=0$. В дальнейшем для простоты условимся во всех высших порядках малости полагать составляющие, соответствующие собственным колебаниям, равными нулю.

Таким образом, окончательный вид решения с точностью до членов порядка $\varepsilon^{2}$ таков:


\begin{equation*}
x \approx a \cos (t+\varphi)+\varepsilon\left[-\frac{a^{2}}{2}+\frac{a^{2}}{6} \cos 2(t+\varphi)\right]+\ldots . \tag{9.15}
\end{equation*}


Как видно из выражения (9.15), в спектре колебаний появляются высшие гармоники: нулевая и вторая, амплитуды которых имеют порядок $\varepsilon a^{2}$, т.е. много меньше амплитуды основной составляющей. Можно продолжить описанную процедуру, продвигаясь во все более высокие порядки малости. В решении появятся и другие гармоники: третья, четвертая и т.д. Однако их амплитуды будут еще меньше (порядка $\varepsilon^{n-1} a^{n}$, где $n-$ номер гармоники). Действительно, поскольку нелинейность является слабой, амплитуды высших гармоник должны быстро уменьшаться с ростом их номера.

Остается только вычислить константы $a$ и $\varphi$. Пусть начальные условия имеют вид


\begin{equation*}
x(0)=x_{0}, \dot{x}(0)=y_{0} . \tag{9.16}
\end{equation*}


Тогда, используя выражение (9.15), легко найти, что


\begin{gather*}
a \cos \varphi-\varepsilon\left[\frac{a^{2}}{2}-\frac{a^{2}}{6} \cos 2 \varphi\right]=x_{0}  \tag{9.17}\\
a \sin \varphi+\frac{\varepsilon a^{2}}{3} \sin 2 \varphi=-y_{0} .
\end{gather*}


Это система трансцендентных уравнений, получить точное решение которой в общем случае не удается. Однако, учитывая, что в (9.17) содержится малый параметр, можно представить решение в виде рядов


\begin{align*}
a & =a_{0}+\varepsilon a_{1}+\ldots \\
\varphi & =\varphi_{0}+\varepsilon \varphi_{1}+\ldots \tag{9.18}
\end{align*}


В разложениях (9.18) нужно учитывать то же число членов, что и в решении (9.15). Пытаться найти $a$ и $\varphi$ с более высокой степенью точности, очевидно, просто не имеет смысла.

Итак, подставим (9.18) в систему (9.17) и выделим члены одинаковых порядков малости. В нулевом порядке по $\varepsilon$ будем иметь


\begin{gather*}
a_{0} \cos \varphi_{0}=x_{0}  \tag{9.19}\\
a_{0} \sin \varphi_{0}=-y_{0}
\end{gather*}


откуда нетрудно найти, что


\begin{gather*}
a_{9}=\sqrt{x_{0}^{2}+y_{0}^{2}} \\
\varphi_{0}=\arg (x-i y)=-2 \operatorname{arctg} \frac{y_{0}}{x_{0}+\sqrt{x_{0}^{2}+y_{0}^{2}}} . \tag{9.20}
\end{gather*}


Члены порядка $\varepsilon$ в (9.17) дают


\begin{gather*}
a_{1} \cos \varphi_{0}-a_{0} \varphi_{1} \sin \varphi_{0}-\frac{a_{0}^{2}}{2}+\frac{a_{0}^{2}}{6} \cos 2 \varphi_{0}=0  \tag{9.21}\\
a_{1} \sin \varphi_{0}+a_{0} \varphi_{1} \cos \varphi_{0}+\frac{a^{2}}{3} \sin 2 \varphi=0
\end{gather*}


Это система линейных уравнений относительно $a_{1}, \varphi_{1}$, найти решение которой не представляет труда. Мы предлагаем читателю проделать это самостоятельно.

\section*{Разложение по степеням параметра нелинейности. Осциллятор Дуффинга}
Столь простой подход, как прямое разложение по степеням малого параметра, не всегда приводит к успеху. Чтобы показать это, рассмотрим осциллятор Дуффинга (осциллятор с кубической нелинейностью)


\begin{equation*}
\ddot{x}+\omega_{0}^{2} x+\beta x^{3}=0 . \tag{9.22}
\end{equation*}


Вновь используем замену переменных (9.2). Тогда уравнение (9.22) примет вид


\begin{equation*}
\ddot{x}+x+\varepsilon x^{3}=0, \tag{9.23}
\end{equation*}


где теперь $\varepsilon=\beta A^{2} / \omega_{0}^{2}$. Как и прежде, будем рассматривать случай слабой нелинейности, т.е. $\varepsilon \ll 1$. Отыскивая решение в виде (9.4), вместо уравнений (9.6)-(9.8) будем иметь


\begin{align*}
& \varepsilon^{0}: \ddot{x}_{1}+x_{1}=0,  \tag{9.24}\\
& \varepsilon^{1}: \ddot{x}_{2}+x_{2}+x_{1}^{3}=0 . \tag{9.25}
\end{align*}


В нулевом порядке по $\varepsilon$, естественно, по-прежнему получаем уравнение гармонического осциллятора, решение которого имеет вид (9.9). Попытаемся найти $x_{2}$. После подстановки выражения для $x_{1}$ (9.9) уравнение (9.7) приводится к виду


\begin{equation*}
\ddot{x}_{2}+x_{2}=-x_{1}^{3}=-a^{3} \cos ^{3}(t+\varphi)=-\frac{a^{3}}{4}[3 \cos (t+\varphi)+\cos 3(t+\varphi)] . \tag{9.26}
\end{equation*}


Нужно найти решение этого уравнения, соответствующее вынужденным колебаниям в членах высшего порядка. Поскольку нелинейность кубичная, в данном случае в спектре внешнего воздействия содержатся первая и третья гармоники. Решение будем искать в виде суперпозиции откликов на эти воздействия:


\begin{equation*}
x_{2}=x_{2}^{(1)}+x_{2}^{(3)}, \tag{9.27}
\end{equation*}


где $x_{2}^{(1)}$ и $x_{2}^{(3)}$ удовлетворяют уравнениям


\begin{align*}
& \ddot{x}_{2}^{(1)}+x_{2}^{(1)}=-\frac{3 a^{3}}{4} \cos (t+\varphi),  \tag{9.28}\\
& \ddot{x}_{2}^{(3)}+x_{2}^{(3)}=-\frac{a^{3}}{4} \cos 3(t+\varphi) . \tag{9.29}
\end{align*}


Решение уравнения (9.29) находится без труда и имеет вид гармонических колебаний на частоте вынуждающей силы:


\begin{equation*}
x_{2}^{(3)}=\frac{a^{3}}{32} \cos 3(t+\varphi) . \tag{9.30}
\end{equation*}


Что же касается уравнения (9.28), то в нем внешнее воздействие имеет частоту, равную частоте собственных колебаний осциллятора. Как известно из теории линейных колебаний, в этом случае возникает резонанс, выражающийся в неограниченном нарастании амплитуды колебаний по линейному закону. Соответствующее решение имеет вид


\begin{equation*}
x_{2}^{(1)}=-\frac{3 a^{3} t}{8} \sin (t+\varphi) \tag{9.30}
\end{equation*}


Это так называемый секулярный или вековой член. (Термин берет свое начало из небесной механики.) Окончательный вид решения с точностью до членов второго порядка малости таков:


\begin{equation*}
x \approx a \cos (t+\varphi)+\varepsilon\left[-\frac{3 a^{3} t}{8} \sin (t+\varphi)+\frac{a^{3}}{32} \cos 3(t+\varphi)\right]+\ldots \tag{9.31}
\end{equation*}


Обратим внимание, что, как бы ни был мал параметр $\varepsilon$, с течением времени второй член в решении (9.31), неограниченно нарастая, становится больше первого. Таким образом, справедливость разложения (9.4) на больших временах нарушается, или, как говорят математики, разложение не является равномерно пригодным по $t$. Это явно нефизический результат. Действительно, как мы показали в лекции 8, решения уравнения Дуффинга имеют вид периодических нелинейных колебаний, и никакого нарастания амплитуды со временем нет.

В чем же причина неудачного результата? Дело в том, что колебания осциллятора Дуффинга являются неизохронными, т.е. их период зависит от амплитуды. Разложение (9.4) принципиально не учитывает неизохронность: в спектре колебаний могут появиться только собственная частота линейных колебаний и её гармоники.

Для осциллятора с квадратичной нелинейностью (9.3) мы на самом деле пришли бы к аналогичному результату, если бы продвинулись в вычислениях ещё на один порядок. Как видно из уравнения (9.8), при попытке найти решение для $x_{3}$ в правой части появится произведение $x_{1} x_{2}$. Поскольку выражение для $x_{1}$ (9.9) содержит первую гармонику, а выражение для $x_{2}$ (9.13) - вторую, их произведение будет содержать первую и третью гармоники. Следовательно, в решении для $x_{3}$ мы также получим секулярно растущее слагаемое.

\section*{Метод Линштедта - Пуанкаре}
Итак, необходимо модифицировать схему решения таким образом, чтобы можно было учесть неизохронность. Наиболее простой способ был предложен А. Линштедтом (1883) и А. Пуанкаре (1892). Введем в уравнении (9.23) новую временную переменную $\tau=\omega t$. Поскольку $d / d t=\omega d / d \tau$, получим


\begin{equation*}
\omega^{2} x^{\prime \prime}+x+\varepsilon x^{3}=0 \tag{9.32}
\end{equation*}


Здесь штрихами обозначены производные по $\tau$. Будем искать решение уравнения (9.32) в виде разложений в степенной ряд как для переменной $x$, так и для частоты $\omega$ :


\begin{align*}
& x=x_{1}+\varepsilon x_{2}+\varepsilon^{2} x_{3}+\ldots  \tag{9.33}\\
& \omega=1+\varepsilon \omega_{1}+\varepsilon^{2} \omega_{2}+\ldots
\end{align*}


Первый член в разложении для $\omega$ должен представлять собой частоту линейных колебаний, которая в принятой нормировке равна единице. Последующие поправки $\omega_{1}, \omega_{2}, \ldots$ будут описывать эффекты неизохронности.

Подставим разложения (9.33) в уравнение (9.32). Получим


\begin{align*}
& {\left[1+2 \varepsilon \omega_{1}+\varepsilon^{2}\left(\omega_{1}^{2}+2 \omega_{2}\right)+\ldots\right]\left[x_{1}^{\prime \prime}+\varepsilon x_{2}^{\prime \prime}+\ldots\right]+}  \tag{9.34}\\
& \quad+x_{1}+\varepsilon x_{2}+\ldots \varepsilon x_{1}^{3}+3 \varepsilon^{2} x_{1}^{2} x_{2}+\ldots=0
\end{align*}


Преобразуем уравнение (9.34). После несложных вычислений приведем его к виду


\begin{equation*}
x_{1}^{\prime \prime}+x_{1}+\varepsilon\left(x_{2}^{\prime \prime}+x_{2}+2 \omega_{1} x_{1}^{\prime \prime}+x_{1}^{3}\right)+\ldots=0 \tag{9.35}
\end{equation*}


Приравнивая к нулю члены нулевого и первого порядков малости, будем иметь


\begin{gather*}
x_{1}^{\prime \prime}+x_{1}=0,  \tag{9.36}\\
x_{2}^{\prime \prime}+x_{2}=-2 \omega_{1} x_{1}^{\prime \prime}-x_{1}^{3} . \tag{9.37}
\end{gather*}


Решение уравнения (9.36) запишем в виде


\begin{equation*}
x_{1}=a \cos (\tau+\varphi)=a \cos (\omega t+\varphi) \tag{9.38}
\end{equation*}


Подставив это соотношение в правую часть (9.37), найдем, что


\begin{equation*}
x_{2}^{\prime \prime}+x_{2}=2 \omega_{1} a \cos (\tau+\varphi)-\frac{a^{3}}{4}[3 \cos (\tau+\varphi)+\cos 3(\tau+\varphi)] . \tag{9.39}
\end{equation*}


Теперь необходимо выбрать $\omega_{1}$ таким образом, чтобы устранить члены, пропорциональные $\cos (\tau+\varphi)$, которые приводят к секулярному росту решения для $x_{2}$. Для этого, очевидно, следует положить


\begin{equation*}
\omega_{1}=\frac{3 a^{2}}{8} \tag{9.40}
\end{equation*}


Теперь уравнение (9.39) принимает вид


\begin{equation*}
x_{2}^{\prime \prime}+x_{2}=-\frac{a^{3}}{4} \cos 3(\tau+\varphi) \tag{9.41}
\end{equation*}


Его решение


\begin{equation*}
x_{2}=\frac{a^{3}}{32} \cos 3(\tau+\varphi) \tag{9.42}
\end{equation*}


не содержит секулярных составляющих и разложение остается равномерно пригодным при всех $t$.

Окончательный вид найденного нами решения с точностью до членов порядка $\varepsilon^{2}$ таков (ср. (9.31)):


\begin{gather*}
x \approx a \cos (\omega t+\varphi)+\frac{\varepsilon a^{3}}{32} \cos 3(\omega t+\varphi),  \tag{9.43}\\
\omega \approx 1+\frac{3 \varepsilon a^{2}}{8} . \tag{9.44}
\end{gather*}


Если параметр $\varepsilon$ считается положительным, то частота колебаний растет с ростом амплитуды, при $\varepsilon<0$ частота, наоборот, уменьшается.

Отметим, что в отличие от осциллятора с квадратичной нелинейностью в спектре колебаний в первую очередь появляется не вторая, а третья гармоника. Если продолжать разложения далее, то можно убедиться, что спектр будет содержать только нечетные гармоники. Это является следствием симметрии уравнения Дуффинга относительно замены $x \rightarrow-x$. Аналогичный результат мы получили при анализе колебаний математического маятника (лекция 7).

Задача 9.1. Получите оценку для частоты слабонелинейных колебаний (9.44) из точного решения, найденного в лекции 8.

Решение. В случае $\varepsilon>0$ для периода справедливо соотношение (8.34). Заменив в этой формуле $x_{0}^{2}$ приближенно на величину $\varepsilon a^{2}$, найдем, что


\begin{equation*}
T=\frac{4 K(m)}{\sqrt{1+\varepsilon a^{2}}}, m^{2}=\frac{\varepsilon a^{2}}{2\left(1+\varepsilon a^{2}\right)} \tag{9.45}
\end{equation*}


где $K(m)$ - полный эллиптический интеграл первого рода. С учетом малости $\varepsilon$ имеем $m^{2} \approx \varepsilon a^{2} / 2$.\\
Получим приближенное выражение для $K(m)$ при малых значениях $m$. В этом случае


\begin{equation*}
K(m)=\int_{0}^{\pi / 2} \frac{d \psi}{\sqrt{1-m^{2} \sin ^{2} \psi}} \approx \int_{0}^{\pi / 2}\left(1+\frac{m^{2}}{2} \sin ^{2} \psi\right) d \psi \tag{9.46}
\end{equation*}


Интеграл (9.46) легко вычисляется:


\begin{equation*}
K(m) \approx \frac{\pi}{2}\left(1+\frac{m^{2}}{4}\right)=\frac{\pi}{2}\left(1+\frac{\varepsilon a^{2}}{8}\right) . \tag{9.47}
\end{equation*}


Подставив это выражение в формулу (9.45) и ограничиваясь членами порядка $\varepsilon$, будем иметь


\begin{equation*}
T \approx 2 \pi\left(1+\frac{\varepsilon a^{2}}{8}\right)\left(1-\frac{\varepsilon a^{2}}{2}\right) \approx 2 \pi\left(1-\frac{3 \varepsilon a^{2}}{8}\right) . \tag{9.48}
\end{equation*}


Тогда видно, что частота $\omega=2 \pi / T$ совпадает с формулой (9.44).\\
В случае $\varepsilon<0$ период колебаний определяется формулой (8.42), которую можно приближенно записать в виде


\begin{equation*}
T=\frac{4 K(m)}{\sqrt{1-|\varepsilon| a^{2} / 2}}, m^{2}=\frac{|\varepsilon| a^{2}}{2\left(1-|\varepsilon| a^{2} / 2\right)} \approx \frac{|\varepsilon| a^{2}}{2} \tag{9.49}
\end{equation*}


С учетом выражения (9.47) получаем


\begin{equation*}
T \approx 2 \pi\left(1+\frac{|\varepsilon| a^{2}}{8}\right)\left(1+\frac{|\varepsilon| a^{2}}{4}\right) \approx 2 \pi\left(1+\frac{3|\varepsilon| a^{2}}{8}\right)=2 \pi\left(1-\frac{3 \varepsilon a^{2}}{8}\right) \tag{9.50}
\end{equation*}


Следовательно, для частоты снова приходим к формуле(9.44).\\
На рис. 9.1 для сравнения приведены зависимости $x(t)$, полученные по различным приближенным методикам, и точное решение (8.36). Параметр а выбран равным 0.5 , т.е. нелинейность, вообще говоря, достаточно сильная. Тем не менее, решение (9.43), полученное методом Линштедта - Пуанкаре, достаточно хорошо согласуется с точным решением. В то же время зависимость $x(t)$, построенная согласно формуле (9.31), демонстрирует очевидный рост амплитуды по линейному закону, и уже на временах порядка периода колебаний расхождение становится существенным.

Задача 9.2 Приведите приближенно задачу о движении частицы в потенциальной яме вида $U(x)=U_{0} \operatorname{tg}^{2} \alpha x$ к модели осциллятора с кубической нелинейностью. В рамках этой модели найдите зависимость периода колебаний от частоты. Сравните полученный результат с точным (задача 4.6), построив соответствующую таблицу.


\end{document}