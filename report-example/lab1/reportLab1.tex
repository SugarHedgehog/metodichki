\documentclass[a4paper, 14pt]{extarticle}
\usepackage{ifxetex,ifluatex,ifpdf}

\ifluatex
    \usepackage{fontspec}
    \usepackage{polyglossia}
    \setmainfont{CMU Serif}
    \newfontfamily{\cyrillicfont}{CMU Serif}
    \setsansfont{CMU Sans Serif}
    \newfontfamily{\cyrillicfontsf}{CMU Sans Serif}
    \setmonofont{CMU Typewriter Text}
    \newfontfamily{\cyrillicfonttt}{CMU Typewriter Text}
    \setdefaultlanguage{russian}
    \else  
    \ifpdf
        \usepackage[english, russian]{babel}
        \usepackage[T2A]{fontenc}
        \else
        \ifxetex
        	\usepackage{fontspec}
        	\defaultfontfeatures{Ligatures={TeX}}
        	\setmainfont{Noto Serif}
        	\setsansfont{Noto Sans}
        	\setmonofont{Noto Sans Mono}
        \fi
    \fi
\fi

\usepackage[left=1cm,right=1cm,top=2cm,bottom=2cm]{geometry}
\usepackage{amsmath}
\usepackage{hyperref}

\begin{document}
\thispagestyle{empty}
\begin{center}
	\hfill \break
	\large{МИНОБРНАУКИ РОССИИ}\\
	\footnotesize{ФЕДЕРАЛЬНОЕ ГОСУДАРСТВЕННОЕ БЮДЖЕТНОЕ ОБРАЗОВАТЕЛЬНОЕ УЧРЕЖДЕНИЕ}\\
	\footnotesize{ВЫСШЕГО ПРОФЕССИОНАЛЬНОГО ОБРАЗОВАНИЯ}\\
	\small{\textbf{«ВОРОНЕЖСКИЙ ГОСУДАРСТВЕННЫЙ УНИВЕРСИТЕТ»}}\\

	\hfill \break
	\hfill \break
	\hfill \break
	\hfill \break
	\hfill \break
	\hfill \break
	\hfill \break
	\hfill \break
	\hfill \break

	\large\textbf{Лабораторная работа №1}\\
	\large\textbf{Шифрование методом простой замены}\\
\end{center}

\hfill \break
\hfill \break
\hfill \break
\hfill \break
\hfill \break
\hfill \break
\hfill \break


\begin{tabular}{p{0.8\textwidth}}
	Студента(ки) 4 курса математического факультета \newline
	\underline{Суматохина А.С.} \\
	\\
	Преподаватель \newline
	\underline{Сидельникова Софья Юрьевна} \\                 \\
\end{tabular}

\hfill \break
\hfill \break
\hfill \break
\hfill \break
\hfill \break
\hfill \break
\hfill \break
\hfill \break
\hfill \break
\hfill \break
\hfill \break
\begin{center}
	Воронеж 2023г
\end{center}
\newpage

\tableofcontents
\newpage

\section{Исходный текст}

Техника — обобщающее наименование устройств, механизмов, машин, систем (включая средства труда).

Также термин может употребляться для обозначения методов, процессов и технологий упорядоченной искусной деятельности, например, для создания, изготовления, обеспечения, использования чего-либо, включая методологически упорядоченные процессы творчества (смотри Техника (значения)).

Понятие техники охватывает технические изделия, ранее не существовавшие в природе и процессы их изготовления человеком для осуществления какой-либо деятельности, в том числе: машины, механизмы, оборудование, аппараты, инструменты, приборы и т. д. — а также системы взаимосвязанных технических устройств (в частности, агрегаты, установки и строительные сооружения). Техника может иметь производственное (промышленное, агропромышленное) или непроизводственное назначение. Последнее включает использование техники в науке, быту, образовании, культуре, медицине, военном деле, освоении космоса и в других областях. С точки зрения управления процессами, техника является средством реализации задач и достижения целей процесса; техника используется в огромном разнообразии процессов, включая технологические процессы и процессы промышленного и сельскохозяйственного производства, измерения, контроля и управления, перевозки, ведения боевых действий, обучения, спорта, отдыха, развлечений и многих других процессов.

Техника разрабатывается и совершенствуется в результате инженерной деятельности. Особенности конструирования и изготовления технических устройств зависят от вида технического устройства, требований заказчика к его техническим характеристикам (производительности, надёжности, экономичности, долговечности и т. д.), качеству, стоимости, технологии изготовления, а также от финансовых и технических возможностей производителя. Так, техническое изделие или агрегат могут быть изготовлены промышленным или кустарным способом, в то время как установки, как правило, собирают из компонентов по месту эксплуатации установки. При этом отдельные компоненты установки — индивидуальные изделия, агрегаты и узлы — могут иметь высокую заводскую готовность и модульность, что позволяет значительно снизить затраты труда и времени на их интеграцию в установку и замену в случае неисправности. Огромную роль в технике играет взаимозаменяемость, которая снижает затраты и облегчает конструирование, изготовление, эксплуатацию, обслуживание и ремонт технических устройств.

Современная техника является продуктом научно-технической революции, а уровень развития техники является показателем научно-технического развития общества. В условиях глобализации мировой экономики передовая техника быстро распространяется по миру. Вместе с тем её использование в отдельно взятой стране или её части зависит от множества факторов, влияющих на доступность техники и эффективность её практического применения — например, уровень экономического развития, рынка, кредитно-финансовой системы, наличие и дееспособность инфраструктуры, покупательская способность, квалификация пользователей техники.

В более широком смысле под техникой понимают совокупность технических средств и технологий, знаний и деятельности, в которых задействованы технические средства. 

\section{Форматированный текст}

Форматирование текста происходит поэтапно функцией \texttt{formatText}
\begin{enumerate}
    \item Все строчные символы меняются на заглавые
    \item Происхоят замены символов (Ё=Е, Й=И, Ы=И), удаляются пробелы и отступы, знаки припинания и буквы "Ъ" "Ь".
\end{enumerate}

ТЕХНИКАОБОБЩАЮЩЕЕНАИМЕНОВАНИЕУСТРОИСТВ
МЕХАНИЗМОВМАШИНСИСТЕМВКЛЮЧАЯСРЕДСТВАТР
УДАТАКЖЕТЕРМИНМОЖЕТУПОТРЕБЛЯТСЯДЛЯОБОЗ
НАЧЕНИЯМЕТОДОВПРОЦЕССОВИТЕХНОЛОГИИУПОР
ЯДОЧЕННОИИСКУСНОИДЕЯТЕЛНОСТИНАПРИМЕРДЛ
ЯСОЗДАНИЯИЗГОТОВЛЕНИЯОБЕСПЕЧЕНИЯИСПОЛЗ
ОВАНИЯЧЕГОЛИБОВКЛЮЧАЯМЕТОДОЛОГИЧЕСКИУП
ОРЯДОЧЕННИЕПРОЦЕССИТВОРЧЕСТВАСМОТРИТЕХ
НИКАЗНАЧЕНИЯПОНЯТИЕТЕХНИКИОХВАТИВАЕТТЕ
ХНИЧЕСКИЕИЗДЕЛИЯРАНЕЕНЕСУЩЕСТВОВАВШИЕВ
ПРИРОДЕИПРОЦЕССИИХИЗГОТОВЛЕНИЯЧЕЛОВЕКО
МДЛЯОСУЩЕСТВЛЕНИЯКАКОИЛИБОДЕЯТЕЛНОСТИВ
ТОМЧИСЛЕ:МАШИНИМЕХАНИЗМИОБОРУДОВАНИЕАП
ПАРАТИИНСТРУМЕНТИПРИБОРИИТДАТАКЖЕСИСТЕ
МИВЗАИМОСВЯЗАННИХТЕХНИЧЕСКИХУСТРОИСТВВ
ЧАСТНОСТИАГРЕГАТИУСТАНОВКИИСТРОИТЕЛНИЕ
СООРУЖЕНИЯТЕХНИКАМОЖЕТИМЕТПРОИЗВОДСТВЕ
ННОЕПРОМИШЛЕННОЕАГРОПРОМИШЛЕННОЕИЛИНЕП
РОИЗВОДСТВЕННОЕНАЗНАЧЕНИЕПОСЛЕДНЕЕВКЛЮ
ЧАЕТИСПОЛЗОВАНИЕТЕХНИКИВНАУКЕБИТУОБРАЗ
ОВАНИИКУЛТУРЕМЕДИЦИНЕВОЕННОМДЕЛЕОСВОЕН
ИИКОСМОСАИВДРУГИХОБЛАСТЯХСТОЧКИЗРЕНИЯУ
ПРАВЛЕНИЯПРОЦЕССАМИТЕХНИКАЯВЛЯЕТСЯСРЕД
СТВОМРЕАЛИЗАЦИИЗАДАЧИДОСТИЖЕНИЯЦЕЛЕИПР
ОЦЕССАТЕХНИКАИСПОЛЗУЕТСЯВОГРОМНОМРАЗНО
ОБРАЗИИПРОЦЕССОВВКЛЮЧАЯТЕХНОЛОГИЧЕСКИЕ
ПРОЦЕССИИПРОЦЕССИПРОМИШЛЕННОГОИСЕЛСКОХ
ОЗЯИСТВЕННОГОПРОИЗВОДСТВАИЗМЕРЕНИЯКОНТ
РОЛЯИУПРАВЛЕНИЯПЕРЕВОЗКИВЕДЕНИЯБОЕВИХД
ЕИСТВИИОБУЧЕНИЯСПОРТАОТДИХАРАЗВЛЕЧЕНИИ
ИМНОГИХДРУГИХПРОЦЕССОВТЕХНИКАРАЗРАБАТИ
ВАЕТСЯИСОВЕРШЕНСТВУЕТСЯВРЕЗУЛТАТЕИНЖЕН
ЕРНОИДЕЯТЕЛНОСТИОСОБЕННОСТИКОНСТРУИРОВ
АНИЯИИЗГОТОВЛЕНИЯТЕХНИЧЕСКИХУСТРОИСТВЗ
АВИСЯТОТВИДАТЕХНИЧЕСКОГОУСТРОИСТВАТРЕБ
ОВАНИИЗАКАЗЧИКАКЕГОТЕХНИЧЕСКИМХАРАКТЕР
ИСТИКАМПРОИЗВОДИТЕЛНОСТИНАДЕЖНОСТИЭКОН
ОМИЧНОСТИДОЛГОВЕЧНОСТИИТДКАЧЕСТВУСТОИМ
ОСТИТЕХНОЛОГИИИЗГОТОВЛЕНИЯАТАКЖЕОТФИНА
НСОВИХИТЕХНИЧЕСКИХВОЗМОЖНОСТЕИПРОИЗВОД
ИТЕЛЯТАКТЕХНИЧЕСКОЕИЗДЕЛИЕИЛИАГРЕГАТМО
ГУТБИТИЗГОТОВЛЕНИПРОМИШЛЕННИМИЛИКУСТАР
НИМСПОСОБОМВТОВРЕМЯКАКУСТАНОВКИКАКПРАВ
ИЛОСОБИРАЮТИЗКОМПОНЕНТОВПОМЕСТУЭКСПЛУА
ТАЦИИУСТАНОВКИПРИЭТОМОТДЕЛНИЕКОМПОНЕНТ
ИУСТАНОВКИИНДИВИДУАЛНИЕИЗДЕЛИЯАГРЕГАТИ
ИУЗЛИМОГУТИМЕТВИСОКУЮЗАВОДСКУЮГОТОВНОС
ТИМОДУЛНОСТЧТОПОЗВОЛЯЕТЗНАЧИТЕЛНОСНИЗИ
ТЗАТРАТИТРУДАИВРЕМЕНИНАИХИНТЕГРАЦИЮВУС
ТАНОВКУИЗАМЕНУВСЛУЧАЕНЕИСПРАВНОСТИОГРО
МНУЮРОЛВТЕХНИКЕИГРАЕТВЗАИМОЗАМЕНЯЕМОСТ
КОТОРАЯСНИЖАЕТЗАТРАТИИОБЛЕГЧАЕТКОНСТРУ
ИРОВАНИЕИЗГОТОВЛЕНИЕЭКСПЛУАТАЦИЮОБСЛУЖ
ИВАНИЕИРЕМОНТТЕХНИЧЕСКИХУСТРОИСТВСОВРЕ
МЕННАЯТЕХНИКАЯВЛЯЕТСЯПРОДУКТОМНАУЧНОТЕ
ХНИЧЕСКОИРЕВОЛЮЦИИАУРОВЕНРАЗВИТИЯТЕХНИ
КИЯВЛЯЕТСЯПОКАЗАТЕЛЕМНАУЧНОТЕХНИЧЕСКОГ
ОРАЗВИТИЯОБЩЕСТВАВУСЛОВИЯХГЛОБАЛИЗАЦИИ
МИРОВОИЭКОНОМИКИПЕРЕДОВАЯТЕХНИКАБИСТРО
РАСПРОСТРАНЯЕТСЯПОМИРУВМЕСТЕСТЕМЕЕИСПО
ЛЗОВАНИЕВОТДЕЛНОВЗЯТОИСТРАНЕИЛИЕЕЧАСТИ
ЗАВИСИТОТМНОЖЕСТВАФАКТОРОВВЛИЯЮЩИХНАДО
СТУПНОСТТЕХНИКИИЭФФЕКТИВНОСТЕЕПРАКТИЧЕ
СКОГОПРИМЕНЕНИЯНАПРИМЕРУРОВЕНЭКОНОМИЧЕ
СКОГОРАЗВИТИЯРИНКАКРЕДИТНОФИНАНСОВОИСИ
СТЕМИНАЛИЧИЕИДЕЕСПОСОБНОСТИНФРАСТРУКТУ
РИПОКУПАТЕЛСКАЯСПОСОБНОСТКВАЛИФИКАЦИЯП
ОЛЗОВАТЕЛЕИТЕХНИКИВБОЛЕЕШИРОКОМСМИСЛЕП
ОДТЕХНИКОИПОНИМАЮТСОВОКУПНОСТТЕХНИЧЕСК
ИХСРЕДСТВИТЕХНОЛОГИИЗНАНИИИДЕЯТЕЛНОСТИ
ВКОТОРИХЗАДЕИСТВОВАНИТЕХНИЧЕСКИЕСРЕДСТВА

\subsection{Статистика для форматированного текста}

Статистика составлется функцией \texttt{statistics}

\noindent Всего символов: 2700\\

\noindent А = 166
\\Б = 29
\\В = 130
\\Г = 39
\\Д = 56
\\Е = 267
\\Ж = 13
\\З = 63
\\И = 333
\\К = 94
\\Л = 89
\\М = 74
\\Н = 195
\\О = 280
\\П = 64
\\Р = 122
\\С = 175
\\Т = 213
\\У = 64
\\Ф = 7
\\Х = 54
\\Ц = 18
\\Ч = 47
\\Ш = 9
\\Щ = 6
\\Э = 7
\\Ю = 14
\\Я = 71

Индекс совпадений считается по формуле $\sum^{28}_{i=1} p_i^2$, где $p_i$ - через вероятности появления $i$-го символа. Расчёт производит фнукция \texttt{indexOfMatches}\\

\textbf{Индекс совпадений: 0.0665156378600823}

\section{Зашифрованный текст}

Ключ формируется каждый раз случайным образом перемешиванием массива со всеми используемыми русскими символами.

Ключ для конкретного примера:\\

\noindent АБВГДЕЖ\\
ЗИКЛМНО\\
ПРСТУФХ\\
ЦЧШЩЭЮЯ\\


\noindent ДТМУСОЮ\\
ЕАЛЧЦНЯ\\
БЭЗГИШП\\
РЖФВЩХК\\


ГОПНАЛДЯТЯТВДХВООНДАЦОНЯМДНАОИЗГЭЯА
ЗГМЦОПДНАЕЦЯМЦДФАНЗАЗГОЦМЛЧХЖДКЗЭОСЗГМДГЭИС
ДГДЛЮОГОЭЦАНЦЯЮОГИБЯГЭОТЧКГЗКСЧКЯТЯЕНДЖ
ОНАКЦОГЯСЯМБЭЯРОЗЗЯМАГОПНЯЧЯУААИБЯЭКСЯЖ
ОННЯААЗЛИЗНЯАСОКГОЧНЯЗГАНДБЭАЦОЭСЧКЗЯЕС
ДНАКАЕУЯГЯМЧОНАКЯТОЗБОЖОНАКАЗБЯЧЕЯМДНАК
ЖОУЯЧАТЯМЛЧХЖДКЦОГЯСЯЧЯУАЖОЗЛАИБЯЭКСЯЖО
ННАОБЭЯРОЗЗАГМЯЭЖОЗГМДЗЦЯГЭАГОПНАЛДЕНДЖ
ОНАКБЯНКГАОГОПНАЛАЯПМДГАМДОГГОПНАЖОЗЛАО
АЕСОЧАКЭДНООНОЗИВОЗГМЯМДМФАОМБЭАЭЯСОАБЭ
ЯРОЗЗААПАЕУЯГЯМЧОНАКЖОЧЯМОЛЯЦСЧКЯЗИВОЗГ
МЧОНАКЛДЛЯАЧАТЯСОКГОЧНЯЗГАМГЯЦЖАЗЧОЦДФА
НАЦОПДНАЕЦАЯТЯЭИСЯМДНАОДББДЭДГААНЗГЭИЦО
НГАБЭАТЯЭААГСДГДЛЮОЗАЗГОЦАМЕДАЦЯЗМКЕДНН
АПГОПНАЖОЗЛАПИЗГЭЯАЗГММЖДЗГНЯЗГАДУЭОУДГ
АИЗГДНЯМЛААЗГЭЯАГОЧНАОЗЯЯЭИЮОНАКГОПНАЛД
ЦЯЮОГАЦОГБЭЯАЕМЯСЗГМОННЯОБЭЯЦАФЧОННЯОДУ
ЭЯБЭЯЦАФЧОННЯОАЧАНОБЭЯАЕМЯСЗГМОННЯОНДЕН
ДЖОНАОБЯЗЧОСНООМЛЧХЖДОГАЗБЯЧЕЯМДНАОГОПН
АЛАМНДИЛОТАГИЯТЭДЕЯМДНААЛИЧГИЭОЦОСАРАНО
МЯОННЯЦСОЧОЯЗМЯОНААЛЯЗЦЯЗДАМСЭИУАПЯТЧДЗ
ГКПЗГЯЖЛАЕЭОНАКИБЭДМЧОНАКБЭЯРОЗЗДЦАГОПН
АЛДКМЧКОГЗКЗЭОСЗГМЯЦЭОДЧАЕДРААЕДСДЖАСЯЗ
ГАЮОНАКРОЧОАБЭЯРОЗЗДГОПНАЛДАЗБЯЧЕИОГЗКМ
ЯУЭЯЦНЯЦЭДЕНЯЯТЭДЕААБЭЯРОЗЗЯММЛЧХЖДКГОП
НЯЧЯУАЖОЗЛАОБЭЯРОЗЗААБЭЯРОЗЗАБЭЯЦАФЧОНН
ЯУЯАЗОЧЗЛЯПЯЕКАЗГМОННЯУЯБЭЯАЕМЯСЗГМДАЕЦ
ОЭОНАКЛЯНГЭЯЧКАИБЭДМЧОНАКБОЭОМЯЕЛАМОСОН
АКТЯОМАПСОАЗГМААЯТИЖОНАКЗБЯЭГДЯГСАПДЭДЕ
МЧОЖОНАААЦНЯУАПСЭИУАПБЭЯРОЗЗЯМГОПНАЛДЭД
ЕЭДТДГАМДОГЗКАЗЯМОЭФОНЗГМИОГЗКМЭОЕИЧГДГ
ОАНЮОНОЭНЯАСОКГОЧНЯЗГАЯЗЯТОННЯЗГАЛЯНЗГЭ
ИАЭЯМДНАКААЕУЯГЯМЧОНАКГОПНАЖОЗЛАПИЗГЭЯА
ЗГМЕДМАЗКГЯГМАСДГОПНАЖОЗЛЯУЯИЗГЭЯАЗГМДГ
ЭОТЯМДНААЕДЛДЕЖАЛДЛОУЯГОПНАЖОЗЛАЦПДЭДЛГ
ОЭАЗГАЛДЦБЭЯАЕМЯСАГОЧНЯЗГАНДСОЮНЯЗГАЩЛЯ
НЯЦАЖНЯЗГАСЯЧУЯМОЖНЯЗГААГСЛДЖОЗГМИЗГЯАЦ
ЯЗГАГОПНЯЧЯУАААЕУЯГЯМЧОНАКДГДЛЮОЯГШАНДН
ЗЯМАПАГОПНАЖОЗЛАПМЯЕЦЯЮНЯЗГОАБЭЯАЕМЯСАГ
ОЧКГДЛГОПНАЖОЗЛЯОАЕСОЧАОАЧАДУЭОУДГЦЯУИГ
ТАГАЕУЯГЯМЧОНАБЭЯЦАФЧОННАЦАЧАЛИЗГДЭНАЦЗ
БЯЗЯТЯЦМГЯМЭОЦКЛДЛИЗГДНЯМЛАЛДЛБЭДМАЧЯЗЯ
ТАЭДХГАЕЛЯЦБЯНОНГЯМБЯЦОЗГИЩЛЗБЧИДГДРААИ
ЗГДНЯМЛАБЭАЩГЯЦЯГСОЧНАОЛЯЦБЯНОНГАИЗГДНЯ
МЛААНСАМАСИДЧНАОАЕСОЧАКДУЭОУДГААИЕЧАЦЯУ
ИГАЦОГМАЗЯЛИХЕДМЯСЗЛИХУЯГЯМНЯЗГАЦЯСИЧНЯ
ЗГЖГЯБЯЕМЯЧКОГЕНДЖАГОЧНЯЗНАЕАГЕДГЭДГАГЭ
ИСДАМЭОЦОНАНДАПАНГОУЭДРАХМИЗГДНЯМЛИАЕДЦ
ОНИМЗЧИЖДОНОАЗБЭДМНЯЗГАЯУЭЯЦНИХЭЯЧМГОПН
АЛОАУЭДОГМЕДАЦЯЕДЦОНКОЦЯЗГЛЯГЯЭДКЗНАЮДО
ГЕДГЭДГААЯТЧОУЖДОГЛЯНЗГЭИАЭЯМДНАОАЕУЯГЯ
МЧОНАОЩЛЗБЧИДГДРАХЯТЗЧИЮАМДНАОАЭОЦЯНГГО
ПНАЖОЗЛАПИЗГЭЯАЗГМЗЯМЭОЦОННДКГОПНАЛДКМЧ
КОГЗКБЭЯСИЛГЯЦНДИЖНЯГОПНАЖОЗЛЯАЭОМЯЧХРА
АДИЭЯМОНЭДЕМАГАКГОПНАЛАКМЧКОГЗКБЯЛДЕДГО
ЧОЦНДИЖНЯГОПНАЖОЗЛЯУЯЭДЕМАГАКЯТВОЗГМДМИ
ЗЧЯМАКПУЧЯТДЧАЕДРААЦАЭЯМЯАЩЛЯНЯЦАЛАБОЭО
СЯМДКГОПНАЛДТАЗГЭЯЭДЗБЭЯЗГЭДНКОГЗКБЯЦАЭ
ИМЦОЗГОЗГОЦООАЗБЯЧЕЯМДНАОМЯГСОЧНЯМЕКГЯА
ЗГЭДНОАЧАООЖДЗГАЕДМАЗАГЯГЦНЯЮОЗГМДШДЛГЯ
ЭЯММЧАКХВАПНДСЯЗГИБНЯЗГГОПНАЛААЩШШОЛГАМ
НЯЗГООБЭДЛГАЖОЗЛЯУЯБЭАЦОНОНАКНДБЭАЦОЭИЭ
ЯМОНЩЛЯНЯЦАЖОЗЛЯУЯЭДЕМАГАКЭАНЛДЛЭОСАГНЯ
ШАНДНЗЯМЯАЗАЗГОЦАНДЧАЖАОАСООЗБЯЗЯТНЯЗГА
НШЭДЗГЭИЛГИЭАБЯЛИБДГОЧЗЛДКЗБЯЗЯТНЯЗГЛМД
ЧАШАЛДРАКБЯЧЕЯМДГОЧОАГОПНАЛАМТЯЧООФАЭЯЛ
ЯЦЗЦАЗЧОБЯСГОПНАЛЯАБЯНАЦДХГЗЯМЯЛИБНЯЗГГ
ОПНАЖОЗЛАПЗЭОСЗГМАГОПНЯЧЯУААЕНДНАААСОКГ
ОЧНЯЗГАМЛЯГЯЭАПЕДСОАЗГМЯМДНАГОПНАЖОЗЛАО
ЗЭОСЗГМД

\subsection{Статистика для зашифрованного текста}

Всего символов: 2699\\

\noindent А = 333
\\Б = 64
\\В = 6
\\Г = 213
\\Д = 166
\\Е = 63
\\Ж = 47
\\З = 175
\\И = 64
\\К = 71
\\Л = 94
\\М = 130
\\Н = 195
\\О = 267
\\П = 54
\\Р = 18
\\С = 56
\\Т = 29
\\У = 39
\\Ф = 9
\\Х = 14
\\Ц = 74
\\Ч = 89
\\Ш = 7
\\Щ = 7
\\Э = 122
\\Ю = 13
\\Я = 280

\textbf{Индекс совпадений: 0.06656493608915573}


Вывод:  успешно проведена программная реализация шифрования методом простой перестановки.

Код реализации: \url{https://github.com/SugarHedgehog/work/tree/main/lab_1}
\end{document}