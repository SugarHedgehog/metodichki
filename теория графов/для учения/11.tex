\textbf{Плоские графы и формула Эйлера}

\noindent\textbf{Планарный граф}: Граф, который можно нарисовать без пересечения рёбер.

\noindent\textbf{Плоский граф}: Граф, нарисованный на плоскости.

\noindent\textbf{Грани}: Области, определяемые плоским графом; внешняя грань — неограниченная.

\noindent\textbf{Цикл}: Путь, начинающийся и заканчивающийся в одной вершине без повторений.

\noindent\textbf{Формула Эйлера}:
Для полиэдров: \( V - E + F = 2 \), где \( V \) — вершины, \( E \) — рёбра, \( F \) — грани.

\noindent\textbf{Графовая версия}:
Для связного плоского графа: \( p - q + r = 2 \).

\noindent\textbf{Следствия и теоремы}:
\begin{itemize}
    \item \textbf{Следствие 11.1 (а)}: Если каждая грань — \( n \)-цикл, то \( q = \frac{n(p-2)}{n-2} \).
    \item \textbf{Максимальный планарный граф}: Граф, который перестаёт быть планарным при добавлении ребра.
    \item \textbf{Следствие 11.1 (б)}: Для максимального плоского графа \( q = 3p - 6 \).
    \item \textbf{Условие планарности}: Для \( p \geq 3 \), \( q \leq 3p - 6 \).
    \item \textbf{Непланарные графы}: \( K_5 \) и \( K_{3,3} \).
    \item \textbf{Теорема Уитни}: Граф планарен, если каждый его блок планарен.
    \item \textbf{Теорема 11.3}: Для любой грани \( f \) двусвязного плоского графа \( G \) найдётся изоморфный плоский граф с внешней гранью \( f \).
\end{itemize}

\noindent\textbf{Дополнительные концепции}:
\begin{itemize}
    \item \textbf{Выпуклый многогранник}: Многогранник, содержащий любые соединяющие его точки отрезки.
    \item \textbf{Теорема Штейница и Радемахера}: Граф — 1-скелет выпуклого многогранника, если он планарен и трёхсвязен.
    \item \textbf{Теорема 11.7}: Любой планарный граф изоморфен плоскому графу с прямыми рёбрами.
\end{itemize}