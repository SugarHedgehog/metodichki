\documentclass[twocolumn]{article}
\usepackage{dashbox}
\setlength{\columnsep}{40pt}
\usepackage[T2A]{fontenc}
\usepackage[utf8]{inputenc}
\usepackage[english,russian]{babel}
\usepackage{graphicx}
\graphicspath{{pictures/}}
\DeclareGraphicsExtensions{.pdf,.png,.jpg}

\linespread{1.15}

\usepackage{egetask}
\usepackage{egetask-math-11-2022}

\def\examyear{2023}
\usepackage[colorlinks,linkcolor=blue]{hyperref}\def\rfoottext{Разрешается свободное копирование в некоммерческих целях с указанием источника }
\def\lfoottext{Источник \href{https://vk.com/egemathika}{https://vk.com/egemathika}}

\begin{document}



\cleardoublepage
\def\examvart{Вариант 1.0}
\normalsize

\begin{center}
	\textbf{
		Единый государственный экзамен\\по МАТЕМАТИКЕ\\Профильный уровень\\ \qquad \\ Инструкция по выполнению работы
	}
\end{center}


\par \qquad Экзаменационная работа состоит из двух частей, включающих в себя 18 заданий. Часть 1 содержит 11 заданий с кратким ответом базового и повышенного уровней сложности. Часть 2 содержит 7 заданий с развёрнутым ответом повышенного и высокого уровней сложности.
\par \qquad На выполнение экзаменационной работы по математике отводится 3 часа 55 минут (235 минут).
\par \qquad Ответы к заданиям 1—11 записываются по приведённому ниже \underline {образцу} в виде целого числа или конечной десятичной дроби. Числа запишите в поля ответов в тексте работы, а затем перенесите их в бланк ответов №1.
%%\includegraphics[width=0.98\linewidth]{obrazec}
\par \qquad При выполнении заданий 12—18 требуется записать полное решение и ответ в бланке ответов №2.
\par \qquad  Все бланки ЕГЭ заполняются яркими чёрными чернилами. Допускается использование гелевой или капиллярной ручки.
\par \qquad При выполнении заданий можно пользоваться черновиком. \textbf{Записи в черновике, а также в тексте контрольных измерительных материалов не учитываются при оценивании работы.}
\par \qquad  Баллы, полученные Вами за выполненные задания, суммируются. Постарайтесь выполнить как можно больше заданий и набрать наибольшее количество баллов.
\par \qquad После завершения работы проверьте, что ответ на каждое задание в бланках ответов №1 и №2 записан под правильным номером.
\begin{center}
	\textit{\textbf{Желаем успеха!}}\\ \qquad \\\textbf{ Справочные материалы} \\
$\sin^2 \alpha + \cos^2 \alpha = 1$ \\
$\sin 2\alpha=2\sin \alpha \cdot \cos \alpha$ \\
$\cos 2\alpha=\cos^2 \alpha-\sin^2 \alpha$ \\
$\sin (\alpha+\beta)=\sin \alpha \cdot \cos \beta+\cos \alpha \cdot \sin\beta$ \\
$\cos (\alpha+\beta)=\cos \alpha \cdot \cos \beta-\sin\alpha \cdot \sin\beta$
\end{center}

\startpartone
\large




\begin{taskBN}{7-1}
\addpictoright[0.4\linewidth]{images/9299084059373277n0}На рисунке изображён график $y=f'(x)$ — производной функции $f(x)$, определенной на интервале $(-3;5)$. В какой точке отрезка $[2; 3]$ функция $f(x)$ принимает наибольшее значение?\vspace{2.5cm}
\end{taskBN}

\begin{taskBN}{7-2}
\addpictoright[0.4\linewidth]{images/776525944899729n0}На рисунке изображен график производной функции $f(x)$, определенной на интервале $(-1; 8)$. Найдите количество точек, в которых касательная к графику функции $f(x)$ параллельна прямой $y=-3x+ 14{,}8 $ или совпадает с ней.\vspace{2.5cm}
\end{taskBN}

\begin{taskBN}{7-3}
\addpictoright[0.4\linewidth]{images/020693809529216n0}На рисунке изображен график функции $y=f(x)$, определенной на интервале $(-6;8)$. Найдите сумму точек экстремума функции $f(x)$.\vspace{2.5cm}
\end{taskBN}

\newpage
 Ответы

\begin{table}[h]\begin{tabular}{|l|l|}
\hline
7-1 & 3
\\
\hline
7-2 & 3
\\
\hline
7-3 & 5
\\
\hline
\end{tabular}\end{table}



\newpage
\end{document}