
\begin{taskBN}{1}
В треугольнике $UVR$ угол $U$ равен $90^\circ$. Сколько составляет  $VR$, если $VU=150$? $\sin{V}=\frac{8}{17}$. 
\end{taskBN}

\begin{taskBN}{2}
Основанием пирамиды является ромб. Чему равен объём пирамиды, если вторая диагональ основания составляет 24? Высота пирамиды составляет 5. Первая диагональ равна 3. 
\end{taskBN}

\begin{taskBN}{3}
В случайном эксперименте симметричную монету бросают 4 раза. Какова вероятность того, что решка выпадет трижды?
\end{taskBN}

\begin{taskBN}{4}
По отзывам заказчиков Станислав Андреевич оценил надёжность двух интернет-магазинов. Вероятность того, что нужный товар доставят из магазина А, равна 0.5. Вероятность того, что этот товар доставят из магазина Б, равна 0.86. Станислав Андреевич заказал товар сразу в обоих магазинах. Считая, что интернет-магазины работают независимо друг от друга, найдите вероятность того, что ни один магазин не доставит товар.
\end{taskBN}

\begin{taskBN}{5}
Найдите корень уравнения $$4x=\sqrt{6x+7}$$ Если корней несколько, в ответе укажите больший из них.
\end{taskBN}

\begin{taskBN}{6}
Найдите значение выражения $$\frac{\log_{6}{36}}{\log_{6}{6}} $$
\end{taskBN}

\begin{taskBN}{7}
\addpictoright[0.8\linewidth]{images/922357129667666n0}На рисунке изображен график производной функции $f(x)$. Найдите абсциссу точки, в которой касательная к графику $y=f(x)$ параллельна прямой $y=-2x+12{,}7$ или совпадает с ней.
\end{taskBN}

\begin{taskBN}{8}
После дождя уровень воды в колодце может повыситься. Мальчик измеряет время $t$ падения небольших камешков в колодец и рассчитывает расстояние до воды по формуле  $h=5t^2$, где $h$ — расстояние в метрах, $t$ — время падения в секундах. До дождя время падения камешков составляло 0,7 с. На сколько должен подняться уровень воды после дождя, чтобы измеряемое время изменилось на 0,4 с? Ответ выразите в метрах.
\end{taskBN}

\begin{taskBN}{9}
Часы со стрелками показывают 10 часов 24 минуты. Через сколько минут минутная стрелка в второй раз поравняется с часовой?
\end{taskBN}

\begin{taskBN}{10}
\addpictocenter[]{images/969616154340588n0}На рисунке изображёны графики двух линейных функций. Найдите абсциссу точки пересечения графиков.
\end{taskBN}

\begin{taskBN}{11}
Определите наибольшее значение функции $y = -25+x^{3}-4x^{2}+6x$ на луче $\left(-\infty;-7 \right]$
\end{taskBN}
%\newpage Ответы

%\begin{table}\begin{tabular}{lll}\\2 & 1 & 170\\2 & 2 & 60\\2 & 3 & 0,25\\2 & 4 & 0,07\\2 & 5 & 0,875\\2 & 6 & 2\\2 & 7 & 5\\2 & 8 & 2\\2 & 9 & 96\\2 & 10 & 48\\2 & 11 & -606\end{tabular}\end{table}