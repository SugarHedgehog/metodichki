\subsection{Хроматическое число графа}

Пусть дано натуральное число \( p \), говорят, что граф \( G \) является \( p \)-хроматическим, если его вершины можно раскрасить \( p \) различными цветами таким образом, чтобы никакие две смежные вершины не были окрашены одинаково. Наименьшее число \( p \), при котором граф \( G \) является \( p \)-хроматическим, называется \textit{хроматическим числом} этого графа и обозначается символом \(\chi(G)\).

Граф \( (X \cup U) \) — симметрический и обладает к тому же весьма примечательным свойством: его можно начертить на плоскости так, чтобы никакие два ребра не пересекались (в точках, отличных от граничных). Такие графы называются \textit{плоскими}. Известно, что хроматическое число плоского графа никогда не превышает 5 (см. гл. 21), таким образом, пяти красок достаточно для раскрашивания карты (плоской), при котором никакие две соседние страны не окрашиваются в один и тот же цвет.

\textit{Хроматическим классом} графа называется натуральное число \( q \), обладающее следующими свойствами:

\begin{enumerate}
    \item каждое ребро графа можно окрасить в какой-нибудь из \( q \) цветов таким образом, чтобы никакие два смежных ребра не были окрашены одинаково;
    \item это невозможно сделать с помощью только \( q - 1 \) цветов.
\end{enumerate}

Хроматический класс графа \( (X \cup U) \) совпадает с хроматическим числом графа \( (U \cup \Gamma) \), определяемого следующим образом: вершинами его служат ребра исходного графа и \( u' \in \Gamma \), когда Граф является двудольным (т.е. имеет хроматическое число 2) в том и только в том случае, если он не содержит циклов нечётной длины.

\subsection*{Доказательство}

(1) Рассмотрим граф \( (X, U) \) без нечётных циклов и покажем, что он — двудольный. Граф можно предполагать связным (в противном случае мы рассмотрели бы все его компоненты связности отдельно). Будем последовательно раскрашивать вершины по следующему правилу:

\begin{enumerate}
    \item[1\textdegree] Произвольную вершину \( a \) окрашиваем в синий цвет.
    \item[2\textdegree] Если вершина \( x \) уже оказалась синей, то все смежные с ней вершины окрашиваем в красный цвет. Если вершина \( y \) — красная, то все смежные с ней окрашиваем в синий цвет.
\end{enumerate}

Так как граф связен, каждая его вершина рано или поздно окажется окрашенной, причём никакая вершина не будет одновременно синей и красной, ибо иначе \( x \) и \( a \) находились бы на одном цикле нечётной длины. Следовательно, граф — двудольный.

(2) Если граф — двудольный, то он, очевидно, не содержит циклов нечётной длины, ибо вершины такого цикла невозможно окрасить двумя цветами в соответствии с указанным требованием.

\subsection*{Замечание}

\textbf{Свойство.} Граф \( G \) не имеет циклов нечётной длины равносильно свойству.

(2) граф \( G \) не имеет элементарных циклов нечётной длины.

Непосредственно ясно, что \((1) \Rightarrow (2)\), для доказательства того, что \((2) \Rightarrow (1)\), допустим что существует цикл \( u = [x_0, x_1, \ldots, x_p = x_0] \) нечётной длины \( p \). Каждый раз, когда имеются такие две вершины \( x_j \) и \( x_k \), что \( j < k < p \) и \( x_j = x_k \), цикл \( u \) можно разбить на два частичных цикла \([x_j, \ldots, x_k]\) и \([x_k, \ldots, x_j]\), причём ровно один из этих двух циклов имеет нечётную длину.

Ясно, что если продолжать таким же образом разбивать цикл \( p \), пока это возможно, то всякий раз будет оставаться в точности один цикл нечётной длины, дойдя в конце концов до элементарных циклов мы получим противоречие с (2).

Эти результаты позволяют легко распознавать бикроматические графы, что же касается других графов, то для них графические методы определения хроматического числа неизвестны. Отметим, однако, что во многих случаях благодаря следующей теореме действенным орудием оказывается понятие функции Гранда.

\textbf{Теорема 4} Пусть \( G \) — симметрический граф. Чтобы граф \( G \) был \( p \)-хроматическим, необходимо и достаточно, чтобы он допускал функцию Гранди \( g(x) \), для которой
\[
\max_{x \in X} g(x) \leq p - 1.
\]

1\textdegree{} Если такая функция \( g(x) \) существует, то граф \( G \) является \( p \)-хроматическим: в самом деле, достаточно числам \( 0, 1, \ldots, p - 1 \) поставить в соответствие различные цвета и окрасить каждую вершину \( x \) в тот цвет, который отвечает числу \( g(x) \).

2\textdegree{} Предположим, что граф \( p \)-хроматический, и докажем, что на \( G \) существует функция Гранди, значения которой не превышают \( p - 1 \).

Пусть \( S_0, S_1, \ldots, S_{p-1} \) — множества вершин с одинаковым цветами. Присоединим к \( S_0 \) все вершины из \( S_1 \), не смежные ни с одной из вершин \( S_0 \). Далее присоединим к \( S_0 \) все вершины из \( S_2 \), не смежные ни с одной из вершин \( S_0 \) и вершин \( S_1 \), ранее присоединенных к \( S_0 \). Затем последовательно поступим подобным же образом с множествами \( S_3, \ldots, S_{p-1} \). В результате получим множество \( \overline{S_0} = S_0 \cup S_1 \cup \ldots \cup S_{p-1} \).

Функция \( g(x) \), равная \( k \) при \( x \in S_k \), есть функция Гранди для графа \( G \), что и требовалось.

\textbf{Теорема 5} Пусть \( G \) — \((p+1)\)-хроматический граф, \( H \) — \((q+1)\)-хроматический граф, обозначим через \( r \) наибольшую из \( d\)-сумм \( p' + q' \), где \( p' \leq p \), \( q' \leq q \), тогда граф \( G \times H \) является \((r+1)\)-хроматическим.

Действительно, всегда можно предположить, что графы \( G \) и \( H \) симметрические (это никак не изменит рёбер графа \( G \times H \)), построим для \( G \) функцию Гранди \( g(x) \) с наибольшим значением, не превосходящим \( p \), а для \( H \) — функцию Гранди \( h(x) \) с наибольшим значением, не превосходящим \( q \) в соответствии с предыдущей теоремой. В силу теоремы 8 (гл. 3) граф \( G \times H \) допускает функцию Гранди с наибольшим значением, не превосходящим \( r \), откуда и следует справедливость утверждения.

Например, читатель легко проверит, что если \( G \) 6-хроматический, а \( H \) 7-хроматический граф, то граф \( G \times H \) является 8-хроматическим, потому что
\[
r = 6 + 1 = (1 \cdot 1) = 7.
\]

\textbf{Теорема 6} Если \( G \) и \( H \) два различных графа с хроматическими числами \( p \) и \( q \), а \( r = \min\{p, q\} \), то граф \( G \times H \) является \( r \)-хроматическим.

Предположим для определенности, что \( p \leq q \), и раскрасим граф \( G \) с помощью \( p \) цветов; в графе \( G \times H \) придумаем вершине \(\xi = (x, y)\) тот же цвет, который имеет \( x \) в \( G \). Тогда смежные вершины графа \( G \times H \) будут иметь различные цвета (ибо иначе в \( G \) имелись бы одинаково окрашенные смежные вершины).

\[
\text{Ч. Т. Д.}
\]

Эту теорему можно выразить еще и так
\[
\gamma(G \times H) \leq \min\{\gamma(G), \gamma(H)\}.
\]

\(^1\) Этот систематический способ, которым мы обязаны Гомори, основывается на симплекс-методе Данцига (G. Dantzig) и в общих чертах состоит в том, что сначала решается обычная линейная программа, а затем если какой-нибудь из переменных в этом решении отвечает нецелое число, то составляется некоторый набор линейных неравенств, которым удовлетворяют все целые решения, но не удовлетворяют уже найденные.

