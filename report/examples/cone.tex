\newpage
\subsubsection*{№318145}

\examples{В сосуде, имеющем форму конуса, уровень жидкости достигает $\frac{2}{3}$ высоты. Объём жидкости равен 864мл. Сколько миллилитров жидкости поместится в весь сосуд?}{2916}{0428175678454303n0}{0.35}

\examples{В сосуде, имеющем форму конуса, уровень жидкости достигает $\frac{1}{2}$ высоты. Объём жидкости равен 72мл. Сколько миллилитров жидкости поместится в весь сосуд?}{576}{075099107295719n0}{0.35}

\subsubsection*{№27052}

\examples{Высота конуса равна $44$. Плоскость, параллельная плоскости основания конуса,  делит его так, что образующие конусов равны $50$ и $55$. Найдите площадь осевого сечения меньшего конуса.}{1200}{680919579681634n0}{0.35}

\newpage
\examples{Высота конуса равна $12$. Плоскость, параллельная плоскости основания конуса,  делит его так, что площади осевых сечений конусов относятся, как $1:16$. Найдите радиус основания меньшего конуса.}{4}{649652582292986n0}{0.35}

\subsubsection*{№27094}

\examples{Во сколько раз уменьшили образующую конуса, если его объём уменьшился в 4 раза? При этом высота не изменилась.}{4}{533652720262299n0}{0.35}

\examples{Во сколько раз увеличили площадь основания конуса, если его длина окружности основания увеличилась в 9 раз?}{81}{377291332736288n0}{0.35}

