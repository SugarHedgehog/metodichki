\documentclass[a4paper, 12pt,fleqn]{article}
\usepackage{fontspec}
\usepackage{polyglossia}
\setmainfont{CMU Serif}
\newfontfamily{\cyrillicfont}{CMU Serif}
\setsansfont{CMU Sans Serif}
\newfontfamily{\cyrillicfontsf}{CMU Sans Serif}
\setmonofont{CMU Typewriter Text}
\newfontfamily{\cyrillicfonttt}{CMU Typewriter Text}
\setdefaultlanguage{russian}
\usepackage{fancyhdr}
%%% Дополнительная работа с математикой
\usepackage{amsfonts,amssymb,amsthm,mathtools}
\usepackage{amsmath}
%%% Отступы
\usepackage[left=1cm, top=0.5cm,right=3cm]{geometry}
\usepackage{fancyhdr}
%%% Номер страницы снизу слева
\pagestyle{fancy}
\fancyhf{}
\fancyfoot[L]{\thepage}
\renewcommand{\headrulewidth}{0pt}
\renewcommand{\footrulewidth}{0pt}
%%%Определение теорем
\newtheorem{theorem}{Теорема}[subsection]
\newenvironment{customthm}[1]
  {\renewcommand\thetheorem{#1}\theorem}
  {\endtheorem}

  \begin{document}
  \pagestyle{fancy}
  \setcounter{page}{18}
  \begin{description}
    \item[б.]Пусть линейный интегральный оператор
  \end{description}
  \begin{equation} \tag{2.5}\label{2.5}
    (Ax)(t)=\underset{Q}{\int} G(t, s)x(s)ds
  \end{equation}
  действует в некотором банаховом пространстве $E$ определенных на функций, причем в
  $E$ выделен конус $K_{+}$ неотрицательных функций. Оператор \eqref{2.5}
  положителен, если ядро
  $G(t, s)$ неотрицательно.
  
  \textbf{2.3. Теорема о непрерывности}. Доказательство непрерывности операторов часто бывает
  связано с громоздкими оценками. Положительные операторы являются счастливым исключением.
  \begin{customthm}{2.1}\label{2.1} Пусть аддитивный оператор $A$ действует из банахова пространства $E_1$ с клином $K_1$ в
  банахово пространство $E_2$ с клином $K_2$. Пусть клин $K_1$ воспроизводящий, а $K_2$ - конус.
  Тогда $A$ линеен.
  \end{customthm}
  
  $\blacksquare$ В силу теоремы 1.4 достаточно установить замкнутость графика $\Gamma(A)$
  оператора $A$. Для этого достаточно установить равенство нулю каждого предела $y\in E_2$
  последовательностей вида $Ax_n$, где $x\in E_1$ и $\| x_n\|\to 0$.
  
  В силу теоремы 1.5 каждому $x_n$ можно сопоставить такой элемент $u_n\in K$, что $-u_n\leq x_n\leq u_n$,
  и $\|u_n\|\leq a\|x_n\|$. Без ограничения общности можно считать
  сходящимся ряд $\|x_1\|+2\|x_2\|+{\dots} +n\|x_n\|+\dots$
  Тогда будет сходиться и ряд $u_1+2u_2+ {\dots} +nu_n+\dots;$ обозначим его сумму через
  $w$. Так как $-w\leq-nu_n\leq nx_n \leq nu_n \leq w$, то из положительности $A$
  вытекают соотношения $-\cfrac{1}{n}Aw\leq Ax_n\leq\cfrac{1}{n}Aw$ и, после перехода к пределу,
   соотношение $0\leq y \leq 0$. Но $K$ является конусом и поэтому $y = 0$. $\blacksquare$
  
  Предположение о том, что клин $K_1$ воспроизводящий, и предположение о том, что $K_2$ является конусом,
  в условиях теоремы \ref{2.1} отбросить нельзя. Теорема \ref{2.1}, в частности, может быть
  применена и к положительным функционалам. Из нее следует непрерывность каждого
  аддитивного функционала, положительного на воспроизводящем клине.
  
  Теорема \ref{2.1} имеет большую историю: для функционалов на телесном конусе она была
  установлена еще М.Г. Крейном, для операторов, преобразующих воспроизводящий конус
  в нормальный, - И.А. Бахтиным, М.А. Красносельским и В.Я. Стеценко (независимо и
  несколько позже - Б.3. Вулихом), в общей форме - Г.Я. Лозановским.
  
  \textbf{2.4. Продолжение положительных функционалов.} В общем анализе важна теорема Хана -
  Банаха о продолжении линейных функционалов с сохранением нормы. В теории
  полуупорядоченных пространств аналогичную роль играет теорема о продолжении
  функционалов с сохранением положительности.
  
  Пусть $E_0$ - подпространство банахова пространства $E$ с клином $K$; пересечение
  $K_0=K\cap  E_0$ - клин в $E_0$. Определенный на $E$ функционал $F$ называется \textit{продолжением}
  определенного на $E_0$ функционала $f$, если $F(x)=f(x)$ при $x\in E_0$. Если функционал $F$
  положителен относительно $K$, то его называют \textit{положительным продолжением} функционала $f$.
  Обсудим возможность построения линейных положительных на $E$ продолжений линейных
  положительных на $E_0$ функционалов. Такое продолжение возможно не всегда.
  
  \end{document}
  