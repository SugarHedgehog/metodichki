\documentclass[a4paper 12pt]{article}
\usepackage[left=0cm,right=0cm,
top=0cm,bottom=0cm,bindingoffset=0cm]{geometry}
%%% Работа с русским языком
\usepackage{cmap}					% поиск в PDF
\usepackage{mathtext} 				% русские буквы в фомулах
\usepackage[T2A]{fontenc}			% кодировка
\usepackage[utf8]{inputenc}			% кодировка исходного текста
\usepackage[english,russian]{babel}	% локализация и переносы

%%% Дополнительная работа с математикой
\usepackage{amsfonts,amssymb,amsthm,mathtools} % AMS
\usepackage{amsmath}
\usepackage{icomma} % "Умная" запятая: $0,2$ --- число, $0, 2$ --- перечисление

%% Номера формул
%\mathtoolsset{showonlyrefs=true} % Показывать номера только у тех формул, на которые есть \eqref{} в тексте.

%% Шрифты
\usepackage{euscript}	 % Шрифт Евклид
\usepackage{mathrsfs} % Красивый матшрифт

%% Свои команды
\DeclareMathOperator{\sgn}{\mathop{sgn}}


%% Перенос знаков в формулах (по Львовскому)
\newcommand*{\hm}[1]{#1\nobreak\discretionary{}
	{\hbox{$\mathsurround=0pt #1$}}{}}

%%% Работа с картинками
\usepackage{graphicx}  % Для вставки рисунков
\graphicspath{{Изображения/}{image}}  % папки с картинками
\setlength\fboxsep{3pt} % Отступ рамки \fbox{} от рисунка
\setlength\fboxrule{1pt} % Толщина линий рамки \fbox{}
\usepackage{wrapfig} % Обтекание рисунков и таблиц текстом

%%% Работа с таблицами
\usepackage{array,tabularx,tabulary,booktabs} % Дополнительная работа с таблицами
\usepackage{longtable}  % Длинные таблицы
\usepackage{multirow} % Слияние строк в таблице
\title{Доклад на тему объективные тесты на интеллект и объективность}
\author{Суматохина Александра 3 курс КТФ}
\date{2 октября 2022}
\begin{document}
\maketitle
\subsection*{Объективные тесты}
Объективный тест \textendash\ это тест, цель которого скрыта от испытуемого, значит 
его результаты нельзя фальсифицировать. 
, и данные, полученные с его помощью, могут быть оценены независимо от лица, 
проводящего тестирование и интерпретацию. 
Преимущество таких тестов состоит в их практическом значении. 
Когда испытуемые не могут произвольно манипулировать показателями, 
такой тест может быть использован в процедурах профотбора.

К числу объективных методик относятся:
\begin{enumerate}
    \item Объективные психологические тесты
    \item Тесты-опросники
    \item Субъективное шкалирование
\end{enumerate} 

\subsection*{Объективные психологические тесты}
Самым известным объективным тестом является тест Германа Роршаха, 
который был создам в 1921 году. Он представляет из себя набор картинок с 
изображением клякс и фигур разных цветов. 

Роршах обнаружил, что те испытуемые, которые видят правильную 
симметричную фигуру в бесформенной чернильной кляксе, обычно 
хорошо понимают реальную ситуацию, способны к самокритике и 
самоконтролю.

Изучая самообладание, понимаемое в основном как господство над 
эмоциями, Роршах использовал чернильные кляксы разных цветов 
(красного, пастельных оттенков) и разной насыщенности серого и 
черного, чтобы ввести факторы, обладающие эмоциональным воздействием. 
Взаимодействие интеллектуального контроля и возникающей эмоции 
определяет то, что испытуемый видит в кляксе. Роршах обнаружил, 
что лица, различное эмоциональное состояние которых было известно 
из клинических наблюдений, действительно по-разному реагируют на 
цвета и оттенки.

Наиболее оригинальное и важное открытие Роршаха, относящееся к 
психодинамике, это ответ, в котором используется 
движение. Некоторые испытуемые видели в чернильных кляксах движущиеся 
человеческие фигуры. Роршах обнаружил, что среди здоровых индивидов 
это чаще всего характерно для тех, кому свойственно богатое 
воображение, а среди лиц с психическими отклонениями \textendash\ для тех, 
кто предрасположен к нереалистичным фантазиям. 

Таким образом, оказалось, что чернильные 
кляксы способны раскрыть глубоко скрытые желания или страхи, 
лежащие в основе длительных неразрешимых личностных конфликтов.
, о том, что делает человека счастливым или печальным, что волнует 
его, а что он вынужден подавлять и переводить в форму подсознательных 
фантазий, может быть извлечена из содержания или «сюжета» ассоциаций, 
вызываемых чернильными пятнами.

\subsection*{Тесты опросники}
Тесты-опросники \textendash\ это тоже тесты с заданными вариантами ответа на 
вопрос. Они применяются в большей степени для 
диагностики личностных черт, а также установок, ценностных 
ориентаций, самооценки.
\subsection*{Субъективное шкалирование}
Методики субъективного шкалирования, в отличие от методов экспертной 
оценки в этом случае шкальные оценки выносит сам испытуемый, а не 
исполнитель методики. Испытуемый оценивает внешние объекты или понятия, 
а выводы делаются о нем самом. 
\subsection*{Тесты интеллекта}
Тесты интеллекта \textendash\ тесты общих способностей. Предназначены для 
измерения уровня интеллектуального развития являются наиболее 
распространенными в психодиагностике. Проявления интеллекта 
многообразны, но им присуще то общее, что позволяет отличить 
их от других особенностей поведения. А именно \textendash\ вовлечение в 
любой интеллектуальный акт мышления, памяти, воображения, всех 
психических функций, которые обеспечивают познание окружающего 
мира. Тесты для оценки различных интеллектуальных функций – это 
тесты логического мышления, смысловой и ассоциативной памяти, 
арифметические, пространственной визуализации и т. д. 
\subsection*{Тесты на креативность}
Тест креативности – методики, предназначенные для изучения и 
оценки творческих способностей личности.

Креативность – способность продуцировать новые идеи, находить 
нетрадиционные способы решения проблемных задач. 
Факторы креативности – беглость, четкость, гибкость мышления, 
чувствительность к проблемам, оригинальность, изобретательность, 
конструктивность при их решении и др.

Если решение тестов креативности может приниматься как одно из 
свидетельств наличия творческих способностей у человека, то не 
решение их еще не является доказательством отсутствия таковых.

\end{document}