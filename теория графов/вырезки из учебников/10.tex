\subsection{Цикломатическое число графа}

Понятие, которое мы собираемся здесь ввести не зависит от ''ориентации''. Для большей общности введем в рассмотрение не просто графы а \textit{мультиграфы}; по определению, \textit{мультиграф} $(X, U)$ это пара, образованная множеством $X$ вершин и множеством $U$ ребер соединяющих некоторые пары вершин, в противоположность обычным графам у мультиграфа одна и та же пара вершин может соединяться более чем одним ребром.

Во многих задачах удобно вместо обычных графов рассматривать мультиграфы.

\textbf{Пример (химия).} Молекула представляется мультиграфом, вершины которого обозначены символами таблицы Менделеева (Пойа [4] применил теорию графов к органической химии для подсчета числа изомеров химических соединений). Говорят также, что этилен является 2-графом, ацетилен --- 3-графом, и т.д. (см. рис. 4--1)

\begin{figure}[h]
\centering
\begin{tikzpicture}
% Здесь можно добавить код для отрисовки молекул этилена и ацетилена
\end{tikzpicture}
\caption{Этилен \hspace{2cm} Ацетилен}
\label{fig:4-1}
\end{figure}

Рассмотрим мультиграф $G$ с $n$ вершинами, $m$ ребрами, $p$ компонентами связности. Положим

\[\rho(G) = n - p,\]
\[\alpha(G) = m - \rho(G) = m - n + p,\]

$\nu(G)$ называется \textit{цикломатическим числом} мультиграфа $G$. Его свойства играют важную роль\footnote{Цикломатическим числом графа называется цикломатическое число того 2-графа который получится, если каждую дугу заменить ребром -- Прим перев}

\textbf{Теорема 1} Пусть $G$ --- мультиграф полученный из мультиграфа $\bar{G}$ добавлением нового ребра между вершинами $a$ и $b$; если $a$ и $b$ совпадают или могут быть соединены цепью в $\bar{G}$, то

\[\rho(G') = \rho(\bar{G}), \quad \nu(G') = \nu(\bar{G}) + 1,\]

в противном случае

\[\rho(G) = \rho(\bar{G}) + 1, \quad \nu(G') = \nu(\bar{G})\]

(Непосредственно)

\textbf{Следствие} \quad $\rho(\bar{G}) \geq 0, \quad \nu(\bar{G}) \geq 0$

В самом деле, для графа, образованного всеми вершинами $G$, но без ребер имеем $\rho = 0$, $\nu = 0$. Каждое добавление ребра либо увеличивает $\rho$, не меняя $\nu$ либо увеличивает $\nu$. Таким образом, в процессе построения графа $G$ числа $\rho$ и $\nu$ могут только возрастать.

Для дальнейшего удобно следующим образом отождествлять циклы мультиграфа с векторами: придадим каждому ребру мультиграфа $G$ произвольную ориентацию, если цикл $\mu$ проходит через ребро $u_k$ в направлении его ориентации $r_k$ раз и в противоположном направлении $s_k$ раз, то полагаем $c^k = r_k - s_k$. Вектор

\[(c^1, c^2, \ldots, c^k, \ldots, c^m)\]

$m$-мерного пространства $\mathbb{R}^m$ будем называть \textit{вектором-циклом}, соответствующим циклу $\mu$ и обозначать через $\boldsymbol{\mu}$ (или опять через $\mu$, если это не может привести к недоразумению).

Циклы $\mu$, $\mu'$, $\mu''$, называются \textit{независимыми}, если соответствующие им векторы линейно независимы\footnote{Напомним некоторые классические определения линейной алгебры}. Отметим, что это свойство не зависит от выбора ориентации ребер.

Если $\mathbf{c} = (c^1, c^2, \ldots, c^m)$ и $\mathbf{d} = (d^1, d^2, \ldots, d^m)$ --- два вектора из $\mathbb{R}^m$, а $\alpha \in \mathbb{R}$, то полагаем

\[\begin{aligned}
\alpha\mathbf{c} &= (\alpha c^1, \alpha c^2, \ldots, \alpha c^m) \\
-\mathbf{c} &= (-c^1, -c^2, \ldots, -c^m) \\
\mathbf{c} + \mathbf{d} &= (c^1 + d^1, c^2 + d^2, \ldots, c^m + d^m) \\
\mathbf{0} &= (0, 0, \ldots, 0)
\end{aligned}\]

Множество $E \subset \mathbb{R}^m$ представляет собой \textit{векторное подпространство} если

\[\begin{aligned}
\alpha \in \mathbb{R}, \quad \mathbf{c} \in E &\Rightarrow \alpha\mathbf{c} \in E \\
\mathbf{c}, \mathbf{d} \in E &\Rightarrow \mathbf{c} + \mathbf{d} \in E
\end{aligned}\]

Говорят, что векторы $\mathbf{c}_1, \ldots, \mathbf{c}_k$ из $\mathbb{R}^m$ \textit{линейно независимы}, если

\[\alpha_1\mathbf{c}_1 + \alpha_2\mathbf{c}_2 + \cdots + \alpha_k\mathbf{c}_k = \mathbf{0} \Rightarrow \alpha_1 = \alpha_2 = \cdots = \alpha_k = 0\]

Напротив, когда для некоторых чисел $\alpha_i$ не равных одновременно нулю,
\[\alpha_1\mathbf{c}_1 + \alpha_2\mathbf{c}_2 + \cdots + \alpha_k\mathbf{c}_k = \mathbf{0},\] говорят, что данные векторы \textit{линейно}

\begin{theorem} 2
    Цикломатическое число $\nu(G)$ мультиграфа $G$ равно наибольшему количеству независимых циклов.
    \end{theorem}
    
    Действительно, возьмем граф без ребер, образованный всеми вершинами $G$ и, добавляя к нему ребро за ребром, построим данный мультиграф $G$. В силу теоремы 1 цикломатическое число увеличивается на единицу, когда добавление ребра приводит к образованию новых циклов и не меняется в противоположном случае. Допустим, что перед добавлением ребра $u_k$ уже имелась база, состоящая из независимых циклов $\rho_1$, $\rho_2$, \ldots, и что добавление $u_k$ повлекло за собой возникновение циклов $\nu_1$, $\nu_2$, \ldots. Среди новых циклов наверняка имеются простые, пусть, например, $\nu_1$ --- простой, $\nu_1^k = 1$. Очевидно $\nu_1$ не может линейно выражаться через $\rho_i$ (ибо $\mu_1^k = \mu_2^k = \cdots = 0$). С другой стороны, $\nu_2$ (и аналогично $\nu_3$, \ldots) можно линейно выразить через $\nu_1$, $\mu_1$, $\mu_2$, \ldots, в самом деле, вектор $\nu_2 - \lambda_1^2\nu_1$ соответствует некоторому циклу, не содержащему $u_k$ (этот цикл получается из $\nu_2$ заменой ребра $u_k$ оставшейся частью $\nu_1$ с измененным направлением обхода), и, значит линейно выражается через $\mu_1$, $\mu_2$, \ldots. Таким образом каждый шаг, увеличивающий на единицу цикломатическое число, в то же время увеличивает на единицу наибольшее количество независимых циклов. Теорема доказана.
    
    \begin{corollary} 1
    Граф $G$ не имеет циклов тогда и только тогда, когда $\nu(G) = 0$.
    \end{corollary}
    
    \begin{corollary} 2
    Граф $G$ имеет один единственный цикл тогда и только тогда, когда $\nu(G) = 1$.
    \end{corollary}
    
    \begin{theorem} 3
    В сильно связном графе цикломатическое число равно наибольшему количеству независимых контуров.
    \end{theorem}

    В самом деле, рассмотрим 2-граф, получающийся заменой дуг данного графа $G$ ребрами, и элементарный цикл $\mu$, вершины, встречающиеся в цикле $\mu$, можно распределить по следующим множествам множество $S$ точек обладающих тем свойством, что одна из дуг $\mu$ исходит из точки, а другая заходит в нее, множество $S'$

зависимы Если $a_1 \neq 0$, то можно также написать

$$c_1 = \frac{a_2}{a_1}c_2 + ... + \frac{a_k}{a_1}c_k,$$

в этом случае говорят, что $c_1$ линейно выражается через $c_2$, $c_3$ ..., $c_k$

База векторного подпространства $E$ есть такое множество векторов $\{e_1, e_2, ..., e_k\}$ из $E$, что каждый вектор подпространства $E$ линейно выражается через векторы $e_i$, наименьшее из чисел $k$ называется размерностью подпространства $E$.

В $E = \mathbb{R}^m$ одна из баз образуется векторами

$$e_1 = (1, 0, 0, ..., 0), e_2 = (0, 1, 0, ..., 0), ..., e_m = (0, 0, 0, ..., 1),$$

точек, из которых исходит по две дуги $\mu$, множество $S''$ точек, в которые заходит по две дуги $\mu$ (см. рис 4-2).

Так как количество конечных точек равно количеству начальных то $|S'| = |S''|$, итак, пусть $x'_1, x'_2, ..., x'_k$ - элементы $S'$, а $x''_1, x''_2, ..., x''_k$ - элементы $S''$

В цикле $\mu$ элементы $S'$ чередуются с элементами $S'$, и мы предположим нумерацию вершин такой что первая вершина после $x'_i$, не принадлежащая $S$ есть $x_i$, наконец, если $\nu_0$ — путь, в котором вершина $x$ встречается раньше вершины $y$, то обозначим через $\mu_0[x,y]$ частичный путь из $x$ в $y$. Поскольку граф сильно связен, существует контур $\nu_i$, проходящий через $x'_{i+1}$ и $x'_i$ и содержащий дуги \mu на пути от $x'_{i+1}$ к $x'_i$. Цикл \mu является линейной комбинацией контуров ибо можно написать

$\mu = \mu[x'_1,x''_1] - \nu_1[x'_2,x'_1] + \mu[x'_2,x''_2] + \cdots = \mu[x'_1,x_1] + \nu_1[x_1,x'_2] + \mu[x'_2,x''_2] + \nu_2[x_2,x'_3] + \cdots = -(\nu_1 + \nu_2 + \cdots)$

Значит каждый элементарный цикл есть линейная комбинация контуров и то же справедливо для произвольного цикла (поскольку он является линейной комбинацией элементарных)

В $R^n$ контуры образуют базу векторного подпространства, порожденного циклами и в силу теоремы 2 эта база имеет размерность $\nu(G)$; поэтому наибольшее число независимых контуров равно $\nu(G)$