\documentclass[10pt, twoside]{article}
\usepackage[russian]{babel}
\usepackage[utf8]{inputenc}
\usepackage[T2A]{fontenc}
\usepackage{amsmath}
\usepackage{amsfonts}
\usepackage{amssymb}
\usepackage{amsthm}
\usepackage{stmaryrd}
\usepackage{mathrsfs}
\usepackage{fancyhdr}

\usepackage[symbol]{footmisc}
\renewcommand{\thefootnote}{\fnsymbol{footnote}}
\renewcommand{\thefootnote}{}

\newtheorem*{theorem}{Теорема}
\pagestyle{fancy}
\fancyhf{}
\fancyhead[LE,RO]{\thepage}
\fancyhead[LO,RE]{\thesubtopic]}
\fancyhead[CO]{\S 5. СВОЙСТВА НЕПРЕРЫВНЫХ ФУНКЦИЙ}
\fancyhead[CE]{ГЛ. II. ФУНКЦИИ ОДНОЙ ПЕРЕМЕННОЙ}
\renewcommand{\headrulewidth}{0pt}

\newcounter{subtopic}

\newcommand{\subtopic}[1]{%
  \stepcounter{subtopic}%
  \gdef\thesubtopic{\arabic{subtopic}}
  \textbf{\arabic{subtopic}. #1}
}

\begin{document}
\setcounter{page}{172}
\setcounter{subtopic}{82}
Мы установили, таким образом, важное свойство функции $f(x)$, непрерывной в промежутке: \textit{переходя от одного своего значения к другому, функция хоть раз принимает, в качестве значения, каждое промежуточное число.}

Иными словами это свойство можно выразить и так: \textit{значения, принимаемые непрерывной фикцией $f(x)$, когда $x$ изменяется в каком-либо промежутке $\mathscr{X}$, сами также заполняют сплошь некоторый промежуток $\mathscr{Y}$.}

Действительно, пусть

$$
  m=\inf f(x), \quad M=\sup f(x)*)
$$
\footnotetext{*) Напоминаем читателю, что если множество $\{f(x)\}$ не ограничено сверху (снизу), то мы условились в 11 полагать $M=+\infty$ ( $m=-\infty$ ).
}
и $y_{0}$ есть произвольное число между $m$ и $M$ :

$$
  m<y_{0}<M .
$$

Необходимо найдутся значения функции $y_{1}=f\left(x_{1}\right)$ и $y_{2}=f\left(x_{2}\right)$ ( $x_{1}$ и $x_{2}$ взяты из промежутка $\mathscr{X}$), такие, что

$$
  m \leqslant y_{1}<y_{0}<y_{2} \leqslant M ;
$$

это вытекает из самого определения точных границ числового множества. Но тогда, по доказанной теореме, существует между $x_{1}$ и $x_{2}$ такое значение $x=x_{0}$ (очевидно, также принадлежащее $\mathscr{X}$), что $f\left(x_{0}\right)$ в точности равно $y_{0}$; следовательно, это число входит в множество $\mathscr{Y}$.

Таким образом, $\mathscr{Y}$ представляет собой промежуток с концами $m$ и $M$ (которые сами могут ему принадлежать или нет - смотря по случаю; ср. \textbf{84}).

Мы видели в \textbf{71}, 2${}^{\circ}$, что в случае монотонной функции упомянутое свойство, обратно, влечет за собой непрерывность. Однако не следует думать, что так будет всегда; легко построить заведомо разрывные функции, которые все же этим свойством обладают. Например, значения функции [\textbf{70}, 4)]:

$$
  f(x)=\sin \frac{1}{x} \quad(x \neq 0), \quad f(0)=0,
$$

когда $x$ изменяется в каком-либо промежутке, содержащем точку разрыва $x=0$, заполняют сплошь промежуток $[-1,+1]$.

\subtopic{Существование обратной функции}. Применим изученные в предыдущем $n^{\circ}$ свойства непрерывной функции к установлению, при некоторых предположениях, существования однозначной обратной функции и ее непрерывности [cp. \textbf{49}].

\begin{theorem}
  Пусть функция $y=f(x)$ определена, монотонно возрастает (убывает)**) и непрерывна в некотором промежутке $\mathcal{X}$.
\end{theorem}

\footnotetext{
  **) В строгом смысле слова (это здесь существенно).
}


\newpage
\setcounter{page}{174}
\setcounter{subtopic}{83}
С помощью доказанной теоремы можно наново установить ряд уже известных нам результатов.

Если применить ее к функции $x^{n}$ ( $n$ - натуральное число) в промежутке $\mathscr{X}=[0,+\infty)$, то придем к существованию и непрерывности (арифметического) корня $x=\sqrt[n]{y}$ для $y$ в $\mathscr{Y}=[0,+\infty$).
Исходя из функции $y=a^{x}$ в промежутке $\mathscr{X}=(-\infty,+\infty)$, докажем существование и непрерывность логарифма $x=\log _{a} y$ в промежутке $\mathscr{Y}=(0,+\infty)$. Наконец, рассматривая функции $y=\sin x$ и $y=\operatorname{tg} x$, первую - в промежутке $\mathscr{X}_{1}=\left[-\frac{\pi}{2}, \frac{\pi}{2}\right]$, а вторую - в открытом промежутке $\mathscr{X}_{2}=\left(-\frac{\pi}{2}, \frac{\pi}{2}\right)$, убедимся в существовании и непрерывности обратных им функций $x=\arcsin y$ и $x=\operatorname{arctg} y$, соответственно, в промежутках $\mathscr{Y}=[-1,+1]$ и, $\mathscr{Y}_{2}=(-\infty,+\infty)$.

  [При этом предполагается, что предварительно уже доказана непрерывность функций $x^{n}$, $a^{x}$, $\sin x$, $\operatorname{tg} x$ - без ссылки на существование обратных им функций (иначе получился бы порочный круг). Такие доказательства и были даны в \textbf{68}; соображения же $n^{\circ}$ \textbf{72}, очевидно, здесь непригодны.]

Рассмотрим еще такой пример.

Пусть для $x$ в $\mathscr{X}=(-\infty,+\infty)$


\begin{equation*}
  y=x-\varepsilon \cdot \sin x, \quad \text { где } 0<\varepsilon<1 . \tag{3}
\end{equation*}


Легко показать, что эта функция будет монотонно возрастающей (в узком смысле). Именно, если $x''>x'$ и $y', y''$ - соответствующие значения $y$, то

$$
  y''-y'=\left(x''-x'\right)-\varepsilon\left(\sin x''-\sin x'\right).
$$

Но [см. (2), \textbf{68}]

$$
  \left|\sin x''-\sin x'\right| \leqslant x''-x',
$$

откуда и следует, что

$$
  y''-y'>0, \quad \text { т.е. } y''>y' .
$$

Применив к этому случаю теорему, убеждаемся в том, что и $x$ является однозначной функцией от $y$, и т. д.

Приведенный пример представляет интерес тем, что соприкасается с одной задачей теоретической астрономии. Уравнение


\begin{equation*}
  E=M+\varepsilon \cdot \sin E \tag{3a}
\end{equation*}


есть знаменитое уравнение Кеплера, которое связывает среднюю аномалию $M$ планеты с ее эксцентрической аномалией $E$ ( $\varepsilon$ есть эксцентриситет планетной орбиты). Мы доказали, таким образом, что, каково бы ни было значение средней аномалии, уравнение Кеплера, действительно, однозначно определяет значение эксцентрической аномалии.

\subtopic{Теорема об ограниченности функции.} Если функция $f(x)$ определена (следовательно, принимает конечные значения) для всех значений $x$ в некотором конечном промежутке, то это не влечёт за собой с необходимостью ограниченности функции, т.е. ограниченности множества $\{f(x)\}$ принимаемых ею значений. Например, пусть функция $f(x)$ определена так:

$$
  f(x)=\frac{1}{x}, \quad \text { если } \quad 0<x \leqslant 1, \quad \text { и } \quad f(0)=0 .
$$

Функция эта принимает только конечные значения, но она не ограничена, ибо при приближении $x$ к $0$ может принимать сколь угодно большие значения. Заметим попутно, что в полуоткрытом промежутке $(0,1]$ она непрерывна, но в точке $x=0$ имеет разрыв.

Иначе обстоит дело с функциями, непрерывными в замкнутом промежутке.

\textit{\textbf{Первая теорема Вейерштрасса.} Если функция $f(x)$ определена и непрерывна в замкнуто.и промежутке $[a, b]$, то она ограничена, т.е. существуют такие постоянные и конечные числа $m$ и $M$, что}

  $$
    m \leqslant f(x) \leqslant M \text{ при } a \leqslant x \leqslant b .
  $$

Доказательство поведем от противного: допустим, что функция $f(x)$ при изменении $x$ в промежутке $[a, b]$ оказывается неограниченной.

В таком случае для каждого натурального числа $n$ найдется в промежутке $[a, b]$ такое значение $x=x_{n}$, что


\begin{equation*}
  \left|f\left(x_{n}\right)\right| \geqslant  n. \tag{4}\label{4}
\end{equation*}


По лемме Больцано-Вейерштрасса [\textbf{41}], из последовательности $\left\{x_{n}\right\}$ можно извлечь частичную последовательность $\left\{x_{n_{k}}\right\}$, сходящуюся к конечному пределу:

$$
  \left.x_{n_{k}} \to x_{0} \quad \text { (при } k \to+\infty\right),
$$

причем, очевидно, $a \leqslant x_{0} \leqslant b$. Вследствие непрерывности функции в точке $x_{0}$, тогда должно быть и

$$
  f\left(x_{n_{k}}\right) \to f\left(x_{0}\right),
$$

а это невозможно, так как из \eqref{4} следует, что

$$
  \left|f\left(x_{n_{k}}\right)\right| \to \infty,
$$

Полученное противоречие и доказывает теорему.\\
\subtopic{Наибольшее и наименьшее значения функции.} Мы знаем, что бесконечное числовое множество, даже ограниченное, может не иметь в своем составе наибольшего (наименьшего) элемента. Если функция $f(x)$ определена и даже ограничена в некотором промежутке изменения $x$, то в составе множества ее значений $\{f(x)\}$ может не оказаться наибольшего (наименьшего). В этом случае точная верхняя (нижняя) граница значений функции $f(x)$

\newpage
\setcounter{page}{177}
\setcounter{subtopic}{85}
Аналогично может быть доказано утверждение и относительно наименьшего значения.

II-е доказательство. Можно и здесь исходить из леммы Больцано-Вейерштрасса [\textbf{41}]. Ограничимся утверждением о наибольшем значении. Если, как и только что,

$$
  M=\sup \{f(x)\},
$$

то по свойству точной верхней границы [\textbf{11}], для любого $n$ найдется такое $x=x_{n}$ в $[a, b]$, что


\begin{equation*}
  f\left(x_{n}\right)>M-\frac{1}{n}. \tag{5}\label{5}
\end{equation*}


Тогда из последовательности $\left\{x_{n}\right\}$ может быть извлечена частичная последовательность $\left\{x_{n_{k}}\right\}$, сходящаяся к некоторому значению $x_{0}$ из $[a, b]$: $x_{n_{k}} \to x_{0}$, так что, ввиду непрерывности функции, и

$$
  f\left(x_{l_{k}}\right) \to f\left(x_{0}\right) .
$$

В то же время из \eqref{5} имеем

$$
  f\left(x_{n_{k}}\right)>M-\frac{1}{n_{k}} \text { и, } \quad \text { в пределе, } \quad f\left(x_{0}\right) \geqslant  M .
$$

Но $f\left(x_{0}\right)$ не может быть больше верхней границы $M$ множества значений функции и, следовательно,

$$
  f\left(x_{0}\right)=M,
$$

что и требовалось доказать.

Отметим, что оба приведенные доказательства суть чистые «доказательства существования». Средств для вычисления, например, значения $x=x_{0}$ никаких не дано. Впоследствии [в главе IV, § 1], правда, при более тяжелых предположениях относительно функции, мы научимся фактически находить значения независимой переменной, доставляющие функции наибольшее или наименьшее значения.

Если функция $f(x)$, при изменении $x$ в каком-либо промежутке $\mathcal{X}$, ограничена, то ее колебанием в этом промежутке называется разность

$$
  \omega=M-m .
$$

Иначе можно определить колебание $\omega$ как точную верхнюю границу множества всевозможных разностей $f\left(x''\right)-f\left(x'\right)$, где $x'$ и $x''$ принимают независимо одно от другого произвольные значения в промежутке $\mathscr{X}$:

$$
  \omega=\sup _{x', x'' \text { из } \mathscr{X}} \left\{f\left(x''\right)-f\left(x'\right)\right\} .
$$

Когда речь идет о непрерывной функции $f(x)$ в замкнутом конечном промежутке $\mathscr{X}=[a, b]$, то, как следует из доказанной теоремы, колебанием будет попросту разность между

\newpage
\setcounter{page}{179}
\setcounter{subtopic}{86}

прос: существует ли, при заданном $\varepsilon$, такое $\delta$, которое годилось бы для всех точек $x_{0}$ из этого промежутка?

\textit{Если для каждого числа $\varepsilon>0$ найдется такое число $\delta>0$, что}

$$
  \left|x-x_{0}\right|<\delta \text { влечет за собой }\left|f(x)-f\left(x_{0}\right)\right|<\varepsilon \text {, }
$$

\textit{где бы в пределах рассматриваемого промежутка $\mathscr{X}$ не лежали точки $x_{0}$ и $x$, то функцию $f(x)$ называют равномерно непрерывной в промежутке $\mathscr{X}$.}

В этом случае число $\delta$ оказывается зависящим только от $\varepsilon$ и может быть указано до выбора точки $x_{0}: \delta$ годится для всех $x_{0}$ одновременно.

Равномерная непрерывность означает, что во всех частях промежутка достаточна одна и та же степень близости двух значений аргумента, чтобы добиться заданной степени близости соответствующих значений функции.

Можно показать на примере, что непрерывность функции во всех точках промежутка не влечет необходимо за собой ее равномерной непрерывности в этом промежутке. Пусть, например, $f(x)=\sin \frac{1}{x}$ для $x$, содержащихся между $0$ и $\frac{2}{\pi}$, исключая $0$ . В этом случае область изменения $x$ есть незамкнуты й промежуток $\left(0, \frac{2}{\pi}\right]$, и в каждой его точке функция непрерывна. Положим теперь $x_{0}=\frac{2}{(2 n+1) \pi}, x=\frac{1}{n \pi}$ (где $n$ - любое натуральное число); тогда

$$
  f\left(x_{0}\right)=\sin (2 n+1) \frac{\pi}{2}= \pm 1, \quad f(x)=\sin n \pi=0,
$$

так что

$$
  \left|f(x)-f\left(x_{0}\right)\right|=1,
$$

несмотря на то, что $\left|x-x_{0}\right|=\frac{1}{n(2 n+1) \pi}$ с возрастанием $n$ может быть сделано сколь угодно малым. Здесь при $\varepsilon=1$ нельзя найти $\delta$, которое годилось бы одновременно для всех точек $x_{0}$ в $\left(0, \frac{2}{\pi}\right]$, хотя для каждого отдельного значения $x_{0}$, ввиду непрерывности функции, такое $\delta$ существует!

Весьма замечательно, что в замкнутом промежутке $[a, b]$ аналогичного положения вещей быть уже не может, как явствует из следующей теоремы, принадлежащей Кантору (G. Cantor).

\subtopic{Теорема Кантора} Если функция $f(x)$ определена и непрерывна в замкнутом промежутке $[a, b]$, то она и равномерно непрерывна в этом промежутке.

Доказательство поведем от противного. Пусть для некоторого определенного числа $\varepsilon>0$ не существует такого числа\\
$\delta>0$, о котором идет речь в определении равномерной непрерывности. В таком случае, какое бы число $\delta>0$ ни взять, найдутся в промежутке $[a, b]$ такие два значения $x_{0}'$ и $x'$, что

$$
  \left|x'-x_{0}'\right|<\delta, \quad \text { и тем не менее }\left|f\left(x'\right)-f\left(x_{0}'\right)\right| \geqslant \varepsilon .
$$

Возьмем теперь последовательность $\left\{\delta_{n}\right\}$ положительных чисел так, что $\delta_{n} \to 0$.

В силу сказанного, для каждого $\delta_{n}$ найдутся в $[a, b]$ значения $x_{0}^{(n)}$ и $x^{(n)}$ (они играют роль $x_{0}'$ и $x'$ ), такие, что (при $n=1,2,3, \ldots$ )

$$
  \left|x^{(n)}-x_{0}^{(n)}\right|<\delta_{n}, \quad \text { и тем не менее }\left|f\left(x^{(n)}\right)-f\left(x_{0}^{(n)}\right)\right| \geqslant \varepsilon .
$$

По лемме Больцано-Вейерштрасса [\textbf{41}] из ограниченной последовательности $\left\{x^{(n)}\right\}$ можно извлечь частичную последовательность, сходящуюся к некоторой точке $x_{0}$ промежутка $[a, b]$. Для того чтобы не осложнять обозначений, будем считать, что уже сама последовательность $\left\{x^{(n)}\right\}$ сходится к $x_{0}$.

Так как $x^{(n)}-x_{0}^{(n)} \to 0$ (ибо $\left|x^{(n)}-x_{0}^{(n)}\right|<\delta_{n}$, а $\delta_{n} \to 0$ ), то одновременно и последовательность $\left\{x_{0}^{(n)}\right\}$ сходится к $x_{0}$. Тогда, ввиду непрерывности функции в точке $x_{0}$, должно быть

$$
  f\left(x^{(n)}\right) \to f\left(x_{0}\right) \quad \text { и } \quad f\left(x_{0}^{(n)}\right) \to f\left(x_{0}\right),
$$

так что

$$
  f\left(x^{(n)}\right)-f\left(x_{0}^{(n)}\right) \to 0,
$$

а это противоречит тому, что при всех значениях $n$

$$
  \left|f\left(x^{(n)}\right)-f\left(x_{0}^{(n)}\right)\right| \geqslant \varepsilon
$$

Теорема доказана.

Из доказанной теоремы непосредственно вытекает такое следствие, которое ниже будет нам полезно:

\textit{\textbf{Следствие.} Пусть функция $f(x)$ определена и непрерывна в замкнутом промежутке $[a, b]$. Тогда по заданному $\varepsilon>0$ найдется такое $\delta>0$, что если промежуток произвольно разбить на частичные промежутки с длинами, меньшими $\delta$, то в каждом из них колебание функции $f(x)$ будет меньше $\varepsilon$.}

Действительно, если, по заданному $\varepsilon$, в качестве $\delta$ взять число, о котором говорится в определении равномерной непрерывности, то в частичном промежутке с длиной, меньшей $\delta$, разность между любыми двумя значениями функции будет по абсолютной величине меньше $\varepsilon$. В частности, это справедливо и относительно н а и большего и наименьшего из этих значений, разность которых и дает колебание функции в упомянутом частичном промежутке [\textbf{85}].

\subtopic{Лемма Бореля} Мы докажем сейчас одно интересное вспомогательное утверждение, которое - подобно лемме Больцано-Вейерштрасса - может быть полезно при проведении многих тонких рассуждений; оно принадлежит Борелю (Е. Borel).


Рассмотрим, наряду с промежутком $[a, b]$, еще некоторую систему $\sum$ открытых промежутков $\sigma$, которая может быть как конечной, так и бесконечной. Условимся говорить, что си стема $\sum$ покрывает промежуток $[a, b]$ (или что этот промежуток покрывается системой $\sum$, и т. п.), если для каждой точки $x$ промежутка $[a, b]$ найдется в $\sum$ промежуток $\sigma$, содержащий ее. Этот способ речи облегчит нам формулировку и доказательство упомянутого утверждения.

\textit{\textbf{Лемма Бореля.}} Если замкнутыи й промежуток $[a, b]$ покрывается бесконечной системой $\Sigma=\{\sigma\}$ открытых промежутков, то из неё всегда можно выделить конечную подсистему

$$
  \Sigma^{*}=\left\{\sigma_{1}, \sigma_{2}, \ldots, \sigma_{n}\right\},
$$

\textit{которая также покрывает весь промежуток $[a, b]$.}

I-е доказательство поведем от противного, применив метод Больцано [\textbf{41}]. Допустим же, что промежуток $[a, b]$ не может быть покрыт конечным числом промежутков $\sigma$ из $\sum$. Разделим промежуток $[a, b]$ пополам. Тогда хоть одна из половин его тоже не может быть покрыта конечным числом $\sigma$; действительно, если бы одна из них могла быть покрыта промежутками $\sigma_{1}, \sigma_{2}, \ldots, \sigma_{m}$ (из $\Sigma$ ), а другая - промежутками $\sigma_{m+1}, \sigma_{m+2}, \ldots, \sigma_{n}$ (из $\sum$ ), то из в с ех этих промежутков составилась бы конечная система $\sum^{*}$, покрывающая уже весь промежуток $[a, b]$, вопреки допущению. Обозначим через $\left[a_{1}, b_{1}\right]$ ту половину промежутка, которая не покрывается конечным числом $\sigma$ (если же обе таковы, то - любую из них). Этот промежуток снова разделим пополам и обозначим через $\left[a_{2}, b_{2}\right]$ ту из его половин, которую нельзя покрыть конечным числом $\sigma$, и т. д.

  Продолжая этот процесс неограниченно, мы получим бесконечную последовательность вложенных промежутков $\left[a_{n}, b_{n}\right](n=1,2,3, \ldots)$, каждый из которых составляет половину предшествующего. Промежутки эти все выбираются так, что ни один из них не покрывается конечным числом промежутков $\sigma$. По лемме о вложенных промежутках [\textbf{38}], существует общая им всем точка $c$, к которой стремятся концы $a_{n}, b_{n}$.

  Эта точка $c$, как и всякая точка промежутка $[a, b]$, лежит в одном из промежутков $\sigma$, скажем в $\sigma_{0}=(\alpha, \beta)$, так что $\alpha<c<\beta$. Но варианты $a_{n}$ и $b_{n}$, стремящиеся к $c$, начиная с некоторого номера будут сами содержаться между $\alpha$ и $\beta$ $[26,1^{\circ}]$, так что определяемый ими промежуток $\left[a_{n}, b_{n}\right]$ окажется покрытым всего лишь одним промежутком $\sigma_{0}$, вопреки самому выбору этих промежутков $\left[a_{n}, b_{n}\right]$. Полученное противоречие и доказывает лемму.

  Приведем еще одно доказательство, построенное на новой идее; она принадлежит Лебегу (H. Lebesgue).

  II-е доказательство. Рассмотрим точки $x^{*}$ промежутка $[a, b]$, обладающие тем свойством, что промежуток $\left[a, x^{*}\right]$ покрывается конечным числом промежутков $\sigma$. Такие точки $x^{*}$, вообще, найдутся: так как, например, точка $a$ лежит в одном из $\sigma$, то и все близлежащие к ней точки содержатся в этом $\sigma$ и, следовательно, оказываются точками $x^{*}$.

  Нашей задачей является установить, что и точка $b$ принадлежит к числу точек $x^{*}$.

  Так как все $x^{*} \leqslant b$, то существует [\textbf{11}] и

$$
  \sup \left\{x^{*}\right\}=c \leqslant b
$$

Как и всякая точка промежутка $[a, b], c$ принадлежит некоторому $\sigma_{0}=(\alpha, \beta)$, $\alpha<c<\beta$. Но, по свойству точной верхней границы, найдется $x_{0}^{*}$, такое, что $\alpha<x_{0}^{*} \leqslant c$. Промежуток $\left[a, x_{0}^{*}\right]$ покрывается конечным числом промежутков $\sigma$ (по самому определению точек $x^{*}$ ); если к этим промежуткам присоединить еще лишь один промежуток $\sigma_{0}$, то покроется и весь промежуток $[a, c]$, так что $c$ есть одна из точек $x^{*}$.

Вместе с тем, ясно, что $c$ не может быть $<b$, ибо иначе между $c$ и $\beta$ нашлись бы еще точки $x^{*}$, вопреки определению числа $c$ как верхней границы всех $x^{*}$. Таким образом, необходимо $b=c$; значит $b$ есть одно из $x^{*}$, т.е. промежуток $[a, b]$ покрывается конечным числом промежутков $\sigma$, ч. и тр. д.

Заметим, что для справедливости заключения леммы в равной мере существенно как предположение о замкнутости основного промежутка $[a, b]$, так и предположение о том, что промежутки $\sigma$, составляющие систему $\Sigma$, - открытые. Например, система открытых промежутков

$$
  \left(\frac{1}{2}, \frac{3}{2}\right),\left(\frac{1}{4}, \frac{3}{4}\right),\left(\frac{1}{8}, \frac{3}{8}\right), \ldots,\left(\frac{1}{2^{n}}, \frac{3}{2^{n}}\right), \ldots
$$

покрывает промежуток ( 0,1$]$, но из них нельзя выделить конечной подсистемы с тем же свойством. Аналогично, система замкнутых промежутков

$$
  \left[0, \frac{1}{2}\right],\left[\frac{1}{2}, \frac{3}{4}\right],\left[\frac{3}{4}, \frac{7}{8}\right], \ldots,\left[\frac{2^{n}-1}{2^{n}}, \frac{2^{n+1}-1}{2^{n+1}}\right], \ldots \quad \text { и } \quad[1,2]
$$

покрывает промежуток [0, 2], но и здесь выделение конечной подсистемы невозможно.

\subtopic{Новые доказательства основных теорем.} Покажем теперь, как лемма Бореля может быть использована для доказательства основных теорем о непрерывных функциях Больцано-Коши, Вейерштрасса и Кантора.

\begin{enumerate}
  \item[$1^{\circ}$] \textit{1-я теорема Больцано-Коши} [\textbf{80}]. На этот раз доказывать ее будем от противного. Допустим, что - при соблюдении предположений
\end{enumerate}

\newpage
\setcounter{page}{184}
\setcounter{subtopic}{89}

Таким образом, в пределах каждой такой окрестности функция $f(x)$ заведомо ограничена: снизу - числом $f\left(x'\right)-\varepsilon$, а сверху - числом $f\left(x'\right)+\varepsilon$.

Читателю ясно, что и здесь к бесконечной системе $\sum$ окрестностей, обладающих указанным свойством, надлежит применить лемму Бореля. Из нее следует, что найдется в $\sum$ конечное число окрестностей (6), также в совокупности покрывающих весь промежуток $[a, b]$. Если

$$
  \begin{aligned}
     & m_{1} \leqslant f(x) \leqslant M_{1} \text { в } \sigma_{1}, \\
     & m_{2} \leqslant f(x) \leqslant M_{2} \text { в } \sigma_{2}, \\
     & \ldots \ldots \ldots \ldots \ldots \ldots \ldots             \\
     & m_{n} \leqslant f(x) \leqslant M_{n} \text { в } \sigma_{n},
  \end{aligned}
$$

то, взяв в качестве $m$ наименьшее из чисел $m_{1}, m_{2}, \ldots, m_{n}$, а в качестве $M$ - наибольшее из чисел $M_{1}, M_{2}, \ldots, M_{n}$, очевидно, будем иметь

$$
  m \leqslant f(x) \leqslant M
$$
во всем промежутке $[a, b]$, ч. и тр. д.

\begin{enumerate}
\item[$3^{\circ}$] \textbf{\textit{Теорема Кантора}}
3.  [87]. Зададимся произвольным числом $\varepsilon>0$. На этот раз каждую точку $x'$ промежутка $[a, b]$ окружим такой окрестностью $\sigma'=\left(x'-\delta', x'+\delta'\right)$, чтобы в ее пределах выполнялось неравенство

$$
  \left|f(x)-f\left(x'\right)\right|<\frac{\varepsilon}{2}.
$$

Если $x_{0}$ также есть точка этой окрестности, то одновременно и

$$
  \left|f\left(x'\right)-f\left(x_{0}\right)\right|<\frac{\varepsilon}{2}.
$$

Таким образом, для любых точек $x$ и $x_{0}$ из $\sigma'$ будем иметь

$$
  \left|f(x)-f\left(x_{0}\right)\right|<\varepsilon
$$

Стянем каждую окрестность $\sigma'$ вдвое, сохраняя ее центр, т.е. вместо $\sigma'$ рассмотрим окрестность

$$
  \overline{\sigma'}=\left(x'-\frac{\delta'}{2}, x'+\frac{\delta'}{2}\right)
$$

Из этих окрестностей также составится система $\overline{\Sigma}$, покрывающая промежуток $[a, b]$, и именно к ней мы применим лемму Бореля. Промежуток $[a, b]$ покроется конечным числом промежутков из $\overline{\Sigma}$:

$$
  \bar{\sigma}_{i}=\left(x_{i}-\frac{\delta_{i}}{2}, x_{i}+\frac{\delta_{i}}{2}\right) \quad(i=1,2, \ldots, n)
$$

Пусть теперь $\delta$ будет наименьшим из всех чисел $\frac{\delta_{i}}{2}$, и $x_{0}$, $x$ - любые две точки нашего промежутка, удовлетворяющие условию:


\begin{equation*}
  \left|x-x_{0}\right|<\delta \tag{7}\label{7}
\end{equation*}


Точка $x_{0}$ должна принадлежать одной из выделенных окрестностей, например, окрестности

$$
  \overline{\sigma_{i_{0}}}=\left(x_{i_{0}}-\frac{\delta_{i_{0}}}{2}, x_{i_{0}} + \frac{\delta_{i_{0}}}{2}\right)
$$

так что $\left|x_{0}-x_{i_{0}}\right|<\frac{\delta_{i_{0}}}{2}$.

Так как $\delta \leqslant \frac{\delta_{i_{0}}}{2}$, то, ввиду \eqref{7}, $\left|x-x_{0}\right|<\frac{\delta_{i_{0}}}{2}$, откуда $\left|x-x_{i_{0}}\right|<\delta_{i_{0}}$, т.е. точка $x$ (а подавно - и точка $x_{0}$ ) принадлежит той первоначально взятой окрестности

$$
  \left(x_{i_{0}}-\delta_{i_{0}}, x_{i_{0}}+\delta_{i_{0}}\right)
$$

стягиванием которой получена окрестность $\bar{\sigma}_{i_{0}}$. В таком случае, по свойству всех первоначально взятых окрестностей,

$$
  \left|f(x)-f\left(x_{0}\right)\right|<\varepsilon.
$$

Поскольку $\delta$ было выбрано вне зависимости от положения точки $x_{0}$, равномерная непрерывность функции $f(x)$ доказана.

Как видно из приведенных рассуждений, лемма Бореля с успехом прилагается в тех случаях, когда «локальное» свойство, связанное с окрестностью отдельной точки, подлежит распространению на весь рассматриваемый промежуток.

\end{enumerate}
\end{document}