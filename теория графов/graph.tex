\documentclass[a4paper, 12pt]{extarticle}
\usepackage{etoolbox}
\usepackage{fontspec}
\usepackage{polyglossia}
\setmainfont{CMU Serif}
\newfontfamily{\cyrillicfont}{CMU Serif}
\setsansfont{CMU Sans Serif}
\newfontfamily{\cyrillicfontsf}{CMU Sans Serif}
\setmonofont{CMU Typewriter Text}
\newfontfamily{\cyrillicfonttt}{CMU Typewriter Text}
\setdefaultlanguage{russian}

%%% Дополнительная работа с математикой
\usepackage{amsfonts,amssymb,amsthm,mathtools} % AMS
\usepackage{amsmath}
\usepackage{icomma} % "Умная" запятая: $0,2$ --- число, $0, 2$ --- перечисление

%% Шрифты
\usepackage{euscript} % Шрифт Евклид
\usepackage{mathrsfs} % Красивый матшрифт

%% Свои команды
\DeclareMathOperator{\sgn}{\mathop{sgn}}

%% Перенос знаков в формулах (по Львовскому)
\newcommand*{\hm}[1]{#1\nobreak\discretionary{}
	{\hbox{$\mathsurround=0pt #1$}}{}}

%%% Работа с картинками
\usepackage{graphicx}  % Для вставки рисунков
\graphicspath{{Изображения/}{image}}  % папки с картинками
\setlength\fboxsep{3pt} % Отступ рамки \fbox{} от рисунка
\setlength\fboxrule{1pt} % Толщина линий рамки \fbox{}
\usepackage{wrapfig} % Обтекание рисунков и таблиц текстом

%%% Работа с таблицами
\usepackage{array,tabularx,tabulary,booktabs} % Дополнительная работа с таблицами
\usepackage{longtable}  % Длинные таблицы
\usepackage{multirow} % Слияние строк в таблице
\usepackage{blindtext}
\usepackage{multicol}
\usepackage{pdfpages}
\usepackage[left=1cm,right=1cm,top=1cm,bottom=1.5cm]{geometry}
\usepackage{tikz}
\usepackage{wrapfig}
\renewcommand{\thesubsection}{\arabic{subsection}}

\begin{document}

% Неориентированные графы, cтепени, изоморфизм.
\subsection{Неориентированные графы, степени, изоморфизм}

\begin{itemize}
	\item \textbf{Граф}:
	\begin{itemize}
		\item Обозначается как $G = (X, \Gamma)$.
		\item Состоит из:
		\begin{enumerate}
			\item[\(1^\circ\)] Непустое множество $X$.
			\item[\(2^\circ\)] Отображение $\Gamma$ множества $X$ в $X$.
		\end{enumerate}
	\end{itemize}

	\item \textbf{Элементы графа}:
	\begin{itemize}
		\item \textbf{Вершина}: Каждый элемент множества $X$ называется точкой или вершиной графа.
		\item \textbf{Дуга}: Пара элементов $(x, y)$, где $y \in \Gamma x$, называется дугой графа.
	\end{itemize}

	\item \textbf{Изображение графа}:
	\begin{itemize}
		\item Элементы $X$ изображаются точками на плоскости.
		\item Пары точек $x$ и $y$, где $y \in \Gamma x$, соединяются непрерывной линией со стрелкой от $x$ к $y$.
	\end{itemize}

	\item \textbf{Множество дуг}:
	\begin{itemize}
		\item Обозначается через $U$.
		\item Дуги обозначаются буквами $\alpha$, $\beta$, $\omega$ (при необходимости с индексами).
	\end{itemize}
\end{itemize}

\textit{Степенью} вершины \(v_i\) в графе \(G\) — обозначается \(d_i\) или \(\deg v_i\) — называется число рёбер, инцидентных \(v_i\) (то есть рёбер, которые соединены с \(v_i\)). Поскольку каждое ребро инцидентно двум вершинам, в сумму степеней вершин графа каждое ребро вносит двойку. Таким образом, мы приходим к утверждению, которое установлено Эйлером и является исторически первой теоремой теории графов.

\subsubsection*{Теорема 2.1}
Сумма степеней вершин графа \(G\) равна удвоенному числу его рёбер:
\[
\sum_i \deg v_i = 2q.
\]

\subsubsection*{Следствие 2.1 (a)}
В любом графе число вершин с нечётными степенями чётно.

В \((p, q)\)-графе \(0 \leq \deg v \leq p-1\) для любой вершины \(v\). Минимальная степень вершин графа \(G\) обозначается через \(\min \deg G\) или \(\delta(G)\), максимальная — через \(\max \deg G = \Delta(G)\). Если \(\delta(G) = \Delta(G) = r\), то все вершины имеют одинаковую степень и такой граф \(G\) называется \textit{регулярным} (или \textit{однородным}) степени \(r\). В этом случае говорят о степени графа и пишут \(\deg G = r\).

Регулярный граф степени 0 совсем не имеет рёбер. Если \(G\) — регулярный граф степени 1, то каждая его компонента содержит точно одно ребро; в регулярном графе степени 2 каждая компонента — цикл, и, конечно, обратно. Первые интересные\(^2\) регулярные графы имеют степень 3; такие графы называются \textit{кубическими}. На рис. 2.11 показаны два регулярных графа с 6 вершинами. Второй из них изоморфен каждому из трёх графов, изображённых на рис. 2.5.

\subsection*{Следствие 2.1 (б)}
Каждый кубический граф имеет чётное число вершин.

Полезно дать названия вершинам с малыми степенями. Вершина \(v\) называется \textit{изолированной}, если \(\deg v = 0\), и \textit{концевой} (или \textit{висячей}), если \(\deg v = 1\).

Два графа $G$ и $H$ изоморфны (записывается $G \cong H$ или иногда $G = H$), если между их множествами вершин существует взаимно однозначное соответствие, сохраняющее смежность. Например, графы $G_1$ и $G_2$ на рис.~2.5 изоморфны при соответствии $v_i \leftrightarrow u_i$, и чисто случайно оказалось, что граф $G_1$ изоморфен каждому из них. Совершенно очевидно, что изоморфизм есть отношение эквивалентности на графах.

% Маршруты, связность, метрика графа.
\textbf{Маршруты, связность, метрика графа}

\noindent\textbf{Определение.} \textit{Маршрут} в графе $G$ (последовательность переходов по вершинам и рёбрам) -- чередующаяся последовательность вершин и рёбер $v_0, x_1, v_1, \ldots, x_n, v_n$, где:
\begin{itemize}
\item Начинается и заканчивается вершиной (точкой графа)
\item Каждое ребро инцидентно (напрямую соединяет) предшествующей и следующей вершинам
\end{itemize}

\noindent\textbf{Обозначение:} $(v_0-v_n)$-маршрут (путь от вершины $v_0$ до $v_n$) записывается как $v_0 v_1 v_2 \ldots v_n$

\noindent\textbf{Классификация маршрутов:}
\begin{itemize}
\item \textit{Замкнутый}: $v_0 = v_n$ (начальная и конечная вершины совпадают)
\item \textit{Открытый}: $v_0 \neq v_n$ (начальная и конечная вершины различны)
\item \textit{Цепь} (trail): все рёбра различны (по каждому ребру проходим не более одного раза)
\item \textit{Простая цепь} (path): все вершины и рёбра различны (нигде не повторяемся)
\item \textit{Цикл}: замкнутая цепь (маршрут возвращается в начальную точку)
\item \textit{Простой цикл}: замкнутый маршрут с $n \geq 3$ различными вершинами (замкнутый путь без повторений вершин, кроме начальной/конечной)
\end{itemize}

\noindent\textbf{Длина маршрута} $v_0 v_1 \ldots v_n$ = $n$ (количество пройденных рёбер)

\noindent\textbf{Важные метрики:}
\begin{itemize}
\item \textit{Обхват графа} $g(G)$: длина кратчайшего простого цикла (минимальное количество рёбер в замкнутом пути без повторений)
\item \textit{Окружение графа} $c(G)$: длина длиннейшего простого цикла (максимальное количество рёбер в замкнутом пути без повторений)
\end{itemize}

\noindent\textbf{Примечание:} $g(G)$ и $c(G)$ не определены для графов без циклов (для деревьев и лесов).

%Самодополнительные графы
\subsection{Самодополнительные графы}

\begin{wrapfigure}{r}{0.35\textwidth}
    \begin{tikzpicture}[scale=1]
    % Первый граф (G)
    \begin{scope}[xshift=0cm]
        \foreach \angle [count=\i] in {90,150,...,450} {
            \node[circle, fill=black, inner sep=1.5pt] (v\i) at (\angle:1) {};
        }
        \foreach \i in {1,...,6} {
            \pgfmathtruncatemacro{\next}{mod(\i,6)+1}
            \draw (v\i) -- (v\next);
        }
        \draw (v1) -- (v4);
        \draw (v2) -- (v5);
        \draw (v3) -- (v6);
        \node at (0,-1.5) {$G$};
    \end{scope}
    
    % Второй граф (G с чертой)
    \begin{scope}[xshift=3.5cm]
        \foreach \angle [count=\i] in {30,90,...,389} {
            \node[circle, fill=black, inner sep=1.5pt] (w\i) at (\angle:1) {};
        }
        \foreach \i/\j in {1/3,3/5,5/1,2/4,4/6,6/2} {
            \draw (w\i) -- (w\j);
        }
        \node at (0,-1.5) {$\overline{G}$};
    \end{scope}
    \node at (1.5,-2) {Рис. 2.12. Граф и его дополнение.};
    \end{tikzpicture}
\end{wrapfigure}

\noindent\textbf{Определение.} \textit{Дополнение графа} $\overline{G}$ (граф с теми же вершинами, но противоположными связями):
\begin{itemize}[noitemsep,topsep=0pt]
\item Множество вершин: $V(\overline{G}) = V(G)$
\item Две вершины смежны в $\overline{G}$ $\Leftrightarrow$ несмежны в $G$
\end{itemize}

\noindent\textbf{Определение.} \textit{Самодополнительный граф} -- граф, изоморфный своему дополнению (структура графа совпадает со структурой его дополнения).

\noindent\textbf{Полный граф} $K_p$ (все вершины попарно соединены):
\begin{itemize}[noitemsep,topsep=0pt]
\item Содержит $p$ вершин
\item Имеет $\binom{p}{2}$ рёбер
\item Является регулярным степени $p-1$
\item Частный случай: $K_3$ -- треугольник
\end{itemize}

\noindent\textbf{Вполне несвязный граф} $\overline{K_p}$ -- дополнение полного графа (регулярный граф степени 0).

% Экстремальные графы
\subsection{Экстремальные графы}

\noindent\textbf{Теорема 2.3 (Турана)} (о максимальном числе рёбер в графе без треугольников):\\
Наибольшее число рёбер у графов с $r$ вершин без треугольников равно $\lfloor r^2/4 \rfloor$.

\noindent\textbf{Доказательство} (по индукции для чётных $r$):
\begin{enumerate}[noitemsep,topsep=0pt]
\item База: очевидна для малых $r$
\item Шаг: для $r = 2n + 2$, где утверждение верно для всех чётных $r \leq 2n$:
   \begin{itemize}[noitemsep]
   \item Пусть $G$ -- граф с $p = 2n + 2$ вершинами без треугольников
   \item Существуют смежные вершины $u$, $v$ (граф не вполне несвязный)
   \item В подграфе $G' = G - \{u, v\}$ максимум $n^2$ рёбер
   \item Нет вершины $w$, смежной с $u$ и $v$ одновременно
   \item Если $w$ смежна с $k$ вершинами $G'$, то $v$ смежна максимум с $(2n - k)$ вершинами
   \item Всего рёбер: $n^2 + k + (2n - k) + 1 = n^2 + 2n + 1 = p^2/4$
   \end{itemize}
\end{enumerate}

\noindent\textbf{Конструктивное доказательство существования:}\\
Для чётного $p$ $(p, p^2/4)$-граф без треугольников строится так:
\begin{itemize}[noitemsep]
\item Берём два множества $V_1$ и $V_2$ по $p/2$ вершин
\item Соединяем каждую вершину из $V_1$ с каждой из $V_2$
\end{itemize}

\noindent\textbf{Примечания:}
\begin{itemize}[noitemsep]
\item Доказательство существования чисел $r(m, n)$ см. у М. Холла
\item По определению бесконечный граф не является графом
\item Обзор бесконечных графов: см. Нэш-Вильямс
\end{itemize}

% Числа Рамсея
\subsection{Числа Рамсея}
Широко известна следующая головоломка.

\textit{Доказать, что среди любых шести человек найдутся либо трое попарно знакомых, либо трое попарно незнакомых.}

\begin{enumerate}
    \item Напоминаем читателю (см. введение), что в тексте не все теоремы доказываются.
    \item По своим структурным свойствам. — \textit{Прим. перев.}
\end{enumerate}

В этих терминах головоломку можно сформулировать так:

\textbf{Теорема 2.2.} Если \( G \) — граф с шестью вершинами, то либо \( G \), либо \( \overline{G} \) содержит треугольник.

\textbf{Доказательство.} Пусть \( v \) — произвольная вершина графа \( G \), имеющего шесть вершин. Так как вершина \( v \) с любой из остальных пяти вершин смежна или в \( G \), или в \( \overline{G} \), то, не теряя общности, можно предположить, что вершины \( u_1, u_2, u_3 \) смежны с \( v \) в \( G \). Если какие-либо две из вершин \( u_1, u_2, u_3 \) смежны в \( G \), то вместе с \( v \) они образуют треугольник. Если никакие две из них не смежны в \( G \), то в графе \( \overline{G} \) вершины \( u_1, u_2, u_3 \) образуют треугольник.

Обобщая теорему 2.2, естественно поставить вопрос: каково наименьшее целое число \( r(m, n) \), для которого каждый граф с \( r(m, n) \) вершинами содержит \( K_m \) или \( K_n \)?

Числа \( r(m, n) \) называются \textit{числами Рамсея} \(^1\). Ясно, что \( r(m, n) = r(n, m) \). Задача, связанная с нахождением чисел Рамсея, остается нерешенной, хотя известна простая верхняя оценка, полученная Эрдёшем и Секерешем \(^1\):

\begin{equation}
r(m, n) \leq \binom{m + n - 2}{m - 1}.
\end{equation}

Постановка этой задачи вытекает из теоремы Рамсея. Бесконечный граф \(^2\) имеет бесконечное множество вершин и не содержит кратных ребер и петель. Рамсей \(^1\) доказал (на языке теории множеств), что каждый бесконечный граф содержит \( \aleph_0 \) попарно смежных вершин или \( \aleph_0 \) попарно несмежных вершин.

Все известные числа Рамсея приведены в табл. 2.1 (взята из обзорной статьи Гравера и Якелл \(^1\)).

% Эйлеровы графы
\subsection{Эйлеровы графы}

\noindent\textbf{Определение.} \textit{Эйлеров граф} -- граф, содержащий цикл со всеми вершинами и рёбрами (имеет эйлеров цикл). Обязательно связный.

\noindent\textbf{Теорема 7.1} (критерий эйлеровости). Для связного графа $G$ эквивалентны:
\begin{enumerate}[noitemsep,topsep=0pt]
\item $G$ -- эйлеров граф
\item Все вершины имеют чётную степень
\item Рёбра можно разбить на простые циклы
\end{enumerate}

\noindent\textbf{Доказательство:}\\
(1)$\Rightarrow$(2): В эйлеровом цикле каждое прохождение вершины даёт +2 к её степени. Каждое ребро используется один раз $\Rightarrow$ степени чётны.

\noindent(2)$\Rightarrow$(3): В связном графе с чётными степенями:
\begin{itemize}[noitemsep]
\item Найдём простой цикл $Z$
\item Удалим его рёбра -- получим граф $G_1$ с чётными степенями
\item Повторяем до пустого графа $G_n$
\end{itemize}

\noindent(3)$\Rightarrow$(1): Имея разбиение на циклы:
\begin{itemize}[noitemsep]
\item Берём цикл $Z_1$
\item Находим цикл $Z_2$ с общей вершиной $v$
\item Строим замкнутую цепь из $Z_1$ и $Z_2$
\item Продолжаем до полного эйлерова цикла
\end{itemize}

\noindent\textbf{Следствие 7.1(a).} В связном графе с $2n$ вершинами нечётной степени ($n \geq 1$) рёбра можно разбить на $n$ открытых цепей.

\noindent\textbf{Следствие 7.1(б).} В связном графе с двумя вершинами нечётной степени существует открытая цепь, содержащая все рёбра (начинается и заканчивается в вершинах нечётной степени).

\subsection{Деревья}

\noindent\textbf{Основные определения:}
\noindent\textbf{Ациклический граф} -- граф без циклов.
\noindent\textbf{Дерево} -- связный ациклический граф.
\noindent\textbf{Лес} -- граф без циклов (компоненты -- деревья).

\noindent\textbf{Теорема 4.1.} Для графа $G$ эквивалентны:
\noindent 1) $G$ -- дерево
\noindent 2) любые две вершины соединены единственной простой цепью
\noindent 3) $G$ связен и $p = q + 1$
\noindent 4) $G$ ациклический и $p = q + 1$
\noindent 5) $G$ ациклический, и добавление любого ребра создаёт ровно один цикл
\noindent 6) $G$ связный, не $K_p$ при $p \geq 3$, добавление ребра создаёт один цикл
\noindent 7) $G$ не $K_3 \cup K_1$ и не $K_3 \cup K_2$, $p = q + 1$, добавление ребра создаёт один цикл

\noindent\textbf{Доказательство} (схема):
\noindent 1$\Rightarrow$2: От противного: две цепи образуют цикл
\noindent 2$\Rightarrow$3: Индукция по числу вершин
\noindent 3$\Rightarrow$4: От противного: цикл длины $n$ требует $q \geq p$
\noindent 4$\Rightarrow$5: Единственность компоненты из $p = q + k$
\noindent 5$\Rightarrow$6: $K_p$ при $p \geq 3$ содержит цикл
\noindent 6$\Rightarrow$7: Анализ возможных циклов
\noindent 7$\Rightarrow$1: Исключение случаев с циклами

\noindent\textbf{Следствие 4.1(а).} В нетривиальном дереве есть минимум две висячие вершины.
\noindent\textit{Доказательство:} Из $\sum d_i = 2(p-1)$ в дереве.
\subsection{Диаметр и радиус графа}
\subsection{Хроматическое число графа}

\noindent\textbf{Определение.} \textit{p-хроматический граф} -- граф, вершины которого можно раскрасить в p цветов так, чтобы смежные вершины имели разные цвета.

\noindent\textbf{Хроматическое число} $\chi(G)$ -- минимальное p, при котором граф p-хроматический.

\noindent\textbf{Хроматический класс} -- минимальное число цветов q для раскраски рёбер без одинаковых смежных рёбер.

\noindent\textbf{Теорема о двудольных графах.} Граф двудольный $(χ(G)=2) \Longleftrightarrow$ не содержит циклов нечётной длины.

\noindent\textbf{Доказательство:}\\
($\Rightarrow$) Алгоритм раскраски в 2 цвета:
1) Выбираем вершину a, красим в синий
2) Смежные с синими красим в красный, с красными -- в синий
3) Отсутствие нечётных циклов гарантирует корректность
 
($\Leftarrow$) От противного: в двудольном графе нельзя раскрасить нечётный цикл в 2 цвета.

\noindent\textbf{Теорема 4.} Для симметрического графа G эквивалентны:
1) G является p-хроматическим
2) Существует функция Гранди g(x) с $\max g(x) \leq p-1$

\noindent\textbf{Теорема 5.} Для графов G (p+1-хром.) и H (q+1-хром.):
$\chi(G \times H) = r+1$, где $r = max{p'+q': p'\leq p, q'\leq q}$

\noindent\textbf{Теорема 6.} Для графов G и H с χ(G)=p, χ(H)=q:
$\chi(G \times H) = \min\{p,q\}$

\noindent\textbf{Важное свойство:} Для плоских графов χ(G)$leq$5 (достаточно 5 цветов для раскраски карты).
\subsection{Цикломатическое число графа}
\subsection{Плоские графы, формула Эйлера}
\subsection{Линейно независимые циклы}
\subsection{Хроматическое число плоского графа}
\subsection{Примеры неплоских графов}
\subsection{Ориентированные графы, порядковая функция}
\subsection{Функция Гранди}
\subsection{Внутреннее устойчивое множество}
\subsection{Внешнее устойчивое множество}
\subsection{Ядро графа}
\subsection{Игры на графе, игра НИМ}
\subsection{Транспортные сети}
\subsection{Теорема Кёнига-Холла}
\subsection{Приложения к матрицам}
\subsection{Бистохастические матрицы}
\subsection{Теорема Биркгофа - фон Неймана}

\end{document}