\subsection{Хроматическое число графа}

\noindent\textbf{Определение.} \textit{p-хроматический граф} -- граф, вершины которого можно раскрасить в p цветов так, чтобы смежные вершины имели разные цвета.

\noindent\textbf{Хроматическое число} $\chi(G)$ -- минимальное p, при котором граф p-хроматический.

\noindent\textbf{Хроматический класс} -- минимальное число цветов q для раскраски рёбер без одинаковых смежных рёбер.

\noindent\textbf{Теорема о двудольных графах.} Граф двудольный $(χ(G)=2) \Longleftrightarrow$ не содержит циклов нечётной длины.

\noindent\textbf{Доказательство:}\\
($\Rightarrow$) Алгоритм раскраски в 2 цвета:
1) Выбираем вершину a, красим в синий
2) Смежные с синими красим в красный, с красными -- в синий
3) Отсутствие нечётных циклов гарантирует корректность
 
($\Leftarrow$) От противного: в двудольном графе нельзя раскрасить нечётный цикл в 2 цвета.

\noindent\textbf{Теорема 4.} Для симметрического графа G эквивалентны:
1) G является p-хроматическим
2) Существует функция Гранди g(x) с $\max g(x) \leq p-1$

\noindent\textbf{Теорема 5.} Для графов G (p+1-хром.) и H (q+1-хром.):
$\chi(G \times H) = r+1$, где r = max{p'+q': p'$leq$p, q'$leq$q}

\noindent\textbf{Теорема 6.} Для графов G и H с χ(G)=p, χ(H)=q:
$\chi(G \times H) = \min\{p,q\}$

\noindent\textbf{Важное свойство:} Для плоских графов χ(G)$leq$5 (достаточно 5 цветов для раскраски карты).