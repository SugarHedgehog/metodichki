\begin{center}
    \hfill \break
    \large{МИНОБРНАУКИ РОССИИ}\\
    \footnotesize{ФЕДЕРАЛЬНОЕ ГОСУДАРСТВЕННОЕ БЮДЖЕТНОЕ ОБРАЗОВАТЕЛЬНОЕ УЧРЕЖДЕНИЕ}\\ 
    \footnotesize{ВЫСШЕГО ПРОФЕССИОНАЛЬНОГО ОБРАЗОВАНИЯ}\\
    \small{\textbf{«ВОРОНЕЖСКИЙ ГОСУДАРСТВЕННЫЙ УНИВЕРСИТЕТ»}}\\
    \hfill \break
    \normalsize{Математический факультет}\\
     \hfill \break
    \normalsize{Кафедра теории функций и геометрии}\\
    \hfill\break
    \hfill \break
    \hfill \break
    \hfill \break
    %%\large{Начально-краевая задача для волнового уравнения на геометрическом графе-звезде}\\
    \hfill \break
    \hfill \break
    \hfill \break
    \hfill \break
    \hfill \break
    \hfill \break\
    \hfill \break
    \hfill \break
    \normalsize{Контрольная работа\\
    \hfill \break
    Направление  010501 Фундаментальные математика и механика\\

    \hfill \break
    }\\
    \hfill \break
    \hfill \break
    \end{center}
    \hfill \break
     
    \normalsize{ 
    \begin{tabular}{cccc}
    Зав.кафедрой & \underline{\hspace{3cm}} &  д.физ.-мат.н.,  проф. & Е.М. Семёнов \\\\
    Обучающийся & \underline{\hspace{3cm}} & &А.С. Суматохина \\\\
    Руководитель & \underline{\hspace{3cm}}&  & О.Б. Мазкина \\\\
    \end{tabular}
    }\\
    \hfill \break
    \hfill \break
    \hfill \break
    \hfill \break
    \hfill \break
    \hfill \break
    \hfill \break
    \hfill \break
    \hfill \break
    \hfill \break
    \begin{center} Воронеж 2023 \end{center}
    \thispagestyle{empty} % выключаем отображение номера для этой страницы
     
    % КОНЕЦ ТИТУЛЬНОГО ЛИСТА