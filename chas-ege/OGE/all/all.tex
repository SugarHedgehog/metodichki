\documentclass[4apaper]{article}
\usepackage{pdfpages}
\usepackage{dashbox}
\usepackage[T2A]{fontenc}
\usepackage[utf8]{inputenc}
\usepackage[english,russian]{babel}
\usepackage{graphicx}
\graphicspath{{pictures/}}
\DeclareGraphicsExtensions{.pdf,.png,.jpg}

\linespread{1.15}

\usepackage{../../egetask_ver}

\def\examyear{2023}
\usepackage[colorlinks,linkcolor=blue]{hyperref}

\begin{document}
\begin{taskBN}{1}
\addpictocenter[scale=0.4]{images/63611096677063n0}Установите соответствие между графиками функций и формулами, которые их задают. Формулы: \\1) $\frac{4}{x+4}+2$\\2) $\frac{5}{x-4}-4$\\3) $\frac{-3}{x-3}+4$\\4) $\frac{2}{x+1}-2$
\end{taskBN}

\begin{taskBN}{2}
\addpictocenter[scale=0.4]{images/663734334474295n0}Установите соответствие между графиками функций и формулами, которые их задают. Формулы: \\1) $-3x^2+10x-8$\\2) $4x^2-7x-2$\\3) $-5x^2+7x+1$\\4) $-1.5x^2-7x-5$
\end{taskBN}

\begin{taskBN}{3}
\addpictocenter[scale=0.4]{images/082481064381035n0}На рисунке изображён график квадратичной функции $y=f(x)$. Какое из следующих утверждений о данной функции неверно?\\1) Наибольшее значение функции равно  $4$\\2) Абсцисса вершины равна $1$\\3) Ордината вершины равна $4$\\4) $f(-4)=$ $-4$\\5) Функция возрастает на промежутке $(-\infty;0]$
\end{taskBN}

\begin{taskBN}{4}
\addpictocenter[scale=0.4]{images/189385112062671n0}Установите соответствие между графиками функций и формулами, которые их задают. Формулы: \\1) $\frac{-3}{x-1}-5$\\2) $\frac{-2}{x+3}+2$\\3) $\frac{-2}{x-4}+1$\\4) $\frac{2}{x-5}+2$
\end{taskBN}

\begin{taskBN}{5}
\addpictocenter[scale=0.4]{images/016845351153553n0}На рисунке изображён график квадратичной функции $y=f(x)$. Какое из следующих утверждений о данной функции неверно?\\1) Абсцисса вершины равна $5$\\2) Функция убывает на промежутке $[0; \infty)$\\3) Наибольшее значение функции равно  $4$\\4) $f(-4)=$ $0$\\5) Ордината вершины равна $4$
\end{taskBN}

\begin{taskBN}{6}
\addpictocenter[scale=0.4]{images/9390474790598589n0}Установите соответствие между графиками функций и формулами, которые их задают. Формулы: \\1) $-1.5\sqrt{x}+5$\\2) $-3\sqrt{x}-2$\\3) $-3\sqrt{x}+2$\\4) $-4\sqrt{x}+3$
\end{taskBN}

\begin{taskBN}{7}
\addpictocenter[scale=0.4]{images/422842652393939n0}На рисунке изображён график квадратичной функции $y=f(x)$. Какое из следующих утверждений о данной функции неверно?\\1) Наибольшее значение функции равно  $2$\\2) $f(-4)=$ $-2$\\3) Функция возрастает на промежутке $(-\infty;0]$\\4) Абсцисса вершины равна $8.5$\\5) Ордината вершины равна $2$
\end{taskBN}

\begin{taskBN}{8}
\addpictocenter[scale=0.4]{images/6850896048894455n0}Установите соответствие между графиками функций и формулами, которые их задают. Формулы: \\1) $-1.5x-6$\\2) $2.5x+2$\\3) $-2x-7$\\4) $3x-1$
\end{taskBN}

\begin{taskBN}{9}
\addpictocenter[scale=0.4]{images/117810923644557n0}Установите соответствие между графиками функций и формулами, которые их задают. Формулы: \\1) $2.5\sqrt{x}-3$\\2) $2\sqrt{x}-3$\\3) $-1.5\sqrt{x}+2$\\4) $1.5\sqrt{x}-1$
\end{taskBN}

\begin{taskBN}{10}
\addpictocenter[scale=0.4]{images/703058456161071n0}Установите соответствие между графиками функций и формулами, которые их задают. Формулы: \\1) $3x-6$\\2) $4x+3$\\3) $2.5x+9$\\4) $3.5x-5$
\end{taskBN}

\begin{taskBN}{11}
\addpictocenter[scale=0.4]{images/334649457440095n0}Установите соответствие между графиками функций и формулами, которые их задают. Формулы: \\1) $\frac{-2}{x+4}+1$\\2) $\frac{-5}{x+3}-1$\\3) $\frac{-5}{x+3}+2$\\4) $\frac{-3}{x-2}-2$
\end{taskBN}

\begin{taskBN}{12}
\addpictocenter[scale=0.4]{images/527994656824931n0}Установите соответствие между графиками функций и формулами, которые их задают. Формулы: \\1) $-4.5x+4$\\2) $2x-1$\\3) $1.5x+1$\\4) $-2x-3$
\end{taskBN}

\begin{taskBN}{13}
\addpictocenter[scale=0.4]{images/004154427257667n0}Установите соответствие между графиками функций и формулами, которые их задают. Формулы: \\1) $x^2+3x-9$\\2) $-2x^2+6x+4$\\3) $3x^2+7x+5$\\4) $4x^2+3x-4$
\end{taskBN}

\begin{taskBN}{14}
\addpictocenter[scale=0.4]{images/601248541243585n0}Установите соответствие между графиками функций и формулами, которые их задают. Формулы: \\1) $-3.5\sqrt{x}+4$\\2) $4\sqrt{x}-4$\\3) $-3\sqrt{x}-2$\\4) $-\sqrt{x}-1$
\end{taskBN}

\begin{taskBN}{15}
\addpictocenter[scale=0.4]{images/537196898410924n0}Установите соответствие между графиками функций и формулами, которые их задают. Формулы: \\1) $-4x^2+5x+4$\\2) $4x^2-8x+3$\\3) $2x^2+6x-4$\\4) $-3x^2-9x-4$
\end{taskBN}

\begin{taskBN}{16}
\addpictocenter[scale=0.4]{images/935647731365426n0}Установите соответствие между графиками функций и формулами, которые их задают. Формулы: \\1) $-4\sqrt{x-1}+3$\\2) $-1.5x-2$\\3) $3x-7$\\4) $-4x+3$
\end{taskBN}

\begin{taskBN}{17}
\addpictocenter[scale=0.4]{images/3139990153771106n0}Установите соответствие между графиками функций и формулами, которые их задают. Формулы: \\1) $-4\sqrt{x+7}+4$\\2) $2x+7$\\3) $\sqrt{x+4}+3$\\4) $-4\sqrt{x+7}+5$
\end{taskBN}

\begin{taskBN}{18}
\addpictocenter[scale=0.4]{images/725318229255183n0}Установите соответствие между графиками функций и формулами, которые их задают. Формулы: \\1) $\frac{1}{x-2}-1$\\2) $\frac{3}{x-5}+5$\\3) $\frac{2}{x+2}+4$\\4) $\frac{4}{x+2}-5$
\end{taskBN}

\begin{taskBN}{19}
\addpictocenter[scale=0.4]{images/317641019774816n0}Установите соответствие между графиками функций и формулами, которые их задают. Формулы: \\1) $\frac{-2}{x-3}-4$\\2) $\frac{3}{x+1}+2$\\3) $\frac{1}{x-2}+3$\\4) $\frac{-3}{x+3}+3$
\end{taskBN}

\begin{taskBN}{20}
\addpictocenter[scale=0.4]{images/903373325007264n0}На рисунке изображён график квадратичной функции $y=f(x)$. Какое из следующих утверждений о данной функции неверно?\\1) Функция возрастает на промежутке $[0; \infty)$\\2) $f(-4)=$ $1$\\3) Наименьшее значение функции равно  $-3$\\4) Ордината вершины равна $-3$\\5) Абсцисса вершины равна $6$
\end{taskBN}

\begin{taskBN}{21}
\addpictocenter[scale=0.4]{images/06781985074358n0}На рисунке изображён график квадратичной функции $y=f(x)$. Какое из следующих утверждений о данной функции верно?\\1) Функция возрастает на промежутке $[1; \infty)$\\2) Абсцисса вершины равна $1$\\3) Ордината вершины равна $4.5$\\4) $f(-3)=$ $10$\\5) Наибольшее значение функции равно  $3.5$
\end{taskBN}

\begin{taskBN}{22}
\addpictocenter[scale=0.4]{images/0262485068019911n0}Установите соответствие между графиками функций и формулами, которые их задают. Формулы: \\1) $2\sqrt{x-2}-3$\\2) $-1.5x-2$\\3) $3x+8$\\4) $-\sqrt{x-5}-3$
\end{taskBN}

\begin{taskBN}{23}
\addpictocenter[scale=0.4]{images/083787158859988n0}Установите соответствие между графиками функций и формулами, которые их задают. Формулы: \\1) $-3.5\sqrt{x}+5$\\2) $1.5\sqrt{x}-2$\\3) $4\sqrt{x}-2$\\4) $3\sqrt{x}-4$
\end{taskBN}

\begin{taskBN}{24}
\addpictocenter[scale=0.4]{images/251096004808276n0}Установите соответствие между графиками функций и формулами, которые их задают. Формулы: \\1) $\frac{-5}{x+2}-4$\\2) $\frac{1}{x-3}-2$\\3) $\frac{3}{x-4}+4$\\4) $\frac{5}{x-4}-3$
\end{taskBN}

\begin{taskBN}{25}
\addpictocenter[scale=0.4]{images/9264738888790345n0}На рисунке изображён график квадратичной функции $y=f(x)$. Какое из следующих утверждений о данной функции неверно?\\1) Абсцисса вершины равна $2.5$\\2) Ордината вершины равна $5$\\3) Функция убывает на промежутке $[-2; \infty)$\\4) $f(-4)=$ $4$\\5) Наибольшее значение функции равно  $5$
\end{taskBN}

\begin{taskBN}{26}
\addpictocenter[scale=0.4]{images/14052743553199n0}На одном из рисунков изображен график функции $f(x)=-4x^2-1$. Укажите номер этого рисунка.
\end{taskBN}

\begin{taskBN}{27}
\addpictocenter[scale=0.4]{images/2343607422159553n0}На одном из рисунков изображен график функции $f(x)=2x^2+2x-3$. Укажите номер этого рисунка.
\end{taskBN}

\begin{taskBN}{28}
\addpictocenter[scale=0.4]{images/316355227579996n0}Установите соответствие между графиками функций и формулами, которые их задают. Формулы: \\1) $\frac{-3}{x+1}+3$\\2) $\frac{-3}{x-4}-2$\\3) $\frac{-4}{x-2}+5$\\4) $\frac{3}{x+6}+2$
\end{taskBN}

\begin{taskBN}{29}
\addpictocenter[scale=0.4]{images/280992017041205n0}Установите соответствие между графиками функций и формулами, которые их задают. Формулы: \\1) $\frac{3}{x+1}-2$\\2) $\frac{1}{x-3}+1$\\3) $\frac{-2}{x-1}+2$\\4) $\frac{5}{x+5}-2$
\end{taskBN}

\begin{taskBN}{30}
\addpictocenter[scale=0.4]{images/766848359334255n0}Установите соответствие между графиками функций и формулами, которые их задают. Формулы: \\1) $\sqrt{x}-3$\\2) $3\sqrt{x}-5$\\3) $5\sqrt{x}-3$\\4) $-2.5\sqrt{x}+2$
\end{taskBN}

\begin{taskBN}{31}
\addpictocenter[scale=0.4]{images/237264538948122n0}На одном из рисунков изображен график функции $f(x)=-4x^2+4$. Укажите номер этого рисунка.
\end{taskBN}

\begin{taskBN}{32}
\addpictocenter[scale=0.4]{images/122158084787271n0}На рисунке изображён график квадратичной функции $y=f(x)$. Какое из следующих утверждений о данной функции верно?\\1) Ордината вершины равна $4$\\2) $f(-2)=$ $3.5$\\3) Наименьшее значение функции равно  $5.5$\\4) Абсцисса вершины равна $2$\\5) Функция возрастает на промежутке $(-\infty;2]$
\end{taskBN}

\begin{taskBN}{33}
\addpictocenter[scale=0.4]{images/107096995400619n0}Установите соответствие между графиками функций и формулами, которые их задают. Формулы: \\1) $\frac{-3}{x+3}-3$\\2) $\frac{-3}{x-3}+3$\\3) $\frac{4}{x+5}-4$\\4) $\frac{-5}{x+3}-2$
\end{taskBN}

\begin{taskBN}{34}
\addpictocenter[scale=0.4]{images/77442957265715n0}На рисунке изображён график квадратичной функции $y=f(x)$. Какое из следующих утверждений о данной функции верно?\\1) Ордината вершины равна $4.5$\\2) $f(-2)=$ $6$\\3) Функция возрастает на промежутке $(-\infty;2]$\\4) Наименьшее значение функции равно  $9.5$\\5) Абсцисса вершины равна $2$
\end{taskBN}

\begin{taskBN}{35}
\addpictocenter[scale=0.4]{images/44631279468112184n0}Установите соответствие между графиками функций и формулами, которые их задают. Формулы: \\1) $\frac{2}{x+2}-3$\\2) $\frac{3}{x+2}+5$\\3) $\frac{5}{x-5}+3$\\4) $\frac{4}{x-4}-4$
\end{taskBN}

\begin{taskBN}{36}
\addpictocenter[scale=0.4]{images/240591659096704n0}Установите соответствие между графиками функций и формулами, которые их задают. Формулы: \\1) $3.5\sqrt{x}-2$\\2) $3\sqrt{x}-2$\\3) $-4\sqrt{x}+4$\\4) $-4\sqrt{x}+5$
\end{taskBN}

\begin{taskBN}{37}
\addpictocenter[scale=0.4]{images/99251461618833n0}Установите соответствие между графиками функций и формулами, которые их задают. Формулы: \\1) $-2x^2-x+4$\\2) $-2x^2-8x-7$\\3) $-2x^2+7x+2$\\4) $-2x^2+8x-9$
\end{taskBN}

\begin{taskBN}{38}
\addpictocenter[scale=0.4]{images/069936739399532n0}Установите соответствие между графиками функций и формулами, которые их задают. Формулы: \\1) $\frac{-4}{x-2}+2$\\2) $\frac{-3}{x+2}+2$\\3) $\frac{2}{x-4}-2$\\4) $\frac{1}{x+3}-3$
\end{taskBN}

\begin{taskBN}{39}
\addpictocenter[scale=0.4]{images/313821291298223n0}Установите соответствие между графиками функций и формулами, которые их задают. Формулы: \\1) $3x+1$\\2) $\frac{2}{x-3}-3$\\3) $\frac{-4}{x-1}-3$\\4) $-2\sqrt{x-7}+2$
\end{taskBN}

\begin{taskBN}{40}
\addpictocenter[scale=0.4]{images/4235216943398776n0}Установите соответствие между графиками функций и формулами, которые их задают. Формулы: \\1) $1.5\sqrt{x}-1$\\2) $-5\sqrt{x}+6$\\3) $3.5\sqrt{x}-3$\\4) $3\sqrt{x}-1$
\end{taskBN}

\begin{taskBN}{41}
\addpictocenter[scale=0.4]{images/3395814593473154n0}Установите соответствие между графиками функций и формулами, которые их задают. Формулы: \\1) $\sqrt{x}-6$\\2) $5\sqrt{x}-4$\\3) $-3.5\sqrt{x}+5$\\4) $-3.5\sqrt{x}+5$
\end{taskBN}

\begin{taskBN}{42}
\addpictocenter[scale=0.4]{images/640351994384081n0}Установите соответствие между графиками функций и формулами, которые их задают. Формулы: \\1) $\frac{1}{x-4}+3$\\2) $\frac{-3}{x+3}-1$\\3) $\frac{-3}{x-4}+2$\\4) $\frac{2}{x+4}+2$
\end{taskBN}

\begin{taskBN}{43}
\addpictocenter[scale=0.4]{images/4314060318899313n0}Установите соответствие между графиками функций и формулами, которые их задают. Формулы: \\1) $3\sqrt{x}+1$\\2) $4.5\sqrt{x}-4$\\3) $-2.5\sqrt{x}+2$\\4) $3.5\sqrt{x}-2$
\end{taskBN}

\begin{taskBN}{44}
\addpictocenter[scale=0.4]{images/715618939587745n0}Установите соответствие между графиками функций и формулами, которые их задают. Формулы: \\1) $1.5x+3$\\2) $2x+2$\\3) $-x+6$\\4) $\frac{4}{x+1}+5$
\end{taskBN}

\begin{taskBN}{45}
\addpictocenter[scale=0.4]{images/193054809997987n0}На одном из рисунков изображен график функции $f(x)=-4x^2+x+1$. Укажите номер этого рисунка.
\end{taskBN}

\begin{taskBN}{46}
\addpictocenter[scale=0.4]{images/02202230568501662n0}Установите соответствие между графиками функций и формулами, которые их задают. Формулы: \\1) $2x+9$\\2) $4x-4$\\3) $-2.5x-8$\\4) $-4x+7$
\end{taskBN}

\begin{taskBN}{47}
\addpictocenter[scale=0.4]{images/02550255328133133n0}Установите соответствие между графиками функций и формулами, которые их задают. Формулы: \\1) $3\sqrt{x-9}-5$\\2) $-4x+8$\\3) $1.5x-1$\\4) $4x+3$
\end{taskBN}

\begin{taskBN}{48}
\addpictocenter[scale=0.4]{images/5006688027056594n0}На рисунке изображён график квадратичной функции $y=f(x)$. Какое из следующих утверждений о данной функции верно?\\1) Абсцисса вершины равна $2$\\2) Функция возрастает на промежутке $(-\infty;2]$\\3) Ордината вершины равна $3$\\4) $f(-2)=$ $3.5$\\5) Наименьшее значение функции равно  $10$
\end{taskBN}

\begin{taskBN}{49}
\addpictocenter[scale=0.4]{images/612601087654425n0}Установите соответствие между графиками функций и формулами, которые их задают. Формулы: \\1) $-3x-6$\\2) $-x+7$\\3) $3.5x-4$\\4) $-2.5x+6$
\end{taskBN}

\begin{taskBN}{50}
\addpictocenter[scale=0.4]{images/54880917815492n0}Установите соответствие между графиками функций и формулами, которые их задают. Формулы: \\1) $5\sqrt{x}-4$\\2) $-2.5\sqrt{x}+2$\\3) $2\sqrt{x}-3$\\4) $-2.5\sqrt{x}+1$
\end{taskBN}

\begin{taskBN}{51}
\addpictocenter[scale=0.4]{images/898004215884411n0}Установите соответствие между графиками функций и формулами, которые их задают. Формулы: \\1) $\frac{1}{x-2}+2$\\2) $\frac{2}{x+4}+4$\\3) $\frac{-3}{x+6}+2$\\4) $\frac{4}{x+4}-1$
\end{taskBN}

\begin{taskBN}{52}
\addpictocenter[scale=0.4]{images/156246252233849n0}Установите соответствие между графиками функций и формулами, которые их задают. Формулы: \\1) $-4x^2+2x+7$\\2) $3x^2+5x-7$\\3) $2x^2-7x-4$\\4) $-2x^2+4x-2$
\end{taskBN}

\begin{taskBN}{53}
\addpictocenter[scale=0.4]{images/221110945967119n0}На рисунке изображён график квадратичной функции $y=f(x)$. Какое из следующих утверждений о данной функции неверно?\\1) $f(-5)=$ $-2$\\2) Ордината вершины равна $-4$\\3) Абсцисса вершины равна $6$\\4) Функция убывает на промежутке $(-\infty;-3]$\\5) Наименьшее значение функции равно  $-4$
\end{taskBN}

\begin{taskBN}{54}
\addpictocenter[scale=0.4]{images/826001181366713n0}Установите соответствие между графиками функций и формулами, которые их задают. Формулы: \\1) $-2\sqrt{x}-3$\\2) $-2\sqrt{x}+5$\\3) $-4\sqrt{x}+1$\\4) $\sqrt{x}-3$
\end{taskBN}

\begin{taskBN}{55}
\addpictocenter[scale=0.4]{images/978889487669146n0}Установите соответствие между графиками функций и формулами, которые их задают. Формулы: \\1) $2x+4$\\2) $-5x+10$\\3) $1.5x+5$\\4) $-4\sqrt{x+4}+3$
\end{taskBN}

\begin{taskBN}{56}
\addpictocenter[scale=0.4]{images/0223298095919564n0}Установите соответствие между графиками функций и формулами, которые их задают. Формулы: \\1) $-4\sqrt{x}+4$\\2) $-\sqrt{x}+4$\\3) $4.5\sqrt{x}-4$\\4) $-2\sqrt{x}+6$
\end{taskBN}

\begin{taskBN}{57}
\addpictocenter[scale=0.4]{images/6187020274250288n0}Установите соответствие между графиками функций и формулами, которые их задают. Формулы: \\1) $\frac{-4}{x+1}+2$\\2) $\frac{3}{x-5}-2$\\3) $\frac{-4}{x+2}+4$\\4) $\frac{-5}{x+3}-3$
\end{taskBN}

\begin{taskBN}{58}
\addpictocenter[scale=0.4]{images/898511412378743n0}Установите соответствие между графиками функций и формулами, которые их задают. Формулы: \\1) $5x+6$\\2) $-x-9$\\3) $2x+5$\\4) $2.5x-3$
\end{taskBN}

\begin{taskBN}{59}
\addpictocenter[scale=0.4]{images/8656465395323314n0}Установите соответствие между графиками функций и формулами, которые их задают. Формулы: \\1) $-3x+6$\\2) $\frac{3}{x-1}+4$\\3) $-3x-9$\\4) $1.5x+3$
\end{taskBN}

\begin{taskBN}{60}
\addpictocenter[scale=0.4]{images/209764444749436n0}На рисунке изображён график квадратичной функции $y=f(x)$. Какое из следующих утверждений о данной функции неверно?\\1) Наибольшее значение функции равно  $2$\\2) Абсцисса вершины равна $3$\\3) Функция возрастает на промежутке $(-\infty;0]$\\4) Ордината вершины равна $2$\\5) $f(-4)=$ $-2$
\end{taskBN}

\begin{taskBN}{61}
\addpictocenter[scale=0.4]{images/45206132549893n0}Установите соответствие между графиками функций и формулами, которые их задают. Формулы: \\1) $1.5\sqrt{x}+2$\\2) $3.5\sqrt{x}-4$\\3) $-2\sqrt{x}+2$\\4) $-2\sqrt{x}+3$
\end{taskBN}

\begin{taskBN}{62}
\addpictocenter[scale=0.4]{images/1014483932795083n0}На одном из рисунков изображен график функции $f(x)=4x^2-x$. Укажите номер этого рисунка.
\end{taskBN}

\begin{taskBN}{63}
\addpictocenter[scale=0.4]{images/8933137381418663n0}Установите соответствие между графиками функций и формулами, которые их задают. Формулы: \\1) $\frac{5}{x-2}+3$\\2) $\frac{-4}{x+1}+2$\\3) $\frac{-3}{x+4}+5$\\4) $\frac{2}{x-5}+4$
\end{taskBN}

\begin{taskBN}{64}
\addpictocenter[scale=0.4]{images/617757236645447n0}Установите соответствие между графиками функций и формулами, которые их задают. Формулы: \\1) $\frac{3}{x+2}+2$\\2) $\frac{-2}{x+1}-3$\\3) $\frac{-2}{x+2}-1$\\4) $\frac{4}{x-4}-4$
\end{taskBN}

\begin{taskBN}{65}
\addpictocenter[scale=0.4]{images/588964034624508n0}Установите соответствие между графиками функций и формулами, которые их задают. Формулы: \\1) $2.5\sqrt{x}-3$\\2) $-3\sqrt{x}+4$\\3) $\sqrt{x}-5$\\4) $3\sqrt{x}-3$
\end{taskBN}

\begin{taskBN}{66}
\addpictocenter[scale=0.4]{images/451526639476451n0}Установите соответствие между графиками функций и формулами, которые их задают. Формулы: \\1) $\frac{-4}{x-2}-2$\\2) $\frac{-2}{x+1}+3$\\3) $\frac{-2}{x-6}+2$\\4) $\frac{4}{x-1}-4$
\end{taskBN}

\begin{taskBN}{67}
\addpictocenter[scale=0.4]{images/598507844935631n0}Установите соответствие между графиками функций и формулами, которые их задают. Формулы: \\1) $\frac{3}{x-4}-2$\\2) $\frac{-5}{x-2}+3$\\3) $\frac{2}{x-1}-5$\\4) $\frac{-3}{x+2}+4$
\end{taskBN}

\begin{taskBN}{68}
\addpictocenter[scale=0.4]{images/831248518277368n0}Установите соответствие между графиками функций и формулами, которые их задают. Формулы: \\1) $2x^2+5x-8$\\2) $3x^2-6x+5$\\3) $-4x^2-3x+3$\\4) $x^2+x-6$
\end{taskBN}

\begin{taskBN}{69}
\addpictocenter[scale=0.4]{images/087107206848234n0}На рисунке изображён график квадратичной функции $y=f(x)$. Какое из следующих утверждений о данной функции неверно?\\1) Функция убывает на промежутке $(-\infty;-2]$\\2) $f(-4)=$ $0$\\3) Ордината вершины равна $-1$\\4) Абсцисса вершины равна $8.5$\\5) Наименьшее значение функции равно  $-1$
\end{taskBN}

\begin{taskBN}{70}
\addpictocenter[scale=0.4]{images/240474984035853n0}Установите соответствие между графиками функций и формулами, которые их задают. Формулы: \\1) $1.5x^2-6x+2$\\2) $-3x^2+7x+2$\\3) $x^2+6x+6$\\4) $4x^2-6x-1$
\end{taskBN}

\begin{taskBN}{71}
\addpictocenter[scale=0.4]{images/032734853479851n0}Установите соответствие между графиками функций и формулами, которые их задают. Формулы: \\1) $-2\sqrt{x}-3$\\2) $5\sqrt{x}-5$\\3) $4\sqrt{x}-4$\\4) $2\sqrt{x}-5$
\end{taskBN}

\begin{taskBN}{72}
\addpictocenter[scale=0.4]{images/33628118907318n0}На рисунке изображён график квадратичной функции $y=f(x)$. Какое из следующих утверждений о данной функции верно?\\1) Ордината вершины равна $9$\\2) Функция возрастает на промежутке $[4; \infty)$\\3) Абсцисса вершины равна $4$\\4) $f(-2)=$ $2.5$\\5) Наибольшее значение функции равно  $3$
\end{taskBN}

\begin{taskBN}{73}
\addpictocenter[scale=0.4]{images/4990351752676667n0}Установите соответствие между графиками функций и формулами, которые их задают. Формулы: \\1) $\frac{-2}{x-3}-4$\\2) $\frac{2}{x-1}+2$\\3) $\frac{-3}{x-5}-3$\\4) $\frac{3}{x+2}+5$
\end{taskBN}

\begin{taskBN}{74}
\addpictocenter[scale=0.4]{images/862275106337597n0}Установите соответствие между графиками функций и формулами, которые их задают. Формулы: \\1) $2.5\sqrt{x}-5$\\2) $-2\sqrt{x}-1$\\3) $5\sqrt{x}-2$\\4) $-2.5\sqrt{x}+5$
\end{taskBN}

\begin{taskBN}{75}
\addpictocenter[scale=0.4]{images/02807354391914n0}На одном из рисунков изображен график функции $f(x)=-2x^2+4$. Укажите номер этого рисунка.
\end{taskBN}

\begin{taskBN}{76}
\addpictocenter[scale=0.4]{images/336607720629683n0}Установите соответствие между графиками функций и формулами, которые их задают. Формулы: \\1) $-4.5x+5$\\2) $-5x-8$\\3) $-2x-4$\\4) $-2.5x-7$
\end{taskBN}

\begin{taskBN}{77}
\addpictocenter[scale=0.4]{images/492308999252552n0}Установите соответствие между графиками функций и формулами, которые их задают. Формулы: \\1) $1.5x+1$\\2) $-x-8$\\3) $\frac{2}{x+2}-4$\\4) $-3\sqrt{x+7}+5$
\end{taskBN}

\begin{taskBN}{78}
\addpictocenter[scale=0.4]{images/278959541818445n0}Установите соответствие между графиками функций и формулами, которые их задают. Формулы: \\1) $\frac{-2}{x+5}+4$\\2) $\frac{5}{x-4}+2$\\3) $\frac{3}{x-1}+4$\\4) $\frac{4}{x+4}-4$
\end{taskBN}

\begin{taskBN}{79}
\addpictocenter[scale=0.4]{images/3374273992493768n0}Установите соответствие между графиками функций и формулами, которые их задают. Формулы: \\1) $5x^2+7x-2$\\2) $-3x^2+5x-1$\\3) $2x^2+6x+3$\\4) $-1.5x^2-6x-5$
\end{taskBN}

\begin{taskBN}{80}
\addpictocenter[scale=0.4]{images/649292902870085n0}Установите соответствие между графиками функций и формулами, которые их задают. Формулы: \\1) $4x^2+5x-5$\\2) $3x^2-6x+3$\\3) $2x^2+3x-7$\\4) $-2x^2+6x+5$
\end{taskBN}

\begin{taskBN}{81}
\addpictocenter[scale=0.4]{images/788210148455199n0}Установите соответствие между графиками функций и формулами, которые их задают. Формулы: \\1) $-4x+7$\\2) $-2x+8$\\3) $-\sqrt{x-6}-2$\\4) $3\sqrt{x-5}-5$
\end{taskBN}

\begin{taskBN}{82}
\addpictocenter[scale=0.4]{images/824286852316451n0}Установите соответствие между графиками функций и формулами, которые их задают. Формулы: \\1) $-x^2+x+10$\\2) $3x^2-7x-2$\\3) $-4x^2-5x+2$\\4) $-4x^2+x+2$
\end{taskBN}

\begin{taskBN}{83}
\addpictocenter[scale=0.4]{images/790176531429735n0}На одном из рисунков изображен график функции $f(x)=-x^2-4x-1$. Укажите номер этого рисунка.
\end{taskBN}

\begin{taskBN}{84}
\addpictocenter[scale=0.4]{images/028452313669612n0}Установите соответствие между графиками функций и формулами, которые их задают. Формулы: \\1) $\frac{2}{x-5}+3$\\2) $\frac{4}{x+4}+2$\\3) $\frac{-1}{x+1}-3$\\4) $\frac{3}{x+4}+2$
\end{taskBN}

\begin{taskBN}{85}
\addpictocenter[scale=0.4]{images/376629200174634n0}Установите соответствие между графиками функций и формулами, которые их задают. Формулы: \\1) $-2\sqrt{x}-2$\\2) $-5\sqrt{x}+6$\\3) $-2\sqrt{x}-1$\\4) $-3.5\sqrt{x}+3$
\end{taskBN}

\begin{taskBN}{86}
\addpictocenter[scale=0.4]{images/440402603876268n0}Установите соответствие между графиками функций и формулами, которые их задают. Формулы: \\1) $-4\sqrt{x}+5$\\2) $2\sqrt{x}-1$\\3) $-3.5\sqrt{x}+5$\\4) $1.5\sqrt{x}+1$
\end{taskBN}

\begin{taskBN}{87}
\addpictocenter[scale=0.4]{images/00902188594010811n0}Установите соответствие между графиками функций и формулами, которые их задают. Формулы: \\1) $-2x+4$\\2) $1.5x+5$\\3) $-\sqrt{x+9}+3$\\4) $2\sqrt{x+6}-2$
\end{taskBN}

\begin{taskBN}{88}
\addpictocenter[scale=0.4]{images/0847625612370564n0}На одном из рисунков изображен график функции $f(x)=x^2-2x-2$. Укажите номер этого рисунка.
\end{taskBN}

\begin{taskBN}{89}
\addpictocenter[scale=0.4]{images/9401265971394885n0}Установите соответствие между графиками функций и формулами, которые их задают. Формулы: \\1) $\frac{4}{x-5}+3$\\2) $\frac{-4}{x+2}+2$\\3) $\sqrt{x-8}+1$\\4) $-2\sqrt{x+1}+1$
\end{taskBN}

\begin{taskBN}{90}
\addpictocenter[scale=0.4]{images/593716191596201n0}Установите соответствие между графиками функций и формулами, которые их задают. Формулы: \\1) $\frac{-2}{x-5}-3$\\2) $\frac{-2}{x+2}-1$\\3) $\frac{5}{x-4}+5$\\4) $\frac{4}{x+3}+2$
\end{taskBN}

\begin{taskBN}{91}
\addpictocenter[scale=0.4]{images/823167377567038n0}На одном из рисунков изображен график функции $f(x)=-4x^2+1$. Укажите номер этого рисунка.
\end{taskBN}

\begin{taskBN}{92}
\addpictocenter[scale=0.4]{images/193251247989517n0}На одном из рисунков изображен график функции $f(x)=-x^2+2x$. Укажите номер этого рисунка.
\end{taskBN}

\begin{taskBN}{93}
\addpictocenter[scale=0.4]{images/2625665661176906n0}На рисунке изображён график квадратичной функции $y=f(x)$. Какое из следующих утверждений о данной функции верно?\\1) Абсцисса вершины равна $-4$\\2) Функция возрастает на промежутке $(-\infty;-4]$\\3) Ордината вершины равна $9$\\4) Наименьшее значение функции равно  $8$\\5) $f(-4)=$ $3.5$
\end{taskBN}

\begin{taskBN}{94}
\addpictocenter[scale=0.4]{images/593181391709269n0}Установите соответствие между графиками функций и формулами, которые их задают. Формулы: \\1) $-1.5\sqrt{x}+3$\\2) $3.5\sqrt{x}-5$\\3) $-\sqrt{x}+6$\\4) $-4.5\sqrt{x}+5$
\end{taskBN}

\begin{taskBN}{95}
\addpictocenter[scale=0.4]{images/1762151091854505n0}Установите соответствие между графиками функций и формулами, которые их задают. Формулы: \\1) $4x^2-3x-5$\\2) $-2x^2-6x+5$\\3) $-2x^2+3x-2$\\4) $-x^2+4x-5$
\end{taskBN}

\begin{taskBN}{96}
\addpictocenter[scale=0.4]{images/6751355067585816n0}Установите соответствие между графиками функций и формулами, которые их задают. Формулы: \\1) $1.5\sqrt{x}-5$\\2) $3.5\sqrt{x}-3$\\3) $4\sqrt{x}-5$\\4) $-3\sqrt{x}+3$
\end{taskBN}

\begin{taskBN}{97}
\addpictocenter[scale=0.4]{images/6879502887555953n0}Установите соответствие между графиками функций и формулами, которые их задают. Формулы: \\1) $5x+10$\\2) $4x+2$\\3) $x+4$\\4) $-2x+9$
\end{taskBN}

\begin{taskBN}{98}
\addpictocenter[scale=0.4]{images/442962125149111n0}Установите соответствие между графиками функций и формулами, которые их задают. Формулы: \\1) $4\sqrt{x+4}-4$\\2) $3\sqrt{x+2}-5$\\3) $-x+6$\\4) $-2\sqrt{x+3}+3$
\end{taskBN}

\begin{taskBN}{99}
\addpictocenter[scale=0.4]{images/645439592994921n0}Установите соответствие между графиками функций и формулами, которые их задают. Формулы: \\1) $-\sqrt{x+6}-1$\\2) $-x-7$\\3) $-3x-5$\\4) $-2\sqrt{x-10}+2$
\end{taskBN}

\begin{taskBN}{100}
\addpictocenter[scale=0.4]{images/728091029957117n0}Установите соответствие между графиками функций и формулами, которые их задают. Формулы: \\1) $-3x^2-6x+4$\\2) $3x^2+x-6$\\3) $-2x^2+6x+4$\\4) $4x^2+7x+2$
\end{taskBN}

\begin{taskBN}{101}
\addpictocenter[scale=0.4]{images/892117256892582n0}Установите соответствие между графиками функций и формулами, которые их задают. Формулы: \\1) $\frac{3}{x+4}+1$\\2) $\frac{-5}{x+3}-2$\\3) $\frac{2}{x-6}+3$\\4) $\frac{1}{x+2}+2$
\end{taskBN}

\begin{taskBN}{102}
\addpictocenter[scale=0.4]{images/4744337180338427n0}Установите соответствие между графиками функций и формулами, которые их задают. Формулы: \\1) $-2x^2+8x-7$\\2) $-2x^2-x+1$\\3) $-2x^2+3x+4$\\4) $3x^2-2x-6$
\end{taskBN}

\begin{taskBN}{103}
\addpictocenter[scale=0.4]{images/751004317504716n0}Установите соответствие между графиками функций и формулами, которые их задают. Формулы: \\1) $-2x+4$\\2) $-5x+5$\\3) $4x-1$\\4) $-3x-7$
\end{taskBN}

\begin{taskBN}{104}
\addpictocenter[scale=0.4]{images/778285208622208n0}Установите соответствие между графиками функций и формулами, которые их задают. Формулы: \\1) $2x+6$\\2) $-3\sqrt{x-4}+3$\\3) $3x+3$\\4) $-x-3$
\end{taskBN}

\begin{taskBN}{105}
\addpictocenter[scale=0.4]{images/337516253723157n0}Установите соответствие между графиками функций и формулами, которые их задают. Формулы: \\1) $-2x^2-8x-8$\\2) $-5x^2+7x+2$\\3) $-3x^2-x+5$\\4) $2x^2+5x-3$
\end{taskBN}

\begin{taskBN}{106}
\addpictocenter[scale=0.4]{images/4599523068933562n0}Установите соответствие между графиками функций и формулами, которые их задают. Формулы: \\1) $\frac{1}{x+2}-4$\\2) $\frac{2}{x-2}+1$\\3) $\frac{-3}{x+5}-2$\\4) $\frac{5}{x+2}-4$
\end{taskBN}

\begin{taskBN}{107}
\addpictocenter[scale=0.4]{images/328193131467475n0}Установите соответствие между графиками функций и формулами, которые их задают. Формулы: \\1) $-1.5x^2+6x-5$\\2) $x^2+x-5$\\3) $2x^2-7x+7$\\4) $-2x^2-5x+2$
\end{taskBN}

\begin{taskBN}{108}
\addpictocenter[scale=0.4]{images/682315994189164n0}Установите соответствие между графиками функций и формулами, которые их задают. Формулы: \\1) $-2.5\sqrt{x}+2$\\2) $-4\sqrt{x}+3$\\3) $2\sqrt{x}-4$\\4) $-4.5\sqrt{x}+4$
\end{taskBN}

\begin{taskBN}{109}
\addpictocenter[scale=0.4]{images/334254169292207n0}Установите соответствие между графиками функций и формулами, которые их задают. Формулы: \\1) $-2.5\sqrt{x}+5$\\2) $-2\sqrt{x}+5$\\3) $4\sqrt{x}-5$\\4) $1.5\sqrt{x}-5$
\end{taskBN}

\begin{taskBN}{110}
\addpictocenter[scale=0.4]{images/718386855638597n0}Установите соответствие между графиками функций и формулами, которые их задают. Формулы: \\1) $\frac{4}{x+2}-3$\\2) $\frac{-3}{x-1}+4$\\3) $\frac{1}{x-4}-3$\\4) $\frac{5}{x+1}-4$
\end{taskBN}

\begin{taskBN}{111}
\addpictocenter[scale=0.4]{images/355334881754803n0}Установите соответствие между графиками функций и формулами, которые их задают. Формулы: \\1) $3x-4$\\2) $1.5x-1$\\3) $2x+5$\\4) $-x-2$
\end{taskBN}

\begin{taskBN}{112}
\addpictocenter[scale=0.4]{images/671435770019045n0}Установите соответствие между графиками функций и формулами, которые их задают. Формулы: \\1) $\frac{3}{x+1}+3$\\2) $\frac{-4}{x+1}+3$\\3) $\frac{1}{x+4}+3$\\4) $\frac{-4}{x+4}-3$
\end{taskBN}

\begin{taskBN}{113}
\addpictocenter[scale=0.4]{images/147469950832352n0}На одном из рисунков изображен график функции $f(x)=4x^2-x-3$. Укажите номер этого рисунка.
\end{taskBN}

\begin{taskBN}{114}
\addpictocenter[scale=0.4]{images/5216983888286928n0}Установите соответствие между графиками функций и формулами, которые их задают. Формулы: \\1) $2x^2-8x+7$\\2) $1.5x^2-6x+1$\\3) $3x^2-3x-2$\\4) $4x^2+6x-5$
\end{taskBN}

\begin{taskBN}{115}
\addpictocenter[scale=0.4]{images/896957185484746n0}Установите соответствие между графиками функций и формулами, которые их задают. Формулы: \\1) $\frac{-1}{x+3}+2$\\2) $\frac{-2}{x+1}-2$\\3) $\frac{5}{x+2}+3$\\4) $\frac{-3}{x+3}+1$
\end{taskBN}

\begin{taskBN}{116}
\addpictocenter[scale=0.4]{images/241027690430064n0}Установите соответствие между графиками функций и формулами, которые их задают. Формулы: \\1) $4x^2+3x-3$\\2) $-2x^2-4x-3$\\3) $-x^2+5x-2$\\4) $2x^2-7x+7$
\end{taskBN}

\begin{taskBN}{117}
\addpictocenter[scale=0.4]{images/304077443226712n0}Установите соответствие между графиками функций и формулами, которые их задают. Формулы: \\1) $x+6$\\2) $2.5x+4$\\3) $3x+4$\\4) $5x-1$
\end{taskBN}

\begin{taskBN}{118}
\addpictocenter[scale=0.4]{images/367782559914664n0}Установите соответствие между графиками функций и формулами, которые их задают. Формулы: \\1) $5\sqrt{x+3}-5$\\2) $-2x-8$\\3) $\frac{4}{x-1}+2$\\4) $-2\sqrt{x+4}+1$
\end{taskBN}

\begin{taskBN}{119}
\addpictocenter[scale=0.4]{images/115214726578003n0}Установите соответствие между графиками функций и формулами, которые их задают. Формулы: \\1) $-3x^2+6x-5$\\2) $-2x^2-9x-2$\\3) $-2x^2-5x-3$\\4) $2x^2-9x+2$
\end{taskBN}

\begin{taskBN}{120}
\addpictocenter[scale=0.4]{images/964274812761608n0}Установите соответствие между графиками функций и формулами, которые их задают. Формулы: \\1) $-2\sqrt{x-2}+3$\\2) $3x-6$\\3) $-2\sqrt{x-8}+2$\\4) $-3\sqrt{x+2}+4$
\end{taskBN}

\begin{taskBN}{121}
\addpictocenter[scale=0.4]{images/5168410430205463n0}Установите соответствие между графиками функций и формулами, которые их задают. Формулы: \\1) $-3\sqrt{x+6}+5$\\2) $5\sqrt{x-4}-5$\\3) $-x+2$\\4) $-\sqrt{x-4}+3$
\end{taskBN}

\begin{taskBN}{122}
\addpictocenter[scale=0.4]{images/543322888977561n0}На рисунке изображён график квадратичной функции $y=f(x)$. Какое из следующих утверждений о данной функции верно?\\1) Абсцисса вершины равна $0$\\2) Наименьшее значение функции равно  $2$\\3) $f(-4)=$ $7$\\4) Функция возрастает на промежутке $(-\infty;0]$\\5) Ордината вершины равна $7.5$
\end{taskBN}

\begin{taskBN}{123}
\addpictocenter[scale=0.4]{images/39140241325147485n0}Установите соответствие между графиками функций и формулами, которые их задают. Формулы: \\1) $3x^2-2x-5$\\2) $4x^2+5x-4$\\3) $-2x^2+9x-8$\\4) $-x^2+3x+9$
\end{taskBN}

\begin{taskBN}{124}
\addpictocenter[scale=0.4]{images/12192958930690612n0}Установите соответствие между графиками функций и формулами, которые их задают. Формулы: \\1) $\frac{4}{x+1}-4$\\2) $\frac{1}{x+2}-4$\\3) $\frac{-3}{x+1}-3$\\4) $\frac{-2}{x-4}-1$
\end{taskBN}

\begin{taskBN}{125}
\addpictocenter[scale=0.4]{images/19004576890685n0}На одном из рисунков изображен график функции $f(x)=-x^2-1$. Укажите номер этого рисунка.
\end{taskBN}

\begin{taskBN}{126}
\addpictocenter[scale=0.4]{images/0516849223500147n0}На рисунке изображён график квадратичной функции $y=f(x)$. Какое из следующих утверждений о данной функции верно?\\1) Функция убывает на промежутке $(-\infty;0]$\\2) Ордината вершины равна $6$\\3) Абсцисса вершины равна $0$\\4) $f(-4)=$ $3$\\5) Наибольшее значение функции равно  $2$
\end{taskBN}

\begin{taskBN}{127}
\addpictocenter[scale=0.4]{images/0027455831544263n0}Установите соответствие между графиками функций и формулами, которые их задают. Формулы: \\1) $-3x^2-6x-3$\\2) $-2x^2+8x-2$\\3) $4x^2-4x-3$\\4) $-2x^2+2x+2$
\end{taskBN}

\begin{taskBN}{128}
\addpictocenter[scale=0.4]{images/779960865463897n0}Установите соответствие между графиками функций и формулами, которые их задают. Формулы: \\1) $\frac{-2}{x+2}-1$\\2) $\frac{-4}{x+6}+3$\\3) $\frac{4}{x-2}-2$\\4) $\frac{-2}{x+3}-3$
\end{taskBN}

\begin{taskBN}{129}
\addpictocenter[scale=0.4]{images/9235936262503175n0}Установите соответствие между графиками функций и формулами, которые их задают. Формулы: \\1) $4.5\sqrt{x}-4$\\2) $3\sqrt{x}+2$\\3) $3.5\sqrt{x}-2$\\4) $-4\sqrt{x}+2$
\end{taskBN}

\begin{taskBN}{130}
\addpictocenter[scale=0.4]{images/514426993106045n0}Установите соответствие между графиками функций и формулами, которые их задают. Формулы: \\1) $\frac{4}{x-3}-3$\\2) $\frac{2}{x+2}-4$\\3) $\frac{-5}{x+1}+4$\\4) $\frac{3}{x+3}+4$
\end{taskBN}

\begin{taskBN}{131}
\addpictocenter[scale=0.4]{images/9220076085135331n0}Установите соответствие между графиками функций и формулами, которые их задают. Формулы: \\1) $-2.5x+6$\\2) $-1.5x-5$\\3) $3x+3$\\4) $x+7$
\end{taskBN}

\begin{taskBN}{132}
\addpictocenter[scale=0.4]{images/268673883922573n0}Установите соответствие между графиками функций и формулами, которые их задают. Формулы: \\1) $-2.5\sqrt{x}+3$\\2) $-1.5\sqrt{x}-2$\\3) $-2\sqrt{x}+4$\\4) $-4\sqrt{x}+5$
\end{taskBN}

\begin{taskBN}{133}
\addpictocenter[scale=0.4]{images/9498321901529563n0}Установите соответствие между графиками функций и формулами, которые их задают. Формулы: \\1) $-x+7$\\2) $1.5x+5$\\3) $-3x+3$\\4) $\frac{-4}{x+4}-2$
\end{taskBN}

\begin{taskBN}{134}
\addpictocenter[scale=0.4]{images/3409298902564675n0}Установите соответствие между графиками функций и формулами, которые их задают. Формулы: \\1) $\frac{-4}{x+6}+3$\\2) $\frac{-4}{x+2}+3$\\3) $\frac{-4}{x+4}-3$\\4) $\frac{3}{x+2}-2$
\end{taskBN}

\begin{taskBN}{135}
\addpictocenter[scale=0.4]{images/14288086121092n0}Установите соответствие между графиками функций и формулами, которые их задают. Формулы: \\1) $-3\sqrt{x}+4$\\2) $4\sqrt{x}-2$\\3) $3.5\sqrt{x}-2$\\4) $-3.5\sqrt{x}+5$
\end{taskBN}

\begin{taskBN}{136}
\addpictocenter[scale=0.4]{images/0556145773842n0}Установите соответствие между графиками функций и формулами, которые их задают. Формулы: \\1) $-2.5\sqrt{x}+1$\\2) $1.5\sqrt{x}+2$\\3) $2.5\sqrt{x}-4$\\4) $4\sqrt{x}-4$
\end{taskBN}

\begin{taskBN}{137}
\addpictocenter[scale=0.4]{images/135969343674636n0}Установите соответствие между графиками функций и формулами, которые их задают. Формулы: \\1) $4x^2-2x-7$\\2) $-2x^2+3x-1$\\3) $-3x^2-5x-3$\\4) $-5x^2-2x+4$
\end{taskBN}

\begin{taskBN}{138}
\addpictocenter[scale=0.4]{images/0632240745201875n0}Установите соответствие между графиками функций и формулами, которые их задают. Формулы: \\1) $2.5x+5$\\2) $-4x-10$\\3) $-2x-4$\\4) $-5x+4$
\end{taskBN}

\begin{taskBN}{139}
\addpictocenter[scale=0.4]{images/975520323227629n0}Установите соответствие между графиками функций и формулами, которые их задают. Формулы: \\1) $4.5x+4$\\2) $-1.5x-7$\\3) $-4x+9$\\4) $-3.5x+9$
\end{taskBN}

\begin{taskBN}{140}
\addpictocenter[scale=0.4]{images/564127339706283n0}Установите соответствие между графиками функций и формулами, которые их задают. Формулы: \\1) $-2.5x+5$\\2) $x-4$\\3) $-2\sqrt{x+1}+2$\\4) $\frac{3}{x-2}+2$
\end{taskBN}

\begin{taskBN}{141}
\addpictocenter[scale=0.4]{images/6249779283552057n0}На рисунке изображён график квадратичной функции $y=f(x)$. Какое из следующих утверждений о данной функции неверно?\\1) $f(-4)=$ $5$\\2) Абсцисса вершины равна $2$\\3) Ордината вершины равна $5$\\4) Наибольшее значение функции равно  $5$\\5) Функция возрастает на промежутке $(-\infty;-4]$
\end{taskBN}

\begin{taskBN}{142}
\addpictocenter[scale=0.4]{images/1253813502803074n0}Установите соответствие между графиками функций и формулами, которые их задают. Формулы: \\1) $\frac{3}{x+5}-1$\\2) $\frac{-3}{x-4}-3$\\3) $\frac{4}{x+3}-4$\\4) $\frac{-3}{x+6}+1$
\end{taskBN}

\begin{taskBN}{143}
\addpictocenter[scale=0.4]{images/8976379516647n0}Установите соответствие между графиками функций и формулами, которые их задают. Формулы: \\1) $-3x+6$\\2) $2x-5$\\3) $2.5x-3$\\4) $-4x+5$
\end{taskBN}

\begin{taskBN}{144}
\addpictocenter[scale=0.4]{images/366625367142634n0}Установите соответствие между графиками функций и формулами, которые их задают. Формулы: \\1) $\frac{5}{x-3}+5$\\2) $\frac{4}{x-2}-2$\\3) $\frac{3}{x+4}+3$\\4) $\frac{-1}{x+4}+4$
\end{taskBN}

\begin{taskBN}{145}
\addpictocenter[scale=0.4]{images/429243972878091n0}Установите соответствие между графиками функций и формулами, которые их задают. Формулы: \\1) $2x+8$\\2) $\frac{-3}{x+1}+2$\\3) $4x-5$\\4) $-\sqrt{x+3}+1$
\end{taskBN}

\begin{taskBN}{146}
\addpictocenter[scale=0.4]{images/630756003808312n0}Установите соответствие между графиками функций и формулами, которые их задают. Формулы: \\1) $-2x+2$\\2) $4\sqrt{x+6}-4$\\3) $-5x-5$\\4) $3x-2$
\end{taskBN}

\begin{taskBN}{147}
\addpictocenter[scale=0.4]{images/4515697090710666n0}Установите соответствие между графиками функций и формулами, которые их задают. Формулы: \\1) $3\sqrt{x}-5$\\2) $2\sqrt{x}-4$\\3) $-3\sqrt{x}-1$\\4) $-5\sqrt{x}+4$
\end{taskBN}

\begin{taskBN}{148}
\addpictocenter[scale=0.4]{images/027198537500882n0}Установите соответствие между графиками функций и формулами, которые их задают. Формулы: \\1) $2\sqrt{x}-6$\\2) $-4\sqrt{x}+2$\\3) $-4\sqrt{x}+2$\\4) $1.5\sqrt{x}-3$
\end{taskBN}

\begin{taskBN}{149}
\addpictocenter[scale=0.4]{images/815912291161287n0}Установите соответствие между графиками функций и формулами, которые их задают. Формулы: \\1) $-3\sqrt{x+4}+2$\\2) $-2x+10$\\3) $-x+1$\\4) $\frac{-3}{x-1}-3$
\end{taskBN}

\begin{taskBN}{150}
\addpictocenter[scale=0.4]{images/622073454857956n0}Установите соответствие между графиками функций и формулами, которые их задают. Формулы: \\1) $-2\sqrt{x+2}+5$\\2) $\frac{-4}{x-6}-5$\\3) $\sqrt{x+5}-3$\\4) $\frac{2}{x+1}-3$
\end{taskBN}

\begin{taskBN}{151}
\addpictocenter[scale=0.4]{images/1593135645282713n0}Установите соответствие между графиками функций и формулами, которые их задают. Формулы: \\1) $-3x+4$\\2) $-1.5x+6$\\3) $-x+8$\\4) $-3x+5$
\end{taskBN}

\begin{taskBN}{152}
\addpictocenter[scale=0.4]{images/1347749326231582n0}Установите соответствие между графиками функций и формулами, которые их задают. Формулы: \\1) $4.5\sqrt{x}-5$\\2) $-2.5\sqrt{x}+2$\\3) $-\sqrt{x}-4$\\4) $-4.5\sqrt{x}+4$
\end{taskBN}

\begin{taskBN}{153}
\addpictocenter[scale=0.4]{images/761215251585657n0}Установите соответствие между графиками функций и формулами, которые их задают. Формулы: \\1) $2x^2-6x+4$\\2) $4x^2-9x+5$\\3) $5x^2+9x+3$\\4) $3x^2+4x-4$
\end{taskBN}

\begin{taskBN}{154}
\addpictocenter[scale=0.4]{images/882851344759508n0}Установите соответствие между графиками функций и формулами, которые их задают. Формулы: \\1) $4x+9$\\2) $3x+7$\\3) $-2x+1$\\4) $-2\sqrt{x+2}+5$
\end{taskBN}

\begin{taskBN}{155}
\addpictocenter[scale=0.4]{images/25110334510013n0}Установите соответствие между графиками функций и формулами, которые их задают. Формулы: \\1) $-x^2-2x-4$\\2) $4x^2+2x-7$\\3) $x^2+4x+8$\\4) $2x^2-5x+2$
\end{taskBN}

\begin{taskBN}{156}
\addpictocenter[scale=0.4]{images/057859595233785n0}Установите соответствие между графиками функций и формулами, которые их задают. Формулы: \\1) $-4x^2+6x+3$\\2) $x^2-2x+2$\\3) $3x^2-3x-2$\\4) $-3x^2+6x+4$
\end{taskBN}

\begin{taskBN}{157}
\addpictocenter[scale=0.4]{images/526940458803965n0}Установите соответствие между графиками функций и формулами, которые их задают. Формулы: \\1) $-4x-9$\\2) $\sqrt{x-5}-5$\\3) $-3\sqrt{x+3}+3$\\4) $-\sqrt{x+7}+5$
\end{taskBN}

\begin{taskBN}{158}
\addpictocenter[scale=0.4]{images/4671252381407238n0}Установите соответствие между графиками функций и формулами, которые их задают. Формулы: \\1) $\frac{4}{x+3}+2$\\2) $\frac{-4}{x+5}+3$\\3) $\frac{-4}{x-4}+1$\\4) $\frac{3}{x+5}-4$
\end{taskBN}

\begin{taskBN}{159}
\addpictocenter[scale=0.4]{images/229898516802411n0}Установите соответствие между графиками функций и формулами, которые их задают. Формулы: \\1) $\frac{-5}{x+1}+3$\\2) $\frac{2}{x-3}+2$\\3) $\frac{-4}{x-6}+1$\\4) $\frac{-3}{x-2}-2$
\end{taskBN}

\begin{taskBN}{160}
\addpictocenter[scale=0.4]{images/983397521484753n0}Установите соответствие между графиками функций и формулами, которые их задают. Формулы: \\1) $-2\sqrt{x}+5$\\2) $2\sqrt{x}-3$\\3) $-\sqrt{x}+2$\\4) $4\sqrt{x}-3$
\end{taskBN}

\begin{taskBN}{161}
\addpictocenter[scale=0.4]{images/745564308798061n0}Установите соответствие между графиками функций и формулами, которые их задают. Формулы: \\1) $-1.5x-5$\\2) $-3\sqrt{x+6}+4$\\3) $\sqrt{x-3}-4$\\4) $2x-7$
\end{taskBN}

\begin{taskBN}{162}
\addpictocenter[scale=0.4]{images/25447320811599394n0}Установите соответствие между графиками функций и формулами, которые их задают. Формулы: \\1) $-4.5\sqrt{x}+4$\\2) $-\sqrt{x}-4$\\3) $1.5\sqrt{x}-1$\\4) $2.5\sqrt{x}-1$
\end{taskBN}

\begin{taskBN}{163}
\addpictocenter[scale=0.4]{images/485791076878449n0}На одном из рисунков изображен график функции $f(x)=-x^2+2x-1$. Укажите номер этого рисунка.
\end{taskBN}

\begin{taskBN}{164}
\addpictocenter[scale=0.4]{images/393563415541224n0}Установите соответствие между графиками функций и формулами, которые их задают. Формулы: \\1) $-4\sqrt{x+7}+3$\\2) $3x-9$\\3) $\frac{-2}{x-4}-3$\\4) $1.5x+5$
\end{taskBN}

\begin{taskBN}{165}
\addpictocenter[scale=0.4]{images/241495602515302n0}Установите соответствие между графиками функций и формулами, которые их задают. Формулы: \\1) $3\sqrt{x+2}-5$\\2) $-2\sqrt{x+10}+2$\\3) $-2\sqrt{x-7}+4$\\4) $-3x-4$
\end{taskBN}

\begin{taskBN}{166}
\addpictocenter[scale=0.4]{images/2188772499625877n0}Установите соответствие между графиками функций и формулами, которые их задают. Формулы: \\1) $\frac{4}{x+5}+1$\\2) $\frac{-2}{x-3}+3$\\3) $\frac{3}{x+4}-2$\\4) $\frac{-4}{x+3}-3$
\end{taskBN}

\begin{taskBN}{167}
\addpictocenter[scale=0.4]{images/234726324007222n0}На одном из рисунков изображен график функции $f(x)=-4x^2+x+3$. Укажите номер этого рисунка.
\end{taskBN}

\begin{taskBN}{168}
\addpictocenter[scale=0.4]{images/8149353308626146n0}Установите соответствие между графиками функций и формулами, которые их задают. Формулы: \\1) $-3x-5$\\2) $3\sqrt{x+4}-4$\\3) $-4\sqrt{x-7}+4$\\4) $4\sqrt{x+6}-5$
\end{taskBN}

\begin{taskBN}{169}
\addpictocenter[scale=0.4]{images/17276375535152n0}На рисунке изображён график квадратичной функции $y=f(x)$. Какое из следующих утверждений о данной функции верно?\\1) Функция убывает на промежутке $[4; \infty)$\\2) Абсцисса вершины равна $4$\\3) Наименьшее значение функции равно  $5$\\4) Ордината вершины равна $2$\\5) $f(-2)=$ $9$
\end{taskBN}

\begin{taskBN}{170}
\addpictocenter[scale=0.4]{images/981158016588498n0}Установите соответствие между графиками функций и формулами, которые их задают. Формулы: \\1) $4\sqrt{x}-3$\\2) $-1.5\sqrt{x}-2$\\3) $3.5\sqrt{x}-2$\\4) $-3.5\sqrt{x}+5$
\end{taskBN}

\begin{taskBN}{171}
\addpictocenter[scale=0.4]{images/0815417716297493n0}Установите соответствие между графиками функций и формулами, которые их задают. Формулы: \\1) $\frac{-2}{x+4}+3$\\2) $\frac{-1}{x-2}+4$\\3) $\frac{3}{x-6}-2$\\4) $\frac{-2}{x-5}-2$
\end{taskBN}

\begin{taskBN}{172}
\addpictocenter[scale=0.4]{images/40179217755191954n0}На рисунке изображён график квадратичной функции $y=f(x)$. Какое из следующих утверждений о данной функции неверно?\\1) Функция убывает на промежутке $[0; \infty)$\\2) Абсцисса вершины равна $6$\\3) $f(-4)=$ $-5$\\4) Ордината вершины равна $-1$\\5) Наибольшее значение функции равно  $-1$
\end{taskBN}

\begin{taskBN}{173}
\addpictocenter[scale=0.4]{images/379114633782971n0}На рисунке изображён график квадратичной функции $y=f(x)$. Какое из следующих утверждений о данной функции верно?\\1) Функция убывает на промежутке $[-2; \infty)$\\2) $f(-4)=$ $5$\\3) Наименьшее значение функции равно  $4$\\4) Ордината вершины равна $7$\\5) Абсцисса вершины равна $-2$
\end{taskBN}

\begin{taskBN}{174}
\addpictocenter[scale=0.4]{images/699303535892758n0}Установите соответствие между графиками функций и формулами, которые их задают. Формулы: \\1) $4x^2+5x-5$\\2) $3x^2-9x+2$\\3) $2x^2+5x+1$\\4) $-2x^2+3x+6$
\end{taskBN}

\begin{taskBN}{175}
\addpictocenter[scale=0.4]{images/051520719568223n0}Установите соответствие между графиками функций и формулами, которые их задают. Формулы: \\1) $3x+3$\\2) $2x-7$\\3) $-\sqrt{x-5}-3$\\4) $4x-8$
\end{taskBN}

\begin{taskBN}{176}
\addpictocenter[scale=0.4]{images/60041891633163n0}Установите соответствие между графиками функций и формулами, которые их задают. Формулы: \\1) $3.5x-9$\\2) $4x-2$\\3) $2.5x-3$\\4) $-x+1$
\end{taskBN}

\begin{taskBN}{177}
\addpictocenter[scale=0.4]{images/721936853624537n0}Установите соответствие между графиками функций и формулами, которые их задают. Формулы: \\1) $2x+9$\\2) $-2x+9$\\3) $-3x+8$\\4) $4x+4$
\end{taskBN}

\begin{taskBN}{178}
\addpictocenter[scale=0.4]{images/558206620248087n0}Установите соответствие между графиками функций и формулами, которые их задают. Формулы: \\1) $-2\sqrt{x+4}-1$\\2) $\frac{4}{x+6}-5$\\3) $-x-5$\\4) $-2x-7$
\end{taskBN}

\begin{taskBN}{179}
\addpictocenter[scale=0.4]{images/022885523803185n0}Установите соответствие между графиками функций и формулами, которые их задают. Формулы: \\1) $x^2+4x+6$\\2) $3x^2+10x+5$\\3) $5x^2-9x+3$\\4) $2x^2+2x-9$
\end{taskBN}

\begin{taskBN}{180}
\addpictocenter[scale=0.4]{images/0150029961708127n0}На одном из рисунков изображен график функции $f(x)=-4x^2-x$. Укажите номер этого рисунка.
\end{taskBN}

\begin{taskBN}{181}
\addpictocenter[scale=0.4]{images/566157552297787n0}Установите соответствие между графиками функций и формулами, которые их задают. Формулы: \\1) $2\sqrt{x}+2$\\2) $-2.5\sqrt{x}+4$\\3) $1.5\sqrt{x}+1$\\4) $-1.5\sqrt{x}+5$
\end{taskBN}

\begin{taskBN}{182}
\addpictocenter[scale=0.4]{images/757693546989537n0}Установите соответствие между графиками функций и формулами, которые их задают. Формулы: \\1) $2x^2+6x-4$\\2) $5x^2+x-5$\\3) $4x^2-4x-4$\\4) $-2x^2-2x+3$
\end{taskBN}

\begin{taskBN}{183}
\addpictocenter[scale=0.4]{images/500122073652516n0}Установите соответствие между графиками функций и формулами, которые их задают. Формулы: \\1) $2\sqrt{x-6}-3$\\2) $-4x-5$\\3) $\frac{-4}{x-3}+1$\\4) $1.5x+4$
\end{taskBN}

\begin{taskBN}{184}
\addpictocenter[scale=0.4]{images/415627061434864n0}Установите соответствие между графиками функций и формулами, которые их задают. Формулы: \\1) $x^2+6x+1$\\2) $-3x^2-2x+6$\\3) $2x^2-5x+3$\\4) $-3x^2-8x-3$
\end{taskBN}

\begin{taskBN}{185}
\addpictocenter[scale=0.4]{images/794834566500153n0}Установите соответствие между графиками функций и формулами, которые их задают. Формулы: \\1) $\frac{1}{x+2}+4$\\2) $\frac{-3}{x+2}+5$\\3) $\frac{2}{x-5}+5$\\4) $\frac{4}{x+1}+4$
\end{taskBN}

\begin{taskBN}{186}
\addpictocenter[scale=0.4]{images/028929815850702n0}Установите соответствие между графиками функций и формулами, которые их задают. Формулы: \\1) $4\sqrt{x}-6$\\2) $-3\sqrt{x}-2$\\3) $2.5\sqrt{x}-3$\\4) $3.5\sqrt{x}-2$
\end{taskBN}

\begin{taskBN}{187}
\addpictocenter[scale=0.4]{images/056338934581665n0}На рисунке изображён график квадратичной функции $y=f(x)$. Какое из следующих утверждений о данной функции неверно?\\1) Функция возрастает на промежутке $[2; \infty)$\\2) Абсцисса вершины равна $8.5$\\3) $f(-2)=$ $5$\\4) Наименьшее значение функции равно  $1$\\5) Ордината вершины равна $1$
\end{taskBN}

\begin{taskBN}{188}
\addpictocenter[scale=0.4]{images/4592042306998076n0}Установите соответствие между графиками функций и формулами, которые их задают. Формулы: \\1) $\frac{4}{x-5}+5$\\2) $2\sqrt{x-8}-2$\\3) $-3\sqrt{x-1}+1$\\4) $-3x-6$
\end{taskBN}

\begin{taskBN}{189}
\addpictocenter[scale=0.4]{images/886177746853228n0}Установите соответствие между графиками функций и формулами, которые их задают. Формулы: \\1) $-3x^2+8x-4$\\2) $-x^2+4x+1$\\3) $-3x^2+7x+3$\\4) $-2x^2-7x-6$
\end{taskBN}

\begin{taskBN}{190}
\addpictocenter[scale=0.4]{images/917131633542872n0}Установите соответствие между графиками функций и формулами, которые их задают. Формулы: \\1) $1.5\sqrt{x}-5$\\2) $4.5\sqrt{x}-5$\\3) $-2\sqrt{x}+2$\\4) $-4.5\sqrt{x}+4$
\end{taskBN}

\begin{taskBN}{191}
\addpictocenter[scale=0.4]{images/064625319614892n0}На одном из рисунков изображен график функции $f(x)=4x^2-x-3$. Укажите номер этого рисунка.
\end{taskBN}

\begin{taskBN}{192}
\addpictocenter[scale=0.4]{images/25440947041224016n0}Установите соответствие между графиками функций и формулами, которые их задают. Формулы: \\1) $-3x^2-3x+4$\\2) $-2x^2+5x+3$\\3) $3x^2-8x-1$\\4) $2x^2-4x-5$
\end{taskBN}

\begin{taskBN}{193}
\addpictocenter[scale=0.4]{images/257543266443781n0}Установите соответствие между графиками функций и формулами, которые их задают. Формулы: \\1) $2x^2-9x+8$\\2) $-x^2-3x+7$\\3) $-4x^2-3x+2$\\4) $4x^2-4x-4$
\end{taskBN}

\begin{taskBN}{194}
\addpictocenter[scale=0.4]{images/502897110657184n0}Установите соответствие между графиками функций и формулами, которые их задают. Формулы: \\1) $\frac{-1}{x+3}+3$\\2) $\frac{-2}{x+2}+3$\\3) $\frac{3}{x-4}-2$\\4) $\frac{4}{x-3}+5$
\end{taskBN}

\begin{taskBN}{195}
\addpictocenter[scale=0.4]{images/556162751000245n0}Установите соответствие между графиками функций и формулами, которые их задают. Формулы: \\1) $1.5\sqrt{x}-5$\\2) $3.5\sqrt{x}-3$\\3) $3\sqrt{x}+1$\\4) $\sqrt{x}-5$
\end{taskBN}

\begin{taskBN}{196}
\addpictocenter[scale=0.4]{images/280201398381318n0}Установите соответствие между графиками функций и формулами, которые их задают. Формулы: \\1) $-2\sqrt{x}+5$\\2) $-3\sqrt{x}+4$\\3) $4.5\sqrt{x}-5$\\4) $-\sqrt{x}-3$
\end{taskBN}

\begin{taskBN}{197}
\addpictocenter[scale=0.4]{images/594585050619617n0}Установите соответствие между графиками функций и формулами, которые их задают. Формулы: \\1) $-\sqrt{x}-3$\\2) $-2\sqrt{x}+5$\\3) $5\sqrt{x}-2$\\4) $-1.5\sqrt{x}-1$
\end{taskBN}

\begin{taskBN}{198}
\addpictocenter[scale=0.4]{images/401376850367286n0}Установите соответствие между графиками функций и формулами, которые их задают. Формулы: \\1) $3\sqrt{x-7}-1$\\2) $-2x+5$\\3) $-4x-5$\\4) $-2x-8$
\end{taskBN}

\begin{taskBN}{199}
\addpictocenter[scale=0.4]{images/5961214821664846n0}Установите соответствие между графиками функций и формулами, которые их задают. Формулы: \\1) $-4x-10$\\2) $3.5x+9$\\3) $3x-4$\\4) $4.5x+5$
\end{taskBN}

\begin{taskBN}{200}
\addpictocenter[scale=0.4]{images/5043953975748394n0}На рисунке изображён график квадратичной функции $y=f(x)$. Какое из следующих утверждений о данной функции верно?\\1) Абсцисса вершины равна $0$\\2) $f(-4)=$ $8.5$\\3) Наименьшее значение функции равно  $2.5$\\4) Ордината вершины равна $2$\\5) Функция возрастает на промежутке $(-\infty;0]$
\end{taskBN}

\newpage
 
\begin{tabular}{*{4}l}
\begin{tabular}[t]{|l|l|}
\hline
1 & 423\\
\hline
2 & 321\\
\hline
3 & 2\\
\hline
4 & 324\\
\hline
5 & 1\\
\hline
6 & 421\\
\hline
7 & 4\\
\hline
8 & 231\\
\hline
9 & 321\\
\hline
10 & 241\\
\hline
11 & 421\\
\hline
12 & 342\\
\hline
13 & 312\\
\hline
14 & 142\\
\hline
15 & 312\\
\hline
16 & 132\\
\hline
17 & 412\\
\hline
18 & 312\\
\hline
19 & 143\\
\hline
20 & 5\\
\hline
21 & 2\\
\hline
22 & 143\\
\hline
23 & 314\\
\hline
24 & 312\\
\hline
25 & 1\\
\hline
26 & 2\\
\hline
27 & 3\\
\hline
28 & 132\\
\hline
29 & 341\\
\hline
30 & 423\\
\hline
31 & 4\\
\hline
32 & 4\\
\hline
33 & 321\\
\hline
34 & 5\\
\hline
35 & 124\\
\hline
36 & 341\\
\hline
37 & 123\\
\hline
38 & 142\\
\hline
39 & 342\\
\hline
40 & 231\\
\hline
41 & 231\\
\hline
42 & 412\\
\hline
43 & 321\\
\hline
44 & 312\\
\hline
45 & 3\\
\hline
46 & 142\\
\hline
47 & 213\\
\hline
48 & 1\\
\hline
49 & 321\\
\hline
50 & 432\\
\hline
\end{tabular}&\begin{tabular}[t]{|l|l|}
\hline
51 & 143\\
\hline
52 & 423\\
\hline
53 & 3\\
\hline
54 & 142\\
\hline
55 & 134\\
\hline
56 & 213\\
\hline
57 & 213\\
\hline
58 & 314\\
\hline
59 & 132\\
\hline
60 & 2\\
\hline
61 & 234\\
\hline
62 & 2\\
\hline
63 & 423\\
\hline
64 & 143\\
\hline
65 & 213\\
\hline
66 & 142\\
\hline
67 & 342\\
\hline
68 & 342\\
\hline
69 & 4\\
\hline
70 & 412\\
\hline
71 & 213\\
\hline
72 & 3\\
\hline
73 & 134\\
\hline
74 & 241\\
\hline
75 & 4\\
\hline
76 & 134\\
\hline
77 & 314\\
\hline
78 & 314\\
\hline
79 & 321\\
\hline
80 & 241\\
\hline
81 & 341\\
\hline
82 & 234\\
\hline
83 & 3\\
\hline
84 & 124\\
\hline
85 & 143\\
\hline
86 & 312\\
\hline
87 & 321\\
\hline
88 & 3\\
\hline
89 & 421\\
\hline
90 & 412\\
\hline
91 & 2\\
\hline
92 & 2\\
\hline
93 & 1\\
\hline
94 & 213\\
\hline
95 & 231\\
\hline
96 & 142\\
\hline
97 & 423\\
\hline
98 & 341\\
\hline
99 & 413\\
\hline
100 & 124\\
\hline
\end{tabular}&\begin{tabular}[t]{|l|l|}
\hline
101 & 432\\
\hline
102 & 431\\
\hline
103 & 341\\
\hline
104 & 123\\
\hline
105 & 231\\
\hline
106 & 341\\
\hline
107 & 214\\
\hline
108 & 321\\
\hline
109 & 143\\
\hline
110 & 134\\
\hline
111 & 124\\
\hline
112 & 132\\
\hline
113 & 1\\
\hline
114 & 132\\
\hline
115 & 142\\
\hline
116 & 214\\
\hline
117 & 413\\
\hline
118 & 412\\
\hline
119 & 213\\
\hline
120 & 134\\
\hline
121 & 143\\
\hline
122 & 1\\
\hline
123 & 234\\
\hline
124 & 421\\
\hline
125 & 1\\
\hline
126 & 3\\
\hline
127 & 423\\
\hline
128 & 142\\
\hline
129 & 213\\
\hline
130 & 321\\
\hline
131 & 143\\
\hline
132 & 124\\
\hline
133 & 234\\
\hline
134 & 214\\
\hline
135 & 243\\
\hline
136 & 142\\
\hline
137 & 314\\
\hline
138 & 231\\
\hline
139 & 421\\
\hline
140 & 421\\
\hline
141 & 2\\
\hline
142 & 234\\
\hline
143 & 412\\
\hline
144 & 234\\
\hline
145 & 143\\
\hline
146 & 431\\
\hline
147 & 324\\
\hline
148 & 122\\
\hline
149 & 324\\
\hline
150 & 214\\
\hline
\end{tabular}&\begin{tabular}[t]{|l|l|}
\hline
151 & 341\\
\hline
152 & 324\\
\hline
153 & 241\\
\hline
154 & 231\\
\hline
155 & 134\\
\hline
156 & 243\\
\hline
157 & 314\\
\hline
158 & 432\\
\hline
159 & 421\\
\hline
160 & 123\\
\hline
161 & 234\\
\hline
162 & 213\\
\hline
163 & 3\\
\hline
164 & 143\\
\hline
165 & 412\\
\hline
166 & 341\\
\hline
167 & 2\\
\hline
168 & 312\\
\hline
169 & 2\\
\hline
170 & 413\\
\hline
171 & 341\\
\hline
172 & 2\\
\hline
173 & 5\\
\hline
174 & 412\\
\hline
175 & 214\\
\hline
176 & 312\\
\hline
177 & 342\\
\hline
178 & 312\\
\hline
179 & 134\\
\hline
180 & 2\\
\hline
181 & 413\\
\hline
182 & 324\\
\hline
183 & 431\\
\hline
184 & 431\\
\hline
185 & 341\\
\hline
186 & 312\\
\hline
187 & 2\\
\hline
188 & 321\\
\hline
189 & 214\\
\hline
190 & 421\\
\hline
191 & 3\\
\hline
192 & 123\\
\hline
193 & 231\\
\hline
194 & 143\\
\hline
195 & 321\\
\hline
196 & 341\\
\hline
197 & 214\\
\hline
198 & 413\\
\hline
199 & 423\\
\hline
200 & 1\\
\hline
\end{tabular}\end{tabular}



\end{document}