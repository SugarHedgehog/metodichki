\documentclass{article}
\usepackage[utf8]{inputenc} % Кодировка UTF-8
\usepackage[T2A]{fontenc}   % Поддержка кириллицы
\usepackage[russian]{babel}  % Поддержка русского языка
\usepackage{array}           % Для настройки ширины столбцов
\usepackage{geometry}
\geometry{top=10mm, bottom=10mm, left=10mm, right=10mm}
\pagestyle{empty}

\begin{document}
\begin{center}
    \textbf{Дневник педагогической практики студентки 5 курса математического факультета Воронежского государственного университета, специальность 01.05.01. Фундаментальные математика и механика }\\
    \textbf{Суматохиной Александры Сергеевны в МБОУ СОШ №10 с 29.10.2024 по 11.11.2024}\\
    \end{center}

Классный руководитель: Федорова Елена Андреевна

Учитель математики: Федорова Елена Андреевна\\


\begin{tabular}{|p{1cm}|p{1.5cm}|p{5.5cm}|p{4.5cm}|p{3.5cm}|}
\hline
\textbf{№} & \textbf{Дата} & \textbf{Выполняемые виды работ} & \textbf{Ожидаемый результат} & \textbf{Примечание} \\ \hline
1 & 29.09.2024 & Начало практики: Введение в тему "Угол" & Ознакомление с темой & \\ \hline
2 & 30.09.2024 & Проведение занятия по теме "Угол. Измерение углов" & Обучение учащихся измерению углов & Упражнения с транспортиром \\ \hline
3 & 01.10.2024 & Практическое занятие: Измерение углов с помощью транспортиров & Учащиеся научатся применять транспортир & Проверка знаний по измерению углов \\ \hline
4 & 02.10.2024 & Посещение мероприятия, посвящённого Дню учителя & Ознакомление с работой педагогов & \\ \hline
5 & 03.10.2024 & Обсуждение прочитанного: Влияние учителей на жизнь ученика & Учащиеся поделятся своими мыслями & \\ \hline
6 & 04.10.2024 & Проведение занятия по теме "Биссектриса угла. Свойство биссектрисы угла" & Обучение учащихся свойствам биссектрисы &  \\ \hline
7 & 05.10.2024 & Практическое занятие: Построение биссектрисы угла & Учащиеся научатся строить биссектрису & Упражнения на построение \\ \hline
8 & 07.10.2024 & Подготовка проверочной работы по теме "Угол. Измерение углов" & Создание теста для проверки знаний & Разработка вопросов для теста \\ \hline
9 & 08.10.2024 & Проведение проверочной работы по теме "Угол. Измерение углов" & Оценка знаний учащихся & Анализ результатов теста \\ \hline
10 & 09.10.2024 & Проведение занятия по теме "Треугольник. Площадь треугольника" & Обучение учащихся вычислению площади треугольника & \\ \hline
11 & 10.10.2024 & Практическое занятие: Вычисление площади треугольника & Учащиеся научатся применять формулы & Упражнения на вычисление \\ \hline
12 & 11.10.2024 & Подготовка проверочной по теме "Биссектриса угла" & Создание теста для проверки знаний & Вопросы по свойствам биссектрисы \\ \hline
13 & 12.10.2024 & Проведение проверочной работы по теме "Биссектриса угла" & Оценка знаний учащихся & Обсуждение результатов \\ \hline
14 & 14.10.2024 & Проведение занятия "Перпендикулярность прямых. Расстояние от точки до прямой. Серединный перпендикуляр" & Обучение учащихся свойствам перпендикулярности & Примеры и задачи на практике \\ \hline
15 & 15.10.2024 & Практическое занятие: Построение перпендикуляров & Учащиеся научатся строить перпендикуляры & Проверка выполненных заданий \\ \hline
% Каникулы
16 & 05.11.2024 & Проведение итоговой проверочной работы & Оценка знаний учащихся по всем темам & \\ \hline
17 & 06.11.2024 & Помощь в мероприятии в библиотеке, посвящённом произведениям Бунина & Поддержка библиотекаря в организации мероприятия & \\ \hline
18 & 10.11.2024 & Завершение практики: Обсуждение полученного опыта & Подведение итогов практики & Обратная связь и рекомендации \\ \hline
\end{tabular}

\end{document}
