\documentclass[a4paper]{article}
\usepackage{dashbox}
\usepackage[T2A]{fontenc}
\usepackage[utf8]{inputenc}
\usepackage[english,russian]{babel}
\usepackage{graphicx}
\DeclareGraphicsExtensions{.pdf,.png,.jpg}

\linespread{1.15}

\usepackage{../egetask_ver}

\def\examyear{2023}
\usepackage[colorlinks,linkcolor=blue]{hyperref}\usepackage{draftwatermark}
\SetWatermarkLightness{0.9}
\SetWatermarkText{https://vk.com/egemathika}
\SetWatermarkScale{ 0.4 }
\def\lfoottext{Источник \href{https://vk.com/egemathika}{https://vk.com/egemathika}}

\begin{document}
\begin{taskBN}{1}
\addpictoright[0.25\textwidth]{images/50569731032854n0}Во сколько раз уменьшили площадь боковой поверхности конуса, если его длина окружности основания уменьшилась в 9 раз? При этом образующая не изменилась.
\end{taskBN}

\begin{taskBN}{2}
\addpictoright[0.25\textwidth]{images/521343110866226n0}В сосуде, имеющем форму конуса, уровень жидкости достигает $\frac{1}{4}$ высоты. Объём жидкости равен 64мл. Сколько миллилитров жидкости нужно долить, чтобы наполнить сосуд доверху?
\end{taskBN}

\begin{taskBN}{3}
\addpictoright[0.25\textwidth]{images/4431341014327124n0}Площадь осевого сечения конуса равна $240$. Плоскость, параллельная плоскости основания конуса,  делит его так, что объёмы конусов равны $240\pi$ и $1920\pi$. Найдите высоту меньшего конуса. 
\end{taskBN}

\begin{taskBN}{4}
\addpictoright[0.25\textwidth]{images/6081255309964166n0}Объём конуса равен $27648\pi$. Плоскость, параллельная плоскости основания конуса,  делит его так, что площади осевых сечений конусов равны $768$ и $1728$. Найдите высоту конуса, отсекаемого от данного конуса проведённой плоскостью. 
\end{taskBN}

\begin{taskBN}{5}
\addpictoright[0.25\textwidth]{images/851172186716015n0}Высота конуса равна $15$. Плоскость, параллельная плоскости основания конуса,  делит его так, что площади осевых сечений конусов равны $192$ и $300$. Найдите объём конуса, отсекаемого от данного конуса проведённой плоскостью. Ответ сократите на $\pi$.
\end{taskBN}

\begin{taskBN}{6}
\addpictoright[0.25\textwidth]{images/901492567607796n0}Образующая конуса равна $35$, высота равна $28$. Найдите объём конуса. Ответ сократите на $\pi$.
\end{taskBN}

\begin{taskBN}{7}
\addpictoright[0.25\textwidth]{images/939309287724563n0}В сосуде, имеющем форму конуса, уровень жидкости достигает $\frac{1}{8}$ высоты. Объём жидкости равен 5мл. Сколько миллилитров жидкости нужно долить, чтобы наполнить сосуд доверху?
\end{taskBN}

\begin{taskBN}{8}
\addpictoright[0.25\textwidth]{images/45943511053382n0}Площадь основания первого конуса в 9 раз меньше, чем площадь основания второго конуса. Во сколько раз радиус основания первого конуса меньше радиуса основания второго конуса?
\end{taskBN}

\begin{taskBN}{9}
\addpictoright[0.25\textwidth]{images/671943515680021n0}Высота конуса равна $9$. Плоскость, параллельная плоскости основания конуса,  делит его так, что объёмы конусов равны $128\pi$ и $432\pi$. Найдите радиус основания конуса, отсекаемого от данного конуса проведённой плоскостью. 
\end{taskBN}

\begin{taskBN}{10}
\addpictoright[0.25\textwidth]{images/73192580679647n0}Площадь осевого сечения конуса равна $1728$. Плоскость, параллельная плоскости основания конуса,  делит его так, что объёмы конусов равны $12\pi$ и $20736\pi$. Найдите длину окружности основания меньшего конуса. Ответ сократите на $\pi$.
\end{taskBN}

\begin{taskBN}{11}
\addpictoright[0.25\textwidth]{images/7463968371628737n0}В сосуде, имеющем форму конуса, уровень жидкости достигает $\frac{3}{7}$ высоты. Объём жидкости равен 81мл. Сколько миллилитров жидкости поместится в весь сосуд?
\end{taskBN}

\begin{taskBN}{12}
\addpictoright[0.25\textwidth]{images/596752905211983n0}Во сколько раз уменьшили площадь боковой поверхности конуса, если его площадь основания уменьшилась в 49 раз? При этом образующая не изменилась.
\end{taskBN}

\begin{taskBN}{13}
\addpictoright[0.25\textwidth]{images/314250217336682n0}В сосуде, имеющем форму конуса, уровень жидкости достигает $\frac{7}{8}$ высоты. Объём жидкости равен 2401мл. Сколько миллилитров жидкости нужно долить, чтобы наполнить сосуд доверху?
\end{taskBN}

\begin{taskBN}{14}
\addpictoright[0.25\textwidth]{images/102830736880413n0}В сосуде, имеющем форму конуса, уровень жидкости достигает $\frac{1}{2}$ высоты. Объём жидкости равен 512мл. Сколько миллилитров жидкости поместится в весь сосуд?
\end{taskBN}

\begin{taskBN}{15}
\addpictoright[0.25\textwidth]{images/9293934063685774n0}Высота конуса равна $24$, объём равен $2592\pi$. Найдите площадь боковой поверхности конуса. Ответ сократите на $\pi$.
\end{taskBN}

\begin{taskBN}{16}
\addpictoright[0.25\textwidth]{images/08219894823480067n0}В сосуде, имеющем форму конуса, уровень жидкости достигает $\frac{2}{7}$ высоты. Объём жидкости равен 16мл. Сколько миллилитров жидкости поместится в весь сосуд?
\end{taskBN}

\begin{taskBN}{17}
\addpictoright[0.25\textwidth]{images/011586950888102n0}Объём первого конуса в 8 раз меньше, чем объём второго конуса. Во сколько раз высота первого конуса меньше высоты второго конуса?
\end{taskBN}

\begin{taskBN}{18}
\addpictoright[0.25\textwidth]{images/743990720403334n0}Объём конуса равен $12000\pi$, высота равна $40$. Найдите площадь боковой поверхности конуса. Ответ сократите на $\pi$.
\end{taskBN}

\begin{taskBN}{19}
\addpictoright[0.25\textwidth]{images/1415072872510095n0}В сосуде, имеющем форму конуса, уровень жидкости достигает $\frac{1}{4}$ высоты. Объём жидкости равен 16мл. Сколько миллилитров жидкости поместится в весь сосуд?
\end{taskBN}

\begin{taskBN}{20}
\addpictoright[0.25\textwidth]{images/8274345646515202n0}В сосуде, имеющем форму конуса, уровень жидкости достигает $\frac{1}{8}$ высоты. Объём жидкости равен 8мл. Сколько миллилитров жидкости поместится в весь сосуд?
\end{taskBN}

\begin{taskBN}{21}
\addpictoright[0.25\textwidth]{images/837358634828798n0}В сосуде, имеющем форму конуса, уровень жидкости достигает $\frac{1}{2}$ высоты. Объём жидкости равен 108мл. Сколько миллилитров жидкости нужно долить, чтобы наполнить сосуд доверху?
\end{taskBN}

\begin{taskBN}{22}
\addpictoright[0.25\textwidth]{images/559825431429979n0}Объём конуса равен $5488\pi$. Плоскость, параллельная плоскости основания конуса,  проходит так, что высота делится на отрезки $15$ и $6$ считая от вершины. Найдите площадь осевого сечения меньшего конуса. 
\end{taskBN}

\begin{taskBN}{23}
\addpictoright[0.25\textwidth]{images/211848045824893n0}Объём конуса равен $768\pi$. Плоскость, параллельная плоскости основания конуса,  делит его так, что площади осевых сечений конусов равны $12$ и $192$. Найдите высоту конуса, отсекаемого от данного конуса проведённой плоскостью. 
\end{taskBN}

\begin{taskBN}{24}
\addpictoright[0.25\textwidth]{images/898622647114941n0}В сосуде, имеющем форму конуса, уровень жидкости достигает $\frac{2}{7}$ высоты. Объём жидкости равен 24мл. Сколько миллилитров жидкости поместится в весь сосуд?
\end{taskBN}

\begin{taskBN}{25}
\addpictoright[0.25\textwidth]{images/9354480107070229n0}В сосуде, имеющем форму конуса, уровень жидкости достигает $\frac{1}{3}$ высоты. Объём жидкости равен 4мл. Сколько миллилитров жидкости поместится в весь сосуд?
\end{taskBN}

\begin{taskBN}{26}
\addpictoright[0.25\textwidth]{images/653646727272481n0}Во сколько раз увеличили площадь основания конуса, если его радиус основания увеличился в 3 раза?
\end{taskBN}

\begin{taskBN}{27}
\addpictoright[0.25\textwidth]{images/635878586977488n0}В сосуде, имеющем форму конуса, уровень жидкости достигает $\frac{2}{9}$ высоты. Объём жидкости равен 64мл. Сколько миллилитров жидкости нужно долить, чтобы наполнить сосуд доверху?
\end{taskBN}

\begin{taskBN}{28}
\addpictoright[0.25\textwidth]{images/255740325312934n0}В сосуде, имеющем форму конуса, уровень жидкости достигает $\frac{4}{9}$ высоты. Объём жидкости равен 640мл. Сколько миллилитров жидкости поместится в весь сосуд?
\end{taskBN}

\begin{taskBN}{29}
\addpictoright[0.25\textwidth]{images/968038754943905n0}В сосуде, имеющем форму конуса, уровень жидкости достигает $\frac{1}{3}$ высоты. Объём жидкости равен 8мл. Сколько миллилитров жидкости нужно долить, чтобы наполнить сосуд доверху?
\end{taskBN}

\begin{taskBN}{30}
\addpictoright[0.25\textwidth]{images/760019930068976n0}Во сколько раз уменьшили объём конуса, если его площадь основания уменьшилась в 7 раз? При этом радиус основания не изменился.
\end{taskBN}

\begin{taskBN}{31}
\addpictoright[0.25\textwidth]{images/228701160462247n0}Во сколько раз уменьшили площадь боковой поверхности конуса, если его длина окружности основания уменьшилась в 6 раз? При этом образующая не изменилась.
\end{taskBN}

\begin{taskBN}{32}
\addpictoright[0.25\textwidth]{images/530208502940084n0}В сосуде, имеющем форму конуса, уровень жидкости достигает $\frac{5}{7}$ высоты. Объём жидкости равен 750мл. Сколько миллилитров жидкости нужно долить, чтобы наполнить сосуд доверху?
\end{taskBN}

\begin{taskBN}{33}
\addpictoright[0.25\textwidth]{images/246369035689775n0}Площадь боковой поверхности конуса равна $720\pi$, высота равна $18$. Найдите длину окружности основания конуса. Ответ сократите на $\pi$.
\end{taskBN}

\begin{taskBN}{34}
\addpictoright[0.25\textwidth]{images/59177258664143n0}Во сколько раз уменьшили объём конуса, если его длина окружности основания уменьшилась в 3 раза? При этом высота не изменилась.
\end{taskBN}

\begin{taskBN}{35}
\addpictoright[0.25\textwidth]{images/2878667545822111n0}Образующая конуса равна $5$, высота равна $4$. Найдите объём конуса. Ответ сократите на $\pi$.
\end{taskBN}

\begin{taskBN}{36}
\addpictoright[0.25\textwidth]{images/677020713183231n0}Образующая первого конуса в 4 раза меньше, чем образующая второго конуса. Во сколько раз площадь боковой поверхности первого конуса меньше площади боковой поверхности второго конуса? При этом у обоих конусов длины окружностей основания равны.
\end{taskBN}

\begin{taskBN}{37}
\addpictoright[0.25\textwidth]{images/001700941163374n0}Радиус основания первого конуса в 8 раз больше, чем радиус основания второго конуса. Во сколько раз площадь основания первого конуса больше площади основания второго конуса?
\end{taskBN}

\begin{taskBN}{38}
\addpictoright[0.25\textwidth]{images/0378064516958434n0}Площадь осевого сечения конуса равна $108$. Плоскость, параллельная плоскости основания конуса,  делит его так, что радиусы оснований конусов равны $4$ и $12$. Найдите объём меньшего конуса. Ответ сократите на $\pi$.
\end{taskBN}

\begin{taskBN}{39}
\addpictoright[0.25\textwidth]{images/323401437688666n0}Объём конуса равен $600\pi$, высота равна $8$. Найдите площадь боковой поверхности конуса. Ответ сократите на $\pi$.
\end{taskBN}

\begin{taskBN}{40}
\addpictoright[0.25\textwidth]{images/84968456961319n0}Во сколько раз уменьшили объём конуса, если его радиус основания уменьшился в 3 раза? При этом высота не изменилась.
\end{taskBN}

\begin{taskBN}{41}
\addpictoright[0.25\textwidth]{images/1842465852278425n0}Во сколько раз уменьшили длину окружности основания конуса, если его объём уменьшился в 100 раз? При этом высота не изменилась.
\end{taskBN}

\begin{taskBN}{42}
\addpictoright[0.25\textwidth]{images/158770130318266n0}Объём первого конуса в 25 раз меньше, чем объём второго конуса. Во сколько раз длина окружности основания первого конуса меньше длины окружности основания второго конуса? При этом у обоих конусов высоты равны.
\end{taskBN}

\begin{taskBN}{43}
\addpictoright[0.25\textwidth]{images/163139323346752n0}В сосуде, имеющем форму конуса, уровень жидкости достигает $\frac{1}{3}$ высоты. Объём жидкости равен 5мл. Сколько миллилитров жидкости нужно долить, чтобы наполнить сосуд доверху?
\end{taskBN}

\begin{taskBN}{44}
\addpictoright[0.25\textwidth]{images/1660609163357383n0}Образующая конуса равна $5$, площадь боковой поверхности равна $15\pi$. Найдите объём конуса. Ответ сократите на $\pi$.
\end{taskBN}

\begin{taskBN}{45}
\addpictoright[0.25\textwidth]{images/853666434625692n0}В сосуде, имеющем форму конуса, уровень жидкости достигает $\frac{1}{2}$ высоты. Объём жидкости равен 32мл. Сколько миллилитров жидкости поместится в весь сосуд?
\end{taskBN}

\begin{taskBN}{46}
\addpictoright[0.25\textwidth]{images/11517472908745n0}Длина окружности основания конуса равна $28\pi$, площадь боковой поверхности равна $700\pi$. Найдите образующую конуса. 
\end{taskBN}

\begin{taskBN}{47}
\addpictoright[0.25\textwidth]{images/4841232579522579n0}В сосуде, имеющем форму конуса, уровень жидкости достигает $\frac{2}{3}$ высоты. Объём жидкости равен 192мл. Сколько миллилитров жидкости поместится в весь сосуд?
\end{taskBN}

\begin{taskBN}{48}
\addpictoright[0.25\textwidth]{images/894341635725352n0}Во сколько раз уменьшили радиус основания конуса, если его площадь боковой поверхности уменьшилась в 3 раза? При этом образующая не изменилась.
\end{taskBN}

\begin{taskBN}{49}
\addpictoright[0.25\textwidth]{images/18326739298989203n0}Во сколько раз уменьшили высоту конуса, если его объём уменьшился в 6 раз? При этом радиус основания не изменился.
\end{taskBN}

\begin{taskBN}{50}
\addpictoright[0.25\textwidth]{images/0273846867167007n0}Площадь осевого сечения конуса равна $108$. Плоскость, параллельная плоскости основания конуса,  проходит так, что высота делится на отрезки $6$ и $3$ считая от основания. Найдите объём меньшего конуса. Ответ сократите на $\pi$.
\end{taskBN}

\begin{taskBN}{51}
\addpictoright[0.25\textwidth]{images/647214853345719n0}Образующая конуса равна $45$, площадь боковой поверхности равна $1620\pi$. Найдите длину окружности основания конуса. Ответ сократите на $\pi$.
\end{taskBN}

\begin{taskBN}{52}
\addpictoright[0.25\textwidth]{images/037800811721979n0}В сосуде, имеющем форму конуса, уровень жидкости достигает $\frac{3}{8}$ высоты. Объём жидкости равен 216мл. Сколько миллилитров жидкости нужно долить, чтобы наполнить сосуд доверху?
\end{taskBN}

\begin{taskBN}{53}
\addpictoright[0.25\textwidth]{images/989778175333685n0}Объём первого конуса в 9 раз меньше, чем объём второго конуса. Во сколько раз длина окружности основания первого конуса меньше длины окружности основания второго конуса? При этом у обоих конусов высоты равны.
\end{taskBN}

\begin{taskBN}{54}
\addpictoright[0.25\textwidth]{images/404623483067396n0}Площадь боковой поверхности конуса равна $444\pi$, высота равна $35$. Найдите образующую конуса. 
\end{taskBN}

\begin{taskBN}{55}
\addpictoright[0.25\textwidth]{images/739312054412178n0}Образующая конуса равна $34$, площадь боковой поверхности равна $1020\pi$. Найдите длину окружности основания конуса. Ответ сократите на $\pi$.
\end{taskBN}

\begin{taskBN}{56}
\addpictoright[0.25\textwidth]{images/17628021403606n0}Во сколько раз увеличили высоту конуса, если его объём увеличился в 8 раз? При этом радиус основания не изменился.
\end{taskBN}

\begin{taskBN}{57}
\addpictoright[0.25\textwidth]{images/987899884423704n0}Высота конуса равна $33$, площадь боковой поверхности равна $2420\pi$. Найдите площадь окружности основания конуса. Ответ сократите на $\pi$.
\end{taskBN}

\begin{taskBN}{58}
\addpictoright[0.25\textwidth]{images/5920647980436522n0}Площадь основания первого конуса в 100 раз больше, чем площадь основания второго конуса. Во сколько раз радиус основания первого конуса больше радиуса основания второго конуса?
\end{taskBN}

\begin{taskBN}{59}
\addpictoright[0.25\textwidth]{images/341235988107236n0}Объём конуса равен $3136\pi$, высота равна $48$. Найдите образующую конуса. 
\end{taskBN}

\begin{taskBN}{60}
\addpictoright[0.25\textwidth]{images/8482577934005542n0}Площадь боковой поверхности конуса равна $1224\pi$, высота равна $45$. Найдите объём конуса. Ответ сократите на $\pi$.
\end{taskBN}

\begin{taskBN}{61}
\addpictoright[0.25\textwidth]{images/7475389871702516n0}Площадь осевого сечения конуса равна $588$. Плоскость, параллельная плоскости основания конуса,  делит его так, что диаметры оснований конусов равны $32$ и $56$. Найдите высоту меньшего конуса. 
\end{taskBN}

\begin{taskBN}{62}
\addpictoright[0.25\textwidth]{images/3522508301807783n0}Высота конуса равна $21$. Плоскость, параллельная плоскости основания конуса,  делит его так, что площади осевых сечений конусов равны $300$ и $588$. Найдите объём меньшего конуса. Ответ сократите на $\pi$.
\end{taskBN}

\begin{taskBN}{63}
\addpictoright[0.25\textwidth]{images/8711630868538347n0}Высота конуса равна $24$. Плоскость, параллельная плоскости основания конуса,  делит его так, что диаметры оснований конусов равны $48$ и $64$. Найдите объём конуса, отсекаемого от данного конуса проведённой плоскостью. Ответ сократите на $\pi$.
\end{taskBN}

\begin{taskBN}{64}
\addpictoright[0.25\textwidth]{images/93757540397605n0}Объём конуса равен $432\pi$. Плоскость, параллельная плоскости основания конуса,  делит его так, что площади осевых сечений конусов равны $12$ и $108$. Найдите высоту меньшего конуса. 
\end{taskBN}

\begin{taskBN}{65}
\addpictoright[0.25\textwidth]{images/756207702658763n0}Высота конуса равна $10$. Плоскость, параллельная плоскости основания конуса,  делит его так, что объёмы конусов равны $240\pi$ и $1920\pi$. Найдите длину окружности основания конуса, отсекаемого от данного конуса проведённой плоскостью. Ответ сократите на $\pi$.
\end{taskBN}

\begin{taskBN}{66}
\addpictoright[0.25\textwidth]{images/753455500955952n0}Высота конуса равна $32$. Плоскость, параллельная плоскости основания конуса,  делит его так, что объёмы конусов равны $324\pi$ и $6144\pi$. Найдите радиус основания меньшего конуса. 
\end{taskBN}

\begin{taskBN}{67}
\addpictoright[0.25\textwidth]{images/833261497279721n0}Объём конуса равен $6400\pi$, длина окружности основания равна $40\pi$. Найдите образующую конуса. 
\end{taskBN}

\begin{taskBN}{68}
\addpictoright[0.25\textwidth]{images/984379677198138n0}Высота конуса равна $20$, длина окружности основания равна $96\pi$. Найдите объём конуса. Ответ сократите на $\pi$.
\end{taskBN}

\begin{taskBN}{69}
\addpictoright[0.25\textwidth]{images/873501144107341n0}Во сколько раз увеличили площадь боковой поверхности конуса, если его образующая увеличилась в 3 раза? При этом длина окружности основания не изменилась.
\end{taskBN}

\begin{taskBN}{70}
\addpictoright[0.25\textwidth]{images/25370068669809398n0}Объём конуса равен $600\pi$, радиус основания равен $15$. Найдите образующую конуса. 
\end{taskBN}

\begin{taskBN}{71}
\addpictoright[0.25\textwidth]{images/770041356897636n0}В сосуде, имеющем форму конуса, уровень жидкости достигает $\frac{3}{10}$ высоты. Объём жидкости равен 108мл. Сколько миллилитров жидкости поместится в весь сосуд?
\end{taskBN}

\begin{taskBN}{72}
\addpictoright[0.25\textwidth]{images/5371790605090654n0}Радиус основания конуса равен $18$. Плоскость, параллельная плоскости основания конуса,  проходит так, что высота делится на отрезки $4$ и $20$ считая от вершины. Найдите объём конуса, отсекаемого от данного конуса проведённой плоскостью. Ответ сократите на $\pi$.
\end{taskBN}

\begin{taskBN}{73}
\addpictoright[0.25\textwidth]{images/235051575337236n0}Образующая конуса равна $60$, радиус основания равен $48$. Найдите площадь боковой поверхности конуса. Ответ сократите на $\pi$.
\end{taskBN}

\begin{taskBN}{74}
\addpictoright[0.25\textwidth]{images/826031096107864n0}Высота конуса равна $9$, площадь боковой поверхности равна $180\pi$. Найдите длину окружности основания конуса. Ответ сократите на $\pi$.
\end{taskBN}

\begin{taskBN}{75}
\addpictoright[0.25\textwidth]{images/1574260535645793n0}Во сколько раз увеличили площадь основания конуса, если его радиус основания увеличился в 4 раза?
\end{taskBN}

\begin{taskBN}{76}
\addpictoright[0.25\textwidth]{images/6486540471565216n0}Во сколько раз увеличили площадь боковой поверхности конуса, если его площадь основания увеличилась в 16 раз? При этом образующая не изменилась.
\end{taskBN}

\begin{taskBN}{77}
\addpictoright[0.25\textwidth]{images/5779515062055487n0}Длина окружности основания конуса равна $16\pi$, высота равна $15$. Найдите объём конуса. Ответ сократите на $\pi$.
\end{taskBN}

\begin{taskBN}{78}
\addpictoright[0.25\textwidth]{images/036960334970529n0}Образующая конуса равна $29$, радиус основания равен $21$. Найдите площадь боковой поверхности конуса. Ответ сократите на $\pi$.
\end{taskBN}

\begin{taskBN}{79}
\addpictoright[0.25\textwidth]{images/081942486408542n0}Высота конуса равна $40$, образующая равна $41$. Найдите объём конуса. Ответ сократите на $\pi$.
\end{taskBN}

\begin{taskBN}{80}
\addpictoright[0.25\textwidth]{images/9065086417938346n0}Площадь боковой поверхности конуса равна $15\pi$, объём равен $12\pi$. Найдите образующую конуса. 
\end{taskBN}

\begin{taskBN}{81}
\addpictoright[0.25\textwidth]{images/857997654103505n0}В сосуде, имеющем форму конуса, уровень жидкости достигает $\frac{3}{5}$ высоты. Объём жидкости равен 189мл. Сколько миллилитров жидкости нужно долить, чтобы наполнить сосуд доверху?
\end{taskBN}

\begin{taskBN}{82}
\addpictoright[0.25\textwidth]{images/269563739927773n0}В сосуде, имеющем форму конуса, уровень жидкости достигает $\frac{1}{3}$ высоты. Объём жидкости равен 24мл. Сколько миллилитров жидкости нужно долить, чтобы наполнить сосуд доверху?
\end{taskBN}

\begin{taskBN}{83}
\addpictoright[0.25\textwidth]{images/314079080650902n0}Объём конуса равен $2000\pi$, высота равна $15$. Найдите образующую конуса. 
\end{taskBN}

\begin{taskBN}{84}
\addpictoright[0.25\textwidth]{images/7969564305063155n0}В сосуде, имеющем форму конуса, уровень жидкости достигает $\frac{1}{2}$ высоты. Объём жидкости равен 24мл. Сколько миллилитров жидкости нужно долить, чтобы наполнить сосуд доверху?
\end{taskBN}

\begin{taskBN}{85}
\addpictoright[0.25\textwidth]{images/958071686627721n0}В сосуде, имеющем форму конуса, уровень жидкости достигает $\frac{1}{6}$ высоты. Объём жидкости равен 5мл. Сколько миллилитров жидкости поместится в весь сосуд?
\end{taskBN}

\begin{taskBN}{86}
\addpictoright[0.25\textwidth]{images/586917716314361n0}Длина окружности основания конуса равна $48\pi$. Плоскость, параллельная плоскости основания конуса,  делит его так, что объёмы конусов равны $432\pi$ и $3456\pi$. Найдите высоту конуса, отсекаемого от данного конуса проведённой плоскостью. 
\end{taskBN}

\begin{taskBN}{87}
\addpictoright[0.25\textwidth]{images/350351880842054n0}Объём конуса равен $16200\pi$. Плоскость, параллельная плоскости основания конуса,  делит его так, что диаметры оснований конусов равны $30$ и $90$. Найдите площадь осевого сечения меньшего конуса. 
\end{taskBN}

\begin{taskBN}{88}
\addpictoright[0.25\textwidth]{images/4175203102183798n0}Площадь окружности основания конуса равна $1936\pi$. Плоскость, параллельная плоскости основания конуса,  проходит так, что высота делится на отрезки $21$ и $12$ считая от основания. Найдите площадь осевого сечения конуса, отсекаемого от данного конуса проведённой плоскостью. 
\end{taskBN}

\begin{taskBN}{89}
\addpictoright[0.25\textwidth]{images/070378980164876n0}Во сколько раз уменьшили площадь боковой поверхности конуса, если его длина окружности основания уменьшилась в 7 раз? При этом образующая не изменилась.
\end{taskBN}

\begin{taskBN}{90}
\addpictoright[0.25\textwidth]{images/614388005549899n0}Объём конуса равен $21296\pi$. Плоскость, параллельная плоскости основания конуса,  делит его так, что длины окружностей оснований конусов равны $80\pi$ и $88\pi$. Найдите площадь осевого сечения конуса, отсекаемого от данного конуса проведённой плоскостью. 
\end{taskBN}

\begin{taskBN}{91}
\addpictoright[0.25\textwidth]{images/755077527972349n0}В сосуде, имеющем форму конуса, уровень жидкости достигает $\frac{1}{2}$ высоты. Объём жидкости равен 48мл. Сколько миллилитров жидкости поместится в весь сосуд?
\end{taskBN}

\begin{taskBN}{92}
\addpictoright[0.25\textwidth]{images/042975895658975n0}Во сколько раз увеличили высоту конуса, если его объём увеличился в 5 раз? При этом радиус основания не изменился.
\end{taskBN}

\begin{taskBN}{93}
\addpictoright[0.25\textwidth]{images/787205126180708n0}Объём конуса равен $15972\pi$. Плоскость, параллельная плоскости основания конуса,  делит его так, что площади осевых сечений конусов равны $432$ и $1452$. Найдите радиус основания меньшего конуса. 
\end{taskBN}

\begin{taskBN}{94}
\addpictoright[0.25\textwidth]{images/827419709208677n0}В сосуде, имеющем форму конуса, уровень жидкости достигает $\frac{7}{8}$ высоты. Объём жидкости равен 1029мл. Сколько миллилитров жидкости нужно долить, чтобы наполнить сосуд доверху?
\end{taskBN}

\begin{taskBN}{95}
\addpictoright[0.25\textwidth]{images/4857400702654113n0}Площадь основания первого конуса в 5 раз меньше, чем площадь основания второго конуса. Во сколько раз объём первого конуса меньше объёма второго конуса?
\end{taskBN}

\begin{taskBN}{96}
\addpictoright[0.25\textwidth]{images/934128206545082n0}Образующая первого конуса в 8 раз меньше, чем образующая второго конуса. Во сколько раз площадь боковой поверхности первого конуса меньше площади боковой поверхности второго конуса? При этом у обоих конусов площади оснований равны.
\end{taskBN}

\begin{taskBN}{97}
\addpictoright[0.25\textwidth]{images/4355602430959953n0}Объём конуса равен $16000\pi$, образующая равна $50$. Найдите площадь боковой поверхности конуса. Ответ сократите на $\pi$.
\end{taskBN}

\begin{taskBN}{98}
\addpictoright[0.25\textwidth]{images/313993119200018n0}Длина окружности основания первого конуса в 4 раза больше, чем длина окружности основания второго конуса. Во сколько раз площадь боковой поверхности первого конуса больше площади боковой поверхности второго конуса? При этом у обоих конусов образующие равны.
\end{taskBN}

\begin{taskBN}{99}
\addpictoright[0.25\textwidth]{images/626236619365577n0}Во сколько раз уменьшили площадь боковой поверхности конуса, если его образующая уменьшилась в 8 раз? При этом длина окружности основания не изменилась.
\end{taskBN}

\begin{taskBN}{100}
\addpictoright[0.25\textwidth]{images/922780560757544n0}Объём конуса равен $23520\pi$, высота равна $40$. Найдите образующую конуса. 
\end{taskBN}

\begin{taskBN}{101}
\addpictoright[0.25\textwidth]{images/4247178553574253n0}Площадь основания конуса в два раза больше площади боковой поверхности. Найдите угол между образующей конуса и плоскостью основания. Ответ дайте в градусах.
\end{taskBN}

\begin{taskBN}{102}
\addpictoright[0.25\textwidth]{images/9276910163178205n0}Площадь основания первого конуса в 81 раз меньше, чем площадь основания второго конуса. Во сколько раз радиус основания первого конуса меньше радиуса основания второго конуса?
\end{taskBN}

\begin{taskBN}{103}
\addpictoright[0.25\textwidth]{images/451438229755469n0}Площадь боковой поверхности конуса равна $375\pi$, объём равен $1500\pi$. Найдите длину окружности основания конуса. Ответ сократите на $\pi$.
\end{taskBN}

\begin{taskBN}{104}
\addpictoright[0.25\textwidth]{images/180191938570782n0}Во сколько раз увеличили площадь боковой поверхности конуса, если его длина окружности основания увеличилась в 4 раза? При этом образующая не изменилась.
\end{taskBN}

\begin{taskBN}{105}
\addpictoright[0.25\textwidth]{images/7259574664714494n0}Площадь основания конуса в два раза больше площади боковой поверхности. Найдите угол между образующей конуса и плоскостью основания. Ответ дайте в градусах.
\end{taskBN}

\begin{taskBN}{106}
\addpictoright[0.25\textwidth]{images/8784125863829626n0}Площадь боковой поверхности конуса в два раза больше площади основания. Найдите угол между образующей конуса и плоскостью основания. Ответ дайте в градусах.
\end{taskBN}

\begin{taskBN}{107}
\addpictoright[0.25\textwidth]{images/072365325531507n0}Радиус основания первого конуса в 5 раз больше, чем радиус основания второго конуса. Во сколько раз объём первого конуса больше объёма второго конуса? При этом у обоих конусов высоты равны.
\end{taskBN}

\begin{taskBN}{108}
\addpictoright[0.25\textwidth]{images/979985295314287n0}Во сколько раз уменьшили площадь боковой поверхности конуса, если его длина окружности основания уменьшилась в 8 раз? При этом образующая не изменилась.
\end{taskBN}

\begin{taskBN}{109}
\addpictoright[0.25\textwidth]{images/351520371361473n0}Площадь боковой поверхности конуса в два раза больше площади основания. Найдите угол между образующей конуса и плоскостью основания. Ответ дайте в градусах.
\end{taskBN}

\begin{taskBN}{110}
\addpictoright[0.25\textwidth]{images/326856874204851n0}В сосуде, имеющем форму конуса, уровень жидкости достигает $\frac{5}{7}$ высоты. Объём жидкости равен 250мл. Сколько миллилитров жидкости нужно долить, чтобы наполнить сосуд доверху?
\end{taskBN}

\begin{taskBN}{111}
\addpictoright[0.25\textwidth]{images/679416562030423n0}Длина окружности основания конуса равна $88\pi$. Плоскость, параллельная плоскости основания конуса,  проходит так, что высота делится на отрезки $12$ и $21$ считая от основания. Найдите объём конуса, отсекаемого от данного конуса проведённой плоскостью. Ответ сократите на $\pi$.
\end{taskBN}

\begin{taskBN}{112}
\addpictoright[0.25\textwidth]{images/0736997697376571n0}Площадь осевого сечения конуса равна $300$. Плоскость, параллельная плоскости основания конуса,  проходит так, что высота делится на отрезки $6$ и $9$ считая от вершины. Найдите объём меньшего конуса. Ответ сократите на $\pi$.
\end{taskBN}

\begin{taskBN}{113}
\addpictoright[0.25\textwidth]{images/17509652717859n0}Площадь основания конуса в два раза больше площади боковой поверхности. Найдите угол между образующей конуса и плоскостью основания. Ответ дайте в градусах.
\end{taskBN}

\begin{taskBN}{114}
\addpictoright[0.25\textwidth]{images/532715930506597n0}Во сколько раз уменьшили длину окружности основания конуса, если его объём уменьшился в 49 раз? При этом высота не изменилась.
\end{taskBN}

\begin{taskBN}{115}
\addpictoright[0.25\textwidth]{images/038006238147749n0}В сосуде, имеющем форму конуса, уровень жидкости достигает $\frac{2}{3}$ высоты. Объём жидкости равен 864мл. Сколько миллилитров жидкости нужно долить, чтобы наполнить сосуд доверху?
\end{taskBN}

\begin{taskBN}{116}
\addpictoright[0.25\textwidth]{images/792784341058225n0}Площадь основания первого конуса в 9 раз меньше, чем площадь основания второго конуса. Во сколько раз площадь боковой поверхности первого конуса меньше площади боковой поверхности второго конуса? При этом у обоих конусов образующие равны.
\end{taskBN}

\begin{taskBN}{117}
\addpictoright[0.25\textwidth]{images/659625511946441n0}Объём конуса равен $432\pi$. Плоскость, параллельная плоскости основания конуса,  делит его так, что диаметры оснований конусов равны $16$ и $24$. Найдите площадь осевого сечения конуса, отсекаемого от данного конуса проведённой плоскостью. 
\end{taskBN}

\begin{taskBN}{118}
\addpictoright[0.25\textwidth]{images/021522924900795n0}Объём конуса равен $11664\pi$. Плоскость, параллельная плоскости основания конуса,  делит его так, что длины окружностей оснований конусов равны $56\pi$ и $72\pi$. Найдите площадь осевого сечения конуса, отсекаемого от данного конуса проведённой плоскостью. 
\end{taskBN}

\begin{taskBN}{119}
\addpictoright[0.25\textwidth]{images/285654255007572n0}В сосуде, имеющем форму конуса, уровень жидкости достигает $\frac{1}{9}$ высоты. Объём жидкости равен 9мл. Сколько миллилитров жидкости нужно долить, чтобы наполнить сосуд доверху?
\end{taskBN}

\begin{taskBN}{120}
\addpictoright[0.25\textwidth]{images/866269176808091n0}Во сколько раз уменьшили площадь основания конуса, если его длина окружности основания уменьшилась в 8 раз?
\end{taskBN}

\begin{taskBN}{121}
\addpictoright[0.25\textwidth]{images/468409667965104n0}Площадь боковой поверхности первого конуса в 7 раз меньше, чем площадь боковой поверхности второго конуса. Во сколько раз площадь основания первого конуса меньше площади основания второго конуса? При этом у обоих конусов образующие равны.
\end{taskBN}

\begin{taskBN}{122}
\addpictoright[0.25\textwidth]{images/55932994152874n0}Площадь осевого сечения конуса равна $300$. Плоскость, параллельная плоскости основания конуса,  делит его так, что объёмы конусов равны $96\pi$ и $1500\pi$. Найдите высоту меньшего конуса. 
\end{taskBN}

\begin{taskBN}{123}
\addpictoright[0.25\textwidth]{images/3239722179880884n0}Во сколько раз увеличили площадь боковой поверхности конуса, если его площадь основания увеличилась в 4 раза? При этом образующая не изменилась.
\end{taskBN}

\begin{taskBN}{124}
\addpictoright[0.25\textwidth]{images/515835252757572n0}Объём конуса равен $4800\pi$, длина окружности основания равна $60\pi$. Найдите образующую конуса. 
\end{taskBN}

\begin{taskBN}{125}
\addpictoright[0.25\textwidth]{images/746446213556804n0}Площадь боковой поверхности конуса в два раза больше площади основания. Найдите угол между образующей конуса и плоскостью основания. Ответ дайте в градусах.
\end{taskBN}

\begin{taskBN}{126}
\addpictoright[0.25\textwidth]{images/9268724571488063n0}Во сколько раз увеличили площадь боковой поверхности конуса, если его длина окружности основания увеличилась в 10 раз? При этом образующая не изменилась.
\end{taskBN}

\begin{taskBN}{127}
\addpictoright[0.25\textwidth]{images/605182294072063n0}Во сколько раз увеличили высоту конуса, если его объём увеличился в 8 раз? При этом радиус основания не изменился.
\end{taskBN}

\begin{taskBN}{128}
\addpictoright[0.25\textwidth]{images/304321058184838n0}Площадь боковой поверхности первого конуса в 5 раз меньше, чем площадь боковой поверхности второго конуса. Во сколько раз радиус основания первого конуса меньше радиуса основания второго конуса? При этом у обоих конусов образующие равны.
\end{taskBN}

\begin{taskBN}{129}
\addpictoright[0.25\textwidth]{images/7794229148285405n0}Высота конуса равна $21$, длина окружности основания равна $40\pi$. Найдите образующую конуса. 
\end{taskBN}

\begin{taskBN}{130}
\addpictoright[0.25\textwidth]{images/540015428100082n0}Объём первого конуса в 25 раз меньше, чем объём второго конуса. Во сколько раз длина окружности основания первого конуса меньше длины окружности основания второго конуса? При этом у обоих конусов высоты равны.
\end{taskBN}

\begin{taskBN}{131}
\addpictoright[0.25\textwidth]{images/018245062013726n0}В сосуде, имеющем форму конуса, уровень жидкости достигает $\frac{2}{7}$ высоты. Объём жидкости равен 32мл. Сколько миллилитров жидкости поместится в весь сосуд?
\end{taskBN}

\begin{taskBN}{132}
\addpictoright[0.25\textwidth]{images/592712330288834n0}Высота конуса равна $21$. Плоскость, параллельная плоскости основания конуса,  делит его так, что длины окружностей оснований конусов равны $48\pi$ и $56\pi$. Найдите объём конуса, отсекаемого от данного конуса проведённой плоскостью. Ответ сократите на $\pi$.
\end{taskBN}

\begin{taskBN}{133}
\addpictoright[0.25\textwidth]{images/1323130510912247n0}Образующая первого конуса в 9 раз больше, чем образующая второго конуса. Во сколько раз площадь боковой поверхности первого конуса больше площади боковой поверхности второго конуса? При этом у обоих конусов площади оснований равны.
\end{taskBN}

\begin{taskBN}{134}
\addpictoright[0.25\textwidth]{images/442805161150767n0}Площадь осевого сечения конуса равна $588$. Плоскость, параллельная плоскости основания конуса,  проходит так, что высота делится на отрезки $3$ и $18$ считая от вершины. Найдите радиус основания конуса, отсекаемого от данного конуса проведённой плоскостью. 
\end{taskBN}

\begin{taskBN}{135}
\addpictoright[0.25\textwidth]{images/601940318482933n0}Площадь осевого сечения конуса равна $48$. Плоскость, параллельная плоскости основания конуса,  делит его так, что радиусы оснований конусов равны $4$ и $8$. Найдите объём меньшего конуса. Ответ сократите на $\pi$.
\end{taskBN}

\begin{taskBN}{136}
\addpictoright[0.25\textwidth]{images/2860349643995992n0}Площадь основания конуса в два раза больше площади боковой поверхности. Найдите угол между образующей конуса и плоскостью основания. Ответ дайте в градусах.
\end{taskBN}

\begin{taskBN}{137}
\addpictoright[0.25\textwidth]{images/4986760834345294n0}Образующая конуса равна $41$, площадь боковой поверхности равна $369\pi$. Найдите объём конуса. Ответ сократите на $\pi$.
\end{taskBN}

\begin{taskBN}{138}
\addpictoright[0.25\textwidth]{images/0548382237472453n0}Радиус основания конуса равен $44$. Плоскость, параллельная плоскости основания конуса,  делит его так, что объёмы конусов равны $11664\pi$ и $21296\pi$. Найдите площадь осевого сечения конуса, отсекаемого от данного конуса проведённой плоскостью. 
\end{taskBN}

\begin{taskBN}{139}
\addpictoright[0.25\textwidth]{images/4428748481238693n0}Длина окружности основания конуса равна $72\pi$, высота равна $48$. Найдите объём конуса. Ответ сократите на $\pi$.
\end{taskBN}

\begin{taskBN}{140}
\addpictoright[0.25\textwidth]{images/951824348473266n0}В сосуде, имеющем форму конуса, уровень жидкости достигает $\frac{1}{3}$ высоты. Объём жидкости равен 10мл. Сколько миллилитров жидкости поместится в весь сосуд?
\end{taskBN}

\begin{taskBN}{141}
\addpictoright[0.25\textwidth]{images/628566919745285n0}Площадь осевого сечения конуса равна $960$. Плоскость, параллельная плоскости основания конуса,  проходит так, что высота делится на отрезки $15$ и $5$ считая от основания. Найдите объём меньшего конуса. Ответ сократите на $\pi$.
\end{taskBN}

\begin{taskBN}{142}
\addpictoright[0.25\textwidth]{images/532399550490992n0}Во сколько раз уменьшили объём конуса, если его высота уменьшилась в 3 раза? При этом радиус основания не изменился.
\end{taskBN}

\begin{taskBN}{143}
\addpictoright[0.25\textwidth]{images/306814541295476n0}Длина окружности основания первого конуса в 10 раз меньше, чем длина окружности основания второго конуса. Во сколько раз площадь боковой поверхности первого конуса меньше площади боковой поверхности второго конуса? При этом у обоих конусов образующие равны.
\end{taskBN}

\begin{taskBN}{144}
\addpictoright[0.25\textwidth]{images/6241203179283163n0}Объём конуса равен $5488\pi$, высота равна $21$. Найдите образующую конуса. 
\end{taskBN}

\begin{taskBN}{145}
\addpictoright[0.25\textwidth]{images/107780942743969n0}Длина окружности основания конуса равна $96\pi$, образующая равна $50$. Найдите площадь боковой поверхности конуса. Ответ сократите на $\pi$.
\end{taskBN}

\begin{taskBN}{146}
\addpictoright[0.25\textwidth]{images/030859940466587n0}Длина окружности основания конуса равна $40\pi$, высота равна $48$. Найдите объём конуса. Ответ сократите на $\pi$.
\end{taskBN}

\begin{taskBN}{147}
\addpictoright[0.25\textwidth]{images/263744919184861n0}Высота конуса равна $24$. Плоскость, параллельная плоскости основания конуса,  делит его так, что радиусы оснований конусов равны $12$ и $18$. Найдите объём меньшего конуса. Ответ сократите на $\pi$.
\end{taskBN}

\begin{taskBN}{148}
\addpictoright[0.25\textwidth]{images/2429547423852225n0}Во сколько раз уменьшили образующую конуса, если его площадь боковой поверхности уменьшилась в 7 раз? При этом радиус основания не изменился.
\end{taskBN}

\begin{taskBN}{149}
\addpictoright[0.25\textwidth]{images/439197633899392n0}В сосуде, имеющем форму конуса, уровень жидкости достигает $\frac{1}{3}$ высоты. Объём жидкости равен 108мл. Сколько миллилитров жидкости нужно долить, чтобы наполнить сосуд доверху?
\end{taskBN}

\begin{taskBN}{150}
\addpictoright[0.25\textwidth]{images/345941105908782n0}В сосуде, имеющем форму конуса, уровень жидкости достигает $\frac{1}{2}$ высоты. Объём жидкости равен 81мл. Сколько миллилитров жидкости поместится в весь сосуд?
\end{taskBN}

\begin{taskBN}{151}
\addpictoright[0.25\textwidth]{images/153584256285706n0}Во сколько раз увеличили радиус основания конуса, если его объём увеличился в 25 раз? При этом высота не изменилась.
\end{taskBN}

\begin{taskBN}{152}
\addpictoright[0.25\textwidth]{images/873524065155783n0}В сосуде, имеющем форму конуса, уровень жидкости достигает $\frac{4}{9}$ высоты. Объём жидкости равен 448мл. Сколько миллилитров жидкости нужно долить, чтобы наполнить сосуд доверху?
\end{taskBN}

\begin{taskBN}{153}
\addpictoright[0.25\textwidth]{images/275423434684176n0}Объём конуса равен $2560\pi$, длина окружности основания равна $32\pi$. Найдите образующую конуса. 
\end{taskBN}

\begin{taskBN}{154}
\addpictoright[0.25\textwidth]{images/507143603580272n0}Длина окружности основания первого конуса в 5 раз меньше, чем длина окружности основания второго конуса. Во сколько раз площадь боковой поверхности первого конуса меньше площади боковой поверхности второго конуса? При этом у обоих конусов образующие равны.
\end{taskBN}

\begin{taskBN}{155}
\addpictoright[0.25\textwidth]{images/762352393880738n0}В сосуде, имеющем форму конуса, уровень жидкости достигает $\frac{6}{7}$ высоты. Объём жидкости равен 2160мл. Сколько миллилитров жидкости нужно долить, чтобы наполнить сосуд доверху?
\end{taskBN}

\begin{taskBN}{156}
\addpictoright[0.25\textwidth]{images/020862708401065n0}Высота конуса равна $9$. Плоскость, параллельная плоскости основания конуса,  делит его так, что объёмы конусов равны $128\pi$ и $432\pi$. Найдите длину окружности основания конуса, отсекаемого от данного конуса проведённой плоскостью. Ответ сократите на $\pi$.
\end{taskBN}

\begin{taskBN}{157}
\addpictoright[0.25\textwidth]{images/2917257617795253n0}Во сколько раз уменьшили радиус основания конуса, если его объём уменьшился в 64 раза? При этом высота не изменилась.
\end{taskBN}

\begin{taskBN}{158}
\addpictoright[0.25\textwidth]{images/254642696522062n0}Объём конуса равен $5488\pi$. Плоскость, параллельная плоскости основания конуса,  проходит так, что высота делится на отрезки $9$ и $12$ считая от вершины. Найдите площадь осевого сечения конуса, отсекаемого от данного конуса проведённой плоскостью. 
\end{taskBN}

\begin{taskBN}{159}
\addpictoright[0.25\textwidth]{images/8766846238691874n0}В сосуде, имеющем форму конуса, уровень жидкости достигает $\frac{3}{4}$ высоты. Объём жидкости равен 216мл. Сколько миллилитров жидкости поместится в весь сосуд?
\end{taskBN}

\begin{taskBN}{160}
\addpictoright[0.25\textwidth]{images/2924191530069606n0}Площадь боковой поверхности конуса равна $369\pi$, высота равна $40$. Найдите образующую конуса. 
\end{taskBN}

\begin{taskBN}{161}
\addpictoright[0.25\textwidth]{images/536450626375958n0}Площадь осевого сечения конуса равна $1200$. Плоскость, параллельная плоскости основания конуса,  проходит так, что высота делится на отрезки $8$ и $32$ считая от вершины. Найдите объём меньшего конуса. Ответ сократите на $\pi$.
\end{taskBN}

\begin{taskBN}{162}
\addpictoright[0.25\textwidth]{images/46158018435603n0}Во сколько раз увеличили радиус основания конуса, если его площадь основания увеличилась в 64 раза?
\end{taskBN}

\begin{taskBN}{163}
\addpictoright[0.25\textwidth]{images/152285381809346n0}Площадь осевого сечения конуса равна $1728$. Плоскость, параллельная плоскости основания конуса,  проходит так, что высота делится на отрезки $9$ и $27$ считая от вершины. Найдите объём конуса, отсекаемого от данного конуса проведённой плоскостью. Ответ сократите на $\pi$.
\end{taskBN}

\begin{taskBN}{164}
\addpictoright[0.25\textwidth]{images/971524850204105n0}Во сколько раз уменьшили площадь основания конуса, если его радиус основания уменьшился в 9 раз?
\end{taskBN}

\begin{taskBN}{165}
\addpictoright[0.25\textwidth]{images/238136020639385n0}Площадь окружности основания конуса равна $1024\pi$. Плоскость, параллельная плоскости основания конуса,  проходит так, что высота делится на отрезки $15$ и $9$ считая от вершины. Найдите объём меньшего конуса. Ответ сократите на $\pi$.
\end{taskBN}

\begin{taskBN}{166}
\addpictoright[0.25\textwidth]{images/109359829820486n0}В сосуде, имеющем форму конуса, уровень жидкости достигает $\frac{1}{3}$ высоты. Объём жидкости равен 72мл. Сколько миллилитров жидкости поместится в весь сосуд?
\end{taskBN}

\begin{taskBN}{167}
\addpictoright[0.25\textwidth]{images/397879324010246n0}Площадь боковой поверхности первого конуса в 6 раз больше, чем площадь боковой поверхности второго конуса. Во сколько раз длина окружности основания первого конуса больше длины окружности основания второго конуса? При этом у обоих конусов образующие равны.
\end{taskBN}

\begin{taskBN}{168}
\addpictoright[0.25\textwidth]{images/6192282400516167n0}Площадь боковой поверхности первого конуса в 5 раз меньше, чем площадь боковой поверхности второго конуса. Во сколько раз площадь основания первого конуса меньше площади основания второго конуса? При этом у обоих конусов образующие равны.
\end{taskBN}

\begin{taskBN}{169}
\addpictoright[0.25\textwidth]{images/5955719415375584n0}Во сколько раз увеличили радиус основания конуса, если его объём увеличился в 9 раз? При этом высота не изменилась.
\end{taskBN}

\begin{taskBN}{170}
\addpictoright[0.25\textwidth]{images/792523623882647n0}Во сколько раз уменьшили площадь основания конуса, если его объём уменьшился в 9 раз? При этом радиус основания не изменился.
\end{taskBN}

\begin{taskBN}{171}
\addpictoright[0.25\textwidth]{images/090064625263451n0}Во сколько раз увеличили радиус основания конуса, если его объём увеличился в 64 раза? При этом высота не изменилась.
\end{taskBN}

\begin{taskBN}{172}
\addpictoright[0.25\textwidth]{images/734917634088808n0}Высота конуса равна $18$. Плоскость, параллельная плоскости основания конуса,  делит его так, что объёмы конусов равны $2000\pi$ и $3456\pi$. Найдите длину окружности основания конуса, отсекаемого от данного конуса проведённой плоскостью. Ответ сократите на $\pi$.
\end{taskBN}

\begin{taskBN}{173}
\addpictoright[0.25\textwidth]{images/44393770613295414n0}В сосуде, имеющем форму конуса, уровень жидкости достигает $\frac{3}{10}$ высоты. Объём жидкости равен 270мл. Сколько миллилитров жидкости нужно долить, чтобы наполнить сосуд доверху?
\end{taskBN}

\begin{taskBN}{174}
\addpictoright[0.25\textwidth]{images/417717234509288n0}В сосуде, имеющем форму конуса, уровень жидкости достигает $\frac{1}{2}$ высоты. Объём жидкости равен 162мл. Сколько миллилитров жидкости поместится в весь сосуд?
\end{taskBN}

\begin{taskBN}{175}
\addpictoright[0.25\textwidth]{images/835040144915376n0}Во сколько раз уменьшили высоту конуса, если его объём уменьшился в 6 раз? При этом радиус основания не изменился.
\end{taskBN}

\begin{taskBN}{176}
\addpictoright[0.25\textwidth]{images/6437417542545731n0}Во сколько раз уменьшили длину окружности основания конуса, если его площадь основания уменьшилась в 49 раз?
\end{taskBN}

\begin{taskBN}{177}
\addpictoright[0.25\textwidth]{images/311636534914331n0}Площадь основания первого конуса в 49 раз больше, чем площадь основания второго конуса. Во сколько раз радиус основания первого конуса больше радиуса основания второго конуса?
\end{taskBN}

\begin{taskBN}{178}
\addpictoright[0.25\textwidth]{images/22350977702767838n0}Площадь осевого сечения конуса равна $1452$. Плоскость, параллельная плоскости основания конуса,  делит его так, что объёмы конусов равны $16\pi$ и $21296\pi$. Найдите высоту меньшего конуса. 
\end{taskBN}

\begin{taskBN}{179}
\addpictoright[0.25\textwidth]{images/513336970901977n0}Объём первого конуса в 5 раз больше, чем объём второго конуса. Во сколько раз площадь основания первого конуса больше площади основания второго конуса?
\end{taskBN}

\begin{taskBN}{180}
\addpictoright[0.25\textwidth]{images/202339140167082n0}Объём конуса равен $2700\pi$. Плоскость, параллельная плоскости основания конуса,  делит его так, что радиусы оснований конусов равны $10$ и $15$. Найдите высоту конуса, отсекаемого от данного конуса проведённой плоскостью. 
\end{taskBN}

\begin{taskBN}{181}
\addpictoright[0.25\textwidth]{images/9407323720927652n0}Объём конуса равен $15360\pi$. Плоскость, параллельная плоскости основания конуса,  делит его так, что длины окружностей оснований конусов равны $24\pi$ и $96\pi$. Найдите площадь осевого сечения конуса, отсекаемого от данного конуса проведённой плоскостью. 
\end{taskBN}

\begin{taskBN}{182}
\addpictoright[0.25\textwidth]{images/04541711540346n0}Образующая конуса равна $51$, площадь окружности основания равна $576\pi$. Найдите объём конуса. Ответ сократите на $\pi$.
\end{taskBN}

\begin{taskBN}{183}
\addpictoright[0.25\textwidth]{images/4820634068351726n0}Диаметр основания конуса равен $88$. Плоскость, параллельная плоскости основания конуса,  делит его так, что объёмы конусов равны $128\pi$ и $21296\pi$. Найдите высоту конуса, отсекаемого от данного конуса проведённой плоскостью. 
\end{taskBN}

\begin{taskBN}{184}
\addpictoright[0.25\textwidth]{images/05449208714100151n0}В сосуде, имеющем форму конуса, уровень жидкости достигает $\frac{7}{8}$ высоты. Объём жидкости равен 1029мл. Сколько миллилитров жидкости нужно долить, чтобы наполнить сосуд доверху?
\end{taskBN}

\begin{taskBN}{185}
\addpictoright[0.25\textwidth]{images/519382977928922n0}Во сколько раз увеличили радиус основания конуса, если его площадь основания увеличилась в 9 раз?
\end{taskBN}

\begin{taskBN}{186}
\addpictoright[0.25\textwidth]{images/952170593637904n0}Площадь окружности основания конуса равна $49\pi$, площадь боковой поверхности равна $175\pi$. Найдите объём конуса. Ответ сократите на $\pi$.
\end{taskBN}

\begin{taskBN}{187}
\addpictoright[0.25\textwidth]{images/1098821098114375n0}Радиус основания первого конуса в 5 раз меньше, чем радиус основания второго конуса. Во сколько раз площадь основания первого конуса меньше площади основания второго конуса?
\end{taskBN}

\begin{taskBN}{188}
\addpictoright[0.25\textwidth]{images/013068610270934n0}Объём конуса равен $2000\pi$. Плоскость, параллельная плоскости основания конуса,  проходит так, что высота делится на отрезки $9$ и $6$ считая от основания. Найдите площадь осевого сечения конуса, отсекаемого от данного конуса проведённой плоскостью. 
\end{taskBN}

\begin{taskBN}{189}
\addpictoright[0.25\textwidth]{images/0123354837880494n0}Площадь осевого сечения конуса равна $300$. Плоскость, параллельная плоскости основания конуса,  проходит так, что высота делится на отрезки $9$ и $6$ считая от вершины. Найдите объём меньшего конуса. Ответ сократите на $\pi$.
\end{taskBN}

\begin{taskBN}{190}
\addpictoright[0.25\textwidth]{images/915312362063925n0}В сосуде, имеющем форму конуса, уровень жидкости достигает $\frac{1}{5}$ высоты. Объём жидкости равен 7мл. Сколько миллилитров жидкости поместится в весь сосуд?
\end{taskBN}

\begin{taskBN}{191}
\addpictoright[0.25\textwidth]{images/463274746583569n0}Площадь боковой поверхности конуса в два раза больше площади основания. Найдите угол между образующей конуса и плоскостью основания. Ответ дайте в градусах.
\end{taskBN}

\begin{taskBN}{192}
\addpictoright[0.25\textwidth]{images/873333118643097n0}Образующая конуса равна $60$, высота равна $48$. Найдите площадь окружности основания конуса. Ответ сократите на $\pi$.
\end{taskBN}

\begin{taskBN}{193}
\addpictoright[0.25\textwidth]{images/37957328441388n0}Образующая первого конуса в 9 раз меньше, чем образующая второго конуса. Во сколько раз площадь боковой поверхности первого конуса меньше площади боковой поверхности второго конуса? При этом у обоих конусов длины окружностей основания равны.
\end{taskBN}

\begin{taskBN}{194}
\addpictoright[0.25\textwidth]{images/7484382150212865n0}В сосуде, имеющем форму конуса, уровень жидкости достигает $\frac{2}{9}$ высоты. Объём жидкости равен 40мл. Сколько миллилитров жидкости нужно долить, чтобы наполнить сосуд доверху?
\end{taskBN}

\begin{taskBN}{195}
\addpictoright[0.25\textwidth]{images/145261425536647n0}Высота конуса равна $16$, объём равен $768\pi$. Найдите образующую конуса. 
\end{taskBN}

\begin{taskBN}{196}
\addpictoright[0.25\textwidth]{images/6916304076484077n0}Во сколько раз увеличили площадь боковой поверхности конуса, если его образующая увеличилась в 9 раз? При этом длина окружности основания не изменилась.
\end{taskBN}

\begin{taskBN}{197}
\addpictoright[0.25\textwidth]{images/7294782143311154n0}В сосуде, имеющем форму конуса, уровень жидкости достигает $\frac{4}{9}$ высоты. Объём жидкости равен 384мл. Сколько миллилитров жидкости поместится в весь сосуд?
\end{taskBN}

\begin{taskBN}{198}
\addpictoright[0.25\textwidth]{images/497983542413781n0}Площадь основания первого конуса в 81 раз меньше, чем площадь основания второго конуса. Во сколько раз длина окружности основания первого конуса меньше длины окружности основания второго конуса?
\end{taskBN}

\begin{taskBN}{199}
\addpictoright[0.25\textwidth]{images/813790083797532n0}Объём конуса равен $8748\pi$. Плоскость, параллельная плоскости основания конуса,  делит его так, что длины окружностей оснований конусов равны $24\pi$ и $54\pi$. Найдите высоту конуса, отсекаемого от данного конуса проведённой плоскостью. 
\end{taskBN}

\begin{taskBN}{200}
\addpictoright[0.25\textwidth]{images/36646800855707n0}В сосуде, имеющем форму конуса, уровень жидкости достигает $\frac{3}{5}$ высоты. Объём жидкости равен 270мл. Сколько миллилитров жидкости поместится в весь сосуд?
\end{taskBN}

\begin{taskBN}{201}
\addpictoright[0.25\textwidth]{images/637602188943198n0}Высота конуса равна $12$. Плоскость, параллельная плоскости основания конуса,  делит его так, что площади окружностей оснований конусов равны $144\pi$ и $256\pi$. Найдите объём конуса, отсекаемого от данного конуса проведённой плоскостью. Ответ сократите на $\pi$.
\end{taskBN}

\begin{taskBN}{202}
\addpictoright[0.25\textwidth]{images/088154887076035n0}В сосуде, имеющем форму конуса, уровень жидкости достигает $\frac{1}{2}$ высоты. Объём жидкости равен 80мл. Сколько миллилитров жидкости поместится в весь сосуд?
\end{taskBN}

\begin{taskBN}{203}
\addpictoright[0.25\textwidth]{images/96625812895043n0}В сосуде, имеющем форму конуса, уровень жидкости достигает $\frac{3}{8}$ высоты. Объём жидкости равен 81мл. Сколько миллилитров жидкости поместится в весь сосуд?
\end{taskBN}

\begin{taskBN}{204}
\addpictoright[0.25\textwidth]{images/704154907540667n0}Объём конуса равен $21296\pi$. Плоскость, параллельная плоскости основания конуса,  проходит так, что высота делится на отрезки $9$ и $24$ считая от вершины. Найдите площадь окружности основания меньшего конуса. Ответ сократите на $\pi$.
\end{taskBN}

\begin{taskBN}{205}
\addpictoright[0.25\textwidth]{images/045941495058712n0}Объём конуса равен $21296\pi$. Плоскость, параллельная плоскости основания конуса,  проходит так, что высота делится на отрезки $12$ и $21$ считая от основания. Найдите площадь осевого сечения меньшего конуса. 
\end{taskBN}

\begin{taskBN}{206}
\addpictoright[0.25\textwidth]{images/2954889965212326n0}Объём конуса равен $128\pi$, радиус основания равен $8$. Найдите площадь боковой поверхности конуса. Ответ сократите на $\pi$.
\end{taskBN}

\begin{taskBN}{207}
\addpictoright[0.25\textwidth]{images/05223821082728897n0}Длина окружности основания конуса равна $96\pi$. Плоскость, параллельная плоскости основания конуса,  проходит так, что высота делится на отрезки $10$ и $10$ считая от основания. Найдите объём конуса, отсекаемого от данного конуса проведённой плоскостью. Ответ сократите на $\pi$.
\end{taskBN}

\begin{taskBN}{208}
\addpictoright[0.25\textwidth]{images/386234641140425n0}Площадь осевого сечения конуса равна $768$. Плоскость, параллельная плоскости основания конуса,  проходит так, что высота делится на отрезки $18$ и $6$ считая от вершины. Найдите объём меньшего конуса. Ответ сократите на $\pi$.
\end{taskBN}

\begin{taskBN}{209}
\addpictoright[0.25\textwidth]{images/542605675813436n0}Площадь боковой поверхности конуса равна $1280\pi$, объём равен $8192\pi$. Найдите длину окружности основания конуса. Ответ сократите на $\pi$.
\end{taskBN}

\begin{taskBN}{210}
\addpictoright[0.25\textwidth]{images/3582091553566047n0}Площадь боковой поверхности конуса равна $1620\pi$, площадь окружности основания равна $1296\pi$. Найдите образующую конуса. 
\end{taskBN}

\begin{taskBN}{211}
\addpictoright[0.25\textwidth]{images/21497960050027576n0}Объём первого конуса в 64 раза больше, чем объём второго конуса. Во сколько раз радиус основания первого конуса больше радиуса основания второго конуса? При этом у обоих конусов высоты равны.
\end{taskBN}

\begin{taskBN}{212}
\addpictoright[0.25\textwidth]{images/15704539172753718n0}Радиус основания конуса равен $40$. Плоскость, параллельная плоскости основания конуса,  делит его так, что объёмы конусов равны $128\pi$ и $16000\pi$. Найдите площадь осевого сечения конуса, отсекаемого от данного конуса проведённой плоскостью. 
\end{taskBN}

\begin{taskBN}{213}
\addpictoright[0.25\textwidth]{images/506967090977659n0}В сосуде, имеющем форму конуса, уровень жидкости достигает $\frac{1}{7}$ высоты. Объём жидкости равен 2мл. Сколько миллилитров жидкости нужно долить, чтобы наполнить сосуд доверху?
\end{taskBN}

\begin{taskBN}{214}
\addpictoright[0.25\textwidth]{images/7169681332984446n0}Длина окружности основания конуса равна $48\pi$. Плоскость, параллельная плоскости основания конуса,  делит его так, что объёмы конусов равны $240\pi$ и $1920\pi$. Найдите высоту конуса, отсекаемого от данного конуса проведённой плоскостью. 
\end{taskBN}

\begin{taskBN}{215}
\addpictoright[0.25\textwidth]{images/8198862894404986n0}Во сколько раз увеличили площадь боковой поверхности конуса, если его длина окружности основания увеличилась в 8 раз? При этом образующая не изменилась.
\end{taskBN}

\begin{taskBN}{216}
\addpictoright[0.25\textwidth]{images/079608517483466n0}Высота конуса равна $48$. Плоскость, параллельная плоскости основания конуса,  делит его так, что радиусы оснований конусов равны $15$ и $36$. Найдите объём меньшего конуса. Ответ сократите на $\pi$.
\end{taskBN}

\begin{taskBN}{217}
\addpictoright[0.25\textwidth]{images/031885583410133n0}Во сколько раз уменьшили площадь боковой поверхности конуса, если его длина окружности основания уменьшилась в 7 раз? При этом образующая не изменилась.
\end{taskBN}

\begin{taskBN}{218}
\addpictoright[0.25\textwidth]{images/757918534777249n0}Площадь боковой поверхности конуса равна $1040\pi$, длина окружности основания равна $40\pi$. Найдите объём конуса. Ответ сократите на $\pi$.
\end{taskBN}

\begin{taskBN}{219}
\addpictoright[0.25\textwidth]{images/9384826232666517n0}Объём конуса равен $320\pi$, образующая равна $17$. Найдите площадь окружности основания конуса. Ответ сократите на $\pi$.
\end{taskBN}

\begin{taskBN}{220}
\addpictoright[0.25\textwidth]{images/468206932395445n0}Площадь окружности основания конуса равна $576\pi$, объём равен $1920\pi$. Найдите образующую конуса. 
\end{taskBN}

\begin{taskBN}{221}
\addpictoright[0.25\textwidth]{images/8485790792689454n0}В сосуде, имеющем форму конуса, уровень жидкости достигает $\frac{1}{4}$ высоты. Объём жидкости равен 6мл. Сколько миллилитров жидкости нужно долить, чтобы наполнить сосуд доверху?
\end{taskBN}

\begin{taskBN}{222}
\addpictoright[0.25\textwidth]{images/369052202136256n0}Во сколько раз увеличили площадь боковой поверхности конуса, если его образующая увеличилась в 9 раз? При этом длина окружности основания не изменилась.
\end{taskBN}

\begin{taskBN}{223}
\addpictoright[0.25\textwidth]{images/578150414299656n0}Во сколько раз уменьшили площадь основания конуса, если его длина окружности основания уменьшилась в 7 раз?
\end{taskBN}

\begin{taskBN}{224}
\addpictoright[0.25\textwidth]{images/318545458173031n0}В сосуде, имеющем форму конуса, уровень жидкости достигает $\frac{3}{4}$ высоты. Объём жидкости равен 1944мл. Сколько миллилитров жидкости нужно долить, чтобы наполнить сосуд доверху?
\end{taskBN}

\begin{taskBN}{225}
\addpictoright[0.25\textwidth]{images/234273155958306n0}В сосуде, имеющем форму конуса, уровень жидкости достигает $\frac{1}{9}$ высоты. Объём жидкости равен 1мл. Сколько миллилитров жидкости нужно долить, чтобы наполнить сосуд доверху?
\end{taskBN}

\begin{taskBN}{226}
\addpictoright[0.25\textwidth]{images/165114220507062n0}Образующая конуса равна $30$, высота равна $24$. Найдите площадь окружности основания конуса. Ответ сократите на $\pi$.
\end{taskBN}

\begin{taskBN}{227}
\addpictoright[0.25\textwidth]{images/803991025725489n0}Образующая конуса равна $53$, площадь боковой поверхности равна $1484\pi$. Найдите объём конуса. Ответ сократите на $\pi$.
\end{taskBN}

\begin{taskBN}{228}
\addpictoright[0.25\textwidth]{images/502292736989675n0}Длина окружности основания конуса равна $96\pi$. Плоскость, параллельная плоскости основания конуса,  проходит так, что высота делится на отрезки $7$ и $7$ считая от вершины. Найдите площадь осевого сечения конуса, отсекаемого от данного конуса проведённой плоскостью. 
\end{taskBN}

\begin{taskBN}{229}
\addpictoright[0.25\textwidth]{images/9538397157871863n0}Площадь боковой поверхности конуса равна $720\pi$, высота равна $18$. Найдите образующую конуса. 
\end{taskBN}

\begin{taskBN}{230}
\addpictoright[0.25\textwidth]{images/8986377121614264n0}Высота конуса равна $15$. Плоскость, параллельная плоскости основания конуса,  делит его так, что площади окружностей оснований конусов равны $16\pi$ и $400\pi$. Найдите площадь осевого сечения меньшего конуса. 
\end{taskBN}

\begin{taskBN}{231}
\addpictoright[0.25\textwidth]{images/509984546514007n0}Во сколько раз увеличили длину окружности основания конуса, если его площадь боковой поверхности увеличилась в 4 раза? При этом образующая не изменилась.
\end{taskBN}

\begin{taskBN}{232}
\addpictoright[0.25\textwidth]{images/219521295802855n0}Радиус основания первого конуса в 8 раз больше, чем радиус основания второго конуса. Во сколько раз площадь основания первого конуса больше площади основания второго конуса?
\end{taskBN}

\begin{taskBN}{233}
\addpictoright[0.25\textwidth]{images/421741282624057n0}В сосуде, имеющем форму конуса, уровень жидкости достигает $\frac{1}{5}$ высоты. Объём жидкости равен 6мл. Сколько миллилитров жидкости поместится в весь сосуд?
\end{taskBN}

\begin{taskBN}{234}
\addpictoright[0.25\textwidth]{images/8963896963580995n0}Объём конуса равен $12000\pi$. Плоскость, параллельная плоскости основания конуса,  делит его так, что длины окружностей оснований конусов равны $24\pi$ и $60\pi$. Найдите высоту конуса, отсекаемого от данного конуса проведённой плоскостью. 
\end{taskBN}

\begin{taskBN}{235}
\addpictoright[0.25\textwidth]{images/69632809311413n0}В сосуде, имеющем форму конуса, уровень жидкости достигает $\frac{1}{6}$ высоты. Объём жидкости равен 2мл. Сколько миллилитров жидкости поместится в весь сосуд?
\end{taskBN}

\begin{taskBN}{236}
\addpictoright[0.25\textwidth]{images/811713592329224n0}В сосуде, имеющем форму конуса, уровень жидкости достигает $\frac{1}{4}$ высоты. Объём жидкости равен 80мл. Сколько миллилитров жидкости поместится в весь сосуд?
\end{taskBN}

\begin{taskBN}{237}
\addpictoright[0.25\textwidth]{images/314450530390741n0}В сосуде, имеющем форму конуса, уровень жидкости достигает $\frac{3}{5}$ высоты. Объём жидкости равен 162мл. Сколько миллилитров жидкости нужно долить, чтобы наполнить сосуд доверху?
\end{taskBN}

\begin{taskBN}{238}
\addpictoright[0.25\textwidth]{images/3451072330528695n0}Во сколько раз увеличили площадь боковой поверхности конуса, если его длина окружности основания увеличилась в 5 раз? При этом образующая не изменилась.
\end{taskBN}

\begin{taskBN}{239}
\addpictoright[0.25\textwidth]{images/566921053405743n0}Площадь осевого сечения конуса равна $1452$. Плоскость, параллельная плоскости основания конуса,  проходит так, что высота делится на отрезки $40$ и $4$ считая от основания. Найдите длину окружности основания конуса, отсекаемого от данного конуса проведённой плоскостью. Ответ сократите на $\pi$.
\end{taskBN}

\begin{taskBN}{240}
\addpictoright[0.25\textwidth]{images/2677320626805957n0}Образующая конуса равна $10$, радиус основания равен $6$. Найдите объём конуса. Ответ сократите на $\pi$.
\end{taskBN}

\begin{taskBN}{241}
\addpictoright[0.25\textwidth]{images/382642329119957n0}В сосуде, имеющем форму конуса, уровень жидкости достигает $\frac{3}{4}$ высоты. Объём жидкости равен 216мл. Сколько миллилитров жидкости нужно долить, чтобы наполнить сосуд доверху?
\end{taskBN}

\begin{taskBN}{242}
\addpictoright[0.25\textwidth]{images/790598372502421n0}Высота конуса равна $14$, площадь окружности основания равна $2304\pi$. Найдите объём конуса. Ответ сократите на $\pi$.
\end{taskBN}

\begin{taskBN}{243}
\addpictoright[0.25\textwidth]{images/4712358896008768n0}Во сколько раз увеличили радиус основания конуса, если его площадь боковой поверхности увеличилась в 2 раза? При этом образующая не изменилась.
\end{taskBN}

\begin{taskBN}{244}
\addpictoright[0.25\textwidth]{images/7183525619117517n0}Во сколько раз уменьшили площадь боковой поверхности конуса, если его длина окружности основания уменьшилась в 10 раз? При этом образующая не изменилась.
\end{taskBN}

\begin{taskBN}{245}
\addpictoright[0.25\textwidth]{images/179967340331908n0}Площадь основания конуса в два раза больше площади боковой поверхности. Найдите угол между образующей конуса и плоскостью основания. Ответ дайте в градусах.
\end{taskBN}

\begin{taskBN}{246}
\addpictoright[0.25\textwidth]{images/9306946400712814n0}Площадь боковой поверхности конуса равна $544\pi$, объём равен $2560\pi$. Найдите образующую конуса. 
\end{taskBN}

\begin{taskBN}{247}
\addpictoright[0.25\textwidth]{images/9087750700717956n0}Площадь осевого сечения конуса равна $432$. Плоскость, параллельная плоскости основания конуса,  делит его так, что объёмы конусов равны $16\pi$ и $3456\pi$. Найдите длину окружности основания меньшего конуса. Ответ сократите на $\pi$.
\end{taskBN}

\begin{taskBN}{248}
\addpictoright[0.25\textwidth]{images/790520614095102n0}Площадь боковой поверхности конуса равна $2420\pi$, высота равна $33$. Найдите длину окружности основания конуса. Ответ сократите на $\pi$.
\end{taskBN}

\begin{taskBN}{249}
\addpictoright[0.25\textwidth]{images/775752548288325n0}Образующая первого конуса в 7 раз больше, чем образующая второго конуса. Во сколько раз объём первого конуса больше объёма второго конуса?
\end{taskBN}

\begin{taskBN}{250}
\addpictoright[0.25\textwidth]{images/702978722032613n0}Радиус основания конуса равен $36$. Плоскость, параллельная плоскости основания конуса,  делит его так, что площади осевых сечений конусов равны $108$ и $972$. Найдите высоту конуса, отсекаемого от данного конуса проведённой плоскостью. 
\end{taskBN}

\begin{taskBN}{251}
\addpictoright[0.25\textwidth]{images/661122141579491n0}Площадь боковой поверхности конуса равна $609\pi$, образующая равна $29$. Найдите длину окружности основания конуса. Ответ сократите на $\pi$.
\end{taskBN}

\begin{taskBN}{252}
\addpictoright[0.25\textwidth]{images/395284209312613n0}Во сколько раз увеличили радиус основания конуса, если его площадь основания увеличилась в 36 раз?
\end{taskBN}

\begin{taskBN}{253}
\addpictoright[0.25\textwidth]{images/3995217943817819n0}В сосуде, имеющем форму конуса, уровень жидкости достигает $\frac{3}{4}$ высоты. Объём жидкости равен 135мл. Сколько миллилитров жидкости поместится в весь сосуд?
\end{taskBN}

\begin{taskBN}{254}
\addpictoright[0.25\textwidth]{images/501461151283968n0}Высота конуса равна $15$. Плоскость, параллельная плоскости основания конуса,  делит его так, что объёмы конусов равны $240\pi$ и $6480\pi$. Найдите площадь осевого сечения конуса, отсекаемого от данного конуса проведённой плоскостью. 
\end{taskBN}

\begin{taskBN}{255}
\addpictoright[0.25\textwidth]{images/3606932821000208n0}Во сколько раз увеличили длину окружности основания конуса, если его площадь боковой поверхности увеличилась в 8 раз? При этом образующая не изменилась.
\end{taskBN}

\begin{taskBN}{256}
\addpictoright[0.25\textwidth]{images/238230485861598n0}Площадь боковой поверхности первого конуса в 8 раз больше, чем площадь боковой поверхности второго конуса. Во сколько раз длина окружности основания первого конуса больше длины окружности основания второго конуса? При этом у обоих конусов образующие равны.
\end{taskBN}

\begin{taskBN}{257}
\addpictoright[0.25\textwidth]{images/204075345691558n0}В сосуде, имеющем форму конуса, уровень жидкости достигает $\frac{2}{9}$ высоты. Объём жидкости равен 56мл. Сколько миллилитров жидкости нужно долить, чтобы наполнить сосуд доверху?
\end{taskBN}

\begin{taskBN}{258}
\addpictoright[0.25\textwidth]{images/1946042735288267n0}Объём конуса равен $432\pi$. Плоскость, параллельная плоскости основания конуса,  делит его так, что длины окружностей оснований конусов равны $16\pi$ и $24\pi$. Найдите высоту конуса, отсекаемого от данного конуса проведённой плоскостью. 
\end{taskBN}

\begin{taskBN}{259}
\addpictoright[0.25\textwidth]{images/3287492764347015n0}Длина окружности основания конуса равна $18\pi$. Плоскость, параллельная плоскости основания конуса,  делит его так, что площади осевых сечений конусов равны $12$ и $108$. Найдите высоту меньшего конуса. 
\end{taskBN}

\begin{taskBN}{260}
\addpictoright[0.25\textwidth]{images/675174186427454n0}Длина окружности основания конуса равна $28\pi$, высота равна $48$. Найдите образующую конуса. 
\end{taskBN}

\begin{taskBN}{261}
\addpictoright[0.25\textwidth]{images/176688014791504n0}Во сколько раз увеличили длину окружности основания конуса, если его площадь основания увеличилась в 25 раз?
\end{taskBN}

\begin{taskBN}{262}
\addpictoright[0.25\textwidth]{images/815614361872439n0}В сосуде, имеющем форму конуса, уровень жидкости достигает $\frac{2}{5}$ высоты. Объём жидкости равен 64мл. Сколько миллилитров жидкости нужно долить, чтобы наполнить сосуд доверху?
\end{taskBN}

\begin{taskBN}{263}
\addpictoright[0.25\textwidth]{images/892061661307037n0}Во сколько раз увеличили площадь основания конуса, если его радиус основания увеличился в 6 раз?
\end{taskBN}

\begin{taskBN}{264}
\addpictoright[0.25\textwidth]{images/996443919311575n0}В сосуде, имеющем форму конуса, уровень жидкости достигает $\frac{1}{2}$ высоты. Объём жидкости равен 162мл. Сколько миллилитров жидкости поместится в весь сосуд?
\end{taskBN}

\begin{taskBN}{265}
\addpictoright[0.25\textwidth]{images/735454640919948n0}Высота конуса равна $12$, образующая равна $20$. Найдите площадь окружности основания конуса. Ответ сократите на $\pi$.
\end{taskBN}

\begin{taskBN}{266}
\addpictoright[0.25\textwidth]{images/8847870547336887n0}Объём конуса равен $8640\pi$, площадь боковой поверхности равна $1224\pi$. Найдите образующую конуса. 
\end{taskBN}

\begin{taskBN}{267}
\addpictoright[0.25\textwidth]{images/2032547262739595n0}Объём конуса равен $4116\pi$. Плоскость, параллельная плоскости основания конуса,  проходит так, что высота делится на отрезки $24$ и $4$ считая от вершины. Найдите радиус основания конуса, отсекаемого от данного конуса проведённой плоскостью. 
\end{taskBN}

\begin{taskBN}{268}
\addpictoright[0.25\textwidth]{images/13552052473429499n0}Объём конуса равен $15360\pi$, образующая равна $52$. Найдите длину окружности основания конуса. Ответ сократите на $\pi$.
\end{taskBN}

\begin{taskBN}{269}
\addpictoright[0.25\textwidth]{images/4852121252391124n0}В сосуде, имеющем форму конуса, уровень жидкости достигает $\frac{1}{4}$ высоты. Объём жидкости равен 72мл. Сколько миллилитров жидкости поместится в весь сосуд?
\end{taskBN}

\begin{taskBN}{270}
\addpictoright[0.25\textwidth]{images/007089880657439318n0}Во сколько раз увеличили образующую конуса, если его площадь боковой поверхности увеличилась в 8 раз? При этом длина окружности основания не изменилась.
\end{taskBN}

\begin{taskBN}{271}
\addpictoright[0.25\textwidth]{images/677601598760429n0}Площадь осевого сечения конуса равна $972$. Плоскость, параллельная плоскости основания конуса,  проходит так, что высота делится на отрезки $12$ и $15$ считая от вершины. Найдите объём конуса, отсекаемого от данного конуса проведённой плоскостью. Ответ сократите на $\pi$.
\end{taskBN}

\begin{taskBN}{272}
\addpictoright[0.25\textwidth]{images/510507389194368n0}Во сколько раз уменьшили длину окружности основания конуса, если его площадь боковой поверхности уменьшилась в 9 раз? При этом образующая не изменилась.
\end{taskBN}

\begin{taskBN}{273}
\addpictoright[0.25\textwidth]{images/604655472757241n0}Площадь осевого сечения конуса равна $432$. Плоскость, параллельная плоскости основания конуса,  делит его так, что радиусы оснований конусов равны $16$ и $24$. Найдите объём меньшего конуса. Ответ сократите на $\pi$.
\end{taskBN}

\begin{taskBN}{274}
\addpictoright[0.25\textwidth]{images/090910881539537n0}Во сколько раз уменьшили радиус основания конуса, если его объём уменьшился в 36 раз? При этом высота не изменилась.
\end{taskBN}

\begin{taskBN}{275}
\addpictoright[0.25\textwidth]{images/891064165419091n0}Во сколько раз уменьшили радиус основания конуса, если его объём уменьшился в 64 раза? При этом высота не изменилась.
\end{taskBN}

\begin{taskBN}{276}
\addpictoright[0.25\textwidth]{images/3400365471701015n0}Длина окружности основания конуса равна $10\pi$, высота равна $12$. Найдите площадь боковой поверхности конуса. Ответ сократите на $\pi$.
\end{taskBN}

\begin{taskBN}{277}
\addpictoright[0.25\textwidth]{images/082827725292049n0}Во сколько раз уменьшили объём конуса, если его площадь основания уменьшилась в 9 раз? При этом радиус основания не изменился.
\end{taskBN}

\begin{taskBN}{278}
\addpictoright[0.25\textwidth]{images/158130704530048n0}Высота конуса равна $36$, объём равен $2700\pi$. Найдите образующую конуса. 
\end{taskBN}

\begin{taskBN}{279}
\addpictoright[0.25\textwidth]{images/741744342416309n0}В сосуде, имеющем форму конуса, уровень жидкости достигает $\frac{3}{4}$ высоты. Объём жидкости равен 162мл. Сколько миллилитров жидкости поместится в весь сосуд?
\end{taskBN}

\begin{taskBN}{280}
\addpictoright[0.25\textwidth]{images/5131098668353846n0}Радиус основания конуса равен $20$, высота равна $21$. Найдите площадь боковой поверхности конуса. Ответ сократите на $\pi$.
\end{taskBN}

\begin{taskBN}{281}
\addpictoright[0.25\textwidth]{images/165477936183601n0}Длина окружности основания конуса равна $48\pi$, объём равен $1920\pi$. Найдите площадь боковой поверхности конуса. Ответ сократите на $\pi$.
\end{taskBN}

\begin{taskBN}{282}
\addpictoright[0.25\textwidth]{images/411579366295572n0}Во сколько раз увеличили длину окружности основания конуса, если его площадь боковой поверхности увеличилась в 7 раз? При этом образующая не изменилась.
\end{taskBN}

\begin{taskBN}{283}
\addpictoright[0.25\textwidth]{images/27559236829056477n0}Объём конуса равен $1680\pi$, длина окружности основания равна $24\pi$. Найдите образующую конуса. 
\end{taskBN}

\begin{taskBN}{284}
\addpictoright[0.25\textwidth]{images/106623705774203n0}Высота конуса равна $27$. Плоскость, параллельная плоскости основания конуса,  делит его так, что площади осевых сечений конусов равны $768$ и $972$. Найдите объём конуса, отсекаемого от данного конуса проведённой плоскостью. Ответ сократите на $\pi$.
\end{taskBN}

\begin{taskBN}{285}
\addpictoright[0.25\textwidth]{images/7903693771070537n0}Площадь основания первого конуса в 25 раз меньше, чем площадь основания второго конуса. Во сколько раз длина окружности основания первого конуса меньше длины окружности основания второго конуса?
\end{taskBN}

\begin{taskBN}{286}
\addpictoright[0.25\textwidth]{images/431797711569951n0}Образующая конуса равна $51$, объём равен $16200\pi$. Найдите площадь боковой поверхности конуса. Ответ сократите на $\pi$.
\end{taskBN}

\begin{taskBN}{287}
\addpictoright[0.25\textwidth]{images/672726807608226n0}Площадь боковой поверхности конуса равна $15\pi$, объём равен $12\pi$. Найдите образующую конуса. 
\end{taskBN}

\begin{taskBN}{288}
\addpictoright[0.25\textwidth]{images/2357423800086025n0}Во сколько раз уменьшили высоту конуса, если его объём уменьшился в 3 раза? При этом радиус основания не изменился.
\end{taskBN}

\begin{taskBN}{289}
\addpictoright[0.25\textwidth]{images/259717423673276n0}Образующая конуса равна $13$, радиус основания равен $12$. Найдите объём конуса. Ответ сократите на $\pi$.
\end{taskBN}

\begin{taskBN}{290}
\addpictoright[0.25\textwidth]{images/9728923744523246n0}Радиус основания первого конуса в 9 раз меньше, чем радиус основания второго конуса. Во сколько раз площадь основания первого конуса меньше площади основания второго конуса?
\end{taskBN}

\begin{taskBN}{291}
\addpictoright[0.25\textwidth]{images/293463623646391n0}Образующая конуса равна $41$, длина окружности основания равна $80\pi$. Найдите площадь боковой поверхности конуса. Ответ сократите на $\pi$.
\end{taskBN}

\begin{taskBN}{292}
\addpictoright[0.25\textwidth]{images/7145372844939089n0}В сосуде, имеющем форму конуса, уровень жидкости достигает $\frac{1}{2}$ высоты. Объём жидкости равен 256мл. Сколько миллилитров жидкости нужно долить, чтобы наполнить сосуд доверху?
\end{taskBN}

\begin{taskBN}{293}
\addpictoright[0.25\textwidth]{images/263130815653077n0}Площадь осевого сечения конуса равна $48$. Плоскость, параллельная плоскости основания конуса,  делит его так, что объёмы конусов равны $12\pi$ и $96\pi$. Найдите радиус основания меньшего конуса. 
\end{taskBN}

\begin{taskBN}{294}
\addpictoright[0.25\textwidth]{images/218245683917018n0}Площадь основания первого конуса в 9 раз больше, чем площадь основания второго конуса. Во сколько раз площадь боковой поверхности первого конуса больше площади боковой поверхности второго конуса? При этом у обоих конусов образующие равны.
\end{taskBN}

\begin{taskBN}{295}
\addpictoright[0.25\textwidth]{images/868428545830508n0}В сосуде, имеющем форму конуса, уровень жидкости достигает $\frac{1}{2}$ высоты. Объём жидкости равен 448мл. Сколько миллилитров жидкости поместится в весь сосуд?
\end{taskBN}

\begin{taskBN}{296}
\addpictoright[0.25\textwidth]{images/911003242439316n0}Во сколько раз уменьшили площадь основания конуса, если его длина окружности основания уменьшилась в 4 раза?
\end{taskBN}

\begin{taskBN}{297}
\addpictoright[0.25\textwidth]{images/710776354818264n0}Высота конуса равна $18$, образующая равна $30$. Найдите длину окружности основания конуса. Ответ сократите на $\pi$.
\end{taskBN}

\begin{taskBN}{298}
\addpictoright[0.25\textwidth]{images/8083098116713583n0}Высота конуса равна $16$, радиус основания равен $12$. Найдите площадь боковой поверхности конуса. Ответ сократите на $\pi$.
\end{taskBN}

\begin{taskBN}{299}
\addpictoright[0.25\textwidth]{images/308300089743213n0}Высота конуса равна $12$. Плоскость, параллельная плоскости основания конуса,  делит его так, что объёмы конусов равны $16\pi$ и $1024\pi$. Найдите радиус основания меньшего конуса. 
\end{taskBN}

\begin{taskBN}{300}
\addpictoright[0.25\textwidth]{images/349718255836607n0}В сосуде, имеющем форму конуса, уровень жидкости достигает $\frac{1}{3}$ высоты. Объём жидкости равен 81мл. Сколько миллилитров жидкости нужно долить, чтобы наполнить сосуд доверху?
\end{taskBN}
\end{document}