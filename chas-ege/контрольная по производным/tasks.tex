
\newpage\section{Вариант}\begin{taskBN}{1}
\addpictoright[0.5\linewidth]{контрольная по производным/images/258173862436623n0}На рисунке изображён график дифференцируемой функции $y=f(x)$. На оси абсцисс отмечены 5 точек : $x_1, x_2, x_3, \dots, x_5$. Среди этих точек найдите все точки, в которых производная функции $f(x)$ отрицательна. В ответе укажите количество найденных точек.
\end{taskBN}

\begin{taskBN}{2}
\addpictoright[0.5\linewidth]{контрольная по производным/images/008960363514145n0}На рисунке изображен график производной функции $f(x)$, определенной на интервале $(-7;8)$. Найдите промежутки возрастания функции $f(x)$. В ответе укажите длину наименьшего из них.
\end{taskBN}

\begin{taskBN}{3}
\addpictoright[0.5\linewidth]{контрольная по производным/images/322502834997053n0}На рисунке изображен график производной функции $f(x)$, определенной на интервале $(-4;8)$. Найдите количество точек минимума функции $f(x)$ на отрезке $[-1;4]$. 
\end{taskBN}

\begin{taskBN}{4}
\addpictoright[0.5\linewidth]{контрольная по производным/images/8663626748749598n0}На рисунке изображён график $y=f'(x)$ — производной функции $f(x)$, определенной на интервале $(-9;0)$. В какой точке отрезка $[-3; -2]$ функция $f(x)$ принимает наибольшее значение?
\end{taskBN}

\newpage\section{Вариант}\begin{taskBN}{1}
\addpictoright[0.5\linewidth]{контрольная по производным/images/551654738411846n0}На рисунке изображен график функции $y=f(x)$ и отмечены точки $6,5$; $1,7$; $3,8$; $-9$. В какой из этих точек значение производной наименьшее? В ответе укажите эту точку. 
\end{taskBN}

\begin{taskBN}{2}
\addpictoright[0.5\linewidth]{контрольная по производным/images/550119340985222n0}На рисунке изображен график производной функции $f(x)$, определенной на интервале $(-6;8)$. Найдите промежутки убывания функции $f(x)$. В ответе укажите длину наибольшего из них.
\end{taskBN}

\begin{taskBN}{3}
\addpictoright[0.5\linewidth]{контрольная по производным/images/1634918708692716n0}На рисунке изображен график функции $y=f(x)$, определенной на интервале $(-4;10)$. Найдите количество точек экстремума функции $f(x)$.
\end{taskBN}

\begin{taskBN}{4}
\addpictoright[0.5\linewidth]{контрольная по производным/images/624673053611057n0}На рисунке изображён график $y=f'(x)$ — производной функции $f(x)$, определенной на интервале $(-6;3)$. В какой точке отрезка $[-4; 1]$ функция $f(x)$ принимает наибольшее значение?
\end{taskBN}

\newpage\section{Вариант}\begin{taskBN}{1}
\addpictoright[0.5\linewidth]{контрольная по производным/images/4756710556320272n0}На рисунке изображен график функции $y=f(x)$ и отмечены точки $4,3$; $-9$; $7,6$; $-5,6$. В какой из этих точек значение производной наименьшее? В ответе укажите эту точку. 
\end{taskBN}

\begin{taskBN}{2}
\addpictoright[0.5\linewidth]{контрольная по производным/images/543949249419382n0}На рисунке изображен график производной функции $f(x)$, определенной на интервале$(-3; 9)$. Найдите промежутки убывания функции $f(x)$. В ответе укажите сумму целых точек, входящих в эти промежутки.
\end{taskBN}

\begin{taskBN}{3}
\addpictoright[0.5\linewidth]{контрольная по производным/images/726958125564379n0}На рисунке изображен график производной функции $f(x)$, определенной на интервале $(-3;9)$. Найдите количество точек минимума функции $f(x)$ на отрезке $[-2;7]$. 
\end{taskBN}

\begin{taskBN}{4}
\addpictoright[0.5\linewidth]{контрольная по производным/images/6402164947194924n0}На рисунке изображён график $y=f'(x)$ — производной функции $f(x)$, определенной на интервале $(-1;7)$. В какой точке отрезка $[4; 6]$ функция $f(x)$ принимает наибольшее значение?
\end{taskBN}

\newpage\section{Вариант}\begin{taskBN}{1}
\addpictoright[0.5\linewidth]{контрольная по производным/images/586722780022524n0}На рисунке изображён график дифференцируемой функции $y=f(x)$. На оси абсцисс отмечены 4 точки : $x_1, x_2, x_3, \dots, x_4$. Среди этих точек найдите все точки, в которых производная функции $f(x)$ отрицательна. В ответе укажите количество найденных точек.
\end{taskBN}

\begin{taskBN}{2}
\addpictoright[0.5\linewidth]{контрольная по производным/images/2999699555620476n0}На рисунке изображен график производной функции $f(x)$, определенной на интервале $(-9;8)$. Найдите промежутки возрастания функции $f(x)$. В ответе укажите длину наибольшего из них.
\end{taskBN}

\begin{taskBN}{3}
\addpictoright[0.5\linewidth]{контрольная по производным/images/5218249111147906n0}На рисунке изображен график производной функции $f(x)$, определенной на интервале $(-6;9)$. Найдите количество точек минимума функции $f(x)$ на отрезке $[-2;6]$. 
\end{taskBN}

\begin{taskBN}{4}
\addpictoright[0.5\linewidth]{контрольная по производным/images/194222218275446n0}На рисунке изображён график $y=f'(x)$ — производной функции $f(x)$, определенной на интервале $(-2;6)$. В какой точке отрезка $[2; 3]$ функция $f(x)$ принимает наибольшее значение?
\end{taskBN}

\newpage\section{Вариант}\begin{taskBN}{1}
\addpictoright[0.5\linewidth]{контрольная по производным/images/8440082135707772n0}На рисунке изображён график $y=f'(x)$ — производной функции $f(x)$. На оси абсцисс отмечены 6 точек: $x_1, x_2, x_3, \dots, x_6$. Сколько из этих точек лежит на промежутках возрастания функции $f(x)$?
\end{taskBN}

\begin{taskBN}{2}
\addpictoright[0.5\linewidth]{контрольная по производным/images/722304426342822n0}На рисунке изображен график производной функции $f(x)$, определенной на интервале $(-6;9)$. Найдите промежутки возрастания функции $f(x)$. В ответе укажите длину наименьшего из них.
\end{taskBN}

\begin{taskBN}{3}
\addpictoright[0.5\linewidth]{контрольная по производным/images/594571799149139n0}На рисунке изображен график производной функции $f(x)$, определенной на интервале $(-5;8)$. Найдите сумму точек экстремума функции $f(x)$ на отрезке $[-3;6]$.
\end{taskBN}

\begin{taskBN}{4}
\addpictoright[0.5\linewidth]{контрольная по производным/images/4806162558639058n0}На рисунке изображён график $y=f'(x)$ — производной функции $f(x)$, определенной на интервале $(-6;3)$. В какой точке отрезка $[1; 2]$ функция $f(x)$ принимает наибольшее значение?
\end{taskBN}

\newpage\section{Вариант}\begin{taskBN}{1}
\addpictoright[0.5\linewidth]{контрольная по производным/images/04235594556756n0}На рисунке изображен график функции $y=f(x)$ и отмечены точки $4,5$; $0,4$; $-9$; $-3,4$. В какой из этих точек значение производной наименьшее? В ответе укажите эту точку. 
\end{taskBN}

\begin{taskBN}{2}
\addpictoright[0.5\linewidth]{контрольная по производным/images/705088375836401n0}На рисунке изображен график производной функции $f(x)$, определенной на интервале$(0; 10)$. Найдите промежутки возрастания функции $f(x)$. В ответе укажите количество целых точек, входящих в эти промежутки.
\end{taskBN}

\begin{taskBN}{3}
\addpictoright[0.5\linewidth]{контрольная по производным/images/399386493982363n0}На рисунке изображен график производной функции $f(x)$, определенной на интервале $(-5;9)$. Найдите количество точек минимума функции $f(x)$ на отрезке $[-3;6]$. 
\end{taskBN}

\begin{taskBN}{4}
\addpictoright[0.5\linewidth]{контрольная по производным/images/379214398351169n0}На рисунке изображён график $y=f'(x)$ — производной функции $f(x)$, определенной на интервале $(-1;9)$. В какой точке отрезка $[4; 7]$ функция $f(x)$ принимает наибольшее значение?
\end{taskBN}

\newpage\section{Вариант}\begin{taskBN}{1}
\addpictoright[0.5\linewidth]{контрольная по производным/images/677765894805497n0}На рисунке изображен график функции $y=f(x)$ и отмечены точки $9,5$; $-3,9$; $4$; $-9$. В какой из этих точек значение производной наибольшая? В ответе укажите эту точку. 
\end{taskBN}

\begin{taskBN}{2}
\addpictoright[0.5\linewidth]{контрольная по производным/images/904827125390049n0}На рисунке изображен график производной функции $f(x)$, определенной на интервале $(-9;8)$. Найдите промежутки возрастания функции $f(x)$. В ответе укажите длину наибольшего из них.
\end{taskBN}

\begin{taskBN}{3}
\addpictoright[0.5\linewidth]{контрольная по производным/images/443646045201571n0}На рисунке изображен график функции $y=f(x)$, определенной на интервале $(-5;8)$. Найдите сумму точек экстремума функции $f(x)$.
\end{taskBN}

\begin{taskBN}{4}
\addpictoright[0.5\linewidth]{контрольная по производным/images/870153039793523n0}На рисунке изображён график $y=f'(x)$ — производной функции $f(x)$, определенной на интервале $(-5;4)$. В какой точке отрезка $[-3; 1]$ функция $f(x)$ принимает наибольшее значение?
\end{taskBN}

\newpage\section{Вариант}\begin{taskBN}{1}
\addpictoright[0.5\linewidth]{контрольная по производным/images/9946614189650935n0}На рисунке изображён график дифференцируемой функции $y=f(x)$. На оси абсцисс отмечены 5 точек : $x_1, x_2, x_3, \dots, x_5$. Среди этих точек найдите все точки, в которых производная функции $f(x)$ положительна. В ответе укажите количество найденных точек.
\end{taskBN}

\begin{taskBN}{2}
\addpictoright[0.5\linewidth]{контрольная по производным/images/2685718227178642n0}На рисунке изображен график производной функции $f(x)$, определенной на интервале$(-6; 5)$. Найдите промежутки возрастания функции $f(x)$. В ответе укажите сумму целых точек, входящих в эти промежутки.
\end{taskBN}

\begin{taskBN}{3}
\addpictoright[0.5\linewidth]{контрольная по производным/images/045244283824458n0}На рисунке изображен график производной функции $f(x)$, определенной на интервале $(-4;9)$. Найдите количество точек минимума функции $f(x)$ на отрезке $[-2;5]$. 
\end{taskBN}

\begin{taskBN}{4}
\addpictoright[0.5\linewidth]{контрольная по производным/images/883624704502203n0}На рисунке изображён график $y=f'(x)$ — производной функции $f(x)$, определенной на интервале $(-5;3)$. В какой точке отрезка $[-4; 0]$ функция $f(x)$ принимает наибольшее значение?
\end{taskBN}

\newpage\section{Вариант}\begin{taskBN}{1}
\addpictoright[0.5\linewidth]{контрольная по производным/images/4644273127925453n0}На рисунке изображен график функции $y=f(x)$ и отмечены точки $8,2$; $-0,7$; $-6,7$; $-4,1$. В какой из этих точек значение производной наибольшая? В ответе укажите эту точку. 
\end{taskBN}

\begin{taskBN}{2}
\addpictoright[0.5\linewidth]{контрольная по производным/images/009708906406699n0}На рисунке изображен график производной функции $f(x)$, определенной на интервале$(-5; 7)$. Найдите промежутки убывания функции $f(x)$. В ответе укажите произведение целых точек, входящих в эти промежутки.
\end{taskBN}

\begin{taskBN}{3}
\addpictoright[0.5\linewidth]{контрольная по производным/images/5097202144457n0}На рисунке изображен график производной функции $f(x)$, определенной на интервале $(-5;9)$. Найдите количество точек минимума функции $f(x)$ на отрезке $[-3;5]$. 
\end{taskBN}

\begin{taskBN}{4}
\addpictoright[0.5\linewidth]{контрольная по производным/images/765313042663516n0}На рисунке изображён график $y=f'(x)$ — производной функции $f(x)$, определенной на интервале $(-7;2)$. В какой точке отрезка $[-2; 1]$ функция $f(x)$ принимает наибольшее значение?
\end{taskBN}

\newpage\section{Вариант}\begin{taskBN}{1}
\addpictoright[0.5\linewidth]{контрольная по производным/images/310721861879657n0}На рисунке изображён график дифференцируемой функции $y=f(x)$. На оси абсцисс отмечены 6 точек : $x_1, x_2, x_3, \dots, x_6$. Среди этих точек найдите все точки, в которых производная функции $f(x)$ положительна. В ответе укажите количество найденных точек.
\end{taskBN}

\begin{taskBN}{2}
\addpictoright[0.5\linewidth]{контрольная по производным/images/6588451188792617n0}На рисунке изображен график производной функции $f(x)$, определенной на интервале$(-6; 7)$. Найдите промежутки возрастания функции $f(x)$. В ответе укажите сумму целых точек, входящих в эти промежутки.
\end{taskBN}

\begin{taskBN}{3}
\addpictoright[0.5\linewidth]{контрольная по производным/images/5536455917538503n0}На рисунке изображен график производной функции $f(x)$, определенной на интервале $(-8;9)$. Найдите количество точек экстремума функции $f(x)$ на отрезке $[-5;6]$.
\end{taskBN}

\begin{taskBN}{4}
\addpictoright[0.5\linewidth]{контрольная по производным/images/133053691864n0}На рисунке изображён график $y=f'(x)$ — производной функции $f(x)$, определенной на интервале $(-1;8)$. В какой точке отрезка $[2; 6]$ функция $f(x)$ принимает наибольшее значение?
\end{taskBN}

\newpage\section{Вариант}\begin{taskBN}{1}
\addpictoright[0.5\linewidth]{контрольная по производным/images/548172561457917n0}На рисунке изображен график функции $y=f(x)$ и отмечены точки $-9$; $-3,5$; $0,3$; $9,7$. В какой из этих точек значение производной наибольшая? В ответе укажите эту точку. 
\end{taskBN}

\begin{taskBN}{2}
\addpictoright[0.5\linewidth]{контрольная по производным/images/6132511196053747n0}На рисунке изображен график производной функции $f(x)$, определенной на интервале $(-3;10)$. Найдите промежутки убывания функции $f(x)$. В ответе укажите длину наименьшего из них.
\end{taskBN}

\begin{taskBN}{3}
\addpictoright[0.5\linewidth]{контрольная по производным/images/312817555476926n0}На рисунке изображен график производной функции $f(x)$, определенной на интервале $(-5;9)$. Найдите количество точек максимума функции $f(x)$ на отрезке $[-3;7]$. 
\end{taskBN}

\begin{taskBN}{4}
\addpictoright[0.5\linewidth]{контрольная по производным/images/173631255709561n0}На рисунке изображён график $y=f'(x)$ — производной функции $f(x)$, определенной на интервале $(-5;3)$. В какой точке отрезка $[-4; 2]$ функция $f(x)$ принимает наибольшее значение?
\end{taskBN}

\newpage\section{Вариант}\begin{taskBN}{1}
\addpictoright[0.5\linewidth]{контрольная по производным/images/917208706462193n0}На рисунке изображен график функции $y=f(x)$ и отмечены точки $-9$; $-5$; $-2,6$; $6,3$. В какой из этих точек значение производной наименьшее? В ответе укажите эту точку. 
\end{taskBN}

\begin{taskBN}{2}
\addpictoright[0.5\linewidth]{контрольная по производным/images/652471389428479n0}На рисунке изображен график производной функции $f(x)$, определенной на интервале$(-5; 5)$. Найдите промежутки возрастания функции $f(x)$. В ответе укажите сумму целых точек, входящих в эти промежутки.
\end{taskBN}

\begin{taskBN}{3}
\addpictoright[0.5\linewidth]{контрольная по производным/images/745659507295777n0}На рисунке изображен график функции $y=f(x)$, определенной на интервале $(-7;9)$. Найдите количество точек экстремума функции $f(x)$.
\end{taskBN}

\begin{taskBN}{4}
\addpictoright[0.5\linewidth]{контрольная по производным/images/7984374840754824n0}На рисунке изображён график $y=f'(x)$ — производной функции $f(x)$, определенной на интервале $(-8;2)$. В какой точке отрезка $[-5; -3]$ функция $f(x)$ принимает наибольшее значение?
\end{taskBN}

\newpage\section{Вариант}\begin{taskBN}{1}
\addpictoright[0.5\linewidth]{контрольная по производным/images/521038083647789n0}На рисунке изображён график дифференцируемой функции $y=f(x)$. На оси абсцисс отмечены 5 точек : $x_1, x_2, x_3, \dots, x_5$. Среди этих точек найдите все точки, в которых производная функции $f(x)$ отрицательна. В ответе укажите количество найденных точек.
\end{taskBN}

\begin{taskBN}{2}
\addpictoright[0.5\linewidth]{контрольная по производным/images/3419468729585584n0}На рисунке изображен график производной функции $f(x)$, определенной на интервале$(-3; 9)$. Найдите промежутки убывания функции $f(x)$. В ответе укажите количество целых точек, входящих в эти промежутки.
\end{taskBN}

\begin{taskBN}{3}
\addpictoright[0.5\linewidth]{контрольная по производным/images/771603716176323n0}На рисунке изображен график производной функции $f(x)$, определенной на интервале $(-5;10)$. Найдите сумму точек экстремума функции $f(x)$ на отрезке $[-1;8]$.
\end{taskBN}

\begin{taskBN}{4}
\addpictoright[0.5\linewidth]{контрольная по производным/images/3130248442107846n0}На рисунке изображён график $y=f'(x)$ — производной функции $f(x)$, определенной на интервале $(-3;7)$. В какой точке отрезка $[4; 6]$ функция $f(x)$ принимает наибольшее значение?
\end{taskBN}

\newpage\section{Вариант}\begin{taskBN}{1}
\addpictoright[0.5\linewidth]{контрольная по производным/images/660979261798655n0}На рисунке изображён график дифференцируемой функции $y=f(x)$. На оси абсцисс отмечены 4 точки : $x_1, x_2, x_3, \dots, x_4$. Среди этих точек найдите все точки, в которых производная функции $f(x)$ отрицательна. В ответе укажите количество найденных точек.
\end{taskBN}

\begin{taskBN}{2}
\addpictoright[0.5\linewidth]{контрольная по производным/images/24195837428892n0}На рисунке изображен график производной функции $f(x)$, определенной на интервале $(-5;9)$. Найдите промежутки возрастания функции $f(x)$. В ответе укажите длину наибольшего из них.
\end{taskBN}

\begin{taskBN}{3}
\addpictoright[0.5\linewidth]{контрольная по производным/images/436267612835267n0}На рисунке изображен график производной функции $f(x)$, определенной на интервале $(-5;9)$. Найдите количество точек максимума функции $f(x)$ на отрезке $[-3;5]$. 
\end{taskBN}

\begin{taskBN}{4}
\addpictoright[0.5\linewidth]{контрольная по производным/images/6161613525937357n0}На рисунке изображён график $y=f'(x)$ — производной функции $f(x)$, определенной на интервале $(-6;4)$. В какой точке отрезка $[-5; 3]$ функция $f(x)$ принимает наибольшее значение?
\end{taskBN}

\newpage\section{Вариант}\begin{taskBN}{1}
\addpictoright[0.5\linewidth]{контрольная по производным/images/769241454098192n0}На рисунке изображен график функции $y=f(x)$ и отмечены точки $-3,9$; $-9$; $1,6$; $7,4$. В какой из этих точек значение производной наименьшее? В ответе укажите эту точку. 
\end{taskBN}

\begin{taskBN}{2}
\addpictoright[0.5\linewidth]{контрольная по производным/images/342888391222708n0}На рисунке изображен график производной функции $f(x)$, определенной на интервале $(-9;8)$. Найдите промежутки возрастания функции $f(x)$. В ответе укажите длину наибольшего из них.
\end{taskBN}

\begin{taskBN}{3}
\addpictoright[0.5\linewidth]{контрольная по производным/images/79477733159526n0}На рисунке изображен график функции $y=f(x)$, определенной на интервале $(-6;10)$. Найдите сумму точек экстремума функции $f(x)$.
\end{taskBN}

\begin{taskBN}{4}
\addpictoright[0.5\linewidth]{контрольная по производным/images/424706190673188n0}На рисунке изображён график $y=f'(x)$ — производной функции $f(x)$, определенной на интервале $(-6;3)$. В какой точке отрезка $[-3; 2]$ функция $f(x)$ принимает наибольшее значение?
\end{taskBN}

\newpage\section{Вариант}\begin{taskBN}{1}
\addpictoright[0.5\linewidth]{контрольная по производным/images/025568800964449n0}На рисунке изображён график дифференцируемой функции $y=f(x)$. На оси абсцисс отмечены 6 точек : $x_1, x_2, x_3, \dots, x_6$. Среди этих точек найдите все точки, в которых производная функции $f(x)$ отрицательна. В ответе укажите количество найденных точек.
\end{taskBN}

\begin{taskBN}{2}
\addpictoright[0.5\linewidth]{контрольная по производным/images/358540787720363n0}На рисунке изображен график производной функции $f(x)$, определенной на интервале $(-4;9)$. Найдите промежутки возрастания функции $f(x)$. В ответе укажите длину наименьшего из них.
\end{taskBN}

\begin{taskBN}{3}
\addpictoright[0.5\linewidth]{контрольная по производным/images/117294468531446n0}На рисунке изображен график производной функции $f(x)$, определенной на интервале $(-6;8)$. Найдите количество точек максимума функции $f(x)$ на отрезке $[-4;7]$. 
\end{taskBN}

\begin{taskBN}{4}
\addpictoright[0.5\linewidth]{контрольная по производным/images/36932557864333n0}На рисунке изображён график $y=f'(x)$ — производной функции $f(x)$, определенной на интервале $(-3;5)$. В какой точке отрезка $[-2; 3]$ функция $f(x)$ принимает наибольшее значение?
\end{taskBN}

\newpage\section{Вариант}\begin{taskBN}{1}
\addpictoright[0.5\linewidth]{контрольная по производным/images/973589702412912n0}На рисунке изображён график дифференцируемой функции $y=f(x)$. На оси абсцисс отмечены 6 точек : $x_1, x_2, x_3, \dots, x_6$. Среди этих точек найдите все точки, в которых производная функции $f(x)$ отрицательна. В ответе укажите количество найденных точек.
\end{taskBN}

\begin{taskBN}{2}
\addpictoright[0.5\linewidth]{контрольная по производным/images/9464387395722966n0}На рисунке изображен график производной функции $f(x)$, определенной на интервале$(-4; 7)$. Найдите промежутки возрастания функции $f(x)$. В ответе укажите произведение целых точек, входящих в эти промежутки.
\end{taskBN}

\begin{taskBN}{3}
\addpictoright[0.5\linewidth]{контрольная по производным/images/228826159509755n0}На рисунке изображен график производной функции $f(x)$, определенной на интервале $(-1;10)$. Найдите количество точек минимума функции $f(x)$ на отрезке $[2;8]$. 
\end{taskBN}

\begin{taskBN}{4}
\addpictoright[0.5\linewidth]{контрольная по производным/images/166745099703823n0}На рисунке изображён график $y=f'(x)$ — производной функции $f(x)$, определенной на интервале $(-8;1)$. В какой точке отрезка $[-4; 0]$ функция $f(x)$ принимает наибольшее значение?
\end{taskBN}

\newpage\section{Вариант}\begin{taskBN}{1}
\addpictoright[0.5\linewidth]{контрольная по производным/images/6696759908037146n0}На рисунке изображён график дифференцируемой функции $y=f(x)$. На оси абсцисс отмечены 6 точек : $x_1, x_2, x_3, \dots, x_6$. Среди этих точек найдите все точки, в которых производная функции $f(x)$ отрицательна. В ответе укажите количество найденных точек.
\end{taskBN}

\begin{taskBN}{2}
\addpictoright[0.5\linewidth]{контрольная по производным/images/110405703364792n0}На рисунке изображен график производной функции $f(x)$, определенной на интервале $(-5;9)$. Найдите промежутки убывания функции $f(x)$. В ответе укажите длину наибольшего из них.
\end{taskBN}

\begin{taskBN}{3}
\addpictoright[0.5\linewidth]{контрольная по производным/images/462739530046981n0}На рисунке изображен график производной функции $f(x)$, определенной на интервале $(-3;10)$. Найдите количество точек максимума функции $f(x)$ на отрезке $[0;8]$. 
\end{taskBN}

\begin{taskBN}{4}
\addpictoright[0.5\linewidth]{контрольная по производным/images/710792659947677n0}На рисунке изображён график $y=f'(x)$ — производной функции $f(x)$, определенной на интервале $(-7;3)$. В какой точке отрезка $[-4; 3]$ функция $f(x)$ принимает наибольшее значение?
\end{taskBN}

\newpage\section{Вариант}\begin{taskBN}{1}
\addpictoright[0.5\linewidth]{контрольная по производным/images/769883924787129n0}На рисунке изображен график функции $y=f(x)$ и отмечены точки $-9$; $7,8$; $1,5$; $4,5$. В какой из этих точек значение производной наименьшее? В ответе укажите эту точку. 
\end{taskBN}

\begin{taskBN}{2}
\addpictoright[0.5\linewidth]{контрольная по производным/images/325620658553047n0}На рисунке изображен график производной функции $f(x)$, определенной на интервале $(-8;9)$. Найдите промежутки возрастания функции $f(x)$. В ответе укажите длину наибольшего из них.
\end{taskBN}

\begin{taskBN}{3}
\addpictoright[0.5\linewidth]{контрольная по производным/images/12305627730663793n0}На рисунке изображен график производной функции $f(x)$, определенной на интервале $(0;9)$. Найдите количество точек максимума функции $f(x)$ на отрезке $[3;7]$. 
\end{taskBN}

\begin{taskBN}{4}
\addpictoright[0.5\linewidth]{контрольная по производным/images/278115376533037n0}На рисунке изображён график $y=f'(x)$ — производной функции $f(x)$, определенной на интервале $(-7;3)$. В какой точке отрезка $[-6; -5]$ функция $f(x)$ принимает наибольшее значение?
\end{taskBN}

\newpage\section{Вариант}\begin{taskBN}{1}
\addpictoright[0.5\linewidth]{контрольная по производным/images/859142460002727n0}На рисунке изображён график дифференцируемой функции $y=f(x)$. На оси абсцисс отмечены 6 точек : $x_1, x_2, x_3, \dots, x_6$. Среди этих точек найдите все точки, в которых производная функции $f(x)$ положительна. В ответе укажите количество найденных точек.
\end{taskBN}

\begin{taskBN}{2}
\addpictoright[0.5\linewidth]{контрольная по производным/images/997341981774047n0}На рисунке изображен график производной функции $f(x)$, определенной на интервале $(-6;9)$. Найдите промежутки возрастания функции $f(x)$. В ответе укажите длину наибольшего из них.
\end{taskBN}

\begin{taskBN}{3}
\addpictoright[0.5\linewidth]{контрольная по производным/images/04746489777399776n0}На рисунке изображен график производной функции $f(x)$, определенной на интервале $(-3;9)$. Найдите количество точек экстремума функции $f(x)$ на отрезке $[-1;5]$.
\end{taskBN}

\begin{taskBN}{4}
\addpictoright[0.5\linewidth]{контрольная по производным/images/2104688995932684n0}На рисунке изображён график $y=f'(x)$ — производной функции $f(x)$, определенной на интервале $(-8;2)$. В какой точке отрезка $[0; 1]$ функция $f(x)$ принимает наибольшее значение?
\end{taskBN}

\newpage\section{Вариант}\begin{taskBN}{1}
\addpictoright[0.5\linewidth]{контрольная по производным/images/0762846190542625n0}На рисунке изображен график функции $y=f(x)$ и отмечены точки $-4,7$; $3,8$; $-9$; $6,5$. В какой из этих точек значение производной наибольшая? В ответе укажите эту точку. 
\end{taskBN}

\begin{taskBN}{2}
\addpictoright[0.5\linewidth]{контрольная по производным/images/772695010916505n0}На рисунке изображен график производной функции $f(x)$, определенной на интервале$(-6; 8)$. Найдите промежутки убывания функции $f(x)$. В ответе укажите произведение целых точек, входящих в эти промежутки.
\end{taskBN}

\begin{taskBN}{3}
\addpictoright[0.5\linewidth]{контрольная по производным/images/556461740738307n0}На рисунке изображен график производной функции $f(x)$, определенной на интервале $(-1;9)$. Найдите точку экстремума функции $f(x)$ на отрезке $[3;6]$.
\end{taskBN}

\begin{taskBN}{4}
\addpictoright[0.5\linewidth]{контрольная по производным/images/2843420089104027n0}На рисунке изображён график $y=f'(x)$ — производной функции $f(x)$, определенной на интервале $(-8;1)$. В какой точке отрезка $[-7; 0]$ функция $f(x)$ принимает наибольшее значение?
\end{taskBN}

\newpage\section{Вариант}\begin{taskBN}{1}
\addpictoright[0.5\linewidth]{контрольная по производным/images/296909925285109n0}На рисунке изображен график функции $y=f(x)$ и отмечены точки $0,4$; $-9$; $-2,1$; $5,7$. В какой из этих точек значение производной наибольшая? В ответе укажите эту точку. 
\end{taskBN}

\begin{taskBN}{2}
\addpictoright[0.5\linewidth]{контрольная по производным/images/42165002585022315n0}На рисунке изображен график производной функции $f(x)$, определенной на интервале $(-6;10)$. Найдите промежутки убывания функции $f(x)$. В ответе укажите длину наибольшего из них.
\end{taskBN}

\begin{taskBN}{3}
\addpictoright[0.5\linewidth]{контрольная по производным/images/562353859751829n0}На рисунке изображен график производной функции $f(x)$, определенной на интервале $(-7;9)$. Найдите количество точек экстремума функции $f(x)$ на отрезке $[-4;7]$.
\end{taskBN}

\begin{taskBN}{4}
\addpictoright[0.5\linewidth]{контрольная по производным/images/0013573553832766n0}На рисунке изображён график $y=f'(x)$ — производной функции $f(x)$, определенной на интервале $(-6;3)$. В какой точке отрезка $[-5; 2]$ функция $f(x)$ принимает наибольшее значение?
\end{taskBN}

\newpage\section{Вариант}\begin{taskBN}{1}
\addpictoright[0.5\linewidth]{контрольная по производным/images/19940137321983187n0}На рисунке изображен график функции $y=f(x)$ и отмечены точки $-9$; $9,5$; $1,1$; $5$. В какой из этих точек значение производной наибольшая? В ответе укажите эту точку. 
\end{taskBN}

\begin{taskBN}{2}
\addpictoright[0.5\linewidth]{контрольная по производным/images/0067700345252946n0}На рисунке изображен график производной функции $f(x)$, определенной на интервале $(-6;10)$. Найдите промежутки возрастания функции $f(x)$. В ответе укажите длину наименьшего из них.
\end{taskBN}

\begin{taskBN}{3}
\addpictoright[0.5\linewidth]{контрольная по производным/images/482635182635817n0}На рисунке изображен график производной функции $f(x)$, определенной на интервале $(-3;9)$. Найдите количество точек экстремума функции $f(x)$ на отрезке $[0;7]$.
\end{taskBN}

\begin{taskBN}{4}
\addpictoright[0.5\linewidth]{контрольная по производным/images/5772758469647121n0}На рисунке изображён график $y=f'(x)$ — производной функции $f(x)$, определенной на интервале $(-2;6)$. В какой точке отрезка $[1; 4]$ функция $f(x)$ принимает наибольшее значение?
\end{taskBN}

\newpage\section{Вариант}\begin{taskBN}{1}
\addpictoright[0.5\linewidth]{контрольная по производным/images/259934613054598n0}На рисунке изображен график функции $y=f(x)$ и отмечены точки $3,9$; $-9$; $9,6$; $-6$. В какой из этих точек значение производной наименьшее? В ответе укажите эту точку. 
\end{taskBN}

\begin{taskBN}{2}
\addpictoright[0.5\linewidth]{контрольная по производным/images/0019443379924065n0}На рисунке изображен график производной функции $f(x)$, определенной на интервале $(-6;8)$. Найдите промежутки возрастания функции $f(x)$. В ответе укажите длину наибольшего из них.
\end{taskBN}

\begin{taskBN}{3}
\addpictoright[0.5\linewidth]{контрольная по производным/images/52312788260087n0}На рисунке изображен график функции $y=f(x)$, определенной на интервале $(-3;10)$. Найдите количество точек экстремума функции $f(x)$.
\end{taskBN}

\begin{taskBN}{4}
\addpictoright[0.5\linewidth]{контрольная по производным/images/679319521950374n0}На рисунке изображён график $y=f'(x)$ — производной функции $f(x)$, определенной на интервале $(-7;3)$. В какой точке отрезка $[-2; 2]$ функция $f(x)$ принимает наибольшее значение?
\end{taskBN}

\newpage\section{Вариант}\begin{taskBN}{1}
\addpictoright[0.5\linewidth]{контрольная по производным/images/693695563508764n0}На рисунке изображен график функции $y=f(x)$ и отмечены точки $6,2$; $8,6$; $3,4$; $-9$. В какой из этих точек значение производной наибольшая? В ответе укажите эту точку. 
\end{taskBN}

\begin{taskBN}{2}
\addpictoright[0.5\linewidth]{контрольная по производным/images/715434612838611n0}На рисунке изображен график производной функции $f(x)$, определенной на интервале $(-5;10)$. Найдите промежутки убывания функции $f(x)$. В ответе укажите длину наибольшего из них.
\end{taskBN}

\begin{taskBN}{3}
\addpictoright[0.5\linewidth]{контрольная по производным/images/4036519184232044n0}На рисунке изображен график производной функции $f(x)$, определенной на интервале $(-4;9)$. Найдите количество точек минимума функции $f(x)$ на отрезке $[-1;7]$. 
\end{taskBN}

\begin{taskBN}{4}
\addpictoright[0.5\linewidth]{контрольная по производным/images/402992476638629n0}На рисунке изображён график $y=f'(x)$ — производной функции $f(x)$, определенной на интервале $(-5;5)$. В какой точке отрезка $[-1; 2]$ функция $f(x)$ принимает наибольшее значение?
\end{taskBN}

\newpage\section{Вариант}\begin{taskBN}{1}
\addpictoright[0.5\linewidth]{контрольная по производным/images/874255559532753n0}На рисунке изображён график дифференцируемой функции $y=f(x)$. На оси абсцисс отмечены 6 точек : $x_1, x_2, x_3, \dots, x_6$. Среди этих точек найдите все точки, в которых производная функции $f(x)$ отрицательна. В ответе укажите количество найденных точек.
\end{taskBN}

\begin{taskBN}{2}
\addpictoright[0.5\linewidth]{контрольная по производным/images/6090804627672273n0}На рисунке изображен график производной функции $f(x)$, определенной на интервале $(-4;9)$. Найдите промежутки возрастания функции $f(x)$. В ответе укажите длину наименьшего из них.
\end{taskBN}

\begin{taskBN}{3}
\addpictoright[0.5\linewidth]{контрольная по производным/images/247365946916637n0}На рисунке изображен график производной функции $f(x)$, определенной на интервале $(-1;9)$. Найдите сумму точек экстремума функции $f(x)$ на отрезке $[1;7]$.
\end{taskBN}

\begin{taskBN}{4}
\addpictoright[0.5\linewidth]{контрольная по производным/images/10725865052653n0}На рисунке изображён график $y=f'(x)$ — производной функции $f(x)$, определенной на интервале $(1;9)$. В какой точке отрезка $[2; 6]$ функция $f(x)$ принимает наибольшее значение?
\end{taskBN}

\newpage\section{Вариант}\begin{taskBN}{1}
\addpictoright[0.5\linewidth]{контрольная по производным/images/358671662610089n0}На рисунке изображён график дифференцируемой функции $y=f(x)$. На оси абсцисс отмечены 6 точек : $x_1, x_2, x_3, \dots, x_6$. Среди этих точек найдите все точки, в которых производная функции $f(x)$ отрицательна. В ответе укажите количество найденных точек.
\end{taskBN}

\begin{taskBN}{2}
\addpictoright[0.5\linewidth]{контрольная по производным/images/0159029847580534n0}На рисунке изображен график производной функции $f(x)$, определенной на интервале$(-6; 6)$. Найдите промежутки убывания функции $f(x)$. В ответе укажите сумму целых точек, входящих в эти промежутки.
\end{taskBN}

\begin{taskBN}{3}
\addpictoright[0.5\linewidth]{контрольная по производным/images/7089396408165765n0}На рисунке изображен график производной функции $f(x)$, определенной на интервале $(-6;9)$. Найдите сумму точек экстремума функции $f(x)$ на отрезке $[-3;8]$.
\end{taskBN}

\begin{taskBN}{4}
\addpictoright[0.5\linewidth]{контрольная по производным/images/695653743826874n0}На рисунке изображён график $y=f'(x)$ — производной функции $f(x)$, определенной на интервале $(-1;8)$. В какой точке отрезка $[2; 4]$ функция $f(x)$ принимает наибольшее значение?
\end{taskBN}

\newpage\section{Вариант}\begin{taskBN}{1}
\addpictoright[0.5\linewidth]{контрольная по производным/images/8699143355971877n0}На рисунке изображён график дифференцируемой функции $y=f(x)$. На оси абсцисс отмечены 6 точек : $x_1, x_2, x_3, \dots, x_6$. Среди этих точек найдите все точки, в которых производная функции $f(x)$ отрицательна. В ответе укажите количество найденных точек.
\end{taskBN}

\begin{taskBN}{2}
\addpictoright[0.5\linewidth]{контрольная по производным/images/7834583635817656n0}На рисунке изображен график производной функции $f(x)$, определенной на интервале $(-7;9)$. Найдите промежутки возрастания функции $f(x)$. В ответе укажите длину наименьшего из них.
\end{taskBN}

\begin{taskBN}{3}
\addpictoright[0.5\linewidth]{контрольная по производным/images/610401721498074n0}На рисунке изображен график производной функции $f(x)$, определенной на интервале $(-2;10)$. Найдите количество точек экстремума функции $f(x)$ на отрезке $[1;8]$.
\end{taskBN}

\begin{taskBN}{4}
\addpictoright[0.5\linewidth]{контрольная по производным/images/9991051642815567n0}На рисунке изображён график $y=f'(x)$ — производной функции $f(x)$, определенной на интервале $(-1;8)$. В какой точке отрезка $[0; 3]$ функция $f(x)$ принимает наибольшее значение?
\end{taskBN}

\newpage\section{Вариант}\begin{taskBN}{1}
\addpictoright[0.5\linewidth]{контрольная по производным/images/515427702411263n0}На рисунке изображён график $y=f'(x)$ — производной функции $f(x)$. На оси абсцисс отмечены 7 точек: $x_1, x_2, x_3, \dots, x_7$. Сколько из этих точек лежит на промежутках возрастания функции $f(x)$?
\end{taskBN}

\begin{taskBN}{2}
\addpictoright[0.5\linewidth]{контрольная по производным/images/460928433712756n0}На рисунке изображен график производной функции $f(x)$, определенной на интервале $(-8;9)$. Найдите промежутки возрастания функции $f(x)$. В ответе укажите длину наибольшего из них.
\end{taskBN}

\begin{taskBN}{3}
\addpictoright[0.5\linewidth]{контрольная по производным/images/687917415195213n0}На рисунке изображен график производной функции $f(x)$, определенной на интервале $(-1;8)$. Найдите количество точек минимума функции $f(x)$ на отрезке $[2;5]$. 
\end{taskBN}

\begin{taskBN}{4}
\addpictoright[0.5\linewidth]{контрольная по производным/images/7938404506727383n0}На рисунке изображён график $y=f'(x)$ — производной функции $f(x)$, определенной на интервале $(-8;2)$. В какой точке отрезка $[-7; -6]$ функция $f(x)$ принимает наибольшее значение?
\end{taskBN}

\newpage\section{Вариант}\begin{taskBN}{1}
\addpictoright[0.5\linewidth]{контрольная по производным/images/1208296655500649n0}На рисунке изображен график функции $y=f(x)$ и отмечены точки $-9$; $5$; $-3,8$; $8,3$. В какой из этих точек значение производной наименьшее? В ответе укажите эту точку. 
\end{taskBN}

\begin{taskBN}{2}
\addpictoright[0.5\linewidth]{контрольная по производным/images/602755582779174n0}На рисунке изображен график производной функции $f(x)$, определенной на интервале $(-5;8)$. Найдите промежутки возрастания функции $f(x)$. В ответе укажите длину наименьшего из них.
\end{taskBN}

\begin{taskBN}{3}
\addpictoright[0.5\linewidth]{контрольная по производным/images/717644011923755n0}На рисунке изображен график функции $y=f(x)$, определенной на интервале $(-4;9)$. Найдите количество точек экстремума функции $f(x)$.
\end{taskBN}

\begin{taskBN}{4}
\addpictoright[0.5\linewidth]{контрольная по производным/images/237353569310933n0}На рисунке изображён график $y=f'(x)$ — производной функции $f(x)$, определенной на интервале $(-3;7)$. В какой точке отрезка $[-2; 0]$ функция $f(x)$ принимает наибольшее значение?
\end{taskBN}
\newpage
\begin{tabular}{lll}
    \begin{tabular}[t]{|l|l|l|}
    \hline 1 & 1 & 2\\\hline1 & 2 & 3\\\hline1 & 3 & 0\\\hline1 & 4 & -2\\\hline2 & 1 & -9\\\hline2 & 2 & 3\\\hline2 & 3 & 5\\\hline2 & 4 & 1\\\hline3 & 1 & -5,6\\\hline3 & 2 & 27\\\hline3 & 3 & 2\\\hline3 & 4 & 4\\\hline4 & 1 & 2\\\hline4 & 2 & 4\\\hline4 & 3 & 2\\\hline4 & 4 & 2\\\hline5 & 1 & 2\\\hline5 & 2 & 2\\\hline5 & 3 & 4\\\hline5 & 4 & 2\\\hline6 & 1 & -3,4\\\hline6 & 2 & 3\\\hline6 & 3 & 0\\\hline6 & 4 & 7\\\hline7 & 1 & 9,5\\\hline7 & 2 & 3\\\hline7 & 3 & 3\\\hline7 & 4 & 1\\\hline8 & 1 & 2\\\hline8 & 2 & -1\\\hline8 & 3 & 1\\\hline8 & 4 & -4\\\hline9 & 1 & -0,7\\\hline9 & 2 & 6\\\hline9 & 3 & 2\\\hline9 & 4 & -2
    \\\hline
    \end{tabular}
    \begin{tabular}[t]{|l|l|l|}
    \hline10 & 1 & 2\\\hline10 & 2 & -4\\\hline10 & 3 & 4\\\hline10 & 4 & 2\\\hline11 & 1 & -9\\\hline11 & 2 & 2\\\hline11 & 3 & 0\\\hline11 & 4 & -4\\\hline12 & 1 & -2,6\\\hline12 & 2 & 1\\\hline12 & 3 & 4\\\hline12 & 4 & -5\\\hline13 & 1 & 2\\\hline13 & 2 & 5\\\hline13 & 3 & 9\\\hline13 & 4 & 6\\\hline14 & 1 & 2\\\hline14 & 2 & 4\\\hline14 & 3 & 0\\\hline14 & 4 & -5\\\hline15 & 1 & 1,6\\\hline15 & 2 & 3\\\hline15 & 3 & 5\\\hline15 & 4 & -3\\\hline16 & 1 & 4\\\hline16 & 2 & 3\\\hline16 & 3 & 2\\\hline16 & 4 & -2\\\hline17 & 1 & 3\\\hline17 & 2 & 24\\\hline17 & 3 & 1\\\hline17 & 4 & -4\\\hline18 & 1 & 2\\\hline18 & 2 & 4\\\hline18 & 3 & 1\\\hline18 & 4 & 3\\\hline19 & 1 & 4,5\\\hline19 & 2 & 4\\\hline19 & 3 & 0\\\hline19 & 4 & -6
    \\\hline
    \end{tabular}
    \begin{tabular}[t]{|l|l|l|}
        \hline20 & 1 & 2\\\hline20 & 2 & 4\\\hline20 & 3 & 2\\\hline20 & 4 & 1\\\hline21 & 1 & 3,8\\\hline21 & 2 & 120\\\hline21 & 3 & 4\\\hline21 & 4 & -7\\\hline22 & 1 & 0,4\\\hline22 & 2 & 4\\\hline22 & 3 & 3\\\hline22 & 4 & -5\\\hline23 & 1 & 9,5\\\hline23 & 2 & 1\\\hline23 & 3 & 2\\\hline23 & 4 & 4\\\hline24 & 1 & -6\\\hline24 & 2 & 3\\\hline24 & 3 & 4\\\hline24 & 4 & -2\\\hline25 & 1 & 8,6\\\hline25 & 2 & 4\\\hline25 & 3 & 2\\\hline25 & 4 & 2\\\hline26 & 1 & 1\\\hline26 & 2 & 1\\\hline26 & 3 & 10\\\hline26 & 4 & 6\\\hline27 & 1 & 2\\\hline27 & 2 & -6\\\hline27 & 3 & 12\\\hline27 & 4 & 4\\\hline28 & 1 & 2\\\hline28 & 2 & 2\\\hline28 & 3 & 2\\\hline28 & 4 & 0\\\hline29 & 1 & 6\\\hline29 & 2 & 4\\\hline29 & 3 & 1\\\hline29 & 4 & -7\\\hline30 & 1 & 8,3\\\hline30 & 2 & 1\\\hline30 & 3 & 4\\\hline30 & 4 & 0
        \\\hline
    \end{tabular}\end{tabular}


