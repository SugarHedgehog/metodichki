\documentclass[a4paper,12pt]{article}
\usepackage[russian]{babel}
\usepackage[utf8]{inputenc}
\usepackage[T2A]{fontenc}
\usepackage{amsmath}
\usepackage{graphicx}
\usepackage{amssymb}

\title{Обобщенная схема радиотехнического генератора. Уравнение Ван-дер-Поля}
\author{}
\date{}

\begin{document}

\maketitle

\section{Обобщенная схема радиотехнического генератора}

Схема радиотехнического генератора представлена на рис. 11.2. Основные структурные элементы включают RLC-контур, который служит колебательной системой. Напряжение с контура подается на вход активного элемента — усилителя. Нелинейная характеристика усилителя аппроксимируется кубическим полиномом:

\begin{equation}
i(u) = g_0 u - g_2 u^3 + \ldots
\end{equation}

где коэффициенты $g_n$ положительны. Выход усилителя нагружен на катушку индуктивности $L_1$, которая индуктивно связана с катушкой контура, обеспечивая обратную связь.


\subsection{Механизм возбуждения автоколебаний}

Даже при отсутствии напряжения на выходе усилителя напряжение в контуре испытывает случайные флуктуации, которые усиливаются и вновь поступают в контур через цепь обратной связи. Если энергия, вносимая в контур, превосходит энергию потерь, амплитуда колебаний нарастает. Коэффициент усиления должен быть достаточно велик, но нелинейность приводит к установлению стационарных автоколебаний с постоянной амплитудой.

\subsection{Дифференциальное уравнение генератора}

Запишем уравнения Кирхгофа для контура:

\begin{equation}
L \frac{dI}{dt} + RI + u = M \frac{di(u)}{dt}
\end{equation}

\begin{equation}
u = \frac{1}{C} \int I \, dt
\end{equation}

Из этих уравнений с учетом выражения для нелинейной характеристики можно получить:

\begin{equation}
\frac{d^2 u}{dt^2} - \omega_0^2 \left( \frac{Mg_0 - RC - 3Mg_2 u^2}{2} \right) \frac{du}{dt} + \omega_0^2 u = 0
\end{equation}

где $\omega_0 = \frac{1}{\sqrt{LC}}$ — собственная частота колебательного контура. Это уравнение называется уравнением Ван-дер-Поля.

\subsection{Условия самовозбуждения}

Линеаризуя уравнение, получаем:

\begin{equation}
\frac{d^2 u}{dt^2} - \frac{Mg_0 - RC}{LC} \frac{du}{dt} + \frac{u}{LC} = 0
\end{equation}

При $RC > Mg_0$ состояние равновесия устойчиво. При $Mg_0 > RC$ малые возмущения нарастают.

\section{Уравнение Ван-дер-Поля}

Уравнение Ван-дер-Поля имеет единственную особую точку $x = \dot{x} = 0$. При $\lambda > 0$ на фазовой плоскости имеется предельный цикл, отвечающий режиму периодических автоколебаний. Фазовые портреты и временные реализации колебаний представлены на рис. 11.3.


\section{Автогенератор на активном элементе с отрицательной дифференциальной проводимостью}

Рассмотрим схему генератора с активным элементом, вольтамперная характеристика которого имеет падающий участок. Уравнение для генератора:

\begin{equation}
C \frac{du}{dt} + Gu + i - g_0 u + g_2 u^3 = 0
\end{equation}

\begin{equation}
u = L \frac{di_L}{dt}
\end{equation}

Дифференцируя и подставляя, приходим к уравнению Ван-дер-Поля:

\begin{equation}
\frac{d^2 u}{dt^2} - \frac{1}{C} (g_0 - G - 3g_2 u^2) \frac{du}{dt} + \omega_0^2 u = 0
\end{equation}

Условие самовозбуждения: $g_0 > G$.

\section{Ламповый генератор Ван-дер-Поля}

Ламповый генератор Ван-дер-Поля служит классической моделью для изучения автоколебательных систем. Основные структурные элементы включают колебательный контур, обратную связь и активный элемент — триод.

\begin{equation}
i_a = i_L + i_C
\end{equation}

\begin{equation}
L \frac{di_L}{dt} + Ri = \frac{1}{C} \int i_C \, dt
\end{equation}

Дифференцируя и подставляя, получаем уравнение, аналогичное уравнению Ван-дер-Поля.

\end{document}