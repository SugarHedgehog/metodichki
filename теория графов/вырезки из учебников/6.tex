\subsection{Эйлеровы графы}

Как мы уже видели в гл. 1, отрицательное решение Эйлером задачи о кёнигсбергских мостах привело к первой опубликованной работе по теории графов. Задачу об обходе мостов можно обобщить и получить следующую задачу теории графов: можно ли найти в данном графе \(G\) цикл, содержащий все вершины и все рёбра? Граф, в котором это возможно, называется \textit{эйлеровым}. Таким образом, эйлеров граф имеет \textit{эйлеров цикл} — замкнутую цепь, содержащую все вершины и все рёбра. Ясно, что эйлеров граф должен быть связным.

\textbf{Теорема 7.1.} Для связного графа \(G\) следующие утверждения эквивалентны:
\begin{enumerate}
    \item \(G\) — эйлеров граф;
    \item каждая вершина графа \(G\) имеет чётную степень;
    \item множество рёбер графа \(G\) можно разбить на простые циклы.
\end{enumerate}

\footnotetext[1]{Ясно, что эта теорема справедлива также и для мультиграфов.}

\textbf{Доказательство.} (1) влечет (2). Пусть \(T\) — эйлеров цикл в \(G\). Каждое прохождение данной вершины в \(T\) вносит 2 в степень этой вершины и, поскольку каждое ребро графа \(G\) появляется точно один раз в \(T\), любая вершина должна иметь чётную степень.

\begin{enumerate}
    \item[(2)] влечет (3). Так как \( G \) — связный и нетривиальный граф, то степень каждой вершины равна по крайней мере 2, так что \( G \) содержит простой цикл \( Z \). Удаление рёбер цикла \( Z \) приводит к остовному подграфу \( G_1 \), в котором также каждая вершина имеет чётную степень. Если в \( G_1 \) нет рёбер, то (3) уже доказано; в противном случае применим высказанные выше соображения к \( G_1 \) и получим граф \( G_2 \), в котором опять степени всех вершин чётны, и т. д. Одновременно с пустым графом \( G_n \) получаем разбиение рёбер графа \( G \) на \( n \) простых циклов.

    \item[(3)] влечет (1). Пусть \( Z_1 \) — один из простых циклов этого разбиения. Если \( G \) состоит только из этого цикла, то очевидно, что \( G \) — эйлеров граф. В противном случае другой простой цикл \( Z_2 \) в \( G \) имеет вершину \( v \), общую с \( Z_1 \). Маршрут, начинающийся с \( v \) и состоящий из цикла \( Z_1 \) и следующего непосредственно за ним цикла \( Z_2 \), является замкнутой цепью, которая содержит рёбра этих двух циклов. Продолжая эту процедуру, мы можем построить замкнутую цепь, содержащую все рёбра графа \( G \); следовательно, \( G \) — эйлеров граф.
\end{enumerate}

Например, связный граф, представленный на рис. 7.1, в котором каждая вершина имеет чётную степень, обладает эйлеровым циклом.
Из теоремы 7.1 следует, что если в связном графе \( G \) нет вершин с нечётными степенями, то в \( G \) есть замкнутая цепь, содержащая все вершины и все рёбра графа.

G. Аналогичный результат справедлив для связных графов, имеющих некоторое число вершин с нечётными степенями.

\textbf{Следствие 7.1 (a).} Пусть \( G \) — связный граф, в котором \( 2n \) вершин имеют нечётные степени, \( n \geq 1 \). Тогда множество рёбер графа \( G \) можно разбить на \( n \) открытых цепей.

\textbf{Следствие 7.1 (б).} Пусть \( G \) — связный граф, в котором две вершины имеют нечётные степени. Тогда в \( G \) есть открытая цепь, содержащая все вершины и все рёбра графа \( G \) (и начинающаяся в одной из вершин с нечётной степенью, а кончающаяся в другой).