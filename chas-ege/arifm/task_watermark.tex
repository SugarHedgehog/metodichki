\documentclass[4apaper]{article}
\usepackage{dashbox}
\usepackage[T2A]{fontenc}
\usepackage[utf8]{inputenc}
\usepackage[english,russian]{babel}
\usepackage{graphicx}
\DeclareGraphicsExtensions{.pdf,.png,.jpg}

\linespread{1.15}

\usepackage{../egetask_ver}

\def\examyear{2023}
\usepackage[colorlinks,linkcolor=blue]{hyperref}\usepackage{draftwatermark}
\SetWatermarkLightness{0.9}
\SetWatermarkText{https://vk.com/egemathika}
\SetWatermarkScale{ 0.4 }
\def\lfoottext{Источник \href{https://vk.com/egemathika}{https://vk.com/egemathika}}

\begin{document}
\begin{taskBN}{1}
Грузовик перевозит партию щебня массой 475 тонн, ежедневно увеличивая норму перевозки на одно и то же число тонн. Известно, что за первый день было перевезено 7 тонн щебня. Определите, сколько тонн щебня было перевезено на тринадцатый день, если вся работа была выполнена за 19 дней.
\end{taskBN}

\begin{taskBN}{2}
Вся работа была выполнена за 49 дней, при этом Яне необходимо было разослать 6762 письма. Сколько писем было разослано за 24-й день, если за 8-й день Яна разослала 53 письма, а ежедневно она рассылала на одно и то же количество писем больше по сравнению с предыдущим днём? 
\end{taskBN}

\begin{taskBN}{3}
Васе надо решить 703 задачи. Он решает на одно и то же количество задач больше по сравнению с предыдущим днем. Известно, что за первый день Вася решил 10 задач. Определите, сколько задач решил Вася в последний день, если со всеми задачами он справился за 19 дней.
\end{taskBN}

\begin{taskBN}{4}
Грузовик перевозит партию щебня массой 549 тонн, ежедневно увеличивая норму перевозки на одно и то же число тонн. Известно, что за первый день было перевезено 5 тонн щебня. Определите, сколько тонн щебня было перевезено на одиннадцатый день, если вся работа была выполнена за 18 дней.
\end{taskBN}

\begin{taskBN}{5}
Дана арифметическая прогрессия, для которой $a_{19} = 158$, $a_{12} = 95$. Найдите $a_{11}$.
\end{taskBN}

\begin{taskBN}{6}
За сколько дней была выполнена вся работа, если за 2-й день Вероника разослала 22 приглашения, при этом ежедневно она рассылала на одно и то же количество приглашений больше по сравнению с предыдущим днём? Веронике необходимо было разослать 224 приглашения, за 1-й день Вероника разослала 17 приглашений. 
\end{taskBN}

\begin{taskBN}{7}
 Улитка ползет от одного дерева до другого. Каждый день она проползает на одно и то же расстояние больше, чем в предыдущий день. Известно, что за первый и последний дни улитка проползла в общей сложности 8 метров. Определите, сколько дней улитка потратила на весь путь, если расстояние между деревьями равно 12 метров.
\end{taskBN}

\begin{taskBN}{8}
 Улитка ползет от одного дерева до другого. Каждый день она проползает на одно и то же расстояние больше, чем в предыдущий день. Известно, что за первый и последний дни улитка проползла в общей сложности 18 метров. Определите, сколько дней улитка потратила на весь путь, если расстояние между деревьями равно 54 метра.
\end{taskBN}

\begin{taskBN}{9}
Дана арифметическая прогрессия, для которой $a_1 = -7$, $d=2$. Найдите $a_{13}$.
\end{taskBN}

\begin{taskBN}{10}
 Рабочие прокладывают тоннель длиной 52 метра, ежедневно увеличивая норму прокладки на одно и то же число метров. Известно, что за первый день рабочие проложили 10 метров туннеля.  Определите, сколько метров туннеля проложили рабочие в последний день, если вся работа была выполнена за 4 дня.
\end{taskBN}

\begin{taskBN}{11}
Для арифметической прогрессии ${a_n}$ известно, что $a_{11} + a_{16} + a_{21}= 294$. Найдите $a_{16}$.
\end{taskBN}

\begin{taskBN}{12}
Оле надо подписать 714 открыток. Ежедневно она подписывает на одно и то же количество открыток больше по сравнению с предыдущим днем. Известно, что за первый день Оля подписала 4 открытки. Определите, сколько открыток было подписано за четырнадцатый день, если вся работа была выполнена за 21 день.
\end{taskBN}

\begin{taskBN}{13}
Грузовик перевозит партию щебня массой 266 тонн, ежедневно увеличивая норму перевозки на одно и то же число тонн. Известно, что за первый день было перевезено 6 тонн щебня. Определите, сколько тонн щебня было перевезено на седьмой день, если вся работа была выполнена за 14 дней.
\end{taskBN}

\begin{taskBN}{14}
Дана арифметическая прогрессия, для которой $a_{14} = -112$, $d=-9$. Найдите $a_{18}$.
\end{taskBN}

\begin{taskBN}{15}
Дана арифметическая прогрессия, для которой $a_{14} = 64$, $a_{17} = 79$. Найдите разность арифметической прогрессии $d$.
\end{taskBN}

\begin{taskBN}{16}
 Бригада маляров красит забор длиной 231 метр, ежедневно увеличивая норму покраски на одно и то же число метров. Известно, что за первый и последний день в сумме бригада покрасила 66 метров забора.  Определите, сколько дней бригада маляров красила весь забор. 
\end{taskBN}

\begin{taskBN}{17}
Грузовик перевозит партию щебня массой 608 тонн, ежедневно увеличивая норму перевозки на одно и то же число тонн. Известно, что за первый день было перевезено 8 тонн щебня. Определите, сколько тонн щебня было перевезено на девятый день, если вся работа была выполнена за 16 дней.
\end{taskBN}

\begin{taskBN}{18}
Грузовик перевозит партию щебня массой 360 тонн, ежедневно увеличивая норму перевозки на одно и то же число тонн. Известно, что за первый день было перевезено 3 тонны щебня. Определите, сколько тонн щебня было перевезено на двенадцатый день, если вся работа была выполнена за 18 дней.
\end{taskBN}

\begin{taskBN}{19}
Руслану надо решить 198 задач. Он решает на одно и то же количество задач больше по сравнению с предыдущим днем. Известно, что за первый день Руслан решил 3 задачи. Определите, сколько задач решил Руслан в последний день, если со всеми задачами он справился за 11 дней.
\end{taskBN}

\begin{taskBN}{20}
Дана арифметическая прогрессия, для которой $a_{13} = -57$, $d=-5$. Найдите $a_{19}$.
\end{taskBN}

\begin{taskBN}{21}
 Бригада маляров красит забор длиной 153 метра, ежедневно увеличивая норму покраски на одно и то же число метров. Известно, что за первый и последний день в сумме бригада покрасила 34 метра забора.  Определите, сколько дней бригада маляров красила весь забор. 
\end{taskBN}

\begin{taskBN}{22}
Дана арифметическая прогрессия, для которой $a_{5} = -16$, $a_{16} = -60$. Найдите разность арифметической прогрессии $d$.
\end{taskBN}

\begin{taskBN}{23}
Турист идет из одного города в другой, каждый день проходя больше, чем в предыдущий день, на одно и то же расстояние. Известно, что за первый день турист прошел 3 километра. Определите, сколько километров прошел турист за тринадцатый день, если весь путь он прошел за 20 дней, а расстояние между городами составляет 630 километров.
\end{taskBN}

\begin{taskBN}{24}
Дана арифметическая прогрессия, для которой $a_1 = -9$, $d=3$. Найдите $a_{8}$.
\end{taskBN}

\begin{taskBN}{25}
Для арифметической прогрессии ${a_n}$ известно, что $a_{14} + a_{19} + a_{24}= 159$. Найдите $a_{19}$.
\end{taskBN}

\begin{taskBN}{26}
Дана арифметическая прогрессия, для которой $a_{16} = 127$, $a_{17} = 135$. Найдите $a_{13}$.
\end{taskBN}

\begin{taskBN}{27}
Дана арифметическая прогрессия, для которой $a_{12} = 44$, $a_{17} = 64$. Найдите разность арифметической прогрессии $d$.
\end{taskBN}

\begin{taskBN}{28}
 Бригада маляров красит забор длиной 80 метров, ежедневно увеличивая норму покраски на одно и то же число метров. Известно, что за первый и последний день в сумме бригада покрасила 32 метра забора.  Определите, сколько дней бригада маляров красила весь забор. 
\end{taskBN}

\begin{taskBN}{29}
Дана арифметическая прогрессия, для которой $a_{16} = -124$, $a_{12} = -92$. Найдите $a_{7}$.
\end{taskBN}

\begin{taskBN}{30}
 Рабочие прокладывают тоннель длиной 75 метров, ежедневно увеличивая норму прокладки на одно и то же число метров. Известно, что за первый день рабочие проложили 9 метров туннеля.  Определите, сколько метров туннеля проложили рабочие в последний день, если вся работа была выполнена за 5 дней.
\end{taskBN}

\begin{taskBN}{31}
Для арифметической прогрессии ${a_n}$ известно, что $a_{6} + a_{18} + a_{30}= 183$. Найдите $a_{18}$.
\end{taskBN}

\begin{taskBN}{32}
Дана арифметическая прогрессия, для которой $a_{14} = -33$, $d=-3$. Найдите $a_{8}$.
\end{taskBN}

\begin{taskBN}{33}
Юлии необходимо было разослать 1716 открыток. Сколько открыток было разослано за 7-й день, если ежедневно Юлия рассылала на одно и то же количество открыток больше по сравнению с предыдущим днём, вся работа была выполнена за 22 дня, при этом за 20-й день она разослала 129 открыток? 
\end{taskBN}

\begin{taskBN}{34}
Турист идет из одного города в другой, каждый день проходя больше, чем в предыдущий день, на одно и то же расстояние. Известно, что за первый день турист прошел 5 километров. Определите, сколько километров прошел турист за девятый день, если весь путь он прошел за 14 дней, а расстояние между городами составляет 434 километра.
\end{taskBN}

\begin{taskBN}{35}
Для арифметической прогрессии ${a_n}$ известно, что $a_{8} + a_{10} + a_{12}= -174$. Найдите $a_{10}$.
\end{taskBN}

\begin{taskBN}{36}
Грузовик перевозит партию щебня массой 364 тонны, ежедневно увеличивая норму перевозки на одно и то же число тонн. Известно, что за первый день было перевезено 10 тонн щебня. Определите, сколько тонн щебня было перевезено на шестой день, если вся работа была выполнена за 13 дней.
\end{taskBN}

\begin{taskBN}{37}
 Улитка ползет от одного дерева до другого. Каждый день она проползает на одно и то же расстояние больше, чем в предыдущий день. Известно, что за первый и последний дни улитка проползла в общей сложности 11 метров. Определите, сколько дней улитка потратила на весь путь, если расстояние между деревьями равно 22 метра.
\end{taskBN}

\begin{taskBN}{38}
Мише надо решить 198 задач. Он решает на одно и то же количество задач больше по сравнению с предыдущим днем. Известно, что за первый день Миша решил 3 задачи. Определите, сколько задач решил Миша в последний день, если со всеми задачами он справился за 11 дней.
\end{taskBN}

\begin{taskBN}{39}
Вся работа была выполнена за 26 дней. Сколько приглашений было разослано за 6-й день, если за 20-й день Элеонора разослала 159 приглашений? Элеоноре требовалось разослать 2782 приглашения, при этом ежедневно она рассылала на одно и то же количество приглашений больше по сравнению с предыдущим днём. 
\end{taskBN}

\begin{taskBN}{40}
Грузовик перевозит партию щебня массой 472 тонны, ежедневно увеличивая норму перевозки на одно и то же число тонн. Известно, что за первый день было перевезено 7 тонн щебня. Определите, сколько тонн щебня было перевезено на девятый день, если вся работа была выполнена за 16 дней.
\end{taskBN}

\begin{taskBN}{41}
Турист идет из одного города в другой, каждый день проходя больше, чем в предыдущий день, на одно и то же расстояние. Известно, что за первый день турист прошел 5 километров. Определите, сколько километров прошел турист за пятнадцатый день, если весь путь он прошел за 21 день, а расстояние между городами составляет 735 километров.
\end{taskBN}

\begin{taskBN}{42}
Дана арифметическая прогрессия, для которой $a_{9} = -33$, $d=-4$. Найдите $a_{12}$.
\end{taskBN}

\begin{taskBN}{43}
Грузовик перевозит партию щебня массой 424 тонны, ежедневно увеличивая норму перевозки на одно и то же число тонн. Известно, что за первый день было перевезено 4 тонны щебня. Определите, сколько тонн щебня было перевезено на деcятый день, если вся работа была выполнена за 16 дней.
\end{taskBN}

\begin{taskBN}{44}
 Улитка ползет от одного дерева до другого. Каждый день она проползает на одно и то же расстояние больше, чем в предыдущий день. Известно, что за первый и последний дни улитка проползла в общей сложности 18 метров. Определите, сколько дней улитка потратила на весь путь, если расстояние между деревьями равно 63 метра.
\end{taskBN}

\begin{taskBN}{45}
 Рабочие прокладывают тоннель длиной 198 метров, ежедневно увеличивая норму прокладки на одно и то же число метров. Известно, что за первый день рабочие проложили 3 метра туннеля.  Определите, сколько метров туннеля проложили рабочие в последний день, если вся работа была выполнена за 11 дней.
\end{taskBN}

\begin{taskBN}{46}
Мише надо решить 338 задач. Он решает на одно и то же количество задач больше по сравнению с предыдущим днем. Известно, что за первый день Миша решил 2 задачи. Определите, сколько задач решил Миша в последний день, если со всеми задачами он справился за 13 дней.
\end{taskBN}

\begin{taskBN}{47}
Турист идет из одного города в другой, каждый день проходя больше, чем в предыдущий день, на одно и то же расстояние. Известно, что за первый день турист прошел 8 километров. Определите, сколько километров прошел турист за шестнадцатый день, если весь путь он прошел за 22 дня, а расстояние между городами составляет 869 километров.
\end{taskBN}

\begin{taskBN}{48}
Грузовик перевозит партию щебня массой 531 тонна, ежедневно увеличивая норму перевозки на одно и то же число тонн. Известно, что за первый день было перевезено 4 тонны щебня. Определите, сколько тонн щебня было перевезено на двенадцатый день, если вся работа была выполнена за 18 дней.
\end{taskBN}

\begin{taskBN}{49}
Жене надо решить 672 задачи. Он решает на одно и то же количество задач больше по сравнению с предыдущим днем. Известно, что за первый день Женя решил 12 задач. Определите, сколько задач решил Женя в последний день, если со всеми задачами он справился за 16 дней.
\end{taskBN}

\begin{taskBN}{50}
Найдите сумму первых 49 чётных натуральных чисел.
\end{taskBN}

\begin{taskBN}{51}
Для арифметической прогрессии ${a_n}$ известно, что $a_{3} + a_{16} + a_{29}= -213$. Найдите $a_{16}$.
\end{taskBN}

\begin{taskBN}{52}
 Рабочие прокладывают тоннель длиной 140 метров, ежедневно увеличивая норму прокладки на одно и то же число метров. Известно, что за первый день рабочие проложили 7 метров туннеля.  Определите, сколько метров туннеля проложили рабочие в последний день, если вся работа была выполнена за 8 дней.
\end{taskBN}

\begin{taskBN}{53}
Дана арифметическая прогрессия, для которой $a_{16} = 21$, $d=1$. Найдите $a_{9}$.
\end{taskBN}

\begin{taskBN}{54}
 Улитка ползет от одного дерева до другого. Каждый день она проползает на одно и то же расстояние больше, чем в предыдущий день. Известно, что за первый и последний дни улитка проползла в общей сложности 28 метров. Определите, сколько дней улитка потратила на весь путь, если расстояние между деревьями равно 126 метров.
\end{taskBN}

\begin{taskBN}{55}
Для арифметической прогрессии ${a_n}$ известно, что $a_{5} + a_{7} + a_{9}= 198$. Найдите $a_{7}$.
\end{taskBN}

\begin{taskBN}{56}
Дана арифметическая прогрессия, для которой $a_1 = -10$, $d=5$. Найдите $a_{13}$.
\end{taskBN}

\begin{taskBN}{57}
Турист идет из одного города в другой, каждый день проходя больше, чем в предыдущий день, на одно и то же расстояние. Известно, что за первый день турист прошел 4 километра. Определите, сколько километров прошел турист за одиннадцатый день, если весь путь он прошел за 17 дней, а расстояние между городами составляет 476 километров.
\end{taskBN}

\begin{taskBN}{58}
Дана арифметическая прогрессия, для которой $a_{16} = -29$, $d=-2$. Найдите $a_{9}$.
\end{taskBN}

\begin{taskBN}{59}
 Бригада маляров красит забор длиной 351 метр, ежедневно увеличивая норму покраски на одно и то же число метров. Известно, что за первый и последний день в сумме бригада покрасила 78 метров забора.  Определите, сколько дней бригада маляров красила весь забор. 
\end{taskBN}

\begin{taskBN}{60}
Дана арифметическая прогрессия, для которой $a_{18} = 19$, $a_{16} = 17$. Найдите $a_{19}$.
\end{taskBN}

\begin{taskBN}{61}
Грузовик перевозит партию щебня массой 390 тонн, ежедневно увеличивая норму перевозки на одно и то же число тонн. Известно, что за первый день было перевезено 5 тонн щебня. Определите, сколько тонн щебня было перевезено на деcятый день, если вся работа была выполнена за 15 дней.
\end{taskBN}

\begin{taskBN}{62}
 Бригада маляров красит забор длиной 156 метров, ежедневно увеличивая норму покраски на одно и то же число метров. Известно, что за первый и последний день в сумме бригада покрасила 52 метра забора.  Определите, сколько дней бригада маляров красила весь забор. 
\end{taskBN}

\begin{taskBN}{63}
Дана арифметическая прогрессия, для которой $a_{17} = -36$, $d=-2$. Найдите $a_{11}$.
\end{taskBN}

\begin{taskBN}{64}
 Бригада маляров красит забор длиной 174 метра, ежедневно увеличивая норму покраски на одно и то же число метров. Известно, что за первый и последний день в сумме бригада покрасила 58 метров забора.  Определите, сколько дней бригада маляров красила весь забор. 
\end{taskBN}

\begin{taskBN}{65}
Дана арифметическая прогрессия, для которой $a_1 = -3$, $d=-3$. Найдите $a_{7}$.
\end{taskBN}

\begin{taskBN}{66}
 Рабочие прокладывают тоннель длиной 52 метра, ежедневно увеличивая норму прокладки на одно и то же число метров. Известно, что за первый день рабочие проложили 7 метров туннеля.  Определите, сколько метров туннеля проложили рабочие в последний день, если вся работа была выполнена за 4 дня.
\end{taskBN}

\begin{taskBN}{67}
 Рабочие прокладывают тоннель длиной 185 метров, ежедневно увеличивая норму прокладки на одно и то же число метров. Известно, что за первый день рабочие проложили 5 метров туннеля.  Определите, сколько метров туннеля проложили рабочие в последний день, если вся работа была выполнена за 10 дней.
\end{taskBN}

\begin{taskBN}{68}
Дана арифметическая прогрессия, для которой $a_{2} = -11$, $a_{8} = -71$. Найдите разность арифметической прогрессии $d$.
\end{taskBN}

\begin{taskBN}{69}
Для арифметической прогрессии ${a_n}$ известно, что $a_{3} + a_{15} + a_{27}= 225$. Найдите $a_{15}$.
\end{taskBN}

\begin{taskBN}{70}
Турист идет из одного города в другой, каждый день проходя больше, чем в предыдущий день, на одно и то же расстояние. Известно, что за первый день турист прошел 4 километра. Определите, сколько километров прошел турист за деcятый день, если весь путь он прошел за 17 дней, а расстояние между городами составляет 340 километров.
\end{taskBN}

\begin{taskBN}{71}
 Улитка ползет от одного дерева до другого. Каждый день она проползает на одно и то же расстояние больше, чем в предыдущий день. Известно, что за первый и последний дни улитка проползла в общей сложности 17 метров. Определите, сколько дней улитка потратила на весь путь, если расстояние между деревьями равно 34 метра.
\end{taskBN}

\begin{taskBN}{72}
Кате надо подписать 730 открыток. Ежедневно она подписывает на одно и то же количество открыток больше по сравнению с предыдущим днем. Известно, что за первый день Катя подписала 8 открыток. Определите, сколько открыток было подписано за четырнадцатый день, если вся работа была выполнена за 20 дней.
\end{taskBN}

\begin{taskBN}{73}
Дана арифметическая прогрессия, для которой $a_{13} = 55$, $a_{16} = 67$. Найдите разность арифметической прогрессии $d$.
\end{taskBN}

\begin{taskBN}{74}
 Бригада маляров красит забор длиной 125 метров, ежедневно увеличивая норму покраски на одно и то же число метров. Известно, что за первый и последний день в сумме бригада покрасила 50 метров забора.  Определите, сколько дней бригада маляров красила весь забор. 
\end{taskBN}

\begin{taskBN}{75}
Дана арифметическая прогрессия, для которой $a_{15} = 136$, $a_{8} = 66$. Найдите $a_{12}$.
\end{taskBN}

\begin{taskBN}{76}
Для арифметической прогрессии ${a_n}$ известно, что $a_{9} + a_{19} + a_{29}= -132$. Найдите $a_{19}$.
\end{taskBN}

\begin{taskBN}{77}
Дана арифметическая прогрессия, для которой $a_{9} = -76$, $a_{12} = -103$. Найдите $a_{18}$.
\end{taskBN}

\begin{taskBN}{78}
Сколько всего открыток необходимо было разослать Олесе, если за 15-й день Олеся разослала 30 открыток, а за 19-й день она разослала 38 открыток? Ежедневно Олеся рассылала на одно и то же количество открыток больше по сравнению с предыдущим днём, при этом вся работа была выполнена за 40 дней. 
\end{taskBN}

\begin{taskBN}{79}
 Бригада маляров красит забор длиной 130 метров, ежедневно увеличивая норму покраски на одно и то же число метров. Известно, что за первый и последний день в сумме бригада покрасила 52 метра забора.  Определите, сколько дней бригада маляров красила весь забор. 
\end{taskBN}

\begin{taskBN}{80}
Дана арифметическая прогрессия, для которой $a_1 = -5$, $d=4$. Найдите $a_{13}$.
\end{taskBN}

\begin{taskBN}{81}
Дана арифметическая прогрессия, для которой $a_{10} = 47$, $d=6$. Найдите $a_{12}$.
\end{taskBN}

\begin{taskBN}{82}
Турист идет из одного города в другой, каждый день проходя больше, чем в предыдущий день, на одно и то же расстояние. Известно, что за первый день турист прошел 3 километра. Определите, сколько километров прошел турист за шестой день, если весь путь он прошел за 11 дней, а расстояние между городами составляет 143 километра.
\end{taskBN}

\begin{taskBN}{83}
 Улитка ползет от одного дерева до другого. Каждый день она проползает на одно и то же расстояние больше, чем в предыдущий день. Известно, что за первый и последний дни улитка проползла в общей сложности 38 метров. Определите, сколько дней улитка потратила на весь путь, если расстояние между деревьями равно 171 метр.
\end{taskBN}

\begin{taskBN}{84}
Дана арифметическая прогрессия, для которой $a_{13} = -81$, $a_{19} = -123$. Найдите $a_{10}$.
\end{taskBN}

\begin{taskBN}{85}
Турист идет из одного города в другой, каждый день проходя больше, чем в предыдущий день, на одно и то же расстояние. Известно, что за первый день турист прошел 6 километров. Определите, сколько километров прошел турист за восьмой день, если весь путь он прошел за 14 дней, а расстояние между городами составляет 357 километров.
\end{taskBN}

\begin{taskBN}{86}
Дана арифметическая прогрессия, для которой $a_1 = -2$, $d=-7$. Найдите $a_{11}$.
\end{taskBN}

\begin{taskBN}{87}
Турист идет из одного города в другой, каждый день проходя больше, чем в предыдущий день, на одно и то же расстояние. Известно, что за первый день турист прошел 6 километров. Определите, сколько километров прошел турист за восемнадцатый день, если весь путь он прошел за 23 дня, а расстояние между городами составляет 644 километра.
\end{taskBN}

\begin{taskBN}{88}
Дана арифметическая прогрессия, для которой $a_1 = -8$, $d=-2$. Найдите $a_{12}$.
\end{taskBN}

\begin{taskBN}{89}
Турист идет из одного города в другой, каждый день проходя больше, чем в предыдущий день, на одно и то же расстояние. Известно, что за первый день турист прошел 10 километров. Определите, сколько километров прошел турист за деcятый день, если весь путь он прошел за 16 дней, а расстояние между городами составляет 520 километров.
\end{taskBN}

\begin{taskBN}{90}
Для арифметической прогрессии ${a_n}$ известно, что $a_{10} + a_{16} + a_{22}= -273$. Найдите $a_{16}$.
\end{taskBN}

\begin{taskBN}{91}
Дана арифметическая прогрессия, для которой $a_{10} = 84$, $a_{18} = 156$. Найдите $a_{8}$.
\end{taskBN}

\begin{taskBN}{92}
Дана арифметическая прогрессия, для которой $a_{4} = -31$, $a_{7} = -55$. Найдите разность арифметической прогрессии $d$.
\end{taskBN}

\begin{taskBN}{93}
Пете надо решить 572 задачи. Он решает на одно и то же количество задач больше по сравнению с предыдущим днем. Известно, что за первый день Петя решил 5 задач. Определите, сколько задач решил Петя в последний день, если со всеми задачами он справился за 22 дня.
\end{taskBN}

\begin{taskBN}{94}
Ежедневно Александра подписывала на одно и то же количество писем больше по сравнению с предыдущим днём, вся работа была выполнена за 31 день. Сколько писем было подписано за 16-й день, если Александре необходимо было подписать 589 писем, за 2-й день она подписала 5 писем? 
\end{taskBN}

\begin{taskBN}{95}
Мише надо решить 162 задачи. Он решает на одно и то же количество задач больше по сравнению с предыдущим днем. Известно, что за первый день Миша решил 6 задач. Определите, сколько задач решил Миша в последний день, если со всеми задачами он справился за 9 дней.
\end{taskBN}

\begin{taskBN}{96}
 Бригада маляров красит забор длиной 140 метров, ежедневно увеличивая норму покраски на одно и то же число метров. Известно, что за первый и последний день в сумме бригада покрасила 40 метров забора.  Определите, сколько дней бригада маляров красила весь забор. 
\end{taskBN}

\begin{taskBN}{97}
Турист идет из одного города в другой, каждый день проходя больше, чем в предыдущий день, на одно и то же расстояние. Известно, что за первый день турист прошел 3 километра. Определите, сколько километров прошел турист за шестой день, если весь путь он прошел за 11 дней, а расстояние между городами составляет 198 километров.
\end{taskBN}

\begin{taskBN}{98}
 Бригада маляров красит забор длиной 152 метра, ежедневно увеличивая норму покраски на одно и то же число метров. Известно, что за первый и последний день в сумме бригада покрасила 38 метров забора.  Определите, сколько дней бригада маляров красила весь забор. 
\end{taskBN}

\begin{taskBN}{99}
Дана арифметическая прогрессия, для которой $a_{3} = 7$, $a_{11} = 71$. Найдите разность арифметической прогрессии $d$.
\end{taskBN}

\begin{taskBN}{100}
Сколько всего открыток нужно было разослать Марии, если за 5-й день Мария разослала 31 открытку, а ежедневно она рассылала на одно и то же количество открыток больше по сравнению с предыдущим днём, при этом за 2-й день Мария разослала 10 открыток? Вся работа была выполнена за 12 дней. 
\end{taskBN}

\begin{taskBN}{101}
Турист идет из одного города в другой, каждый день проходя больше, чем в предыдущий день, на одно и то же расстояние. Известно, что за первый день турист прошел 3 километра. Определите, сколько километров прошел турист за девятый день, если весь путь он прошел за 15 дней, а расстояние между городами составляет 255 километров.
\end{taskBN}

\begin{taskBN}{102}
Для арифметической прогрессии ${a_n}$ известно, что $a_{6} + a_{8} + a_{10}= 195$. Найдите $a_{8}$.
\end{taskBN}

\begin{taskBN}{103}
Дана арифметическая прогрессия, для которой $a_{12} = -38$, $a_{9} = -29$. Найдите $a_{14}$.
\end{taskBN}

\begin{taskBN}{104}
Ире надо подписать 1219 открыток. Ежедневно она подписывает на одно и то же количество открыток больше по сравнению с предыдущим днем. Известно, что за первый день Ира подписала 9 открыток. Определите, сколько открыток было подписано за шестнадцатый день, если вся работа была выполнена за 23 дня.
\end{taskBN}

\begin{taskBN}{105}
Дана арифметическая прогрессия, для которой $a_1 = 4$, $d=6$. Найдите $a_{9}$.
\end{taskBN}

\begin{taskBN}{106}
Толе надо решить 722 задачи. Он решает на одно и то же количество задач больше по сравнению с предыдущим днем. Известно, что за первый день Толя решил 11 задач. Определите, сколько задач решил Толя в последний день, если со всеми задачами он справился за 19 дней.
\end{taskBN}

\begin{taskBN}{107}
Турист идет из одного города в другой, каждый день проходя больше, чем в предыдущий день, на одно и то же расстояние. Известно, что за первый день турист прошел 4 километра. Определите, сколько километров прошел турист за восьмой день, если весь путь он прошел за 13 дней, а расстояние между городами составляет 208 километров.
\end{taskBN}

\begin{taskBN}{108}
Турист идет из одного города в другой, каждый день проходя больше, чем в предыдущий день, на одно и то же расстояние. Известно, что за первый день турист прошел 7 километров. Определите, сколько километров прошел турист за шестнадцатый день, если весь путь он прошел за 22 дня, а расстояние между городами составляет 847 километров.
\end{taskBN}

\begin{taskBN}{109}
Дана арифметическая прогрессия, для которой $a_{17} = 52$, $a_{18} = 55$. Найдите $a_{8}$.
\end{taskBN}

\begin{taskBN}{110}
 Бригада маляров красит забор длиной 308 метров, ежедневно увеличивая норму покраски на одно и то же число метров. Известно, что за первый и последний день в сумме бригада покрасила 56 метров забора.  Определите, сколько дней бригада маляров красила весь забор. 
\end{taskBN}

\begin{taskBN}{111}
Дана арифметическая прогрессия, для которой $a_{11} = -49$, $a_{14} = -64$. Найдите $a_{10}$.
\end{taskBN}

\begin{taskBN}{112}
Дана арифметическая прогрессия, для которой $a_{8} = 27$, $d=5$. Найдите $a_{19}$.
\end{taskBN}

\begin{taskBN}{113}
Сколько открыток было оформлено за 31-й день, если вся работа была выполнена за 46 дней, а Елене надо было оформить 10143 открытки? За 8-й день Елена оформила 81 открытку, ежедневно она оформляла на одно и то же количество открыток больше по сравнению с предыдущим днём. 
\end{taskBN}

\begin{taskBN}{114}
 Рабочие прокладывают тоннель длиной 225 метров, ежедневно увеличивая норму прокладки на одно и то же число метров. Известно, что за первый день рабочие проложили 9 метров туннеля.  Определите, сколько метров туннеля проложили рабочие в последний день, если вся работа была выполнена за 9 дней.
\end{taskBN}

\begin{taskBN}{115}
Для арифметической прогрессии ${a_n}$ известно, что $a_{17} + a_{20} + a_{23}= 189$. Найдите $a_{20}$.
\end{taskBN}

\begin{taskBN}{116}
Дана арифметическая прогрессия, для которой $a_{13} = 55$, $a_{11} = 47$. Найдите $a_{15}$.
\end{taskBN}

\begin{taskBN}{117}
Найдите сумму первых 33 нечётных натуральных чисел.
\end{taskBN}

\begin{taskBN}{118}
Грузовик перевозит партию щебня массой 630 тонн, ежедневно увеличивая норму перевозки на одно и то же число тонн. Известно, что за первый день было перевезено 3 тонны щебня. Определите, сколько тонн щебня было перевезено на четырнадцатый день, если вся работа была выполнена за 20 дней.
\end{taskBN}

\begin{taskBN}{119}
Дана арифметическая прогрессия, для которой $a_1 = 5$, $d=-4$. Найдите $a_{13}$.
\end{taskBN}

\begin{taskBN}{120}
Турист идет из одного города в другой, каждый день проходя больше, чем в предыдущий день, на одно и то же расстояние. Известно, что за первый день турист прошел 6 километров. Определите, сколько километров прошел турист за двенадцатый день, если весь путь он прошел за 18 дней, а расстояние между городами составляет 720 километров.
\end{taskBN}

\begin{taskBN}{121}
 Рабочие прокладывают тоннель длиной 119 метров, ежедневно увеличивая норму прокладки на одно и то же число метров. Известно, что за первый день рабочие проложили 8 метров туннеля.  Определите, сколько метров туннеля проложили рабочие в последний день, если вся работа была выполнена за 7 дней.
\end{taskBN}

\begin{taskBN}{122}
Грузовик перевозит партию щебня массой 990 тонн, ежедневно увеличивая норму перевозки на одно и то же число тонн. Известно, что за первый день было перевезено 3 тонны щебня. Определите, сколько тонн щебня было перевезено на шестнадцатый день, если вся работа была выполнена за 22 дня.
\end{taskBN}

\begin{taskBN}{123}
Дана арифметическая прогрессия, для которой $a_{9} = -32$, $a_{15} = -56$. Найдите разность арифметической прогрессии $d$.
\end{taskBN}

\begin{taskBN}{124}
Для арифметической прогрессии ${a_n}$ известно, что $a_{6} + a_{15} + a_{24}= 54$. Найдите $a_{15}$.
\end{taskBN}

\begin{taskBN}{125}
Дана арифметическая прогрессия, для которой $a_{9} = 18$, $d=2$. Найдите $a_{11}$.
\end{taskBN}

\begin{taskBN}{126}
 Улитка ползет от одного дерева до другого. Каждый день она проползает на одно и то же расстояние больше, чем в предыдущий день. Известно, что за первый и последний дни улитка проползла в общей сложности 25 метров. Определите, сколько дней улитка потратила на весь путь, если расстояние между деревьями равно 75 метров.
\end{taskBN}

\begin{taskBN}{127}
Дана арифметическая прогрессия, для которой $a_{13} = 70$, $d=6$. Найдите $a_{18}$.
\end{taskBN}

\begin{taskBN}{128}
 Улитка ползет от одного дерева до другого. Каждый день она проползает на одно и то же расстояние больше, чем в предыдущий день. Известно, что за первый и последний дни улитка проползла в общей сложности 28 метров. Определите, сколько дней улитка потратила на весь путь, если расстояние между деревьями равно 126 метров.
\end{taskBN}

\begin{taskBN}{129}
Олегу надо решить 525 задач. Он решает на одно и то же количество задач больше по сравнению с предыдущим днем. Известно, что за первый день Олег решил 14 задач. Определите, сколько задач решил Олег в последний день, если со всеми задачами он справился за 15 дней.
\end{taskBN}

\begin{taskBN}{130}
Для арифметической прогрессии ${a_n}$ известно, что $a_{4} + a_{13} + a_{22}= -261$. Найдите $a_{13}$.
\end{taskBN}

\begin{taskBN}{131}
 Улитка ползет от одного дерева до другого. Каждый день она проползает на одно и то же расстояние больше, чем в предыдущий день. Известно, что за первый и последний дни улитка проползла в общей сложности 10 метров. Определите, сколько дней улитка потратила на весь путь, если расстояние между деревьями равно 20 метров.
\end{taskBN}

\begin{taskBN}{132}
Грузовик перевозит партию щебня массой 513 тонн, ежедневно увеличивая норму перевозки на одно и то же число тонн. Известно, что за первый день было перевезено 3 тонны щебня. Определите, сколько тонн щебня было перевезено на одиннадцатый день, если вся работа была выполнена за 18 дней.
\end{taskBN}

\begin{taskBN}{133}
Известно, что вся работа была выполнена за 31 день. Сколько приглашений было оформлено за 3-й день, если ежедневно Надежда оформляла на одно и то же количество приглашений больше по сравнению с предыдущим днём, а за 6-й день она оформила 19 приглашений, а Надежде необходимо было оформить 899 приглашений? 
\end{taskBN}

\begin{taskBN}{134}
Вся работа была выполнена за 28 дней. Сколько открыток было разослано за 6-й день, если ежедневно Яна рассылала на одно и то же количество открыток больше по сравнению с предыдущим днём, а за 3-й день она разослала 12 открыток, а Яне необходимо было разослать 1946 открыток? 
\end{taskBN}

\begin{taskBN}{135}
Грузовик перевозит партию щебня массой 405 тонн, ежедневно увеличивая норму перевозки на одно и то же число тонн. Известно, что за первый день было перевезено 6 тонн щебня. Определите, сколько тонн щебня было перевезено на деcятый день, если вся работа была выполнена за 15 дней.
\end{taskBN}

\begin{taskBN}{136}
 Бригада маляров красит забор длиной 170 метров, ежедневно увеличивая норму покраски на одно и то же число метров. Известно, что за первый и последний день в сумме бригада покрасила 68 метров забора.  Определите, сколько дней бригада маляров красила весь забор. 
\end{taskBN}

\begin{taskBN}{137}
Дана арифметическая прогрессия, для которой $a_{16} = 91$, $a_{8} = 43$. Найдите $a_{18}$.
\end{taskBN}

\begin{taskBN}{138}
Грузовик перевозит партию щебня массой 322 тонны, ежедневно увеличивая норму перевозки на одно и то же число тонн. Известно, что за первый день было перевезено 10 тонн щебня. Определите, сколько тонн щебня было перевезено на девятый день, если вся работа была выполнена за 14 дней.
\end{taskBN}

\begin{taskBN}{139}
 Бригада маляров красит забор длиной 100 метров, ежедневно увеличивая норму покраски на одно и то же число метров. Известно, что за первый и последний день в сумме бригада покрасила 40 метров забора.  Определите, сколько дней бригада маляров красила весь забор. 
\end{taskBN}

\begin{taskBN}{140}
 Улитка ползет от одного дерева до другого. Каждый день она проползает на одно и то же расстояние больше, чем в предыдущий день. Известно, что за первый и последний дни улитка проползла в общей сложности 12 метров. Определите, сколько дней улитка потратила на весь путь, если расстояние между деревьями равно 30 метров.
\end{taskBN}

\begin{taskBN}{141}
Найдите сумму первых 66 чётных натуральных чисел.
\end{taskBN}

\begin{taskBN}{142}
Дана арифметическая прогрессия, для которой $a_{4} = 17$, $a_{15} = 50$. Найдите разность арифметической прогрессии $d$.
\end{taskBN}

\begin{taskBN}{143}
Сколько писем было оформлено за 2-й день, если ежедневно Олеся оформляла на одно и то же количество писем больше по сравнению с предыдущим днём, при этом вся работа была выполнена за 26 дней, при этом Олесе нужно было оформить 1794 письма? За 12-й день она оформила 63 письма. 
\end{taskBN}

\begin{taskBN}{144}
Дана арифметическая прогрессия, для которой $a_{19} = 176$, $d=10$. Найдите $a_{17}$.
\end{taskBN}

\begin{taskBN}{145}
Турист идет из одного города в другой, каждый день проходя больше, чем в предыдущий день, на одно и то же расстояние. Известно, что за первый день турист прошел 3 километра. Определите, сколько километров прошел турист за пятый день, если весь путь он прошел за 11 дней, а расстояние между городами составляет 143 километра.
\end{taskBN}

\begin{taskBN}{146}
Васе надо решить 65 задач. Он решает на одно и то же количество задач больше по сравнению с предыдущим днем. Известно, что за первый день Вася решил 7 задач. Определите, сколько задач решил Вася в последний день, если со всеми задачами он справился за 5 дней.
\end{taskBN}

\begin{taskBN}{147}
Турист идет из одного города в другой, каждый день проходя больше, чем в предыдущий день, на одно и то же расстояние. Известно, что за первый день турист прошел 6 километров. Определите, сколько километров прошел турист за семнадцатый день, если весь путь он прошел за 24 дня, а расстояние между городами составляет 1248 километров.
\end{taskBN}

\begin{taskBN}{148}
Дана арифметическая прогрессия, для которой $a_{20} = 65$, $a_{19} = 62$. Найдите $a_{18}$.
\end{taskBN}

\begin{taskBN}{149}
Сколько писем было подписано за 11-й день, если вся работа была выполнена за 38 дней, а ежедневно Софья подписывала на одно и то же количество писем больше по сравнению с предыдущим днём, а за 4-й день она подписала 45 писем? При этом Софье необходимо было подписать 7600 писем. 
\end{taskBN}

\begin{taskBN}{150}
Дана арифметическая прогрессия, для которой $a_{20} = 102$, $a_{11} = 57$. Найдите $a_{19}$.
\end{taskBN}

\begin{taskBN}{151}
 Улитка ползет от одного дерева до другого. Каждый день она проползает на одно и то же расстояние больше, чем в предыдущий день. Известно, что за первый и последний дни улитка проползла в общей сложности 20 метров. Определите, сколько дней улитка потратила на весь путь, если расстояние между деревьями равно 50 метров.
\end{taskBN}

\begin{taskBN}{152}
 Бригада маляров красит забор длиной 244 метра, ежедневно увеличивая норму покраски на одно и то же число метров. Известно, что за первый и последний день в сумме бригада покрасила 61 метр забора.  Определите, сколько дней бригада маляров красила весь забор. 
\end{taskBN}

\begin{taskBN}{153}
 Бригада маляров красит забор длиной 106 метров, ежедневно увеличивая норму покраски на одно и то же число метров. Известно, что за первый и последний день в сумме бригада покрасила 53 метра забора.  Определите, сколько дней бригада маляров красила весь забор. 
\end{taskBN}

\begin{taskBN}{154}
Дана арифметическая прогрессия, для которой $a_{14} = -56$, $a_{10} = -40$. Найдите разность арифметической прогрессии $d$.
\end{taskBN}

\begin{taskBN}{155}
За 31-й день Анастасия разослала 227 открыток, вся работа была выполнена за 39 дней, ежедневно она рассылала на одно и то же количество открыток больше по сравнению с предыдущим днём. Сколько всего открыток необходимо было разослать Анастасии, если за 6-й день Анастасия разослала 52 открытки? 
\end{taskBN}

\begin{taskBN}{156}
Дана арифметическая прогрессия, для которой $a_{13} = -53$, $d=-4$. Найдите $a_{14}$.
\end{taskBN}

\begin{taskBN}{157}
Грузовик перевозит партию щебня массой 972 тонны, ежедневно увеличивая норму перевозки на одно и то же число тонн. Известно, что за первый день было перевезено 6 тонн щебня. Определите, сколько тонн щебня было перевезено на девятнадцатый день, если вся работа была выполнена за 24 дня.
\end{taskBN}

\begin{taskBN}{158}
Дана арифметическая прогрессия, для которой $a_{17} = 123$, $d=8$. Найдите $a_{13}$.
\end{taskBN}

\begin{taskBN}{159}
 Бригада маляров красит забор длиной 252 метра, ежедневно увеличивая норму покраски на одно и то же число метров. Известно, что за первый и последний день в сумме бригада покрасила 72 метра забора.  Определите, сколько дней бригада маляров красила весь забор. 
\end{taskBN}

\begin{taskBN}{160}
Грузовик перевозит партию щебня массой 308 тонн, ежедневно увеличивая норму перевозки на одно и то же число тонн. Известно, что за первый день было перевезено 8 тонн щебня. Определите, сколько тонн щебня было перевезено на шестой день, если вся работа была выполнена за 11 дней.
\end{taskBN}

\begin{taskBN}{161}
Дана арифметическая прогрессия, для которой $a_{3} = 10$, $a_{12} = 46$. Найдите разность арифметической прогрессии $d$.
\end{taskBN}

\begin{taskBN}{162}
Турист идет из одного города в другой, каждый день проходя больше, чем в предыдущий день, на одно и то же расстояние. Известно, что за первый день турист прошел 8 километров. Определите, сколько километров прошел турист за шестой день, если весь путь он прошел за 12 дней, а расстояние между городами составляет 228 километров.
\end{taskBN}

\begin{taskBN}{163}
Для арифметической прогрессии ${a_n}$ известно, что $a_{8} + a_{11} + a_{14}= -216$. Найдите $a_{11}$.
\end{taskBN}

\begin{taskBN}{164}
 Бригада маляров красит забор длиной 170 метров, ежедневно увеличивая норму покраски на одно и то же число метров. Известно, что за первый и последний день в сумме бригада покрасила 68 метров забора.  Определите, сколько дней бригада маляров красила весь забор. 
\end{taskBN}

\begin{taskBN}{165}
 Улитка ползет от одного дерева до другого. Каждый день она проползает на одно и то же расстояние больше, чем в предыдущий день. Известно, что за первый и последний дни улитка проползла в общей сложности 22 метра. Определите, сколько дней улитка потратила на весь путь, если расстояние между деревьями равно 99 метров.
\end{taskBN}

\begin{taskBN}{166}
Пете надо решить 208 задач. Он решает на одно и то же количество задач больше по сравнению с предыдущим днем. Известно, что за первый день Петя решил 12 задач. Определите, сколько задач решил Петя в последний день, если со всеми задачами он справился за 8 дней.
\end{taskBN}

\begin{taskBN}{167}
Грузовик перевозит партию щебня массой 406 тонн, ежедневно увеличивая норму перевозки на одно и то же число тонн. Известно, что за первый день было перевезено 3 тонны щебня. Определите, сколько тонн щебня было перевезено на седьмой день, если вся работа была выполнена за 14 дней.
\end{taskBN}

\begin{taskBN}{168}
Маше надо подписать 240 открыток. Ежедневно она подписывает на одно и то же количество открыток больше по сравнению с предыдущим днем. Известно, что за первый день Маша подписала 6 открыток. Определите, сколько открыток было подписано за четвёртый день, если вся работа была выполнена за 10 дней.
\end{taskBN}

\begin{taskBN}{169}
Грузовик перевозит партию щебня массой 680 тонн, ежедневно увеличивая норму перевозки на одно и то же число тонн. Известно, что за первый день было перевезено 8 тонн щебня. Определите, сколько тонн щебня было перевезено на одиннадцатый день, если вся работа была выполнена за 17 дней.
\end{taskBN}

\begin{taskBN}{170}
Грузовик перевозит партию щебня массой 779 тонн, ежедневно увеличивая норму перевозки на одно и то же число тонн. Известно, что за первый день было перевезено 5 тонн щебня. Определите, сколько тонн щебня было перевезено на тринадцатый день, если вся работа была выполнена за 19 дней.
\end{taskBN}

\begin{taskBN}{171}
Турист идет из одного города в другой, каждый день проходя больше, чем в предыдущий день, на одно и то же расстояние. Известно, что за первый день турист прошел 8 километров. Определите, сколько километров прошел турист за восьмой день, если весь путь он прошел за 15 дней, а расстояние между городами составляет 435 километров.
\end{taskBN}

\begin{taskBN}{172}
Дана арифметическая прогрессия, для которой $a_1 = -6$, $d=-2$. Найдите $a_{10}$.
\end{taskBN}

\begin{taskBN}{173}
Турист идет из одного города в другой, каждый день проходя больше, чем в предыдущий день, на одно и то же расстояние. Известно, что за первый день турист прошел 9 километров. Определите, сколько километров прошел турист за седьмой день, если весь путь он прошел за 14 дней, а расстояние между городами составляет 308 километров.
\end{taskBN}

\begin{taskBN}{174}
 Улитка ползет от одного дерева до другого. Каждый день она проползает на одно и то же расстояние больше, чем в предыдущий день. Известно, что за первый и последний дни улитка проползла в общей сложности 19 метров. Определите, сколько дней улитка потратила на весь путь, если расстояние между деревьями равно 57 метров.
\end{taskBN}

\begin{taskBN}{175}
Вся работа была выполнена за 44 дня, при этом ежедневно Екатерина рассылала на одно и то же количество приглашений больше по сравнению с предыдущим днём. Сколько всего приглашений было поручено разослать Екатерине, если за 2-й день она разослала 14 приглашений, а за 20-й день Екатерина разослала 194 приглашения? 
\end{taskBN}

\begin{taskBN}{176}
 Улитка ползет от одного дерева до другого. Каждый день она проползает на одно и то же расстояние больше, чем в предыдущий день. Известно, что за первый и последний дни улитка проползла в общей сложности 30 метров. Определите, сколько дней улитка потратила на весь путь, если расстояние между деревьями равно 135 метров.
\end{taskBN}

\begin{taskBN}{177}
 Рабочие прокладывают тоннель длиной 286 метров, ежедневно увеличивая норму прокладки на одно и то же число метров. Известно, что за первый день рабочие проложили 6 метров туннеля.  Определите, сколько метров туннеля проложили рабочие в последний день, если вся работа была выполнена за 11 дней.
\end{taskBN}

\begin{taskBN}{178}
 Улитка ползет от одного дерева до другого. Каждый день она проползает на одно и то же расстояние больше, чем в предыдущий день. Известно, что за первый и последний дни улитка проползла в общей сложности 24 метра. Определите, сколько дней улитка потратила на весь путь, если расстояние между деревьями равно 108 метров.
\end{taskBN}

\begin{taskBN}{179}
 Бригада маляров красит забор длиной 273 метра, ежедневно увеличивая норму покраски на одно и то же число метров. Известно, что за первый и последний день в сумме бригада покрасила 78 метров забора.  Определите, сколько дней бригада маляров красила весь забор. 
\end{taskBN}

\begin{taskBN}{180}
Для арифметической прогрессии ${a_n}$ известно, что $a_{5} + a_{17} + a_{29}= 135$. Найдите $a_{17}$.
\end{taskBN}

\begin{taskBN}{181}
Ире надо подписать 468 открыток. Ежедневно она подписывает на одно и то же количество открыток больше по сравнению с предыдущим днем. Известно, что за первый день Ира подписала 9 открыток. Определите, сколько открыток было подписано за одиннадцатый день, если вся работа была выполнена за 18 дней.
\end{taskBN}

\begin{taskBN}{182}
 Бригада маляров красит забор длиной 288 метров, ежедневно увеличивая норму покраски на одно и то же число метров. Известно, что за первый и последний день в сумме бригада покрасила 64 метра забора.  Определите, сколько дней бригада маляров красила весь забор. 
\end{taskBN}

\begin{taskBN}{183}
 Рабочие прокладывают тоннель длиной 102 метра, ежедневно увеличивая норму прокладки на одно и то же число метров. Известно, что за первый день рабочие проложили 7 метров туннеля.  Определите, сколько метров туннеля проложили рабочие в последний день, если вся работа была выполнена за 6 дней.
\end{taskBN}

\begin{taskBN}{184}
Грузовик перевозит партию щебня массой 322 тонны, ежедневно увеличивая норму перевозки на одно и то же число тонн. Известно, что за первый день было перевезено 10 тонн щебня. Определите, сколько тонн щебня было перевезено на восьмой день, если вся работа была выполнена за 14 дней.
\end{taskBN}

\begin{taskBN}{185}
Грузовик перевозит партию щебня массой 777 тонн, ежедневно увеличивая норму перевозки на одно и то же число тонн. Известно, что за первый день было перевезено 7 тонн щебня. Определите, сколько тонн щебня было перевезено на шестнадцатый день, если вся работа была выполнена за 21 день.
\end{taskBN}

\begin{taskBN}{186}
Лене надо подписать 648 открыток. Ежедневно она подписывает на одно и то же количество открыток больше по сравнению с предыдущим днем. Известно, что за первый день Лена подписала 4 открытки. Определите, сколько открыток было подписано за девятнадцатый день, если вся работа была выполнена за 24 дня.
\end{taskBN}

\begin{taskBN}{187}
Для арифметической прогрессии ${a_n}$ известно, что $a_{10} + a_{19} + a_{28}= 201$. Найдите $a_{19}$.
\end{taskBN}

\begin{taskBN}{188}
 Бригада маляров красит забор длиной 116 метров, ежедневно увеличивая норму покраски на одно и то же число метров. Известно, что за первый и последний день в сумме бригада покрасила 58 метров забора.  Определите, сколько дней бригада маляров красила весь забор. 
\end{taskBN}

\begin{taskBN}{189}
Грузовик перевозит партию щебня массой 390 тонн, ежедневно увеличивая норму перевозки на одно и то же число тонн. Известно, что за первый день было перевезено 5 тонн щебня. Определите, сколько тонн щебня было перевезено на девятый день, если вся работа была выполнена за 15 дней.
\end{taskBN}

\begin{taskBN}{190}
Дана арифметическая прогрессия, для которой $a_{15} = 18$, $a_{9} = 12$. Найдите разность арифметической прогрессии $d$.
\end{taskBN}

\begin{taskBN}{191}
 Улитка ползет от одного дерева до другого. Каждый день она проползает на одно и то же расстояние больше, чем в предыдущий день. Известно, что за первый и последний дни улитка проползла в общей сложности 15 метров. Определите, сколько дней улитка потратила на весь путь, если расстояние между деревьями равно 30 метров.
\end{taskBN}

\begin{taskBN}{192}
Найдите сумму первых 17 чётных натуральных чисел.
\end{taskBN}

\begin{taskBN}{193}
 Бригада маляров красит забор длиной 400 метров, ежедневно увеличивая норму покраски на одно и то же число метров. Известно, что за первый и последний день в сумме бригада покрасила 80 метров забора.  Определите, сколько дней бригада маляров красила весь забор. 
\end{taskBN}

\begin{taskBN}{194}
Грузовик перевозит партию щебня массой 205 тонн, ежедневно увеличивая норму перевозки на одно и то же число тонн. Известно, что за первый день было перевезено 7 тонн щебня. Определите, сколько тонн щебня было перевезено на четвёртый день, если вся работа была выполнена за 10 дней.
\end{taskBN}

\begin{taskBN}{195}
Дана арифметическая прогрессия, для которой $a_{15} = -46$, $a_{18} = -55$. Найдите $a_{9}$.
\end{taskBN}

\begin{taskBN}{196}
 Улитка ползет от одного дерева до другого. Каждый день она проползает на одно и то же расстояние больше, чем в предыдущий день. Известно, что за первый и последний дни улитка проползла в общей сложности 20 метров. Определите, сколько дней улитка потратила на весь путь, если расстояние между деревьями равно 50 метров.
\end{taskBN}

\begin{taskBN}{197}
Дана арифметическая прогрессия, для которой $a_{16} = 120$, $a_{17} = 128$. Найдите $a_{11}$.
\end{taskBN}

\begin{taskBN}{198}
 Улитка ползет от одного дерева до другого. Каждый день она проползает на одно и то же расстояние больше, чем в предыдущий день. Известно, что за первый и последний дни улитка проползла в общей сложности 14 метров. Определите, сколько дней улитка потратила на весь путь, если расстояние между деревьями равно 28 метров.
\end{taskBN}

\begin{taskBN}{199}
За сколько дней была выполнена вся работа, если за 2-й день Вероника разослала 18 писем, при этом за 7-й день она разослала 53 письма? Веронике нужно было разослать 689 писем, а ежедневно Вероника рассылала на одно и то же количество писем больше по сравнению с предыдущим днём. 
\end{taskBN}

\begin{taskBN}{200}
Грише надо решить 55 задач. Он решает на одно и то же количество задач больше по сравнению с предыдущим днем. Известно, что за первый день Гриша решил 7 задач. Определите, сколько задач решил Гриша в последний день, если со всеми задачами он справился за 5 дней.
\end{taskBN}
\end{document}