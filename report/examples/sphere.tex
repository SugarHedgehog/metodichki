

\subsubsection*{№27163}

\examples{Радиусы двух шаров равны $17$ и $7$. Найдите радиус шара, площадь поверхности которого равна сумме площадей поверхности двух данных шаров. Ответ умножьте на $\sqrt{2}$.}{26}{544171422141239n0}{0.35}

\examples{Радиусы двух шаров равны $10$ и $5$. Найдите радиус шара, площадь большого круга которого равна сумме площадей больших кругов двух данных шаров. Ответ разделите на $\sqrt{5}$.}{5}{8023855670545808n0}{0.35}

\subsubsection*{№27072}

\examples{Во сколько раз объём первого шара больше объёма второго шара, если радиус первого шара в 7 раз больше, чем радиус второго шара?}{343}{544171422141239n0}{0.35}

\examples{Площадь поверхности первого шара в 36 раз меньше, чем площадь поверхности второго шара. Во сколько раз площадь большого круга первого шара меньше площади большого круга второго шара?}{36}{8952607108443624n0}{0.35}

\subsubsection*{№27125}

\examples{Радиусы трёх шаров равны $\sqrt[3]{12}$, $\sqrt[3]{14}$, $\sqrt[3]{2}$. Найдите радиус шара, объем которого равен сумме их объемов.}{3}{845422661216762n0}{0.25}

\examples{Радиусы четырёх шаров равны $\sqrt[3]{14}$, $2$, $\sqrt[3]{2}$, $\sqrt[3]{3}$. Найдите радиус шара, объем которого равен сумме их объемов.}{3}{1086043814507218n0}{0.25}
