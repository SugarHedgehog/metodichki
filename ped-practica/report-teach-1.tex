\documentclass[a4paper,12pt]{article}
\usepackage[utf8]{inputenc}
\usepackage[russian]{babel}
\usepackage{geometry}
\geometry{top=10mm, bottom=10mm, left=10mm, right=10mm}
\pagestyle{empty}

\begin{document}
\begin{center}
    \textbf{Муниципальное бюджетное общеобразовательное учреждение}\\
    \textbf{«Средняя общеобразовательная школа № 10» г. Нижняя Салда Свердловской области}\\
    \end{center}
\begin{center}
    \textbf{\large ОТЗЫВ ОБ УЧЕБНО-ВОСПИТАТЕЛЬНОЙ РАБОТЕ}\\
    \textbf{студентки-практикантки 5-го курса математического факультета}\\
    \textbf{Суматохиной Александры Сергеевны}
    \end{center}
\vspace{10mm}

Суматохина Александра Сергеевна прошла педагогическую практику в МБОУ «Средняя общеобразовательная школа № 10» с 2 сентября по 28 сентября 2024 года. В течение этого периода Александра выполняла функции учителя математики в 5А классе и активно включилась в учебно-воспитательный процесс.

На занятиях, которые проводила Александра Сергеевна, наблюдалась высокая степень вовлечённости учащихся. Она проявила себя как грамотный организатор, тщательно продумывая каждый урок и классное мероприятие. Планирование и проведение урока по математике по теме "Углы: острые, прямые, тупые; измерение углов с помощью транспортира" было выполнено на высоком уровне. Александра также разработала и провела классное мероприятие по профилактике детского дорожно-транспортного травматизма, включающее презентацию, интерактивные игры и практические задания. Ее подход к планированию и проведению мероприятия был структурированным и нацеленным на повышение осведомленности детей о правилах дорожного движения и безопасном поведении на дороге.

Александра Сергеевна умело использовала различные методы и технологии, такие как макеты дорог и интерактивные игры, чтобы сделать материал более доступным и интересным для учащихся.

\vspace{5mm}

\textbf{Результаты практики:} За период своей педагогической деятельности Александра Сергеевна проявила высокий уровень профессиональных навыков и глубокую вовлеченность в учебно-воспитательный процесс. Считаем, что её заслуги и результаты позволяют рекомендовать оценку за практику \textbf{«отлично»}.

\vspace{10mm}

\noindent
\begin{tabbing}
\hspace{4cm} \= \kill
Учитель математики \> \underline{\hspace{5cm}} Федорова Елена Андреевна
\end{tabbing}
Дата \underline{\hspace{5cm}}

\vspace{5mm}

\noindent
\textbf{Предмет: Математика}\\
\textbf{Класс: 5А}\\

\end{document}
