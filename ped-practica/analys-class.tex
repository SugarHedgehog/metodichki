\documentclass[a4paper,12pt]{article}
\usepackage[utf8]{inputenc}
\usepackage[russian]{babel}
\usepackage{graphicx}
\usepackage{hyperref}
\usepackage{enumitem}
\usepackage{geometry}
\geometry{top=10mm, bottom=10mm, left=10mm, right=10mm}
\pagestyle{empty}

\begin{document}

\begin{center}
    \large{\textbf{Анализ урока по теме "Углы: острые, прямые, тупые; измерение углов с помощью транспортира" для 5 класса}}
\end{center}

\subsection*{Общий обзор}

Урок по теме "Углы: острые, прямые, тупые; измерение углов с помощью транспортира" для 5 класса был проведен с целью ознакомления учащихся с видами углов и измерением углов с помощью транспортира. В целом, урок достиг своих целей, несмотря на некоторые сложности.

\subsection*{Сложности}

\begin{itemize}
    \item Вначале урока возникла проблема с авторитетом преподавателя, так как учащиеся не воспринимали меня как учителя и реагировали только на замечания классного руководителя.
    \item Необходимо было обеспечить контроль за написанием конспектов, чтобы гарантировать полное понимание материала.
\end{itemize}

\subsection*{Положительные стороны}

\begin{itemize}
    \item Учащиеся активно выполняли задания в парах, что свидетельствует о эффективности практических занятий.
    \item Урок прошел в спокойной обстановке, без значительных нарушений дисциплины.
\end{itemize}

\subsection*{Рекомендации для будущих занятий}

\begin{itemize}
    \item В будущих занятиях необходимо уделить больше внимания адаптации материала к возрасту и интересам учащихся.
    \item Необходимо разработать дополнительные меры контроля за написанием конспектов, чтобы обеспечить полное понимание материала.
\end{itemize}

\subsection*{Заключение}

Урок по теме "Углы: острые, прямые, тупые; измерение углов с помощью транспортира" для 5 класса был успешным, несмотря на некоторые сложности. Учащиеся получили полезную информацию о видах углов и измерении углов с помощью транспортира, и были заинтересованы в практических занятиях. В будущих занятиях необходимо уделить больше внимания адаптации материала к возрасту и интересам учащихся, а также разработать дополнительные меры контроля за написанием конспектов.

\end{document}
