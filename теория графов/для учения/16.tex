\textbf{Функция Гранди}

\begin{itemize}
    \item \textbf{Функция Гранди:} Для конечного графа $(X, \Gamma)$ функция $g(x)$ — это наименьшее неотрицательное целое число, не принадлежащее множеству $g(\Gamma x) = \{g(y) \mid y \in \Gamma x\}$.
    \item \textbf{Пример 1:} Граф на рис. 3-3 допускает две функции Гранди. Если $\varGamma x = \{y_1, y_2, \ldots\}$, то $g(x)$ — наименьшее число, отличное от $g(y_1), g(y_2)$.
    \item \textbf{Пример 2:} Граф на рис. 3-2 допускает единственную функцию Гранди $g(x)$, где $g(x) = o(x)$ для $x \neq a$, а в $a$ принимает значение $\omega$ (трансфинитное число).
    \item \textbf{Теорема 5:} Прогрессивно конечный граф допускает одну функцию Гранди $g(x)$, и $g(x) \leq o(x)$.
    \item \textbf{Доказательство:} Индукция по множествам:
    \begin{align*}
        X(0) &= \{x \mid \varGamma x = \emptyset\}, \\
        X(1) &= \{x \mid \varGamma x \subseteq X(0)\}, \\
        X(2) &= \{x \mid \varGamma x \subseteq X(1)\}.
    \end{align*}
    \item \textbf{Теорема 6:} Если $|X| < \infty$, то $g(x) \leq \Gamma$. Если $g(x) = n$, то $g$ принимает в $\varGamma x$ все значения $0, 1, 2, \ldots, n-1$, следовательно, $|X| \geq n - g(x)$.
    \item \textbf{Заключение:} Для $\Gamma$-конечного или прогрессивно ограниченного графа значения $g(x)$ остаются конечными.
\end{itemize}