\subsection{Цикломатическое число графа}

\noindent\textbf{Мультиграф} $(X,U)$ -- пара из множества вершин $X$ и множества рёбер $U$, где пара вершин может соединяться несколькими рёбрами.

\noindent\textbf{Важные числовые характеристики:}
\begin{itemize}
    \item Для мультиграфа $G$ с $n$ вершинами, $m$ рёбрами, $p$ компонентами:
    \begin{itemize}
        \item Ранг графа: $\rho(G) = n - p$
        \item Цикломатическое число: $\nu(G) = m - n + p = m - \rho(G)$
    \end{itemize}
\end{itemize}

\noindent\textbf{Теорема 1.} При добавлении ребра между $a$ и $b$:
\begin{itemize}
    \item Если $a,b$ соединены цепью или совпадают:
    \begin{itemize}
        \item $\rho(G') = \rho(\bar{G})$
        \item $\nu(G') = \nu(\bar{G}) + 1$
    \end{itemize}
    \item Иначе:
    \begin{itemize}
        \item $\rho(G) = \rho(\bar{G}) + 1$
        \item $\nu(G') = \nu(\bar{G})$
    \end{itemize}
\end{itemize}

\noindent\textbf{Векторное представление циклов:}
\begin{itemize}
    \item Каждому ребру присваивается ориентация
    \item Для цикла $\mu$: $c^k = r_k - s_k$, где $r_k,s_k$ -- число проходов по/против ориентации
    \item Цикл представляется вектором $(c^1,\ldots,c^m)$
    \item Циклы независимы $\Leftrightarrow$ их векторы линейно независимы
\end{itemize}

\noindent\textbf{Теорема 2.} Цикломатическое число $\nu(G)$ равно максимальному количеству независимых циклов.

\noindent\textbf{Следствия:}
\begin{enumerate}
    \item $\nu(G)=0$ $\Leftrightarrow$ граф без циклов
    \item $\nu(G)=1$ $\Leftrightarrow$ граф содержит ровно один цикл
\end{enumerate}

\noindent\textbf{Теорема 3.} В сильно связном графе цикломатическое число равно максимальному количеству независимых контуров.