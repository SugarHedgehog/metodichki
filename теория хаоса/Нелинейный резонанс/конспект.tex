\documentclass[10pt]{article}
\usepackage[russian]{babel}
\usepackage[utf8]{inputenc}
\usepackage[T2A]{fontenc}
\usepackage{amsmath}
\usepackage{graphicx}

\begin{document}

\section*{Конспект по теме: Нелинейный резонанс}

\subsection*{Основные понятия}

1. **Линейный резонанс**:
   - Характерен для линейного осциллятора.
   - Амплитуда колебаний увеличивается, когда частота воздействия $\omega$ близка к собственной частоте $\omega_0$.
   - Уравнение вынужденных колебаний: 
     \[
     \ddot{x} + \omega_0^2 x = f \cos \omega t
     \]
   - Амплитуда $A$ определяется как:
     \[
     |A| = \frac{f}{|\omega^2 - \omega_0^2|}
     \]

2. **Нелинейный резонанс**:
   - Частота свободных колебаний зависит от амплитуды (неизохронность).
   - Пример: кубический осциллятор
     \[
     \ddot{x} + \omega_0^2 x + \beta x^3 = 0
     \]
   - Частота сдвигается на $\Delta \omega(A) \approx \frac{3 \beta A^2}{8 \omega_0}$.
   - Амплитуда вынужденных колебаний:
     \[
     |A| \approx \frac{f}{|\omega^2 - \omega_0^2 - \frac{3 \beta A^2 \omega_0^2}{4}|}
     \]

\subsection*{Графики и их интерпретация}

- Рис. 14.4 демонстрирует резонансные кривые:
  - (a) Линейный осциллятор.
  - (б) Нелинейный осциллятор с $\beta < 0$.
  - (в) Нелинейный осциллятор с $\beta > 0$.
- Различия в наклоне кривых зависят от знака $\beta$.

\subsection*{Особенности нелинейного резонанса}

- Амплитуда вынужденных колебаний остается конечной даже при совпадении $\omega$ и $\omega_0$.
- Нелинейный сдвиг частоты нарушает резонансные условия.
- Зависимость амплитуды от частоты может быть неоднозначной.

\subsection*{Консервативные и диссипативные осцилляторы}

- Консервативный осциллятор сохраняет «память» о начальном состоянии.
- Движение может быть периодическим или квазипериодическим в зависимости от соотношения частот.
- В диссипативных системах, при малой диссипации, устанавливается режим вынужденных колебаний, не зависящий от начальных условий.

\end{document}