\documentclass[a4paper, 12pt]{article}
\usepackage{fontspec}
\usepackage{polyglossia}
\setmainfont{CMU Serif}
\newfontfamily{\cyrillicfont}{CMU Serif}
\setsansfont{CMU Sans Serif}
\newfontfamily{\cyrillicfontsf}{CMU Sans Serif}
\setmonofont{CMU Typewriter Text}
\newfontfamily{\cyrillicfonttt}{CMU Typewriter Text}
\setdefaultlanguage{russian}
\usepackage[left=1cm,bottom=2cm,top=1cm,right=1cm]{geometry}
\begin{document}
\pagestyle{empty}
\subsection*{\textbf{1. Предмет философии.}}

Философия возникла в Древней Греции в 6-м веке до н.э. как попытка совместить мудрость и рациональное мышление.
Имеем первое определение философии:
философия - это попытка выразить мудрость в формах рациональной мысли.
Рациональная мысль порождает знание.
Знание отвечает на вопрос: как устроен мир, каковы законы природы и общества?
Знание объективно, отчуждаемо, легко передаётся, не связано с волей и опредмечиваемо.
Мудрость интересуется: каков смысл человеческой жизни, ради чего жить?
Мудрость субъективна, неотчуждаема, непередаваема (можно передать рассказ о мудрости, но не саму мудрость), срощена с душой и волей и непредметна.

Во времена зарождения философии мудрость была мифична, но миф утрачивал свой авторитет.
Итак, предстояло:
1. Осознать и просветлить мудрость мыслью.
2. Сделать мысль мудрой.

Соединение знания и мудрости - вопрос выживания цивилизации, но это сложно, так как:
1. Мудрость нельзя полностью выразить знанием.
2. Мудрость изменяется со временем, каждая эпоха требует своей мудрости.

Рациональное мышление - это мышление с помощью понятий.
Сущность (субстанция) - это порождающее основание.
Предельное основание - это основание, ничем не порождённое.

Имеем второе определения:
философия есть попытка объяснить мир исходя из предельных оснований.

Докажем, что два определения эквивалентны.
Действительно, основания будут предельными тогда и только тогда, когда соединяют знания и мудрость.


\subsection*{\textbf{2. Основные философские направления и принципы их выделения. «Основной вопрос» философии.}}

В основе классификации философских учений лежит их подход к решению фундаментальных, всеобщих проблем.
Так как при развитии философии не только изучаются существующие проблемы, но и открываются новые, то такая классификация не может быть абсолютно полна и всеобъемлюща.

1. По способу соединения знания и мудрости.
а) Придерживающиеся сохранения действующей мудрости, пусть и в ущерб рациональному знанию - например, Ницше, восточные философы и т. д.
Истина ищется в интуиции и мистическом опыте.
Соединение знания и мудрости нужно только на пути к истине, и при её достижении философии можно отбросить.
б) Рационализировать всё, облечь мудрость в понятия - например, Декарт.
Истина ищется в понятиях и теории.

2. По выбранным ценностям.
Например, в античной философии ценность - космос (а не человек!), для Декарта и Канта основная ценность - разум.

3. По ответу на т. наз. "основной вопрос философии" - о выборе предельной субстанции в двух областях:

3.1. При обосновании бытия.
Что первично -  дух или материя?
а) Материализм: материя первична.
Пример: Маркс, Локк.
б) Идеализм субъективный: первичен дух, являющийся обобщением человеческого "Я".
Пример: Фихте.
в) Идеализм объективный: первичен мировой, надчеловеческий дух, цели и логика которого отличны от целей и логики отдельных индивидов.
Пример: Гегель.

3.2. При обосновании познания.
а) Эмпиризм: первичен опыт.
Примеры: Локк, Юм, Беркли.
б) Рационализм: первичны разум и теория.
Пример: Декарт.

\subsection*{\textbf{3. Сущность мифа и религии.}}

В настоящее время часто под словом "миф" подразумевают мифические сюжеты.
На самом деле миф - это не набор эпосов, а способ восприятия мира.
Древний человек был слаб и не мог воспринимать мир "как чужое" и покорять его.
А чтобы воспринимать мир "как своё", нужно было его одушевить.
Древний человек (и далее вплоть до Аристотеля включительно) воспринимал мир как единое совершенное, завершённое и всеподчиняющее целое.
Такой мир называется космосом.
Кроме того, неотъемлемой частью мифологического мышления является мистическая сопричастность - явление неразличения смысловой и природной связи.
И правда, если весь мир - космос, человек - лишь его часть, разум - это космическая структура, то как можно отличить связь, установленную природой, от связи, установленной разумом?
Просить у окружающего мира было унизительно, договариваться - рискованно (а вдруг не получится?), и человек изобрёл магию.
Магия - это попытка использования ассоциативной связи для воздействия на окружающий мир.
Следует отметить, что мифологическое мышление частично сохраняется в приметах.
Мифологический человек не мог совладать со своей свободой, и ему приходилось придумывать себе контролёра (например, Афину, которая била по рукам при попытке достать меч).
Так в мифе появилась мораль, требующая не совершать необдуманных поступков.

Первым шагом к свободе от мифа стала религия.
В религии обожествляется уже не природа, а человеческая свобода.
Бог развитой религии - трансцендентен, потусторонен, а человек создан по образу и подобию.
Такой подход освобождает, выделяет человека из природы, даёт право говорит от собственного "Я".
Религия позволяет справиться со страхом смерти и любой безнадёгой.

Ещё большУю свободу даёт философия.
В религии есть догматы, которые требуют веры и сомнению не подлежат, что вызывает определённую духовную несвободу.
В философии догматов нет.
Философия тоже помогает достойно встречать безнадёжные ситуации.
Например, в случае, когда человек поступил честно, а его предали, учение Канта будет утверждать, что он прав.


\subsection*{\textbf{4. Первые философские учения.}}

Первые философские учения (как и вообще настоящая наука) зародились в Древней Греции.
В силу политического устройства (демократии) там были критицизм и доказательность, которых не было на деспотическом Востоке.
Поэтому за знаниями и мудростью на Восток ездили, а в Греции из них собирали науку.
Рациональное мышление уже начинало использоваться, но было достаточно слабо, и потому вместо понятий использовались образы, называемые смыслообразами.
Мышление смыслообразами - это плохо, потому что нестрого.

Первое философское учение выдвинул Фалес: "всё есть вода".
Заслуга Фалеса в том, что он впервые попытался найти первоначало, отказавшись от сотворения мира в ходе разборок мифологических божеств.

Анаксимандр увидел, что фалесовское первоначало обладает определёнными качествами, и потому получить объект с другими качествами из него проблематично.
Тогда Анаксимандр ввёл апейрон - "беспредельное" первоначало, не обладающее никакими чувственно воспринимаемыми качествами.
Апейрон рождает противоположности, как именно - Анаксимандр не уточняет.

Анаксимен считал, что первоначало - воздух, а все физические процессы есть сгущение и разрежение воздуха.
Эмпедокл свёл всю природу к четырём первоэлементам - Земле, Воздуху, Воде и Огню, соединением и разъединением которых управляют Любовь и Вражда.

Наконец, Гераклит достиг вершины философии смыслообразов.
Первоначалом он взял Огонь, который управляется Логосом.
Логос - это идея всеобщей закономерности, которая связывает противоположности.
При соединении противоположностей возникает Гармония.
Огонь олицетворяет текучесть, Логос - порядок, а космос обновляется сгоранием, как феникс.
Гераклит первый соединил порядок с изменчивостью.


\subsection*{\textbf{5. Парменидовское бытие, его значение для развития философской мысли.}}

Парменид просто и красиво ввёл первый теоретический философский объект - бытие.
Бытие - это то, что есть для мысли, т. е. то, во что можно мысль упереть.

Как ни странно, это утверждение приводит к некоторому разнообразию следствий.

Во-первых, небытия нет.
Это доказывается от противного.

Во-вторых, бытие вечно и неизменно, так как не имеет градаций и не может скатиться в небытие.

В-третьих, бытие завершённо, так как объясняет только само себя.

Мир объяснить бытием нельзя, так как в бытии нет ни движения, ни множественности, которые вроде как есть в мире.
Ученик Парменида - Зенон - решил показать, что множественности и движения нет и в мире,  помощью парадоксов.

* Парадокс летящей стрелы.
Летящая стрела летит - следовательно, она движется.
С другой стороны, летящая стрела существует - следовательно, она находится в некотором месте пространства.
Определив это место, мы получим, что стрела неподвижна.
* Парадокс бесконечно малого и бесконечно большого.
Если существует множественность, то каждая вещь бесконечно мала и бесконечно велика одновременно.
Действительно, в таком случае любую вещь можно разделить бесконечное число раз, и получится бесконечно большое количество бесконечно малых частей.
Раз вещь состоит из бесконечно малых частей, то и сама она бесконечно мала, а раз вещь состоит из бесконечного числа частей, то она бесконечно велика.
Таким образом, любая вещь бесконечно мала и бесконечно велика одновременно.


\subsection*{\textbf{6. Философия Демокрита.}}
Демокрит сумел применить парменидовское бытие для объяснения вещей (множественных и движущихся).
Он разбил бытие небытием на атомы.
Здесь отсутствует противоречие с тем, что небытия нет: небытия по-прежнему нет для мысли, мысль Демокрита в него не упирается.
Атомы неделимы, неизменны и тупы.
Атомы носятся в пустоте, сталкиваясь друг с другом.
Возникновение вещи - соединение атомов, уничтожение - разъединение.
Таким образом:
1) Легализовались движение и множественность, но не объяснялось, откуда они возникают, потому что каждый атом также бессмертен и неизменен, как парменидовское бытие.
Атомы внутри себя не содержат источники движения.
2) Оставалось неясно, почему атомы соединяются и разъединяются?
3) Не было понятия целостной вещи, у вещей нет качеств, а человек получался механической куклой.

\subsection*{\textbf{7. Сократ и софисты.}}

И Сократ, и софисты сами пользовались разумом и агитировали других греков делать то же самое.

Софисты - это учителя риторики, ораторского искусства, которые учили греков пользоваться разумом в целях отстаивания собственных интересов, в частности, в суде.
Для софистов не было безусловных оснований, они часто меняли основания даже в пределах одного рассуждения.

И софисты, и Сократ были солидарны в том, что старая, традиционная мораль несовершенна, так как правило, основанное на отдельных поступках, лишено всеобщности, а потому не может применяться абсолютно.
Софисты утверждали, что относительно вообще всё.

Сократ же утверждал, что существуют некие безусловные основания - идеи, которые можно встретить только в разуме.
Идеи задают структуру парменидовского бытия.
Идея - это принцип космического порядка, открывающийся в разуме, но не зависящий от человека.
Идея добра, к примеру, есть родовое качество всех добрых поступков.
Сократовские идеи, в отличие от демокритовских атомов, связаны между собой.

Сократ предлагал строить мораль не на основе традиций, а на основе идей справедливости, нравственности и т.д.


\subsection*{\textbf{8. Философия Платона. Платоновская теория идей.}}
Есть 2 пути мышления: путь мнения и путь истины.
На пути мнения мысль подчиняется диктату чувств.
На пути истины - работает с предметом, который сама определяет.

Парменид просто и красиво ввёл первый теоретический философский объект - бытие.
Бытие - это то, что есть для мысли, т. е. то, во что можно мысль упереть; предмет, полностью определённый строгой мыслью.
Без фокусировки на бытии мысль не может осуществиться.
Бытие обладает фундаментальным свойством - оно есть.

Как ни странно, это утверждение приводит к некоторому разнообразию следствий.

Во-первых, небытия нет. Это доказывается от противного.

Во-вторых, бытие вечно и неизменно, так как не имеет градаций и не может скатиться в небытие.

В-третьих, бытие завершённо, так как объясняет только само себя, единственно (Парменид уподобляет бытие шару - совершеннейшему из трёхмерных тел с точки зрения древних греков).
Платон расширил истолкование идеи.
Платоновская идея, в отличие от сократовской, есть сущность не только поступка, но и вещи.
Идея вещи - вне вещи, в мире идей.
Вещь есть жалкое подражание идеи, потому что материя, в которой отпечатывается идея, изначально несовершенна.
Идея отвечает за то в вещи, что можно выразить логически, материя - за то, что можно ощутить.


\subsection*{\textbf{9. Философия Аристотеля. Аристотелевский космос.}}
Чем отличаются идеи по Платону и по Аристотелю?

Платоновская идея находится вне вещи, в мире идей. Вещь является лишь жалким подражанием идее.
Аристотелевская идея, называемая формой, содержится в вещи, вещь является полноценным воплощением своей формы.

Что такое материя по Аристотелю?

Материя по аристотелю - это чувственно воспринимаемая вещь, которая содержит возможности соответствующих форм.

Где прячутся возможные формы по Аристотелю?
В материи.

Что такое движение по Аристотелю?

Всякое движение есть переход из возможного в действительное, направляемый целью.

Чем движение по Аристотелю отличается от движения по Демокриту?

Демокритовкое движение атомов бесцельно, они тупо толкаются и отскакивают. Аристотелевское же движение целенаправленно.

Чем отличается материя по Платону и по Аристотелю?

Материя Аристотеля бескачественна и согласна отпечатать всё. Криво, убого, но отпечатать. Аристотелеская же материя куда более капризна. Она может принять только те формы, которые содержатся в ней в модусе возможного, зато делает это относительно хорошо.

Как растёт дуб из жёлудя по Аристотелю?

В жёлуде содержится форма жёлудя в модусе действительного и форма дуба в модусе возможного. Когда жёлудь прорастает, форма жёлудя переходит из модуса действительного в модус возможного, а форма дуба, наоборот, из возможного становится действительным. При этом по Аристотелю всё это происходит целесообразно: желудь имеет цель – стать дубом.

Как сгорает ёлка по Платону и по Аристотелю?

По Платону, когда ёлка сгорает, разрушается связь между куском материи и идеей ёлки. Вечная и неизменная идея ёлки остаётся в мире идей, но причастности этого конкретного куска материи, который сгорел, к идее ёлки больше нет.
По Аристотелю, когда ёлка сгорает, то форма ёлки (тоже вечная и неизменная) переходит из модуса действительного в модус возможного, а форма головёшки - из модуса возможного в модус действительного, оставаясь в том же куске материи.

Теория четырёх причин

Это вообще причины чего?

Причины движения.

Какие это причины?

Целевая, действующая, формальная и материя.

Почему материя тоже относится к причинам?

Потому что материя содержит возможность формы.

Что выступает в качестве целевой причины на примере врача и скульптора?

В качестве целевой причины выступает форма, отвечающая на вопрос "зачем?".
Когда скульптор лепит статую, целевой причиной является образ будущей статуи, который скульптор представляет.
Когда врач лечит больного, целевая причина - это форма здоровья так, как его предполагает врач.

Что выступает в качестве действующей причины на примере врача и скульптора?

В качестве действующей причины выступает форма, отвечающая на вопрос "как?".
В случае скульптора - форма статуи, направляющая движения рук скульптора. Не сами руки - руками можно и ругательство нацарапать.
В случае врача - форма здоровья, вносимая в виде лекарств или применяемая в виде процедур.

Что выступает в качестве формальной причины на примере врача и скульптора?

В качестве формальной причины выступает форма, отвечающая на вопрос "что?", т. е. форма как сущность.
На примере скульптора - форма как то, что делает статую самой собой.
На примере врача - форма здоровья как то, что делает человека здоровым.


\subsection*{\textbf{10. Христианство и его роль в европейской культуре.}}
Почему христианство - стержень европейской культуры?
Потому что оно принесло новое отношение человека к миру, сохраняющееся по сей день.
Это отношение основано, во-первых, на обожествлении свободного духа человека, во-вторых, на противопоставлении человека природе.

Что значит "человек противопоставляет себя природе"?
Ранее человек считал и себя, и природу частью мира-космоса.
Теперь же человек - венец Творения, царь природы, образ и подобие Божие (хоть и жалкое).
Природа же создана Богом для человека и потому есть объект покорения.

Что изменилось с появлением христианства?
Во-первых, природа из части священного космоса стала объектом покорения.
Во-вторых, в человеке открылась личность: были объявленными существенными не только дела и слова,
но мысли и переживания (которые по Новому Завету также подлежат Суду Божиему).
В-третьих, история выделилась из неизменной, мифологичной части космоса в направленный процесс, зависящий от Бога и человека.

Почему история изменилась именно так?
Потому что раньше был космос, который сам себя поддерживал.
Античные божества были частью гармоничного космоса (наряду с природой и человеком) и, по большому счёту,
ни на что не влияли, а подчинялись космосу.
Теперь появился всемогущий Бог. Он сотворил (направленно) природу, затем человека и влияет на историю.
Человек, как личность, обладающая свободным духом, также может, хоть и не всегда значительно, на историю влиять.

В чём значение христианства для развития науки?
Мир из статичного космоса стал динамичной историей, направляемой Богом и людьми.
Человек стал противопоставлен природе, природа стала объектом покорения.

Как изменилось отношение к природе с приходом христианства?
Раньше и человек, и природа были частями совершенного и завершённого космоса.
С приходим христианства человек стал противопоставлен природе, природа стала объектом покорения.


\subsection*{\textbf{11. Философия средневековья, ее отличие от античной философии.}}
Философия средневековья. 

Черты:

1)	Религиозная философия - высшая истина, внечеловеческая, идущая от Бога и ведущая к Нему. Причина интереса: возникают новые отношения к человеку и миру. Проблемы свободы, личности, свободного выбора. Эти проблемы обсуждались на фоне религии.

2)	Схоластика - рациональное христианство, стремление выразить истину в рациональных формах. 
Схоласты были нужны для ведения религиозной пропаганды. Церковь в Европе (католическая) не могла, в отличие от Православной, поддерживать свой авторитет силой. Поэтому Церковь готовила схоластов - людей, способных поддерживать веру логикой. Параллельно схоласты рассуждали и о вопросе свободы, который примыкал к вопросам веры и религии (мы помним, что христианство обожествило свободный дух).

3 проблемы (актуальны и сейчас):

1)	Знание и вера

2)	Новое понимание бытия

3)	Номинализм и реализм


1) Соотношение знания и веры будем разбирать на примере религии. Вообще говоря, у этой проблемы есть и внерелигиозные приложения, например, вера в свободу. Вообще вера - это отношение человека к трансцендентному, стремление души и духа к запредельному. Бывает и так, что чтобы доказать истину (например, новую теорему), в неё надо сначала поверить, а потом уже доказывать рационально.

I)	Концепция Тертуллиана проста. Вера ведёт к спасению души, знание - нет. Следовательно, знание не нужно, нужна только вера.

II)	Концепция Августина Аврелия (Блаженного) трактует соотношение между знанием и верой иначе, уделяя основное внимание вере в Бога и знанию о Боге. Стандартный ответ: вера первична, знание вторично. Знаю о Боге, потому что верю в Него. Августин засомневался: а вдруг, не зная о Боге, поверю в кого-то не того? Сначала нужно знать, потом верить. Но как знать о Боге, если Бог трансцендентен? Августин предлагает решение: знание приходит вместе с верой и есть вознаграждение веры. Августин осознал, что есть вещи, которые открываются только через веру.

III)	Концепция Фомы Аквинского значительно сложнее. Он выделяет три типа истин:

а) Истины, в которые можно только верить. Например, догмат о Святой Троице.

б) Истины, для которых достаточно знания. Например, теорема Пифагора.

в) Истины, которые можно доказать знанием, но для этого в них нужно сначала поверить. Фома Аквинский относил сюда, например, существование Бога.

Философия по Аквинскому - служанка богословия, она весьма полезна ему, без неё нельзя обойтись, хоть она и на вторых ролях.

Пример рассуждения схоластов, подчёркивающий специфичность этих рассуждений.

Однажды Фома Аквинский с Альбертом Великим гуляли по монастырскому саду и рассуждали, есть ли у крота глаза. К ним подошёл монастырский садовник и предложил принести крота. Фома обругал садовника, заявив, что садовник ничего не понимает в схоластике, а их интересует, "принципиально есть ли принципиальные глаза у принципиального крота". Они реалисты, были бы номиналистами - рассмотрели бы живого крота.

Томизм - это учение схоласта Фомы (Томаса) Аквинского.
Аквинский узаконил учение Аристотеля для логизации христианства (насколько это возможно). Формы (идеи) Аристотеля существуют сами по себе (в рамках космоса), для Аквинского же бытие - подарок Божий. Вообще религия не даёт рациональных понятий, поэтому их пришлось брать у Аристотеля. Фома Аквинский перенёс всю "лесенку" аристотелевских форм, но последней, "верхней" целевой и движущей причиной является не аристотелевская форма всех форм, а христианский Бог

2) В Античности идея - бытие. Оно следует за формой. В христианстве форма существует, но пока Бог не сотворит - вещи нет. Бытие - результат творческого акта. Божетственный акт - выделение бытия.

3) Номинализм и реализм - это философские учения, отвечающие на вопрос: что первично - единичное или общее? Ранее этот вопрос просто не возникал, так как в мире-космосе ничего единичного в принципе не могло быть. Уникальность появилась в Средневековье. Вопрос в постановке схоластов звучит так: реально ли общее? Ответ: "да, реально" - порождает реализм. Ответ: "нет, чисто номинально, т. е. есть результат человеческого обобщение, просто так названо" - порождает номинализм.
Реализм: для познания вещей не надо пачкать руки, достаточно умственно созерцать идеи. Схоласты ещё добавляли: "...вложенные Господом в наш разум, ибо Он творил мир по идеям".
Номинализм: общего нет как самостоятельно существующего, предшествующего отдельным вещам. Общее есть результат человеческого обобщения. Бог - художник, создаёт единичные вещи. Понятие – общий термин. Номинализм - это учение, утверждающее, что единичное первично, т. е. предшествует общему.

И номинализм, и реализм подготавливали науку Нового времени: реализм - требованием строить красивые теории, номинализм - требование опираться на опыт.
Более широко: заслуга номинализма в том, что он указал, что общее надо уметь выделять (и раздогматизировал его), реализма - в том, что общее надо выделять не как попало, а в соответствии с природой этого общего.
\subsection*{\textbf{12. Знание и вера.}}
Соотношение знания и веры будем разбирать на примере религии.
Вообще говоря, у этой проблемы есть и внерелигиозные приложения, например, вера в свободу.
Вообще вера - это отношение человека к трансцендентному.
Бывает и так, что чтобы доказать истину (например, новую теорему), в неё надо сначала поверить, а потом уже доказывать рационально.

1. Концепция Тертуллиана проста.
Вера ведёт к спасению души, знание - нет.
Следовательно, знание не нужно, нужна только вера.

2. Концепция Фомы Аквинского значительно сложнее.
Он выделяет три типа истин:

а) Истины, в которые можно только верить.
Например, догмат о Святой Троице.

б) Истины, для которых достаточно знания.
Например, теорема Пифагора.

в) Истины, которые можно доказать знанием, но для этого в них нужно сначала поверить.
Фома Аквинский относил сюда, например, существование Бога.

Философия по Аквинскому - служанка богословия, она весьма полезна ему.

3. Концепция Августина Аврелия (Блаженного) трактует соотношение между знанием и верой иначе, уделяя основное внимание вере в Бога и знанию о Боге.
Стандартный ответ: вера первична, знание вторично.
Знаю о Боге, потому что верю в Него.
Августин засомневался: а вдруг, не зная о Боге, поверю в кого-то не того?
Сначала нужно знать, потом верить.
Но как знать о Боге, если Бог трансцендентен?
Августин предлагает решение: знание приходит вместе с верой и есть вознаграждение веры.


\subsection*{\textbf{13. Номинализм и реализм.}}

Номинализм и реализм - это философские учения, отвечающие на вопрос: что первично - единичное или общее?
Ранее этот вопрос просто не возникал, так как в мире-космосе ничего единичного попросту не могло быть.
Уникальность появилась в Средневековье.
Вопрос в постановке схоластов звучит так: реально ли общее?
Ответ: "да, реально" - порождает реализм.
Ответ: "нет, чисто номинально, т. е. есть результат человеческого обобщение, просто так названо" - порождает номинализм.

Реализм: для познания вещей не надо пачкать руки, достаточно умственно созерцать идеи.
Схоласты ещё добавляли: "...вложенные Господом в наш разум, ибо Он творил мир по идеям".

Номинализм: общего нет как самостоятельно существующего, предшествующего отдельным вещам.
Общее есть результат человеческого обобщения.

И номинализм, и реализм подготавливали науку Нового времени:
реализм - требованием строить красивые теории, номинализм - требование опираться на опыт.

Более широко: заслуга номинализма в том, что он указал, что общее надо уметь выделять (и раздогматизировал его), реализма - в том, что общее надо выделять не как попало, а в соответствии с природой этого общего.


\subsection*{\textbf{14. Философия эпохи Возрождения. Возрожденческий гуманизм.}}

В Эпоху Возрождения произошёл переход от теоцентризма к антропоцентризму.
Вектор свободы, возникший благодаря христианству, развернулся изнутри вовне - и появились методы нерелигиозного воплощения свободы и самореализации - в искусстве, в политике и т.д.
Но неверно было бы считать, что произошло возвращение к античности: в античности гражданин - это член полиса, часть части космоса, а в Эпоху Возрождения - свободный индивид.

Гуманизм - это учение, провозглашающее человека высшей ценностью, а его призванием - не служение религии, а творческое самовыражение.

Пантеизм - это учением, обосновывающее гуманизм.
Суть пантеизма - в объявлении Бога творческой силой природы.
Тогда человек - не жалкое и грязненькое подобие Бога, а вершина божественного творчества.


\subsection*{\textbf{15. Возникновение науки Нового времени. Бэкон о методе новой науки.}}

В античности занятие наукой было приобщением к законам космической гармонии.
Так как разум - структура космическая, то достаточно было созерцания, а всякое искусственное изменение порядка вещей лишь мешало постигать их сущность.
А в Новое Время появился принципиально новый метод науки - эксперимент.
Это стало возможным благодаря возросшему авторитету рационального мышления.
Знание перестало быть бескорыстным созерцанием, человек Нового Времени вовсю покорял природу.
Бэкон: "Знание есть могущество" ("Knowledge is power itself").

Сам способ объяснения мира претерпел существенные изменения.
Аристотель выводил законы из представлений о мировом порядке.
Ньютон выводил порядок из законов природы.

Примечательно, что идея покорения природы была во многом взята у алхимиков, но мышление алхимиков было не рациональным, а мистическим.
Также заметим, что популяризации рационального мышления во многом вызвана деятельностью Лютера по Реформации Церкви.


\subsection*{\textbf{16. Рационализм и эмпиризм (общая характеристика).}}

Рационализм и эмпиризм отвечают на вопрос: каков метод получения истинного знания?

Эмпиризм: источник знания - опыт.
От опыта учёные с помощью таинственного умения - индукции переходят к теории.
Проблема в том, что индукция ненадёжна.
Природа-то, конечно, говорит на языке опыта, и теорию высасывать из пальца не стоит, но вот как её получить?
Например, если по индукции рассуждать о том, что вода при нагревании только немного расширяется, то кипение будет полной неожиданностью.

В рационализме ситуация обратна.
Пусть от разума к теории надёжен.
Но где гарантия, что построенная теория будет согласована с природой?


\subsection*{\textbf{17. Английские эмпирики XVIII в. Учения Локка, Юма, Беркли.}}

Локк, Юм и Беркли - эмпирики-сенсуалисты, они считали, что за основу познания надо брать опт, и не различали его научную и донаучную формы.

Локк упростил и прояснил понятие опыта:
опыт - это совокупность ощущений.

Ощущения делятся на "первичные качества" - инвариантные относительно наблюдателя (например, масса) и "вторичные" - зависящие от наблюдателя (например, вкус).
Строить науку нужно, основываясь на первичных качествах.
В разуме по Локку не ничего, чего бы не было в опыте.

Контрпример к учению Локка: вчера солнышко встало на востоке; сегодня солнышко встало на востоке.
Это мы знаем из вчерашних и сегодняшних ощущений соответственно.
А откуда мы знаем, что завтра солнышко встанет на востоке, если у нас нет завтрашних ощущений?

Юм предлагает устранить этот недостаток, добавив привычку:
знание есть сумма ощущений плюс привычка.
Это, однако, тоже не панацея, потому что привычка не даёт обязательности и всеобщности.
Заслуга Юма в том, что он догадался, что к ощущениям нужно добавить нечто.
Позже Кант в качестве этого нечто использует априорные формы.

Беркли развил теорию Локка в другом направлении.
Он поставил вопрос: почему за ощущениями должно что-то стоять?
Вещь по Беркли - это просто комплекс ощущений.
Возникает вопрос: а что их соединяет?
"Бог" - отвечает Беркли.

\subsection*{\textbf{18. Рационализм Декарта.}}

Декарт был рационалистом, следовательно, считал, что теорию нужно выводить из идей.
Более того, Декарт дал метод выведения теории.

Первый акт: подвергнуть всё сомнениям, авторитетов нет.
Какие идеи выстоят в огне сомнений, те и истинные.

Второй акт: применить выстоявшие - истинные - идеи для объяснения мира, т. е. дедукция, переход от абстрактного к конкретному.

Что же остаётся, если бросить всё в огонь сомнений?
Активное сомневающееся "Я" Декарта.
Это результат формулируется фразой "Cogito ergo sum" - "мыслю, следовательно, существую".

Декарт выделяет две первичные субстанции: материю-протяжённость и мир идей-мышление.
Они невыводимы друг из друга и согласованы Богом.

\subsection*{\textbf{19. Учение Гоббса и Локка о государстве. Понятия правового государства и гражданского общества.}}

Гоббс (и Локк) были эмпириками, следовательно, номиналистами.
Номинализм способен дать глубокие результаты, будучи приложенным к обществу, так как единица общества - индивид - обладает собственной глубиной.

Гоббс впервые построил теория государства, исходя из отдельны индивидов.
До этого государство считалось божественным установлением, вспомним коронации и "царь - помазанник Божий", "император - Сын Солнца" и т.д.

Государство - это власть над обществом, обладающая монополией на насилие.
Гоббс заявил, что государство создаётся свободными индивидами в результате общественного договора  с целью поддержания порядка и защиты прав и свобод индивидов.

До создания государства индивиды находятся в естественном состоянии.
По Гоббсу естественное состояние предполагаем у всех право на всё, при попытке осуществить которое начинается война всех против всех.
Таким образом, граждане лишаются возможности контролировать государство, потому что плохое государство - лучше, чем никакого.
Имеем парадокс: государство, созданное для того, чтобы защищать права граждан, может безнаказанно их ущемлять.

Этот парадокс был разрешён Локком.
Локк постулировал, что естественное состояние общества есть гражданское общество, т. е. общество, в котором наблюдается первичный порядок.
Первичный порядок - это порядок, основанный на моральном законе и трудовой (чтобы уважали!) собственности.
Государство создаётся лишь для защиты этого порядка, следовательно, граждане могут его контролировать.

Локк выдвинул два механизма контроля власти: выборы и разделение властей.
Выборы позволяют решать, кто именно будет олицетворять волю общества.
Разделение властей (как правило, на три ветви - законодательную, исполнительную и судебную) создаёт некую конкуренцию внутри власти, противодействующую её монополизации.

Идеи Гоббса и Локка легли в основу многих современных конституций, в частности, Конституции РФ.


\subsection*{\textbf{20. Кант и проблемы теории познания.}}

Кант фактически даёт ответ на вопрос, неявно поставленный Юмом: какой Х нужно добавить к сумме ощущения, чтобы получить знание?
Покажем сначала, что Х не равен нулю, т. е. что что-то добавить надо.
*Картинка с профилями и вазой.*
В наших ощущениях есть только изображение линий, но нет ни профиля, ни вазы, тем не менее, мы их видим.
Причём одно и то же изображение линий может быть как профилями, так и вазой.
Что мы добавляем к линиям, чтобы получить профили или вазу?
Принципы профиля и вазы, взятые из предыдущего опыта.
Если некто ни разу не видел вазы, то и на картинке он её не увидит.
Итак, некий Х к ощущениям добавлять всё-таки нужно, но пока из наших рассуждений неясно, почему Х не может быть взят из опыта.

Рассмотрим аудиторию.
В ней стоят столы, висят лампы, сидят студенты и преподаватель.
Уберём теперь всё, что взято из опыта.
Стены аудитории убрали. Людей убрали. Лампы, столы, доску, мел - всё убрали!
Что осталось? Пространство, в котором аудитория была, и время, в которое "подопытная" лекция проходила.
Ни пространство, ни время из опыта не выводятся.

Определение. Априорными формами называются принципы познания, невыводимые из опыта.

Роль априорных форм в познании заключается в том, что они встраиваются в опыт и "форматируют" его, делая пригодным для обработки разумом.
Априорные формы содержатся в разуме и не имеют ничего общего с миром в себе.

Пространство и время - это априорные формы созерцания.
Кроме них, существуют и априорные формы мышления - например, причинность.
Как и всякая априорная форма, причинность не выводится из опыта.
В опыте нет причин и следствий, есть только последовательность событий.

Именно наличие априорных форм позволяет разуму извлекать из опыта всеобщие и обязательные закономерности и таким образом познавать мир.

По Канту разум воспринимает ощущения, встраивая в них априорные принципы.
Результат форматирования ощущений принципами есть опыт.
Опыт согласован с разумом, следовательно, построенная разумом теория без проблем согласуется с опытом.
Кантов субъект - разум - формирует воспринимаемый в опыте мир, форматируя ощущения с помощью принципов.
Этот мир называется миром для нас.
Но есть и другой мир - мир, из которого ощущения приходят.
Он называется миром в себе.
Мир в себе непознаваем, так как разум не может с ним взаимодействовать.
И обратно, всё, с чем может встретиться разум, принадлежит миру для нас и, следовательно, познаваемо.

Агностицизм Канта в том и заключается, что по Канту существует непознаваемый мир в себе.


\subsection*{\textbf{21. Соотношение нравственности и свободы. Категорический императив нравственного сознания (по Канту).}}

Кант впервые дал нерелигиозное обоснование свободы и нравственности, а затем показал, что основа нравственности - свобода.
Человек вменяем, следовательно, подчиняется нравственному закону, следовательно, свободен.

По Канту человека делает свободным разум.
Разум имеет два лица: теоретический и практический.
Теоретический разум даёт законы внешнему миру (строит теорию, объясняющую внешний мир) и, следовательно, несвободен.
Практический разум даёт принципы поступков и не зависит от устройства мира.
Свобода есть автономия воли, а волю Кант отождествляет с практическим разумом.
Свобода вовсе не означает произвол - нет, произвол скорее присущ человеку, ведомому за ниточки страстей.
Свободный разум не может быть произволен и беззаконен, но чтобы оставаться свободным, должен давать закон самому себе, подобно тому, как свободное государство само определяет свои законы (а не отказывается от них).

Закон, который свободный разум даёт себе, и называется нравственным законом.

Категорический императив нравственного сознания имеет две формы.

Первая форма: поступай так, чтобы принцип твоего поступка мог быть взят в качестве всеобщего закона.

Вторая форма: поступай так, чтобы человек был целью твоего поступка, а не средством.

Покажем, как они выводятся.
Разум свободен тогда и только тогда, когда сам даёт себе закон.
Этот закон и удовлетворяет требованию, называемому категорическим императивом.

Во-первых, разум универсален (так как является субъектом).
Следовательно, зов разума не зависит ни от человека, ни от внешнего мира, то есть является всеобщим.
Отсюда первая форма.

Во-вторых, разум автономен, следовательно, уважает только самого себя.
Так как разум универсален, то в каждом человеке нужно видеть носителя разума, отсюда вторая форма.


\subsection*{\textbf{22. Философия Фихте и Гегеля (общая характеристика).}}

Фихте считает, что весь мир формируется субъектом - обобщённым человеческим "Я".
"Я" формирует весь мир, следовательно, оно изначально.
"Я" изначально, следовательно, бессознательно.
"Я" приходит к самосознанию и развивает себя с помощью опредмечивания.
Так как "Я" - обобщение человеческого "я", то его опредмечивание выводится из человеческого опредмечивания.
Чтобы человеку познать себя, надо создать предмет в широком смысле - например, доказать теорему, написать стих, высечь статую.
Это первый акт самопознания, он называется "положить иное".
Результат в человеке как бы содержался, но пока человек его не достиг, он не знал, что в нём это есть.
Это второй акт самопознания, он называется "узнать себя в ином".

Общая эту цепочку рассуждений, Фихте приходит к выводу, что "Я" полагает "не-Я" - природу.
"Я" руками людей творит природу.
Человек становится умнее и могущественнее только через преобразование природы, а природа - это просто тренажёр для человека.

Так как субъект Фихте - идея, выводимая обобщением из человеческого мышления, то Фихте - субъективный идеалист.

В отличие от Фихте, Гегель является объективным идеалистом и провозглашает субъектом мировой разум, невыводимый из человеческого.
Мировой разум стремится себя осознать, воплощая себя в истории через человечество.
Люди сделали - посмотрели - осознали, а вместе с ними осознал и мировой разум.
Несмотря на то, что история состоит из действий отдельных индивидов с раздельными целями, она подчиняется некоей особой, надчеловеческой логике (и выделение этой логики - заслуга Гегеля).
Чтобы направить историю в нужное русло, мировой разум расставляет между людьми приманки в виде страстей.


\subsection*{\textbf{23. Диалектика как учение о всеобщей связи и развитии (общая характеристика).}}

Диалектика - это учение о всеобщей связи и развитии.
Также говорят, что диалектика - это учение о единстве и борьбе противоположностей.
(Закон единства и борьбы противоположностей как раз даёт эквивалентность этих определений.)

Развитие - это движение системы, в ходе которого изменяется уровень организации системы.

В диалектике важную роль играет саморазвитие - развитие, происходящее в силу внутренних причин.

Развитие, в ходе которого нарастает уровень организации системы, называется прогрессивным (прогрессом системы).
Развитие, в ходе которого уровень организации системы падает, называют регрессивным (регрессом системы).

Диалектическое противоречие - это отношение между двумя диалектическими противоположностями.

Диалектические противоположности - это две противоборствующие стороны системы, которые взаимно предполагают друг друга.

Диалектическое противоречие является источником развития.


\subsection*{\textbf{24. Закон единства и борьбы противоположностей.}}

Для саморазвития системы необходимо внутреннее противоречие этой системы, порождённое ей самой.
Саморазвитие происходит следующим образом:
система порождает противоречие внутри себя - появляются противоположности;
противоречие нарастает - разгорается борьба противоположностей;
противоречие разрешается (тем или иным способом) - качество присваивает свою противоположность, возникает новое качество.

Например, покажем диалектическое саморазвитие совести человека.
Перед человеком стоит нравственный выбор: совершить выгодный, но бесчестный поступок или честный, но невыгодный.
Шкурные интересы противоречат интересам совести - появилось противоречие.
Внутри человека совесть борется с жадностью - это борьба противоположностей.
Противоречие может разрешиться двумя способами:

а) Совесть побеждает жадность и получает новое качество, поднимается на новый уровень, т. е. человек становится более совестливым.

б) Жадность побеждает совесть. Но и тут совесть получает новое качество - привычку "прогибаться" под жадность, человек становится менее совестливым.

В обоих случаях происходит развитие совести: в первом случае - прогрессивное, во втором - регрессивное.


\subsection*{\textbf{25. Закон перехода количественных изменений в качественные.}}

Закон перехода количественных изменений в качественные:
когда количественные изменения превосходят свою меру, появляется новое качество.

Дадим теперь определения всем понятиям, используемых в законе.

Качество - это то, что делает систему самой собой.

Количество - это степень интенсивности качества.

Мера - это интервал изменения количества, изменения внутри которого не приводят к изменению качества.

Заметим, что меру иногда можно определить теоретически, но очень сложно практически избежать выхода за её пределы.
Например, американские экономисты построили теорию, согласно которой экономика для развития нуждается в кредитах.
Экономисты знали, что чрезмерное кредитование ведёт к кризису.
Тем не менее, в 2008 году соблюсти меру им не удалось, и кризис разразился.

Другим классическим примером меры является пример с агрегатными состояниями воды.
Мера количественных изменений температуры, при которых вода остаётся жидкой - от 0 до 100 градусов по Цельсию.
Стоит выйти за границы этой меры - и вода начинает либо замерзать, либо испаряться.

А ещё можно заметить, что попытка вернуться назад после качественного скачка не всегда заканчивается успехом.
Например, если превысить меру охлаждения сосуда с водой, то лёд разорвёт сосуд, а попытка вернуть систему в изначальное состояние путём повышения температуры даст не целый сосуд с жидкой водой, а пустой лопнувший сосуд и лужу.


\subsection*{\textbf{26. Закон отрицания отрицания.}}

Закон отрицания отрицания, в отличие от двух других законов диалектики, относится только к прогрессивному развитию системы и формулируется следующим образом: чтобы развитие было прогрессивным, необходимо диалектическое отрицание.

Отрицание - это смена одной ступени развития на другую.

Диалектическое отрицание - это "неполное" отрицание, сохраняющее нечто от предыдущей ступени.
Так как диалектическое отрицание отрицает понятие отрицания, то его каламбура ради называют "отрицанием отрицания".

Без отрицаний никакое развитие невозможно по определению.
Но для прогрессивного развития необходимо накопление сложности организации, следовательно, отрицание должно быть не полным, т. е. диалектическим.
Отсюда имеем формулировку закона.


\subsection*{\textbf{27. Понятие материи. Виды материализма. Специфика материализма К. Маркса.}}

Ранний материализм не мог взять за основу Платоновскую материю (а именно идеалист Платон впервые ввёл это понятие), так как платоновская материя бескачественна и полностью зависит от идей.
По сходным причинам не подошла и аристотелева материя.
В результате за основу были взяты демокритовские атомы, а материя отождествлена с веществом.
Демокритовские атомы были лишены эвристичности, и потому и движение, и развитие приходилось постулировать.
Это делало материализм крайне малопригодным для объяснения мира и уж совсем непригодным для объяснения общества.

До Маркса материализм не был диалектичен, т. к. из атомов не выводятся ни связь, ни развитие, а вот идеализм мог быть диалектичен.

Маркс вышел к новому пониманию материи, сделав её диалектичной - так появился диалектический материализм.
Более того, Маркс применил диалектический материализм к объяснению истории - так появился исторический материализм.

Определение материи Маркс не дал, но указал на то, что не нужно отождествлять материю с неделимыми частицами вещества.

По Марксу, в основе общества - общественные отношения, которые делятся на идеологические, зависящие от разума, и материальные, от разума и сознания не зависящие.
Основным видом материальных отношений являются производственные.
Производственные отношения - это форма существования производительных сил.
Производственные отношения Маркс называет социальной материей.
Производственные отношения определяют базис формации - типа общества, а идеологические являются надстройкой.
Производительные силы постоянно развиваются, стремясь удовлетворить увеличивающиеся потребности человека.
Развитие производительных сил, опережающее развитие производственных отношений (которые более консервативны), приводит к революции и смене формации.

Маркс, совершив революцию в материализме, не дал определения материи.
Определение материи дал Ленин.

Первое определение материи, данное Лениным:
материя - это объективная реальность, отражаемая в наших ощущениях, но существующая вне и независимо от наших ощущений.

Проблема в том, что это определение не охватывает социальную материю - социальная материя не отражается в наших ощущениях.
Тогда Ленин даёт следующее определение:
материя - это объективная реальность, отражаемая в нашем сознании и существующая вне и независимо от нашего сознания.

Теперь проблема заключается в том, что под это определение подходит, например, и гегелевская мировая идея.
Но можно избавиться от этой проблемы, потребовав от материи неисчерпаемости.
Тогда получим:
материя - это неисчерпаемая объективная реальность, отражаемая в нашем сознании и существующая вне и независимо от него.


\subsection*{\textbf{28. Проблема пространства и времени в философии и современной науке.}}

Существует две концепции понимания пространства и времени: субстанционалистическая и релятивистская.

Субстанционалистическая концепция утверждает, что пространство и время - это две независимые сущности (субстанции).
Пространство - это "пустая сцена" для процессов.
Процессы идут, потому что время движется.
Сторонниками этой концепции были Демокрит (неявно), Декарт и Ньютон.

Релятивистская концепция утверждает, что пространство есть порядок сосуществования объектов, а время есть порядок следования процессов.
Время движется, потому что процессы идут.
Такое определение допускает некоторую хрупкость пространства и времени: пространство или время есть риск "сломать", просто переставив местами объекты или провоцируя необычные процессы.
Эту хрупкость устранил Лейбниц, поняв, что нужно говорить о всевозможных объектах и процессах.
Итак, по Лейбницу пространство - это порядок сосуществования всевозможных объектов, время - порядок следования всевозможных процессов.

Однако возможность изменения пространства долгое время не использовалась.
Формально её заблокировал Кант, постулировав, что пространство и время приходят "сверху" как априорные принципы.

Эйнштейн очертил границы "хрупкости" пространства-времени теорией относительности.
Изменение пространства-времени - неразделимого континуума - в одной системе отсчёта относительно другой системы отсчёта зависит от скорости их движения друг относительно друга. Скорость - характеристика "хорошая", "спокойная", кроме того, чтобы влияние изменения скорости на пространство-время было заметно, скорость должна быть очень значительной.

\subsection*{\textbf{29. Проблема истины.}}

В нашем знании есть объективное содержание, которое не зависит ни от человека, ни от человечества.
Это знание отражает объективную реальность.
Оно и называется объективной истиной.
Критерием объективной истины является практика.
Любая объективная истина является относительной, то есть может быть уточнена и дополнена.
Например, законы Ньютона являются объективной истиной, так как соответствуют объективной реальности.
Тем не менее они могут быть дополнены теорией относительности Эйнштейна.

Любая объективная истина относительна, то есть может быть уточнена и дополнена.
Относительная истина не даёт границ собственной применимости.
Например, законы Ньютона являются объективной истиной, так как соответствуют объективной реальности.
Но в законах Ньютона не содержится указание не применять их на скоростях, близких к скорости света.
Это ограничение на законы Ньютона даёт уже другая относительная истина - теория относительности Эйнштейна, одновременно уточняя и дополняя аконы Ньютона.
Тем не менее, в каждой относительной истине содержится некое непроверяемое, абсолютное зерно.
Значит, существует абсолютная истина.
Каждая новая относительная истина, уточняя и дополняя предыдущую, расширяет это зерно и тем самым приближается к абсолютной истине, но никогда не достигнет её, потому что мир неисчерпаем.
Но мы не можем даже заранее сказать, где именно содержится это абсолютное зерно, потому что не знаем, в какой именно части новая относительная истина дополнит и уточнит предыдущую.


Практика как критерий истины предложена Марксом.
Выведем этот критерий.
Практика - это деятельность общественного человека по преобразованию мира.
В ходе практики человек создаёт мир техники ("вторую природу").
С одной стороны, при создании мира техники человек руководствуется научными теориями.
С другой стороны, мир техники работает по законам природы.
Следовательно, если мир техники работает так, как ожидалось, значит, научные теории соответствуют законам природы, причём ровно в той мере, в какой эти теории воплощены в технике.

Сила критерия практики в том, что он позволяет заявить: учёные не зря едят свой хлеб, наука полезна.
Слабость критерия практики в том, что он не позволяет проверить на истинность отдельно взятую гипотезу, так как любой объект мира техники воплощает одновременно несколько научных теорий.

Другой подход предлагают позитивизм и его наследник - неопозитивизм.
Оба - ответвления эмпиризма.
Три основных постулата неопозитивизма:
1. Принцип демаркации: наука отделена от мировоззрения, философия науки выводится из самой науки и отделена от "старой" философии.
2. Принцип верификации: научная теория должна подтверждаться фактами.
3. Принцип кумулятивности: развитие науки происходит путём накопления фактов.

Неопозитивисты, будучи эмпириками, считали, что наука выводится из опыта.
Опыт неопозитивизм рассматривает как совокупность фактов, зафиксированным простейшими высказываниями - протокольными предложениями.
Факт - главный судья теории и одновременно её источник.
Теория, прошедшая верификацию, считается истинной.

Проблема заключается в том, что факт оказывается теоретически нагружен, и теория из фактов не выводится.

Поппер предложил заменить критерий верификации на критерий фальсификации: теория истинна, пока не опровергнута фактами.
Недостаток такого подхода в том, что учёный будет рьяно защищать свою теорию.

Постпозитивизм серьёзно пересмотрел позитивистские идеи:
1. Отказ от демаркации: допускается влияние мировоззрения на науку.
2. Отказ от эмпиризма: признаётся наличие в науке невыводимых из опыта утверждений.
3. Отказ от кумулятивизма: наука развивается не накопительно, а скачками, научными революциями.

Кун выдвинул идею, что теория определяется не только разумом, но и научным сообществом.
Именно научное сообщество выбирает парадигму - фундаментальную теорию, которая даёт общее видение мира, образцы постановки и решения исследовательских задач.
Смена парадигмы не обосновывается внутринаучными причинами и влечёт научную революцию.

Локатес попытался объяснить скачкообразное развитие науки, исходя из внутринаучных причин.
С точки зрения Локатеса, исследовательская программа есть фундаментальная теория, порождающая серию других теорий, которые её развивают.
В исследовательскую программу входит ядро - сама теория, эвристика - принцип выдвижения развивающих теорию гипотез, и защитный пояс - серия положений, соотносящих ядро с экспериментом.
Например, законы Ньютона - это ядро, эвристика - это приёмы трехмерного интегрирования/дифференцирования, развивающие теории - это гидродинамика, небесная механика и т. д.
Исследовательские программы конкурируют между собой, в основном - по предсказательной силе.
Та исследовательская программа, которая лучше предсказывает вновь открываемые явления, побеждает.

\subsection*{\textbf{30. Сознание, его происхождение и сущность.}}

Сознание отличается от мира вещей тем, что ему присуща идеальность.
Идеальность - особое свойство сознания, заключающееся в том, что объекты мира сознания бестелесны, не действуют на органы чувств и не располагаются в пространстве.
Существуют две концепции сознания: идеалистическая и материалистическая.
Идеализму объяснить сознание относительно просто: идеальный дух был всегда, вопрос лишь в том, как он приходит к самосознанию.
Но сознание оказывается запрограммированным изнутри.
Материализму сложнее: надо объяснить, во-первых, происхождение идеального, во-вторых, связь идеального сознания с материальным мозгом, в-третьих - с внешним миром.
Домарксовский материализм либо не касался этого вопроса, либо отрицал идеальность сознания: "мозг выделяет сознание, как печень выделяет желчь".
Марксовский (диалектический) материализм объяснил сознание как отражение объективной реальности.
Сознание оказывается неисчерпаемым, но запрограммированным извне.
Отражение - это неотъемлемый спутник взаимодействия двух материальных систем, оно заключается в том, что каждая из провзаимодействовавших систем в своей структуре сохраняет некоторые характеристики системы-партнёра.
Содержание отражения - это те характеристики, которые оно сохраняет, а структура отражения - это способ, которым оно это делает.
Например, след ноги на песке - это отражение ноги песком.
Его форма - возмущение ровной поверхности песка, его содержание - размер и рельеф низа стопы.
Содержание отражения, взятое как предмет, имеет все признаки идеального образа (но только тогда, когда его опредметили!).

Эволюция отражения прошла три ступени: неживую природу, живую природу, человека.
В неживой природе основной метод бытия - пассивное взаимодействие, отражение не играет самостоятельной роли по сравнению с взаимодействием.
Основной метод бытия в живой природе - активное приспособление, объект живой природы пользуется отражением, но не опредмечивает его содержание.
Человек е опредмечивает содержание отражения - и получает идеальный образ.
Так возникает сознание.

Идеальный образ возникает у человека благодаря присвоению схемы социальных трудовых действий, направленных на изготовление орудий труда.
Такие действия лишены прямого биологического смысла, а значит, не вписываются в рамки биологических программ.

Язык - это система знаков, т. е. предметов произвольной природы, которые в ходе исполнения социальной функции начинает нести новое, неприродное содержание.
Основные функции языка:
1. Коммуникативная - передача информации между людьми.
2. Экспрессивная - выражение чувств.
3. Моделирующая - замена операций над реальными объектами операциями над языком.

Мышление есть присутствие идеальных образов.
Мышление невозможно без языка.
Действительно, для того, чтобы с идеальными образами что-то делать, нужно приписать им значение, т. е. сделать знаками языка.


\subsection*{\textbf{31. Сознание и бессознательное. Концепции З. Фрейда и  К.Г. Юнга.}}

До Фрейда единственным известным видом бессознательного был автоматизм - например, слова мы пишем сознательно, а буквы - благодаря автоматизму.
Автоматизм - это бессознательное совершение рутинных (нетворческих) действий.
Фрейд (как и его ученик Юнг) был врачом.
Фрейд открыл бессознательное как самостоятельную силу, т. е. силу, действующую в обход сознания и часто вопреки ему.
Фрейдовское бессознательное делится на вытесненное и исходное.
Вытесненное бессознательное формируется комплексами, неизжитыми переживаниями и т.д.
Вытесненное бессознательное можно изжить, особенно под контролем специалиста.
Исходное бессознательное неизживаемо, носит природное происхождение и состоит из двух сил: Эрос и Танатоса.
Эрос - это сила, стремящаяся к удовольствию любой ценой (в том числе к сексуальному).
Танатос - это сила, стремящаяся к разрушению: агрессия, если направлена вовне, и суицидальные стремления, если направлена вовнутрь.
Вся жизненная энергия сосредоточена в Эросе, и если сознанию нужно её взять, Эрос приходится обманывать.
Процесс получения энергии Эроса на цели творчества и последующего получения удовольствия от достижения упомянутых целей называется сублимацией.

Так как при попытке запретить что-либо бессознательному от своего имени сознание может быть проигнорировано, культура внедряет в бессознательное своего резидента - Super Ego.
Исходное бессознательное слишком примитивно, чтобы Super Ego могло содержать что-то, кроме запретов, сформированных культурой.

Таким образом, из идей Фрейда следует, что, во-первых, бессознательное есть самостоятельная сила, действующая независимо от сознания, во-вторых, сущность человека не такая уж белая и пушистая (там есть Эрос и Танатос), в-третьих, обрести в жизни гармонию нельзя, ибо придётся вечно бороться с самим собой - сознание с бессознательным.

Юнг был учеником Фрейда и дал принципиально другую структуру исходного бессознательного.
Исходное бессознательное по Юнгу называется коллективным, так как оно сформировано коллективными мифологическими переживаниями первобытного человека, склонность к которым наследуется.
Структурной единицей коллективного бессознательного является архетип.
Например, архетип "Тень" отвечает за первобытную ответную агрессию, архетип "нуминозное" - за тягу с пугающей, но в то же время манящей тайне, архетипы "анима" и "анимус" - за тягу к женственному и мужественному соответственно.
Так как коллективное бессознательное имеет культурное происхождение, то его можно изжить (удовлетворить) символически, безопасно для культуры.
Например, архетип "нуминозное" удовлетворяется просмотром ужастиков.

\subsection*{\textbf{32. Философия экзистенциализма.}}

Предпосылкой для возникновения экзистенциализма стала Первая Мировая война.
Бойцы следовали своей сущности, сущности вполне одобряемой классикой и воплощали такие качества, как смелость, верность присяге, изобретательность (в том числе при изобретении химического оружия) и т. д., но это привело к кровавой бойне.
И люди задумались: а если следование сущности ведёт к такому ужасу, то может, и не стоит ей следовать?
Кроме того, встаёт вопрос: а если свобода заключается в том, чтобы следовать своей сущности, то получается, что человек запрограммирован?

Ответы на эти вопросы пытается дать экзистенциализм, название которого происходит от лат. existance - существование.
Классическая философия утверждает, что сущность первична, существование есть лишь следование сущности, а свобода - это такое бесплатное приложение к существованию, которое можно получить, если хорошо его исполнять.
Экзистенциализм утверждает: в случае человека первично существование, а сущность человек может выбирать сам.
Но подлинное существование - экзистенция, она же свобода - присутствует не в течение всей жизни, а только в те моменты, когда человек находится вне заданности, например, в творчестве.
Недостаток экзистенциализма в том, что с его позиций нельзя выработать всеобщий моральный закон (хотя можно осуждать конформизм), например, нельзя осудить нациста, который утверждает, что он нацист по убеждениям.




\end{document}