\documentclass[aspectratio=169,12pt]{beamer}
\usepackage{fontspec}
\setmainfont{CMU Serif}
\newfontfamily{\cyrillicfont}{CMU Serif}
\setsansfont{CMU Sans Serif}
\newfontfamily{\cyrillicfontsf}{CMU Sans Serif}
\setmonofont{CMU Typewriter Text}
\newfontfamily{\cyrillicfonttt}{CMU Typewriter Text}

%%% Fonts and language setup.
\usepackage{polyglossia}
%% Math
\usepackage{amsmath, amsfonts, amssymb, amsthm, mathtools} % Advanced math tools.
\usepackage{icomma} % "Умная" запятая: $0,2$ --- число, $0, 2$ --- перечисление

\usepackage{tikz,lipsum,lmodern}
\usepackage[most]{tcolorbox}

\usetheme{boxes}
\usecolortheme[style=light]{Nord}
\usefonttheme{Nord}
\setbeamertemplate{navigation symbols}{}%remove navigation symbols
\usepackage{minted}
\usemintedstyle{nord}
%\setbeamerfont{footline}{size=\large}

\usepackage[dvipsnames]{xcolor}
\usepackage{verbatim}
\usepackage{hyperref}
\usepackage{listings}
\usepackage{tikz,tikz-3dplot}   
\usetikzlibrary{arrows, positioning, shapes} 

\usepackage{bbding}
\newcommand{\greencheck}{{\color{ForestGreen}\Checkmark}}
\newcommand{\redsolid}{{\color{BrickRed}\XSolid}}

\lstdefinelanguage{JavaScript}{
  keywords={let, typeof, new, true, false, catch, function, return, null, catch, switch, var, if, in, while, do, else, case, break},
  ndkeywords={class, export, boolean, throw, implements, import, this},
  identifierstyle=\color{black},
  sensitive=false,
  comment=[l]{//},
  morecomment=[s]{/*}{*/},
  commentstyle=\color{purple}\ttfamily,
  stringstyle=\color{red}\ttfamily,
  morestring=[b]',
  morestring=[b]",
}

\lstset{
  language=JavaScript,
  extendedchars=true,
  basicstyle=\footnotesize\ttfamily,
  showstringspaces=false,
  breakatwhitespace=true,
  showspaces=false,
  numbers=none,
  keepspaces=false,
  breaklines=false,
  showtabs=false,
  captionpos=b
  escapechar =|,
  frame=none,
  commentstyle=\itshape ,
  stringstyle =\bfseries,
  linewidth = \linewidth,
  xleftmargin = -30pt,
  xrightmargin = -15pt,
}

\newtcolorbox{leftBox}{
  colback=NordWhite,colframe=NordBlack, 
  width = 0.97\linewidth,
  sharp corners = southwest,
}

\newtcblisting{oldrightBox}{
  before=\begin{flushright},
  after=\end{flushright},
  colback=NordWhite,colframe=NordBlack, 
  width = 0.97\linewidth,
  sharp corners = southeast,
  listing only,
}

\newtcolorbox{rightBox}{
  before=\begin{flushright},
  after=\end{flushright},
  colback=NordWhite,colframe=NordBlack, 
  width = 0.97\linewidth,
  sharp corners = southeast,
  listing only,
}
\setbeamerfont{footline}{size=\footnotesize}
\setbeamertemplate{footline}[frame number]

%%% Polyglossia setup after  (nearly) everything as described in documentation.
\setdefaultlanguage{russian}
\setotherlanguage{english}

\usepackage{multicol}
\usepackage{xcolor}
\setlength{\columnsep}{0.5cm}

\begin{document}
\begin{frame}
	\begin{center}
		\vspace{1cm}
		\Large\textcolor{NordBrightBlue}{\textbf{Применение нейронных сетей к проблемам генерации
				задач по планиметрии}}\\
	\end{center}
	\vspace{0.5cm}
	\large\textcolor{NordBlue}{\textit{Докладчик: Суматохина А.С.}}\\
	\large\textcolor{NordBlue}{\textit{Научный руководитель: д.ф.-м.н., проф. Семенов Е.М.}}\\
	\large\textcolor{NordBlue}{\textit{Научный консультант: асп. Авдеев Н.Н.}}\\
	\vspace{0.5cm}
	\begin{center}
		16 апреля 2025 г.
	\end{center}
	\begin{center}
		Воронеж, ВГУ
	\end{center}

\end{frame}

\begin{frame}{Существующие проблемы}
	\begin{itemize}
		%TODO: Тут написан какой-то бред
		\item Дефицит заданий для подготовки
		\item При появлении новых заданий в экзамене — дефицит материалов увеличивается в разы
		\item Списывание ответов учениками
		\item Несоответствие чертежей с условиями задачи
	\end{itemize}
\end{frame}

\begin{frame}{Проект «Час ЕГЭ»}
	\large
	«Час ЕГЭ» — компьютерный образовательный проект, разрабатываемый с 2013~года при математическом факультете ВГУ в рамках «OpenSource кластера» и предназначенный для помощи учащимся старших классов подготовиться к тестовой части единого государственного экзамена
\end{frame}

%\begin{frame}{Достижения}
%	\begin{itemize}
%		%Там уже было, но я ДОПИСАЛА за год
%		%добавить
%		\item Полностью покрыт открытый банк заданий ФИПИ ЕГЭ профильного уровня по темам:
%		      \begin{itemize}
%			      \item Планиметрия — 26 шаблонов принято
%			      \item Вектора — 18 шаблонов (10 принято, 8 на рецензировании)
%			      \item Стереометрия — 61 шаблонов (8 принято, 53 на рецензировании)
%			      \item Теория вероятности — 10 шаблонов (9 принято, 1  на рецензировании)
%			      \item Теория вероятности (повышенной сложности) — 11 шаблонов принято
%			      \item Производная и первообразная — 17 шаблонов на рецензировании
%		      \end{itemize}
%	\end{itemize}
%\end{frame}

\begin{frame}{Достижения}
	\begin{itemize}
		\item Покрыт ОБЗ ФИПИ ЕГЭ базового уровня по темам:
		      \begin{itemize}
			      \item Планиметрии — 60 шаблонов на внутреннем рецензировании (code review)
			      \item Теории вероятности — 11 шаблонов (2  принято, 9 на code review)
		      \end{itemize}
		\item Покрыт ОБЗ ФИПИ ОГЭ по темам:
		\begin{itemize}
			\item Планиметрия по теме треугольники — 16 шаблонов на code review
		\end{itemize}
		\item По теме графики функций — 2 шаблона на code review
	\end{itemize}
\end{frame}

\begin{frame}{Библиотека для работы с геометрическими фигурами}
	\begin{itemize}
		\item Разработана библиотека \texttt{flatten-shape-geometry} и опубликована на NPM, зависящая от библиотек:
		      \texttt{@flatten-js/core}, \texttt{mathjs} и других
		\item \texttt{flatten-shape-geometry} содержит классы: ShapeWithConnectionMatrix, Angle,  Triangle, Quadrilateral, Square, Rectangle, Rhombus, Parallelogram, Trapezoid
	\end{itemize}
\end{frame}

\begin{frame}{Библиотека для работы с геометрическими фигурами}
	\begin{itemize}
		\item В библиотеке пресутсвуют вспомогательные функции:
		\begin{itemize}
			\item Нахождения перпендикуляра опущенного из точки на отрезок
			\item Сдвига координат
			\item Нахождения центра описанной окружности
		\end{itemize}
		\item Для класса Circle из \texttt{@flatten-js/core} написаны методы:
		      \begin{itemize}
			      \item Получения точки на окружности по заданному углу
			      \item Построения отрезков: радиусов, диаметров и хорд
			      \item Построения касательных из точки к окружности
		      \end{itemize}
	\end{itemize}
\end{frame}

\begin{frame}{Устройство flatten-shape-geometry}
	\begin{center}
		\begin{tikzpicture}[
				roundnode/.style={ellipse, draw=NordBlack, fill=NordBlack!5, very thick, minimum size=5mm},
				squarednodeclass/.style={rectangle, draw=NordGreen, fill=NordGreen!10, very thick, minimum size=5mm},
				squarednodefunc/.style={rectangle, draw=NordOrange, fill=NordOrange!10, very thick, minimum size=5mm},
				squarednodemethod/.style={rectangle, draw=NordMagenta, fill=NordMagenta!10, very thick, minimum size=5mm},
				line width=1.2pt
			]

			% Nodes
			\node[roundnode]      (polygon)      					{Polygon};
			\node[squarednodeclass]    (swcm)       [below=of polygon] 	{SWCM};
			\node[roundnode]      (vector)     [right=of polygon]  {Vector};
			\node[roundnode]      (point)     [right=of vector] {Point};
			\node[squarednodeclass]    (angle)      [below=of point] 		{Angle};
			\node[squarednodefunc]      (general)    	[right=of point] {general};
			\node[roundnode]      (line)    [right=of general] {Line};
			\node[roundnode]      (segment)    [below=of general] {Segment};
			\node[roundnode]      (circle)    	[right=of segment] {Circle};
			\node[squarednodemethod]      (ecm)    	[below=of circle] {ECM};

			\node[squarednodeclass](triangle)  	   [below=of vector]  {Traingle};
			\node[squarednodeclass](quadrilateral)  [below=of triangle]  {Quadrilateral};

			\node[squarednodeclass](rhombus)  	[below=of quadrilateral]  {Rhombus};
			\node[squarednodeclass](trapezoid)  	[right=of rhombus]  {Trapezoid};
			\node[squarednodeclass](rectangle)  	[left=of rhombus]  {Rectangle};
			\node[squarednodeclass](square)  	[right=of trapezoid]  {Square};

			%Edges
			\draw[->, dashed] (angle) --(vector) ;
			\draw[->, dashed] (angle) -- (segment);
			\draw[->, dashed] (angle) -- (point);

			\draw[->, dashed] (general) --(line) ;
			\draw[->, dashed] (general) -- (segment);
			\draw[->, dashed] (general) -- (point);

			\draw[->] (swcm) -- (polygon);
			\draw[->] (triangle) -- (swcm);
			\draw[->, dashed] (triangle) -- (angle);
			\draw[->] (quadrilateral) -- (swcm);
			\draw[->, dashed] (quadrilateral) -- (angle);

			\draw[->] (rhombus) -- (quadrilateral);
			\draw[->] (trapezoid) -- (quadrilateral);
			\draw[->] (rectangle) -- (quadrilateral);
			\draw[->] (square) -- (quadrilateral);

			\draw[->, dashed] (ecm) -- (circle);
			\draw[->, dashed] (ecm) -- (segment);

		\end{tikzpicture}
	\end{center}
	SWCM - ShapeWithConnectionMatrix; \\ ECM - Extra Circle Methods;
\end{frame}

\begin{frame}[fragile]{Раннее успешное использование нейросетей для генерации кода}
	\begin{leftBox}
		write a 2D to 3D projector in JS without using libraries. input:  point = \{x,y,z\}, camera= \{x,y,z, angleX, angleY, angleZ, scale\}
	\end{leftBox}

	\begin{oldrightBox}
		function project2Dto3D(point, camera) { ...	return { x: translatedX, y: translatedY, z: translatedZ };
	\end{oldrightBox}

	\begin{leftBox}
		shorten function project2Dto3D
	\end{leftBox}

	\begin{oldrightBox}
		function project2Dto3D(point, camera) { ...	return { x: rx + cx, y: ry + cy, z: rz + cz };
	\end{oldrightBox}
\end{frame}

\begin{frame}[fragile]{Тестирование класса Triangle при помощи DeepSeek R1}
	\begin{leftBox}
		Напиши тест для инициализации треугольника через его координаты $A(0,0), B(4,0), C(0,3)$
	\end{leftBox}
	
	\begin{rightBox}
		\greencheck Корректная инициализация вершин
		\\\greencheck Правильные длины сторон
		\\\greencheck Корректное значение угла A
		\\\greencheck Корректное значение угла B
		\\\greencheck Корректное значение угла C
		\\\greencheck Обработка невалидных входных данных
	\end{rightBox}
\end{frame}

\begin{frame}[fragile]{Тестирование класса Triangle при помощи DeepSeek R1}
	\begin{leftBox}
		Напиши тест для инициализации треугольника через его координаты $A(0,0), B(4,4), C(5,0)$
	\end{leftBox}
	
	\begin{rightBox}
		\greencheck Корректное значение угла A
		\\ \redsolid Корректное значение угла B
		\\\textcolor{ForestGreen}{Expected: 104.04} 
		\\\textcolor{BrickRed}{Received: 59.03624346792648} 
		\\\redsolid Корректное значение угла C
		\\\textcolor{ForestGreen}{Expected: 30.96} 
		\\\textcolor{BrickRed}{Received: 75.96375653207353} 
	\end{rightBox}
\end{frame}

\begin{frame}[fragile]{Тестирование метода Circle.tangentsFromPoint при помощи DeepSeek R1}
	\begin{leftBox}
		Напиши тесты для метода tangentsFromPoint для окружности с центром в $(0,0)$, радиусом $5$ и точкой $(10, 0)$
	\end{leftBox}
	
	\begin{rightBox}
		%\greencheck должен обрабатывать окружность со смещенным центром
		%\\\greencheck должен выбрасывать ошибку для точки на окружности
		%\redsolid должен возвращать корректные длины отрезков без дополнительных настроек
		%\\\textcolor{ForestGreen}{Expected: 0} 
		%\textcolor{BrickRed}{Received: 8.660254037844387} 
		\redsolid корректность координаты x первой точки касания
		\\\textcolor{ForestGreen}{Expected: 3.5355} 
		\textcolor{BrickRed}{Received: -4.330127018922194} 
		\\\redsolid корректность координаты y первой точки касания
		\\\textcolor{ForestGreen}{Expected: 3.5355} 
		\textcolor{BrickRed}{Received: 2.4999999999999996} 
		\\\redsolid корректность координаты x второй точки касания
		\\\textcolor{ForestGreen}{Expected: -3.5355} 
		\textcolor{BrickRed}{Received: 4.330127018922194} 
		\\\redsolid корректность координаты y второй точки касания
		\\\textcolor{ForestGreen}{Expected: 3.5355} 
		\textcolor{BrickRed}{Received: 2.4999999999999996} 
	\end{rightBox}
\end{frame}

\begin{frame}{Итоги}
	\begin{itemize}
		\item Нейросети способны генерировать простейшие тесты без уточнений от программиста
		\item При усложнении задачи качество тестов существенно падает
		\item Нейросети не могут полностью заменить программиста в написании тестов
		\item Требуется экспертная оценка и доработка сгенерированных тестов человеком
		\item Написано 1115 шаблонов 
		\item Отредактировано 369 шаблонов
	\end{itemize}
\end{frame}

\begin{frame}{Список используемых источников}
	\begin{thebibliography}{10}
		\bibitem{chasdok1} Момот Е. А., Арахов Н. Д. Разработка и внедрение ПО для сбора статистики результатов подготовки к ЕГЭ по математике профильного уровня //Актуальные проблемы прикладной математики, информатики и механики. – 2021. – С. 1-2.
		\bibitem{npmjs} Node Package Manager. – URL: https://www.npmjs.com/
		\bibitem{egemath} Открытый банк заданий ЕГЭ. – URL: https://fipi.ru/ege/otkrytyy-bank-zadaniy-ege
		\bibitem{ogemath} Открытый банк заданий ОГЭ – URL:  https://fipi.ru/oge/otkrytyy-bank-zadaniy-oge
	\end{thebibliography}
\end{frame}

\begin{frame}
	\center\large\textcolor{NordBrightBlue}{\textbf{Спасибо за внимание}}\\
	\hfill \break
	\normalsize
	Все добавленные в проект задания можно сгенерировать по ссылке\\
	\hfill \break

	\includegraphics[width=0.25\textwidth]{images/QR-code}\\
	\hfill \break

	\url{https://math.vsu.ru/chas-ege/sh/katalog.html}
\end{frame}

\end{document}
