
\begin{taskBN}{1}
Площадь параллелограмма $DFZK$ равна 32. Точка $M$ – середина стороны $ZK$. Найдите площадь треугольника $DKM$.
\end{taskBN}

\begin{taskBN}{2}
Основанием пирамиды является прямоугольник. Чему равен объём пирамиды, если высота пирамиды составляет 5, при этом вторая сторона основания составляет 3? Первая сторона составляет 3. 
\end{taskBN}

\begin{taskBN}{3}
В среднем из 1000 бензопил, поступивших в продажу, 995 не имеют дефектов. Найдите вероятность того, что один случайным образом выбранный экземпляр товара не имеет дефектов.
\end{taskBN}

\begin{taskBN}{4}
Из города в деревню ежедневно ходит электричка. Вероятность того, что в среду в электричке окажется меньше 23 пассажиров, равна 0.85. Вероятность того, что окажется меньше 18 пассажиров, равна 0.67. Найдите вероятность того, что число пассажиров будет от 18 до 22.
\end{taskBN}

\begin{taskBN}{5}
Найдите корень уравнения $$\frac{47}{5x-35}=\frac{47}{13x+5}$$
\end{taskBN}

\begin{taskBN}{6}
Найдите значение выражения $${\frac{\sqrt{5,2}\cdot\sqrt{6,8}}{\sqrt{2,21}}}$$
\end{taskBN}

\begin{taskBN}{7}
    \addpictoright[0.8\linewidth]{images/9265036894634555n0}На рисунке изображены график функции $y=f(x)$ и касательная к этому графику, проведённая в точке $d$. Уравнение касательной имеет вид $y=0{,}17 x+2{,}68$. Найдите значение функции $g(x) = 2{,}5\cdot(f'(x)+6{,}5)$ в точке $d$.
\end{taskBN}

\begin{taskBN}{8}
Зависимость объёма спроса $q$ (единиц в месяц) на продукцию предприятия-монополиста от цены $p$ (тыс. руб.) задаётся формулой $q=289-17p$. Выручка предприятия за месяц $r$ (в тыс. руб.) вычисляется по формуле $r(p)=q\cdot p$. Определите наибольшую цену $p$, при которой месячная выручка $r(p)$ составит не менее 884 тыс. руб. Ответ приведите в тыс. руб.
\end{taskBN}

\begin{taskBN}{9}
На сколько деталей в час первый рабочий делает больше, чем второй, если первый рабочий выполняет заказ на 9 часов быстрее, чем второй, а количество деталей в заказе равно 108? Первый рабочий делает в час 12 деталей. 
\end{taskBN}

\begin{taskBN}{10}
\addpictocenter[]{images/8457782638685436n0}На рисунке изображён график функции $f(x)=\frac{kx+a}{x+b}$. Найдите значение $x$, при котором $f(x)=1{,}75$.
\end{taskBN}

\begin{taskBN}{11}
Найдите наименьшее значение функции $y=e^{42x}-42e^x+37$ на отрезке $[-40; 3]$.
\end{taskBN}
%%\newpage Ответы

%%\begin{table}\begin{tabular}{lll}\\2 & 1 & 8\\2 & 2 & 15\\2 & 3 & 0,995\\2 & 4 & 0,18\\2 & 5 & -5\\2 & 6 & 4\\2 & 7 & 16,675\\2 & 8 & 13\\2 & 9 & 6\\2 & 10 & 10\\2 & 11 & -4\end{tabular}\end{table}