\documentclass[a4paper, 12pt]{extarticle}
\usepackage{fontspec}
\usepackage{polyglossia}
\setmainfont{CMU Serif}
\newfontfamily{\cyrillicfont}{CMU Serif}
\setsansfont{CMU Sans Serif}
\newfontfamily{\cyrillicfontsf}{CMU Sans Serif}
\setmonofont{CMU Typewriter Text}
\newfontfamily{\cyrillicfonttt}{CMU Typewriter Text}
\setdefaultlanguage{russian}
\usepackage[left=1cm,right=1cm,
top=2cm,bottom=2cm]{geometry}
%%% Дополнительная работа с математикой
\usepackage{amsfonts,amssymb,amsthm,mathtools} % AMS
\usepackage{amsmath}
\usepackage{icomma} % "Умная" запятая: $0,2$ --- число, $0, 2$ --- перечисление

%% Шрифты
\usepackage{euscript} % Шрифт Евклид
\usepackage{mathrsfs} % Красивый матшрифт

%% Свои команды
\DeclareMathOperator{\sgn}{\mathop{sgn}}


%% Перенос знаков в формулах (по Львовскому)
\newcommand*{\hm}[1]{#1\nobreak\discretionary{}
	{\hbox{$\mathsurround=0pt #1$}}{}}

%%% Работа с картинками
\usepackage{graphicx}  % Для вставки рисунков
\graphicspath{{Изображения/}{image}}  % папки с картинками
\setlength\fboxsep{3pt} % Отступ рамки \fbox{} от рисунка
\setlength\fboxrule{1pt} % Толщина линий рамки \fbox{}
\usepackage{wrapfig} % Обтекание рисунков и таблиц текстом

%%% Работа с таблицами
\usepackage{array,tabularx,tabulary,booktabs} % Дополнительная работа с таблицами
\usepackage{longtable}  % Длинные таблицы
\usepackage{multirow} % Слияние строк в таблице
\usepackage{csquotes}

\title{Доклад на тему История педагогических учений К.Д. Ушинский}
\author{Суматохина Александра 3 курс 4 группа}
\date{6 марта 2023}
\begin{document}
\maketitle

\subsection*  {Деятельность}
Константин Дмитриевич Ушинский (1824–1871) — родоначальник научного подхода к педагогике в России. Он проделал долгий путь до того, как стал известным педагогом. За свою жизнь Ушинский успел отучиться на юридическом ( где его и "заразил" теорией падагогики профессор Пётр Редкин), поработать педагогом, уйти в чиновники, зарабатывать переводами, и в конце концов вернуться в педагоги.

В 1854 году Ушинскому удалось устроиться учителем русской словесности и юридических предметов в Гатчинский сиротский приют. Это заведение славилось жестокими методами воспитания. За пять лет работы там Ушинскому удалось изменить местные порядки, а ещё искоренить доносительство и воровство среди воспитанников.

Однажды он обнаружил в лицее пыльный шкаф с никому не нужными книгами по педагогике — и нырнул в них с головой. Боже мой! — от скольких бы грубых ошибок был избавлен я, если бы познакомился с ними прежде, чем вступил на педагогическое поприще! — вспоминал он потом.

Это открытие вдохновило Ушинского, он стал публиковаться в Журнале для воспитания, и на его свежие идеи обратил внимание не кто-нибудь, а сама императрица Мария Александровна. Он показался ей подходящей кандидатурой для реформ в Смольном институте благородных девиц, о которых она давно задумывалась. Институт тогда представлял собой просто пародию на образование — воспитывавшиеся там девицы толком ничего не знали, кроме французского, и выпускались жеманными невеждами.

Так в 1859 году 35-летний Ушинский стал инспектором классов Смольного института. Эта должность главного по содержанию и методикам обучения. Ушинский везде был реформатором, но в Смольном впервые получил на это неограниченные полномочия. Однако от проблем это его не спасло.

Ушинский  заменил механическую бездумную зубрёжку, которая там раньше была единственным методом, на настоящее преподавание и обучение, поменял учителей. 
При нём девочек начали обучать естественным наукам (раньше считалось, что им это ни к чему), а ещё он открыл дополнительный старший класс, в котором могли остаться выпускницы, желавшие получить профессию домашней учительницы.

Но главное, Ушинский поменял мировоззрение институток, включил у них живое мышление и постарался сделать более свободной царившую там атмосферу ханжеской морали и подавления индивидуальности.

Всё это Ушинский проделал меньше чем за три года, но ломкой традиций сразу нажил себе врагов в лице начальницы института и истеричных классных дам. Ему строили козни, на него писали жалобы, и работать в состоянии вечного конфликта стало невозможно — снова пришлось уйти.

После этого Ушинского командировали за границу — изучать и описывать опыт передовых европейских школ. К педагогической практике он больше не вернулся, но написал первые массовые учебники и руководства для начального обучения.

\section*  {Книги}
Главные книги Ушинского: Детский мир. Хрестоматия (1861), Родное слово (1864), фундаментальный труд Человек как предмет воспитания. Опыт педагогической антропологии (1868–1869).

\section*  {Идеи}
Ушинский придерживался и последовательно развивал много идей:
\begin{itemize}
        \item Образование должно быть обязательным для всех вне зависимости от сословия. Женщины имеют такое же право на образование, как и мужчины (напомним, в педагогику Ушинский пришёл ещё до отмены крепостного права, а за равные права на образование женщины в Российской империи боролись вплоть до революции 1917 года, так что эти его идеи сильно опередили время).
        \item Педагогика не может опираться лишь на личный опыт учителя, ведь тот может оказаться ошибочным. Она должна опираться на теорию, то есть всестороннее изучение человека и систематизированный опыт. Поэтому педагогические идеи должны развиваться в университете, опираясь на науку. Так, в Человеке как предмете воспитания Ушинский подробно описывал не только психологию, но и физиологию человека. Благодаря этому он объяснял, например, почему дети не могут долго заниматься одним монотонным делом и почему обучение для них должно быть наглядным.
    
        \item Обучение — не механическая зубрёжка, а развитие умственных способностей ученика, наблюдательности, воображения, фантазии, желания и способности дальше приобретать знания самостоятельно. Обучение должно быть сознательным, то есть до учащихся нужно донести, зачем они учатся и чему в итоге научатся.
    
        \item Обучение должно быть системным и последовательным. 

        \item От конкретного — к отвлечённому, от знакомого — к незнакомому, от единичного — к сложному, от частного — к общему. В учебном материале нужно определённым образом расположить материал для повторения и практические задачи. Это обеспечит прочность знаний. Важно научить применять знания на практике, оперировать ими в разных ситуациях.
    
        \item Задача первоначального обучения — сделать серьёзное занятие увлекательным для ребёнка. Но с увлекательностью важно не переборщить. Если превратить вообще всё обучение в игру, то ребёнок не сможет дальше осваивать не очень интересные, но важные знания. Он должен привыкнуть к тому, что обучение — всё-таки труд.
    
        \item Главная задача педагогики — воспитание нравственности, а не наполнение головы знаниями. 
    
        \item Обучение — лишь средство воспитания. Школа должна готовить человека к жизни и труду.
    
        \item Воспитание должно быть гуманным. Физические наказания и унижения неприемлемы. Воспитание — это убеждение личным примером, а не слепое повиновение.
    
        \item Воспитание и образование должны учитывать культурные и языковые особенности народа. Однако это не значит, что русская школа должна быть какой-то уникальной, не как у людей. Законы души и её развития везде одинаковы.
\end{itemize}

\subsection*{Несколько цитат Ушинского.}

\begin{quote}
    \textit{Если педагогика хочет воспитывать человека во всех отношениях, то она должна прежде узнать его тоже во всех отношениях.}
\end{quote}

\begin{quote}
    \textit{ Если всякий преподаватель станет произвольно выбирать для себя методу преподавания, а всякий воспитатель — методу воспитания, то в общественных заведениях, особенно в больших, из такого разнообразия может произойти значительный вред. Но как бы ни было вредно разнообразие, происходящее от различных убеждений, оно во всяком случае полезнее мёртвого однообразия.
}\end{quote}

\begin{quote}
    \textit{Учение, лишённое всякого интереса и взятое только силой принуждения, убивает в ученике охоту к учению, без которой он далеко не уйдёт, а учение, основанное только на интересе, не даёт возможности окрепнуть самообладанию и воле ученика, так как не всё в учении интересно и придёт многое, что надобно будет взять силою воли.}
\end{quote}

\begin{quote}
    \textit{Приступая к святому делу воспитания детей, мы должны глубоко сознавать, что наше собственное воспитание было далеко не удовлетворительно, что результаты его большею частью печальны и жалки и что, во всяком случае, нам надо изыскивать средства сделать детей наших лучше нас.
    В преподавателе среднего учебного заведения знание предмета далеко не составляет главного достоинства. <…> Но главное достоинство гимназического преподавателя состоит в том, чтобы он умел воспитывать учеников своим предметом.}
\end{quote}

\begin{quote}
    \textit{В воспитании всё должно основываться на личности воспитателя, потому что воспитательная сила изливается только из живого источника человеческой личности. Никакие уставы и программы, никакой искусственный организм заведения, как бы хитро он ни был придуман, не может заменить личности в деле воспитания.
    }
\end{quote}

\begin{quote}
    \textit{Страх телесного наказания не сделает злого сердца добрым, а смешение страха со злостью — самое отвратительное явление в человеке.}
\end{quote}
\end{document}