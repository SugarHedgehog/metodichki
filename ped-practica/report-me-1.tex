\documentclass[a4paper,12pt]{article}
\usepackage[utf8]{inputenc}
\usepackage[russian]{babel}
\usepackage{geometry}
\geometry{top=10mm, bottom=10mm, left=10mm, right=10mm}
\pagestyle{empty}
\begin{document}

\begin{center}
\textbf{ОТЧЕТ}\\
о прохождении учебной педагогической практики\\
студентки 5-го курса математического факультета\\
Суматохиной Александры Сергеевны
\end{center}

\noindent
\textbf{Предмет: Математика}\\
\textbf{Класс: 5-й}\\

\subsection*{Введение}

В ходе педагогической практики был составлен план конспекта урока по теме "Углы: острые, прямые, тупые; измерение углов с помощью транспортира и план воспитательного мероприятия по теме "Профилактика детского дорожно-транспортного травматизма". Целью урока является познакомить учеников с видами углов (острым, прямым, тупым) и научить измерять углы с помощью транспортира. Целью мероприятия является повышение осведомленности учащихся о правилах дорожного движения и мерах безопасности на дороге.

\subsection*{Описание урока по математике}
Урок по математике состоит из шести этапов:
\begin{itemize}
    \item Вводное слово (2 минуты)
    \item Презентация материала (12 минут)
    \item Пример определения угла с помощью ладони (8 минут)
    \item Практическая работа (12 минут)
    \item Обсуждение и заключение (6 минут)
    \item Домашнее задание (5 минут)
\end{itemize}

\subsection*{Описание классного часа по теме "Профилактика детского дорожно-транспортного травматизма"}
Классный час включает в себя следующие этапы:
\begin{itemize}
    \item Введение (5 минут): краткое обсуждение важности безопасности на дороге.
    \item Презентация правил безопасного поведения (10 минут): демонстрация слайдов с правилами и примерами.
    \item Практическая игра (15 минут): игра «Улица без опасности», где дети выполняют задачи по безопасному поведению на дороге.
    \item Обсуждение и выводы (10 минут): обсуждение правил и их применения в реальной жизни.
    \item Домашнее задание (5 минут): рассказать родителям о правилах безопасного поведения на дороге и предложить им помочь в их изучении.
\end{itemize}

Материалы: презентация, листы бумаги, карандаши, фишки для игры.

\subsection*{Оценка результатов}

Урок по математике и классный час были спланированы для эффективного обучения детей новым знаниям и навыкам. Практические задания и игры помогли сделать обучение интересным и запоминающимся.

\subsection*{Личный вклад и выводы}

Разработка плана урока и классного часа позволила мне развить навыки планирования и организации учебного процесса. Опыт работы над этими мероприятиями дал ценные навыки, которые будут использованы в дальнейшей профессиональной деятельности.
\vspace{10mm}

\noindent
Дата \underline{\hspace{5cm}}
\begin{tabbing}
\hspace{2cm} \= \kill
Подпись \> \underline{\hspace{4cm}}
\end{tabbing}


\end{document}
