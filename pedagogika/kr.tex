\documentclass[a4paper, 14pt]{article}
\usepackage{fontspec}
\usepackage{polyglossia}
\setmainfont{CMU Serif}
\newfontfamily{\cyrillicfont}{CMU Serif}
\setsansfont{CMU Sans Serif}
\newfontfamily{\cyrillicfontsf}{CMU Sans Serif}
\setmonofont{CMU Typewriter Text}
\newfontfamily{\cyrillicfonttt}{CMU Typewriter Text}
\setdefaultlanguage{russian}

%%% Дополнительная работа с математикой
\usepackage{amsfonts,amssymb,amsthm,mathtools} % AMS
\usepackage{amsmath}
\usepackage{icomma} % "Умная" запятая: $0,2$ --- число, $0, 2$ --- перечисление

%% Шрифты
\usepackage{euscript} % Шрифт Евклид
\usepackage{mathrsfs} % Красивый матшрифт

%% Свои команды
\DeclareMathOperator{\sgn}{\mathop{sgn}}


%% Перенос знаков в формулах (по Львовскому)
\newcommand*{\hm}[1]{#1\nobreak\discretionary{}
	{\hbox{$\mathsurround=0pt #1$}}{}}

%%% Работа с картинками
\usepackage{graphicx}  % Для вставки рисунков
\graphicspath{{Изображения/}{image}}  % папки с картинками
\setlength\fboxsep{3pt} % Отступ рамки \fbox{} от рисунка
\setlength\fboxrule{1pt} % Толщина линий рамки \fbox{}
\usepackage{wrapfig} % Обтекание рисунков и таблиц текстом

%%% Работа с таблицами
\usepackage{array,tabularx,tabulary,booktabs} % Дополнительная работа с таблицами
\usepackage{longtable}  % Длинные таблицы
\usepackage{multirow} % Слияние строк в таблице
\usepackage{blindtext}
\usepackage{multicol}
\usepackage{pdfpages}
\usepackage[left=1cm,right=1cm,top=1cm,bottom=1cm]{geometry}
\begin{document}
\begin{center}
    \hfill \break
    \large{МИНОБРНАУКИ РОССИИ}\\
    \footnotesize{ФЕДЕРАЛЬНОЕ ГОСУДАРСТВЕННОЕ БЮДЖЕТНОЕ ОБРАЗОВАТЕЛЬНОЕ УЧРЕЖДЕНИЕ}\\ 
    \footnotesize{ВЫСШЕГО ПРОФЕССИОНАЛЬНОГО ОБРАЗОВАНИЯ}\\
    \small{\textbf{«ВОРОНЕЖСКИЙ ГОСУДАРСТВЕННЫЙ УНИВЕРСИТЕТ»}}\\
    \hfill \break
    \normalsize{Математический факультет}\\
     \hfill \break
    \normalsize{Кафедра теории функций и геометрии}\\
    \hfill\break
    \hfill \break
    \hfill \break
    \hfill \break
    %%\large{Начально-краевая задача для волнового уравнения на геометрическом графе-звезде}\\
    \hfill \break
    \hfill \break
    \hfill \break
    \hfill \break
    \hfill \break
    \hfill \break\
    \hfill \break
    \hfill \break
    \normalsize{Контрольная работа\\
    \hfill \break
    Направление  010501 Фундаментальные математика и механика\\

    \hfill \break
    }\\
    \hfill \break
    \hfill \break
    \end{center}
    \hfill \break
     
    \normalsize{ 
    \begin{tabular}{cccc}
    Зав.кафедрой & \underline{\hspace{3cm}} &  д.физ.-мат.н.,  проф. & Е.М. Семёнов \\\\
    Обучающийся & \underline{\hspace{3cm}} & &А.С. Суматохина \\\\
    Руководитель & \underline{\hspace{3cm}}&  & О.Б. Мазкина \\\\
    \end{tabular}
    }\\
    \hfill \break
    \hfill \break
    \hfill \break
    \hfill \break
    \hfill \break
    \hfill \break
    \hfill \break
    \hfill \break
    \hfill \break
    \hfill \break
    \begin{center} Воронеж 2023 \end{center}
    \thispagestyle{empty} % выключаем отображение номера для этой страницы
     
    % КОНЕЦ ТИТУЛЬНОГО ЛИСТА

\documentclass[a4paper, 12pt]{extarticle}
\usepackage{fontspec}
\usepackage{polyglossia}
\setmainfont{CMU Serif}
\newfontfamily{\cyrillicfont}{CMU Serif}
\setsansfont{CMU Sans Serif}
\newfontfamily{\cyrillicfontsf}{CMU Sans Serif}
\setmonofont{CMU Typewriter Text}
\newfontfamily{\cyrillicfonttt}{CMU Typewriter Text}
\setdefaultlanguage{russian}

\usepackage[left=1.5cm,right=1cm,top=2cm,bottom=2cm]{geometry}
\title{Доклад на тему Программная реализация(на языке JavaScript) алгоритмов генерации ФОС по математике 2023}
\author{Суматохина Александра 3 курс Кафедра Теории функции и геометрии}
\date{13 марта 2023}
\begin{document}
    \maketitle
\subsection*{Существующие проблемы}
Единый государственный экзамен (ЕГЭ) — централизованно проводимый в Российской Федерации экзамен в средних учебных заведениях — школах, лицеях и гимназиях, форма проведения ГИА(Государственный Итоговая Аттестация) по образовательным программам среднего общего образования. 
Служит одновременно выпускным экзаменом из школы и вступительным экзаменом в вузы.

За два года подготовки к ЕГЭ школьники сталкиваются с дефицитом заданий для подготовки.
А учителя со списыванием ответов при решении задач экзамена учениками. 
Также в в конце 2021 года в список заданий ЕГЭ были добавлены новые задания под номером 9, 
количество которых для прорешивания очень мало. 
Также существуют задания, решение которых занимает менее минуты, а их составление вручную занимает несоразмерно много времени. 
Проект «Час ЕГЭ» позволяет решить все эти проблемы.

«Час ЕГЭ» — компьютерный образовательный проект, разрабатываемый при математическом 
факультете ВГУ в рамках «OpenSource кластера» и предназначенный для помощи учащимся 
старших классов подготовиться к тестовой части единого государственного экзамена.

Задания в «Час ЕГЭ» генерируются случайным образом по специализированным алгоритмам, называемых шаблонами, каждый из которых охватывает множество вариантов соответствующей ему задачи. 

\subsection*{Демонстрация генерации заданий}
%%Тут нужна подводка к примерам
\subsection*{Этапы генерации заданий № 10}

Первый этап: Генерация коэффициентов функций

Второй этап: Подсчитываются и находятся точки, которые находятся в узлах целочисленной сетки (функция \texttt{intPoints}).

Третий этап: Отрисовывается целочисленная сетка, оси координат и единичный  (функция \texttt{drawCoordinatePlane})

Четвёртый этап: Отрисовка графика (функция \texttt{graph9AdrawFunction})

Пятый этап: Отображение нескольких точек, найденных на втором этапе (функция \texttt{graph9AmarkCircles})

\subsection*{Достижения}
\begin{itemize}
    \item Полностью покрыт открытый банк заданий ФИПИ
    \item Разработано 35 шаблонов
    \item В ядро добавлено несколько вспомогательных функций, которые позволят быстро разрабатывать новые шаблоны при добавлении новых прототипов в открытый банк заданий
\end{itemize}

\subsection*{Этапы генерации заданий № 7}
    Первый этап: Генерация точек, через которые будет проходить функция
        
    Первый этап: Использование сторонней библиотеки \texttt{cubic-spline} для построения графика функции по точкам сплайна третьего порядка.
        
    Второй этап: Проверка того, что функция не вышла за рамки видимости, и все экстремумы видны. %что за бред?
    
    Третий этап: Нахождение количества экстремумов функции.
    Здесь же проводятся дополнительные проверки, в примеру, экстремумы должны быть явно видны для решающего. 
        
    Четвёртый этап: Отрисовка графика функции и краевых точек.
    
\subsection*{Кубический сплайн}

Кубическим сплайном функции $y = f(x)$, $x\in[a, b]$ на сетке $a=x_0<x_1<x_2< \dots <x_n=b$ назовём функцию $S(x)$, удовлетворяющую условиям:
    \begin{enumerate}
        \item На каждом отрезке $[x_{i-1},x_i]$, функция $S(x)$ является полиномом третьей степени.
        \item Функция $S(x)$, ее первая $S'(x)$ и вторая $S''(x)$ производные непрерывны на сегменте $[a, b]$.
        \item $S(x_i)=f(x_i)=f_i, i=0,\dots,n$.
\end{enumerate}

Этот метод построения является оптимальным, так как для заданий этого типа необходимы функции с большим количеством экстремумов на небольшом отрезке. При этом функция должна быть гладкой и непрерывной на этом отрезке.  

\end{document}

\end{document}