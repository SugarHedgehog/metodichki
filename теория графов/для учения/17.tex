\textbf{Внутреннее устойчивое множество}
Граф $G = (X, \Gamma)$, множество $S \subseteq V$ называется \textit{внутренне устойчивым}, если $\Gamma S \cap S = \emptyset$.
\textbf{Число внутренней устойчивости}
\[
\alpha(G) = \max_{S \in \mathfrak{S}} |S|
\]
\textbf{Связь с хроматическим числом}
\[
\alpha(G) \gamma(G) \geq |X|
\]
\textbf{Пример}
Граф с $\gamma(G) = 4$, где белые вершины образуют наибольшее внутренне устойчивое множество.

\textbf{Лемма 1}
\[
\alpha(G \times H) \geq \alpha(G) \cdot \alpha(H)
\]
\textbf{Емкость графа}
\[
\theta(G) = \sup_{n} \sqrt[n]{\alpha(G^n)}
\]

\textbf{Лемма 2}
Сохраняющее отображение $\sigma$ переводит $S$ во внутренне устойчивое множество $\sigma(S)$.

\textbf{Лемма 3}
Если $\sigma(X)$ внутренне устойчиво, то $\theta(G) = \alpha(G)$.

\textbf{Теорема 7 (Шеннон)}
Если для $G$ или $H$ существует $\sigma$, то
\[
\alpha(G \times H) = \alpha(G)\alpha(H)
\]
\textbf{Следствие}
Если $\sigma$ переводит вершины $G$ во внутренне устойчивое множество, то
\[
\alpha(G) = \sup_n \sqrt[n]{\alpha(G^n)} = \alpha(G)
\]