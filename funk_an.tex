\documentclass[a4paper, 12pt]{extarticle}
\usepackage{fontspec}
\usepackage{polyglossia}
\setmainfont{CMU Serif}
\newfontfamily{\cyrillicfont}{CMU Serif}
\setsansfont{CMU Sans Serif}
\newfontfamily{\cyrillicfontsf}{CMU Sans Serif}
\setmonofont{CMU Typewriter Text}
\newfontfamily{\cyrillicfonttt}{CMU Typewriter Text}
\setdefaultlanguage{russian}
\usepackage[left=1cm,right=1cm,
top=2cm,bottom=2cm]{geometry}
%%% Дополнительная работа с математикой
\usepackage{amsfonts,amssymb,amsthm,mathtools} % AMS
\usepackage{amsmath}
\usepackage{icomma} % "Умная" запятая: $0,2$ --- число, $0, 2$ --- перечисление

%% Шрифты
\usepackage{euscript} % Шрифт Евклид
\usepackage{mathrsfs} % Красивый матшрифт

%% Свои команды
\DeclareMathOperator{\sgn}{\mathop{sgn}}


%% Перенос знаков в формулах (по Львовскому)
\newcommand*{\hm}[1]{#1\nobreak\discretionary{}
	{\hbox{$\mathsurround=0pt #1$}}{}}

%%% Работа с картинками
\usepackage{graphicx}  % Для вставки рисунков
\graphicspath{{Изображения/}{image}}  % папки с картинками
\setlength\fboxsep{3pt} % Отступ рамки \fbox{} от рисунка
\setlength\fboxrule{1pt} % Толщина линий рамки \fbox{}
\usepackage{wrapfig} % Обтекание рисунков и таблиц текстом

%%% Работа с таблицами
\usepackage{array,tabularx,tabulary,booktabs} % Дополнительная работа с таблицами
\usepackage{longtable}  % Длинные таблицы
\usepackage{multirow} % Слияние строк в таблице
\begin{document}
\section{Метрики}
\begin{tabular}{|m{2.5cm}|m{6cm}|m{7cm}|}
    \hline
    Название              & Метрика                                                                  & Какое множество или пространство                                                                                  \\
    \hline

    Дискретная            & $
        \rho (x,y) =
    \begin{cases}
            1 & x=y      \\
            0 & x \neq y \\
        \end{cases}$       & X - произвольное непустое множество

    \\
    \hline

    $\mathbb{R}^n_p$      & $\rho_p (x,y)=(\sum_{k = 1}^{\infty}|x_k-y_k|^p)^\frac{1}{p}$            & \multirow{2}{6cm}{$\mathbb{R}^n$ - множество n-метрных векторов $x=(x_1,x_2,\dots,x_n)$  }                        \\
    \cline{1-2}
    $\mathbb{R}^n_\infty$ & $\rho_\infty (x,y)=max|x_k-y_k|$                                         &                                                                                                                   \\
    \hline
    $C[a,b]$              & $\rho (x,y)=max|x(t)-y(t)|$                                              & \multirow{2}{6cm}{Пространство числовых функций, непрерывных на $[a,b]$}                                          \\
    \cline{1-2}
    $C_1[a,b]$            & $\rho (x,y)=\int_{b}^{a}|x(t)-y(t)|\,dx $                                &                                                                                                                   \\
    \hline
    $M[a,b]$              & $\rho (x,y)=sup|x(t)-y(t)|$                                              & Пространство числовых функций, определённых и ограниченных на $[a,b]$                                             \\
    \hline
    $l_p$                 & $\rho_p (x,y)=(\sum_{k = 1}^{\infty}|x_k-y_k|^p)^\frac{1}{p}$            & Пространство числовых последовательностей $x=(x_1,x_2,\dots,x_k,\dots)$, суммируемых с $p$-той степенью           \\
    \hline
    $m$                   & $\rho (x,y)=sup|x_k-y_k|$                                                & Пространство произвольных числовых последовательностей $x=(x_1,x_2,\dots,x_k,\dots)$, таких что $sup|x_k|<\infty$ \\
    \hline
    $s$                   & $\rho_p (x,y)=\sum_{k = 1}^{\infty} \cfrac{|x_k-y_k|}{2^k(1+|x_k-y_k|)}$ & Пространство произвольных числовых последовательностей $x=(x_1,x_2,\dots,x_k,\dots)$                              \\
    \hline
\end{tabular}


\section{Определения}
\textbf{Метрика} - это число, поставленное в соответствие элементам $x$ и $y$ из произвольного множества $X$, такое что выполняются аксиомы для любых $x,y,z\in X$:
\[\rho(x,y)\geq 0,\rho(x,y)=0\longleftrightarrow x=y\]
\[\rho(x,y)=\rho(y,x)=\]
\[\rho(x,y)\leqslant \rho(x,z)+\rho(z,y)\]
\textbf{Метрическое пространство} - это $\{X,\rho\}$, если выполняются аксиомы.


b) открытый шар;
c) замкнутый шар;
d) ограниченное множество;
e) сходимость в метрическом пространстве;
f) фундаментальная последовательность;
g) полное метрическое пространство;
h) точка прикосновения множества;
i) замыкание множества;
j) изолированная точка множества;
k) предельная точка множества;
l) замкнутое множество;
m) внутренняя точка множества;
n) внутренность множества;
o) открытое множество;
p) подпространство метрического пространства;
q) совершенное множество;
r) всюду плотное множество;
s) множество, плотное в другом множестве;
t) нигде не плотное множество;
u) множество первой категории;
v) множество второй категории.
\end{document}