
\cleardoublepage

\def\examvart{Вариант 13.1}

\normalsize


\begin{center}
	\textbf{
		Единый государственный экзамен\\по МАТЕМАТИКЕ\\Профильный уровень\\ \qquad \\ Инструкция по выполнению работы
	}
\end{center}


\par \qquad Экзаменационная работа состоит из двух частей, включающих в себя 18 заданий. Часть 1 содержит 11 заданий с кратким ответом базового и повышенного уровней сложности. Часть 2 содержит 7 заданий с развёрнутым ответом повышенного и высокого уровней сложности.
\par \qquad На выполнение экзаменационной работы по математике отводится 3 часа 55 минут (235 минут).
\par \qquad Ответы к заданиям 1—11 записываются по приведённому ниже \underline {образцу} в виде целого числа или конечной десятичной дроби. Числа запишите в поля ответов в тексте работы, а затем перенесите их в бланк ответов №1.
%%\includegraphics[width=0.98\linewidth]{obrazec}
\par \qquad При выполнении заданий 12—18 требуется записать полное решение и ответ в бланке ответов №2.
\par \qquad  Все бланки ЕГЭ заполняются яркими чёрными чернилами. Допускается использование гелевой или капиллярной ручки.
\par \qquad При выполнении заданий можно пользоваться черновиком. \textbf{Записи в черновике, а также в тексте контрольных измерительных материалов не учитываются при оценивании работы.}
\par \qquad  Баллы, полученные Вами за выполненные задания, суммируются. Постарайтесь выполнить как можно больше заданий и набрать наибольшее количество баллов.
\par \qquad После завершения работы проверьте, что ответ на каждое задание в бланках ответов №1 и №2 записан под правильным номером.
\begin{center}
	\textit{\textbf{Желаем успеха!}}\\ \qquad \\\textbf{ Справочные материалы} \\
$\sin^2 \alpha + \cos^2 \alpha = 1$ \\
$\sin 2\alpha=2\sin \alpha \cdot \cos \alpha$ \\
$\cos 2\alpha=\cos^2 \alpha-\sin^2 \alpha$ \\
$\sin (\alpha+\beta)=\sin \alpha \cdot \cos \beta+\cos \alpha \cdot \sin\beta$ \\
$\cos (\alpha+\beta)=\cos \alpha \cdot \cos \beta-\sin\alpha \cdot \sin\beta$
\end{center}


\startpartone
\large

\begin{taskBN}{1}
Найдите хорду, на которую опирается угол $45^\circ$, вписанный в окружность диаметра $158\sqrt{2}$.
\end{taskBN}

\begin{taskBN}{2}
Объём куба равен $10,125\sqrt{3}$. Найдите диагональ куба.
\end{taskBN}

\begin{taskBN}{3}
Песенный конкурс проводится в 15 дней. Всего запланировано 440 выступлений — последние 4 дня по 55 выступлений, остальные распределены поровну между оставшимися днями. Порядок выступлений определяется жеребьёвкой. Какова вероятность, что выступление товарища П. окажется запланированным на предпоследний день мероприятия?
\end{taskBN}

\begin{taskBN}{4}
На фабрике игрушек 21\% изготовленных машинок имеют дефект. При контроле качества продукции выявляется 63\% дефектных машинок. Остальные машинки поступают в продажу. Найдите вероятность того, что случайно выбранная при покупке машинка с дефектами. Результат округлите до тысячных.
\end{taskBN}

\begin{taskBN}{5}
Найдите корень уравнения $$\log_{51}20-\log_{51}(17+3x)=0$$
\end{taskBN}

\begin{taskBN}{6}
Найдите значение выражения $$1:    8^{-6,5}\cdot9^{5,5}:    72^{5,5}$$
\end{taskBN}

\begin{taskBN}{7}
    \addpictoright[0.8\linewidth]{images/369283051242371n0}На рисунке изображен график производной функции $f(x)$, определенной на интервале $(-2; 8)$. Найдите количество точек, в которых касательная к графику функции $f(x)$ параллельна прямой $y=8{,}2$ или совпадает с ней.
\end{taskBN}

\begin{taskBN}{8}
При температуре $0^\circ {\rm{C}}$ рельс имеет длину $l_0=40$ м. При возрастании температуры происходит тепловое расширение рельса, и его длина, выраженная в метрах, меняется по закону $l(t^\circ ) = l_0 (1 + \alpha  \cdot t^\circ)$, где $\alpha=1.3\cdot 10^{-5}(^\circ {\rm{C}})^{-1}$ — коэффициент теплового расширения, $t^\circ$  — температура (в градусах Цельсия). При какой температуре рельс удлинится на 15.6 мм? Ответ выразите в градусах Цельсия.
\end{taskBN}

\begin{taskBN}{9}
Первая труба пропускает на 5 литров жидкости в минуту меньше, чем вторая. Сколько литров жидкости в минуту пропускает вторая труба, если ёмкость объёмом 20 литров она опустошает на 2 минуты быстрее, чем первая труба?

\end{taskBN}

\begin{taskBN}{10}
\addpictocenter[]{images/582540854558522n0}На рисунке изображён график функции $f(x)=\frac{kx+a}{x+b}$. Найдите значение $x$, при котором $f(x)=-6{,}5$.
\end{taskBN}

\begin{taskBN}{11}
Определите наименьшее значение функции $y =90-\frac{9\sqrt{3}}{2}x+\frac{3\sqrt{3}\pi}{2}-9\cos x$ на полуинтервале $\left[-\frac{\pi}{2};\frac{\pi}{2} \right)$
\end{taskBN}
\newpage Ответы

\begin{table}\begin{tabular}{ll}
   1 & 158\\2 & 4,5\\3 & 0,125\\4 & 0,09\\5 & 1\\6 & 8\\7 & 4\\8 & 30\\9 & 10\\10 & -19\\11 & 85,5\\
\end{tabular}\end{table}

\cleardoublepage

\def\examvart{Вариант 13.2}

\normalsize


\begin{center}
	\textbf{
		Единый государственный экзамен\\по МАТЕМАТИКЕ\\Профильный уровень\\ \qquad \\ Инструкция по выполнению работы
	}
\end{center}


\par \qquad Экзаменационная работа состоит из двух частей, включающих в себя 18 заданий. Часть 1 содержит 11 заданий с кратким ответом базового и повышенного уровней сложности. Часть 2 содержит 7 заданий с развёрнутым ответом повышенного и высокого уровней сложности.
\par \qquad На выполнение экзаменационной работы по математике отводится 3 часа 55 минут (235 минут).
\par \qquad Ответы к заданиям 1—11 записываются по приведённому ниже \underline {образцу} в виде целого числа или конечной десятичной дроби. Числа запишите в поля ответов в тексте работы, а затем перенесите их в бланк ответов №1.
%%\includegraphics[width=0.98\linewidth]{obrazec}
\par \qquad При выполнении заданий 12—18 требуется записать полное решение и ответ в бланке ответов №2.
\par \qquad  Все бланки ЕГЭ заполняются яркими чёрными чернилами. Допускается использование гелевой или капиллярной ручки.
\par \qquad При выполнении заданий можно пользоваться черновиком. \textbf{Записи в черновике, а также в тексте контрольных измерительных материалов не учитываются при оценивании работы.}
\par \qquad  Баллы, полученные Вами за выполненные задания, суммируются. Постарайтесь выполнить как можно больше заданий и набрать наибольшее количество баллов.
\par \qquad После завершения работы проверьте, что ответ на каждое задание в бланках ответов №1 и №2 записан под правильным номером.
\begin{center}
	\textit{\textbf{Желаем успеха!}}\\ \qquad \\\textbf{ Справочные материалы} \\
$\sin^2 \alpha + \cos^2 \alpha = 1$ \\
$\sin 2\alpha=2\sin \alpha \cdot \cos \alpha$ \\
$\cos 2\alpha=\cos^2 \alpha-\sin^2 \alpha$ \\
$\sin (\alpha+\beta)=\sin \alpha \cdot \cos \beta+\cos \alpha \cdot \sin\beta$ \\
$\cos (\alpha+\beta)=\cos \alpha \cdot \cos \beta-\sin\alpha \cdot \sin\beta$
\end{center}


\startpartone\large

\begin{taskBN}{1}
К окружности, вписанной в треугольник $GSN$, проведены три касательные. Периметры отсеченных треугольников равны 23, 48, 83. Найдите периметр треугольника $GSN$.
\end{taskBN}

\begin{taskBN}{2}
Ребро правильного тетраэдра равно 22. Найдите площадь сечения, проходящего через середины четырёх рёбер правильного тетраэдра.
\end{taskBN}

\begin{taskBN}{3}
Вероятность того, что батарейка бракованная, равна 0,16. Покупатель в магазине выбирает случайную упаковку, в которой две таких батарейки. Найдите вероятность того, что исправной окажется ровно одна батарейка.
\end{taskBN}

\begin{taskBN}{4}
Вероятность того, что на вступительном испытании по физике абитуриент Ш. верно решит больше 7 задач, равна 0.64. Вероятность того, что Ш. верно решит больше 6 задач, равна 0.77. Найдите вероятность того, что Ш. верно решит ровно 7 задач.
\end{taskBN}

\begin{taskBN}{5}
Найдите корень уравнения $$x-\sqrt{-8+6x}=0$$ Если корней несколько, в ответе укажите их сумму.
\end{taskBN}

\begin{taskBN}{6}
Найдите значение выражения $${4}^{\log_{64}{27}}$$
\end{taskBN}

\begin{taskBN}{7}
Прямая $y=-2x+2$ параллельна касательной к графику функции $y=-6x^{2}-2x-5$. Найдите ординату точки касания.
\end{taskBN}

\begin{taskBN}{8}
Некоторая компания продает свою продукцию. Месячная операционная прибыль предприятия (в рублях) вычисляется по формуле $a(q)=q(p-v)-f$. Постоянные расходы предприятия $f=311200~руб.$, а цена за единицу продукции $p=380~руб.$ Чему равна месячный объём производства продукции $q$, выраженная в шт., если месячная операционная прибыль предприятия равна 333.8 тыс. руб.? Известно, что переменные затраты на производство одной единицы продукции $v=230~руб.$ 
\end{taskBN}

\begin{taskBN}{9}
Два мотоциклиста стартуют одновременно в одном направлении из двух диаметрально противоположных точек кольцевой трассы, длина которой равна 281,6 км. Через сколько минут мотоциклисты поравняются в первый раз, если скорость одного из них на 16 км/ч больше скорости другого? 
\end{taskBN}

\begin{taskBN}{10}
\addpictocenter[]{images/2217368472699128n0}На рисунке изображён график функции $f(x)=\frac{k}{x+a}$. Найдите значение $x$, при котором $f(x)=10$.
\end{taskBN}

\begin{taskBN}{11}
Найдите наибольшее значение функции $y = 8\mathrm{tg} x-37-8x$ на отрезке $\left[-\frac{\pi}{4};0 \right]$
\end{taskBN}

\newpage Ответы

\begin{table}\begin{tabular}{ll}
   1 & 154\\2 & 121\\3 & 0,2688\\4 & 0,13\\5 & 6\\6 & 3\\7 & -5\\8 & 4300\\9 & 528\\10 & 3,5\\11 & -37\\
\end{tabular}\end{table}

\cleardoublepage

\def\examvart{Вариант 13.3}

\normalsize


\begin{center}
	\textbf{
		Единый государственный экзамен\\по МАТЕМАТИКЕ\\Профильный уровень\\ \qquad \\ Инструкция по выполнению работы
	}
\end{center}


\par \qquad Экзаменационная работа состоит из двух частей, включающих в себя 18 заданий. Часть 1 содержит 11 заданий с кратким ответом базового и повышенного уровней сложности. Часть 2 содержит 7 заданий с развёрнутым ответом повышенного и высокого уровней сложности.
\par \qquad На выполнение экзаменационной работы по математике отводится 3 часа 55 минут (235 минут).
\par \qquad Ответы к заданиям 1—11 записываются по приведённому ниже \underline {образцу} в виде целого числа или конечной десятичной дроби. Числа запишите в поля ответов в тексте работы, а затем перенесите их в бланк ответов №1.
%%\includegraphics[width=0.98\linewidth]{obrazec}
\par \qquad При выполнении заданий 12—18 требуется записать полное решение и ответ в бланке ответов №2.
\par \qquad  Все бланки ЕГЭ заполняются яркими чёрными чернилами. Допускается использование гелевой или капиллярной ручки.
\par \qquad При выполнении заданий можно пользоваться черновиком. \textbf{Записи в черновике, а также в тексте контрольных измерительных материалов не учитываются при оценивании работы.}
\par \qquad  Баллы, полученные Вами за выполненные задания, суммируются. Постарайтесь выполнить как можно больше заданий и набрать наибольшее количество баллов.
\par \qquad После завершения работы проверьте, что ответ на каждое задание в бланках ответов №1 и №2 записан под правильным номером.
\begin{center}
	\textit{\textbf{Желаем успеха!}}\\ \qquad \\\textbf{ Справочные материалы} \\
$\sin^2 \alpha + \cos^2 \alpha = 1$ \\
$\sin 2\alpha=2\sin \alpha \cdot \cos \alpha$ \\
$\cos 2\alpha=\cos^2 \alpha-\sin^2 \alpha$ \\
$\sin (\alpha+\beta)=\sin \alpha \cdot \cos \beta+\cos \alpha \cdot \sin\beta$ \\
$\cos (\alpha+\beta)=\cos \alpha \cdot \cos \beta-\sin\alpha \cdot \sin\beta$
\end{center}


\startpartone\large

\begin{taskBN}{1}
В треугольнике $MHS$ угол $M$ равен $90^\circ$. Чему равен  $\cos^2{H}$, если $SH=8$, а  $SM=\sqrt{48}$? 
\end{taskBN}

\begin{taskBN}{2}
Площадь поверхности куба составляет 150. Найдите квадрат диагонали куба.
\end{taskBN}

\begin{taskBN}{3}
В случайном эксперименте симметричную монету бросают 2 раза. Какова вероятность того, что орёл выпадет один раз?
\end{taskBN}

\begin{taskBN}{4}
Из деревни в город каждый час ходит электричка. Вероятность того, что в 21:00 в электричке окажется меньше 29 пассажиров, равна 0.88. Вероятность того, что окажется меньше 17 пассажиров, равна 0.57. Найдите вероятность того, что число пассажиров будет от 17 до 28.
\end{taskBN}

\begin{taskBN}{5}
Найдите корень уравнения $$-2x=\sqrt{7+3x}$$ Если корней несколько, в ответе укажите их произведение.
\end{taskBN}

\begin{taskBN}{6}
Найдите значение выражения $$\frac{9^{6,8}\cdot{5^{5,8}}}{45^{4,8}}$$
\end{taskBN}
\begin{taskBN}{7}
    \addpictoright[0.8\linewidth]{images/439501106528127n0}На рисунке изображены график функции $y = f(x)$ и касательная к нему в точке с абсциссой $i$. Найдите значение производной функции $f(x)$ в точке $i$.
\end{taskBN}

\begin{taskBN}{8}
Перед отправкой тепловоз издал гудок с частотой $f_0 = 135$ Гц. Чуть позже издал гудок подъезжающий к платформе тепловоз. Из-за эффекта Доплера частота второго гудка $f$ больше первого: она зависит от скорости тепловоза по закону $f(v)=\frac{f_0}{1-\frac{v}{c}}$ (Гц), где $c$ — скорость звука в воздухе (в м/с). Человек, стоящий на платформе, различает сигналы по тону, если они отличаются не менее чем на 9 Гц. Определите, с какой максимальной скоростью приближался к платформе тепловоз, если человек не смог различить сигналы, а $c = 320$ м/с. Ответ выразите в м/с.
\end{taskBN}

\begin{taskBN}{9}
Поезд, двигаясь равномерно со скоростью 89 км/ч, проезжает мимо придорожного столба за 27 секунд. Найдите длину поезда в метрах.
\end{taskBN}

\begin{taskBN}{10}
\addpictocenter[]{images/055647402697582n0}На рисунке изображён график функции $f(x)=\log{_a}{(x+b)}+6$. Найдите $f(-6,875)$. 
\end{taskBN}

\begin{taskBN}{11}
Вычислите наибольшее значение функции $y = -9-\frac{75}{\pi}x+9\sin x$ на отрезке $\left[0;\frac{23\pi}{6} \right]$
\end{taskBN}

\newpage Ответы

\begin{table}\begin{tabular}{ll}
    1 & 0,25\\2 & 75\\3 & 0,5\\4 & 0,31\\5 & -1\\6 & 405\\7 & -0,5\\8 & 20\\9 & 667,5\\10 & 9\\11 & -9
\end{tabular}\end{table}