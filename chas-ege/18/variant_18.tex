\documentclass[twocolumn]{article}
\usepackage{dashbox}
\setlength{\columnsep}{40pt}
\usepackage[T2A]{fontenc}
\usepackage[utf8]{inputenc}
\usepackage[english,russian]{babel}
\usepackage{graphicx}
\graphicspath{{pictures/}}
\DeclareGraphicsExtensions{.pdf,.png,.jpg}

\linespread{1.15}

\usepackage{../egetask}
\usepackage{../egetask-math-11-2022}

\def\examyear{2023}
\usepackage[colorlinks,linkcolor=blue]{hyperref}\def\rfoottext{Разрешается свободное копирование в некоммерческих целях с указанием источника }
\def\lfoottext{Источник \href{https://vk.com/egemathika}{https://vk.com/egemathika}}

\begin{document}



\cleardoublepage
\def\examvart{Вариант 18.1}
\normalsize

\begin{center}
	\textbf{
		Единый государственный экзамен\\по МАТЕМАТИКЕ\\Профильный уровень\\ \qquad \\ Инструкция по выполнению работы
	}
\end{center}


\par \qquad Экзаменационная работа состоит из двух частей, включающих в себя 18 заданий. Часть 1 содержит 11 заданий с кратким ответом базового и повышенного уровней сложности. Часть 2 содержит 7 заданий с развёрнутым ответом повышенного и высокого уровней сложности.
\par \qquad На выполнение экзаменационной работы по математике отводится 3 часа 55 минут (235 минут).
\par \qquad Ответы к заданиям 1—11 записываются по приведённому ниже \underline {образцу} в виде целого числа или конечной десятичной дроби. Числа запишите в поля ответов в тексте работы, а затем перенесите их в бланк ответов №1.
%%\includegraphics[width=0.98\linewidth]{obrazec}
\par \qquad При выполнении заданий 12—18 требуется записать полное решение и ответ в бланке ответов №2.
\par \qquad  Все бланки ЕГЭ заполняются яркими чёрными чернилами. Допускается использование гелевой или капиллярной ручки.
\par \qquad При выполнении заданий можно пользоваться черновиком. \textbf{Записи в черновике, а также в тексте контрольных измерительных материалов не учитываются при оценивании работы.}
\par \qquad  Баллы, полученные Вами за выполненные задания, суммируются. Постарайтесь выполнить как можно больше заданий и набрать наибольшее количество баллов.
\par \qquad После завершения работы проверьте, что ответ на каждое задание в бланках ответов №1 и №2 записан под правильным номером.
\begin{center}
	\textit{\textbf{Желаем успеха!}}\\ \qquad \\\textbf{ Справочные материалы} \\
$\sin^2 \alpha + \cos^2 \alpha = 1$ \\
$\sin 2\alpha=2\sin \alpha \cdot \cos \alpha$ \\
$\cos 2\alpha=\cos^2 \alpha-\sin^2 \alpha$ \\
$\sin (\alpha+\beta)=\sin \alpha \cdot \cos \beta+\cos \alpha \cdot \sin\beta$ \\
$\cos (\alpha+\beta)=\cos \alpha \cdot \cos \beta-\sin\alpha \cdot \sin\beta$
\end{center}

\startpartone
\large




\begin{taskBN}{1}
Основания равнобедренной трапеции равны 36 и 66. Боковые стороны равны 17. Найдите тангенс острого угла трапеции. Ответ округлите до сотых.
\end{taskBN}

\begin{taskBN}{2}
Площадь сечения, проходящего через середины четырёх рёбер правильного тетраэдра, равна 1225. Найдите ребро правильного тетраэдра.
\end{taskBN}

\begin{taskBN}{3}
В среднем из 500 чайников, поступивших в продажу, 490 не имеют дефектов. Найдите вероятность того, что один случайным образом выбранный экземпляр товара имеет дефекты.
\end{taskBN}

\begin{taskBN}{4}
Вероятность того, что в случайный момент времени температура тела здорового человека окажется выше чем 36.7 °С, равна 0.72. Найдите вероятность того, что в случайный момент времени у здорового человека температура окажется 36.7 °С или ниже.
\end{taskBN}

\begin{taskBN}{5}
Найдите корень уравнения $${1}\frac{7}{8}=-\frac{1}{8}x$$
\end{taskBN}

\begin{taskBN}{6}
Найдите значение выражения $$\frac{-84\cos362^\circ}{-50(1-2\cos^2{181^\circ})}$$
\end{taskBN}

\begin{taskBN}{7}
\addpictoright[0.4\linewidth]{images/3420125922923112n0}На рисунке изображен график производной функции $f(x)$, определенной на интервале $(-8; 1)$. Найдите количество точек, в которых касательная к графику функции $f(x)$ параллельна прямой $y=7x+ 14{,}5 $ или совпадает с ней.\vspace{2.5cm}
\end{taskBN}

\begin{taskBN}{8}
Ёмкость высоковольтного конденсатора в телевизоре $C=6$ мкФ. Параллельно с конденсатором подключён резистор с сопротивлением $R=8$ МОм. Во время работы телевизора напряжение на конденсаторе $U_{0}=24$ кВ. После выключения телевизора напряжение на конденсаторе убывает до значения $U$ кВ за время, определяемое выражением $t=\alpha RC \log _{2} {\frac{U_{0}}{U}}$, где $ \alpha=0.7$ - постоянная. С момента выключения телевизора прошло 67.2 с. Определите напряжение на конденсаторе в кВ.
\end{taskBN}

\begin{taskBN}{9}
Два водителя стартуют одновременно в одном направлении из двух диаметрально противоположных точек кольцевой трассы, длина которой равна 115,9 км. Через сколько минут водители поравняются в первый раз, если скорость одного из них на 19 км/ч больше скорости другого? 
\end{taskBN}

\begin{taskBN}{10}
\addpictoright[0.4\linewidth]{images/66369859233442n0}На рисунке изображён график функции $f(x)=\frac{k}{x}+b$. Найдите значение $x$, при котором $f(x)=-1,75$.\vspace{2.5cm}
\end{taskBN}

\begin{taskBN}{11}
Определите точку максимума функции $y = 99x-39+\frac{7}{3}x^{3}+21x^{2}$ на отрезке $\left[3;9 \right]$
\end{taskBN}

\newpage
 Ответы

\begin{table}[h]\begin{tabular}{|l|l|}
\hline
1 & 0,53
\\
\hline
2 & 70
\\
\hline
3 & 0,02
\\
\hline
4 & 0,28
\\
\hline
5 & -15
\\
\hline
6 & -1,68
\\
\hline
7 & 0
\\
\hline
8 & 6
\\
\hline
9 & 183
\\
\hline
10 & -12
\\
\hline
11 & 9
\\
\hline
\end{tabular}\end{table}



\newpage




\cleardoublepage
\def\examvart{Вариант 18.2}
\normalsize

\begin{center}
	\textbf{
		Единый государственный экзамен\\по МАТЕМАТИКЕ\\Профильный уровень\\ \qquad \\ Инструкция по выполнению работы
	}
\end{center}


\par \qquad Экзаменационная работа состоит из двух частей, включающих в себя 18 заданий. Часть 1 содержит 11 заданий с кратким ответом базового и повышенного уровней сложности. Часть 2 содержит 7 заданий с развёрнутым ответом повышенного и высокого уровней сложности.
\par \qquad На выполнение экзаменационной работы по математике отводится 3 часа 55 минут (235 минут).
\par \qquad Ответы к заданиям 1—11 записываются по приведённому ниже \underline {образцу} в виде целого числа или конечной десятичной дроби. Числа запишите в поля ответов в тексте работы, а затем перенесите их в бланк ответов №1.
%%\includegraphics[width=0.98\linewidth]{obrazec}
\par \qquad При выполнении заданий 12—18 требуется записать полное решение и ответ в бланке ответов №2.
\par \qquad  Все бланки ЕГЭ заполняются яркими чёрными чернилами. Допускается использование гелевой или капиллярной ручки.
\par \qquad При выполнении заданий можно пользоваться черновиком. \textbf{Записи в черновике, а также в тексте контрольных измерительных материалов не учитываются при оценивании работы.}
\par \qquad  Баллы, полученные Вами за выполненные задания, суммируются. Постарайтесь выполнить как можно больше заданий и набрать наибольшее количество баллов.
\par \qquad После завершения работы проверьте, что ответ на каждое задание в бланках ответов №1 и №2 записан под правильным номером.
\begin{center}
	\textit{\textbf{Желаем успеха!}}\\ \qquad \\\textbf{ Справочные материалы} \\
$\sin^2 \alpha + \cos^2 \alpha = 1$ \\
$\sin 2\alpha=2\sin \alpha \cdot \cos \alpha$ \\
$\cos 2\alpha=\cos^2 \alpha-\sin^2 \alpha$ \\
$\sin (\alpha+\beta)=\sin \alpha \cdot \cos \beta+\cos \alpha \cdot \sin\beta$ \\
$\cos (\alpha+\beta)=\cos \alpha \cdot \cos \beta-\sin\alpha \cdot \sin\beta$
\end{center}

\startpartone
\large




\begin{taskBN}{1}
В треугольнике $CUN$ угол $C$ равен $90^\circ$. Сколько составляет  $UN$, если $\sin{N}=\sqrt{0,51}$, а  $NC=3,5$? 
\end{taskBN}

\begin{taskBN}{2}
Основанием призмы является  прямоугольный треугольник. Найдите второй катет, если первый катет составляет 7, при этом объём призмы равен 168, а высота призмы равна 8. 
\end{taskBN}

\begin{taskBN}{3}
Научная конференция проводится в 17 дней. Всего запланировано 900 выступлений — последние 2 дня по 135 выступлений, остальные распределены поровну между оставшимися днями. Порядок выступлений определяется жеребьёвкой. Какова вероятность, что выступление товарища Н. окажется запланированным на последний день мероприятия?
\end{taskBN}

\begin{taskBN}{4}
Из деревни в город каждый час ходит маршрутка. Вероятность того, что в 9:00 в маршрутке окажется меньше 18 пассажиров, равна 0.97. Вероятность того, что окажется меньше 15 пассажиров, равна 0.63. Найдите вероятность того, что число пассажиров будет от 15 до 17.
\end{taskBN}

\begin{taskBN}{5}
Найдите корень уравнения $${5}\frac{1}{4}=-\frac{3}{8}x$$
\end{taskBN}

\begin{taskBN}{6}
Вычислить значение выражения: $$ {-12}\cos2\alpha\mbox{, если }\cos\alpha = {0,1}$$
\end{taskBN}

\begin{taskBN}{7}
Материальная точка движется прямолинейно по закону $x(t)=\frac{1}{2}t^{2}-3t+3$, где $x$ — расстояние от точки отсчета в метрах, $t$ — время в секундах, измеренное с начала движения. В какой момент времени (в секундах) ее скорость была равна $3$ м/с?
\end{taskBN}

\begin{taskBN}{8}
На верфи инженеры проектируют новый аппарат для погружения на небольшие глубины. Конструкция имеет форму сферы, а значит, действующая на аппарат выталкивающая (архимедова) сила, выражаемая в ньютонах, будет определяться по формуле:  $F_{\rm{A}}  = \alpha \rho gr^3$, где $\alpha  = 4,2$ — постоянная, $r$ — радиус аппарата в метрах, $\rho  = 1000~\mbox{кг}/\mbox{м}^3$ — плотность воды, а $g$ — ускорение свободного падения (считайте $g = 10$ Н/кг). Каков может быть минимальный радиус аппарата, чтобы выталкивающая сила при погружении была не меньше, чем $656250$ Н? Ответ выразите в метрах.
\end{taskBN}

\begin{taskBN}{9}
Два байкера стартуют одновременно в одном направлении из двух диаметрально противоположных точек круговой трассы, длина которой равна 198,9 км. Через сколько минут байкеры поравняются в первый раз, если скорость одного из них на 13 км/ч больше скорости другого? 
\end{taskBN}

\begin{taskBN}{10}
\addpictoright[0.4\linewidth]{images/389062221021743n0}На рисунке изображён график функции $f(x)=\frac{k}{x}+b$. Найдите $f(-20)$.\vspace{2.5cm}
\end{taskBN}

\begin{taskBN}{11}
Найдите наименьшее значение функции $y = 9+52\cos x-69x$ на отрезке $[-\frac{8\pi}{5};0]$
\end{taskBN}

\newpage
 Ответы

\begin{table}[h]\begin{tabular}{|l|l|}
\hline
1 & 5
\\
\hline
2 & 6
\\
\hline
3 & 0,15
\\
\hline
4 & 0,34
\\
\hline
5 & -14
\\
\hline
6 & 11,76
\\
\hline
7 & 6
\\
\hline
8 & 2.5
\\
\hline
9 & 459
\\
\hline
10 & -0,95
\\
\hline
11 & 61
\\
\hline
\end{tabular}\end{table}



\newpage




\cleardoublepage
\def\examvart{Вариант 18.3}
\normalsize

\begin{center}
	\textbf{
		Единый государственный экзамен\\по МАТЕМАТИКЕ\\Профильный уровень\\ \qquad \\ Инструкция по выполнению работы
	}
\end{center}


\par \qquad Экзаменационная работа состоит из двух частей, включающих в себя 18 заданий. Часть 1 содержит 11 заданий с кратким ответом базового и повышенного уровней сложности. Часть 2 содержит 7 заданий с развёрнутым ответом повышенного и высокого уровней сложности.
\par \qquad На выполнение экзаменационной работы по математике отводится 3 часа 55 минут (235 минут).
\par \qquad Ответы к заданиям 1—11 записываются по приведённому ниже \underline {образцу} в виде целого числа или конечной десятичной дроби. Числа запишите в поля ответов в тексте работы, а затем перенесите их в бланк ответов №1.
%%\includegraphics[width=0.98\linewidth]{obrazec}
\par \qquad При выполнении заданий 12—18 требуется записать полное решение и ответ в бланке ответов №2.
\par \qquad  Все бланки ЕГЭ заполняются яркими чёрными чернилами. Допускается использование гелевой или капиллярной ручки.
\par \qquad При выполнении заданий можно пользоваться черновиком. \textbf{Записи в черновике, а также в тексте контрольных измерительных материалов не учитываются при оценивании работы.}
\par \qquad  Баллы, полученные Вами за выполненные задания, суммируются. Постарайтесь выполнить как можно больше заданий и набрать наибольшее количество баллов.
\par \qquad После завершения работы проверьте, что ответ на каждое задание в бланках ответов №1 и №2 записан под правильным номером.
\begin{center}
	\textit{\textbf{Желаем успеха!}}\\ \qquad \\\textbf{ Справочные материалы} \\
$\sin^2 \alpha + \cos^2 \alpha = 1$ \\
$\sin 2\alpha=2\sin \alpha \cdot \cos \alpha$ \\
$\cos 2\alpha=\cos^2 \alpha-\sin^2 \alpha$ \\
$\sin (\alpha+\beta)=\sin \alpha \cdot \cos \beta+\cos \alpha \cdot \sin\beta$ \\
$\cos (\alpha+\beta)=\cos \alpha \cdot \cos \beta-\sin\alpha \cdot \sin\beta$
\end{center}

\startpartone
\large




\begin{taskBN}{1}
В треугольнике $CUG$ угол $G$ равен $90^\circ$.Чему равна  $UG$, если $\ctg^2{U}=12,25$?  $CG=2$. 
\end{taskBN}

\begin{taskBN}{2}
\addpictoright[0.4\linewidth]{images/026949902908446n0}В правильном тетраэдре площадь полной поверхности равна $36\sqrt{3}$. Чему равна площадь боковой поверхности тетраэдра? Ответ умножьте на $\sqrt{3}$.\vspace{2.5cm}
\end{taskBN}

\begin{taskBN}{3}
В чемпионате по лёгкой атлетике участвуют 40 спортсменок: 9 из Словении, 4 из ЮАР, остальные — из Венесуэлы. Порядок, в котором выступают спортсменки, определяется жребием. Найдите вероятность того, что спортсменка, выступающая последней, окажется из ЮАР.
\end{taskBN}

\begin{taskBN}{4}
Если гроссмейстер В. играет белыми, то он выигрывает у гроссмейстера Ё. с вероятностью 0,83. Если В. играет черными, то В. выигрывает у Ё. с вероятностью 0,18. Гроссмейстеры В. и Ё. играют две партии, причем во второй партии меняют цвет фигур. Найдите вероятность того, что Ё. выиграет оба раза.
\end{taskBN}

\begin{taskBN}{5}
Найдите корень уравнения $${5}^{-5x+38}=\frac{1}{25}$$
\end{taskBN}

\begin{taskBN}{6}
Найдите значение выражения $$\frac{\log_{5}{1296}}{\log_{5}{6}} $$
\end{taskBN}

\begin{taskBN}{7}
Материальная точка движется прямолинейно по закону $x(t)=3t^{2}-18t+17$, где $x$ — расстояние от точки отсчета в метрах, $t$ — время в секундах, измеренное с начала движения. В какой момент времени (в секундах) ее скорость была равна $18$ м/с?
\end{taskBN}

\begin{taskBN}{8}
По закону Ома для полной цепи сила тока, измеряемая в амперах, равна $I=\frac{\varepsilon}{R+r}$, где $\varepsilon$ — ЭДС источника (в вольтах), $r= 4{,}5 $ Ом — его внутреннее сопротивление, $R$ — сопротивление цепи (в омах). При каком наименьшем сопротивлении цепи сила тока будет составлять не более $36\%$ от силы тока короткого замыкания $I_{\mbox{кз}}=\frac{\varepsilon}{r}$? (Ответ выразите в омах.)
\end{taskBN}

\begin{taskBN}{9}
Между остановками Ц. и Н. 1,5 километров прямой трассы. Олеся выехала на велосипеде из леса между станциями в 375 метрах от Ц. и увидела, что к Ц. в направлении к Н. с постоянной скоростью подъезжает автобус, на который Олесе нужно успеть. Она заметила, что если она сейчас поедет в сторону Ц., она окажется там одновременно с автобусом. Но и если она поедет в сторону Н., она также окажется там одновременно с автобусом, который успеет преодолеть весь участок от Ц. до Н., не останавливаясь на остановке Ц. Во сколько раз скорость велосипеда меньше скорости автобуса? (Считайте, что автобус и велосипед движутся с постоянными скоростями и останавливаются мгновенно.)
\end{taskBN}

\begin{taskBN}{10}
\addpictoright[0.4\linewidth]{images/593422273170903n0}На рисунке изображёны графики двух линейных функций. Найдите абсциссу точки пересечения графиков.\vspace{2.5cm}
\end{taskBN}

\begin{taskBN}{11}
Вычислите наименьшее значение функции $y =95-(12-x)e^{x-11}$ на отрезке $\left[-17;38 \right]$
\end{taskBN}

\newpage
 Ответы

\begin{table}[h]\begin{tabular}{|l|l|}
\hline
1 & 7
\\
\hline
2 & 81
\\
\hline
3 & 0,1
\\
\hline
4 & 0,1394
\\
\hline
5 & 8
\\
\hline
6 & 4
\\
\hline
7 & 6
\\
\hline
8 & 8
\\
\hline
9 & 2
\\
\hline
10 & 8
\\
\hline
11 & 94
\\
\hline
\end{tabular}\end{table}



\newpage
\end{document}