\subsection{Внешнее устойчивое множество}
Граф $G = (X, \Gamma)$, множество $T \subseteq X$ внешне устойчиво, если для каждой вершины $x \notin T$ имеем $\Gamma_x \cap T \neq \varnothing$ (каждая вершина вне $T$ соединена с $T$). Если $\mathcal{T}$ — все внешне устойчивые множества, то $X \in \mathcal{T}$ и $T \in \mathcal{T} \quad A \supseteq T \Rightarrow A \in \mathcal{T}$.

\textbf{Число внешней устойчивости}
\[\beta(G) = \min_{T \in \mathcal{T}} |T|\] (минимальное внешне устойчивое множество).

\textbf{Алгоритм нахождения наименьшего внешне устойчивого множества}
\\1. Удаляем вершину $x$, если $\Delta x \subseteq \Delta y$ для $y \neq x$ (вершина $y$ заменяет $x$). Пример: удаляем $c, d, f$.
\\2. Если есть висячее ребро $(x, y)$, то $x \in T$. Пример: $a \in T$.
\\3. Исключаем $a$ и $\Delta a = \{\overline{a}, \overline{b}, \overline{c}\}$.
\\4. Повторяем шаги 1 и 2. Если граф неприводим, временно добавляем в $T$ вершину, например $b$.
\\5. Исключаем $b$ и $\Delta b = \{a, e, f\}$.
\\6. Упрощаем граф: исключаем $g$, так как $\Delta g \subseteq \Delta e = \{g\}$. Включаем $e$ в $T$, получаем $T = \{a, b, e\}$.