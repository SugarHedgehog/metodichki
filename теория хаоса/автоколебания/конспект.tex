\documentclass[a4paper,12pt]{article}
\usepackage[utf8]{inputenc}
\usepackage[russian]{babel}
\usepackage{amsmath}
\usepackage{amssymb}
\usepackage{graphicx}
\usepackage{hyperref}

\title{Автоколебательный режим и связанные с ним системы}
\author{}
\date{}

\begin{document}

\maketitle

\section{Основные определения и понятия}

Наряду с колебательными системами, в которых энергия с течением времени может только уменьшаться из-за диссипации, существуют и такие, в которых возможно пополнение энергии колебаний за счет неустойчивостей. Это может иметь место, когда система в состоянии обмениваться с окружающей средой энергией или веществом, т.е. является энергетически неизолированной (открытой). В открытых системах возникает множество принципиально новых явлений, в первую очередь — генерация автоколебаний. Термин «автоколебания» ввел А.А. Андронов в 1928 г. Он же заложил основы теории автоколебаний, впервые связав их с предельными циклами Пуанкаре.

Современное определение автоколебаний можно сформулировать следующим образом. Автоколебания — это незатухающие колебания в нелинейной диссипативной системе, вид и свойства которых определяются самой системой и не зависят от начальных условий (по крайней мере, в конечных пределах). Ключевым в этом определении является требование независимости от начальных условий. С течением времени фазовая траектория стремится к некоторому притягивающему множеству, называемому аттрактором. После переходного процесса в системе устанавливаются колебания, которым отвечает движение изображающей точки по аттрактору. Такие колебания, очевидно, будут зависеть только от параметров системы, а не от начальных условий.

\subsection{Аттракторы и предельные циклы}

Аттракторами, соответствующими периодическим автоколебаниям, являются устойчивые \textit{предельные циклы}. Под предельным циклом понимается замкнутая изолированная фазовая траектория. Термин «изолированная» означает, что в её достаточно малой (кольцеобразной) окрестности не существует других замкнутых фазовых траекторий. Это отличает предельные циклы от замкнутых фазовых траекторий, соответствующих периодическим колебаниям консервативного нелинейного осциллятора. Предельный цикл является устойчивым, если все соседние траектории приближаются к нему при \( t \to \infty \), и неустойчивым, если соседние траектории удаляются от него при \( t \to \infty \).

\subsection{Типы автоколебаний}

Автоколебания не обязательно должны быть периодическими. Различают также \textit{квазипериодические}, т.е. содержащие несколько независимых спектральных компонент, находящихся в иррациональном соотношении, а также \textit{хаотические} автоколебания, которые являются случайными, хотя совершаются под действием нес-лучайных источников энергии. Спектр хаотических автоколебаний сплошной. Математическим образом квазипериодических автоколебаний в фазовом пространстве является \( n \)-мерный тор, а стохастических — \textit{странный аттрактор}, т.е. притягивающее множество, имеющее чрезвычайно сложную внутреннюю структуру, на котором все (или почти все) траектории неустойчивы.

\section{Примеры автоколебательных систем}

Класс автоколебательных систем очень широк: механические часы, радиотехнические, электронные и квантовые генераторы электромагнитных колебаний, духовые и смычковые музыкальные инструменты и др. Автоколебательный характер носят некоторые химические реакции, процессы в биологических популяциях и живых организмах.

\section{Задача 11.1}

В сосуд с поперечным сечением \( S_1 \) из крана с сечением \( S_2 \) поступает со скоростью \( V \) вода. Вода может выливаться через узкую сифонную трубку с поперечным сечением \( S_3 \). Высота левого колена трубки равна \( h \), а правого — \( H \). Постройте график зависимости уровня воды в сосуде от времени и обоснуйте автоколебательный характер поведения системы. Найдите период установившихся автоколебаний. Считайте, что скорость вытекания воды через трубку определяется формулой Торричелли.

\section{Элементы автоколебательных систем}

В простейших автоколебательных системах можно, как правило, выделить следующие основные элементы:
\begin{itemize}
    \item Колебательная система с затуханием;
    \item Усилитель, содержащий источник энергии и преобразователь энергии источника в энергию колебаний;
    \item Нелинейный ограничитель;
    \item Звено обратной связи.
\end{itemize}

Рассмотрим эти элементы на классическом примере радиотехнического генератора, обобщенная схема которого приведена на рис. 11.2.

\end{document}