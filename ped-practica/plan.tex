\documentclass[a4paper,12pt]{article}
\usepackage[utf8]{inputenc}
\usepackage[russian]{babel}
\usepackage{geometry}
\geometry{top=10mm, bottom=10mm, left=10mm, right=10mm}
\pagestyle{empty}

\begin{document}
Урок "Углы: острые, прямые, тупые; измерение углов с помощью транспортира" (40 минут)

Цель урока:
\begin{itemize}
\item Познакомить учеников с видами углов (острый, прямой, тупой)
\item Научить измерять углы с помощью транспортира
\end{itemize}

Материалы: транспортиры, листы бумаги, линейка, карандаш.

Время (40 минут):
\begin{enumerate}
\item Вводное слово (2 минуты)
	\begin{itemize}
	\item Урок начинается с короткого вводного слова. Сегодня ученики узнают о видах углов и научатся измерять углы с помощью транспортира.
	\end{itemize}

\item Презентация материала (12 минут)
	\begin{itemize}
	\item Острый угол - угол меньше 90 градусов, но больше 0 градусов.
	\item Прямой угол - угол равен 90 градусов
	\item Тупой угол - угол больше прямого, но меньше 180 градусов.
	\item Демонстрация примеров каждого вида угла на доске или листе бумаги.
	\end{itemize}

\item Пример определения угла с помощью ладони (8 минут)
	\begin{itemize}
	\item Предложить ученикам положить ладонь на стол или на доску так, чтобы пальцы были направлены вверх.
	\item Попросить учеников указать на угол между большим и средним пальцами.
	\item Объясните, что угол между большим и средним пальцами равен примерно 45°.
	\item Просьба учеников указать на угол между средним и указательным пальцами.
	\item Объяснить, что угол между средним и указательным пальцами равен примерно 15°.
	\item Демонстрация, как можно использовать эти углы для определения общего угла между большим и указательным пальцами.
	\end{itemize}

\item Практическая работа (12 минут)
	\begin{itemize}
	\item Распределение учеников на пары и дать каждой паре лист бумаги, линейку, карандаш и транспортир.
	\item Просьба учеников нарисовать различные углы на листе бумаги.
	\item Затем просьба их измерить углы с помощью транспортира и записать результаты.
	\item Ученики могут меняться ролями, чтобы каждый измерял углы.
	\end{itemize}

\item Обсуждение и заключение (6 минут)
	\begin{itemize}
	\item Опрос учеников, какие углы они нарисовали и измерили.
	\item Обсуждение результатов и ответы на вопросы учеников.
	\item Заключение урока, подводя итоги и поблагодарив учеников за их участие.
	\end{itemize}

\item Домашнее задание (5 минут)
	\begin{itemize}
	\item Домашнее задание: нарисовать и измерить еще несколько углов дома и принести результаты на следующий урок.
	\end{itemize}
\end{enumerate}
\end{document}