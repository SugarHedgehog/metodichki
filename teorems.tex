\documentclass[a4paper 12pt]{article}
\usepackage[left=0cm,right=0cm,
top=0cm,bottom=0cm,bindingoffset=0cm]{geometry}
%%% Работа с русским языком
\usepackage{cmap}					% поиск в PDF
\usepackage{mathtext} 				% русские буквы в фомулах
\usepackage[T2A]{fontenc}			% кодировка
\usepackage[utf8]{inputenc}			% кодировка исходного текста
\usepackage[english,russian]{babel}	% локализация и переносы

%%% Дополнительная работа с математикой
\usepackage{amsfonts,amssymb,amsthm,mathtools} % AMS
\usepackage{amsmath}
\usepackage{icomma} % "Умная" запятая: $0,2$ --- число, $0, 2$ --- перечисление

%% Номера формул
%\mathtoolsset{showonlyrefs=true} % Показывать номера только у тех формул, на которые есть \eqref{} в тексте.

%% Шрифты
\usepackage{euscript}	 % Шрифт Евклид
\usepackage{mathrsfs} % Красивый матшрифт

%% Свои команды
\DeclareMathOperator{\sgn}{\mathop{sgn}}


%% Перенос знаков в формулах (по Львовскому)
\newcommand*{\hm}[1]{#1\nobreak\discretionary{}
	{\hbox{$\mathsurround=0pt #1$}}{}}

%%% Работа с картинками
\usepackage{graphicx}  % Для вставки рисунков
\graphicspath{{Изображения/}{image}}  % папки с картинками
\setlength\fboxsep{3pt} % Отступ рамки \fbox{} от рисунка
\setlength\fboxrule{1pt} % Толщина линий рамки \fbox{}
\usepackage{wrapfig} % Обтекание рисунков и таблиц текстом

%%% Работа с таблицами
\usepackage{array,tabularx,tabulary,booktabs} % Дополнительная работа с таблицами
\usepackage{longtable}  % Длинные таблицы
\usepackage{multirow} % Слияние строк в таблице
\newcommand{\br}{\geqslant}
\newcommand{\mr}{\leqslant}
\newcommand{\Fi}{\varphi}
\newcommand{\mes}{\{X, \rho\}}
\newcommand{\ria}{\rightarrow}
\newcommand{\ovrl}{\overline}
\newcommand{\MS}{Пусть $\{X, \rho\}$~--- метрическое пространство}
\newcommand{\tttk}{тогда и только тогда, когда }
\newcommand{\Rnp}{\mathbb{R}^n_p}
\newcommand{\crk}{^{\circ}}
\linespread{1.15}
\begin{document}
\pagestyle{empty}
Без доков:

Примеры метрических пространств

{\setlength{\extrarowheight}{5pt}
\begin{tabularx}{\textwidth}{||l|l|X||}

    \hline
    Название              & Метрика                                                                   & Какое множество или пространство                                                                                        \\
    \hline

    Дискретная            & $
        \rho (x, y) =
    \begin{cases}
            1 & x=y      \\
            0 & x \neq y \\
        \end{cases}$       & $X$ - произвольное непустое множество

    \\
    \hline

    $\Rnp$      & $\rho_p (x, y)=(\sum_{k = 1}^{n}|x_k-y_k|^p)^\frac{1}{p}$                 & \multirow{2}{6cm}{$\mathbb{R}^n$ - множество n-мерных векторов $x=(x_1, x_2, \dots, x_n)$  }                            \\
    \cline{1-2}
    $\mathbb{R}^n_\infty$ & $\rho_\infty (x, y)=\underset{1\leq k\leq n}{\max}|x_k-y_k|$              &                                                                                                                         \\
    \hline
    $C[a, b]$             & $\rho (x, y)=\underset{a \leq t \leq b}{\max}|x(t)-y(t)|$                 & \multirow{2}{6cm}{Пространство числовых функций, непрерывных на $[a, b]$}                                               \\
    \cline{1-2}
    $C_1[a, b]$           & $\rho (x, y)=\int_{b}^{a}|x(t)-y(t)|dx $                                  &                                                                                                                         \\[5pt]
    \hline
    $M[a, b]$             & $\rho (x, y)=\underset{a \leq t \leq b}{\sup}|x(t)-y(t)|$                 & Пространство числовых функций, определённых и ограниченных на $[a, b]$                                                  \\
    \hline
    $l_p$                 & $\rho_p (x, y)=(\sum_{k = 1}^{\infty}|x_k-y_k|^p)^\frac{1}{p}$            & Пространство числовых последовательностей $x=~(x_1, x_2, \dots, x_k, \dots)$, суммируемых с $p$-той степенью            \\
    \hline
    $m$                   & $\rho (x, y)=\underset{k \in \mathbb{N} }{\sup}|x_k-y_k|$                 & Пространство произвольных числовых последовательностей $x=~(x_1, x_2, \dots, x_k, \dots)$, таких что $\sup|x_k|<\infty$ \\
    \hline
    $s$                   & $\rho_p (x, y)=\sum_{k = 1}^{\infty} \cfrac{|x_k-y_k|}{2^k(1+|x_k-y_k|)}$ & Пространство произвольных числовых последовательностей $x=~(x_1, x_2, \dots, x_k, \dots)$                               \\
    \hline
\end{tabularx}}
\hfill \break
\hspace*{8mm}\textbf{Лемма 1}

Пусть на $[a, b]$ функция $\Fi (t) \br 0$, непрерывна на $\int_{b}^{a}\Fi
    (t)\, dx=0$, тогда $\Fi\equiv 0$ на $[a, b]$.

\textbf{Теорема 1 (свойства замыкания) }

\MS. $\forall(M, M_1, M_2\subset X)[(M\subset \ovrl{M})
        \wedge \newline (\ovrl{\ovrl{M}}\subset \ovrl{M})
        \wedge (M_1 \subset M_2 \ria \ovrl{M_1} \subset \ovrl{M_2})
        \wedge(\ovrl{M_1\cup M_2}=\ovrl M_1 \cup \ovrl M_2)]$

\textbf{Теорема 2 (критерий точки прикосновения)}

\MS и $M \subset X$. Точка $x \in \ovrl M $ \tttk
$\exists( \{ x_n \} \subset M)[\underset{n \ria \infty }{x_n \ria x}]$

\textbf{Примеры замкнутых множеств}

\begin{tabular}{c c c}
    1. $\forall B[x_0, r]\subset \mes $& 
    2. $\varnothing \subset \mes $&
    3. $X \subset \mes$
\end{tabular}

\textbf{Теорема 3}

В метрическом пространстве объединение конечного числа и пересечение любого числа 
замкнутых множеств ~--- замкнутое множество.

\textbf{Свойства сходящихся последовательностей  }
\begin{enumerate}
    \item Пусть в метрическом пространстве $\{x_n\}$ сходится. $\forall(x_{n_k} \subset 
    {x_n})$ сходится к тому же пределу, что и $\{x_n\}$
    \item Последовательность в метрическом пространстве может иметь только один 
    предел
    \item Любая сходящаяся последов в метрическом пространстве ограничена.
    \item Пусть в метрическом пространстве даны две последовательности 
    
    $\underset{n \ria \infty }{x_n \ria x \text{ и } y_n \ria y} $. Тогда 
    $\underset{n \ria \infty }{\rho(x_n, y_n) \ria \rho(x, y)}$
\end{enumerate}

\textbf{Сходимость в пространстве $\Rnp$   }

Пусть $1 \mr p < \infty$, ${x^m}\subset \Rnp $ и $\underset{m \ria \infty }{x^m \ria x}$

$\rho(x^m, x)= (\sum_{k = 1}^{n}|x^m_k-x_k|^p)^{1/p} \ria 0$ (покоординатная сходимость)

\textbf{Сходимость в пространстве $C[a, b]$}

Пусть $\{ x_n \}\subset C[a, b]$ сходится по метрике к функции $x \in C[a, b]$. 

$\underset{n \ria \infty }{\rho(x_n, x)}=\underset{a \mr t \mr b}{max}|x_n(t)-x(t)| \ria 0$

\hspace*{30mm}$\Updownarrow$

$\forall(\varepsilon>0)\exists(N\in \mathbb{N})\forall(n\br N)\forall(t\in[a, b])
[|x_n(t)-x(t)|<\varepsilon]$(равномерная сходимость)

\textbf{Теорема 4 (свойства внутренности)}

\MS. 

$\forall(M, M_1, M_2\subset X)[(M\crk \subset M)
\wedge((M\crk)\crk=M)
\wedge(M_1 \subset M_2 \ria M_1\crk \subset M_2\crk) 
\wedge \\((M_1 \cap M_2)\crk=M_1\crk \cap  M_2\crk)]$

\textbf{Теорема 5 (связь внутренности и замыкания)}

\MS и $M \subset X$. Тогда:

$X\backslash M\crk=\ovrl{X\backslash M}$ $X 
\backslash \ovrl M = (X \backslash M)\crk$

\textbf{Примеры открытых множеств  }

\begin{tabular}{c c c}
    1. $\forall B(x_0, r)\subset \mes $& 
    2. $\varnothing \subset \mes $&
    3. $X \subset \mes$
\end{tabular}

\textbf{Теорема 6 (о дополнении)  }

В метрическом пространстве дополнение открытого множество замкнуто, 
а дополнение замкнутого множества открыто

\textbf{Теорема 7  }

В метрическом пространстве объединение любого числа и пересечения конечного числа 
открытых множеств ~--- открытое множество. 

\textbf{Теорема 10 (о полноте подпространства) }

\MS и $M \subset X$. Для того чтобы $ \{M, \rho\}$ было полным подпространством 
$\mes$, необходимо и достаточно, а в случае полноты $\mes$ и достаточно, чтобы 
множество $M$ было замкнуто в $\mes$.

\textbf{Теорема 11 (о вложенных шарах) }

\MS и в нём задана последовательность $ \{ B[a_n, r_n] \} $ замкнутых шаров таких, 
что $\underset{n \ria \infty}{r_n \ria 0}$ и $\forall(n \in \mathbb{N})[B[a_{n+1}, 
r_{n+1}]\subset B[a_n, r_n]]$

Тогда в $X$ существует единственная точка, принадлежащая всем этим шарам.

\textbf{Теорема 14 (об эквивалентности определений)   }

Непрерывность по Гейне в точке $x_0 \in M$:

$\forall( \{ x_n \}\subset M)[(\underset{n \ria \infty}{\lim} x_n = x_0 )\ria 
(\underset{n \ria \infty}{\lim}f(x_n)=f(x_0))]$

\hspace*{30mm}$\Updownarrow$

Непрерывность по Коши в точке $x_0 \in M$:

$\forall(\varepsilon>0)\exists(\delta>0)\forall(x \in M)[\rho_X(x, x_0)<\delta\ria 
\rho(f(x), f(x_0))<\varepsilon]$

\textbf{Следствие из теоремы 14}

Для непрерывности функции $f:M \ria Y$ в точке $x_0 \in M$ достаточно, чтобы для 
$\forall(x_n \ria x \in X)$ последовательность ${f(x_n)}$ сходилась в $Y$.

\textbf{Теорема 15 (о прообразах открытых множеств) }

Пусть $\{X, \rho_X \}$, $\{Y \rho_Y \}$ - метрические пространства. Функция $f: 
X \ria Y$ непрерывна на $X$ \tttk для любого открытого множества $A \subset Y$ 
прообраз $f^{-1}[A]$~--- открытое множество в $X$.

\textbf{Теорема 16 (о прообразах замкнутых множеств)}

Пусть $\{X, \rho_X \}$, $\{Y \rho_Y \}$ - метрические пространства. Функция $f: 
X \ria Y$ непрерывна на $X$ \tttk для любого замкнутого множества $A \subset Y$ 
прообраз $f^{-1}[A]$~--- замкнутое множество в $X$.

\textbf{Теорема 17 (о неподвижной точке) }

Пусть \mes~--- полное метрическое пространство, множество $M= \ovrl M \subset X 
\text{ и } f:M \ria M$~--- сжимающее отображение. Тогда $f$ имеет в $M$ 
единственную неподвижную точку $x^*=f(x^*)$.Кроме того, для любого $x_0 
\in M $ последовательность $\underset{n \in \mathbb{N} }{x_n=f(x_{n-1})}$ сходится 
к $x^*$ и справедлива оценка погрешности :

$\rho(x_n, x^*)\br \cfrac{q^n}{1-q}\rho(f(x_0), x_0)$

\textbf{Теорема 18   }

В метрическом пространстве любое относите компактное множество ограниченно.

\textbf{Теорема 19 (Вейерштрасса)}

Пусть $M$ компактное множество в метрическом пространстве $\mes$ и функция
$f:M \ria \mathbb{R}^1$ непрерывна на $M$. Тогда множество $f[M]$ отграничено в 
$\mathbb{R}^1$ и $\exists(a \in M)[f(a)=\underset{x \in M}{\inf} f(x)]\wedge 
\exists(b\in M)[f(b)=
\underset{x \in M}{\sup} f(x)] $

\textbf{Теорема 20}

Вполне ограниченное множество в метрическом пространстве ограниченно.

\textbf{Следствие из теоремы Хаусдорфа}

\MS. Для того чтобы множество $M \subset X$ было относительно компактным 
необходимо, а в случае полноты пространства $X$ и достаточно, чтобы множество $M$
было вполне ограниченным.

\textbf{Теорема 22 (относительная компактность в $ \Rnp $) }

Множество $M \subset \Rnp$ относительно компактно \tttk это множество в $\Rnp$
ограниченно.

%%%%%%%%%%%%%%%%%%%%%%%%%%%%%%%%%%%%%%%%%%%%%%%%%%

С  доками:

\textbf{Примеры полных метрических пространств  }

\textbf{Примеры метрических пространств, не являющихся полными}

\textbf{Теорема 8 (об открытом множестве на прямой) }

\textbf{Теорема 9 (о замкнутом множестве на прямой)}

\textbf{Теорема 13 (Бэра)}

\textbf{Примеры сепарабельных пространств}

\textbf{Примеры пространств, не являющихся сепарабельными}

\textbf{Теорема 21 (Хаусдорфа) }

\textbf{Теорема 23 (Арцела) }
\end{document}