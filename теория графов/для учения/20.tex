\textbf{Игры на графе, игра НИМ}

\noindent\textbf{Определение игры на графе:} \\
Граф $(X, \Gamma)$ определяет игру двух игроков $(A)$ и $(B)$. Положениями игры служат вершины графа. Начальная вершина $x_0$ выбирается жребием. Игроки ходят поочередно: $(A)$ выбирает $x_1 \in \Gamma x_0$, затем $(B)$ выбирает $x_2 \in \Gamma x_1$, и так далее. Если $\Gamma x_n = \emptyset$, игрок, выбравший $x_n$, выигрывает.

\noindent\textbf{Игра НИМ:} \\
Эта игра называется \textit{игрой Ним}. Задача — охарактеризовать выигрышные положения, т.е. вершины, выбор которых обеспечивает выигрыш независимо от ответов противника.

\noindent\textbf{Теорема 1:} \textit{Если граф имеет ядро $S$, и игрок выбрал вершину в $S$, то это обеспечивает ему выигрыш или ничью.}

\noindent\textbf{Доказательство:} \\
Если $(A)$ выбрал $x_1 \in S$, то либо $\Gamma x_1 = \emptyset$, и он выиграл, либо $(B)$ выбирает $x_2 \in X \setminus S$, и $(A)$ может выбрать $x_3 \in S$.

\noindent\textbf{Метод вычисления выигрышных позиций:} \\
Основной метод — вычисление функции Гранди $g(x)$. Ядро $S = \{ x | g(x) = 0 \}$. Если $g(x_0) = 0$, $(A)$ в критическом положении. Если $g(x_0) \neq 0$, $(A)$ может выиграть, выбрав $x_1$ с $g(x_1) = 0$.

\noindent\textbf{Следствие:} \\
Если граф прогрессивно конечен, существует единственная функция Гранди $g(x)$. Выбор $y$ с $g(y) = 0$ — выигрышный, $z$ с $g(z) \neq 0$ — проигрышный.
