\textbf{Экстремальные графы}

\noindent\textbf{Теорема 2.3 (Турана)}: Наибольшее число рёбер у графов с $r$ вершинами без треугольников равно $\lfloor r^2/4 \rfloor$.

\noindent\textbf{Доказательство} (по индукции для чётных $r$):
\begin{enumerate}
    \item База: очевидна для малых $r$.
    \item Шаг: для $r = 2n + 2$, где утверждение верно для всех чётных $r \leq 2n$:
    \begin{itemize}[noitemsep]
        \item Пусть $G$ — граф с $p = 2n + 2$ вершинами без треугольников.
        \item Существуют смежные вершины $u$, $v$ (граф не вполне несвязный).
        \item В подграфе $G' = G - \{u, v\}$ максимум $n^2$ рёбер.
        \item Нет вершины $w$, смежной с $u$ и $v$ одновременно.
        \item Если $w$ смежна с $k$ вершинами $G'$, то $v$ смежна максимум с $(2n - k)$ вершинами.
        \item Всего рёбер: $n^2 + k + (2n - k) + 1 = n^2 + 2n + 1 = p^2/4$.
    \end{itemize}
\end{enumerate}

\noindent\textbf{Конструктивное доказательство существования:}\\
Для чётного $p$ $(p, p^2/4)$-граф без треугольников строится так:
\begin{itemize}[noitemsep]
    \item Берём два множества $V_1$ и $V_2$ по $p/2$ вершин.
    \item Соединяем каждую вершину из $V_1$ с каждой из $V_2$.
\end{itemize}

\noindent\textbf{Примечания:}
\begin{itemize}[noitemsep]
    \item Доказательство существования чисел $r(m, n)$ см. у М. Холла.
    \item По определению бесконечный граф не является графом.
    \item Обзор бесконечных графов: см. Нэш-Вильямс.
\end{itemize}

\noindent\textbf{Теорема 2.4}: Граф является двудольным тогда и только тогда, когда все его простые циклы чётны.

\noindent\textbf{Доказательство:}
\begin{itemize}[noitemsep]
    \item Если $G$ — двудольный граф, то его вершины можно разбить на $V_1$ и $V_2$, и любое ребро соединяет вершины из разных множеств.
    \item Каждый простой цикл $v_1v_2\ldots v_nv_1$ содержит вершины из $V_1$ и $V_2$, так что длина $n$ цикла чётна.
    \item Обратное: если все простые циклы чётны, то каждое ребро соединяет $V_1$ и $V_2$.
\end{itemize}

\noindent\textbf{Дополнительные результаты:}
\begin{itemize}[noitemsep]
    \item $ex(p, C_p) = \left\lfloor \frac{1}{2} + \frac{p(p-1)}{2} \right\rfloor$
    \item $ex(p, K_{4-x}) = \left\lfloor \frac{p^2}{4} \right\rfloor$
    \item $ex(p, K_{3, x} - x) = \left\lfloor \frac{p^2}{4} \right\rfloor$
\end{itemize}

\noindent\textbf{Обобщение Турана}: $ex(p, K_n) = \frac{(n-2)(p^2 - r^2)}{2(n-1)} + \binom{r}{2}$, где $p \equiv r \pmod{(n-1)}$ и $0 \leq r < n-1$.