\subsection{Внешнее устойчивое множество}

Пусть дан граф $G = (X, \Gamma)$, говорят, что множество $T \subseteq X$ внешне устойчиво, если для каждой вершины $x \notin T$ имеем $\Gamma_x \cap T \neq \varnothing$, иначе говорят, если $\Gamma' T \supseteq X \setminus T$,

Если $\mathcal{T}$ — семейство всех внешне устойчивых множеств графа, то $X \in \mathcal{T}$

$T \in \mathcal{T} \quad A \supseteq T \Rightarrow A \in \mathcal{T}$

По определению, \textit{число внешней устойчивости} графа $G$ есть

\[
\beta(G) = \min_{T \in \mathcal{T}} |T|
\]

Задача, которая нас сейчас интересует, заключается в построении внешне устойчивого множества с наименьшим числом элементов.

\subsubsection*{Алгоритм для нахождения наименьшего внешне устойчивого множества}

Рассмотрим для примера граф $G = (X, \Gamma)$ изображенный на рис. 4--11 и определим отображение $\Delta$ множества $X = \{a, b\}$ в новое множество

\[
\overline{X} = \left\{
\begin{array}{c}
\overline{a} \\
\overline{b}
\end{array}
\right\}
\]

следующим образом

\[
\overline{y} \in \Delta x \iff y = x \text{ или } y \in \Gamma^{-1} x
\]

Тем самым построен так называемый простой граф\footnote{Простой граф --- это граф без петель и кратных рёбер.}, который мы обозначим через $(X, \overline{X})$ (рис. 4--12), если $T$ --- внешне устойчивое множество графа $G$, то $\Delta T = \overline{X}$. Наоборот, если $\Delta T = \overline{X}$, то множество $T$ внешне устойчиво в $G$. Задача свелась таким образом, к определению наименьшего множества $T \subseteq X$, для которого $\Delta T = \overline{X}$.

1\textdegree{} Удаляем из простого графа каждую такую вершину $x$, что $\Delta x \subseteq \Delta y$ для некоторой вершины $y \neq x$ (в самом деле, с точки зрения нашей задачи вершина $y$ будет полностью заменять вершину $x$). В нашем примере мы удаляем, таким образом, вершины $c, d, f$.

2\textdegree{} Если в простом графе имеется висячее ребро $(x, y)$, то очевидно, $x \in T$. В данном примере множеству $T$ заведомо принадлежит вершина $a$.

3\textdegree{} Исключим из простого графа вершину $a$, уже входящую в $T$, и множество $\Delta a = \{\overline{a}, \overline{b}, \overline{c}\}$, в результате получается граф, изображенный на рис. 4--13.

4\textdegree{} Снова пытаемся удалить некоторую вершину, как в 1\textdegree{} или исключив вершину, заведомо принадлежащую $T$, как в 2\textdegree{}, если упростить граф уже нельзя (как в данном примере), то назовем его неприводимым. Временно отнесем в $T$ произвольную вершину, скажем $b$.

5\textdegree{} Исключим, как в 3\textdegree{}, вершину $b$ и множество $\Delta b = \{a, e, f\}$.

6\textdegree{} Продолжаем упрощение, как выше: из полученного графа можно исключить вершину $g$, так как $\Delta g \subseteq \Delta e = \{g\}$. Включая в $T$ последнюю вершину $e$, получаем решение $T = \{a, b, e\}$.
