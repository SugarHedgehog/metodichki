\documentclass[a4paper, 12pt]{extarticle}
\usepackage{fontspec}
\usepackage{polyglossia}
\setmainfont{CMU Serif}
\newfontfamily{\cyrillicfont}{CMU Serif}
\setsansfont{CMU Sans Serif}
\newfontfamily{\cyrillicfontsf}{CMU Sans Serif}
\setmonofont{CMU Typewriter Text}
\newfontfamily{\cyrillicfonttt}{CMU Typewriter Text}
\setdefaultlanguage{russian}
\usepackage[left=1cm,right=1cm,top=1cm,bottom=2cm]{geometry}
\title{Доклад на тему педагогическое мышление}
\author{Суматохина Александра 3 курс КТФ}
\date{6 марта 2023}
\begin{document}
\maketitle

Профессиональное мышление педагога - есть сложное образование, характеризующееся своеобразием структуры, содержательного и практически - действенного фондов, качественных характеристик и направленностью на решение практических задач по преобразованию деятельности и личности учащегося.

Согласно точке зрения М.М. Кашапова. Т.Г. Киселевой, профессиональное педагогическое мышление осуществляется на двух уровнях: ситуативном и надситуативном. 

Ситуативность - надситуативность мышления педагога приводит к психологическим отличиям в целемотивационном компоненте профессиональной деятельности, в выделении информационной основы, в составлении программы деятельности и принятии педагогического решения.

Это позволяет дать следующее рабочее определение: \textbf{педагогическое мышление - это высший познавательный процесс поиска, обнаружения педагогической проблемности в ходе профессиональной деятельности педагога}. Осознание проблемности приводит к возникновению педагогической проблемной ситуации и стимулирует решение педагогической задачи.

\subsubsection*{Структура педагогического мышления педагога}

Современного учителя должны отличать инициатива и ответственность, потребность в постоянном обновлении и обогащении своих знаний, способность смело принимать новаторские решения и активно проводить их в жизнь.

Мышление учителя включено в педагогическую деятельность и направлено на решение специфических для нее задач. Это поиск педагогических идей, средств преобразования педагогического процесса. Объект познания учителя - педагогическая реальность, включающая личность и коллектив, содержание, формы и методы учебно-воспитательного процесса, педагогические ситуации и явления.

Педагогическое мышление существует как процесс решения педагогических задач. Его характеризуют полифункциональность, иерархизированность, эвристичность поисковых структур, многокомпонентность.

В настоящее время обозначилось несколько путей развития мышления учителя: 

функционально-операциональный, суть которого состоит в том, что студентам предлагаются для решения отдельные педагогические задачи, выделенные в соответствии с основными структурными компонентами профессионально-педагогической деятельности (гностические, конструктивные, организаторские, коммуникативные, проектировочные); 

конструктивно-методический - будущие учителя решают конкретные методические проблемные ситуации, в ходе которых у них формируется так называемое методическое мышление;

проблемно-методический - студенты ищут ответ на типичные вопросы, возникающие в практике учебно-воспитательной работы, развивают свои способности творчески разрабатывать решения, оптимальные для конкретных условий практической деятельности. 

Основные задачи, решаемые педагогом, - это идейно-нравственное, трудовое, эстетическое и физическое воспитание. Каждая из них состоит из более частных задач. Так, например, задача умственного развития включает формирование системы знаний, развитие познавательных процессов, мышления. Каждая из этих задач реализуется через более конкретные составляющие - формирование понятий, мыслительных действий и операций, интересов и т. д.

Собственно педагогические цели достигаются конструктивным путем - разработкой тематических планов, планов-конспектов уроков, конструированием конкретного учебного содержания, организационных форм и методов обучения в логике решения тех или иных педагогических задач.

Конструктивные задачи (планы) воплощаются в жизнь благодаря решению организационных (разработка форм обучения, решение конкретных задач) и коммуникативных (на общение) задач.

В ходе решения каждой задачи актуализируется определенный мыслительный процесс.

Используя средства формализации, можно представить нормативную модель педагогического мышления учителя. Она включает цели умственного, трудового, идейно-нравственного, эстетического и физического воспитания; гностические, проектировочно-конструктивные задачи, результаты решения различных задач.

В качестве показателей развития профессионального мышления использовались следующие характеристики: отношение количества правильно решенных экспериментальных педагогических задач к набору задач, входящих в нормативную модель мыслительной деятельности учителя (коэффициент профессионализма); качественный анализ решения задач на основе разработанной нами шкалы уровней решения, основанной на концепции П.Я. Пономарева о развитии внутреннего плана деятельности.

\end{document}