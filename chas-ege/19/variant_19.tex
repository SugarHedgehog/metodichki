\documentclass[twocolumn]{article}
\usepackage{dashbox}
\setlength{\columnsep}{40pt}
\usepackage[T2A]{fontenc}
\usepackage[utf8]{inputenc}
\usepackage[english,russian]{babel}
\usepackage{graphicx}
\graphicspath{{pictures/}}
\DeclareGraphicsExtensions{.pdf,.png,.jpg}

\linespread{1.15}

\usepackage{../egetask}
\usepackage{../egetask-math-11-2022}

\def\examyear{2023}
\usepackage[colorlinks,linkcolor=blue]{hyperref}\def\rfoottext{Разрешается свободное копирование в некоммерческих целях с указанием источника }
\def\lfoottext{Источник \href{https://vk.com/egemathika}{https://vk.com/egemathika}}

\begin{document}



\cleardoublepage
\def\examvart{Вариант 19.1}
\normalsize

\begin{center}
	\textbf{
		Единый государственный экзамен\\по МАТЕМАТИКЕ\\Профильный уровень\\ \qquad \\ Инструкция по выполнению работы
	}
\end{center}


\par \qquad Экзаменационная работа состоит из двух частей, включающих в себя 18 заданий. Часть 1 содержит 11 заданий с кратким ответом базового и повышенного уровней сложности. Часть 2 содержит 7 заданий с развёрнутым ответом повышенного и высокого уровней сложности.
\par \qquad На выполнение экзаменационной работы по математике отводится 3 часа 55 минут (235 минут).
\par \qquad Ответы к заданиям 1—11 записываются по приведённому ниже \underline {образцу} в виде целого числа или конечной десятичной дроби. Числа запишите в поля ответов в тексте работы, а затем перенесите их в бланк ответов №1.
%%\includegraphics[width=0.98\linewidth]{obrazec}
\par \qquad При выполнении заданий 12—18 требуется записать полное решение и ответ в бланке ответов №2.
\par \qquad  Все бланки ЕГЭ заполняются яркими чёрными чернилами. Допускается использование гелевой или капиллярной ручки.
\par \qquad При выполнении заданий можно пользоваться черновиком. \textbf{Записи в черновике, а также в тексте контрольных измерительных материалов не учитываются при оценивании работы.}
\par \qquad  Баллы, полученные Вами за выполненные задания, суммируются. Постарайтесь выполнить как можно больше заданий и набрать наибольшее количество баллов.
\par \qquad После завершения работы проверьте, что ответ на каждое задание в бланках ответов №1 и №2 записан под правильным номером.
\begin{center}
	\textit{\textbf{Желаем успеха!}}\\ \qquad \\\textbf{ Справочные материалы} \\
$\sin^2 \alpha + \cos^2 \alpha = 1$ \\
$\sin 2\alpha=2\sin \alpha \cdot \cos \alpha$ \\
$\cos 2\alpha=\cos^2 \alpha-\sin^2 \alpha$ \\
$\sin (\alpha+\beta)=\sin \alpha \cdot \cos \beta+\cos \alpha \cdot \sin\beta$ \\
$\cos (\alpha+\beta)=\cos \alpha \cdot \cos \beta-\sin\alpha \cdot \sin\beta$
\end{center}

\startpartone
\large




\begin{taskBN}{1}
В треугольнике $RPL$ угол $R$ равен $90^\circ$.Чему равна  $LR$, если $PL=\sqrt{360}$, а  $\ctg{P}=3$? 
\end{taskBN}

\begin{taskBN}{2}
\addpictoright[0.4\linewidth]{images/7381500162809165n0}В правильном тетраэдре апофема равна $2\sqrt{3}$. Чему равна площадь основания тетраэдра? Ответ разделите на $\sqrt{3}$.\vspace{2.5cm}
\end{taskBN}

\begin{taskBN}{3}
В чемпионате по вольной борьбе участвуют 20 спортсменок: 6 из Бельгии, 3 из Мексики, 11 из Австралии, остальные — из Англии. Порядок, в котором выступают спортсменки, определяется жребием. Найдите вероятность того, что спортсменка, выступающая предпоследней, окажется из Австралии.
\end{taskBN}

\begin{taskBN}{4}
На коллоквиуме по математическому анализу студенту достаётся один из вопросов. Вероятность того, что это вопрос по теме "Правило Лопиталя", равна 0,4. Вероятность того, что это вопрос по теме "Теорема Лагранжа", равна 0,12. Вопросов, которые одновременно относятся к этим двум темам, нет. Найдите вероятность того, что на коллоквиуме студенту достанется вопрос по одной из этих двух тем.
\end{taskBN}

\begin{taskBN}{5}
Найдите корень уравнения $$4^{\log_{16}{(-4x-9)}}=1$$
\end{taskBN}

\begin{taskBN}{6}
Вычислить значение выражения: $$ \frac{{41}\cos{186}^\circ}{\cos{6}^\circ}$$
\end{taskBN}

\begin{taskBN}{7}
Прямая $y=41x-77$ является касательной к графику функции $y=-7x^{2}+bx-189$. Найдите $b$, зная, что оно меньше -11.
\end{taskBN}

\begin{taskBN}{8}
По закону Ома для полной цепи сила тока, измеряемая в амперах, равна $I=\frac{\varepsilon}{R+r}$, где $\varepsilon$ — ЭДС источника (в вольтах), $r= 3{,}2 $ Ом — его внутреннее сопротивление, $R$ — сопротивление цепи (в омах). При каком наименьшем сопротивлении цепи сила тока будет составлять не более $50\%$ от силы тока короткого замыкания $I_{\mbox{кз}}=\frac{\varepsilon}{r}$? (Ответ выразите в омах.)
\end{taskBN}

\begin{taskBN}{9}
От пристани С к пристани А  отправился с постоянной скоростью первый теплоход, через некоторое время следом за ним отправился второй.Скорость второго теплохода больше скорости первого на 6 км/ч. Известно, что скорость первого теплохода составляет 12 км/ч. Чему равно расстояние между пристанями, выраженное в км, если в пункт А оба теплохода прибыли одновременно, а второй теплоход вышел на 2 часа позже первого? 
\end{taskBN}

\begin{taskBN}{10}
\addpictoright[0.4\linewidth]{images/256129981221142n0}На рисунке изображён график функции $f(x)=a\tg x+b$. Найдите $b$.\vspace{2.5cm}
\end{taskBN}

\begin{taskBN}{11}
Определите наименьшее значение функции $y =\frac{8\sqrt{3}\pi}{3}-16\cos x-8\sqrt{3}x-83$ на полуинтервале $\left[-\frac{\pi}{3};\frac{\pi}{2} \right)$
\end{taskBN}

\newpage
 Ответы


\begin{tabular}{*{1}l}
\begin{tabular}[t]{|l|l|}
\hline
1 & 6\\
\hline
2 & 4\\
\hline
3 & 0,55\\
\hline
4 & 0,52\\
\hline
5 & -2,5\\
\hline
6 & -41\\
\hline
7 & -15\\
\hline
8 & 3,2\\
\hline
9 & 72\\
\hline
10 & -1\\
\hline
11 & -91\\
\hline
\end{tabular}\end{tabular}



\newpage




\cleardoublepage
\def\examvart{Вариант 19.2}
\normalsize

\begin{center}
	\textbf{
		Единый государственный экзамен\\по МАТЕМАТИКЕ\\Профильный уровень\\ \qquad \\ Инструкция по выполнению работы
	}
\end{center}


\par \qquad Экзаменационная работа состоит из двух частей, включающих в себя 18 заданий. Часть 1 содержит 11 заданий с кратким ответом базового и повышенного уровней сложности. Часть 2 содержит 7 заданий с развёрнутым ответом повышенного и высокого уровней сложности.
\par \qquad На выполнение экзаменационной работы по математике отводится 3 часа 55 минут (235 минут).
\par \qquad Ответы к заданиям 1—11 записываются по приведённому ниже \underline {образцу} в виде целого числа или конечной десятичной дроби. Числа запишите в поля ответов в тексте работы, а затем перенесите их в бланк ответов №1.
%%\includegraphics[width=0.98\linewidth]{obrazec}
\par \qquad При выполнении заданий 12—18 требуется записать полное решение и ответ в бланке ответов №2.
\par \qquad  Все бланки ЕГЭ заполняются яркими чёрными чернилами. Допускается использование гелевой или капиллярной ручки.
\par \qquad При выполнении заданий можно пользоваться черновиком. \textbf{Записи в черновике, а также в тексте контрольных измерительных материалов не учитываются при оценивании работы.}
\par \qquad  Баллы, полученные Вами за выполненные задания, суммируются. Постарайтесь выполнить как можно больше заданий и набрать наибольшее количество баллов.
\par \qquad После завершения работы проверьте, что ответ на каждое задание в бланках ответов №1 и №2 записан под правильным номером.
\begin{center}
	\textit{\textbf{Желаем успеха!}}\\ \qquad \\\textbf{ Справочные материалы} \\
$\sin^2 \alpha + \cos^2 \alpha = 1$ \\
$\sin 2\alpha=2\sin \alpha \cdot \cos \alpha$ \\
$\cos 2\alpha=\cos^2 \alpha-\sin^2 \alpha$ \\
$\sin (\alpha+\beta)=\sin \alpha \cdot \cos \beta+\cos \alpha \cdot \sin\beta$ \\
$\cos (\alpha+\beta)=\cos \alpha \cdot \cos \beta-\sin\alpha \cdot \sin\beta$
\end{center}

\startpartone
\large




\begin{taskBN}{1}
К окружности, вписанной в треугольник $LWJ$, проведены три касательные. Периметры отсеченных треугольников равны 31, 42, 67. Найдите периметр треугольника $LWJ$.
\end{taskBN}

\begin{taskBN}{2}
\addpictoright[0.4\linewidth]{images/596948642733143n0}В правильном тетраэдре площадь боковой поверхности равна $48\sqrt{3}$. Чему равна площадь основания тетраэдра? Ответ разделите на $\sqrt{3}$.\vspace{2.5cm}
\end{taskBN}

\begin{taskBN}{3}
В случайном эксперименте симметричную монету бросают 4 раза. Какова вероятность того, что решка выпадет трижды?
\end{taskBN}

\begin{taskBN}{4}
При изготовлении подшипников радиусом 20 мм вероятность того, что радиус будет отличаться от заданного не больше, чем на 0,013 мм, равна 0,963. Найдите вероятность того, что случайный подшипник будет иметь радиус меньше чем 19,987 мм или больше чем 20,013 мм.
\end{taskBN}

\begin{taskBN}{5}
Найдите корень уравнения $$-49x^2-43=-(7x+43)^2$$
\end{taskBN}

\begin{taskBN}{6}
Найдите значение выражения $$\frac{\log_{35,5}\sqrt[20]{13}}{\log_{35,5}{13}}.$$
\end{taskBN}

\begin{taskBN}{7}
\addpictoright[0.4\linewidth]{images/223207196501726n0}На рисунке изображены график функции $y=f(x)$ и касательная к этому графику, проведённая в точке $v$. Уравнение касательной имеет вид $y= 0{,}13  x+ 4 $. Найдите значение производной функции $g(x) =  4 f(x)+ 2{,}5 x+\frac{71428571}{1000000000}$ в точке $v$.\vspace{2.5cm}
\end{taskBN}

\begin{taskBN}{8}
В ходе распада радиоактивного изотопа его масса уменьшается по закону $m(t) = m_02^{-t/T}$, где $m_0$ — начальная масса изотопа, $t$ (сек) — прошедшее от начального момента время, $T$ — период полураспада в секундах. В лаборатории получили вещество, содержащее в начальный момент времени $m (t) = 928$ г изотопа Z, период полураспада которого $T = 7$ сек. В течение скольких секунд масса изотопа будет больше 58 г?
\end{taskBN}

\begin{taskBN}{9}
Часы со стрелками показывают 9 часов 57 минут. Через сколько минут минутная стрелка в тринадцатый раз поравняется с часовой?
\end{taskBN}

\begin{taskBN}{10}
\addpictoright[0.4\linewidth]{images/188492710027687n0}На рисунке изображёны графики двух линейных функций. Найдите абсциссу точки пересечения графиков.\vspace{2.5cm}
\end{taskBN}

\begin{taskBN}{11}
Вычислите точку минимума функции $y = -(x-11)^{2}e^{x-56}-10$
\end{taskBN}

\newpage
 Ответы


\begin{tabular}{*{1}l}
\begin{tabular}[t]{|l|l|}
\hline
1 & 140\\
\hline
2 & 16\\
\hline
3 & 0,25\\
\hline
4 & 0,037\\
\hline
5 & -3\\
\hline
6 & 0,05\\
\hline
7 & 3,02\\
\hline
8 & 28\\
\hline
9 & 843\\
\hline
10 & -4,2\\
\hline
11 & 9\\
\hline
\end{tabular}\end{tabular}



\newpage




\cleardoublepage
\def\examvart{Вариант 19.3}
\normalsize

\begin{center}
	\textbf{
		Единый государственный экзамен\\по МАТЕМАТИКЕ\\Профильный уровень\\ \qquad \\ Инструкция по выполнению работы
	}
\end{center}


\par \qquad Экзаменационная работа состоит из двух частей, включающих в себя 18 заданий. Часть 1 содержит 11 заданий с кратким ответом базового и повышенного уровней сложности. Часть 2 содержит 7 заданий с развёрнутым ответом повышенного и высокого уровней сложности.
\par \qquad На выполнение экзаменационной работы по математике отводится 3 часа 55 минут (235 минут).
\par \qquad Ответы к заданиям 1—11 записываются по приведённому ниже \underline {образцу} в виде целого числа или конечной десятичной дроби. Числа запишите в поля ответов в тексте работы, а затем перенесите их в бланк ответов №1.
%%\includegraphics[width=0.98\linewidth]{obrazec}
\par \qquad При выполнении заданий 12—18 требуется записать полное решение и ответ в бланке ответов №2.
\par \qquad  Все бланки ЕГЭ заполняются яркими чёрными чернилами. Допускается использование гелевой или капиллярной ручки.
\par \qquad При выполнении заданий можно пользоваться черновиком. \textbf{Записи в черновике, а также в тексте контрольных измерительных материалов не учитываются при оценивании работы.}
\par \qquad  Баллы, полученные Вами за выполненные задания, суммируются. Постарайтесь выполнить как можно больше заданий и набрать наибольшее количество баллов.
\par \qquad После завершения работы проверьте, что ответ на каждое задание в бланках ответов №1 и №2 записан под правильным номером.
\begin{center}
	\textit{\textbf{Желаем успеха!}}\\ \qquad \\\textbf{ Справочные материалы} \\
$\sin^2 \alpha + \cos^2 \alpha = 1$ \\
$\sin 2\alpha=2\sin \alpha \cdot \cos \alpha$ \\
$\cos 2\alpha=\cos^2 \alpha-\sin^2 \alpha$ \\
$\sin (\alpha+\beta)=\sin \alpha \cdot \cos \beta+\cos \alpha \cdot \sin\beta$ \\
$\cos (\alpha+\beta)=\cos \alpha \cdot \cos \beta-\sin\alpha \cdot \sin\beta$
\end{center}

\startpartone
\large




\begin{taskBN}{1}
В треугольнике $HES$ угол $H$ равен $90^\circ$.Чему равна  $SH$, если $\ctg^2{E}=\frac{4}{81}$? При этом  $EH=5$. 
\end{taskBN}

\begin{taskBN}{2}
\addpictoright[0.4\linewidth]{images/315962875718251n0}В правильной шестиугольной пирамиде апофема составляет 16; площадь боковой поверхности составляет 768. Чему равна сторона основания пирамиды?\vspace{2.5cm}
\end{taskBN}

\begin{taskBN}{3}
Перед началом первого тура чемпионата по настольному теннису участниц разбивают на игровые пары случайным образом с помощью жребия. Всего в чемпионате участвует 186 спортсменок, среди которых 75 участниц из Австралии, в том числе Олеся. Найдите вероятность того, что в первом туре Олеся будет играть с какой-либо спортсменкой не из Австралии.
\end{taskBN}

\begin{taskBN}{4}
Вероятность того, что на контрольной работе по математике ученик Ф. верно решит больше 9 задач, равна 0.57. Вероятность того, что Ф. верно решит больше 8 задач, равна 0.81. Найдите вероятность того, что Ф. верно решит ровно 9 задач.
\end{taskBN}

\begin{taskBN}{5}
Найдите корень уравнения $$-\frac{x+8}{-11x+5}=-\frac{x+8}{2x-8}$$ Если корней несколько, в ответе укажите их произведение.
\end{taskBN}

\begin{taskBN}{6}
Найдите значение выражения $$44\cos\frac{3\pi}{4}\sin\frac{\pi}{4}$$
\end{taskBN}

\begin{taskBN}{7}
Прямая $y=71x+4$ параллельна касательной к графику функции $y=4x^{2}+7x+3$. Найдите ординату точки касания.
\end{taskBN}

\begin{taskBN}{8}
Некоторая компания продает свою продукцию. Месячная операционная прибыль предприятия (в рублях) вычисляется по формуле $a(q)=q(p-v)-f$. Цена за единицу продукции $p=360~\mbox{руб}.$. Чему равны переменные затраты на производство одной единицы продукции $v$, выраженные в \mbox{руб}., если месячный объём производства продукции $q=13200~\mbox{шт}.$? При этом месячная операционная прибыль предприятия составляет 2.1097 \mbox{млн. руб}., а постоянные расходы предприятия $f=1718300~\mbox{руб}.$. 
\end{taskBN}

\begin{taskBN}{9}
Между остановками К. и Л. 6 километров прямой трассы. Кристина выехала на велосипеде из леса между станциями в 600 метрах от К. и увидела, что к К. в направлении к Л. с постоянной скоростью подъезжает автобус, на который Кристине нужно успеть. Она заметила, что если она сейчас поедет в сторону К., она окажется там одновременно с автобусом. Но и если она поедет в сторону Л., она также окажется там одновременно с автобусом, который успеет преодолеть весь участок от К. до Л., не останавливаясь на остановке К. Каково сейчас расстояние (в км) между Кристиной и автобусом?  (Считайте, что автобус и велосипед движутся с постоянными скоростями и останавливаются мгновенно.)
\end{taskBN}

\begin{taskBN}{10}
\addpictoright[0.4\linewidth]{images/0276487289126877n0}На рисунке изображён график функции $f(x)=k\sqrt{x}$. Найдите значение $x$, при котором $f(x)= -45 $. \vspace{2.5cm}
\end{taskBN}

\begin{taskBN}{11}
Вычислите точку максимума функции $y = -88-(-18+x)e^{x+18}$
\end{taskBN}

\newpage
 Ответы


\begin{tabular}{*{1}l}
\begin{tabular}[t]{|l|l|}
\hline
1 & 22,5\\
\hline
2 & 16\\
\hline
3 & 0.6\\
\hline
4 & 0,24\\
\hline
5 & -8\\
\hline
6 & -22\\
\hline
7 & 315\\
\hline
8 & 70\\
\hline
9 & 1,35\\
\hline
10 & 225\\
\hline
11 & 17\\
\hline
\end{tabular}\end{tabular}



\newpage
\end{document}