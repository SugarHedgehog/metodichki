\documentclass[a4paper, 12pt]{extarticle}
\usepackage{fontspec}
\usepackage{polyglossia}
\setmainfont{CMU Serif}
\newfontfamily{\cyrillicfont}{CMU Serif}
\setsansfont{CMU Sans Serif}
\newfontfamily{\cyrillicfontsf}{CMU Sans Serif}
\setmonofont{CMU Typewriter Text}
\newfontfamily{\cyrillicfonttt}{CMU Typewriter Text}
\setdefaultlanguage{russian}
\usepackage{amsfonts}

\usepackage[left=1.5cm,right=1cm,top=2cm,bottom=2cm]{geometry}
\title{Применение нейронных сетей к проблемам генерации
задач по планиметрии}
\author{Суматохина Александра 5 курс Кафедра Теории функции и геометрии\\Научный руководитель: д.ф.-м.н., проф. Семенов Е.М.\\Научный консультант: асп. Авдеев Н.Н}
\date{16 апреля 2025 г}
\begin{document}
\maketitle

Здравствуйте, меня зовут Александра Суматохина , я обучаюсь на 5 курсе, мой научный руководитель доктор физико-математических наук Евгений Михайлович Семенов, научный консультант аспирант Николай Николаевич Авдеев. Тема моего доклада "применение нейронных сетей к проблемам генерации задач по планиметрии".

\subsection*{Существующие проблемы}
При подготовке к ЕГЭ школьники и педагоги сталкиваются с нехваткой заданий по определённым темам. Например, после включения в ЕГЭ заданий по графикам функций (2021) и векторам (2023) возникла проблема дефицита материалов для подготовки. Также существует проблема с заданиями по теме "Производная и первообразная", где банк заданий быстро исчерпывается.

Кроме того, доступные материалы часто имеют готовые решения, что приводит к списыванию, а задания с чертежами не всегда корректно иллюстрируют условия задачи.

Проект «Час ЕГЭ» позволяет решить все эти проблемы.

\subsection*{Проект «Час ЕГЭ»}
«Час ЕГЭ» — компьютерный образовательный проект, разрабатываемый при математическом
факультете ВГУ в рамках «OpenSource кластера» и предназначенный для помощи учащимся
старших классов, учителями и репетиторам при подготовке к тестовой части единого государственного экзамена.

Задания в «Час ЕГЭ» генерируются случайным образом по специализированным алгоритмам, называемых шаблонами, каждый из которых охватывает множество вариантов соответствующей ему задачи.

%\subsection*{За этот год были достигнуты следующие пункты в ЕГЭ профиле}%TODO:А надо ли?
%\begin{itemize}
%    \item В разделе планиметрии разработано и принято 26 шаблонов
%    \item По теме векторов написано 18 шаблонов, из которых 10 уже приняты, а 8 находятся на рецензировании
%    \item В стереометрии разработан 61 шаблон, где 8 приняты и 53 проходят рецензирование
%    \item По теории вероятности написано 10 шаблонов базового уровня (9 принято, 1 на рецензировании) и 11 шаблонов повышенной сложности уже приняты
%    \item В разделе производной и первообразной 17 шаблонов находятся на рецензировании
%\end{itemize}

\subsection*{В ЕГЭ Базе покрыт открытый банк заданий по темам:}
\begin{itemize}
	\item По планиметрии написано 60 шаблонов, находящихся на внутреннем рецензировании (code review)
	\item По теории вероятности написано 11 шаблонов (2  принято, 9 на code review)
\end{itemize}

\subsection*{По ОГЭ покрыт открытый банк заданий по темам:}
\begin{itemize}
	\item По планиметрии написано 16 шаблонов по теме треугольники (code review)
\end{itemize}

\begin{itemize}
	\item По теме графики функций написано 2 шаблона (code review)
\end{itemize}


Важным техническим достижением стала разработка библиотеки \texttt{flatten-shape-geometry}, которая опубликована в NPM(Node Package Managers). Библиотека включает в себя две ключевые зависимости: \texttt{@flatten-js/core}, \texttt{mathjs} и др.

В рамках библиотеки реализованы следующие классы:
\begin{itemize}
    \item ShapeWithConnectionMatrix
    \item Angle
    \item Triangle
    \item Quadrilateral
    \item Square
    \item Rectangle
    \item Rhombus
    \item Parallelogram
    \item Trapezoid
\end{itemize}

Библиотека содержит множество вспомогательных функций, включая:
\begin{itemize}
    \item Функция нахождения перпендикуляра опущенного из точки на отрезок
    \item Функция сдвига координат
    \item Функция нахождение центра описанной окружности
    \item Другие геометрические операции
\end{itemize}

Для класса Circle из \texttt{@flatten-js/core} были дополнительно разработаны методы:
\begin{itemize}
    \item Получение точки на окружности по заданному углу
    \item Построение радиусов, диаметров и хорд
    \item Построение касательных из точки к окружности
\end{itemize}

\section{Устройство flatten-shape-geometry}

Библиотека построена на основе иерархической структуры классов, где базовыми элементами являются Point, Vector, Line, Segment и Circle. На их основе построены более сложные геометрические объекты, такие как Polygon и его наследники. Особое место занимает класс ShapeWithConnectionMatrix (SWCM), который служит основой для создания остальных геометрических фигур.

Перед публикацией любой библиотеки на NPM необходимо провести тестирование. 

\section{Использование нейросетей}

В процессе разработки мы успешно применяли нейросети для генерации кода. Например, был успешно реализован проектор 3D в 2D без использования сторонних библиотек. 
\subsection*{Тестирование инициализации Треугольника}
При тестировании класса Triangle с помощью DeepSeek R1 были выявлено:
\begin{itemize}
    \item Успешно протестирована инициализация треугольника с координатами $A(0,0)$, $B(4,0)$, $C(0,3)$
    \item Выявлены неточности при работе с углами в треугольнике с координатами $A(0,0)$, $B(4,4)$, $C(5,0)$. При этом с остальными тестами справляется.
\end{itemize}

\subsection*{Тестирование метода построения касательных}
При тестировании метода Circle.tangentsFromPoint с помощью DeepSeek R1 для окружности с центром в точке (0,0), радиусом 5 и точкой (10, 0) были выявлены следующие проблемы:
\begin{itemize}
    \item Некорректный расчет координат точек касания (ожидаемые значения x: ±3.5355, полученные: 4.3301)
    \item Неточности в вычислении y-координат точек касания (ожидаемые значения: 3.5355, полученные: 2.5)
    \item Проблемы с длинами касательных отрезков
\end{itemize}
\subsection*{Итоги}
Как итог, можно сделать следующие выводы:
\begin{itemize}
        \item Полностью покрыт открытый банк заданий ФИПИ ЕГЭ базового уровня по планиметрии и теории вероятности, а в ОГЭ по теме треугольники
		\item Разработано 89 шаблонов
		\item Нейросеть DeepSeek R1 способна генерировать простейшие тесты
		\item При усложнении задачи качество тестов существенно падает
		\item Нейросети не могут полностью заменить программиста при реализации математических алгоритмов в ближайшие 2-3 года
\end{itemize}

\end{document}
