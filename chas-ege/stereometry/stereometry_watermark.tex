\documentclass[4apaper]{article}
\usepackage{indentfirst}
\usepackage{dashbox}
\usepackage[T2A]{fontenc}
\usepackage[utf8]{inputenc}
\usepackage[english,russian]{babel}
\usepackage{graphicx}
\DeclareGraphicsExtensions{.pdf,.png,.jpg}

\linespread{1.15}

\usepackage{../egetask_ver}

\def\examyear{2023}
\usepackage[colorlinks,linkcolor=blue]{hyperref}\usepackage{draftwatermark}
\SetWatermarkLightness{0.9}
\SetWatermarkText{https://vk.com/egemathika}
\SetWatermarkScale{ 0.4 }
\def\lfoottext{Источник \href{https://vk.com/egemathika}{https://vk.com/egemathika}}

\begin{document}
\newpage\section*{Вариант № 1}

\begin{taskBN}{1}
Три ребра прямоугольного параллелепипеда, выходящие из одной вершины, равны 42, 36 и 49. Найдите ребро равновеликого ему куба.
\end{taskBN}

\begin{taskBN}{2}
\addpictoright[0.4\textwidth]{images/406280940191281n0}Во сколько раз увеличится объём правильного тетраэдра, если все его полная площадь поверхности увеличить в 26 раз?
\end{taskBN}

\begin{taskBN}{3}
\addpictoright[0.4\textwidth]{images/10667297604536n0}Площадь основания конуса в два раза меньше площади боковой поверхности. Найдите угол между образующей конуса и плоскостью основания. Ответ дайте в градусах.
\end{taskBN}

\begin{taskBN}{4}
\addpictoright[0.4\textwidth]{images/9032041227657506n0}Радиусы четырёх шаров равны $\sqrt[3]{38}$, $\sqrt[3]{17}$, $\sqrt[3]{7}$, $\sqrt[3]{2}$. Найдите радиус шара, объем которого равен сумме их объемов.
\end{taskBN}

\begin{taskBN}{5}
\addpictoright[0.4\textwidth]{images/434346432538081n0} Первая цилиндрическая кружка в 4 раза выше второй, а вторая в 9 раз шире первой. Найдите отношение объёма второй кружки к объёму первой.
\end{taskBN}

\begin{taskBN}{6}
\addpictoright[0.4\textwidth]{images/483676841594316n0}Во сколько раз увеличится диагональ куба, если его ребро увеличится в 16 раз?
\end{taskBN}

\newpage\section*{Вариант № 2}

\begin{taskBN}{1}
В прямоугольном параллелепипеде  $DQTRD_{1}Q_{1}T_{1}R_{1}$  известно, что  $DQ = 8$, $TQ = 15$, $QQ_{1} = 2$. Найдите площадь сечения, проходящего через вершины $Q_{1}$, $Q$ и $R$.
\end{taskBN}

\begin{taskBN}{2}
\addpictoright[0.4\textwidth]{images/160755205923462n0}В правильной треугольной пирамиде высота составляет 2; сторона основания составляет 12. Чему равна апофема пирамиды?
\end{taskBN}

\begin{taskBN}{3}
\addpictoright[0.4\textwidth]{images/23549823036985384n0}Площадь боковой поверхности конуса в два раза меньше площади основания. Найдите угол между образующей конуса и плоскостью основания. Ответ дайте в градусах.
\end{taskBN}

\begin{taskBN}{4}
\addpictoright[0.4\textwidth]{images/0577880532766648n0}Площадь большого круга первого шара в 16 раз меньше, чем площадь большого круга второго шара. Во сколько раз радиус первого шара меньше радиуса второго шара?
\end{taskBN}

\begin{taskBN}{5}
\addpictoright[0.4\textwidth]{images/294502562300874n0}Объём цилиндра равен $15552\pi$, площадь боковой поверхности равна $1728\pi$. Найдите площадь полной поверхности цилиндра. Ответ сократите на $\pi$.
\end{taskBN}

\begin{taskBN}{6}
\addpictoright[0.4\textwidth]{images/932780634122585n0}Объём куба равен $397,953\sqrt{3}$. Найдите диагональ куба.
\end{taskBN}
\newpage\section*{Вариант № 3}

\begin{taskBN}{1}
Два ребра прямоугольного параллелепипеда, выходящие из одной вершины, равны 25 и 16. Ребро куба, равновеликого данному параллелепипеду, равно 20. Найдите третье ребро параллелепипеда, выходящее из той же вершины.
\end{taskBN}

\begin{taskBN}{2}
\addpictoright[0.4\textwidth]{images/678626629265761n0}В правильной четырёхугольной пирамиде сторона основания составляет 9; площадь боковой поверхности составляет 135. Чему равна высота пирамиды?
\end{taskBN}

\begin{taskBN}{3}
\addpictoright[0.4\textwidth]{images/9572182791636425n0}Площадь боковой поверхности конуса в два раза больше площади основания. Найдите угол между образующей конуса и плоскостью основания. Ответ дайте в градусах.
\end{taskBN}

\begin{taskBN}{4}
\addpictoright[0.4\textwidth]{images/831728192468953n0}Радиусы двух шаров равны $14$ и $2$. Найдите радиус шара, площадь большого круга которого равна сумме площадей большого круг двух данных шаров. Ответ разделите на $\sqrt{2}$.
\end{taskBN}

\begin{taskBN}{5}
\addpictoright[0.4\textwidth]{images/586961253027308n0}В цилиндрическом сосуде уровень жидкости достигает 88 см. На какой высоте будет находиться уровень жидкости, если её перелить во второй цилиндрический сосуд, радиус которого в 2 раза больше радиуса первого? Ответ выразите в сантиметрах.
\end{taskBN}

\begin{taskBN}{6}
\addpictoright[0.4\textwidth]{images/73861549247152n0}Диагональ куба равна $6,9\sqrt{3}$. Найдите объём куба.
\end{taskBN}
\newpage\section*{Вариант № 4}

\begin{taskBN}{1}
Два ребра прямоугольного параллелепипеда, выходящие из одной вершины, равны 3 и 6. Известно, что третье выходящее из той же вершины ребро составляет 2. Найдите диагональ параллелепипеда.
\end{taskBN}

\begin{taskBN}{2}
\addpictoright[0.4\textwidth]{images/999729413772876n0}Во сколько раз увеличится площадь грани правильного тетраэдра, если все его объём увеличить в 9 раз?
\end{taskBN}

\begin{taskBN}{3}
\addpictoright[0.4\textwidth]{images/92135662053859n0}Высота конуса равна $14$. Плоскость, параллельная плоскости основания конуса,  делит его так, что объёмы конусов равны $1344\pi$ и $10752\pi$. Найдите радиус основания конуса, отсекаемого от данного конуса проведённой плоскостью. 
\end{taskBN}

\begin{taskBN}{4}
\addpictoright[0.4\textwidth]{images/08535060865785n0}Радиусы трёх шаров равны $\sqrt[3]{49}$, $\sqrt[3]{12}$, $\sqrt[3]{3}$. Найдите радиус шара, объем которого равен сумме их объемов.
\end{taskBN}

\begin{taskBN}{5}
\addpictoright[0.4\textwidth]{images/9062977687860962n0}В цилиндрическом сосуде уровень жидкости достигает 8 см. На какой высоте будет находиться уровень жидкости, если её перелить во второй цилиндрический сосуд, диаметр которого в 2 раза меньше диаметра первого? Ответ выразите в сантиметрах.
\end{taskBN}

\begin{taskBN}{6}
\addpictoright[0.4\textwidth]{images/968271256697992n0}Во сколько раз увеличится квадрат диагонали куба, если его ребро увеличится в 15 раз?
\end{taskBN}
\newpage\section*{Вариант № 5}

\begin{taskBN}{1}
Два ребра прямоугольного параллелепипеда, выходящие из одной вершины, равны 64 и 48. Ребро куба, равновеликого данному параллелепипеду, равно 48. Найдите третье ребро параллелепипеда, выходящее из той же вершины.
\end{taskBN}

\begin{taskBN}{2}
\addpictoright[0.4\textwidth]{images/4189453351795691n0}В правильной четырёхугольной пирамиде сторона основания составляет 8; апофема равна 5. Чему равна площадь боковой поверхности пирамиды?
\end{taskBN}

\begin{taskBN}{3}
\addpictoright[0.4\textwidth]{images/895870693696007n0}Площадь боковой поверхности конуса в два раза меньше площади основания. Найдите угол между образующей конуса и плоскостью основания. Ответ дайте в градусах.
\end{taskBN}

\begin{taskBN}{4}
\addpictoright[0.4\textwidth]{images/84801142143021n0}Радиусы трёх шаров равны $\sqrt[3]{15}$, $\sqrt[3]{11}$, $\sqrt[3]{2}$. Найдите радиус шара, объем которого равен сумме их объемов.
\end{taskBN}

\begin{taskBN}{5}
\addpictoright[0.4\textwidth]{images/8065687015638463n0}В цилиндрическом сосуде уровень жидкости достигает 96 см. На какой высоте будет находиться уровень жидкости, если её перелить во второй цилиндрический сосуд, диаметр которого в 4 раза меньше диаметра первого? Ответ выразите в сантиметрах.
\end{taskBN}

\begin{taskBN}{6}
\addpictoright[0.4\textwidth]{images/6475707529057135n0}Eсли ребро куба увеличить на 7, то площадь поверхности увеличится на 798. Найдите площадь поверхности исходного куба.
\end{taskBN}
\newpage\section*{Вариант № 6}

\begin{taskBN}{1}
Два ребра прямоугольного параллелепипеда, выходящие из одной вершины, равны 8 и 8. Известно, что объём составляет 256. Найдите площадь поверхности параллелепипеда.
\end{taskBN}

\begin{taskBN}{2}
\addpictoright[0.4\textwidth]{images/4620132935345507n0}В правильном тетраэдре апофема равна $4\sqrt{3}$. Чему равна площадь полной поверхности тетраэдра? Ответ разделите на $\sqrt{3}$.
\end{taskBN}

\begin{taskBN}{3}
\addpictoright[0.4\textwidth]{images/452279848322535n0}Площадь осевого сечения конуса равна $300$. Плоскость, параллельная плоскости основания конуса,  делит его так, что длины окружностей оснований конусов равны $16\pi$ и $40\pi$. Найдите высоту конуса, отсекаемого от данного конуса проведённой плоскостью. 
\end{taskBN}

\begin{taskBN}{4}
\addpictoright[0.4\textwidth]{images/559272086621903n0}Радиусы трёх шаров равны $\sqrt[3]{52}$, $\sqrt[3]{9}$, $\sqrt[3]{3}$. Найдите радиус шара, объем которого равен сумме их объемов.
\end{taskBN}

\begin{taskBN}{5}
\addpictoright[0.4\textwidth]{images/4046260451614232n0}В цилиндрический сосуд налили $3400\mbox{см}^3$ воды. Уровень воды при этом достигает высоты $425$см. В жидкость полностью погрузили деталь. При этом уровень жидкости в сосуде поднялся на $105$см. Чему равен объем детали? Ответ выразите в $\mbox{см}^3$.
\end{taskBN}

\begin{taskBN}{6}
\addpictoright[0.4\textwidth]{images/1134696050451587n0}Eсли ребро куба уменьшить на 8, то квадрат диагонали уменьшится на 432. Найдите объём исходного куба.
\end{taskBN}
\newpage\section*{Вариант № 7}

\begin{taskBN}{1}
\addpictoright[0.4\textwidth]{images/658682891343357n0}Найдите объём параллелепипеда, изображённого на рисунке
\end{taskBN}
\vspace*{1.3cm}

\begin{taskBN}{2}
\addpictoright[0.4\textwidth]{images/883019443867187n0}В правильной треугольной пирамиде площадь боковой поверхности равна 72; высота равна 2. Чему равна апофема пирамиды?
\end{taskBN}

\begin{taskBN}{3}
\addpictoright[0.4\textwidth]{images/084084865530843n0}Площадь боковой поверхности конуса в два раза больше площади основания. Найдите угол между образующей конуса и плоскостью основания. Ответ дайте в градусах.
\end{taskBN}

\begin{taskBN}{4}
\addpictoright[0.4\textwidth]{images/1156658329533586n0}Радиусы двух шаров равны $5$ и $10$. Найдите радиус шара, площадь большого круга которого равна сумме площадей больших кругов двух данных шаров. Ответ умножьте на $\sqrt{5}$.
\end{taskBN}

\begin{taskBN}{5}
\addpictoright[0.4\textwidth]{images/21939254579562584n0}В цилиндрическом сосуде уровень жидкости достигает 68 см. На какой высоте будет находиться уровень жидкости, если её перелить во второй цилиндрический сосуд, диаметр которого в 2 раза меньше диаметра первого? Ответ выразите в сантиметрах.
\end{taskBN}

\begin{taskBN}{6}
\addpictoright[0.4\textwidth]{images/308732186857008n0}Объём куба составляет 64. Найдите ребро куба.
\end{taskBN}
\newpage\section*{Вариант № 8}

\begin{taskBN}{1}
Два ребра прямоугольного параллелепипеда, выходящие из одной вершины, равны 4 и 10. Ребро куба, равновеликого данному параллелепипеду, равно 10. Найдите третье ребро параллелепипеда, выходящее из той же вершины.
\end{taskBN}

\begin{taskBN}{2}
\addpictoright[0.4\textwidth]{images/442956590538788n0}В правильной шестиугольной пирамиде площадь боковой поверхности равна 252; высота равна 13. Чему равна сторона основания пирамиды?
\end{taskBN}

\begin{taskBN}{3}
\addpictoright[0.4\textwidth]{images/321200406092748n0}Площадь основания конуса в два раза меньше площади боковой поверхности. Найдите угол между образующей конуса и плоскостью основания. Ответ дайте в градусах.
\end{taskBN}

\begin{taskBN}{4}
\addpictoright[0.4\textwidth]{images/411290597836345n0}Радиус первого шара в 8 раз больше, чем радиус второго шара. Во сколько раз площадь большого круга первого шара больше площади большого круга второго шара?
\end{taskBN}

\begin{taskBN}{5}
\addpictoright[0.4\textwidth]{images/033519800587483n0}В цилиндрический сосуд налили $300\mbox{см}^3$ воды. Уровень воды при этом достигает высоты $75$см. В жидкость полностью погрузили деталь. При этом уровень жидкости в сосуде поднялся в ${1}\frac{73}{75}$ раза. Чему равен объем детали? Ответ выразите в $\mbox{см}^3$.
\end{taskBN}

\begin{taskBN}{6}
\addpictoright[0.4\textwidth]{images/391894661352282n0}Квадрат диагонали куба составляет 48. Найдите объём куба.
\end{taskBN}
\newpage\section*{Вариант № 9}

\begin{taskBN}{1}
Два ребра прямоугольного параллелепипеда, выходящие из одной вершины, равны 1 и 4. Известно, что объём составляет 32. Найдите площадь поверхности параллелепипеда.
\end{taskBN}

\begin{taskBN}{2}
\addpictoright[0.4\textwidth]{images/027014714008182n0}В правильной шестиугольной пирамиде апофема равна 13; сторона основания равна 8. Чему равна высота пирамиды?
\end{taskBN}

\begin{taskBN}{3}
\addpictoright[0.4\textwidth]{images/4228392380149602n0}Во сколько раз уменьшили площадь боковой поверхности конуса, если его образующая уменьшилась в 5 раз? При этом длина окружности основания не изменилась.
\end{taskBN}

\begin{taskBN}{4}
\addpictoright[0.4\textwidth]{images/30248975251169874n0}Радиусы двух шаров равны $4$ и $3$. Найдите радиус шара, площадь поверхности которого равна сумме площадей поверхностей двух данных шаров.
\end{taskBN}

\begin{taskBN}{5}
\addpictoright[0.4\textwidth]{images/25548504802893n0}В цилиндрическом сосуде уровень жидкости достигает 36 см. На какой высоте будет находиться уровень жидкости, если её перелить во второй цилиндрический сосуд, радиус которого в 6 раз меньше радиуса первого? Ответ выразите в сантиметрах.
\end{taskBN}

\begin{taskBN}{6}
\addpictoright[0.4\textwidth]{images/2810795068279057n0}Ребро куба составляет 2. Найдите квадрат диагонали куба.
\end{taskBN}
\newpage\section*{Вариант № 10}

\begin{taskBN}{1}
Найдите объём многогранника с вершинами в точках $G_1,C,C_1,Y,D_1,Y_1$ прямоугольного параллелепипеда $DYCGD_1Y_1C_1G_1$, у которого $CG = 6$, $YC = 14$, если $GG_1=9$. 
\end{taskBN}

\begin{taskBN}{2}
\addpictoright[0.4\textwidth]{images/3948859548553396n0}В правильной четырёхугольной пирамиде апофема составляет 19.5; высота составляет 18. Чему равна площадь боковой поверхности пирамиды?
\end{taskBN}

\begin{taskBN}{3}
\addpictoright[0.4\textwidth]{images/794347312186359n0}Площадь осевого сечения конуса равна $300$. Плоскость, параллельная плоскости основания конуса,  делит его так, что радиусы оснований конусов относятся, как $2:5$. Найдите образующую меньшего конуса. 
\end{taskBN}

\begin{taskBN}{4}
\addpictoright[0.4\textwidth]{images/978539723304121n0}Радиусы двух шаров равны $7$ и $1$. Найдите радиус шара, площадь поверхности которого равна сумме площадей поверхностей двух данных шаров. Ответ умножьте на $\sqrt{2}$.
\end{taskBN}

\begin{taskBN}{5}
\addpictoright[0.4\textwidth]{images/578073965496307n0}В цилиндрическом сосуде уровень жидкости достигает 36 см. На какой высоте будет находиться уровень жидкости, если её перелить во второй цилиндрический сосуд, диаметр которого в 6 раз меньше диаметра первого? Ответ выразите в сантиметрах.
\end{taskBN}

\begin{taskBN}{6}
\addpictoright[0.4\textwidth]{images/19302082969487766n0}Объём куба равен $10,125\sqrt{3}$. Найдите диагональ куба.
\end{taskBN}
\newpage\section*{Вариант № 11}

\begin{taskBN}{1}
\addpictoright[0.4\textwidth]{images/100686004190526n0}Найдите площадь поверхности параллелепипеда, изображённого на рисунке
\end{taskBN}
\vspace*{1cm}

\begin{taskBN}{2}
\addpictoright[0.4\textwidth]{images/839798212195793n0}Даны две правильные четырёхугольные пирамиды. Сторона основания первой пирамиды составляет 8. У второй пирамиды объём в 85,5 раз больше, а высота в 9,5 раз больше, чем у первой. Найдите сторону основания второй пирамиды.
\end{taskBN}

\begin{taskBN}{3}
\addpictoright[0.4\textwidth]{images/1732253630343334n0}Площадь боковой поверхности конуса в два раза меньше площади основания. Найдите угол между образующей конуса и плоскостью основания. Ответ дайте в градусах.
\end{taskBN}

\begin{taskBN}{4}
\addpictoright[0.4\textwidth]{images/7464073401458693n0}Радиусы трёх шаров равны $\sqrt[3]{22}$, $\sqrt[3]{3}$, $\sqrt[3]{2}$. Найдите радиус шара, объем которого равен сумме их объемов.
\end{taskBN}

\begin{taskBN}{5}
\addpictoright[0.4\textwidth]{images/280836419795824n0} Первая цилиндрическая кружка в 3 раза выше второй, а вторая в 5 раз шире первой. Найдите отношение объёма первой кружки к объёму второй.
\end{taskBN}

\begin{taskBN}{6}
\addpictoright[0.4\textwidth]{images/387400909410955n0}Объём куба равен $20,577\sqrt{3}$. Найдите диагональ куба.
\end{taskBN}
\newpage\section*{Вариант № 12}

\begin{taskBN}{1}
Два ребра прямоугольного параллелепипеда, выходящие из одной вершины, равны 16 и 49. Ребро куба, равновеликого данному параллелепипеду, равно 28. Найдите третье ребро параллелепипеда, выходящее из той же вершины.
\end{taskBN}

\begin{taskBN}{2}
\addpictoright[0.4\textwidth]{images/993455483787637n0}Даны две правильные четырёхугольные пирамиды. Сторона основания первой пирамиды составляет 3. У второй пирамиды объём в 90 раз больше, а высота в 10 раз больше, чем у первой. Найдите сторону основания второй пирамиды.
\end{taskBN}

\begin{taskBN}{3}
\addpictoright[0.4\textwidth]{images/218806700806139n0}Длина окружности основания конуса равна $88\pi$. Плоскость, параллельная плоскости основания конуса,  делит его так, что объёмы конусов относятся, как $125:1331$. Найдите площадь осевого сечения конуса, отсекаемого от данного конуса проведённой плоскостью. 
\end{taskBN}

\begin{taskBN}{4}
\addpictoright[0.4\textwidth]{images/8778137661928613n0}Радиусы трёх шаров равны $\sqrt[3]{210}$, $\sqrt[3]{5}$, $\sqrt[3]{2}$. Найдите радиус шара, объем которого равен сумме их объемов.
\end{taskBN}

\begin{taskBN}{5}
\addpictoright[0.4\textwidth]{images/987241215488467n0}Объём цилиндра равен $59200\pi$, площадь основания равна $1600\pi$. Найдите площадь полной поверхности цилиндра. Ответ сократите на $\pi$.
\end{taskBN}

\begin{taskBN}{6}
\addpictoright[0.4\textwidth]{images/587028654420799n0}Eсли ребро куба уменьшить на 4, то квадрат диагонали уменьшится на 168. Найдите квадрат диагонали исходного куба.
\end{taskBN}
\newpage\section*{Вариант № 13}

\begin{taskBN}{1}
Три ребра прямоугольного параллелепипеда, выходящие из одной вершины, равны 49, 64 и 56. Найдите ребро равновеликого ему куба.
\end{taskBN}

\begin{taskBN}{2}
\addpictoright[0.4\textwidth]{images/027100289933516n0}Даны две правильные четырёхугольные пирамиды. Сторона основания первой пирамиды составляет 9. У второй пирамиды площадь боковой поверхности в 22 раза больше, а высота в 5,5 раз больше, чем у первой. Найдите сторону основания второй пирамиды.
\end{taskBN}

\begin{taskBN}{3}
\addpictoright[0.4\textwidth]{images/8316531922394756n0}Площадь основания конуса в два раза меньше площади боковой поверхности. Найдите угол между образующей конуса и плоскостью основания. Ответ дайте в градусах.
\end{taskBN}

\begin{taskBN}{4}
\addpictoright[0.4\textwidth]{images/883944259156221n0}Радиусы четырёх шаров равны $\sqrt[3]{318}$, $\sqrt[3]{21}$, $\sqrt[3]{3}$, $1$. Найдите радиус шара, объем которого равен сумме их объемов.
\end{taskBN}

\begin{taskBN}{5}
\addpictoright[0.4\textwidth]{images/4932588706540866n0}В цилиндрическом сосуде уровень жидкости достигает 56 см. На какой высоте будет находиться уровень жидкости, если её перелить во второй цилиндрический сосуд, диаметр которого в 2 раза больше диаметра первого? Ответ выразите в сантиметрах.
\end{taskBN}

\begin{taskBN}{6}
\addpictoright[0.4\textwidth]{images/25526579973932706n0}Диагональ куба равна $0,6\sqrt{3}$. Найдите объём куба.
\end{taskBN}
\newpage\section*{Вариант № 14}

\begin{taskBN}{1}
Два ребра прямоугольного параллелепипеда, выходящие из одной вершины, равны 9 и 6. Ребро куба, равновеликого данному параллелепипеду, равно 6. Найдите третье ребро параллелепипеда, выходящее из той же вершины.
\end{taskBN}

\begin{taskBN}{2}
\addpictoright[0.4\textwidth]{images/725209813946493n0}В правильной четырёхугольной пирамиде площадь боковой поверхности составляет 260; высота составляет 12. Чему равна сторона основания пирамиды?
\end{taskBN}

\begin{taskBN}{3}
\addpictoright[0.4\textwidth]{images/456581269529505n0}Во сколько раз увеличили площадь основания конуса, если его длина окружности основания увеличилась в 9 раз?
\end{taskBN}

\begin{taskBN}{4}
\addpictoright[0.4\textwidth]{images/039789703031094n0}Радиусы трёх шаров равны $\sqrt[3]{12}$, $\sqrt[3]{13}$, $\sqrt[3]{2}$. Найдите радиус шара, объем которого равен сумме их объемов.
\end{taskBN}

\begin{taskBN}{5}
\addpictoright[0.4\textwidth]{images/491636457363429n0} Первая цилиндрическая кружка в 2 раза шире второй, а вторая в 5 раз выше первой. Найдите отношение объёма второй кружки к объёму первой.
\end{taskBN}

\begin{taskBN}{6}
\addpictoright[0.4\textwidth]{images/8450410484073632n0}Eсли ребро куба увеличить на 7, то квадрат диагонали увеличится на 273. Найдите объём исходного куба.
\end{taskBN}
\newpage\section*{Вариант № 15}

\begin{taskBN}{1}
Два ребра прямоугольного параллелепипеда, выходящие из одной вершины, равны 64 и 40. Ребро куба, равновеликого данному параллелепипеду, равно 40. Найдите третье ребро параллелепипеда, выходящее из той же вершины.
\end{taskBN}

\begin{taskBN}{2}
\addpictoright[0.4\textwidth]{images/020882901277926n0}В правильной четырёхугольной пирамиде сторона основания составляет 15; апофема равна 8.5. Чему равна высота пирамиды?
\end{taskBN}

\begin{taskBN}{3}
\addpictoright[0.4\textwidth]{images/235256942611118n0}Высота конуса равна $24$. Плоскость, параллельная плоскости основания конуса,  делит его так, что диаметры оснований конусов относятся, как $1:3$. Найдите площадь осевого сечения меньшего конуса. 
\end{taskBN}

\begin{taskBN}{4}
\addpictoright[0.4\textwidth]{images/57461205188257n0}Во сколько раз радиус первого шара больше радиуса второго шара, если площадь большого круга первого шара в 16 раз больше, чем площадь большого круга второго шара?
\end{taskBN}

\begin{taskBN}{5}
\addpictoright[0.4\textwidth]{images/144293263987201n0}Высота цилиндра равна $28$, площадь полной поверхности равна $7800\pi$. Найдите площадь боковой поверхности цилиндра. Ответ сократите на $\pi$.
\end{taskBN}

\begin{taskBN}{6}
\addpictoright[0.4\textwidth]{images/969459139921462n0}Объём куба равен $2187\sqrt{3}$. Найдите диагональ куба.
\end{taskBN}
\newpage\section*{Вариант № 16}

\begin{taskBN}{1}
Два ребра прямоугольного параллелепипеда, выходящие из одной вершины, равны 16 и 36. Ребро куба, равновеликого данному параллелепипеду, равно 36. Найдите третье ребро параллелепипеда, выходящее из той же вершины.
\end{taskBN}

\begin{taskBN}{2}
\addpictoright[0.4\textwidth]{images/1864853529856865n0}В правильной четырёхугольной пирамиде сторона основания равна 8; площадь боковой поверхности равна 80. Чему равна апофема пирамиды?
\end{taskBN}

\begin{taskBN}{3}
\addpictoright[0.4\textwidth]{images/0555065017414356n0}Площадь основания конуса в два раза больше площади боковой поверхности. Найдите угол между образующей конуса и плоскостью основания. Ответ дайте в градусах.
\end{taskBN}

\begin{taskBN}{4}
\addpictoright[0.4\textwidth]{images/9885426228487395n0}Радиусы трёх шаров равны $\sqrt[3]{89}$, $\sqrt[3]{34}$, $\sqrt[3]{2}$. Найдите радиус шара, объем которого равен сумме их объемов.
\end{taskBN}

\begin{taskBN}{5}
\addpictoright[0.4\textwidth]{images/769256825381532n0} Первая цилиндрическая кружка в 8,5 раз шире второй, а вторая в 5 раз выше первой. Найдите отношение объёма первой кружки к объёму второй.
\end{taskBN}

\begin{taskBN}{6}
\addpictoright[0.4\textwidth]{images/538129695270728n0}Eсли ребро куба уменьшить на 3, то квадрат диагонали уменьшится на 171. Найдите квадрат диагонали исходного куба.
\end{taskBN}
\newpage\section*{Вариант № 17}

\begin{taskBN}{1}
\addpictoright[0.4\textwidth]{images/190373330997235n0}Найдите площадь поверхности параллелепипеда, изображённого на рисунке
\end{taskBN}

\begin{taskBN}{2}
\addpictoright[0.4\textwidth]{images/2558826014183637n0}Во сколько раз увеличится площадь боковой поверхности правильного тетраэдра, если все его объём увеличить в 5 раз?
\end{taskBN}

\begin{taskBN}{3}
\addpictoright[0.4\textwidth]{images/483603450087289n0}В сосуде, имеющем форму конуса, уровень жидкости достигает $\frac{1}{5}$ высоты. Объём жидкости равен 3мл. Сколько миллилитров жидкости поместится в весь сосуд?
\end{taskBN}

\begin{taskBN}{4}
\addpictoright[0.4\textwidth]{images/220619770002915n0}Радиусы четырёх шаров равны $\sqrt[3]{19}$, $\sqrt[3]{4}$, $\sqrt[3]{3}$, $\sqrt[3]{2}$. Найдите радиус шара, объем которого равен сумме их объемов.
\end{taskBN}

\begin{taskBN}{5}
\addpictoright[0.4\textwidth]{images/4950184225686745n0} Первая цилиндрическая кружка в 3 раза выше второй, а вторая в 2,5 раза шире первой. Найдите отношение объёма первой кружки к объёму второй.
\end{taskBN}

\begin{taskBN}{6}
\addpictoright[0.4\textwidth]{images/962194509627077n0}Во сколько раз увеличится диагональ куба, если его ребро увеличится в 19,5 раз?
\end{taskBN}
\newpage\section*{Вариант № 18}

\begin{taskBN}{1}
Два ребра прямоугольного параллелепипеда, выходящие из одной вершины, равны 25 и 9. Ребро куба, равновеликого данному параллелепипеду, равно 15. Найдите третье ребро параллелепипеда, выходящее из той же вершины.
\end{taskBN}

\begin{taskBN}{2}
\addpictoright[0.4\textwidth]{images/083913920323999n0}В правильном тетраэдре высота равна $2\sqrt{6}$. Чему равна площадь полной поверхности тетраэдра? Ответ разделите на $\sqrt{3}$.
\end{taskBN}

\begin{taskBN}{3}
\addpictoright[0.4\textwidth]{images/9765577041095357n0}Во сколько раз увеличили объём конуса, если его длина окружности основания увеличилась в 2 раза? При этом высота не изменилась.
\end{taskBN}

\begin{taskBN}{4}
\addpictoright[0.4\textwidth]{images/867641702942638n0}Радиусы трёх шаров равны $\sqrt[3]{22}$, $\sqrt[3]{4}$, $1$. Найдите радиус шара, объем которого равен сумме их объемов.
\end{taskBN}

\begin{taskBN}{5}
\addpictoright[0.4\textwidth]{images/916539913476407n0}Длина окружности основания цилиндра равна $54\pi$, объём равен $21141\pi$. Найдите площадь боковой поверхности цилиндра. Ответ сократите на $\pi$.
\end{taskBN}

\begin{taskBN}{6}
\addpictoright[0.4\textwidth]{images/93909014946527n0}Во сколько раз увеличится диагональ куба, если его площадь поверхности увеличится в 9 раз?
\end{taskBN}
\newpage\section*{Вариант № 19}

\begin{taskBN}{1}
\addpictoright[0.4\textwidth]{images/385274347836138n0}Найдите площадь поверхности параллелепипеда, изображённого на рисунке
\end{taskBN}

\begin{taskBN}{2}
\addpictoright[0.4\textwidth]{images/9535673685622164n0}В правильном тетраэдре площадь боковой поверхности равна $12\sqrt{3}$. Чему равна площадь основания тетраэдра? Ответ умножьте на $\sqrt{3}$.
\end{taskBN}

\begin{taskBN}{3}
\addpictoright[0.4\textwidth]{images/736454125097876n0}В сосуде, имеющем форму конуса, уровень жидкости достигает $\frac{2}{5}$ высоты. Объём жидкости равен 448мл. Сколько миллилитров жидкости поместится в весь сосуд?
\end{taskBN}

\begin{taskBN}{4}
\addpictoright[0.4\textwidth]{images/6494339705699974n0}Радиусы двух шаров равны $10$ и $5$. Найдите радиус шара, площадь поверхности которого равна сумме площадей поверхностей двух данных шаров. Ответ разделите на $\sqrt{5}$.
\end{taskBN}

\begin{taskBN}{5}
\addpictoright[0.4\textwidth]{images/056390263801122n0}В цилиндрическом сосуде уровень жидкости достигает 72 см. На какой высоте будет находиться уровень жидкости, если её перелить во второй цилиндрический сосуд, радиус которого в 6 раз меньше радиуса первого? Ответ выразите в сантиметрах.
\end{taskBN}

\begin{taskBN}{6}
\addpictoright[0.4\textwidth]{images/012992037174403n0}Во сколько раз увеличится площадь поверхности куба, если его квадрат диагонали увеличится в 17 раз?
\end{taskBN}
\newpage\section*{Вариант № 20}

\begin{taskBN}{1}
Три ребра прямоугольного параллелепипеда, выходящие из одной вершины, равны 16, 25 и 20. Найдите ребро равновеликого ему куба.
\end{taskBN}

\begin{taskBN}{2}
\addpictoright[0.4\textwidth]{images/7462567763615402n0}В правильной шестиугольной пирамиде апофема составляет 6; площадь боковой поверхности составляет 108. Чему равна сторона основания пирамиды?
\end{taskBN}

\begin{taskBN}{3}
\addpictoright[0.4\textwidth]{images/316402790836639n0}Площадь основания конуса в два раза меньше площади боковой поверхности. Найдите угол между образующей конуса и плоскостью основания. Ответ дайте в градусах.
\end{taskBN}

\begin{taskBN}{4}
\addpictoright[0.4\textwidth]{images/555075282944786n0}Радиусы двух шаров равны $6$ и $8$. Найдите радиус шара, площадь поверхности которого равна сумме площадей поверхностей двух данных шаров.
\end{taskBN}

\begin{taskBN}{5}
\addpictoright[0.4\textwidth]{images/000317117009005n0}В цилиндрическом сосуде уровень жидкости достигает 12 см. На какой высоте будет находиться уровень жидкости, если её перелить во второй цилиндрический сосуд, диаметр которого в 2 раза больше диаметра первого? Ответ выразите в сантиметрах.
\end{taskBN}

\begin{taskBN}{6}
\addpictoright[0.4\textwidth]{images/5614330801046226n0}Eсли ребро куба уменьшить на 4, то площадь поверхности уменьшится на 288. Найдите площадь поверхности исходного куба.
\end{taskBN}
\newpage\section*{Вариант № 21}

\begin{taskBN}{1}
Три ребра прямоугольного параллелепипеда, выходящие из одной вершины, равны 40, 64 и 25. Найдите ребро равновеликого ему куба.
\end{taskBN}

\begin{taskBN}{2}
\addpictoright[0.4\textwidth]{images/675008818005209n0}Даны две правильные четырёхугольные пирамиды. Высота первой пирамиды составляет 9. У второй пирамиды сторона основания в 7,5 раз больше, а объём в 168,75 раз больше, чем у первой. Найдите высоту второй пирамиды.
\end{taskBN}

\begin{taskBN}{3}
\addpictoright[0.4\textwidth]{images/359136080953532n0}Высота конуса равна $36$. Плоскость, параллельная плоскости основания конуса,  делит его так, что образующие конусов относятся, как $1:4$. Найдите радиус основания меньшего конуса. 
\end{taskBN}

\begin{taskBN}{4}
\addpictoright[0.4\textwidth]{images/670554099278913n0}Во сколько раз уменьшили площадь большого круга шара, если его объём уменьшился в 729 раз?
\end{taskBN}

\begin{taskBN}{5}
\addpictoright[0.4\textwidth]{images/463892743944151n0}Площадь основания цилиндра равна $9\pi$, площадь полной поверхности равна $294\pi$. Найдите площадь боковой поверхности цилиндра. Ответ сократите на $\pi$.
\end{taskBN}

\begin{taskBN}{6}
\addpictoright[0.4\textwidth]{images/54244106429051864n0}Во сколько раз увеличится квадрат диагонали куба, если его ребро увеличится в 3 раза?
\end{taskBN}
\newpage\section*{Вариант № 22}

\begin{taskBN}{1}
\addpictoright[0.4\textwidth]{images/12284821245560962n0}Найдите площадь основания параллелепипеда, изображённого на рисунке
\end{taskBN}

\begin{taskBN}{2}
\addpictoright[0.4\textwidth]{images/93438940637665n0}В правильном тетраэдре апофема равна $3\sqrt{3}$. Чему равна площадь основания тетраэдра? Ответ разделите на $\sqrt{3}$.
\end{taskBN}

\begin{taskBN}{3}
\addpictoright[0.4\textwidth]{images/6263955063931581n0}Высота конуса равна $18$. Плоскость, параллельная плоскости основания конуса,  делит его так, что объёмы конусов относятся, как $1:8$. Найдите площадь осевого сечения конуса, отсекаемого от данного конуса проведённой плоскостью. 
\end{taskBN}

\begin{taskBN}{4}
\addpictoright[0.4\textwidth]{images/9919397844127056n0}Радиусы трёх шаров равны $\sqrt[3]{100}$, $\sqrt[3]{23}$, $\sqrt[3]{2}$. Найдите радиус шара, объем которого равен сумме их объемов.
\end{taskBN}

\begin{taskBN}{5}
\addpictoright[0.4\textwidth]{images/5505580174750166n0}Объём цилиндра равен $1444\pi$, площадь боковой поверхности равна $76\pi$. Найдите длину окружности основания цилиндра. Ответ сократите на $\pi$.
\end{taskBN}

\begin{taskBN}{6}
\addpictoright[0.4\textwidth]{images/996621880850787n0}Eсли ребро куба уменьшить на 3, то площадь поверхности уменьшится на 162. Найдите объём исходного куба.
\end{taskBN}
\newpage\section*{Вариант № 23}

\begin{taskBN}{1}
Два ребра прямоугольного параллелепипеда, выходящие из одной вершины, равны 6 и 9. Известно, что квадрат диагонали составляет 166. Найдите объём параллелепипеда.
\end{taskBN}

\begin{taskBN}{2}
\addpictoright[0.4\textwidth]{images/818590792695314n0}Даны две правильные четырёхугольные пирамиды. Сторона основания первой пирамиды составляет 9. У второй пирамиды площадь боковой поверхности в 30,25 раз больше, а высота в 5,5 раз больше, чем у первой. Найдите сторону основания второй пирамиды.
\end{taskBN}

\begin{taskBN}{3}
\addpictoright[0.4\textwidth]{images/515654792333638n0}В сосуде, имеющем форму конуса, уровень жидкости достигает $\frac{1}{2}$ высоты. Объём жидкости равен 64мл. Сколько миллилитров жидкости нужно долить, чтобы наполнить сосуд доверху?
\end{taskBN}

\begin{taskBN}{4}
\addpictoright[0.4\textwidth]{images/620634421277477n0}Радиусы двух шаров равны $7$ и $1$. Найдите радиус шара, площадь поверхности которого равна сумме площадей поверхностей двух данных шаров. Ответ умножьте на $\sqrt{2}$.
\end{taskBN}

\begin{taskBN}{5}
\addpictoright[0.4\textwidth]{images/3334140960137215n0} Первая цилиндрическая кружка в 5 раз шире второй, а вторая в 7,5 раз выше первой. Найдите отношение объёма второй кружки к объёму первой.
\end{taskBN}

\begin{taskBN}{6}
\addpictoright[0.4\textwidth]{images/6655547368661603n0}Объём куба составляет 125. Найдите квадрат диагонали куба.
\end{taskBN}
\newpage\section*{Вариант № 24}

\begin{taskBN}{1}
\addpictoright[0.4\textwidth]{images/535535185394258n0}Найдите площадь основания параллелепипеда, изображённого на рисунке
\end{taskBN}
\vspace*{1cm}

\begin{taskBN}{2}
\addpictoright[0.4\textwidth]{images/821610187728548n0}В правильной четырёхугольной пирамиде сторона основания равна 12; высота составляет 8. Чему равна площадь боковой поверхности пирамиды?
\end{taskBN}

\begin{taskBN}{3}
\addpictoright[0.4\textwidth]{images/182804806993736n0}Площадь боковой поверхности конуса равна $580\pi$, длина окружности основания равна $40\pi$. Найдите образующую конуса. 
\end{taskBN}

\begin{taskBN}{4}
\addpictoright[0.4\textwidth]{images/144563999844687n0}Радиусы четырёх шаров равны $\sqrt[3]{51}$, $\sqrt[3]{142}$, $\sqrt[3]{149}$, $\sqrt[3]{2}$. Найдите радиус шара, объем которого равен сумме их объемов.
\end{taskBN}

\begin{taskBN}{5}
\addpictoright[0.4\textwidth]{images/04536973813498n0}В цилиндрическом сосуде уровень жидкости достигает 8 см. На какой высоте будет находиться уровень жидкости, если её перелить во второй цилиндрический сосуд, радиус которого в 2 раза больше радиуса первого? Ответ выразите в сантиметрах.
\end{taskBN}

\begin{taskBN}{6}
\addpictoright[0.4\textwidth]{images/0776838468450722n0}Во сколько раз увеличится квадрат диагонали куба, если его ребро увеличится в 4,5 раз?
\end{taskBN}
\newpage\section*{Вариант № 25}

\begin{taskBN}{1}
Сколько составляет  $ZI$, если $ZZ_1=3$, при этом объём многогранника, вершинами которого являются точки $Z_1,D_1,D,Z,C$ прямоугольного параллелепипеда $DZICD_1Z_1I_1C_1$, у которого $IC = 12$, равен 144? 
\end{taskBN}

\begin{taskBN}{2}
\addpictoright[0.4\textwidth]{images/9808172470141918n0}Во сколько раз увеличится объём правильного тетраэдра, если все его площадь грани увеличить в 7 раз?
\end{taskBN}

\begin{taskBN}{3}
\addpictoright[0.4\textwidth]{images/152687813875625n0}Площадь основания конуса в два раза больше площади боковой поверхности. Найдите угол между образующей конуса и плоскостью основания. Ответ дайте в градусах.
\end{taskBN}

\begin{taskBN}{4}
\addpictoright[0.4\textwidth]{images/862997171734132n0}Радиусы двух шаров равны $12$ и $16$. Найдите радиус шара, площадь большого круга которого равна сумме площадей больших кругов двух данных шаров.
\end{taskBN}

\begin{taskBN}{5}
\addpictoright[0.4\textwidth]{images/434696727929457n0}Объём цилиндра равен $35280\pi$, площадь боковой поверхности равна $2520\pi$. Найдите площадь полной поверхности цилиндра. Ответ сократите на $\pi$.
\end{taskBN}

\begin{taskBN}{6}
\addpictoright[0.4\textwidth]{images/311745482699317n0}Диагональ куба равна $2,4\sqrt{3}$. Найдите объём куба.
\end{taskBN}
\newpage\section*{Вариант № 26}

\begin{taskBN}{1}
В прямоугольном параллелепипеде  $NAPFN_{1}A_{1}P_{1}F_{1}$  известно, что  $NA = 8$, $N_{1}F_{1} = 15$, $AA_{1} = 12$. Найдите площадь сечения, проходящего через вершины $N_{1}$, $P_{1}$ и $N$.
\end{taskBN}

\begin{taskBN}{2}
\addpictoright[0.4\textwidth]{images/564702599919252n0}Во сколько раз увеличится объём правильного тетраэдра, если все его площадь боковой поверхности увеличить в 7 раз?
\end{taskBN}

\begin{taskBN}{3}
\addpictoright[0.4\textwidth]{images/123388155236377n0}Во сколько раз уменьшили площадь боковой поверхности конуса, если его радиус основания уменьшился в 8 раз? При этом образующая не изменилась.
\end{taskBN}

\begin{taskBN}{4}
\addpictoright[0.4\textwidth]{images/51846394330687495n0}Площадь поверхности первого шара в 9 раз больше, чем площадь поверхности второго шара. Во сколько раз площадь большого круга первого шара больше площади большого круга второго шара?
\end{taskBN}

\begin{taskBN}{5}
\addpictoright[0.4\textwidth]{images/207547464385075n0} Первая цилиндрическая кружка в 2,5 раза выше второй, а вторая в 4,5 раз шире первой. Найдите отношение объёма второй кружки к объёму первой.
\end{taskBN}

\begin{taskBN}{6}
\addpictoright[0.4\textwidth]{images/782649907475517n0}Eсли ребро куба увеличить на 6, то площадь поверхности увеличится на 432. Найдите площадь поверхности исходного куба.
\end{taskBN}
\newpage\section*{Вариант № 27}

\begin{taskBN}{1}
Чему равна  $I_1Y_1$, если $SS_1=6$? Известно, что  $IS=6$. Объём многогранника с вершинами в точках $S,Y,I,I_1,Y_1$ прямоугольного параллелепипеда $IYUSI_1Y_1U_1S_1$ составляет 108. 
\end{taskBN}

\begin{taskBN}{2}
\addpictoright[0.4\textwidth]{images/47799651122262n0}В правильной четырёхугольной пирамиде площадь боковой поверхности равна 240; сторона основания составляет 12. Чему равна высота пирамиды?
\end{taskBN}

\begin{taskBN}{3}
\addpictoright[0.4\textwidth]{images/986209924169948n0}Площадь основания конуса в два раза больше площади боковой поверхности. Найдите угол между образующей конуса и плоскостью основания. Ответ дайте в градусах.
\end{taskBN}

\begin{taskBN}{4}
\addpictoright[0.4\textwidth]{images/51951496581508n0}Радиусы двух шаров равны $4$ и $8$. Найдите радиус шара, площадь большого круга которого равна сумме площадей больших кругов двух данных шаров. Ответ разделите на $\sqrt{5}$.
\end{taskBN}

\begin{taskBN}{5}
\addpictoright[0.4\textwidth]{images/679672624615342n0}В цилиндрический сосуд налили $700\mbox{м}^3$ воды. Уровень воды при этом достигает высоты $14$м. В жидкость полностью погрузили деталь. При этом уровень жидкости в сосуде поднялся на $9$м. Чему равен объем детали? Ответ выразите в $\mbox{м}^3$.
\end{taskBN}

\begin{taskBN}{6}
\addpictoright[0.4\textwidth]{images/5435272023411784n0}Объём куба равен $472,392\sqrt{3}$. Найдите диагональ куба.
\end{taskBN}
\newpage\section*{Вариант № 28}

\begin{taskBN}{1}
Два ребра прямоугольного параллелепипеда, выходящие из одной вершины, равны 4 и 16. Ребро куба, равновеликого данному параллелепипеду, равно 16. Найдите третье ребро параллелепипеда, выходящее из той же вершины.
\end{taskBN}

\begin{taskBN}{2}
\addpictoright[0.4\textwidth]{images/355682747245434n0}Во сколько раз увеличится площадь грани правильного тетраэдра, если все его объём увеличить в 5 раз?
\end{taskBN}

\begin{taskBN}{3}
\addpictoright[0.4\textwidth]{images/887763658468549n0}Высота конуса равна $12$. Плоскость, параллельная плоскости основания конуса,  делит его так, что образующие конусов относятся, как $1:2$. Найдите площадь осевого сечения меньшего конуса. 
\end{taskBN}

\begin{taskBN}{4}
\addpictoright[0.4\textwidth]{images/687916274990708n0}Радиусы трёх шаров равны $\sqrt[3]{57}$, $\sqrt[3]{3}$, $\sqrt[3]{4}$. Найдите радиус шара, объем которого равен сумме их объемов.
\end{taskBN}

\begin{taskBN}{5}
\addpictoright[0.4\textwidth]{images/5914083126345537n0}Площадь полной поверхности цилиндра равна $3780\pi$, площадь боковой поверхности равна $2322\pi$. Найдите объём цилиндра. Ответ сократите на $\pi$.
\end{taskBN}

\begin{taskBN}{6}
\addpictoright[0.4\textwidth]{images/240821746997458n0}Во сколько раз увеличится квадрат диагонали куба, если его ребро увеличится в 9,5 раз?
\end{taskBN}
\newpage\section*{Вариант № 29}

\begin{taskBN}{1}
Чему равен объём многогранника, вершинами которого являются точки $C_1,C,U,M_1,D,M$ прямоугольного параллелепипеда $MDUCM_1D_1U_1C_1$, у которого $U_1C_1 = 6$, $D_1U_1 = 11$, если $MM_1=4$? 
\end{taskBN}

\begin{taskBN}{2}
\addpictoright[0.4\textwidth]{images/413480805874306n0}В правильной четырёхугольной пирамиде высота равна 4; апофема составляет 5. Чему равна площадь боковой поверхности пирамиды?
\end{taskBN}

\begin{taskBN}{3}
\addpictoright[0.4\textwidth]{images/8506714830719586n0}Площадь боковой поверхности конуса в два раза меньше площади основания. Найдите угол между образующей конуса и плоскостью основания. Ответ дайте в градусах.
\end{taskBN}

\begin{taskBN}{4}
\addpictoright[0.4\textwidth]{images/1232429205258236n0}Радиусы четырёх шаров равны $\sqrt[3]{2}$, $\sqrt[3]{19}$, $\sqrt[3]{42}$, $1$. Найдите радиус шара, объем которого равен сумме их объемов.
\end{taskBN}

\begin{taskBN}{5}
\addpictoright[0.4\textwidth]{images/895541034279967n0}Длина окружности основания цилиндра равна $98\pi$, площадь боковой поверхности равна $588\pi$. Найдите объём цилиндра. Ответ сократите на $\pi$.
\end{taskBN}

\begin{taskBN}{6}
\addpictoright[0.4\textwidth]{images/530277204619659n0}Eсли ребро куба увеличить на 7, то квадрат диагонали увеличится на 441. Найдите квадрат диагонали исходного куба.
\end{taskBN}
\newpage\section*{Вариант № 30}

\begin{taskBN}{1}
Чему равен объём многогранника с вершинами в точках $U_1,N,N_1,Z_1,Z,X_1$ прямоугольного параллелепипеда $ZNUXZ_1N_1U_1X_1$, у которого $U_1X_1 = 9$, $ZX = 12$, если $NN_1=2$? 
\end{taskBN}

\begin{taskBN}{2}
\addpictoright[0.4\textwidth]{images/9468920570618646n0}В правильном тетраэдре площадь боковой поверхности равна $27\sqrt{3}$. Чему равен объём тетраэдра? Ответ разделите на $\sqrt{2}$.
\end{taskBN}

\begin{taskBN}{3}
\addpictoright[0.4\textwidth]{images/3262962214018497n0}Во сколько раз уменьшили площадь основания конуса, если его длина окружности основания уменьшилась в 8 раз?
\end{taskBN}

\begin{taskBN}{4}
\addpictoright[0.4\textwidth]{images/899842581393574n0}Объём первого шара в 512 раз меньше, чем объём второго шара. Во сколько раз площадь большого круга первого шара меньше площади большого круга второго шара?
\end{taskBN}

\begin{taskBN}{5}
\addpictoright[0.4\textwidth]{images/074568981337698n0}В цилиндрический сосуд налили $4100\mbox{м}^3$ воды. Уровень воды при этом достигает высоты $100$м. В жидкость полностью погрузили деталь. При этом уровень жидкости в сосуде поднялся в $ 1{,}68 $ раза. Чему равен объем детали? Ответ выразите в $\mbox{м}^3$.
\end{taskBN}

\begin{taskBN}{6}
\addpictoright[0.4\textwidth]{images/976478723976701n0}Площадь поверхности куба составляет 24. Найдите квадрат диагонали куба.
\end{taskBN}
\end{document}