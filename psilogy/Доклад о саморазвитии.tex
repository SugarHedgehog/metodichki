\documentclass[a4paper, 12pt]{report}
\usepackage{fontspec}
\usepackage{amsmath}
\usepackage{polyglossia}
\setmainfont{CMU Serif}
\newfontfamily{\cyrillicfont}{CMU Serif}
\setsansfont{CMU Sans Serif}
\newfontfamily{\cyrillicfontsf}{CMU Sans Serif}
\setmonofont{CMU Typewriter Text}
\newfontfamily{\cyrillicfonttt}{CMU Typewriter Text}
\setdefaultlanguage{russian}
\usepackage[left=1cm,right=1cm,
top=1.5cm,bottom=1.5cm]{geometry}
\usepackage{enumitem}
\begin{document}
\renewcommand{\labelenumi}{\textbf{\arabic{enumi}.}}
\renewcommand{\labelenumii}{\textbf{\arabic{enumi}.\arabic{enumii}}}
\renewcommand{\labelenumiii}{\arabic{enumi}.\arabic{enumii}.\arabic{enumiii}}
\renewcommand{\labelenumiv}{\arabic{enumi}.\arabic{enumii}.\arabic{enumiii}.\arabic{enumiv}}
    \subsection*{Понятие психологического консультирования}  
Психологическое консультирование – прикладная отрасль современной
психологии. Ее задачей является разработка теоретических основ и
прикладных программ оказания психологической помощи психически и
соматически здоровым людям в ситуациях, когда они сталкиваются со
своими проблемами. 
    \subsection*{Цели и задачи психологического консультирования}
Существуют разные взгляды на цели и задачи консультирования. 
Но самой понятной и очевидной целью консультирования яыляется 
оказание психологической помощи, то есть разговор с психологом 
должен помочь человеку в решении его
проблем и налаживании взаимоотношений с окружающими.

В определении задачи всё более однозначно. Задачами считаются:

    \begin{itemize}[noitemsep]
        \item Выслушивание клиента.
        \item Облегчение эмоционального состояния клиента.
        \item Принятие клиентом ответственности за происходящее с ним.
        \item Помощь психолога в определении того, что именно и как можно 
        изменить в ситуации.
    \end{itemize}
     
    \subsection*{Первая встреча с клиентом}
Первое знакомство с психологом должно проходить в расслабленной и спокойной 
обстановке. Этому может способствовать:
\begin{itemize}[noitemsep]
    \item Можно встать навстречу клиенту или встретить его в дверях кабинета, что
    будет восприниматься клиентом как демонстрация доброжелательности и
    заинтересованности.
    \item Желательно подбадривать клиента словами типа «Проходите,
    пожалуйста», «Усаживайтесь поудобнее».
    \item После первых минут контакта с клиентом рекомендуется предоставить
    ему паузу 45 – 60 секунд, чтобы клиент мог собраться с мыслями и
    оглядеться.
    \item После паузы желательно начать собственно знакомство. Можно сказать
    клиенту: «Давайте познакомимся. Как мне Вас называть?» После этого
    необходимо представиться клиенту. Представляться лучше всего так, как
    представился клиент. Можно обсудить, удобно ли будет клиенту, если его
    будут называть таким образом.
    \item Далее предоставить полную информацию о последующей работе(методики, 
    продолжительности консультаций, и так далее)
    \item После всего можно переходить к опросу клиента
\end{itemize}
    \subsection*{Создание обоюдного доверия}
Создавать обоюдное доверие - значит создавать терапевтический климат, 
способствующий обсуждению значимых проблем. Успешность создания 
терапевтического климата обусловлена личностными 
качествами консультанта и его отношением к консультированию: - необходим 
искренний интерес к людям и подлинное желание помогать им.
    \subsection*{Продолжительность психологического консультирования}
Обычно консультативная беседа со взрослым человеком длится от 50
минут до одного часа. Психотерапия и консультирование требуют
концентрации внимания и бдительности, а, как известно, концентрацию
внимания трудно сохранять более 45 — 50 минут. Традиционные 50 минут,
регламентирующие консультативную встречу, позволяют продуктивно
обсудить несколько вопросов, а затем 10 минут уделить записи основных
аспектов прошедшей беседы или просто отдыху за чашечкой кофе.
Продолжительность беседы может изменяться в зависимости от
возраста клиента. 
    \subsection*{Структурирование времени консультирования}
    
    \subsection*{Психологический анамнез}
Психологический амнамнез (от греч. воспоминание) — совокупность 
сведений о человеке, полученных различными методами с целью 
организации с ним эффективной работы. 
Оно свободно от обязательной связи с процессом болезни и употребляется 
как синоним понятия "история индивидуального развития человека". 

    \begin{enumerate}
        \setlength\itemsep{0pt}
        \item Демографические данные
        \begin{enumerate}
        \setlength\itemsep{-0.1mm}
            \item возраст клиента \item семейное положение
            \item профессия \item образование
        \end{enumerate} 
        \item Актуальные проблемы и нарушения
        \begin{enumerate}
        
        \setlength\itemsep{0pt}
            \item возникновение, развитие и 
            продолжительность затруднений 
            \item события в жизни, обусловленные 
            возникновением, обострением и разрешением проблем 
            \item возраст, в котором 
            возникли проблемы 
            \item изменение отношений личности (особенно к значимым 
            людям), перемена интересов, ухудшение физического состояния (сон, аппетит), 
            обусловленные возникновением проблем 
            \item непосредственная причина обращения
            клиента 
            \item предшествующие попытки разрешения проблем (самостоятельно или 
            с помощью других специалистов) и результаты 
            \item употребление лекарств 
            \item семейный анамнез (особенно психические болезни, алкоголизм, наркомания, 
            самоубийства).
        \end{enumerate} 
        \item Психосоциальный анамнез (значимые межличностные отношения): 
        \begin{enumerate}
        \setlength\itemsep{0pt}
            \item раннее детство (обстоятельства и очередность рождения, основные 
            воспитатели, отношения в семье)
            \item дошкольный период (рождение братьев и
            сестер, другие значительные события в семье, первые воспоминания)
            \item младший школьный возраст (успехи и неудачи в учебе, проблемы с учителями 
            и ровесниками в школе, отношения в семье)
            \item отрочество и юность 
            (отношения с ровесниками, лицами другого пола, родителями, успехи и 
            неудачи в школе, идеалы и устремления)
            \item взрослый возраст (социальные 
            отношения, удовлетворенность работой, браком, отношения в семье, половая 
            жизнь, экономические условия жизни, утрата близких людей, возрастные 
            изменения, употребление алкоголя, наркотиков, психологические и 
            экзистенциальные кризисы, планы на будущее). 
        \end{enumerate}
    \end{enumerate}
    \subsection*{Использование тестов во время консультирования}
    Время от времени в процессе проведения психологического 
    консультирования возникает срочная необходимость психологического 
    тестирования клиента. Когда, при каких обстоятельствах появляется 
    необходимость в этом?
    \begin{enumerate}

        \item Во-первых, тогда, когда у психолога-консультанта не хватает сведений для того, чтобы сделать правильные выводы о сути проблемы клиента и с учетом его индивидуальности предложить действенные меры по практическому решению возникшей проблемы. В этом случае необходимо бывает разносторонне изучить личность клиента, с тем, чтобы узнать о тех его индивидуальных особенностях, которые важны для прояснения проблемы и для точного определения путей ее решения.
        Нередко на основе одной и особенно первой встречи с клиентом 
        трудно бывает судить о нем как о личности, тем более что в 
        психологической консультации в незнакомой обстановке человек 
        ведет себя весьма сдержанно и играет обычно не вполне свойственную 
        ему жизненную роль, не отображающую полностью его индивидуальность.
        
        \item Во-вторых, тогда, когда необходимо оценить такие индивидуальные 
        особенности клиента, которые и в обычной, повседневной жизни слабо 
        или почти не проявляются в поведении человека.
                  
        \item В-третьих, тогда, когда психологу-консультанту необходимо 
        точно установить, какое влияние на данного клиента оказало или 
        может оказать посещение психологической консультации, получение 
        и выполнение им рекомендаций психолога консультанта.
                    
        \item В-четвертых, тогда, когда в практике консультирования 
        применяется новый, достаточно не проверенный метод, эффективность 
        которого окончательно не установлена, не оценена ни количественно, 
        ни качественно.
                
        \item В-пятых, тогда, когда сама по себе процедура 
        психодиагностики, применяемая в практике психологического 
        консультирования, имеет известное психотерапевтическое значение, 
        оказывая благотворное влияние на клиента.
                  
        \end{enumerate}

    \subsection*{Процедуры и техники психологического консультирования}
    Под процедурами в психологическом консультировании принято понимать 
    группы приемов ведения специалистом клиента, объединенные по целевому 
    назначению, с помощью которых решается одна из задач данного этапа.

    Каждой процедуре соответствуют определенные техники — специальные 
    приемы, применяемые консультантом для решения задач процедур на 
    каждом из этапов психологического консультирования.
    
    Рассмотрим процедуры и техники во взаимосвязи с этапами 
    психологического консультирования.
    
    Этап первый
    
    Как мы определили, первым этапом в психологическом консультировании 
    будет этап, который условно можно назвать «начало работы».
    
    Первой процедурой на этом этапе, безусловно, является встреча клиента 
    с консультантом. Техники, уместные для решения данной задачи: 
    приветствие клиента, проведение клиента на место, выбор клиентом 
    своего места, выбор психологом места для себя, приемы установления 
    психологического контакта.    
   
    Техника приветствия. Осуществляется при помощи стандартных фраз: 
    «Рад вас видеть...», «Приятно познакомиться...». Составным элементом 
    данной техники, который может быть применен специалистом по его 
    усмотрению, является встреча специалистом клиента у входа в 
    консультацию или в фойе, перед кабинетом.
    
    Техника «проведение клиента на место». Особенно уместна в случаях 
    первичного посещения клиентом психологической консультации: психолог 
    идет впереди клиента, указывая ему дорогу и пропуская вперед себя при 
    входе в кабинет. В данном моменте техника «проведение клиента 
    на место» тесно соприкасается с техниками «выбор клиентом своего 
    места» и «выбор своего места психологом-консультантом». 
    Предоставление права выбора клиенту места носит диагностический 
    характер, поскольку то, куда и как сядет клиент, дает специалисту 
    информацию о нем и его психологическом состоянии (стул, кресло, 
    диван выберет клиент, сядет в тень или на освещенное пространство). 
    Однако психолог может взять инициативу на себя и предложить клиенту 
    заранее приготовленное для него место. Данный прием будет иметь 
    положительный аспект в тех случаях, когда клиент относится к 
    личностям зависимым или авторитарным. В первом случае решается 
    задача присоединения и ведения, во втором — установления иерархии.    

    Вторая процедура первого этапа: установление положительного 
    эмоционального настроя клиента. Рассмотрим техники, применяемые в 
    рамках данной процедуры. Прежде всего это установление раппорта. 
    Он устанавливается в течение первых тридцати секунд. Существует 
    прекрасная аксиома: у вас никогда не будет второй возможности 
    произвести первое впечатление. Что же поможет произвести приятное 
    впечатление? Опрятный внешний вид, благожелательное выражение лица, 
    соблюдение социальной зоны общений. Последний фактор весьма условен. 
    Помните, что границы социальной зоны определяются не 
    среднестатистическими расчетами, а личностными особенностями 
    (национальность, воспитание, местность проживания). Процедура: 
    снятие психологических барьеров. Естественно, что клиент испытывает 
    волнение, снять которое помогут специальные техники. 
    Прежде всего можно дать клиенту побыть некоторое время одному. 
    Например, извинившись, попросить несколько минут для «завершения» 
    какого-либо дела («дело» придумайте сами). Мягкая, спокойная, 
    ненавязчивая музыка также будет способствовать созданию благоприятной 
    атмосферы. Хорошо снимают психологическое напряжение размеренные 
    движения рук. С этой целью можно дать клиенту в руки предмет (книгу, 
    журнал, игрушку) или под предлогом помощи попросить что-либо сделать. 
    На вербальном уровне эффективен прием «кавычек» («Был случай, когда 
    клиент очень волновался, но все закончилось хорошо») и прямое 
    разделение эмоций клиента: «Я тоже волнуюсь...».
    
    Этап второй. Сбор информации
    
    Процедура первая: диагностика личности клиента. В рамках данной 
    процедуры применяются техники: беседа, наблюдение, тесты.
    
    Беседа, целенаправленно организованная специалистом, является 
    основным методом ведения психологической консультации.
  
    В ходе нее предполагает выявление интересующих исследователя связей 
    на основе эмпирических данных, полученных в процессе общения 
    «клиент-консультант». Применяется на разных стадиях психологического 
    консультирования. В зависимости от целей этапа консультативного 
    процесса меняются и цели беседы. Как специфический вид беседы 
    выделяется интервью.
    
    Интервью — способ получения социально-психологической информации с 
    помощью устного опроса. Выделяют два вида интервью: свободное и 
    стандартизированное.
    
    Свободное интервью не зависит от темы и формы беседы и предполагает 
    сотрудничество клиента при поиске необходимой информации. К его 
    достоинствам можно отнести большой диапазон непосредственных 
    поведенческих реакций клиента, которые получает возможность наблюдать 
    специалист, что способствует более полному получению информации о 
    личности клиента и сути его проблемы, к недостаткам — достаточно 
    большой временной отрезок.
    
    Стандартизированное интервью — близко по форме к анкете, отличаясь 
    от него большей свободой клиента при формулировании ответов. 
    Наиболее полезно в начале консультативного процесса при ориентировке 
    в проблеме. Полученные данные впоследствии уточняются и используются 
    для выдвижения новых гипотез. Достоинство: информативность, экономия 
    времени, количественная выраженность результата. Недостатки: 
    вероятность потери эмоционального контакта с клиентом, снижение 
    непосредственных поведенческих реакций вследствие активизации 
    механизмов защиты.
      
    Наблюдение — эмпирический метод психологического исследования, 
    заключающийся в целенаправленном и осознанном восприятии специалистом 
    психических проявлений клиента. Позволяет специалисту собрать 
    информацию о клиенте, выявить наиболее значимые событийные моменты, 
    отследить изменение его состояния на разных этапах консультативного 
    процесса. К недостаткам можно отнести: субъективизм специалиста, 
    который может выразиться в фиксации внимания на факте, значимом с 
    позиции консультанта, а не клиента, и последующей его интерпретации, 
    а не прояснении и кбнтрперенос. Применяется на всех этапах 
    консультативного процесса.
    
    Психологические тесты — стандартизированный метод для измерения 
    уровня развития или состояния какого-либо психологического качества 
    или свойства отдельного индивида. В рамках психологического 
    консультирования применяются тесты, соответствующие определенным 
    критериям. Данные критерии будут рассмотрены в главе «Особенности 
    психодиагностики в процессе консультирования».
    
    Процедура вторая: прояснение сути проблемы клиента, определение его 
    ресурсов. Основные техники: диалог, слушание.
    
    Диалог. Определяется как речевое общение между двумя и более людьми, 
    предполагающее обмен репликами. В широком смысле репликой считается 
    и ответ в виде действия, жеста, молчания. Опирается на традицию 
    устного интеллектуального общения в Древней Греции. В ее истоках — 
    деятельность Сократа, которая, кроме того, нашла свое отражение в 
    психотерапевтической методике «Сократовский диалог». В основе 
    данной методики лежит логическая аргументация, преподносимая 
    специалистом клиенту в виде вопросов, которые предполагают только 
    положительные ответы. При формулировании вопросов консультант 
    сознательно игнорирует непоследовательные, противоречивые, 
    бездоказательные суждения клиента.
    
    66
    
    В результате клиент планомерно подводится к принятию суждения, 
    которое перед этим им не принималось или не понималось.
    
    Техника «слушание». Прежде всего предполагает «слышание» другого 
    человека, в данном случае клиента. Иначе: в процесс «слышания» 
    включается не только восприятие произносимых слов, но и фона, на 
    котором они были произнесены.
    
    Следовательно, можно выделить два аспекта слушания: вербальный 
    аспект и невербальный.
    
    К вербальному аспекту относятся непосредственно слова, словосочетания, 
    метафоры, которые употребляет клиент в своей речи.
    
    К невербальному аспекту (фону):
    
    «язык тела» (позы, жесты, мимика);
    
    психофизиологические реакции (изменение цвета кожного покрова, 
    частота и глубина дыхания, степень потоотделения);
    
    голосовые характеристики (тон, тембр, темп, интонации).
    
    Истинность предположений, сделанных на. основе невербальных данных, 
    необходимо перепроверять вместе с клиентом. Это обусловлено тем, что 
    невербальные сообщения определяются целым рядом факторов. Например, 
    контекстом событий, личностными особенностями клиента, спецификой 
    взаимоотношений со значимым близким. О существовании многих из них 
    консультант может не знать, что, безусловно, приведет к искажению 
    воспринимаемой им информации и, как следствию, неправильному 
    представлению о сути проблемы и способов ее разрешения.
    \subsection*{Навыки поддержания консультативного контакта}
\begin{itemize}
    \item Проявление глубокого интереса к людям, следствием чего является
    терпение в общении с ними.
    \item Чувствительность к установкам и поведению других людей, способность
    отождествляться с самыми разными людьми.
    \item Эмоциональная стабильность и объективность.
    \item Аутентичность.
    \item Открытость собственному опыту. Означает искренность в восприятии
    собственных чувств. Консультант должен знать, замечать свои чувства, в
    том числе и отрицательные, не вытеснять их.
    \item Развитое самопознание. Чем больше консультант знает о самом себе, тем
    больше он поймет своих клиентов. Этому способствует умение слышать
    то, что творится внутри нас.
    \item Сильная идентичность. Идентичность – сложная динамическая структура,
    формирующаяся и развивающаяся на протяжении всей жизни человека.
    Единицей этой структуры является самоопределение – некоторое решение
    относительно себя, своей жизни, своих ценностей, принятое в результате
    интериоризации родительских ожиданий («Преждевременная
    идентичность») или в результате преодоления кризиса идентичности
    («Достигнутая идентичность»).
    \item Толерантность к неопределенности. Психолог-консультант должен уметь
    без значительного дискомфорта переносить ситуации неопределенности.
    Это достигается через: 
    \begin{enumerate} \setlength\itemsep{0pt}
        \item уверенность в своей интуиции и адекватности 
        чувств; 
        \item убежденность в правильности принимаемых решений;
        \item способность рисковать. Все эти качества приобретаются по мере
    \end{enumerate}
  
    личного и профессионального опыта.
    \item Принятие личной ответственности. Критика не вызывает у такого
    человека механизмов психологической защиты, а служит полезной
    обратной связью, улучшающей эффективность деятельности и даже
    организацию жизни.
    \item Стремление к глубине межличностных отношений. Обычно этому
    препятствуют: страх потерять свободу, быть более уязвимым, страх
    непринятия другим положительных чувств, отклонение их. Для того,
    чтобы эти факторы не мешали стремлению к развитию глубоких
    межличностных отношений, в среде, где живет психолог-консультант,
    необходимо стараться создавать такую атмосферу, чтобы люди избегали
    осуждения, «наклеивания ярлыков».
    \item Постановка реалистичных целей. Правильная оценка собственных
    возможностей позволяет ставить перед собой лишь достижимые цели.
    Следует не винить себя за ошибки, а делать полезные выводы. 
\end{itemize}
    \end{document}