
\documentclass[a4paper, 12pt]{extarticle}
\usepackage{fontspec}
\usepackage{polyglossia}
\setmainfont{CMU Serif}
\newfontfamily{\cyrillicfont}{CMU Serif}
\setsansfont{CMU Sans Serif}
\newfontfamily{\cyrillicfontsf}{CMU Sans Serif}
\setmonofont{CMU Typewriter Text}
\newfontfamily{\cyrillicfonttt}{CMU Typewriter Text}
\setdefaultlanguage{russian}
\usepackage[left=1cm,right=1cm,
top=2cm,bottom=2cm]{geometry}
%%% Дополнительная работа с математикой
\usepackage{amsfonts,amssymb,amsthm,mathtools} % AMS
\usepackage{amsmath}
\usepackage{icomma} % "Умная" запятая: $0,2$ --- число, $0, 2$ --- перечисление

%% Шрифты
\usepackage{euscript} % Шрифт Евклид
\usepackage{mathrsfs} % Красивый матшрифт

%% Свои команды
\DeclareMathOperator{\sgn}{\mathop{sgn}}


%% Перенос знаков в формулах (по Львовскому)
\newcommand*{\hm}[1]{#1\nobreak\discretionary{}
	{\hbox{$\mathsurround=0pt #1$}}{}}

%%% Работа с картинками
\usepackage{graphicx}  % Для вставки рисунков
\graphicspath{{Изображения/}{image}}  % папки с картинками
\setlength\fboxsep{3pt} % Отступ рамки \fbox{} от рисунка
\setlength\fboxrule{1pt} % Толщина линий рамки \fbox{}
\usepackage{wrapfig} % Обтекание рисунков и таблиц текстом

%%% Работа с таблицами
\usepackage{array,tabularx,tabulary,booktabs} % Дополнительная работа с таблицами
\usepackage{longtable}  % Длинные таблицы
\usepackage{multirow} % Слияние строк в таблице
\begin{document}
\pagestyle{empty}
\section{Метрики}
{\setlength{\extrarowheight}{5pt}
  \begin{tabularx}{\textwidth}{||l|l|X||}

      \hline
      Название              & Метрика                                                                  & Какое множество или пространство                                                                                   \\
      \hline

      Дискретная            & $
          \rho (x, y) =
      \begin{cases}
              1 & x=y      \\
              0 & x \neq y \\
          \end{cases}$       & $X$ - произвольное непустое множество

      \\
      \hline

      $\mathbb{R}^n_p$      & $\rho_p (x, y)=(\sum_{k = 1}^{n}|x_k-y_k|^p)^\frac{1}{p}$            & \multirow{2}{6cm}{$\mathbb{R}^n$ - множество n-мерных векторов $x=(x_1, x_2, \dots, x_n)$  }                         \\
      \cline{1-2}
      $\mathbb{R}^n_\infty$ & $\rho_\infty (x, y)=\underset{1\leq k\leq n}{\max}|x_k-y_k|$              &                                                                                                                    \\
      \hline
      $C[a, b]$              & $\rho (x, y)=\underset{a \leq t \leq b}{\max}|x(t)-y(t)|$                 & \multirow{2}{6cm}{Пространство числовых функций, непрерывных на $[a, b]$}                                           \\
      \cline{1-2}
      $C_1[a, b]$            & $\rho (x, y)=\int_{b}^{a}|x(t)-y(t)|dx $                                &                                                                                                                    \\[5pt]
      \hline
      $M[a, b]$              & $\rho (x, y)=\underset{a \leq t \leq b}{\sup}|x(t)-y(t)|$                 & Пространство числовых функций, определённых и ограниченных на $[a, b]$                                              \\
      \hline
      $l_p$                 & $\rho_p (x, y)=(\sum_{k = 1}^{\infty}|x_k-y_k|^p)^\frac{1}{p}$            & Пространство числовых последовательностей $x=~(x_1, x_2, \dots, x_k, \dots)$, суммируемых с $p$-той степенью            \\
      \hline
      $m$                   & $\rho (x, y)=\underset{k \in \mathbb{N} }{\sup}|x_k-y_k|$                 & Пространство произвольных числовых последовательностей $x=~(x_1, x_2, \dots, x_k, \dots)$, таких что $\sup|x_k|<\infty$ \\
      \hline
      $s$                   & $\rho_p (x, y)=\sum_{k = 1}^{\infty} \cfrac{|x_k-y_k|}{2^k(1+|x_k-y_k|)}$ & Пространство произвольных числовых последовательностей $x=~(x_1, x_2, \dots, x_k, \dots)$                               \\
      \hline
  \end{tabularx}}


\section{Определения}

\textbf{Метрика} - это число, поставленное в соответствие элементам $x$ и $y$ из произвольного множества $X$, такое что выполняются аксиомы для любых $x, y, z\in X$:
\[\rho(x, y)\geq 0, \rho(x, y)=0\Longleftrightarrow x=y\]
\[\rho(x, y)=\rho(y, x)\]
\[\rho(x, y)\leqslant \rho(x, z)+\rho(z, y)\]

\textbf{Метрическое пространство} - это $\{X, \rho\}$, если выполняются аксиомы.

\textbf{Открытый шар} - это множество с радиусом $r\geqslant 0$ и центром в точке $x_0\in X$ в метрическом пространстве  $\{X, \rho\}$.
$B(x_0, r)=\{x \in X|\rho(x, x_0)<r \}$

\textbf{Замкнутый шар} - это множество с радиусом $r\geqslant 0$ и центром в точке $x_0\in X$ в метрическом пространстве  $\{X, \rho\}$.
$B(x_0, r)=\{x \in X|\rho(x, x_0)\leqslant r \}$

\textbf{Ограниченное множество} - это множество($M$) полностью входящее в замкнутый шар конечного радиуса в множестве $X$ из $\{X, \rho\}$.
$(M \subset  X) $.

\textbf{Сходимость в метрическом пространстве}

Последовательность элементов $\{x_n \} \subset X$
сходится к элементу $x \in X$ по метрике, 
если $\rho(x_n, x)~\longrightarrow~0$ при $n~\longrightarrow~\infty$ в $\{X, \rho\}$.

\textbf{Фундаментальная последовательность} $\{x_n\}\subset X$

$\forall(\varepsilon >0)\exists(N \in \mathbb{N})\forall(n, m\geqslant N)[\rho(x_n, x)<\varepsilon]$

\textbf{Полное метрическое пространство}

Метрическое пространство полное, если всякая фундаментальная последовательность в нём сходится.

\textbf{Точка прикосновения множества}

$x \in X$ точка прикосновения множества $M \subset X$, если $ \forall(\varepsilon>0)[B(x, \varepsilon)\cap M\neq \varnothing ]$. $X \in \{X, \rho\}$.

\textbf{Замыкание множества} - это множество всех точек прикосновения множества $M$.

\textbf{Изолированная точка множества}

($x \in M$), если $\forall(\varepsilon >0)[B(x, \varepsilon)\cap M = \{x\} ]$

\textbf{Предельная точка множества}

($x \in M$), если $ \forall(\varepsilon >0)\exists(y\neq x)[y \in B(x, \varepsilon)\cap M]$

\textbf{Замкнутое множество}

Множество $F$ в метрическом пространстве называется замкнутым, если $F=\overline{F}(\overline{F}\subset F)$.

\textbf{Внутренняя точка множества}

$(x \in X)$ внутренняя точка множества $M$, если $\exists(\varepsilon>0)[B(x, \varepsilon)\subset M]$

\textbf{Внутренность множества} - это множество всех внутренних точек множества $M$.

\textbf{Открытое множество} - множество, где всякая точка является внутренней.

\textbf{Подпространство метрического пространства}

Всякий открытый шар $B(x, \varepsilon), \varnothing, X $ в метрическом пространстве $\{X, \rho \}$ являются открытыми множествами. 

\textbf{Совершенное множество} - это замкнутое и не имеющие изолированных точек множество.

\textbf{Всюду плотное множество} - это множество  $ M \subset X$, где $ \overline{M}=X$.

\textbf{Множество, плотное в другом множестве} - это множество $ M \subset X$, где $ B \subset X$ и $B \subset \overline{M}$

\textbf{Нигде не плотное множество}  - это множество $ M \subset X$:

$\forall(B(x, \varepsilon) \subset X) \exists(B(y, \varepsilon) \subset B(x, \varepsilon))[B(y, \varepsilon)\cap M=\varnothing]$

\textbf{Множество первой категории} - это множество $M \subset X$ представимое, как объединение конечного или счётного числа нигде не плотных множеств.

\textbf{Множество второй категории} - множество не являющиеся множеством первой категории.
\end{document}