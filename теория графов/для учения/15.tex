\subsection{Порядковые числа в графах}

\noindent\textbf{Пример использования порядковых чисел}

\noindent\textbf{Порядковое число} графа $G$ — минимальное число цветов, необходимых для раскраски вершин графа так, чтобы никакие две смежные вершины не имели одинаковый цвет.

\noindent\textbf{Теорема 5.1}: Для любого графа $G$ его порядковое число $\chi(G)$ удовлетворяет неравенству:
\[
\chi(G) \leq \Delta(G) + 1
\]
где $\Delta(G)$ — максимальная степень вершины в графе $G$.

\noindent\textbf{Пример}: Рассмотрим граф $K_4$ (полный граф с четырьмя вершинами).
\begin{itemize}
    \item Все вершины соединены друг с другом.
    \item $\Delta(K_4) = 3$.
    \item $\chi(K_4) = 4$, так как каждая вершина должна иметь уникальный цвет.
\end{itemize}

\noindent\textbf{Алгоритм раскраски графа}:
\begin{enumerate}
    \item Выберите вершину $v$ с максимальной степенью.
    \item Назначьте $v$ минимально возможный цвет, не совпадающий с цветами её соседей.
    \item Повторите для всех вершин графа.
\end{enumerate}

\noindent\textbf{Замечание}: Порядковое число графа может быть равно $\Delta(G)$, если граф является двудольным.

\noindent\textbf{Следствие 5.1(a)}: Если граф $G$ планарен, то $\chi(G) \leq 4$ (теорема о четырёх красках).

\noindent\textbf{Применение}: Раскраска графов используется в задачах планирования, таких как распределение частот в беспроводных сетях и составление расписаний.