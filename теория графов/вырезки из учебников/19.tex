\subsection{Ядро графа}

Пусть $G = (X, \Gamma)$ — конечный или бесконечный граф. Множество $S \subseteq X$ называется \textit{ядром} графа, если $S$ устойчиво как внутренне, так и внешне, т.е. если

\begin{equation}
x \in S \Rightarrow \Gamma x \cap S = \varnothing,
\end{equation}

\begin{equation}
x \notin S \Rightarrow \Gamma x \cap S \neq \varnothing.
\end{equation}

Из условия (1) вытекает, что ядро $S$ не содержит петель. Из условия (2) — что $S$ содержит все такие вершины $x$, для которых $\Gamma x = \varnothing$. Очевидно, пустое множество $\varnothing$ не может быть ядром.

\textbf{Теорема 1} Если $S$ -- ядро графа $(X, \Gamma)$, то множество $S$ -- максимальное в семействе $\mathfrak{S}$ внутренне устойчивых множеств, т.е.
\[
A \in \mathfrak{S}, \, A \supseteq S \Rightarrow A = S
\]

Пусть $A$ внутренне устойчивое множество, содержащее ядро $S$; предположим, что $A$ строго содержит $S$ и покажем, что это приводит к противоречию. В самом деле, тогда существовала бы такая вершина $a$, что $a \in A$, $a \notin S$, откуда $\Gamma a \cap S = \varnothing$ и, значит, $\Gamma a \cap A \neq \varnothing$ в противоречии с условием $A \in \mathfrak{S}$.

\textbf{Теорема 2} В симметрическом графе без петель каждое максимальное множество семейства $\mathfrak{S}$ внутренне устойчивых множеств представляет собой ядро.

Пусть $S$ -- максимальное множество из $\mathfrak{S}$, надо показать что для любой вершины $x \notin S$ имеет место $\Gamma x \cap S \neq \varnothing$. В самом деле, если $\Gamma x \cap S = \varnothing$ для некоторой вершины $x \in S$, то множество $A = S \cup \{x\}$ внутренне устойчиво (поскольку $x \notin \Gamma x$) и в то же время $A \supset S$, что противоречит предположению о максимальности $S$ в $\mathfrak{S}$.

\textbf{Следствие} \textit{Симметрический граф без петель обладает ядром}

В самом деле, образуем вспомогательный граф $(S, \Gamma)$, вершинами которого служат внутренне устойчивые множества данного симметрического графа $a \subseteq S' \subseteq S$ тогда и только тогда, когда $S = S'$. Вспомогательный граф --- индуктивный следовательно по лемме Цорна (гл. 3) существует вершина $S \in S$ без строго последующих. Множество $S$ является максимальным внутренне устойчивым и значит, в силу теоремы 2, ядром.

В случае когда данный симметрический граф конечен это следствие становится очевидным и процесс нахождения ядра состоит в следующем:

Берем произвольную вершину $x_0$ и полагаем $S_0 = \{x_0\}$; затем берем некоторую вершину $x_1 \notin \Gamma S_0$ и полагаем $S_1 = \{x_0, x_1\}$, далее берем вершину $x_2 \notin \Gamma S_1$, и т.д. Так как граф конечен, то рано или поздно мы получим $\Gamma S_n = X$ и $S_n$ как максимальное множество в $S$ будет ядром.

\textbf{Характеристической функцией} $\varphi_S(x)$ множества $S$ называется функция

\[
\varphi_S(x) = 
\begin{cases} 
1, & \text{при } x \in S \\ 
0, & \text{при } x \notin S 
\end{cases}
\]

Если $\Gamma x = \emptyset$, то условимся считать, что $\max_{y \in \Gamma x} \varphi_S(y) = 0$.

\textbf{Теорема 3} \textit{Для того чтобы множество $S$ было ядром, необходимо и достаточно чтобы для характеристической функции $\varphi_S(x)$ выполнялось соотношение}

\[
\varphi_S(x) = 1 - \max_{y \in \Gamma x} \varphi_S(y)
\]

1\textdegree \ Пусть $S$ — ядро. В силу внутренней устойчивости

\[
\varphi_S(x) = 1 \implies x \in S \implies \max_{y \in \Gamma x} \varphi_S(y) = 0.
\]

В силу внешней устойчивости

\[
\varphi_S(x) = 0 \implies x \notin S \implies \max_{y \in \Gamma x} \varphi_S(y) = 1.
\]

Отсюда получается требуемое соотношение.

2\textdegree  Пусть $\varphi_S(x)$  — характеристическая функция некоторого множества S,  если рассматриваемое соотношение выполнено, то

$x \in S \implies \varphi_S(x) = 1 \implies \max_{y \in \Gamma_x} \varphi_S(y) = 0 \implies \Gamma_x \cap S = \varnothing$,

$x \notin S \implies \varphi_S(x) = 0 \implies \max_{y \in \Gamma_x} \varphi_S(y) = 1 \implies \Gamma_x \cap S \neq \varnothing$.

Следовательно  $S$  — ядро

\textbf{Теорема 4}  Прогрессивно конечный граф обладает ядром

Доказательство получается сразу, если заметить, что характеристическая функция $\varphi_S(x)$,  удовлетворяющая соотношению предыдущей теоремы, по индукции определяется на множествах

$X(0) = \{x | \Gamma_x = \varnothing\}$,

$X(1) = \{x | \Gamma_x \subseteq X(0)\}$,

$X(2) = \{x | \Gamma_x \subseteq X(1)\}$,

Теорема 4 (гл 3) показывает, что таким путем $\varphi_S$  будет определена на всем X.

\textbf{Теорема Ричардсона}  Конечный граф, не содержащий контуров нечетной длины, обладает ядром

Пусть $(X, \Gamma)$  — конечный граф без нечетных контуров, будем последовательно определять множества $Y_0, Y_1, Y_2$, $\subseteq X$  следующим образом

1\textdegree Берем $Y_0 = \varnothing$,  обозначим через $B_0$  базу (см. гл. 2) подграфа, порождаемого множеством  $X \setminus Y_0$,  эта база существует в силу теоремы 1 (гл. 2). Полагаем $Y_1 = B_0 \cup \Gamma^{-1} B_0$.

2\textdegree  Если множество $Y_n$  уже определено, то обозначим через  $B_n$  какую-либо базу подграфа, порождаемого множеством $X \setminus Y_n$,  удовлетворяющую условию $B_n \subseteq I^1 (\Gamma^{-1} B_{n-1} \setminus Y_{n-1})$.  Легко видеть,что такая база всегда существует\footnote{1}; далее полагаем

\[
Y_{n+1} = Y_n \cup B_n \cup \Gamma^{-1} B_a
\]

Тогда

\[
\emptyset = Y_0 \subseteq Y_1 \subseteq Y_2 \subseteq \cdots
\]

Так как граф предполагается конечным, то существует такой номер $m$, что $Y_m = X$, пусть

\[
S = \bigcup_{n=0}^{m-1} B_n
\]

Покажем что $S$ — ядро графа $(X, \Gamma)$.

1\textdegree $S$ внешне устойчиво, ибо если $x \notin S$, то $x \in \Gamma^{-1} B_k \setminus Y_k$ для некоторого номера $k$, значит $\Gamma x \cap B_k \neq \emptyset$ и $\Gamma x \cap S \neq \emptyset$.

2\textdegree $S$ внутренне устойчиво. В самом деле, никакие два элемента из $B_n$ не могут быть смежны (ибо $B_n$ является базой некоторого подграфа); рассмотрим две смежные вершины, одна из которых $\in B_n$, другая $\in B_p$, где $p < n$ (если такие вершины есть). Имеем $B_n \cap \Gamma^{-1} B_p = \emptyset$, ибо $\Gamma^{-1} B_p \subseteq Y_{p+1}$ и $B_n \subseteq X \setminus Y_n \subseteq X \setminus Y_{p+1}$. Точно так же $B_p \cap \Gamma^{-1} B_n = \emptyset$, так как в противном случае можно было бы

\footnote{1} Для доказательства того что граф $(X \setminus Y_n, \Gamma x \setminus Y_n)$ имеет базу

\[
B_n \subseteq \Gamma^{-1} (\Gamma^{-1} B_{n-1} \setminus Y_{n-1}),
\]

рассмотрим какую-нибудь базу $B'_n$ и вершину $a_0$ без строго последующих, выбранную для построения $B'_n$ в соответствии с теоремой 3 (п. 2), достаточно показать, что в $X \setminus Y_n$ найдётся вершина $b$, которой можно заменить вершину $a_0$ и для которой

\begin{enumerate}
    \item $b \notin a_0 \quad (\text{где} \quad \succsim \text{--- отношение квазипорядка в графе, порожденном множеством } X \setminus Y_n)$
    \item $b \in \Gamma^{-1}(B_{n-1} \setminus Y_{n-1})$
\end{enumerate}

В графе $(X \setminus Y_{n-1}, \Gamma \setminus Y_{n-1})$ существует путь $\mu = [a_0, a_1, \ldots, b]$, ведущий из $a_0$ в $B_{n-1}$, пусть $a_k$ --- первая вершина пути $\mu$, принадлежащая $Y_n$. Так как $a_k \in X \setminus Y_{n-1}$, $a_k \in Y_n$, $a_k \notin B_{n-1}$, то $a_k \in \Gamma^{-1}B_{n-1} \setminus Y_{n-1}$ (предполагается, что множество $Y_n$ определено по формуле $Y_n = Y_{n-1} \cup B_{n-2} \cup B_{n-1}$. Если бы $a_k \in B_{n-1}$, то вершина $a_{k-1}$ принадлежала бы $\Gamma^{-1}B_{n-1} \setminus Y_{n-1}$ и таким образом, $a_k$ не была бы первой вершиной пути $\mu$, принадлежащей $Y_n$. Следовательно $a_k \notin B_n$, и так как $a_k \notin Y_{n-1}$, то $a_k \in \Gamma^{-1}B_{n-1} \setminus Y_{n-1}$. \textit{Прим. ред.} Вершина $b = a_k$, удовлетворяет как условию (1), так и условию (2).

Построить такой путь $\mu = [x_0, \lambda_\rho, y_\rho, x_{n-1}, y_{n-2}, x_{n-2}, \ldots, y_\rho, x_\rho]$, что

\begin{align*}
    &x_0 \in B_n, \quad x_1 \in \Gamma^{-1}B_n, \\
    &x_n \in B_n \subseteq \Gamma^{-1}(\Gamma^{-1}B_{n-1} \setminus Y_{n-1}), \\
    &y_{n-1} \in \Gamma^{-1}B_{n-1} \setminus Y_{n-1}, \\
    &x_{n-1} \in B_{n-1}, \\
    &y_\rho \in \Gamma^{-1}B_\rho \setminus Y_\rho, \\
    &x_\rho \in B_\rho.
\end{align*}

Заметим, что все вершины пути $\mu$ принадлежат $X \setminus Y_p$ и что $\mu$ идет из $x_0 \in B_p$ в $x_p \in B_p$ (см. рис. 5--3), так как $B_p$ --- база подграфа, порожденного множеством $X \setminus Y_p$, то $x_0 = x_p$. Таким образом, путь $\mu$ представляет собой контур нечетной длины, что противоречит условию теоремы.