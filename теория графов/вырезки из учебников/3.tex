\subsection{Самодополнительный графы}

Указанную ситуацию можно описать графом \( G \) с шестью вершинами, представляющими людей; смежность двух вершин соответствует знакомству. Требуется показать, что в \( G \) найдутся либо три попарно смежные, либо три попарно несмежные вершины. Дополнение \( \overline{G} \) графа \( G \) имеет в качестве множества вершин множество \( V(G) \), две вершины в \( G \) смежны тогда и только тогда, когда они не смежны в \( G \). На рис. 2.12 в графе \( G \) нет треугольников, а в графе \( \overline{G} \) их ровно два\(^1\). \textit{Самодополнительный граф} — это граф, изоморфный своему дополнению. Примеры таких графов приведены на рис. 2.13.

В полном графе \( K_p \) каждая пара его \( p \) вершин\(^2\) смежна. Таким образом, граф \( K_p \) имеет \(\binom{p}{2}\) рёбер и является регулярным степени \( p-1 \). Граф \( K_3 \) — треугольник. Графы \( \overline{K_p} \) — вполне несвязные (или регулярные степени 0).