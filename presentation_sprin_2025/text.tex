\documentclass[a4paper, 12pt]{extarticle}
\usepackage{fontspec}
\usepackage{polyglossia}
\setmainfont{CMU Serif}
\newfontfamily{\cyrillicfont}{CMU Serif}
\setsansfont{CMU Sans Serif}
\newfontfamily{\cyrillicfontsf}{CMU Sans Serif}
\setmonofont{CMU Typewriter Text}
\newfontfamily{\cyrillicfonttt}{CMU Typewriter Text}
\setdefaultlanguage{russian}
\usepackage{amsfonts}

\usepackage[left=1.5cm,right=1cm,top=2cm,bottom=2cm]{geometry}
\title{Применение нейронных сетей к проблемам генерации
задач по планиметрии}
\author{Суматохина Александра 5 курс Кафедра Теории функции и геометрии\\Научный руководитель: д.ф.-м.н., проф. Семенов Е.М.\\Научный консультант: асп. Авдеев Н.Н}
\date{16 апреля 2025 г}
\begin{document}
\maketitle

Здравствуйте, меня зовут Суматохина Александра, я обучаюсь на 5 курсе, мой научный руководитель Семёнов Евгений Михайлович, научный консультант Авдеев Николай Николаевич. Тема моего доклада применение нейронных сетей к проблемам генерации
задач по планиметрии".

\subsection*{Существующие проблемы}
Но за 10 и 11 класс при подготовке к ЕГЭ школьники сталкиваются с дефицитом заданий по определённым категориям.
Так в конце 2021 года в список заданий ЕГЭ были добавлены новые задания под номером 11 по теме "графики функции", а в конце 2023 - задание номер 2 по теме "вектора", количество которых, для прорешивания было очень мало. 
А по теме "Производная и первообразная" банк заданий с невероятной скоростью.

Так как это преимущественно графические задания, решение их занимает менее минуты, а их составление вручную занимает несоразмерно много времени.

ЕГЭ является относительно неизменяемым экзаменом, поэтому все материалы, которые уже были выложены в открытый доступ имеют полные решения, что приводят к списыванию учениками.

При этом существуют задания с вспомогательным чертежом. Чаще всего для целого ряда заданий используется одна и та же иллюстрация, которая не всегда соответствуют условиям задачи, а иногда отвлекают от решения.
Проект «Час ЕГЭ» позволяет решить все эти проблемы.

\subsection*{Проект «Час ЕГЭ»}
«Час ЕГЭ» — компьютерный образовательный проект, разрабатываемый при математическом
факультете ВГУ в рамках «OpenSource кластера» и предназначенный для помощи учащимся
старших классов, учителями и репетиторам при подготовке к тестовой части единого государственного экзамена.

Задания в «Час ЕГЭ» генерируются случайным образом по специализированным алгоритмам, называемых шаблонами, каждый из которых охватывает множество вариантов соответствующей ему задачи.

\subsection*{За этот год были достигнуты следующие пункты в ЕГЭ профиле}%TODO:А надо ли?
\begin{itemize}
    \item В разделе планиметрии разработано и принято 26 шаблонов
    \item По теме векторов написано 18 шаблонов, из которых 10 уже приняты, а 8 находятся на рецензировании
    \item В стереометрии разработан 61 шаблон, где 8 приняты и 53 проходят рецензирование
    \item По теории вероятности написано 10 шаблонов базового уровня (9 принято, 1 на рецензировании) и 11 шаблонов повышенной сложности уже приняты
    \item В разделе производной и первообразной 17 шаблонов находятся на рецензировании
\end{itemize}

\subsection*{В ЕГЭ Базе покрыты темы:}
\begin{itemize}
	\item По планиметрии написано 60 шаблонов, находящихся на рецензировании
	\item По теории вероятности написано 11 шаблонов (2  принято, 9 на рецензировании)
\end{itemize}

\subsection*{По ОГЭ написаны:}
\begin{itemize}
	\item По планиметрии написано 16 шаблонов по теме треугольники проходят рецензирование
	\item По теме графики функций написано 2 шаблона находятся на рецензировании
\end{itemize}


Важным техническим достижением стала разработка библиотеки \texttt{flatten-shape-geometry}, которая опубликована в NPM. Библиотека включает в себя несколько ключевых зависимостей: \texttt{@flatten-js/core}, \texttt{mathjs}, \texttt{radians-degrees} и \texttt{degrees-radians}.

В рамках библиотеки реализованы следующие классы:
\begin{itemize}
    \item ShapeWithConnectionMatrix
    \item Angle
    \item Triangle
    \item Square
    \item Rectangle
    \item Rhombus
    \item Parallelogram
    \item Trapezoid
\end{itemize}

Библиотека содержит множество вспомогательных функций, включая:
\begin{itemize}
    \item Поиск перпендикуляра из(ОТ?) точки к отрезку
    \item Сдвиг координат
    \item Нахождение центра описанной окружности
    \item Другие геометрические операции
\end{itemize}

Для класса Circle из \texttt{@flatten-js/core} были дополнительно разработаны методы:
\begin{itemize}
    \item Получение точки на окружности по заданному углу
    \item Построение радиусов, диаметров и хорд
    \item Построение касательных из точки к окружности заданной длины
\end{itemize}

\section{Устройство flatten-shape-geometry}

Библиотека построена на основе иерархической структуры классов, где базовыми элементами являются Point, Vector, Line, Segment и Circle. На их основе построены более сложные геометрические объекты, такие как Polygon и его наследники. Особое место занимает класс ShapeWithConnectionMatrix (SWCM), который служит основой для создания сложных геометрических фигур.

\section{Использование нейросетей}

В процессе разработки мы успешно применяли нейросети для генерации кода. Например, был успешно реализован проектор 2D в 3D без использования сторонних библиотек. При тестировании класса Triangle с помощью DeepSeek R1 были выявлены как успешные результаты, так и области, требующие доработки:
\subsection*{Тестирование инициализации Треугольника}
\begin{itemize}
    \item Успешно протестирована инициализация треугольника с координатами $A(0,0)$, $B(4,0)$, $C(0,3)$
    \item Выявлены неточности при работе с углами в треугольнике с координатами $A(0,0)$, $B(4,4)$, $C(5,0)$. При этом с остальными тестами справляется.
\end{itemize}

\subsection*{Тестирование метода построения касательных}
При тестировании метода Circle.tangentsFromPoint с помощью DeepSeek R1 для окружности с центром в точке (0,0), радиусом 5 и точкой (10, 0) были выявлены следующие проблемы:
\begin{itemize}
    \item Некорректный расчет координат точек касания (ожидаемые значения x: ±3.5355, полученные: м4.3301)
    \item Неточности в вычислении y-координат точек касания (ожидаемые значения: 3.5355, полученные: 2.5)
    \item Проблемы с длинами касательных отрезков
\end{itemize}
\subsection*{Итоги}
Как итог, можно сделать следующие выводы:
\begin{itemize}
    \item Нейросети способны генерировать простейшие тесты без уточнений
    \item При усложнении задачи качество тестов существенно падает
    \item Нейросети не могут полностью заменить программиста в написании тестов
    \item Требуется экспертная оценка и доработка сгенерированных тестов человеком
\end{itemize}

\end{document}
