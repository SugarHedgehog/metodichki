\subsection{Экстремальные графы}

\noindent\textbf{Теорема 2.3 (Турана)} (о максимальном числе рёбер в графе без треугольников):\\
Наибольшее число рёбер у графов с $r$ вершин без треугольников равно $\lfloor r^2/4 \rfloor$.

\noindent\textbf{Доказательство} (по индукции для чётных $r$):
\begin{enumerate}[noitemsep,topsep=0pt]
\item База: очевидна для малых $r$
\item Шаг: для $r = 2n + 2$, где утверждение верно для всех чётных $r \leq 2n$:
   \begin{itemize}[noitemsep]
   \item Пусть $G$ -- граф с $p = 2n + 2$ вершинами без треугольников
   \item Существуют смежные вершины $u$, $v$ (граф не вполне несвязный)
   \item В подграфе $G' = G - \{u, v\}$ максимум $n^2$ рёбер
   \item Нет вершины $w$, смежной с $u$ и $v$ одновременно
   \item Если $w$ смежна с $k$ вершинами $G'$, то $v$ смежна максимум с $(2n - k)$ вершинами
   \item Всего рёбер: $n^2 + k + (2n - k) + 1 = n^2 + 2n + 1 = p^2/4$
   \end{itemize}
\end{enumerate}

\noindent\textbf{Конструктивное доказательство существования:}\\
Для чётного $p$ $(p, p^2/4)$-граф без треугольников строится так:
\begin{itemize}[noitemsep]
\item Берём два множества $V_1$ и $V_2$ по $p/2$ вершин
\item Соединяем каждую вершину из $V_1$ с каждой из $V_2$
\end{itemize}

\noindent\textbf{Примечания:}
\begin{itemize}[noitemsep]
\item Доказательство существования чисел $r(m, n)$ см. у М. Холла
\item По определению бесконечный граф не является графом
\item Обзор бесконечных графов: см. Нэш-Вильямс
\end{itemize}