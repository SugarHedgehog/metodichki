\documentclass[10pt]{article}
\usepackage[russian]{babel}
\usepackage[utf8]{inputenc}
\usepackage[T2A]{fontenc}
\usepackage{amsmath}
\usepackage{amsfonts}
\usepackage{amssymb}
\usepackage[version=4]{mhchem}
\usepackage{stmaryrd}

\begin{document}

\title{Конспект Лекции 9: Асимптотические методы теории нелинейных колебаний}
\author{}
\date{}
\maketitle

\section*{Введение}
Лекция посвящена асимптотическим методам в теории нелинейных колебаний, которые применяются, когда точные аналитические решения уравнений движения найти невозможно. Основное внимание уделяется методам разложения по малому параметру.

\section*{Осциллятор с квадратичной нелинейностью}
Рассматривается уравнение движения осциллятора с квадратичной нелинейностью:
\[
\ddot{x} + \omega_{0}^{2} x + \alpha x^{2} = 0
\]
Путем введения безразмерных переменных и параметра $\varepsilon = \alpha A / \omega_{0}^{2}$, уравнение преобразуется к виду:
\[
\ddot{x} + x + \varepsilon x^{2} = 0
\]
Для случая слабой нелинейности ($\varepsilon \ll 1$) решение ищется в виде ряда по степеням $\varepsilon$.

\section*{Метод разложения по малому параметру}
Решение представляется в виде:
\[
x(t) = x_{1}(t) + \varepsilon x_{2}(t) + \varepsilon^{2} x_{3}(t) + \ldots
\]
Подстановка этого ряда в уравнение движения приводит к системе уравнений для каждого порядка $\varepsilon$. Решение для $x_{1}$ соответствует гармоническому осциллятору, а для $x_{2}$ учитывает влияние нелинейности.

\section*{Осциллятор Дуффинга}
Рассматривается осциллятор с кубической нелинейностью:
\[
\ddot{x} + \omega_{0}^{2} x + \beta x^{3} = 0
\]
Аналогично предыдущему случаю, уравнение преобразуется и решается методом разложения по малому параметру. Однако, в этом случае возникает резонанс, который приводит к секулярному росту амплитуды.

\section*{Метод Линштедта-Пуанкаре}
Для устранения секулярных членов и учета неизохронности, вводится новая временная переменная $\tau = \omega t$. Частота $\omega$ также раскладывается в ряд по $\varepsilon$. Это позволяет получить решение, которое остается корректным на больших временах.

\section*{Заключение}
Методы, рассмотренные в лекции, позволяют находить приближенные решения для нелинейных осцилляторов, учитывая слабую нелинейность. Метод Линштедта-Пуанкаре особенно полезен для устранения секулярных членов и учета зависимости частоты от амплитуды.

\end{document}