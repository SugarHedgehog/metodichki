\subsection{Неориентированные графы, степени, изоморфизм}

\begin{itemize}
	\item \textbf{Граф}:
	\begin{itemize}
		\item Обозначается как $G = (X, \Gamma)$.
		\item Состоит из:
		\begin{enumerate}
			\item[\(1^\circ\)] Непустое множество $X$.
			\item[\(2^\circ\)] Отображение $\Gamma$ множества $X$ в $X$.
		\end{enumerate}
	\end{itemize}

	\item \textbf{Элементы графа}:
	\begin{itemize}
		\item \textbf{Вершина}: Каждый элемент множества $X$ называется точкой или вершиной графа.
		\item \textbf{Дуга}: Пара элементов $(x, y)$, где $y \in \Gamma x$, называется дугой графа.
	\end{itemize}

	\item \textbf{Изображение графа}:
	\begin{itemize}
		\item Элементы $X$ изображаются точками на плоскости.
		\item Пары точек $x$ и $y$, где $y \in \Gamma x$, соединяются непрерывной линией со стрелкой от $x$ к $y$.
	\end{itemize}

	\item \textbf{Множество дуг}:
	\begin{itemize}
		\item Обозначается через $U$.
		\item Дуги обозначаются буквами $\alpha$, $\beta$, $\omega$ (при необходимости с индексами).
	\end{itemize}
\end{itemize}

\textit{Степенью} вершины \(v_i\) в графе \(G\) — обозначается \(d_i\) или \(\deg v_i\) — называется число рёбер, инцидентных \(v_i\) (то есть рёбер, которые соединены с \(v_i\)). Поскольку каждое ребро инцидентно двум вершинам, в сумму степеней вершин графа каждое ребро вносит двойку. Таким образом, мы приходим к утверждению, которое установлено Эйлером и является исторически первой теоремой теории графов.

\subsubsection*{Теорема 2.1}
Сумма степеней вершин графа \(G\) равна удвоенному числу его рёбер:
\[
\sum_i \deg v_i = 2q.
\]

\subsubsection*{Следствие 2.1 (a)}
В любом графе число вершин с нечётными степенями чётно.

В \((p, q)\)-графе \(0 \leq \deg v \leq p-1\) для любой вершины \(v\). Минимальная степень вершин графа \(G\) обозначается через \(\min \deg G\) или \(\delta(G)\), максимальная — через \(\max \deg G = \Delta(G)\). Если \(\delta(G) = \Delta(G) = r\), то все вершины имеют одинаковую степень и такой граф \(G\) называется \textit{регулярным} (или \textit{однородным}) степени \(r\). В этом случае говорят о степени графа и пишут \(\deg G = r\).

Регулярный граф степени 0 совсем не имеет рёбер. Если \(G\) — регулярный граф степени 1, то каждая его компонента содержит точно одно ребро; в регулярном графе степени 2 каждая компонента — цикл, и, конечно, обратно. Первые интересные\(^2\) регулярные графы имеют степень 3; такие графы называются \textit{кубическими}. На рис. 2.11 показаны два регулярных графа с 6 вершинами. Второй из них изоморфен каждому из трёх графов, изображённых на рис. 2.5.

\subsection*{Следствие 2.1 (б)}
Каждый кубический граф имеет чётное число вершин.

Полезно дать названия вершинам с малыми степенями. Вершина \(v\) называется \textit{изолированной}, если \(\deg v = 0\), и \textit{концевой} (или \textit{висячей}), если \(\deg v = 1\).