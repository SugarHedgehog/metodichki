\textbf{Деревья}

\noindent\textbf{Основные определения:}
\noindent\textbf{Ациклический граф} -- граф без циклов.
\noindent\textbf{Дерево} -- связный ациклический граф.
\noindent\textbf{Лес} -- граф без циклов (компоненты -- деревья).

\noindent\textbf{Теорема 4.1.} Для графа $G$ эквивалентны:
\noindent 1) $G$ -- дерево
\noindent 2) любые две вершины соединены единственной простой цепью
\noindent 3) $G$ связен и $p = q + 1$
\noindent 4) $G$ ациклический и $p = q + 1$
\noindent 5) $G$ ациклический, и добавление любого ребра создаёт ровно один цикл
\noindent 6) $G$ связный, не $K_p$ при $p \geq 3$, добавление ребра создаёт один цикл
\noindent 7) $G$ не $K_3 \cup K_1$ и не $K_3 \cup K_2$, $p = q + 1$, добавление ребра создаёт один цикл

\noindent\textbf{Доказательство} (схема):
\noindent 1$\Rightarrow$2: От противного: две цепи образуют цикл
\noindent 2$\Rightarrow$3: Индукция по числу вершин
\noindent 3$\Rightarrow$4: От противного: цикл длины $n$ требует $q \geq p$
\noindent 4$\Rightarrow$5: Единственность компоненты из $p = q + k$
\noindent 5$\Rightarrow$6: $K_p$ при $p \geq 3$ содержит цикл
\noindent 6$\Rightarrow$7: Анализ возможных циклов
\noindent 7$\Rightarrow$1: Исключение случаев с циклами

\noindent\textbf{Следствие 4.1(а).} В нетривиальном дереве есть минимум две висячие вершины.
\noindent\textit{Доказательство:} Из $\sum d_i = 2(p-1)$ в дереве.