\subsection{Цикломатическое число графа}

\noindent\textbf{Определение.} \textit{Мультиграф} $(X,U)$ -- пара из множества вершин $X$ и множества рёбер $U$, где пара вершин может соединяться несколькими рёбрами.

\noindent\textbf{Важные числовые характеристики:}
\noindent Для мультиграфа $G$ с $n$ вершинами, $m$ рёбрами, $p$ компонентами:
\[\rho(G) = n - p \quad \text{(ранг графа)}\]
\[\nu(G) = m - n + p = m - \rho(G) \quad \text{(цикломатическое число)}\]

\noindent\textbf{Теорема 1.} При добавлении ребра между $a$ и $b$:
\noindent Если $a,b$ соединены цепью или совпадают:
\[\rho(G') = \rho(\bar{G}), \quad \nu(G') = \nu(\bar{G}) + 1\]
\noindent Иначе:
\[\rho(G) = \rho(\bar{G}) + 1, \quad \nu(G') = \nu(\bar{G})\]

\noindent\textbf{Векторное представление циклов:}
\noindent • Каждому ребру присваивается ориентация
\noindent • Для цикла $\mu$: $c^k = r_k - s_k$, где $r_k,s_k$ -- число проходов по/против ориентации
\noindent • Цикл представляется вектором $(c^1,\ldots,c^m)$
\noindent • Циклы независимы $\Leftrightarrow$ их векторы линейно независимы

\noindent\textbf{Теорема 2.} Цикломатическое число $\nu(G)$ равно максимальному количеству независимых циклов.

\noindent\textbf{Следствия:}
\noindent 1) $\nu(G)=0$ $\Leftrightarrow$ граф без циклов
\noindent 2) $\nu(G)=1$ $\Leftrightarrow$ граф содержит ровно один цикл

\noindent\textbf{Теорема 3.} В сильно связном графе цикломатическое число равно максимальному количеству независимых контуров.