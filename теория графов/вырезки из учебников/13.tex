\subsection{Хроматическое число плоского графа}

Очевидно, что
\[
\chi(H) \leq \chi(G) + 1 \quad \text{и} \quad \overline{\chi}(H) \leq \overline{\chi}(G) + 1.
\]
Если
\[
\chi(H) < \chi(G) + 1 \quad \text{или} \quad \overline{\chi}(H) < \overline{\chi}(G) + 1,
\]
то \(\chi(H) + \overline{\chi}(H) \leq p + 1\). Предположим теперь, что \(\chi(H) = \chi(G) + 1\) и \(\overline{\chi}(H) = \overline{\chi}(G) + 1\). Тогда удаление вершины \(v\) из \(H\), приводящее к образованию графа \(G\), уменьшает хроматическое число, так что \(d \geq \chi(G)\). Аналогично
\[
p - d - 1 \geq \overline{\chi}(G).
\]
Таким образом, \(\chi(G) + \overline{\chi}(G) \leq p - 1\). Следовательно, всегда
\[
\chi(H) + \overline{\chi}(H) \leq p + 1.
\]
Наконец, используя неравенство \(4\overline{\chi}\chi \leq (\chi + \overline{\chi})^2\), получаем
\[
\overline{\chi}\chi \leq \left(\frac{p + 1}{2}\right)^2.
\]

\textbf{Теорема о пяти красках}

Хотя не известно, все ли планарные графы 4-раскрашиваемы, все они, несомненно, 5-раскрашиваемы. Мы приведем доказательство этого известного утверждения, принадлежащее Хивуду.

\textbf{Теорема 12.7.} \textit{Каждый планарный граф 5-раскрашиваем.}

\textbf{Доказательство.} Будем доказывать индукцией по числу $p$ вершин. Для любого планарного графа с $p \leq 5$ вершинами результат тривиален, поскольку такой граф $p$-раскрашиваем.

Допустим, что все планарные графы с $p$ вершинами ($p \geq 5$) 5-раскрашиваемы. Пусть $G$ — плоский граф с $p+1$ вершинами. В силу следствия 11.1 (д) в графе $G$ найдется вершина $v$ степени 5 или менее. По предположению индукции плоский граф $G - v$ 5-раскрашиваем.

Рассмотрим приписывание цветов вершинам графа $G - v$, при котором получается 5-раскраска; цвета будем обозначать через $c_i$, $1 \leq i \leq 5$. Ясно, что если некоторый цвет, скажем $c_j$, не используется в раскраске вершин, смежных с $v$, то, приписав цвет $c_j$ вершине $v$, получим 5-раскраску графа $G$.

Осталось рассмотреть случай, когда \(\deg v = 5\) и для вершин графа \(G\), смежных с \(v\), используются все пять цветов. Переставим номера цветов, если это необходимо, чтобы вершины, смежные с \(v\) и окрашенные в цвета \(c_1, c_2, c_3, c_4, c_5\), были циклически упорядочены относительно \(v\). Пометим теперь вершину, смежную с \(v\) и окрашенную цветом \(c_i\), буквой \(v_i\), \(1 \leq i \leq 5\) (рис. 12.2).

Обозначим через \(G_{13}\) подграф графа \(G - v\), порожденный всеми вершинами, окрашенными в один из цветов \(c_1\) и \(c_3\). Если вершины \(v_1\) и \(v_3\) принадлежат различным компонентам графа \(G_{13}\), то 5-раскраску графа \(G - v\) можно получить, поменяв друг с другом ($c_1$ на $c_3$ и обратно) цвета вершин той компоненты графа \(G_{13}\), которая содержит \(v_1\). В этой 5-раскраске уже нет вершин, смежных с \(v\) и окрашенных в цвет \(c_1\); поэтому, окрасив \(v\) в цвет \(c_1\), образуем 5-раскраску графа \(G\).

Если же вершины \(v_1\) и \(v_3\) принадлежат одной и той же компоненте графа \(G_{13}\), то в \(G\) между \(v_1\) и \(v_3\) существует простая цепь, все вершины которой окрашены в цвета \(c_1\) и \(c_3\). Эта цепь вместе с цепью \(v_1 v_3\) образует простой цикл, который обязательно окружает или вершину \(v_2\), или вершины \(v_4\) и \(v_5\). В любом из этих случаев \(v_2\) и \(v_4\) нельзя соединить простой цепью, все вершины которой окрашены в цвета \(c_2\) и \(c_4\). Следовательно, рассматривая подграф \(G_{24}\) графа \(G - v\), порожденный всеми вершинами, окрашенными в цвета \(c_2\) и \(c_4\), заключаем, что вершины \(v_2\) и \(v_4\) принадлежат различным его компонентам. Таким образом, если поменять между собой цвета вершин в компоненте подграфа \(G_{24}\), содержащей \(v_2\), получим 5-раскраску графа \(G - v\), и в ней ни одна из вершин, смежных с \(v\), не будет окрашена в цвет \(c_2\). Поэтому, окрасив вершину \(v\) в цвет \(c_2\), образуем 5-раскраску всего графа \(G\).

\textbf{Теорема о пяти красках}

Хотя не известно, все ли планарные графы 4-раскрашиваемы, все они, несомненно, 5-раскрашиваемы. Мы приведем доказательство этого известного утверждения, принадлежащее Хивуду \cite{1}.

\begin{theorem}
Каждый планарный граф 5-раскрашиваем.
\end{theorem}

Будем доказывать индукцией по числу $p$ вершин. Для любого планарного графа с $p \leq 5$ вершинами результат тривиален, поскольку такой граф $p$-раскрашиваем.

Допустим, что все планарные графы с $p$ вершинами $(p \geq 5)$ 5-раскрашиваемы. Пусть $G$ — плоский граф с $p+1$ вершинами. В силу следствия 11.1 (д) в графе $G$ найдется вершина $v$ степени 5 или менее. По предположению индукции плоский граф $G - v$ 5-раскрашиваем.

Рассмотрим приписывание цветов вершинам графа $G - v$, при котором получается 5-раскраска; цвета будем обозначать через $c_i$, $1 \leq i \leq 5$. Ясно, что если некоторый цвет, скажем $c_j$, не используется в раскраске вершин, смежных с $v$, то, приписав цвет $c_j$ вершине $v$, получим 5-раскраску графа $G$.

Осталось рассмотреть случай, когда \(\deg v = 5\) и для вершин графа \(G\), смежных с \(v\), используются все пять цветов. Переставим номера цветов, если это необходимо, чтобы вершины, смежные с \(v\) и окрашенные в цвета \(c_1, c_2, c_3, c_4, c_5\), были циклически упорядочены относительно \(v\). Пометим теперь вершину, смежную с \(v\) и окрашенную цветом \(c_i\), буквой \(v_i\), \(1 \leq i \leq 5\) (рис. 12.2).

Обозначим через \(G_{13}\) подграф графа \(G - v\), порожденный всеми вершинами, окрашенными в один из цветов \(c_1\) и \(c_3\). Если вершины \(v_1\) и \(v_3\) принадлежат различным компонентам графа \(G_{13}\), то 5-раскраску графа \(G - v\) можно получить, поменяв друг с другом ($c_1$ на $c_3$ и обратно) цвета вершин той компоненты графа \(G_{13}\), которая содержит \(v_1\). В этой 5-раскраске уже нет вершин, смежных с \(v\) и окрашенных в цвет \(c_1\); поэтому, окрасив \(v\) в цвет \(c_1\), образуем 5-раскраску графа \(G\).

Если же вершины \(v_1\) и \(v_3\) принадлежат одной и той же компоненте графа \(G_{13}\), то в \(G\) между \(v_1\) и \(v_3\) существует простая цепь, все вершины которой окрашены в цвета \(c_1\) и \(c_3\). Эта цепь вместе с цепью \(v_1 v_3\) образует простой цикл, который обязательно окружает или вершину \(v_2\), или вершины \(v_4\) и \(v_5\). В любом из этих случаев \(v_2\) и \(v_4\) нельзя соединить простой цепью, все вершины которой окрашены в цвета \(c_2\) и \(c_4\). Следовательно, рассматривая подграф \(G_{24}\) графа \(G - v\), порожденный всеми вершинами, окрашенными в цвета \(c_2\) и \(c_4\), заключаем, что вершины \(v_2\) и \(v_4\) принадлежат различным его компонентам. Таким образом, если поменять между собой цвета вершин в компоненте подграфа \(G_{24}\), содержащей \(v_2\), получим 5-раскраску графа \(G - v\), и в ней ни одна из вершин, смежных с \(v\), не будет окрашена в цвет \(c_2\). Поэтому, окрасив вершину \(v\) в цвет \(c_2\), образуем 5-раскраску всего графа \(G\).

\textbf{Гипотеза четырёх красок}

В гл. 1 уже упоминалось, что гипотеза четырёх красок, благодаря попыткам решить её, служила катализатором для теории графов. Мы здесь представим теоретико-графовое обсуждение этой бесславной проблемы. \textit{Раскраской плоской карты} $G$ называется такое приписывание цветов областям в $G$, что никакие две смежные области не получают одинакового цвета. Карта $G$ называется $n$-раскрашиваемой, если существует её раскраска, использующая $n$ или менее цветов. Первоначальная формулировка гипотезы, упомянутой в гл. 1, выглядит так: каждая плоская карта 4-раскрашиваема.

\textbf{Гипотеза четырёх красок.} Каждый планарный граф 4-раскрашиваем.

Ещё раз подчёркиваем, что под раскраской графа всегда понимается раскраска его вершин, в то время как раскраска карты означает, что раскрашиваются именно её области. Таким образом, предположение, что каждая плоская карта 4-раскрашиваема, на самом деле эквивалентно приведённой только что формулировке гипотезы четырёх красок. Чтобы убедиться в этом, предположим, что гипотеза четырёх красок справедлива, и возьмём произвольную плоскую карту \( G \). Пусть \( G^* \) — граф, являющийся основой карты, и геометрически двойственной к карте \( G \). Так как две области карты \( G \) смежны тогда и только тогда, когда соответствующие им вершины графа \( G^* \) смежны, то карта \( G \) 4-раскрашиваема, поскольку граф \( G^* \) 4-раскрашиваем.

Обратно, предположим, что каждая плоская карта 4-раскрашиваема. Пусть $H$ — любой планарный граф, а $H^*$ — граф, двойственный к графу $H$ и нарисованный так, что каждая его область содержит точно одну вершину графа $H$. Связный плоский псевдограф $H^*$ можно перевести в плоский граф $H'$, добавляя две новые вершины на каждую петлю графа $H^*$ и одну новую вершину на каждое ребро из множества кратных ребер. Теперь 4-раскрашиваемость графа $H'$ означает 4-раскрашиваемость графа $H$. Таким образом, эквивалентность обеих формулировок доказана.

Если будет доказана гипотеза четырех красок, то результат будет неулучшаем, поскольку легко привести примеры планарных 4-хроматических графов. Таковы графы $K_4$ и $W_5$, изображенные на рис. 12.3. В каждом из этих графов не менее четырех треугольников, что является в силу теоремы Грюнбаума [1] необходимым условием 4-хроматичности.

\textbf{Теорема 12.8.} \textit{Каждый планарный граф, имеющий меньше четырех треугольников, 3-раскрашиваем.}

Отсюда немедленно вытекает следующее утверждение, первоначально доказанное Грёшем [1].

\textbf{Следствие 12.8 (a).} \textit{Каждый планарный граф, не содержащий треугольников, 3-раскрашиваем.}

Орe и Стемпл \cite{1} показали, что все плоские карты, имеющие не более 39 областей, 4-раскрашиваемы, и тем самым увеличили на 4 число областей в более раннем результате такого типа \footnote{1}. Все эти примеры подтверждают гипотезу четырех красок. Как мы сейчас увидим, эту гипотезу в ее формулировке для плоских карт можно попробовать доказывать для специального класса плоских карт.

\textbf{Теорема 12.9.} Гипотеза четырех красок справедлива тогда и только тогда, когда каждая кубическая плоская карта, не имеющая мостов, 4-раскрашиваема.

\textit{Доказательство.} Мы уже видели, что любая плоская карта 4-раскрашиваема тогда и только тогда, когда справедлива гипотеза четырех красок. В свою очередь это эквивалентно предложению, что каждая плоская карта, не содержащая мостов, 4-раскрашиваема, так как элементарное стягивание с помощью отождествления висячих вершин моста не изменяет числа областей карты и не нарушает смежность любых ее областей.

Ясно, что если 4-раскрашиваем всякая плоская карта, не содержащая мостов, то и всякая кубическая плоская карта, не содержащая мостов, также 4-раскрашиваема. Чтобы проверить обратное, предположим, что \( G \) — плоская карта без мостов и что все кубические плоские карты без мостов 4-раскрашиваемы. Так как \( G \) не содержит мостов, то в ней нет висячих вершин. Если в \( G \) существует вершина \( v \) степени 2, инцидентная ребрам \( y \) и \( z \), то произведем подразделение ребер \( y \) и \( z \), обозначая дополнительные вершины через \( i \) и \( w \) соответственно. Удалим теперь \( v \), отождествляя вершину \( i \) с одной из вершин степени 2 в некоторой копии графа \( K_4 - x \), а вершину \( w \) — с другой вершиной степени 2 в \( K_4 - x \). Очевидно, что каждая из новых вершин имеет степень 3 (рис. 12.4). Если \( G \) содержит вершину \( v_0 \) степени \( n \geq 4 \), инцидентную ребрам \( x_1, x_2, \ldots, x_n \), упорядоченным циклически относительно \( v_0 \), то, добавляя новую вершину \( v_i \), подразделяем каждое ребро \( x_i \). Затем удалим \( v_0 \) и добавим новые ребра \( v_1v_2, v_2v_3, \ldots, v_n-1v_n, v_nv_1 \). Опять каждая из дополнительных вершин имеет степень 3.

Обозначим полученную кубическую плоскую карту, которая, очевидно, не содержит мостов, через \( G' \). Эта карта по предположению 4-раскрашиваема. Рассмотрим в карте \( G \) произвольную вершину \( v \), у которой \( \deg v \neq 3 \). Если в \( G' \) отождествим между собой все добавленные при построении карты \( G' \) вершины, соответствующие вершине \( v \) (причем сделать это для всех вершин у карты \( G \), степень которых отлична от 3), то получим карту \( G \). Поэтому, имея некоторую 4-раскраску карты \( G' \) и осуществляя указанное выше стягивание карты \( G' \) в карту \( G \), получаем \( m \)-раскраску карты \( G \), где \( m \leq 4 \). Теорема доказана.