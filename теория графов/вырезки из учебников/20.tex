\subsection{Игры на графе, игра НИМ}

Граф $(X, \Gamma)$ дает возможность определить некоторую игру двух игроков, которых мы назовем $(A)$ и $(B)$. Положениями этой игры служат вершины графа, начальная вершина $x_0$ выбирается жребием, и противники играют поочередно: сперва игрок $(A)$ выбирает вершину $x_1$ в множестве $\Gamma x_0$, затем $(B)$ выбирает вершину $x_2$ в множестве $\Gamma x_1$, после этого $(A)$ опять выбирает вершину $x_3$ в $\Gamma x_2$, и т.д. Если один из игроков выбрал вершину $x_n$, для которой $\Gamma x_n = \emptyset$, то партия оканчивается, игрок, выбравший вершину последним, выиграл, а его противник проиграл. Ясно, что если граф не является прогрессивно конечным, то партия может никогда не окончиться.

В честь известного развлечения, которое здесь обобщено, будем описанную только что игру называть \textit{игрой Ним}, а определяющий ее граф обозначать через $(X, \Gamma)$; сейчас наша задача состоит в том, чтобы охарактеризовать выигрышные положения, т.е. те вершины графа, выбор которых обеспечивает выигрыш партии независимо от ответов противника. Главным результатом является следующая

\textbf{Теорема 1.} \textit{Если граф имеет ядро $S$ и если один из игроков выбрал вершину в ядре, то этот выбор обеспечивает ему выигрыш или ничью.}

Действительно, если игрок $(A)$ выбрал вершину $x_1 \in S$, то либо $\Gamma x_1 = \emptyset$, и тогда он уже выиграл партию, либо его противник $(B)$ вынужден выбрать вершину $x_2 \in X \setminus S$, а значит, следующим ходом игрок $(A)$ может выбрать $x_3$ опять в $S$ и продолжать в том же духе. Если в какой-либо определенный момент один из игроков выбрал вершину $x_n$, для которой $\Gamma x_n = \emptyset$, то $x_n \in S$, и выигравшим партнером необходимо является $(A)$.

сновной метод для хорошего игрока состоит следовательно, в вычислении какой-либо функции Гранди, если она существует, с помощью этой функции \( g(x) \) получаем ядро
\[
S = \{ x | g(x) = 0 \}
\]

рассматриваемого графа. Если начальная вершина \( x_0 \) такова, что \( g(x_0) = 0 \), то игрок (A) находится в критическом положении, ибо его противник может обеспечить себе выигрыш или ничью. Напротив, если \( g(x_0) \neq 0 \), то игрок (A) сам обеспечивает себе выигрыш или ничью, выбирая такую вершину \( x_1 \), что \( g(x_1) = 0 \).

\textbf{Следствие.} Если граф прогрессивно конечен, то существует одна и только одна функция Гранди \( g(x) \), каждый выбор такой вершины \( y \), для которой \( g(y) = 0 \), является выигрышным, а каждый выбор такой вершины \( z \), что \( g(z) \neq 0 \), — проигрышным. (Непосредственно)

