\documentclass[twocolumn]{article}
\usepackage{dashbox}
\setlength{\columnsep}{40pt}
\usepackage[T2A]{fontenc}
\usepackage[utf8]{inputenc}
\usepackage[english,russian]{babel}
\usepackage{graphicx}
\graphicspath{{pictures/}}
\DeclareGraphicsExtensions{.pdf,.png,.jpg}

\linespread{1.15}

\usepackage{../egetask}
\usepackage{../egetask-math-11-2022}

\def\examyear{2023}
\usepackage[colorlinks,linkcolor=blue]{hyperref}\usepackage{draftwatermark}
\SetWatermarkLightness{0.9}
\SetWatermarkText{https://vk.com/egemathika}
\SetWatermarkScale{ 0.4 }
\def\rfoottext{Разрешается свободное копирование в некоммерческих целях с указанием источника }
\def\lfoottext{Источник \href{https://vk.com/egemathika}{https://vk.com/egemathika}}

\begin{document}



\cleardoublepage
\def\examvart{Вариант 19.1}
\normalsize

\begin{center}
	\textbf{
		Единый государственный экзамен\\по МАТЕМАТИКЕ\\Профильный уровень\\ \qquad \\ Инструкция по выполнению работы
	}
\end{center}


\par \qquad Экзаменационная работа состоит из двух частей, включающих в себя 18 заданий. Часть 1 содержит 11 заданий с кратким ответом базового и повышенного уровней сложности. Часть 2 содержит 7 заданий с развёрнутым ответом повышенного и высокого уровней сложности.
\par \qquad На выполнение экзаменационной работы по математике отводится 3 часа 55 минут (235 минут).
\par \qquad Ответы к заданиям 1—11 записываются по приведённому ниже \underline {образцу} в виде целого числа или конечной десятичной дроби. Числа запишите в поля ответов в тексте работы, а затем перенесите их в бланк ответов №1.
%%\includegraphics[width=0.98\linewidth]{obrazec}
\par \qquad При выполнении заданий 12—18 требуется записать полное решение и ответ в бланке ответов №2.
\par \qquad  Все бланки ЕГЭ заполняются яркими чёрными чернилами. Допускается использование гелевой или капиллярной ручки.
\par \qquad При выполнении заданий можно пользоваться черновиком. \textbf{Записи в черновике, а также в тексте контрольных измерительных материалов не учитываются при оценивании работы.}
\par \qquad  Баллы, полученные Вами за выполненные задания, суммируются. Постарайтесь выполнить как можно больше заданий и набрать наибольшее количество баллов.
\par \qquad После завершения работы проверьте, что ответ на каждое задание в бланках ответов №1 и №2 записан под правильным номером.
\begin{center}
	\textit{\textbf{Желаем успеха!}}\\ \qquad \\\textbf{ Справочные материалы} \\
$\sin^2 \alpha + \cos^2 \alpha = 1$ \\
$\sin 2\alpha=2\sin \alpha \cdot \cos \alpha$ \\
$\cos 2\alpha=\cos^2 \alpha-\sin^2 \alpha$ \\
$\sin (\alpha+\beta)=\sin \alpha \cdot \cos \beta+\cos \alpha \cdot \sin\beta$ \\
$\cos (\alpha+\beta)=\cos \alpha \cdot \cos \beta-\sin\alpha \cdot \sin\beta$
\end{center}

\startpartone
\large




\begin{taskBN}{1}
В треугольнике $VXL$ угол $V$ равен $90^\circ$.  $XL=6$. Чему равна  $LV$, если $XV=\sqrt{27}$? 
\end{taskBN}

\begin{taskBN}{2}
Ребро правильного тетраэдра равно 71. Найдите площадь сечения, проходящего через середины четырёх рёбер правильного тетраэдра.
\end{taskBN}

\begin{taskBN}{3}
В чемпионате по вольной борьбе участвуют 100 спортсменок: 14 из Бразилии, 19 из России, 67 из Венесуэлы, остальные — из Австралии. Порядок, в котором выступают спортсменки, определяется жребием. Найдите вероятность того, что спортсменка, выступающая третьей, окажется из России.
\end{taskBN}

\begin{taskBN}{4}
В кофейне три официанта. Каждый из них занят с клиентом с вероятностью 0.2. Найдите вероятность того, что в случайный момент времени все три официанта заняты одновременно (считайте, что клиенты заходят независимо друг от друга).
\end{taskBN}

\begin{taskBN}{5}
Найдите корень уравнения $$2x-\sqrt{6x-2}=0$$ Если корней несколько, в ответе укажите меньший из них.
\end{taskBN}

\begin{taskBN}{6}
Найдите значение выражения $$\frac{97\cos504^\circ}{-2000(1-2\cos^2{252^\circ})}$$
\end{taskBN}

\begin{taskBN}{7}
\addpictoright[0.4\linewidth]{images/56704021781943n0}На рисунке изображены график функции $y=f(x)$ и касательная к этому графику, проведённая в точке $p$. Уравнение касательной имеет вид $y= 0{,}17  x+ 3{,}32 $. Найдите значение производной функции $g(x) =  5 f(x)+ -9 x+\frac{285106383}{200000000}$ в точке $p$.\vspace{2.5cm}
\end{taskBN}

\begin{taskBN}{8}
При температуре $0^\circ {\rm{C}}$ рельс имеет длину $l_0=20$ м. При возрастании температуры происходит тепловое расширение рельса, и его длина, выраженная в метрах, меняется по закону $l(t^\circ ) = l_0 (1 + \alpha  \cdot t^\circ)$, где $\alpha=1.2\cdot 10^{-5}(^\circ {\rm{C}})^{-1}$ — коэффициент теплового расширения, $t^\circ$  — температура (в градусах Цельсия). При какой температуре рельс удлинится на 9.6 мм? Ответ выразите в градусах Цельсия.
\end{taskBN}

\begin{taskBN}{9}
Теплоход проходит по течению реки до пункта назначения и после стоянки возвращается в пункт отправления. Скорость теплохода в неподвижной воде составляет 14 км/ч, а скорость течения составляет 12 км/ч. Через сколько часов после отплытия теплоход возвращается в пункт отправления, если расстояние от пункта отправления до пункта назначения составляет 143 км, а стоянка длится 9 часов? 
\end{taskBN}

\begin{taskBN}{10}
\addpictoright[0.4\linewidth]{images/529720192758127n0}На рисунке изображён график функции $f(x)=a\tg x+b$. Найдите $b$.\vspace{2.5cm}
\end{taskBN}

\begin{taskBN}{11}
Найдите наименьшее значение функции $y = -28x-3-10\sin x$ на отрезке $\left[-\frac{\pi}{2};0 \right]$
\end{taskBN}




\cleardoublepage
\def\examvart{Вариант 19.2}
\normalsize

\begin{center}
	\textbf{
		Единый государственный экзамен\\по МАТЕМАТИКЕ\\Профильный уровень\\ \qquad \\ Инструкция по выполнению работы
	}
\end{center}


\par \qquad Экзаменационная работа состоит из двух частей, включающих в себя 18 заданий. Часть 1 содержит 11 заданий с кратким ответом базового и повышенного уровней сложности. Часть 2 содержит 7 заданий с развёрнутым ответом повышенного и высокого уровней сложности.
\par \qquad На выполнение экзаменационной работы по математике отводится 3 часа 55 минут (235 минут).
\par \qquad Ответы к заданиям 1—11 записываются по приведённому ниже \underline {образцу} в виде целого числа или конечной десятичной дроби. Числа запишите в поля ответов в тексте работы, а затем перенесите их в бланк ответов №1.
%%\includegraphics[width=0.98\linewidth]{obrazec}
\par \qquad При выполнении заданий 12—18 требуется записать полное решение и ответ в бланке ответов №2.
\par \qquad  Все бланки ЕГЭ заполняются яркими чёрными чернилами. Допускается использование гелевой или капиллярной ручки.
\par \qquad При выполнении заданий можно пользоваться черновиком. \textbf{Записи в черновике, а также в тексте контрольных измерительных материалов не учитываются при оценивании работы.}
\par \qquad  Баллы, полученные Вами за выполненные задания, суммируются. Постарайтесь выполнить как можно больше заданий и набрать наибольшее количество баллов.
\par \qquad После завершения работы проверьте, что ответ на каждое задание в бланках ответов №1 и №2 записан под правильным номером.
\begin{center}
	\textit{\textbf{Желаем успеха!}}\\ \qquad \\\textbf{ Справочные материалы} \\
$\sin^2 \alpha + \cos^2 \alpha = 1$ \\
$\sin 2\alpha=2\sin \alpha \cdot \cos \alpha$ \\
$\cos 2\alpha=\cos^2 \alpha-\sin^2 \alpha$ \\
$\sin (\alpha+\beta)=\sin \alpha \cdot \cos \beta+\cos \alpha \cdot \sin\beta$ \\
$\cos (\alpha+\beta)=\cos \alpha \cdot \cos \beta-\sin\alpha \cdot \sin\beta$
\end{center}

\startpartone
\large




\begin{taskBN}{1}
Четырехугольник ABCD вписан в окружность. Угол ABC равен 53°, угол CAD равен 14°. Найдите угол ABD. Ответ дайте в градусах.
\end{taskBN}

\begin{taskBN}{2}
Два ребра прямоугольного параллелепипеда, выходящие из одной вершины, равны 6 и 7. Известно, что диагональ составляет 11. Найдите объём параллелепипеда.
\end{taskBN}

\begin{taskBN}{3}
На семинар приехали 19 учёных из Эквадора, 9 из Австралии и 52 из Венесуэлы. Порядок докладов определяется жеребьёвкой. Найдите вероятность того, что третьим окажется доклад учёного из Венесуэлы.
\end{taskBN}

\begin{taskBN}{4}
Вероятность того, что на тестировании по физике ученик Т. верно решит больше 10 задач, равна 0.7. Найдите вероятность того, что Т. верно решит ровно 10 задач  или меньше.
\end{taskBN}

\begin{taskBN}{5}
Найдите корень уравнения $$\log_{81}{3^{-31x-6}}=14$$
\end{taskBN}

\begin{taskBN}{6}
Найдите значение выражения $$\frac{\left ({32\sqrt{99}}\right )^{2}}{2048}$$
\end{taskBN}

\begin{taskBN}{7}
\addpictoright[0.4\linewidth]{images/20717210117155n0}На рисунке изображён график функции $y=f(x)$ и касательная к нему в точке с абсциссой $a$. Найдите $f'(a)$.\vspace{2.5cm}
\end{taskBN}

\begin{taskBN}{8}
Молот бросили под углом к плоской горизонтальной поверхности земли. Время полёта молота задаётся законом $t=\frac{2v_0\sin\beta}{g}$ с. При каком наименьшем значении угла $\beta$ время полёта будет не меньше, чем 2 секунды,  если молот бросают с начальной скоростью $v_0=18\frac{\mbox{м}}{\mbox{с}}$? Считайте, что ускорение свободного падения $g=$10$\frac{\mbox{м}}{\mbox{с}^2}$. Дайте в качестве ответа синус угла $\beta$. Ответ округлите до десятых.
\end{taskBN}

\begin{taskBN}{9}
Из пункта L в пункт G одновременно выехали два велосипеда. Первый проехал с постоянной скоростью весь путь. Второй проехал первую половину пути со скоростью, на 12 км/ч большей скорости первого, а вторую половину пути — со скоростью 8 км/ч, в результате чего прибыл в пункт G одновременно c первым велосипедом. Найдите скорость первого велосипеда. Ответ дайте в км/ч.
\end{taskBN}

\begin{taskBN}{10}
\addpictoright[0.4\linewidth]{images/784755726898981n0}На рисунке изображён график функции $f(x)=\frac{kx+a}{x-b}$. Найдите значение $x$, при котором $f(x)= -8{,}1 $.\vspace{2.5cm}
\end{taskBN}

\begin{taskBN}{11}
Определите наибольшее значение функции $y = 31\cos x+86+\frac{107}{\pi}x$ на отрезке $\left[\frac{2\pi}{3};\frac{7\pi}{2} \right]$
\end{taskBN}




\cleardoublepage
\def\examvart{Вариант 19.3}
\normalsize

\begin{center}
	\textbf{
		Единый государственный экзамен\\по МАТЕМАТИКЕ\\Профильный уровень\\ \qquad \\ Инструкция по выполнению работы
	}
\end{center}


\par \qquad Экзаменационная работа состоит из двух частей, включающих в себя 18 заданий. Часть 1 содержит 11 заданий с кратким ответом базового и повышенного уровней сложности. Часть 2 содержит 7 заданий с развёрнутым ответом повышенного и высокого уровней сложности.
\par \qquad На выполнение экзаменационной работы по математике отводится 3 часа 55 минут (235 минут).
\par \qquad Ответы к заданиям 1—11 записываются по приведённому ниже \underline {образцу} в виде целого числа или конечной десятичной дроби. Числа запишите в поля ответов в тексте работы, а затем перенесите их в бланк ответов №1.
%%\includegraphics[width=0.98\linewidth]{obrazec}
\par \qquad При выполнении заданий 12—18 требуется записать полное решение и ответ в бланке ответов №2.
\par \qquad  Все бланки ЕГЭ заполняются яркими чёрными чернилами. Допускается использование гелевой или капиллярной ручки.
\par \qquad При выполнении заданий можно пользоваться черновиком. \textbf{Записи в черновике, а также в тексте контрольных измерительных материалов не учитываются при оценивании работы.}
\par \qquad  Баллы, полученные Вами за выполненные задания, суммируются. Постарайтесь выполнить как можно больше заданий и набрать наибольшее количество баллов.
\par \qquad После завершения работы проверьте, что ответ на каждое задание в бланках ответов №1 и №2 записан под правильным номером.
\begin{center}
	\textit{\textbf{Желаем успеха!}}\\ \qquad \\\textbf{ Справочные материалы} \\
$\sin^2 \alpha + \cos^2 \alpha = 1$ \\
$\sin 2\alpha=2\sin \alpha \cdot \cos \alpha$ \\
$\cos 2\alpha=\cos^2 \alpha-\sin^2 \alpha$ \\
$\sin (\alpha+\beta)=\sin \alpha \cdot \cos \beta+\cos \alpha \cdot \sin\beta$ \\
$\cos (\alpha+\beta)=\cos \alpha \cdot \cos \beta-\sin\alpha \cdot \sin\beta$
\end{center}

\startpartone
\large




\begin{taskBN}{1}
В треугольнике $AQY$ угол $A$ равен $90^\circ$. Чему равна  $YQ$, если $\sin^2{Y}=0,01$, при этом  $YA=\sqrt{8,91}$? 
\end{taskBN}

\begin{taskBN}{2}
Основанием призмы является  прямоугольный треугольник. Определите высоту пирамиды, если второй катет основания составляет 3, а первый катет равен 7. При этом объём призмы составляет 42. 
\end{taskBN}

\begin{taskBN}{3}
На борту самолёта 12 кресел расположены рядом с запасными выходами и 18 — за перегородками, разделяющими салоны. Все эти места удобны для пассажира высокого роста. Остальные места неудобны. Пассажир Й. высокого роста. Найдите вероятность того, что на регистрации при случайном выборе места пассажиру Й. достанется удобное место, если всего в самолёте 250 мест.
\end{taskBN}

\begin{taskBN}{4}
Вероятность того, что на тестировании по физике учащийся П. верно решит больше 10 задач, равна 0.69. Найдите вероятность того, что П. верно решит ровно 10 задач  или меньше.
\end{taskBN}

\begin{taskBN}{5}
Найдите корень уравнения $$\log_{1296}{6^{6x+12}}=6$$
\end{taskBN}

\begin{taskBN}{6}
Найдите значение выражения $$ \log_{12}1728 $$
\end{taskBN}

\begin{taskBN}{7}
Прямая $y=-11x+x+f$ является касательной к графику функции $y=-6x^{2}+13x-33$. Найдите $f$.
\end{taskBN}

\begin{taskBN}{8}
Перед отправкой тепловоз издал гудок с частотой $f_0 = 214,5$ Гц. Чуть позже издал гудок подъезжающий к платформе тепловоз. Из-за эффекта Доплера частота второго гудка $f$ больше первого: она зависит от скорости тепловоза по закону $f(v)=\frac{f_0}{1-\frac{v}{c}}$ (Гц), где $c$ — скорость звука в воздухе (в м/с). Человек, стоящий на платформе, различает сигналы по тону, если они отличаются не менее чем на 13 Гц. Определите, с какой максимальной скоростью приближался к платформе тепловоз, если человек не смог различить сигналы, а $c = 315$ м/с. Ответ выразите в м/с.
\end{taskBN}

\begin{taskBN}{9}
Первая труба пропускает на 3 кубометра жидкости в минуту меньше, чем вторая. Сколько кубометров жидкости в минуту пропускает вторая труба, если цистерну ёмкостью 341,25 кубометров она опустошает на 13 минут быстрее, чем первая труба?

\end{taskBN}

\begin{taskBN}{10}
\addpictoright[0.4\linewidth]{images/209885634063079n0}На рисунке изображены графики функций $f(x)=3x^{2}+21x+36$ и $g(x)=ax^{2} +bx+c$, которые пересекаются в точках $A$ и $B$. Найдите ординату точки $B$.\vspace{2.5cm}
\end{taskBN}

\begin{taskBN}{11}
Вычислите точку максимума функции $y = -3x^{3}+40$ на отрезке $\left[-4;18 \right]$
\end{taskBN}
\end{document}