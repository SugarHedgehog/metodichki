\textbf{Линейно независимые циклы}

\noindent\textbf{Линейно независимые циклы}
\begin{itemize}
    \item \textbf{Пространство циклов} и \textbf{пространство коциклов} определяются над полем $F_2 = \{0, 1\}$.
    \item \textbf{0-цепь} — линейная комбинация вершин $\Sigma e_i v_i$.
    \item \textbf{1-цепь} — линейная комбинация рёбер $\Sigma e_i x_i$.
    \item \textbf{Граничный оператор} $\partial$: переводит 1-цепи в 0-цепи.
    \begin{itemize}
        \item $\partial$ — линейный оператор.
        \item Если $x = uv$, то $\partial x = u + v$.
    \end{itemize}
    \item \textbf{Кограничный оператор} $\delta$: переводит 0-цепи в 1-цепи.
    \begin{itemize}
        \item $\delta$ — линейный оператор.
        \item $\delta v = \Sigma e_i x_i$, где $e_i = 1$, если ребро $x_i$ инцидентно $v$.
    \end{itemize}
\end{itemize}

\noindent\textbf{Циклы и Коциклы}
\begin{itemize}
    \item \textbf{Циклический вектор} — 1-цепь с границей 0 (набор простых циклов без общих рёбер).
    \item \textbf{Пространство циклов} — векторное пространство всех циклических векторов.
    \item \textbf{Базис циклов} — максимальный набор независимых простых циклов.
    \item \textbf{Коцикл} — минимальный разрез графа.
    \item \textbf{Пространство коциклов} — множество всех кограниц графа.
    \item \textbf{Базис коциклов} — базис пространства коциклов, состоящий из коциклов.
\end{itemize}

\noindent\textbf{Циклический ранг}
\begin{itemize}
    \item \textbf{Теорема 4.5}: Циклический ранг $m(G)$ равен числу хорд любого остова в $G$.
    \item \textbf{Следствие 4.5 (а)}: $m(G) = q - p + 1$ для связного $(p, q)$-графа.
    \item \textbf{Следствие 4.5 (б)}: $m(G) = q - p + k$ для $(p, q)$-графа с $k$ компонентами.
\end{itemize}

\noindent\textbf{Коциклический ранг}
\begin{itemize}
    \item \textbf{Теорема 4.6}: Коциклический ранг $t^*(G)$ равен числу рёбер любого остова.
    \item \textbf{Следствие 4.6 (а)}: $t^*(G) = p - 1$ для связного $(p, q)$-графа.
    \item \textbf{Следствие 4.6 (б)}: $t^*(G) = p - k$ для $(p, q)$-графа с $k$ компонентами.
\end{itemize}

\noindent\textbf{Замечания}
\begin{itemize}
    \item Уравнение Эйлера — Пуанкаре: $p - q = k - m(G)$.
    \item Графы как симплициальные комплексы: вершины — 0-симплексы, рёбра — 1-симплексы.
\end{itemize}