\documentclass[a4paper,12pt]{article}
\usepackage[utf8]{inputenc}
\usepackage[russian]{babel}
\usepackage{amsmath}
\usepackage{amssymb}
\usepackage{graphicx}
\usepackage{hyperref}

\title{Отображение Эно: Подробный Конспект}
\author{}
\date{}

\begin{document}

\maketitle

\tableofcontents

\section{Введение}
Отображение Эно (Essentially Non-Oscillatory, ENO) — это метод численного анализа, используемый для решения гиперболических уравнений. Он был разработан для минимизации численных осцилляций, которые могут возникать при решении задач с разрывами или крутыми градиентами.

\section{Основные концепции}
\subsection{Гиперболические уравнения}
Гиперболические уравнения описывают волновые процессы и распространение сигналов. Примером является уравнение переноса:
\begin{equation}
\frac{\partial u}{\partial t} + a \frac{\partial u}{\partial x} = 0
\end{equation}

\subsection{Численные осцилляции}
При использовании стандартных численных методов, таких как метод конечных разностей, могут возникать осцилляции вблизи разрывов. Это связано с дисперсионными свойствами численных схем.

\section{Метод Эно}
\subsection{Основная идея}
Метод Эно выбирает локально гладкие интервалы для интерполяции, избегая разрывов. Это достигается за счет адаптивного выбора шаблона интерполяции.

\subsection{Алгоритм}
1. \textbf{Выбор шаблона:} Для каждого узла выбирается шаблон, который минимизирует осцилляции.
2. \textbf{Интерполяция:} Используется полином для интерполяции значений на выбранном шаблоне.
3. \textbf{Обновление:} Вычисленные значения используются для обновления решения на следующем временном шаге.

\section{Применение}
Метод Эно широко применяется в задачах, где важна точность вблизи разрывов, например, в аэродинамике и гидродинамике.

\section{Преимущества и недостатки}
\subsection{Преимущества}
- Устойчивость к осцилляциям.
- Высокая точность вблизи разрывов.

\subsection{Недостатки}
- Сложность реализации.
- Большие вычислительные затраты.

\section{Заключение}
Отображение Эно является мощным инструментом для решения задач с разрывами. Несмотря на сложность реализации, его преимущества делают его незаменимым в ряде приложений.

\begin{thebibliography}{9}
\bibitem{Harten1987}
A. Harten, B. Engquist, S. Osher, and S. R. Chakravarthy, \textit{Uniformly high order accurate essentially non-oscillatory schemes, III}, Journal of Computational Physics, 1987.

\bibitem{Shu1988}
C.-W. Shu and S. Osher, \textit{Efficient implementation of essentially non-oscillatory shock-capturing schemes}, Journal of Computational Physics, 1988.
\end{thebibliography}

\end{document}