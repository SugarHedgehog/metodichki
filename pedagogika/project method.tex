\documentclass[a4paper, 12pt]{extarticle}
\usepackage{fontspec}
\usepackage{polyglossia}
\setmainfont{CMU Serif}
\newfontfamily{\cyrillicfont}{CMU Serif}
\setsansfont{CMU Sans Serif}
\newfontfamily{\cyrillicfontsf}{CMU Sans Serif}
\setmonofont{CMU Typewriter Text}
\newfontfamily{\cyrillicfonttt}{CMU Typewriter Text}
\setdefaultlanguage{russian}
\usepackage[left=1cm,right=1cm,top=1cm,bottom=1.5cm]{geometry}

\title{Проектный метод. Принципы и правила реализации.}
\author{Суматохина Александра 3 курс КТФ}
\date{17 апреля 2023}
\begin{document}
\maketitle

Определение. Проектным методом называют образовательную технологию, центр которой — сам учащийся, а цель — формирование у него самостоятельности, инициативности и креативности. Эти качества развиваются благодаря его собственным действиям в процессе познания интересных и значимых тем. Разбираемся, в чем суть этой технологии и какие у нее отличительные черты.
\subsection*{Суть метода проектного обучения}

Технология проектного обучения возникла в 20-е годы XX века в США и была связана с гуманистическим течением в образовании, которое уделяло повышенное внимание личности и индивидуальности человека. Считалось, что обучение должно быть связано с темами, которые интересуют каждого ученика по отдельности или в группе. Ребенок вовлекается в реальные и близкие ему ситуации, проживает их на собственном опыте, находит методы решения задач и так осваивает навыки и компетенции, новые способы взаимодействия в социокультурной среде. Обучение переходит от теории к практике, теоретические знания соединяются с эмпирическими. Важно, чтобы ребенок воспринимал знания как действительно важные и необходимые.

    В современном мире технология проектного обучения заключается в тезисе «Все, что я познаю, — я знаю, для чего это мне надо, а также где и как я могу эти знания применить».

    \textbf{Основные принципы метода} проектного обучения — диалогичность, проблемность, интегративность и контекстность.

    \textbf{Диалогичность} предполагает вступление учащегося в диалог с собственным «Я» и с другими участниками проекта — так раскрываются особенности личности.

    \textbf{Проблемность} знаменует начало энергичной мыслительной работы, связанной с необходимостью решения заданной ситуации.

    \textbf{Интегративность} определяется наилучшим соединением давно сформировавшихся систем усвоения знаний и правил обучения.

    \textbf{Контекстность} подразумевает разработку проектов, близких к жизни учащихся, и осознание их важности для общества.

    \textbf{Основная цель проектного обучения} — научить детей находить решения без вмешательства взрослого. Учитель лишь мотивирует и направляет ребенка, в случае необходимости подсказывает, где найти нужную информацию.

Кроме основной, у метода проектного обучения есть и другие цели:
\begin{itemize}
    \item создавать мотивацию к обучению;
    \item привлекать каждого члена группы к самостоятельной работе;
    \item совершенствовать познавательные, организаторские, профессиональные и другие способности учащихся;
    \item повышать самооценку детей;
    \item развивать системное, критическое и аналитическое мышление;
    \item учить использовать полученные знания для решения, в т.ч. практических жизненных задач.
\end{itemize}
    

\textbf{Эффективность} проектного обучения заключается в том, что учащиеся мотивированы на самостоятельную работу и поиск информации в разных источниках от библиотек до интернета. Они учатся распределять время и работать с полученными данными, организуют работу в группах и приобретают навыки коллективного или индивидуального принятия решений.

   \textit{Обязательное условие создания проекта — существование четких представлений о конечном итоге деятельности, об этапах работы над проектом и способах его реализации.}

\subsection*{Классификация технологий проектного обучения}

\textbf{По содержанию:}

    
    Монопредметные — на базе одного предмета.

    Межпредметные — в них объединяются знания разных предметов.
    
    Надпредметные — основаны на изучении информации, не входящей в школьную программу.

\textbf{По основному методу:}

    Игровые, приключенческие — основой является ролевая игра по художественным произведениям или историческим событиям. Также это может быть имитация археологической или морской экспедиции. Финал, как правило, не запланирован.
    
    Исследовательские, творческие — обладают четко поставленной целью и понятной структурой. Имеют много общего с научными исследованиями. Например, социологические опросы небольшой группы людей на важную для учеников тему.
    
    Информационные, ориентированные на практический результат — обычно затрагивают темы, интересные участникам проекта. Примером может служить создание альбома или справочного материала, организация театральной постановки, очистка природных зон. 
    
    Популярны международные проекты, поучаствовать в которых можно через интернет.

    \textbf{По характеру координирования проекта:}

    С явной координацией — учитель активно включается в работу.
    
    Со скрытой координацией — учитель лишь направляет и дает подсказки.
    
    По включенности проектов в учебные планы:
    
    Текущие — затрагивают учебное время.
    
    Итоговые — помогают оценить, как материал усвоен учащимися.

    \textbf{По продолжительности выполнения:}

    Мини-проекты — занимают один-два урока.
    Средние проекты — от нескольких уроков до двух недель.
    Долгосрочные проекты — от нескольких недель до нескольких месяцев. Обычно работа над ними проводится в свободные от занятий часы.

    \textbf{По количеству участников:}

    Индивидуальные — работает один ребенок.
    Групповые — вовлечены несколько учеников.
    Коллективные — затрагивает учащихся нескольких классов или всей школы.

\subsection*{Этапы проектной деятельности}

\textbf{1 этап. Подготовка}

Педагог выбирает интересные темы, а затем обсуждает их вместе с учащимися. Вместе они делают выбор в пользу одной из них, если это групповой проект. Ребята могут и сами предложить актуальную для них тему.

Затем учитель разделяет тему на подтемы, из которых вновь нужно выбрать интересующую.

Класс самостоятельно делится на подгруппы, однако преподаватель может распределить учеников так, как считает нужным.

Учитель ищет подходящую литературу и другие источники информации, ставит вопросы, требующие ответа. По желанию ученики принимают в этом участие.

Учитель или учащиеся принимают решение по поводу оформления результатов работы.

\textbf{2 этап. Разработка проекта}

Ученики активно работают над проектом, а учитель мотивирует, направляет и консультирует по любым возникающим вопросам. Этот период занимает больше всего времени.

\textbf{3 этап. Оформление итогов}

Оформление результатов проекта в соответствии с принятыми ранее правилами. Это может быть доклад, макет, презентация и пр.

\textbf{4 этап. Презентация}

Демонстрация итогов работы над проектом перед учителем, родителями, учениками других классов или школ. Может проходить в игровой форме. Учащиеся рассказывают, какие приемы использовали для получения информации, с какими проблемами столкнулись, чему научились и к каким выводам пришли, показывают результат работы.

\textbf{5 этап. Рефлексия}

Учитель и ученики оценивают проделанную работу, делятся впечатлениями, обсуждают и ставят оценки.

Педагоги отмечают, достигнута ли цель проекта, полностью ли раскрыта тема и какова социальная значимость проведенной работы. Также важны доступность изложения, умение ученика ответить на вопросы по теме и влияние проекта на развитие его личности.

Чтобы проектное обучение приносило пользу и отвечало поставленным задачам, необходима правильная подготовка как учителей, так и учеников. Это должен быть целостный системный подход.

Педагог перестает быть просто источником знаний, а становится организатором и координатором исследовательской работы. Он поддерживает учеников на каждом этапе работы над проектом, создает творческую среду, стимулирует постоянное личностное развитие.

Приоритетное условие успеха — мотивированность учащихся, их заинтересованность и понимание важности подобной работы. Учащиеся детально прорабатывают интересующую тему, пробуют себя в разных ролях, что может оказаться полезным при выборе будущей профессии.

\end{document}