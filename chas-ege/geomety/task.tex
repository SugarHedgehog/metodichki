\documentclass[4apaper]{article}
\usepackage{dashbox}
\usepackage[T2A]{fontenc}
\usepackage[utf8]{inputenc}
\usepackage[english,russian]{babel}
\usepackage{graphicx}
\DeclareGraphicsExtensions{.pdf,.png,.jpg}

\linespread{1.15}

\usepackage{../egetask_ver}

\def\examyear{2023}
\usepackage[colorlinks,linkcolor=blue]{hyperref}

\begin{document}
\begin{taskBN}{1}
Дана геометрическая прогрессия, для которой $b_1 = -7$, $q=0,4$. Найдите $S_{3}$.
\end{taskBN}

\begin{taskBN}{2}
Дана геометрическая прогрессия, для которой $b_{3} = 100$, $b_{6}=-12500$. Найдите $b_{4}$.
\end{taskBN}

\begin{taskBN}{3}
Дана геометрическая прогрессия, для которой $b_1 = 8$, $q=4$. Найдите $S_{4}$.
\end{taskBN}

\begin{taskBN}{4}
Компания «Альфа» начала инвестировать средства в перспективную отрасль в 2001 году, имея капитал в размере 6500 долларов. Каждый год, начиная с 2002 года, она получала прибыль, которая составляла 200\% от капитала предыдущего года. А компания «Бета» начала инвестировать средства в другую отрасль в 2005 году, имея капитал в размере 8000 долларов, и, начиная с 2006 года, ежегодно получала прибыль, составляющую 400\% от капитала предыдущего года. На сколько долларов капитал одной из компаний был больше капитала другой к концу 2007 года, если прибыль из оборота не изымалась?
\end{taskBN}

\begin{taskBN}{5}
Компания «Альфа» начала инвестировать средства в перспективную отрасль в 2001 году, имея капитал в размере 5000 долларов. Каждый год, начиная с 2002 года, она получала прибыль, которая составляла 100\% от капитала предыдущего года. А компания «Бета» начала инвестировать средства в другую отрасль в 2004 году, имея капитал в размере 6500 долларов, и, начиная с 2005 года, ежегодно получала прибыль, составляющую 200\% от капитала предыдущего года. На сколько долларов капитал одной из компаний был больше капитала другой к концу 2006 года, если прибыль из оборота не изымалась?
\end{taskBN}

\begin{taskBN}{6}
Дана геометрическая прогрессия, для которой $b_1 = -4$, $b_{5}=-5184$. Найдите $q$.
\end{taskBN}

\begin{taskBN}{7}
Дана геометрическая прогрессия, для которой $b_{4} = 162$, $b_{5}=486$. Найдите $b_{3}$.
\end{taskBN}

\begin{taskBN}{8}
Дана геометрическая прогрессия, для которой $b_1 = 8$, $q=8$. Найдите $b_{8}$.
\end{taskBN}

\begin{taskBN}{9}
Бизнесмен Плюшкин получил в 2000 году прибыль в размере 1400000 рублей. Каждый следующий год его прибыль увеличивалась на 10\% по сравнению с предыдущим годом. Сколько рублей заработал Плюшкин за 2004 год?
\end{taskBN}

\begin{taskBN}{10}
Дана геометрическая прогрессия, для которой $b_{4} = -512$, $b_{5}=2048$. Найдите $b_{7}$.
\end{taskBN}

\begin{taskBN}{11}
Чему равен знаменатель бесконечной геометрической прогрессии, если $b_1 = -28$ и сумма всех её членов $S=-112$?
\end{taskBN}

\begin{taskBN}{12}
Дана геометрическая прогрессия, для которой $b_1 = -7$, $b_{8}=-15309$. Найдите $b_{6}$.
\end{taskBN}

\begin{taskBN}{13}
Дана геометрическая прогрессия, для которой $b_1 = -3$, $b_{6}=98304$. Найдите $b_{7}$.
\end{taskBN}

\begin{taskBN}{14}
Чему равен знаменатель бесконечной геометрической прогрессии, если $b_1 = 3$ и сумма всех её членов $S=30$?
\end{taskBN}

\begin{taskBN}{15}
Дана геометрическая прогрессия, для которой $b_1 = -4$, $b_{5}=-1024$. Найдите $q$.
\end{taskBN}

\begin{taskBN}{16}
Дана геометрическая прогрессия, для которой $b_{2} = 10$, $b_{7}=-31250$. Найдите $b_{4}$.
\end{taskBN}

\begin{taskBN}{17}
Чему равна сумма бесконечной геометрической прогрессии, для которой $b_1 = 22$ и $q=0,75$?
\end{taskBN}

\begin{taskBN}{18}
Дана геометрическая прогрессия, для которой $b_1 = -2$, $q=3$. Найдите $S_{3}$.
\end{taskBN}

\begin{taskBN}{19}
Чему равна сумма бесконечной геометрической прогрессии, для которой $b_1 = 4$ и $q=0,5$?
\end{taskBN}

\begin{taskBN}{20}
Дана геометрическая прогрессия, для которой $b_1 = -3$, $b_{8}=-839808$. Найдите $q$.
\end{taskBN}

\begin{taskBN}{21}
Дана геометрическая прогрессия, для которой $b_1 = -4$, $b_{4}=256$. Найдите $q$.
\end{taskBN}

\begin{taskBN}{22}
Дана геометрическая прогрессия, для которой $b_1 = 8$, $q=6$. Найдите $b_{5}$.
\end{taskBN}

\begin{taskBN}{23}
Чему равна сумма бесконечной геометрической прогрессии, для которой $b_1 = -7$ и $q=0,8$?
\end{taskBN}

\begin{taskBN}{24}
Дана геометрическая прогрессия, для которой $b_1 = -3$, $q=0,4$. Найдите $S_{6}$.
\end{taskBN}

\begin{taskBN}{25}
Дана геометрическая прогрессия, для которой $b_1 = 5$, $b_{6}=295245$. Найдите $b_{4}$.
\end{taskBN}

\begin{taskBN}{26}
Дана геометрическая прогрессия, для которой $b_1 = -8$, $q=0,4$. Найдите $S_{4}$.
\end{taskBN}

\begin{taskBN}{27}
Бизнесмен Печенов получил в 2000 году прибыль в размере 800000 рублей. Каждый следующий год его прибыль увеличивалась на 7\% по сравнению с предыдущим годом. Сколько рублей заработал Печенов за 2002 год?
\end{taskBN}

\begin{taskBN}{28}
Дана геометрическая прогрессия, для которой $b_1 = -4$, $q=0,3$. Найдите $S_{5}$.
\end{taskBN}

\begin{taskBN}{29}
Дана геометрическая прогрессия, для которой $b_{5} = 405$, $b_{4}=-135$. Найдите $b_{7}$.
\end{taskBN}

\begin{taskBN}{30}
Дана геометрическая прогрессия, для которой $b_1 = 2$, $q=0,2$. Найдите $S_{5}$.
\end{taskBN}

\begin{taskBN}{31}
Дана геометрическая прогрессия, для которой $b_1 = -4$, $q=2$. Найдите $b_{8}$.
\end{taskBN}

\begin{taskBN}{32}
Дана геометрическая прогрессия, для которой $b_{7} = 139968$, $b_{4}=648$. Найдите $b_{5}$.
\end{taskBN}

\begin{taskBN}{33}
Бизнесмен Батонов получил в 2000 году прибыль в размере 500000 рублей. Каждый следующий год его прибыль увеличивалась на 17\% по сравнению с предыдущим годом. Сколько рублей заработал Батонов за 2002 год?
\end{taskBN}

\begin{taskBN}{34}
Дана геометрическая прогрессия, для которой $b_1 = -6$, $q=4$. Найдите $S_{7}$.
\end{taskBN}

\begin{taskBN}{35}
Бизнесмен Печенов получил в 2000 году прибыль в размере 1500000 рублей. Каждый следующий год его прибыль увеличивалась на 10\% по сравнению с предыдущим годом. Сколько рублей заработал Печенов за 2004 год?
\end{taskBN}

\begin{taskBN}{36}
Чему равен знаменатель бесконечной геометрической прогрессии, если $b_1 = 26$ и сумма всех её членов $S=130$?
\end{taskBN}

\begin{taskBN}{37}
Дана геометрическая прогрессия, для которой $b_1 = -4$, $q=3$. Найдите $S_{5}$.
\end{taskBN}

\begin{taskBN}{38}
Компания «Альфа» начала инвестировать средства в перспективную отрасль в 2001 году, имея капитал в размере 7000 долларов. Каждый год, начиная с 2002 года, она получала прибыль, которая составляла 100\% от капитала предыдущего года. А компания «Бета» начала инвестировать средства в другую отрасль в 2004 году, имея капитал в размере 9000 долларов, и, начиная с 2005 года, ежегодно получала прибыль, составляющую 200\% от капитала предыдущего года. На сколько долларов капитал одной из компаний был больше капитала другой к концу 2006 года, если прибыль из оборота не изымалась?
\end{taskBN}

\begin{taskBN}{39}
Дана геометрическая прогрессия, для которой $b_1 = -2$, $q=2$. Найдите $S_{4}$.
\end{taskBN}

\begin{taskBN}{40}
Дана геометрическая прогрессия, для которой $b_1 = -2$, $b_{4}=-432$. Найдите $b_{7}$.
\end{taskBN}

\begin{taskBN}{41}
Дана геометрическая прогрессия, для которой $b_1 = -3$, $b_{8}=-2470629$. Найдите $b_{3}$.
\end{taskBN}

\begin{taskBN}{42}
Чему равна сумма бесконечной геометрической прогрессии, для которой $b_1 = 6$ и $q=0,8$?
\end{taskBN}

\begin{taskBN}{43}
Дана геометрическая прогрессия, для которой $b_1 = -5$, $q=3$. Найдите $S_{6}$.
\end{taskBN}

\begin{taskBN}{44}
Дана геометрическая прогрессия, для которой $b_1 = 3$, $q=4$. Найдите $S_{5}$.
\end{taskBN}

\begin{taskBN}{45}
Дана геометрическая прогрессия, для которой $b_1 = -3$, $q=3$. Найдите $b_{4}$.
\end{taskBN}

\begin{taskBN}{46}
Чему равна сумма бесконечной геометрической прогрессии, для которой $b_1 = 3$ и $q=0,75$?
\end{taskBN}

\begin{taskBN}{47}
Компания «Альфа» начала инвестировать средства в перспективную отрасль в 2001 году, имея капитал в размере 5500 долларов. Каждый год, начиная с 2002 года, она получала прибыль, которая составляла 100\% от капитала предыдущего года. А компания «Бета» начала инвестировать средства в другую отрасль в 2006 году, имея капитал в размере 7500 долларов, и, начиная с 2007 года, ежегодно получала прибыль, составляющую 200\% от капитала предыдущего года. На сколько долларов капитал одной из компаний был больше капитала другой к концу 2008 года, если прибыль из оборота не изымалась?
\end{taskBN}

\begin{taskBN}{48}
Дана геометрическая прогрессия, для которой $b_1 = 6$, $b_{6}=1458$. Найдите $b_{5}$.
\end{taskBN}

\begin{taskBN}{49}
Чему равна сумма бесконечной геометрической прогрессии, для которой $b_1 = 25$ и $q=0,8$?
\end{taskBN}

\begin{taskBN}{50}
Дана геометрическая прогрессия, для которой $b_1 = -6$, $b_{4}=384$. Найдите $b_{6}$.
\end{taskBN}

\begin{taskBN}{51}
Компания «Альфа» начала инвестировать средства в перспективную отрасль в 2001 году, имея капитал в размере 5000 долларов. Каждый год, начиная с 2002 года, она получала прибыль, которая составляла 200\% от капитала предыдущего года. А компания «Бета» начала инвестировать средства в другую отрасль в 2004 году, имея капитал в размере 6500 долларов, и, начиная с 2005 года, ежегодно получала прибыль, составляющую 400\% от капитала предыдущего года. На сколько долларов капитал одной из компаний был больше капитала другой к концу 2006 года, если прибыль из оборота не изымалась?
\end{taskBN}

\begin{taskBN}{52}
Дана геометрическая прогрессия, для которой $b_1 = 9$, $b_{6}=531441$. Найдите $q$.
\end{taskBN}

\begin{taskBN}{53}
Дана геометрическая прогрессия, для которой $b_{4} = -1944$, $b_{7}=419904$. Найдите $b_{5}$.
\end{taskBN}

\begin{taskBN}{54}
Дана геометрическая прогрессия, для которой $b_1 = -8$, $q=-8$. Найдите $b_{7}$.
\end{taskBN}

\begin{taskBN}{55}
Компания «Альфа» начала инвестировать средства в перспективную отрасль в 2001 году, имея капитал в размере 6000 долларов. Каждый год, начиная с 2002 года, она получала прибыль, которая составляла 100\% от капитала предыдущего года. А компания «Бета» начала инвестировать средства в другую отрасль в 2006 году, имея капитал в размере 7500 долларов, и, начиная с 2007 года, ежегодно получала прибыль, составляющую 300\% от капитала предыдущего года. На сколько долларов капитал одной из компаний был больше капитала другой к концу 2008 года, если прибыль из оборота не изымалась?
\end{taskBN}

\begin{taskBN}{56}
Чему равен знаменатель бесконечной геометрической прогрессии, если $b_1 = -15$ и сумма всех её членов $S=-60$?
\end{taskBN}

\begin{taskBN}{57}
Компания «Альфа» начала инвестировать средства в перспективную отрасль в 2001 году, имея капитал в размере 5000 долларов. Каждый год, начиная с 2002 года, она получала прибыль, которая составляла 200\% от капитала предыдущего года. А компания «Бета» начала инвестировать средства в другую отрасль в 2003 году, имея капитал в размере 7000 долларов, и, начиная с 2004 года, ежегодно получала прибыль, составляющую 400\% от капитала предыдущего года. На сколько долларов капитал одной из компаний был больше капитала другой к концу 2006 года, если прибыль из оборота не изымалась?
\end{taskBN}

\begin{taskBN}{58}
Компания «Альфа» начала инвестировать средства в перспективную отрасль в 2001 году, имея капитал в размере 3500 долларов. Каждый год, начиная с 2002 года, она получала прибыль, которая составляла 100\% от капитала предыдущего года. А компания «Бета» начала инвестировать средства в другую отрасль в 2006 году, имея капитал в размере 6000 долларов, и, начиная с 2007 года, ежегодно получала прибыль, составляющую 300\% от капитала предыдущего года. На сколько долларов капитал одной из компаний был больше капитала другой к концу 2008 года, если прибыль из оборота не изымалась?
\end{taskBN}

\begin{taskBN}{59}
Компания «Альфа» начала инвестировать средства в перспективную отрасль в 2001 году, имея капитал в размере 2500 долларов. Каждый год, начиная с 2002 года, она получала прибыль, которая составляла 200\% от капитала предыдущего года. А компания «Бета» начала инвестировать средства в другую отрасль в 2005 году, имея капитал в размере 3500 долларов, и, начиная с 2006 года, ежегодно получала прибыль, составляющую 400\% от капитала предыдущего года. На сколько долларов капитал одной из компаний был больше капитала другой к концу 2007 года, если прибыль из оборота не изымалась?
\end{taskBN}

\begin{taskBN}{60}
Дана геометрическая прогрессия, для которой $b_1 = 3$, $b_{4}=1536$. Найдите $q$.
\end{taskBN}

\begin{taskBN}{61}
Дана геометрическая прогрессия, для которой $b_1 = -6$, $q=-3$. Найдите $b_{8}$.
\end{taskBN}

\begin{taskBN}{62}
Дана геометрическая прогрессия, для которой $b_1 = -7$, $q=0,5$. Найдите $S_{5}$.
\end{taskBN}

\begin{taskBN}{63}
Дана геометрическая прогрессия, для которой $b_1 = 3$, $q=0,6$. Найдите $S_{4}$.
\end{taskBN}

\begin{taskBN}{64}
Дана геометрическая прогрессия, для которой $b_1 = -7$, $q=0,3$. Найдите $S_{5}$.
\end{taskBN}

\begin{taskBN}{65}
Дана геометрическая прогрессия, для которой $b_1 = 6$, $q=6$. Найдите $S_{7}$.
\end{taskBN}

\begin{taskBN}{66}
Бизнесмен Баранкин получил в 2000 году прибыль в размере 1000000 рублей. Каждый следующий год его прибыль увеличивалась на 10\% по сравнению с предыдущим годом. Сколько рублей заработал Баранкин за 2004 год?
\end{taskBN}

\begin{taskBN}{67}
Дана геометрическая прогрессия, для которой $b_1 = 3$, $q=3$. Найдите $S_{3}$.
\end{taskBN}

\begin{taskBN}{68}
Бизнесмен Прянников получил в 2000 году прибыль в размере 1300000 рублей. Каждый следующий год его прибыль увеличивалась на 10\% по сравнению с предыдущим годом. Сколько рублей заработал Прянников за 2004 год?
\end{taskBN}

\begin{taskBN}{69}
Дана геометрическая прогрессия, для которой $b_1 = -4$, $q=3$. Найдите $b_{3}$.
\end{taskBN}

\begin{taskBN}{70}
Дана геометрическая прогрессия, для которой $b_1 = 7$, $q=0,4$. Найдите $S_{3}$.
\end{taskBN}

\begin{taskBN}{71}
Дана геометрическая прогрессия, для которой $b_1 = 3$, $b_{6}=729$. Найдите $q$.
\end{taskBN}

\begin{taskBN}{72}
Дана геометрическая прогрессия, для которой $b_1 = -6$, $b_{6}=-100842$. Найдите $b_{4}$.
\end{taskBN}

\begin{taskBN}{73}
Чему равна сумма бесконечной геометрической прогрессии, для которой $b_1 = -26$ и $q=0,9$?
\end{taskBN}

\begin{taskBN}{74}
Бизнесмен Коржов получил в 2000 году прибыль в размере 1000000 рублей. Каждый следующий год его прибыль увеличивалась на 20\% по сравнению с предыдущим годом. Сколько рублей заработал Коржов за 2004 год?
\end{taskBN}

\begin{taskBN}{75}
Чему равна сумма бесконечной геометрической прогрессии, для которой $b_1 = -16$ и $q=0,75$?
\end{taskBN}

\begin{taskBN}{76}
Дана геометрическая прогрессия, для которой $b_1 = -4$, $q=-3$. Найдите $b_{7}$.
\end{taskBN}

\begin{taskBN}{77}
Дана геометрическая прогрессия, для которой $b_1 = -5$, $q=-6$. Найдите $b_{3}$.
\end{taskBN}

\begin{taskBN}{78}
Дана геометрическая прогрессия, для которой $b_1 = -2$, $q=0,4$. Найдите $S_{5}$.
\end{taskBN}

\begin{taskBN}{79}
Дана геометрическая прогрессия, для которой $b_1 = 6$, $q=-9$. Найдите $b_{8}$.
\end{taskBN}

\begin{taskBN}{80}
Дана геометрическая прогрессия, для которой $b_1 = 7$, $b_{6}=117649$. Найдите $q$.
\end{taskBN}

\begin{taskBN}{81}
Дана геометрическая прогрессия, для которой $b_{7} = 2187$, $b_{8}=6561$. Найдите $b_{5}$.
\end{taskBN}

\begin{taskBN}{82}
Компания «Альфа» начала инвестировать средства в перспективную отрасль в 2001 году, имея капитал в размере 6500 долларов. Каждый год, начиная с 2002 года, она получала прибыль, которая составляла 200\% от капитала предыдущего года. А компания «Бета» начала инвестировать средства в другую отрасль в 2006 году, имея капитал в размере 8500 долларов, и, начиная с 2007 года, ежегодно получала прибыль, составляющую 300\% от капитала предыдущего года. На сколько долларов капитал одной из компаний был больше капитала другой к концу 2008 года, если прибыль из оборота не изымалась?
\end{taskBN}

\begin{taskBN}{83}
Дана геометрическая прогрессия, для которой $b_1 = -8$, $b_{4}=1728$. Найдите $q$.
\end{taskBN}

\begin{taskBN}{84}
Дана геометрическая прогрессия, для которой $b_{6} = -2048$, $b_{3}=32$. Найдите $b_{5}$.
\end{taskBN}

\begin{taskBN}{85}
Компания «Альфа» начала инвестировать средства в перспективную отрасль в 2001 году, имея капитал в размере 6000 долларов. Каждый год, начиная с 2002 года, она получала прибыль, которая составляла 200\% от капитала предыдущего года. А компания «Бета» начала инвестировать средства в другую отрасль в 2006 году, имея капитал в размере 8500 долларов, и, начиная с 2007 года, ежегодно получала прибыль, составляющую 300\% от капитала предыдущего года. На сколько долларов капитал одной из компаний был больше капитала другой к концу 2008 года, если прибыль из оборота не изымалась?
\end{taskBN}

\begin{taskBN}{86}
Бизнесмен Плюшкин получил в 2000 году прибыль в размере 1000000 рублей. Каждый следующий год его прибыль увеличивалась на 10\% по сравнению с предыдущим годом. Сколько рублей заработал Плюшкин за 2003 год?
\end{taskBN}

\begin{taskBN}{87}
Дана геометрическая прогрессия, для которой $b_1 = 2$, $b_{7}=235298$. Найдите $q$.
\end{taskBN}

\begin{taskBN}{88}
Дана геометрическая прогрессия, для которой $b_1 = -8$, $b_{6}=134456$. Найдите $q$.
\end{taskBN}

\begin{taskBN}{89}
Чему равна сумма бесконечной геометрической прогрессии, для которой $b_1 = 19$ и $q=0,8$?
\end{taskBN}

\begin{taskBN}{90}
Дана геометрическая прогрессия, для которой $b_1 = 6$, $b_{8}=-98304$. Найдите $q$.
\end{taskBN}

\begin{taskBN}{91}
Дана геометрическая прогрессия, для которой $b_1 = 6$, $b_{6}=354294$. Найдите $q$.
\end{taskBN}

\begin{taskBN}{92}
Дана геометрическая прогрессия, для которой $b_{5} = 2048$, $b_{6}=-8192$. Найдите $b_{3}$.
\end{taskBN}

\begin{taskBN}{93}
Дана геометрическая прогрессия, для которой $b_1 = -6$, $q=5$. Найдите $S_{4}$.
\end{taskBN}

\begin{taskBN}{94}
Дана геометрическая прогрессия, для которой $b_{5} = -5184$, $b_{6}=-31104$. Найдите $b_{3}$.
\end{taskBN}

\begin{taskBN}{95}
Дана геометрическая прогрессия, для которой $b_{8} = -625000$, $b_{5}=-5000$. Найдите $b_{6}$.
\end{taskBN}

\begin{taskBN}{96}
Дана геометрическая прогрессия, для которой $b_1 = -4$, $q=-6$. Найдите $b_{5}$.
\end{taskBN}

\begin{taskBN}{97}
Дана геометрическая прогрессия, для которой $b_{2} = -16$, $b_{5}=1024$. Найдите $b_{7}$.
\end{taskBN}

\begin{taskBN}{98}
Компания «Альфа» начала инвестировать средства в перспективную отрасль в 2001 году, имея капитал в размере 3000 долларов. Каждый год, начиная с 2002 года, она получала прибыль, которая составляла 200\% от капитала предыдущего года. А компания «Бета» начала инвестировать средства в другую отрасль в 2004 году, имея капитал в размере 5000 долларов, и, начиная с 2005 года, ежегодно получала прибыль, составляющую 300\% от капитала предыдущего года. На сколько долларов капитал одной из компаний был больше капитала другой к концу 2007 года, если прибыль из оборота не изымалась?
\end{taskBN}

\begin{taskBN}{99}
Чему равна сумма бесконечной геометрической прогрессии, для которой $b_1 = 19$ и $q=0,8$?
\end{taskBN}

\begin{taskBN}{100}
Дана геометрическая прогрессия, для которой $b_{3} = 200$, $b_{4}=-1000$. Найдите $b_{8}$.
\end{taskBN}

\begin{taskBN}{101}
Компания «Альфа» начала инвестировать средства в перспективную отрасль в 2001 году, имея капитал в размере 5000 долларов. Каждый год, начиная с 2002 года, она получала прибыль, которая составляла 100\% от капитала предыдущего года. А компания «Бета» начала инвестировать средства в другую отрасль в 2006 году, имея капитал в размере 6000 долларов, и, начиная с 2007 года, ежегодно получала прибыль, составляющую 200\% от капитала предыдущего года. На сколько долларов капитал одной из компаний был больше капитала другой к концу 2008 года, если прибыль из оборота не изымалась?
\end{taskBN}

\begin{taskBN}{102}
Дана геометрическая прогрессия, для которой $b_1 = 3$, $b_{6}=-23328$. Найдите $q$.
\end{taskBN}

\begin{taskBN}{103}
Дана геометрическая прогрессия, для которой $b_{7} = 12288$, $b_{8}=49152$. Найдите $b_{5}$.
\end{taskBN}

\begin{taskBN}{104}
Дана геометрическая прогрессия, для которой $b_1 = 3$, $q=5$. Найдите $S_{6}$.
\end{taskBN}

\begin{taskBN}{105}
Чему равна сумма бесконечной геометрической прогрессии, для которой $b_1 = -20$ и $q=0,8$?
\end{taskBN}

\begin{taskBN}{106}
Компания «Альфа» начала инвестировать средства в перспективную отрасль в 2001 году, имея капитал в размере 7000 долларов. Каждый год, начиная с 2002 года, она получала прибыль, которая составляла 200\% от капитала предыдущего года. А компания «Бета» начала инвестировать средства в другую отрасль в 2005 году, имея капитал в размере 8000 долларов, и, начиная с 2006 года, ежегодно получала прибыль, составляющую 300\% от капитала предыдущего года. На сколько долларов капитал одной из компаний был больше капитала другой к концу 2008 года, если прибыль из оборота не изымалась?
\end{taskBN}

\begin{taskBN}{107}
Чему равна сумма бесконечной геометрической прогрессии, для которой $b_1 = 29$ и $q=0,5$?
\end{taskBN}

\begin{taskBN}{108}
Дана геометрическая прогрессия, для которой $b_1 = 9$, $q=0,6$. Найдите $S_{4}$.
\end{taskBN}

\begin{taskBN}{109}
Дана геометрическая прогрессия, для которой $b_1 = -9$, $b_{3}=-324$. Найдите $q$.
\end{taskBN}

\begin{taskBN}{110}
Бизнесмен Батонов получил в 2000 году прибыль в размере 800000 рублей. Каждый следующий год его прибыль увеличивалась на 18\% по сравнению с предыдущим годом. Сколько рублей заработал Батонов за 2002 год?
\end{taskBN}

\begin{taskBN}{111}
Дана геометрическая прогрессия, для которой $b_1 = -3$, $b_{8}=-2470629$. Найдите $b_{6}$.
\end{taskBN}

\begin{taskBN}{112}
Дана геометрическая прогрессия, для которой $b_1 = 8$, $b_{8}=-131072$. Найдите $b_{7}$.
\end{taskBN}

\begin{taskBN}{113}
Компания «Альфа» начала инвестировать средства в перспективную отрасль в 2001 году, имея капитал в размере 5000 долларов. Каждый год, начиная с 2002 года, она получала прибыль, которая составляла 100\% от капитала предыдущего года. А компания «Бета» начала инвестировать средства в другую отрасль в 2004 году, имея капитал в размере 6500 долларов, и, начиная с 2005 года, ежегодно получала прибыль, составляющую 300\% от капитала предыдущего года. На сколько долларов капитал одной из компаний был больше капитала другой к концу 2007 года, если прибыль из оборота не изымалась?
\end{taskBN}

\begin{taskBN}{114}
Бизнесмен Булкин получил в 2000 году прибыль в размере 600000 рублей. Каждый следующий год его прибыль увеличивалась на 10\% по сравнению с предыдущим годом. Сколько рублей заработал Булкин за 2002 год?
\end{taskBN}

\begin{taskBN}{115}
Дана геометрическая прогрессия, для которой $b_1 = -4$, $b_{8}=19131876$. Найдите $q$.
\end{taskBN}

\begin{taskBN}{116}
Дана геометрическая прогрессия, для которой $b_1 = 4$, $b_{6}=12500$. Найдите $b_{7}$.
\end{taskBN}

\begin{taskBN}{117}
Дана геометрическая прогрессия, для которой $b_1 = 6$, $b_{5}=24576$. Найдите $q$.
\end{taskBN}

\begin{taskBN}{118}
Дана геометрическая прогрессия, для которой $b_1 = 4$, $q=-6$. Найдите $b_{6}$.
\end{taskBN}

\begin{taskBN}{119}
Дана геометрическая прогрессия, для которой $b_1 = -2$, $b_{6}=-2048$. Найдите $b_{8}$.
\end{taskBN}

\begin{taskBN}{120}
Бизнесмен Плюшкин получил в 2000 году прибыль в размере 700000 рублей. Каждый следующий год его прибыль увеличивалась на 9\% по сравнению с предыдущим годом. Сколько рублей заработал Плюшкин за 2002 год?
\end{taskBN}

\begin{taskBN}{121}
Чему равен знаменатель бесконечной геометрической прогрессии, если $b_1 = 19$ и сумма всех её членов $S=76$?
\end{taskBN}

\begin{taskBN}{122}
Бизнесмен Коржов получил в 2000 году прибыль в размере 1100000 рублей. Каждый следующий год его прибыль увеличивалась на 10\% по сравнению с предыдущим годом. Сколько рублей заработал Коржов за 2003 год?
\end{taskBN}

\begin{taskBN}{123}
Дана геометрическая прогрессия, для которой $b_1 = -7$, $q=6$. Найдите $S_{4}$.
\end{taskBN}

\begin{taskBN}{124}
Дана геометрическая прогрессия, для которой $b_1 = -4$, $b_{6}=236196$. Найдите $b_{4}$.
\end{taskBN}

\begin{taskBN}{125}
Дана геометрическая прогрессия, для которой $b_{1} = -6$, $b_{6}=6144$. Найдите $b_{3}$.
\end{taskBN}

\begin{taskBN}{126}
Чему равна сумма бесконечной геометрической прогрессии, для которой $b_1 = 22$ и $q=0,8$?
\end{taskBN}

\begin{taskBN}{127}
Компания «Альфа» начала инвестировать средства в перспективную отрасль в 2001 году, имея капитал в размере 2500 долларов. Каждый год, начиная с 2002 года, она получала прибыль, которая составляла 200\% от капитала предыдущего года. А компания «Бета» начала инвестировать средства в другую отрасль в 2004 году, имея капитал в размере 3000 долларов, и, начиная с 2005 года, ежегодно получала прибыль, составляющую 300\% от капитала предыдущего года. На сколько долларов капитал одной из компаний был больше капитала другой к концу 2007 года, если прибыль из оборота не изымалась?
\end{taskBN}

\begin{taskBN}{128}
Дана геометрическая прогрессия, для которой $b_{6} = -28125$, $b_{7}=140625$. Найдите $b_{4}$.
\end{taskBN}

\begin{taskBN}{129}
Дана геометрическая прогрессия, для которой $b_1 = 3$, $b_{6}=177147$. Найдите $q$.
\end{taskBN}

\begin{taskBN}{130}
Дана геометрическая прогрессия, для которой $b_{8} = -546875$, $b_{5}=4375$. Найдите $b_{7}$.
\end{taskBN}

\begin{taskBN}{131}
Дана геометрическая прогрессия, для которой $b_1 = 2$, $q=-4$. Найдите $b_{7}$.
\end{taskBN}

\begin{taskBN}{132}
Дана геометрическая прогрессия, для которой $b_1 = 8$, $q=3$. Найдите $S_{4}$.
\end{taskBN}

\begin{taskBN}{133}
Чему равна сумма бесконечной геометрической прогрессии, для которой $b_1 = 22$ и $q=0,8$?
\end{taskBN}

\begin{taskBN}{134}
Чему равен знаменатель бесконечной геометрической прогрессии, если $b_1 = -12$ и сумма всех её членов $S=-24$?
\end{taskBN}

\begin{taskBN}{135}
Дана геометрическая прогрессия, для которой $b_1 = -5$, $q=3$. Найдите $S_{4}$.
\end{taskBN}

\begin{taskBN}{136}
Бизнесмен Баранкин получил в 2000 году прибыль в размере 1400000 рублей. Каждый следующий год его прибыль увеличивалась на 20\% по сравнению с предыдущим годом. Сколько рублей заработал Баранкин за 2003 год?
\end{taskBN}

\begin{taskBN}{137}
Дана геометрическая прогрессия, для которой $b_1 = 6$, $q=-10$. Найдите $b_{8}$.
\end{taskBN}

\begin{taskBN}{138}
Чему равна сумма бесконечной геометрической прогрессии, для которой $b_1 = 8$ и $q=0,75$?
\end{taskBN}

\begin{taskBN}{139}
Дана геометрическая прогрессия, для которой $b_1 = -3$, $q=9$. Найдите $b_{8}$.
\end{taskBN}

\begin{taskBN}{140}
Компания «Альфа» начала инвестировать средства в перспективную отрасль в 2001 году, имея капитал в размере 6000 долларов. Каждый год, начиная с 2002 года, она получала прибыль, которая составляла 100\% от капитала предыдущего года. А компания «Бета» начала инвестировать средства в другую отрасль в 2006 году, имея капитал в размере 6500 долларов, и, начиная с 2007 года, ежегодно получала прибыль, составляющую 300\% от капитала предыдущего года. На сколько долларов капитал одной из компаний был больше капитала другой к концу 2009 года, если прибыль из оборота не изымалась?
\end{taskBN}

\begin{taskBN}{141}
Дана геометрическая прогрессия, для которой $b_1 = 9$, $q=9$. Найдите $b_{5}$.
\end{taskBN}

\begin{taskBN}{142}
Дана геометрическая прогрессия, для которой $b_1 = 3$, $q=0,2$. Найдите $S_{5}$.
\end{taskBN}

\begin{taskBN}{143}
Дана геометрическая прогрессия, для которой $b_1 = 8$, $q=3$. Найдите $b_{8}$.
\end{taskBN}

\begin{taskBN}{144}
Дана геометрическая прогрессия, для которой $b_{8} = -49152$, $b_{5}=768$. Найдите $b_{4}$.
\end{taskBN}

\begin{taskBN}{145}
Чему равна сумма бесконечной геометрической прогрессии, для которой $b_1 = 15$ и $q=0,75$?
\end{taskBN}

\begin{taskBN}{146}
Дана геометрическая прогрессия, для которой $b_1 = -8$, $b_{7}=-32768$. Найдите $q$.
\end{taskBN}

\begin{taskBN}{147}
Дана геометрическая прогрессия, для которой $b_1 = -3$, $q=-7$. Найдите $b_{6}$.
\end{taskBN}

\begin{taskBN}{148}
Дана геометрическая прогрессия, для которой $b_1 = 5$, $b_{6}=-15625$. Найдите $q$.
\end{taskBN}

\begin{taskBN}{149}
Дана геометрическая прогрессия, для которой $b_1 = -6$, $b_{6}=6144$. Найдите $b_{7}$.
\end{taskBN}

\begin{taskBN}{150}
Дана геометрическая прогрессия, для которой $b_1 = 3$, $q=-4$. Найдите $b_{5}$.
\end{taskBN}

\begin{taskBN}{151}
Компания «Альфа» начала инвестировать средства в перспективную отрасль в 2001 году, имея капитал в размере 3500 долларов. Каждый год, начиная с 2002 года, она получала прибыль, которая составляла 200\% от капитала предыдущего года. А компания «Бета» начала инвестировать средства в другую отрасль в 2007 году, имея капитал в размере 4500 долларов, и, начиная с 2008 года, ежегодно получала прибыль, составляющую 400\% от капитала предыдущего года. На сколько долларов капитал одной из компаний был больше капитала другой к концу 2009 года, если прибыль из оборота не изымалась?
\end{taskBN}

\begin{taskBN}{152}
Дана геометрическая прогрессия, для которой $b_1 = 5$, $q=2$. Найдите $S_{5}$.
\end{taskBN}

\begin{taskBN}{153}
Дана геометрическая прогрессия, для которой $b_1 = -2$, $q=4$. Найдите $S_{6}$.
\end{taskBN}

\begin{taskBN}{154}
Бизнесмен Баранкин получил в 2000 году прибыль в размере 1300000 рублей. Каждый следующий год его прибыль увеличивалась на 20\% по сравнению с предыдущим годом. Сколько рублей заработал Баранкин за 2004 год?
\end{taskBN}

\begin{taskBN}{155}
Компания «Альфа» начала инвестировать средства в перспективную отрасль в 2001 году, имея капитал в размере 4000 долларов. Каждый год, начиная с 2002 года, она получала прибыль, которая составляла 100\% от капитала предыдущего года. А компания «Бета» начала инвестировать средства в другую отрасль в 2006 году, имея капитал в размере 6500 долларов, и, начиная с 2007 года, ежегодно получала прибыль, составляющую 300\% от капитала предыдущего года. На сколько долларов капитал одной из компаний был больше капитала другой к концу 2008 года, если прибыль из оборота не изымалась?
\end{taskBN}

\begin{taskBN}{156}
Дана геометрическая прогрессия, для которой $b_{5} = -64$, $b_{8}=512$. Найдите $b_{2}$.
\end{taskBN}

\begin{taskBN}{157}
Дана геометрическая прогрессия, для которой $b_{3} = 72$, $b_{4}=-216$. Найдите $b_{5}$.
\end{taskBN}

\begin{taskBN}{158}
Компания «Альфа» начала инвестировать средства в перспективную отрасль в 2001 году, имея капитал в размере 6000 долларов. Каждый год, начиная с 2002 года, она получала прибыль, которая составляла 100\% от капитала предыдущего года. А компания «Бета» начала инвестировать средства в другую отрасль в 2004 году, имея капитал в размере 7000 долларов, и, начиная с 2005 года, ежегодно получала прибыль, составляющую 200\% от капитала предыдущего года. На сколько долларов капитал одной из компаний был больше капитала другой к концу 2007 года, если прибыль из оборота не изымалась?
\end{taskBN}

\begin{taskBN}{159}
Бизнесмен Батонов получил в 2000 году прибыль в размере 600000 рублей. Каждый следующий год его прибыль увеличивалась на 4\% по сравнению с предыдущим годом. Сколько рублей заработал Батонов за 2002 год?
\end{taskBN}

\begin{taskBN}{160}
Чему равна сумма бесконечной геометрической прогрессии, для которой $b_1 = -24$ и $q=0,8$?
\end{taskBN}

\begin{taskBN}{161}
Компания «Альфа» начала инвестировать средства в перспективную отрасль в 2001 году, имея капитал в размере 4500 долларов. Каждый год, начиная с 2002 года, она получала прибыль, которая составляла 200\% от капитала предыдущего года. А компания «Бета» начала инвестировать средства в другую отрасль в 2005 году, имея капитал в размере 5500 долларов, и, начиная с 2006 года, ежегодно получала прибыль, составляющую 300\% от капитала предыдущего года. На сколько долларов капитал одной из компаний был больше капитала другой к концу 2008 года, если прибыль из оборота не изымалась?
\end{taskBN}

\begin{taskBN}{162}
Дана геометрическая прогрессия, для которой $b_1 = -2$, $b_{4}=128$. Найдите $q$.
\end{taskBN}

\begin{taskBN}{163}
Чему равен знаменатель бесконечной геометрической прогрессии, если $b_1 = -4$ и сумма всех её членов $S=-16$?
\end{taskBN}

\begin{taskBN}{164}
Дана геометрическая прогрессия, для которой $b_1 = 8$, $b_{6}=-1944$. Найдите $b_{5}$.
\end{taskBN}

\begin{taskBN}{165}
Чему равна сумма бесконечной геометрической прогрессии, для которой $b_1 = 6$ и $q=0,8$?
\end{taskBN}

\begin{taskBN}{166}
Чему равна сумма бесконечной геометрической прогрессии, для которой $b_1 = 9$ и $q=0,5$?
\end{taskBN}

\begin{taskBN}{167}
Бизнесмен Коржов получил в 2000 году прибыль в размере 1500000 рублей. Каждый следующий год его прибыль увеличивалась на 10\% по сравнению с предыдущим годом. Сколько рублей заработал Коржов за 2003 год?
\end{taskBN}

\begin{taskBN}{168}
Дана геометрическая прогрессия, для которой $b_1 = 4$, $q=4$. Найдите $S_{7}$.
\end{taskBN}

\begin{taskBN}{169}
Чему равна сумма бесконечной геометрической прогрессии, для которой $b_1 = 24$ и $q=0,75$?
\end{taskBN}

\begin{taskBN}{170}
Дана геометрическая прогрессия, для которой $b_{6} = -38880$, $b_{3}=-180$. Найдите $b_{5}$.
\end{taskBN}

\begin{taskBN}{171}
Дана геометрическая прогрессия, для которой $b_1 = 6$, $b_{4}=162$. Найдите $q$.
\end{taskBN}

\begin{taskBN}{172}
Чему равна сумма бесконечной геометрической прогрессии, для которой $b_1 = -17$ и $q=0,8$?
\end{taskBN}

\begin{taskBN}{173}
Дана геометрическая прогрессия, для которой $b_1 = -6$, $q=-6$. Найдите $b_{3}$.
\end{taskBN}

\begin{taskBN}{174}
Дана геометрическая прогрессия, для которой $b_1 = 4$, $b_{8}=-65536$. Найдите $b_{7}$.
\end{taskBN}

\begin{taskBN}{175}
Бизнесмен Ватрушкин получил в 2000 году прибыль в размере 600000 рублей. Каждый следующий год его прибыль увеличивалась на 16\% по сравнению с предыдущим годом. Сколько рублей заработал Ватрушкин за 2002 год?
\end{taskBN}

\begin{taskBN}{176}
Дана геометрическая прогрессия, для которой $b_1 = 8$, $q=-8$. Найдите $b_{8}$.
\end{taskBN}

\begin{taskBN}{177}
Дана геометрическая прогрессия, для которой $b_1 = 8$, $q=0,3$. Найдите $S_{4}$.
\end{taskBN}

\begin{taskBN}{178}
Компания «Альфа» начала инвестировать средства в перспективную отрасль в 2001 году, имея капитал в размере 3500 долларов. Каждый год, начиная с 2002 года, она получала прибыль, которая составляла 200\% от капитала предыдущего года. А компания «Бета» начала инвестировать средства в другую отрасль в 2005 году, имея капитал в размере 5000 долларов, и, начиная с 2006 года, ежегодно получала прибыль, составляющую 300\% от капитала предыдущего года. На сколько долларов капитал одной из компаний был больше капитала другой к концу 2008 года, если прибыль из оборота не изымалась?
\end{taskBN}

\begin{taskBN}{179}
Дана геометрическая прогрессия, для которой $b_1 = -7$, $q=2$. Найдите $S_{4}$.
\end{taskBN}

\begin{taskBN}{180}
Дана геометрическая прогрессия, для которой $b_{8} = -2239488$, $b_{5}=-10368$. Найдите $b_{7}$.
\end{taskBN}

\begin{taskBN}{181}
Бизнесмен Батонов получил в 2000 году прибыль в размере 800000 рублей. Каждый следующий год его прибыль увеличивалась на 20\% по сравнению с предыдущим годом. Сколько рублей заработал Батонов за 2002 год?
\end{taskBN}

\begin{taskBN}{182}
Дана геометрическая прогрессия, для которой $b_1 = -3$, $q=0,4$. Найдите $S_{5}$.
\end{taskBN}

\begin{taskBN}{183}
Дана геометрическая прогрессия, для которой $b_{7} = -2916$, $b_{4}=-108$. Найдите $b_{6}$.
\end{taskBN}

\begin{taskBN}{184}
Дана геометрическая прогрессия, для которой $b_1 = 7$, $q=5$. Найдите $S_{5}$.
\end{taskBN}

\begin{taskBN}{185}
Дана геометрическая прогрессия, для которой $b_1 = -2$, $q=-4$. Найдите $b_{4}$.
\end{taskBN}

\begin{taskBN}{186}
Компания «Альфа» начала инвестировать средства в перспективную отрасль в 2001 году, имея капитал в размере 7000 долларов. Каждый год, начиная с 2002 года, она получала прибыль, которая составляла 200\% от капитала предыдущего года. А компания «Бета» начала инвестировать средства в другую отрасль в 2006 году, имея капитал в размере 8000 долларов, и, начиная с 2007 года, ежегодно получала прибыль, составляющую 400\% от капитала предыдущего года. На сколько долларов капитал одной из компаний был больше капитала другой к концу 2009 года, если прибыль из оборота не изымалась?
\end{taskBN}

\begin{taskBN}{187}
Дана геометрическая прогрессия, для которой $b_1 = 8$, $q=4$. Найдите $S_{6}$.
\end{taskBN}

\begin{taskBN}{188}
Дана геометрическая прогрессия, для которой $b_1 = -8$, $b_{8}=2239488$. Найдите $b_{6}$.
\end{taskBN}

\begin{taskBN}{189}
Чему равна сумма бесконечной геометрической прогрессии, для которой $b_1 = -15$ и $q=0,8$?
\end{taskBN}

\begin{taskBN}{190}
Дана геометрическая прогрессия, для которой $b_1 = -8$, $q=0,5$. Найдите $S_{3}$.
\end{taskBN}

\begin{taskBN}{191}
Дана геометрическая прогрессия, для которой $b_1 = -6$, $q=0,3$. Найдите $S_{5}$.
\end{taskBN}

\begin{taskBN}{192}
Дана геометрическая прогрессия, для которой $b_1 = 8$, $q=-6$. Найдите $b_{5}$.
\end{taskBN}

\begin{taskBN}{193}
Бизнесмен Булкин получил в 2000 году прибыль в размере 1400000 рублей. Каждый следующий год его прибыль увеличивалась на 20\% по сравнению с предыдущим годом. Сколько рублей заработал Булкин за 2004 год?
\end{taskBN}

\begin{taskBN}{194}
Дана геометрическая прогрессия, для которой $b_1 = -2$, $b_{5}=-8192$. Найдите $q$.
\end{taskBN}

\begin{taskBN}{195}
Дана геометрическая прогрессия, для которой $b_1 = 7$, $b_{5}=16807$. Найдите $q$.
\end{taskBN}

\begin{taskBN}{196}
Дана геометрическая прогрессия, для которой $b_1 = 3$, $b_{6}=729$. Найдите $b_{5}$.
\end{taskBN}

\begin{taskBN}{197}
Дана геометрическая прогрессия, для которой $b_1 = 6$, $b_{6}=-18750$. Найдите $b_{8}$.
\end{taskBN}

\begin{taskBN}{198}
Дана геометрическая прогрессия, для которой $b_1 = 2$, $q=2$. Найдите $S_{4}$.
\end{taskBN}

\begin{taskBN}{199}
Дана геометрическая прогрессия, для которой $b_1 = 9$, $b_{6}=-2187$. Найдите $q$.
\end{taskBN}

\begin{taskBN}{200}
Дана геометрическая прогрессия, для которой $b_1 = -9$, $q=5$. Найдите $b_{5}$.
\end{taskBN}

\newpage
 
\begin{tabular}{*{4}l}
\begin{tabular}[t]{|l|l|l|}
\hline
1 & 1 & -10,92\\
\hline
1 & 2 & -500\\
\hline
1 & 3 & 680\\
\hline
1 & 4 & 4538500\\
\hline
1 & 5 & 101500\\
\hline
1 & 6 & -6\\
\hline
1 & 7 & 54\\
\hline
1 & 8 & 16777216\\
\hline
1 & 9 & 2049740\\
\hline
1 & 10 & 32768\\
\hline
1 & 11 & 0,75\\
\hline
1 & 12 & -1701\\
\hline
1 & 13 & -786432\\
\hline
1 & 14 & 0,9\\
\hline
1 & 15 & 4\\
\hline
1 & 16 & 250\\
\hline
1 & 17 & 88\\
\hline
1 & 18 & -26\\
\hline
1 & 19 & 8\\
\hline
1 & 20 & 6\\
\hline
1 & 21 & -4\\
\hline
1 & 22 & 10368\\
\hline
1 & 23 & -35\\
\hline
1 & 24 & -4,97952\\
\hline
1 & 25 & 3645\\
\hline
1 & 26 & -12,992\\
\hline
1 & 27 & 915920\\
\hline
1 & 28 & -5,7004\\
\hline
1 & 29 & 3645\\
\hline
1 & 30 & 2,4992\\
\hline
1 & 31 & -512\\
\hline
1 & 32 & 3888\\
\hline
1 & 33 & 684450\\
\hline
1 & 34 & -32766\\
\hline
1 & 35 & 2196150\\
\hline
1 & 36 & 0,8\\
\hline
1 & 37 & -484\\
\hline
1 & 38 & 143000\\
\hline
1 & 39 & -30\\
\hline
1 & 40 & -93312\\
\hline
1 & 41 & -147\\
\hline
1 & 42 & 30\\
\hline
1 & 43 & -1820\\
\hline
1 & 44 & 1023\\
\hline
1 & 45 & -81\\
\hline
1 & 46 & 12\\
\hline
1 & 47 & 636500\\
\hline
1 & 48 & 486\\
\hline
1 & 49 & 125\\
\hline
\end{tabular}&\begin{tabular}[t]{|l|l|l|}
\hline
1 & 50 & 6144\\
\hline
1 & 51 & 1052500\\
\hline
1 & 52 & 9\\
\hline
1 & 53 & 11664\\
\hline
1 & 54 & -2097152\\
\hline
1 & 55 & 648000\\
\hline
1 & 56 & 0,75\\
\hline
1 & 57 & 340000\\
\hline
1 & 58 & 352000\\
\hline
1 & 59 & 1735000\\
\hline
1 & 60 & 8\\
\hline
1 & 61 & 13122\\
\hline
1 & 62 & -13,5625\\
\hline
1 & 63 & 6,528\\
\hline
1 & 64 & -9,9757\\
\hline
1 & 65 & 335922\\
\hline
1 & 66 & 1464100\\
\hline
1 & 67 & 39\\
\hline
1 & 68 & 1903330\\
\hline
1 & 69 & -36\\
\hline
1 & 70 & 10,92\\
\hline
1 & 71 & 3\\
\hline
1 & 72 & -2058\\
\hline
1 & 73 & -260\\
\hline
1 & 74 & 2073600\\
\hline
1 & 75 & -64\\
\hline
1 & 76 & -2916\\
\hline
1 & 77 & -180\\
\hline
1 & 78 & -3,2992\\
\hline
1 & 79 & -28697814\\
\hline
1 & 80 & 7\\
\hline
1 & 81 & 243\\
\hline
1 & 82 & 14079500\\
\hline
1 & 83 & -6\\
\hline
1 & 84 & 512\\
\hline
1 & 85 & 12986000\\
\hline
1 & 86 & 1331000\\
\hline
1 & 87 & 7\\
\hline
1 & 88 & -7\\
\hline
1 & 89 & 95\\
\hline
1 & 90 & -4\\
\hline
1 & 91 & 9\\
\hline
1 & 92 & 128\\
\hline
1 & 93 & -936\\
\hline
1 & 94 & -144\\
\hline
1 & 95 & -25000\\
\hline
1 & 96 & -5184\\
\hline
1 & 97 & 16384\\
\hline
1 & 98 & 1867000\\
\hline
1 & 99 & 95\\
\hline
\end{tabular}&\begin{tabular}[t]{|l|l|l|}
\hline
1 & 100 & -625000\\
\hline
1 & 101 & 586000\\
\hline
1 & 102 & -6\\
\hline
1 & 103 & 768\\
\hline
1 & 104 & 11718\\
\hline
1 & 105 & -100\\
\hline
1 & 106 & 14797000\\
\hline
1 & 107 & 58\\
\hline
1 & 108 & 19,584\\
\hline
1 & 109 & -6\\
\hline
1 & 110 & 1113920\\
\hline
1 & 111 & -50421\\
\hline
1 & 112 & 32768\\
\hline
1 & 113 & 96000\\
\hline
1 & 114 & 726000\\
\hline
1 & 115 & -9\\
\hline
1 & 116 & 62500\\
\hline
1 & 117 & -8\\
\hline
1 & 118 & -31104\\
\hline
1 & 119 & -32768\\
\hline
1 & 120 & 831670\\
\hline
1 & 121 & 0,75\\
\hline
1 & 122 & 1464100\\
\hline
1 & 123 & -1813\\
\hline
1 & 124 & 2916\\
\hline
1 & 125 & -96\\
\hline
1 & 126 & 110\\
\hline
1 & 127 & 1630500\\
\hline
1 & 128 & -1125\\
\hline
1 & 129 & 9\\
\hline
1 & 130 & 109375\\
\hline
1 & 131 & 8192\\
\hline
1 & 132 & 320\\
\hline
1 & 133 & 110\\
\hline
1 & 134 & 0,5\\
\hline
1 & 135 & -200\\
\hline
1 & 136 & 2419200\\
\hline
1 & 137 & -60000000\\
\hline
1 & 138 & 32\\
\hline
1 & 139 & -14348907\\
\hline
1 & 140 & 1120000\\
\hline
1 & 141 & 59049\\
\hline
1 & 142 & 3,7488\\
\hline
1 & 143 & 17496\\
\hline
1 & 144 & -192\\
\hline
1 & 145 & 60\\
\hline
1 & 146 & -4\\
\hline
1 & 147 & 50421\\
\hline
1 & 148 & -5\\
\hline
1 & 149 & -24576\\
\hline
\end{tabular}&\begin{tabular}[t]{|l|l|l|}
\hline
1 & 150 & 768\\
\hline
1 & 151 & 22851000\\
\hline
1 & 152 & 155\\
\hline
1 & 153 & -2730\\
\hline
1 & 154 & 2695680\\
\hline
1 & 155 & 408000\\
\hline
1 & 156 & 8\\
\hline
1 & 157 & 648\\
\hline
1 & 158 & 195000\\
\hline
1 & 159 & 648960\\
\hline
1 & 160 & -120\\
\hline
1 & 161 & 9489500\\
\hline
1 & 162 & -4\\
\hline
1 & 163 & 0,75\\
\hline
1 & 164 & 648\\
\hline
1 & 165 & 30\\
\hline
1 & 166 & 18\\
\hline
1 & 167 & 1996500\\
\hline
1 & 168 & 21844\\
\hline
1 & 169 & 96\\
\hline
1 & 170 & -6480\\
\hline
1 & 171 & 3\\
\hline
1 & 172 & -85\\
\hline
1 & 173 & -216\\
\hline
1 & 174 & 16384\\
\hline
1 & 175 & 807360\\
\hline
1 & 176 & -16777216\\
\hline
1 & 177 & 11,336\\
\hline
1 & 178 & 7334500\\
\hline
1 & 179 & -105\\
\hline
1 & 180 & -373248\\
\hline
1 & 181 & 1152000\\
\hline
1 & 182 & -4,9488\\
\hline
1 & 183 & -972\\
\hline
1 & 184 & 5467\\
\hline
1 & 185 & 128\\
\hline
1 & 186 & 44927000\\
\hline
1 & 187 & 10920\\
\hline
1 & 188 & 62208\\
\hline
1 & 189 & -75\\
\hline
1 & 190 & -14\\
\hline
1 & 191 & -8,5506\\
\hline
1 & 192 & 10368\\
\hline
1 & 193 & 2903040\\
\hline
1 & 194 & -8\\
\hline
1 & 195 & -7\\
\hline
1 & 196 & 243\\
\hline
1 & 197 & -468750\\
\hline
1 & 198 & 30\\
\hline
1 & 199 & -3\\
\hline
1 & 200 & -5625\\
\hline
\end{tabular}\end{tabular}



\end{document}