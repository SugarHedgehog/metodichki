\documentclass[10pt]{article}
\usepackage[russian]{babel}
\usepackage[utf8]{inputenc}
\usepackage[T2A]{fontenc}
\usepackage{amsmath}
\usepackage{amsfonts}
\usepackage{amssymb}
\usepackage[version=4]{mhchem}
\usepackage{stmaryrd}
\usepackage{graphicx}
\usepackage[export]{adjustbox}
\graphicspath{ {./images/} }

\begin{document}

% Section: Introduction to Nonlinear Pendulum
\section*{§ 3. Пример: нелинейный маятник •}
В этом разделе рассматриваются три модели, которые иллюстрируют различные виды нелинейных колебаний для системы с одной степенью свободы.

% Subsection: Hamiltonian and Equations of Motion
\subsection*{Гамильтониан и уравнения движения}
Гамильтониан нелинейного маятника с единичной массой представлен как:
\begin{equation*}
H=1 / 2 \dot{x}^{2}-\omega_{0}^{2} \cos x, \tag{3.1}
\end{equation*}
где $q=x$ и $p=\dot{x}$. Уравнения движения:
\begin{equation*}
\ddot{x}+\omega_{0}^{2} \sin x=0 . \tag{3.2}
\end{equation*}

% Subsection: Equilibrium States
\subsection*{Состояния равновесия}
Состояния равновесия определяются уравнениями:
\begin{equation*}
\dot{x}_{s}=0, \quad \sin x_{s}=0 . \tag{3.3}
\end{equation*}
Решения: $\dot{x}_{s}=0, x_{s}=\pi n, n=0, \pm 1, \ldots$

% Subsection: Phase Portrait and Separatrix
\subsection*{Фазовый портрет и сепаратриса}
Фазовые траектории при $H<\omega_{0}^{2}$ и $H>\omega_{0}^{2}$ описываются, включая сепаратрису с энергией $H_{s}=\omega_{0}^{2}$.

% Subsection: Soliton-like Solutions
\subsection*{Солитоноподобные решения}
Решение на сепаратрисе:
\begin{equation*}
v= \pm 2 \omega_{0} / \operatorname{ch}\left(\omega_{0} t\right) \tag{3.7}
\end{equation*}

% Subsection: General Solution and Action-Angle Variables
\subsection*{Общее решение и переменные действие-угол}
Используются переменные действие-угол для нахождения общего решения уравнения движения.

% Subsection: Nonlinear Oscillation Frequency
\subsection*{Частота нелинейных колебаний}
Частота нелинейных колебаний маятника определяется как:
\begin{equation*}
\omega(H)=\frac{\pi}{2} \omega_{0} \begin{cases}\frac{1}{F(\pi / 2 ; x)} & (x \leqslant 1), \\ \frac{x}{F(\pi / 2 ; 1 / x)} & (x \geqslant 1) .\end{cases} \tag{3.12}
\end{equation*}

% Subsection: Spectrum of Nonlinear Pendulum
\subsection*{Спектр нелинейного маятника}
Анализируется спектр колебаний маятника и его зависимость от энергии $H$.

% Subsection: General Properties of Oscillation Period
\subsection*{Общие свойства периода колебаний}
Исследуется период колебаний системы в потенциальной яме и его зависимость от энергии.

\end{document}