\documentclass[a4paper 12pt]{article}
\usepackage[left=0cm,right=0cm,
top=0cm,bottom=0cm,bindingoffset=0cm]{geometry}
%%% Работа с русским языком
\usepackage{cmap}					% поиск в PDF
\usepackage{mathtext} 				% русские буквы в фомулах
\usepackage[T2A]{fontenc}			% кодировка
\usepackage[utf8]{inputenc}			% кодировка исходного текста
\usepackage[english,russian]{babel}	% локализация и переносы

%%% Дополнительная работа с математикой
\usepackage{amsfonts,amssymb,amsthm,mathtools} % AMS
\usepackage{amsmath}
\usepackage{icomma} % "Умная" запятая: $0,2$ --- число, $0, 2$ --- перечисление

%% Номера формул
%\mathtoolsset{showonlyrefs=true} % Показывать номера только у тех формул, на которые есть \eqref{} в тексте.

%% Шрифты
\usepackage{euscript}	 % Шрифт Евклид
\usepackage{mathrsfs} % Красивый матшрифт

%% Свои команды
\DeclareMathOperator{\sgn}{\mathop{sgn}}


%% Перенос знаков в формулах (по Львовскому)
\newcommand*{\hm}[1]{#1\nobreak\discretionary{}
	{\hbox{$\mathsurround=0pt #1$}}{}}

%%% Работа с картинками
\usepackage{graphicx}  % Для вставки рисунков
\graphicspath{{Изображения/}{image}}  % папки с картинками
\setlength\fboxsep{3pt} % Отступ рамки \fbox{} от рисунка
\setlength\fboxrule{1pt} % Толщина линий рамки \fbox{}
\usepackage{wrapfig} % Обтекание рисунков и таблиц текстом

%%% Работа с таблицами
\usepackage{array,tabularx,tabulary,booktabs} % Дополнительная работа с таблицами
\usepackage{longtable}  % Длинные таблицы
\usepackage{multirow} % Слияние строк в таблице
\begin{document}

\textbf{Линейный оператор}

Пусть $E, F$ - линейные нормированные пространства. Отображение $A$ назовем отображением из $E$ в $F$, если для $A$ область определения $D(A) \subset E$, а множество значений $D(A) \subset F$. В таком случае пишем $A: D(A) \subset E \rightarrow F$.

Предположим, что пространства $E, F$ оба вещественные, или оба комплексные. Отображение $A$ из $E$ в $F$ называется линейным оператором, если:

\begin{enumerate}
	\item $D(A)$ - линейное многообразие в $E$;

	\item $A(\lambda x)=\lambda A x$, где $x \in D(A)$ и $\lambda$ число;

	\item $A(x+y)=A x+A y$, где $x, y \in D(A)$.
\end{enumerate}

\textbf{Ограниченный линейный оператор}

Линейный оператор $A: D(A) \subset E \rightarrow F$ называется ограниченным на $D(A)$, если

$$
	(\exists C \geq 0)(\forall x \in D(A))\left[\|A x\|_{F} \leq C\|x\|_{E}\right] .
$$\\

\textbf{Норма линейного ограниченного оператора}

Пусть $E, F$ - линейные нормированные пространства. Пусть Линейный оператор $A: D(A) \subset E \rightarrow F$ ограниченный на $D(A)$. Тогда из (1.1) следует, что числовое множество

$$
	M=\left\{\frac{\|A x\|_{F}}{\|x\|_{E}} \mid(x \in D(A)) \wedge(x \neq \Theta)\right\}
$$

ограничено сверху константой $C \geq 0$. Обозначим

$$
	\|A\|=\sup M=\sup _{\substack{x \in D(A) \\ x \neq \Theta}} \frac{\|A x\|_{F}}{\|x\|_{E}} \leq C<\infty .
$$

Величина $\|A\|$ называется нормой оператора $A$ на $D(A)$.\\

\textbf{$L(E, F)$ множество всех линейных ограниченных операторов}

Пусть $E, F$ - линейные нормированные пространства, причем оба вещественные или оба комплексные. Через $L(E, F)$ обозначим множество всех линейных ограниченных операторов $A: E \rightarrow F$. В случае $F=E$ вместо $L(E, E)$ пишут $L(E)$.

Определим на множестве $L(E, F)$ операции умножения на число и сложение. Считаем для числа $\lambda$ и $A, B \in L(E, F)$ операторы $\lambda A$ и $A+B$ такие, что для $x \in E$

$$
	(\lambda A) x=(\lambda) A x, \quad(A+B) x=A x+B x .
$$\\

\textbf{Сильно фундаментальная последовательность}
Последовательность операторов $\left\{A_{n}\right\} \subset L(E, F)$ называется сильно фундаментальной, если для любого $x \in E$ последовательность $\left\{A_{n} x\right\} \subset F$ фундаментальна.\\

\textbf{Сильно полное пространство}
Пространство $L(E, F)$ называется сильно полным, если для всякой сильно фундаментальной последовательности $\left\{A_{n}\right\} \subset L(E, F)$ найдется оператор $A \subset L(E, F)$ такой, что $A_{n} \stackrel{\text { сильно }}{\longrightarrow} A$.\\

\textbf{$\widetilde{A}$ Продолжение оператора по непрерывности на всё пространство}
Пусть $E$ - линейное нормированное пространство и F банахово пространство. Линейный оператор $A: D(A) \subset E \rightarrow F$ ограничен на своей области определения $D(A)$ и множество $D(A)$ плотно в E. Тогда существует оператор $\widetilde{A} \in L(E, F)$ такой, что:

$$
\text { 1) }(\forall x \in D(A))[\widetilde{A} x=A x], \quad \text { 2) }\|\widetilde{A}\|=\|A\|
$$\\

\textbf{Обратимый оператор}
Оператор $A$ называется обратимым, если

$$
(\forall y \in R(A))(\exists x \in D(A) \text { единственный })[A x=y] .
$$\\

\textbf{Ядро оператора или нуль-многообразие}

Для линейного оператора $A$ определим множество

$$
N(A)=\{x \in D(A) \mid A x=\Theta\}
$$

называемое ядром или нуль-многообразием оператора $A$. Нетрудно видеть, что $N(A)$ - линейное многообразие в пространстве $E$.\\

\textbf{Непрерывно обратимый оператор}

Оператор $A$ называется непрерывно обратимым, если оператор $A$ обратим и обратный $A^{-1} \in L(F, E)$.

\end{document}