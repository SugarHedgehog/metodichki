\documentclass[a4paper,12pt]{article}
\usepackage[utf8]{inputenc}
\usepackage[russian]{babel}
\usepackage{geometry}
\geometry{top=10mm, bottom=10mm, left=10mm, right=10mm}
\pagestyle{empty}

\begin{document}

\begin{center}
	\hfill \break
	\large{МИНОБРНАУКИ РОССИИ}\\
	\footnotesize{ФЕДЕРАЛЬНОЕ ГОСУДАРСТВЕННОЕ БЮДЖЕТНОЕ ОБРАЗОВАТЕЛЬНОЕ УЧРЕЖДЕНИЕ}\\
	\footnotesize{ВЫСШЕГО ПРОФЕССИОНАЛЬНОГО ОБРАЗОВАНИЯ}\\
	\small{\textbf{«ВОРОНЕЖСКИЙ ГОСУДАРСТВЕННЫЙ УНИВЕРСИТЕТ»}}\\
	\hfill \break
	\normalsize{Математический факультет}\\
	\hfill \break
	\normalsize{Кафедра теории функций и геометрии}\\
	\hfill\break
	\hfill \break
	\hfill \break
	\hfill \break
	\hfill \break
	\hfill \break
	\hfill \break
	\hfill \break
	\hfill \break
	\large{Психолого-педагогическая характеристика классного коллектива	}\\
	\hfill \break
	\hfill \break
	\hfill \break
\hfill \break
\hfill \break
\hfill \break
\hfill \break
\hfill \break
\hfill \break
\normalsize{
	\begin{tabular}{cccc}
		Выполнила студентка 5 курса & \underline{\hspace{3cm}} &                       &  Суматохина А.С. \\\\
		Руководитель пед. практики & \underline{\hspace{3cm}} & проф. & Прядиев В.Л.    \\\\
	\end{tabular}
}\\
\hfill \break
\hfill \break
\hfill \break
\hfill \break
\hfill \break
\hfill \break
\hfill \break
\hfill \break
\hfill \break
\hfill \break
\hfill \break
\hfill \break
\hfill \break
\hfill \break
\hfill \break
\hfill \break
\hfill \break
Воронеж 2024 \end{center}
\newpage

В 5 «А» классе обучается 24 детей, из которых 14 мальчиков и 10 девочек. Большинство ребят (19) родились в 2013 году, а ещё пятеро – в 2012. Коллектив сформировался ещё в детском саду и начальной школе, благодаря чему дети уже привыкли к совместной учебе и дружеским отношениям, которые успели укрепиться с течением времени.

В учебной деятельности класс демонстрирует средний уровень способностей. Большая часть учеников успешно осваивает учебный материал, старательно и ответственно выполняет задания, проявляя усидчивость и доводя начатое до конца. Большинство учеников отличается внимательностью, усидчивостью и высоким уровнем ответственности, а также стремлением к приобретению знаний. В целом, темп работы на уроках умеренный. Наибольшую активность проявляют Скок Д., Кузнецов М., Миронова В., Диер Д., Калентьева Д. и Шимина У., которые вносят значительный вклад в атмосферу класса. Их речь четкая, словарный запас соответствует возрасту. Внешний вид ребят и организация их рабочих мест поддерживаются в порядке, что говорит об их организованности и аккуратности. Учащиеся охотно берутся за поручения учителя: раздают тетради, готовят класс к занятиям, ухаживают за растениями.

Класс отличается дружелюбием. Дети активно участвуют в подвижных играх и с радостью принимают участие в спортивных соревнованиях. Мальчики и девочки поддерживают хорошие отношения. Все ребята активно вовлечены во внеклассную деятельность, участвуют в школьных мероприятиях, посещают кружки и секции. Так, Рустамова Ф., Артюков А. и Захаров И. посещают спортивные секции, увлекаясь футболом и хоккеем. Пухальская Д., Назарова С., Елфимов А., Тарасова Л. и Денисов Д. занимаются в театральном и музыкальном кружках. В культурно-массовой жизни класса особенно активны Долматова Д., Селюнин Д., Хабибяров А., Бакланов Е., Елатова П. и Суководицин Д., которые часто берут на себя организацию мероприятий.

В классе отсутствует ярко выраженный лидер, однако дети легко идут на контакт и каждый охотно выполняет свою роль в коллективе. Класс неконфликтный, атмосфера способствует развитию коммуникативных навыков. Среди учеников нет изолированных или отвергаемых детей – все вовлечены в коллективные дела и взаимодействие.

Выводы и рекомендации:  
Рекомендуется продолжать поддерживать интерес учащихся к учёбе через игровые и интерактивные подходы, укреплять командный дух через коллективные мероприятия и развивать лидерские качества, давая каждому ученику возможность проявить себя. Особое внимание следует уделить развитию коммуникативных навыков и поддержке разнообразных интересов детей через внеклассные занятия, что поможет раскрыть их потенциал, укрепит уверенность в себе и сплотит класс.
\end{document}