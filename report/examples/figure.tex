\examples{Прямоугольный параллелепипед описан около цилиндра. Площадь полной поверхности и сторона основания параллелепипеда равны $528$ и $12$ соотвественно. Найдите высоту цилиндра.}{5}{960181242289771n0}{0.25}

\examples{Прямоугольный параллелепипед описан около цилиндра. Диагональ основания и диагональ одной из боковых сторон параллелепипеда равны $12\sqrt{2}$ и $3\sqrt{17}$ соотвественно. Найдите объём цилиндра, делённый на $\pi$.}{108}{888566803844101n0}{0.25}

\examples{Цилиндр, полная площадь поверхности которого равна $486\pi$, описан около шара. Найдите площадь поверхности шара, делённую на $\pi$.}{324}{828485373750551n0}{0.25}

\examples{Цилиндр, объём которого равен $1024\pi$, описан около шара. Найдите диаметр шара.}{16}{248685672456981n0}{0.25}

\examples{Цилиндр и конус имеют общие основание и высоту. Объём конуса равен $162$. Найдите объём цилиндра.}{486}{65541990142505n0}{0.25}

\examples{Цилиндр и конус имеют общие основание и высоту. Объём конуса равен $243$. Найдите объём цилиндра.}{729}{8890110895540526n0}{0.25}

