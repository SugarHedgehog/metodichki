\documentclass[a4paper 12pt]{article}
\usepackage[left=0cm,right=0cm,
top=0cm,bottom=0cm,bindingoffset=0cm]{geometry}
%%% Работа с русским языком
\usepackage{cmap}					% поиск в PDF
\usepackage{mathtext} 				% русские буквы в фомулах
\usepackage[T2A]{fontenc}			% кодировка
\usepackage[utf8]{inputenc}			% кодировка исходного текста
\usepackage[english,russian]{babel}	% локализация и переносы

%%% Дополнительная работа с математикой
\usepackage{amsfonts,amssymb,amsthm,mathtools} % AMS
\usepackage{amsmath}
\usepackage{icomma} % "Умная" запятая: $0,2$ --- число, $0, 2$ --- перечисление

%% Номера формул
%\mathtoolsset{showonlyrefs=true} % Показывать номера только у тех формул, на которые есть \eqref{} в тексте.

%% Шрифты
\usepackage{euscript}	 % Шрифт Евклид
\usepackage{mathrsfs} % Красивый матшрифт

%% Свои команды
\DeclareMathOperator{\sgn}{\mathop{sgn}}


%% Перенос знаков в формулах (по Львовскому)
\newcommand*{\hm}[1]{#1\nobreak\discretionary{}
	{\hbox{$\mathsurround=0pt #1$}}{}}

%%% Работа с картинками
\usepackage{graphicx}  % Для вставки рисунков
\graphicspath{{Изображения/}{image}}  % папки с картинками
\setlength\fboxsep{3pt} % Отступ рамки \fbox{} от рисунка
\setlength\fboxrule{1pt} % Толщина линий рамки \fbox{}
\usepackage{wrapfig} % Обтекание рисунков и таблиц текстом

%%% Работа с таблицами
\usepackage{array,tabularx,tabulary,booktabs} % Дополнительная работа с таблицами
\usepackage{longtable}  % Длинные таблицы
\usepackage{multirow} % Слияние строк в таблице
\newcommand{\br}{\geqslant}
\newcommand{\mr }{\leqslant}
\newcommand{\mes}{\{X, \rho\}}
\newcommand{\MS}{Пусть $\{X, \rho\}$~--метрическое пространство}
\newcommand{\tttk}{тогда и только тогда, когда }
\newcommand{\Rnp}{\mathbb{R}^n_p}
\newcommand{\crk}{^{\circ}}
\linespread{1.15}
\usepackage{xhfill}
\begin{document}
\pagestyle{empty}
\footnotesize{
\textbf{Лемма 1}Пусть на $[a, b]$ функция $\varphi (t) \br 0$, непрерывна на $\int_{b}^{a}\varphi
    (t)\, dx=0$, тогда $\varphi\equiv 0$ на $[a, b]$.
Доказательство. Если $\varphi(t) \not\equiv  0$, то
$ \exists(c \in [a, b]) [\varphi(c) > 0]$.
ТК $\varphi(t)$ непрерывна в $[a, b]$ есть
целый отрезок $c\in[\alpha, \beta]$ такой, что
$\forall(t \in [\alpha, \beta])[ \varphi(t) \br \varphi(c)/2 > 0 ]$.
$\int^b_a\varphi(t) dt = \int^\beta_\alpha \varphi(t) dt+
    \int_{t\notin(\alpha,\beta)}\varphi(t) dt\br
    \int^\beta_\alpha\varphi(t) dt \br (\beta \alpha)\varphi(c)/2 > 0. \amalg$
%%\noindent\dotfill


2)	Из 1) следует включение $\overline M \subset \overline{\overline{M}}$ .
Покажем, что $\overline{\overline{M}} \subset \overline M $.
Возьмем $x \in \overline{\overline{M}}$ ,
тогда $\forall(\varepsilon > 0)[B(x, \varepsilon) \cap \overline M  \neq \varnothing]$.
Поэтому найдется $y  \in B(x, \varepsilon) \cap \overline M$.
Обозначим $\delta = \rho(x, y) < \varepsilon$.
Покажем, что $B(y, \varepsilon - \delta) \subset B(x, \varepsilon)$.
Действительно, если $z \in B(y, \varepsilon-\delta), то \rho(z, x) \mr
    \rho(z, y)+\rho(y, x) < \varepsilon- \delta+\delta = \varepsilon$.
Следовательно, $z \in B(x, \varepsilon)$.
Далее воспользуемся тем,  что  $y  \in \overline M$ ,
тогда $B(y, \varepsilon - \delta) \cap M \neq \varnothing$.
Тем более $B(x, \varepsilon) \cap M \neq \varnothing$, то есть $x \in M$ .
3)	Пусть  $x   \in  \overline{M_1}$.
Тогда  $\forall(\varepsilon   >   0)[B(x, \varepsilon)  \cap M_1   \neq   \varnothing],  следовательно,
    \forall(\varepsilon > 0)[B(x, \varepsilon) \cap M_2 \neq \varnothing]$.
Таким образом, $x \in \overline {M_2}$, то есть $\overline {M_1} \subset \overline {M_2}$.

4)	Так как $M_1 \subset \overline{M_1 \cup M_2}$  и $\overline M_2
    \subset \overline{M_1 \cup M_2}$,
то по третьему свойству
$\overline{M_1 }\subset M_1 \cup M_2$ и $M_2 \subset M_1 \cup M_2$.
Отсюда $\overline M_1 \cup \overline M_2 \subset \overline{M_1 \cup M_2}$.
Установим обратное включение.
Возьмем $x \in\overline{ M_1 \cup M_2}$, что означает
$B(x, \varepsilon) \cap (M_1 \cup M_2) \neq	\varnothing$ $\forall$ $\varepsilon > 0$.
Справедливо равенство
$B(x, \varepsilon) \cap (M_1 \cup M_2) =
    (B(x, \varepsilon) \cap M_1) \cup (B(x, \varepsilon) \cap M_2)$.
Поэтому $[B(x, \varepsilon) \cap M_1 \neq \varnothing]
    \wedge [B(x, \varepsilon) \cap M_2 \neq \varnothing] $
и тогда $(x \in \overline M_1) \wedge (x \in \overline M_2)$.

Следовательно, $x \overline \in M_1 \cup \overline M_2$.

%%\noindent\dotfill

\textbf{Теорема 2 (критерий точки прикосновения)}

\MS и $M \subset X$. Точка $x \in \overline M $ \tttk
$\exists( \{ x_n \} \subset M)[\underset{n \rightarrow \infty  }{x_n \rightarrow x}]$

Доказательство. Пусть $x \in M$ .
Возьмем последовательность  $\{\varepsilon_n\}$ такую,  что
$\varepsilon_n \searrow  0$  при  $n  \to  \infty$.
Тогда  $\forall( n) [ B(x, \varepsilon_n) \cap M   \neq  \varnothing]$.
Выберем $x_n \in B(x, \varepsilon_n) \cap M$ .
Получили последовательность $\{x_n\} \subset M$ такую,
что выполняется $\rho(x_n, x) < \varepsilon_n$.
Это означает, что $x_n \to x$ при $n \to \infty$.
Пусть теперь последовательность $\{x_n\} \subset M и x_n \to x$ при $n \to \infty$.
Тогда  $\forall(\varepsilon  >  0)\exists(n)[ \rho(x_n, x)  <  \varepsilon]$.
Следовательно,  $x_n  \in  B(x, \varepsilon) \cap M $,
то  есть $\forall(\varepsilon > 0)[ B(x, \varepsilon) \cap M  \neq \varnothing]$ и
$x \in M$ .

%%\noindent\dotfill

\textbf{Примеры замкнутых множеств}

\begin{tabular}{c c c}
    1. $\forall B[x_0, r]\subset \mes $ &
    2. $\varnothing \subset \mes $      &
    3. $X \subset \mes$
\end{tabular}

%%\noindent\dotfill

\textbf{Теорема 3}

В метрическом пространстве объединение конечного числа и пересечение любого числа
замкнутых множеств ~--замкнутое множество.

Доказательство. 1. Пусть $F = \bigcup^n_{k=1}	$ и $\overline F_k =
    F_k \overline{(k = 1, n)}$. Из теоремы 1 (свойство 4) следует, что
$\overline F=\bigcup^n_{k=1}\overline{F_k}=\bigcup^n_{k=1}{F_k}=F$
то есть $F = \overline F$ и множество $F$ замкнуто.
2.   Пусть   $F    = \bigcap_\xi F_\xi$,
где   $\forall(\xi) [F_\xi = \overline{F_\xi}]$.
Возьмем   $x    \in  \overline F$ ,   тогда
$\forall(\varepsilon > 0)[ B(x, \varepsilon) \cap F  \neq \varnothing]$.
Следовательно, $\forall(\varepsilon > 0)\forall(\xi)
    [ B(x, \varepsilon) \cap F_\xi  \neq \varnothing]$.
Таким образом, $\forall(\xi)[ x \in \overline F_\xi  =  F_\xi]$,
что означает $x \in F$ . Следовательно, $ \overline F  \subset F$
и множество $F$ замкнуто.

%%\noindent\dotfill

\textbf{Свойства сходящихся последовательностей  }
    1) Пусть в метрическом пространстве $\{x_n\}$ сходится. $\forall(x_{n_k} \subset
              \{x_n \})$ сходится к тому же пределу, что и $\{x_n\}$
    2) Последовательность в метрическом пространстве может иметь только один
          предел
    3) Любая сходящаяся последов в метрическом пространстве ограничена.
    4) Пусть в метрическом пространстве даны две последовательности

          $\underset{n \rightarrow \infty  }{x_n \rightarrow x \text{ и } y_n \rightarrow y} $. Тогда
          $\underset{n \rightarrow \infty  }{\rho(x_n, y_n) \rightarrow \rho(x, y)}$

%%\noindent\dotfill

\textbf{Сходимость в пространстве $\Rnp$   }

Пусть $1 \mr  p < \infty $, ${x^m}\subset \Rnp $ и $\underset{m \rightarrow \infty  }{x^m \rightarrow x}$

$\rho(x^m, x)= (\sum_{k = 1}^{n}|x^m_k-x_k|^p)^{1/p} \rightarrow 0$ (покоординатная сходимость)

%%\noindent\dotfill

\textbf{Сходимость в пространстве $C[a, b]$}

Пусть $\{ x_n \}\subset C[a, b]$ сходится по метрике к функции $x \in C[a, b]$.

$\underset{n \rightarrow \infty  }{\rho(x_n, x)}=\underset{a \mr  t \mr  b}{\max}|x_n(t)-x(t)| \rightarrow 0$

\hspace*{30mm}$\Updownarrow$

$\forall(\varepsilon>0)\exists(N\in \mathbb{N})\forall(n\br N)\forall(t\in[a, b])
    [|x_n(t)-x(t)|<\varepsilon]$(равномерная сходимость)

%%\noindent\dotfill

\textbf{Теорема 4 (свойства внутренности)}

\MS.

$\forall(M, M_1, M_2\subset X)[(M\crk \subset M)
        \wedge((M\crk)\crk=M)
        \wedge(M_1 \subset M_2 \rightarrow M_1\crk \subset M_2\crk)
        \wedge \\((M_1 \cap M_2)\crk=M_1\crk \cap  M_2\crk)]$

Доказательство. Свойство 1) очевидно.
2)	Из 1) следует, что $(M \crk )\crk  \subset M \crk$ .
Покажем обратное включение.
Пусть $x \in M \crk$ .
Тогда $\exists(\varepsilon > 0)[ B(x, \varepsilon) \subset M ]$.
Возьмем $y \in B(x, \varepsilon)$. Пусть
$\rho(y, x) = \delta < \varepsilon$,
тогда $B(y, \varepsilon \delta) \subset B(x, \varepsilon) \subset M$ .
Итак, $y \in M \crk  и B(x, \varepsilon) \subset M \crk$.


Следовательно, $x \in (M \crk )\crk$  и $M \crk  \subset (M \crk )\crk$ .
3)	Пусть $x \in M_1\crk$ . Тогда $\exists(\varepsilon > 0)[B(x, \varepsilon) \subset M_1]$.
Но $M_1 \subset M_2$, поэтому
$B(x, \varepsilon) \subset M_2$.
Таким образом, $x \in M_2\crk$  и $M_1\crk  \subset M_2\crk$ .
4)	Заметим, что $M_1\cap M_2 \subset M_1 и M_1\cap M_2 \subset M_2$.
Из свойства 3) следует, что
$(M_1 \cap M_2)\crk  \subset M_1\crk  и (M_1 \cap M_2)\crk  \subset M_2\crk$
. Следовательно, $(M_1 \cap M_2)\crk  \subset M_1\crk  \cap M_2\crk$ .
Покажем обратное включение. Пусть $x \in M_1\crk
    (x \in M_2\crk )$. Это означает, что
$\cap M_2\crk . Тогда (x \in M_1\crk )\wedge
    \exists(\varepsilon1 > 0)[B(x, \varepsilon1) \subset M_1] \wedge
    \exists(\varepsilon2 > 0)[B(x, \varepsilon2) \subset M_2].
    Возьмем \varepsilon = min{\varepsilon1, \varepsilon2}$.
Тогда $B(x, \varepsilon) \subset M_1 \cap M_2$. Таким образом, доказали
$x \in (M_1 \cap M_2)\crk  и M_1\crk  \cap M_2\crk  \subset (M_1 \cap M_2)\crk$ .
%%\noindent\dotfill

\textbf{Теорема 5 (связь внутренности и замыкания)}

\MS и $M \subset X$. Тогда:

$X\backslash M\crk=\overline{X\backslash M},$ $X
    \backslash \overline M = (X \backslash M)\crk$

Доказательство. Свойства 1) и 2) следуют из соответствующих цепочек равносильных утверждений.
1)	$(x \in X\backslash M \crk ) \leftrightarrow (x \notin  M \crk ) \leftrightarrow
    \forall(\varepsilon > 0)[B(x, \varepsilon) \not\subset M ] \leftrightarrow
    \forall(\varepsilon > 0)[B(x, \varepsilon) \cap (X\backslash M ) \neq \varnothing] \leftrightarrow
    (x \in \overline {X\backslash M} )$.
2)	$(x \in X\backslash M ) \leftrightarrow (x \notin \overline  M ) \leftrightarrow
    \exists(\varepsilon > 0)[B(x, \varepsilon) \cap M  = \varnothing] \leftrightarrow
    \exists(\varepsilon > 0)[B(x, \varepsilon) \subset X\backslash M ]
    \leftrightarrow (x \in (X\backslash M )\crk )$.

%%\noindent\dotfill

\textbf{Примеры открытых множеств  }

\begin{tabular}{c c c}
    1. $\forall B(x_0, r)\subset \mes $ &
    2. $\varnothing \subset \mes $      &
    3. $X \subset \mes$
\end{tabular}

%%\noindent\dotfill

\textbf{Теорема 6 (о дополнении)  }

В метрическом пространстве дополнение открытого множество замкнуто,
а дополнение замкнутого множества открыто

Доказательство. Пусть $\{X, \rho\}$ – метрическое пространство и $M  \subset X$.
Будем обозначать CM = $X\backslash M$ – дополнение множества $M$ .
Пусть множество $M$ – открыто, то есть $M = M \crk$ .
Тогда, используя теорему 5 (свойство 1), получим $CM = X\backslash M = X\backslash M \crk
    = \overline {(X\backslash M )} = \overline CM$ .
Итак, множество CM замкнуто.
Пусть множество $M$ – замкнуто, то есть $M =\overline M$ .
Тогда, используя теорему 5 (свойство 2),
получим $CM = X\backslash M = X\backslash \overline M = (X\backslash M )\crk  = (CM )\crk$ .
Итак, множество $CM$ открыто.

%%\noindent\dotfill

\textbf{Теорема 7  }

В метрическом пространстве объединение любого числа и пересечения конечного числа
открытых множеств~-открытое множество.

Доказательство. Пусть$ A =\bigcup_\xi A_\xi $, где $(\forall\xi)[ A_\xi	= A_\xi\crk ]$.
Запишем равенства
$A  = C(CA)  = C\bigcap_\xi	(CA_\xi )$. Из теоремы 6 следует, что множества $CA_\xi$
замкнуты, но тогда из теоремы 3 получим замкнутость множества	$\bigcap_\xi(CA_\xi)$.
Теперь, вновь из теоремы 6, получим открытость множества $A$.
Пусть $A = \bigcap^n_{k=1}A_k$, где $\forall(k)[ A_k  = A\crk_k ]$.
Запишем равенства $A = C(CA) =C\bigcup^n_{k=1}(CA_k)$
. Из теоремы 6 следует, что множества $CA_k$ замкнуты, но тогда

из теоремы 3 получим замкнутость множества $\bigcup^n_{k=1}(CA_k)$.
Теперь, вновь из теоремы 6, получим открытость множества $A$.

%%\noindent\dotfill

\textbf{Теорема 10 (о полноте подпространства) }

\MS и $M \subset X$. Для того чтобы $ \{M, \rho\}$ было полным подпространством
$\mes$, необходимо и достаточно, а в случае полноты $\mes$ и достаточно, чтобы
множество $M$ было замкнуто в $\mes$.

Доказательство. Необходимость. Пусть пространство $\{M, \rho\}$ полное.
Покажем, что множество M замкнуто. Возьмем $x \in \overline M$ .
Тогда $\exists$ (теорема 2) последовательность $\{x_n\} \subset M$ такая, что
$x_n \to x$ при $n \to \infty$.
Сходящаяся последовательность является фундаментальной в $\{X, \rho\}$,
а значит и в $\{M, \rho\}$. Но $\{M, \rho\}$ –
полное пространство, следовательно,
последовательность $\{x_n\}$ сходится и в $\{M, \rho\}$.
Итак, $\exists$ $y \in M$ , что$ \rho(x_n, y) \to 0$.
Кроме того, $\rho(x_n, x) \to 0$, поэтому $x = y \in M$ .
Таким образом, множество $M$ замкнуто.

Достаточность. Пусть пространство $\{X, \rho\}$ полное и множество $M$ замкнуто.
Покажем, что$ \{M, \rho\}$ полное пространство.
Возьмем фундаментальную в $\{M, \rho\}$ последовательность $\{x_n\} \subset M$ .
Так как метрика  в пространстве
$\{X, \rho\}$ такая же, как и в $\{M, \rho\}$, то $\{x_n\}$ фундаментальна и в
полном пространстве $\{X, \rho\}$.
Следовательно, $\exists$ $x \in X$ такой, что  $\rho(x_n, x)  \to 0$ при $n \to \infty$.
Тогда$ x \in \overline M = M$ и последовательность $\{x_n\}$ сходится в $\{M, \rho\}$.
Получили, что $\{M, \rho\}$ полное пространство.

%%\noindent\dotfill

\textbf{Теорема 11 (о вложенных шарах) }

\MS и в нём задана последовательность $ \{ B[a_n, r_n] \} $ замкнутых шаров таких,
что $\underset{n \rightarrow \infty }{r_n \rightarrow 0}$ и
$\forall(n \in \mathbb{N})[B[a_{n+1},
                r_{n+1}]\subset B[a_n, r_n]]$

Тогда в $X$ $\exists$ единственная точка, принадлежащая всем этим шарам.

Доказательство. Рассмотрим $\{a_n\}$ – последовательность центров всех шаров.
Заметим, что $\rho(a_{n+p}, a_n) \mr r_n \to 0$.
Это означает, что последовательность
$\{a_n\}$ фундаментальная, а значит является сходящейся. Следовательно,
$\exists(a \in X)[ a_n \to a]$. Зафиксируем $n \in \mathbb{N} $.
Тогда последовательность
$\{a_k\}^\infty_n   \subset  B[a_n, r_n]$.
Отсюда  следует,  что  $a   \in \overline {B[a_n, r_n]}   =   B[a_n, r_n]$.
Так как $n$  было любое, получаем, что $a$  принадлежит любому шару, то есть
$a \in   \bigcap^\infty_{n=1} B[x_n, r_n]$.
Покажем, что такая точка $a \in X$ единственная.
Предположим, что $\exists$ точка $a' \neq a$ и
$a' \in \bigcap^\infty_{n=1} B[x_n, r_n]$.
Обозначим $\rho(a, a') = \lambda > 0$.
Далее $\exists(m \in N)[ r_m < \lambda/3]$.
Заметим теперь, что $a, a' \in B[a_m, r_m]$.
Следовательно, $0 < \lambda = \rho(a, a') \mr \rho(a, a_m) + \rho(a_m, a') \mr 2rm < 2λ/3$.
Получили противоречие. Значит, точка a единственная.

%%\noindent\dotfill

\textbf{Теорема 14 (об эквивалентности определений)   }

Непрерывность по Гейне в точке $x_0 \in M$:

$\forall( \{ x_n \}\subset M)[(\underset{n \rightarrow \infty }{\lim} x_n = x_0 )\rightarrow
        (\underset{n \rightarrow \infty }{\lim}f(x_n)=f(x_0))]$

\hspace*{30mm}$\Updownarrow$

Непрерывность по Коши в точке $x_0 \in M$:

$\forall(\varepsilon>0)\exists(\delta>0)\forall(x \in M)[\rho_X(x, x_0)<\delta\rightarrow
        \rho(f(x), f(x_0))<\varepsilon]$

Доказательство. Пусть функция $f : M \to Y$ непрерывна в точке $x_0 \in M$
по Коши.
Возьмем последовательность
$\{x_n\} \subset M$ такую, что $x_n \to x_0$ при $n \to \infty$.
Зададим \varepsilon > 0 и найдем по определению 2 соответствующее \delta > 0.
Затем  найдем  $N   =  N (\delta)$  такое,
что  $\forall(n  \br  N )[\rho_X(x_n, x_0)  <  \delta]$.
Но  тогда $\rho_Y (f (x_n), f (x_0)) < \varepsilon$.
Следовательно, $f (x_n) \to f (x_0)$ при $n \to \infty$,
то есть установили непрерывность по Гейне.
Пусть теперь функция $f : M \to Y$ непрерывна в точке $x_0 \in M$ по Гейне.
Предположим, что в смысле определения 2 функция $f$ не является
непрерывной в точке $x_0$. Это означает, что
$\exists(\varepsilon > 0)\forall(\delta > 0)
    \exists(x \in M )\rho X(x, x_0) < \delta
    \wedge	\rho_Y (f (x), f (x_0)) \br \varepsilon$.
Выберем последовательность $\{\delta_n\}$ такую, что $\delta_n \searrow  0$
при $n \to \infty$.
Для каждого $\delta_n$ найдем соответствующее значение $x_n \in M$ такое,
что $\rho_X(x_n, x_0) < \delta_n$ и $\rho_Y (f (x_n), f (x_0)) \br \varepsilon$.
Очевидно, что $x_n \to x_0$.
В то же время последовательность $\{f (x_n)\}$ не может сходиться к $f (x_0)$.
Получили противоречие с тем, что функция $f$ непрерывна в $x_0$ по определению 1.
Итак, и из непрерывности функции в определении 1 следует ее непрерывность в
определении 2.

%%\noindent\dotfill

\textbf{Следствие из теоремы 14}

Для непрерывности функции $f:M \rightarrow Y$ в точке $x_0 \in M$ достаточно, чтобы для
$\forall(x_n \rightarrow x \in X)$ последовательность ${f(x_n)}$ сходилась в $Y$.

%%\noindent\dotfill

\textbf{Теорема 15 (о прообразах открытых множеств) }

Пусть $\{X, \rho_X \}$, $\{Y \rho_Y \}$ метрические пространства. Функция $f:
    X \rightarrow Y$ непрерывна на $X$ \tttk $\forall$ открытого множества $A \subset Y$
прообраз $f^{-1}[A]$~--открытое множество в $X$.

Доказательство.
Пусть функция $f$ непрерывна на $X$ и множество $ A \subset Y$ открыто.
Рассмотрим прообраз $z = {x \in X| f (x) \in A}$.
Возьмем точку $x_0 \in f^{-1}[A]$  и  покажем,  что  она  является  внутренней.
Так  как  $f (x_0)  \in A$ и $A$ открыто,
то $\exists(\varepsilon > 0)[ B(f (x_0), \varepsilon) \subset A ]$.
Так как функция $f$  непрерывна,
то $\exists(\delta > 0)
    [ (\rho_X(x, x_0)  <  \delta)  \to (\rho_Y (f (x), f (x_0))  <  \varepsilon) ]$.
Рассмотрим шар $B(x_0, \delta)  \subset X$.
Получим  $f [B(x_0, \delta)]  \subset B[f (x_0), \varepsilon]  \subset A$,
то  есть $B(x_0, \delta) \subset f^{-1}[A]$ и, следовательно, множество $f^{-1}[A]$
открыто в $X$.
Пусть  теперь  $\forall(A  =  A\crk   \subset Y )[ f^{-1}[A]$  открыто в  $X]$.
Возьмем  $x_0  \in X$  и $\varepsilon > 0$. Рассмотрим $B(f (x_0), \varepsilon) \subset Y$ .
Тогда $f^{-1}[ B(f (x_0), \varepsilon) ]$ – открытое множество в $X$ и
точка $x_0 \in f^{-1}[B(f (x_0), \varepsilon)]$.
Следовательно, $\exists(\delta > 0)[ B(x_0, \delta) \subset f^{-1}[B(f (x_0), \varepsilon)]$.
Таким образом, в произвольной точке $x_0 \in X$ выполняется (14),
то есть функция $f$ непрерывна на $X$.

%%\noindent\dotfill

\textbf{Теорема 16 (о прообразах замкнутых множеств)}

Пусть $\{X, \rho_X \}$, $\{Y \rho_Y \}$ метрические пространства. Функция $f:
    X \rightarrow Y$ непрерывна на $X$ \tttk $\forall$ замкнутого множества $A \subset Y$
прообраз $f^{-1}[A]$~--замкнутое множество в $X$.

Доказательство. Пусть функция f непрерывна на X. Возьмем замкнутое множество
$F = F \subset Y$ и рассмотрим его прообраз
$f^{-1}[F ] = f^{-1}[C(CF )] = f^{-1}[Y \backslash CF ] = f^{-1}[Y ]
    \backslash f^{-1}[CF ] = X\backslash f^{-1}[CF ]$.
Множество $CF$  открыто в $Y$ . В силу теоремы 15 множество $f^{-1}[CF ]$ открыто в $X$.
Следовательно, его дополнение $f^{-1}[F ]$ замкнуто в $X$.
Пусть теперь$ \forall(F  = F  \subset Y )[ f^{-1}[F ] замкнуто в X]$.
Возьмем произвольное открытое множество $A = A\crk  \subset X$ и рассмотрим его прообраз
$f^{-1}[A] = f^{-1}[C(CA)] = f^{-1}[Y \backslash CA] = f^{-1}[Y ]\backslash f^{-1}[CA] =
    X\backslash f^{-1}[CA]$.
Множество $CA$ замкнуто в $Y$ . Тогда множество $f^{-1}[CA]$ замкнуто в $X$,
а его дополнение $f^{-1}[A]$ открыто в X. Следовательно, по теореме 15 функция
$f$ непрерывна на $X$.

%%\noindent\dotfill

\textbf{Теорема 17 (о неподвижной точке) }

Пусть \mes~--полное метрическое пространство, множество $M= \overline M \subset X
    \text{ и } f:M \rightarrow M$~--сжимающее отображение. Тогда $f$ имеет в $M$
единственную неподвижную точку $x^*=f(x^*)$.Кроме того, $\forall$ $x_0
    \in M $ последовательность $\underset{n \in \mathbb{N} }{x_n=f(x_{n-1})}$ сходится
к $x^*$ и справедлива оценка погрешности :

$\rho(x_n, x^*)\br \cfrac{q^n}{1-q}\rho(f(x_0), x_0)$

Доказательство. Возьмем  точку  $x_0  \in M$  и  построим  $x_n  =  f (x_{n-1})  =
    f^n(x_0)$, где $n \in \mathbb{N}$. Покажем, что последовательность $\{x_n\}$ фундаментальна.

$\rho (x_n+p, x_n) = \rho (f^{n+p}(x_0), f^n(x_0)) \mr q^n \rho (f^p(x_0), x_0) \mr
    q^n{\rho (f^p(x_0), f^{p-1}(x_0)) + \rho (f^{p-1}(x_0), f^{p-2}(x_0)) + ... + \rho (f (x_0), x_0)} \mr
    q^n\rho(f(x_0),x_0)\sum^{p-1}_{k=0}q^k\mr\frac{q^n}{1-q}\rho(f(x_0),x_0)$

Так как $q < 1$, то из (16) следует фундаментальность последовательности
$\{x_n\}$. В силу полноты пространства $\{X, \rho\}$ и замкнутости множества $M$
последовательность $x_n \to x^{*}$ и $x^{*} \in M = M$ .
Учитывая непрерывность функции $f$ , в равенстве $x_n  = f (x_n-1)$
перейдем к пределу при $n \to \infty$, получим $x^{*} = f (x^{*})$.
Покажем единственность неподвижной точки. Пусть для $x^{+} \in M$ также выполняется
$x^{+} = f (x^{+})$. Тогда $\rho (x^{*}, x^{+}) = \rho (f (x^{*}), f (x^{+})) \mr q \rho (x^{*}, x^{+})$.
Так как $q < 1$, то отсюда следует $\rho (x^{*}, x^{+}) = 0$, то есть $x^{*} = x^{+}$.
Осталось заметить, что справедливость оценки погрешности (15) следует из (16)
при $p \to \infty$.

%%\noindent\dotfill

\textbf{Теорема 18   }

В метрическом пространстве любое относите компактное множество ограниченно.

Доказательство. Предположим, что в метрическом пространстве $\{X, \rho\}$
множество $M$ относительно компактно, но не ограничено. 
Возьмем произвольный  элемент  $x_0  \in X$.  
Тогда  $\forall(r  >  0)\exists(x  \in M ) [ \rho(x, x_0)  >  r ]$.  
Следовательно, $\exists$ последовательность $\{x_n\} \subset M$  такая,  
что  $\rho(x_n, x_0)  >  n$ для всех $n \in N$ . 
Выделим подпоследовательность $\{x_{n_{k}} \} \subset \{x_n\}$ такую, что
$x_{n_{k}}   \to x \in M$  при $k \to \infty$. 
Получили $\rho(x_{n_{k}}, x_0) \to \rho(x, x_0) \in \mathbb{R}$. Но, с другой
стороны, $\rho(x_{n_{k}}, x_0) > n_k \to \infty$. Это противоречие означает, что множество M
ограничено.
%%\noindent\dotfill

\textbf{Теорема 19 (Вейерштрасса)}

Пусть $M$ компактное множество в метрическом пространстве $\mes$ и функция
$f:M \rightarrow \mathbb{R}^1$ непрерывна на $M$. Тогда множество $f[M]$ отграничено в
$\mathbb{R}^1$ и $\exists(a \in M)[f(a)=\underset{x \in M}{\\inf} f(x)]\wedge
    \exists(b\in M)[f(b)=
        \underset{x \in M}{\sup} f(x)] $

%%\noindent\dotfill

\textbf{Теорема 20}

Вполне ограниченное множество в метрическом пространстве ограниченно.

Доказательство. Пусть M вполне ограниченное множество. 
Для $\varepsilon = 1$ построим для M конечную 1-сеть: $S =  {x_1, x_2, . . . , x_n}$. 
Возьмем $r  =  1 + max_{ 2\mr k \mr n} \rho(x_1, x_k)$ и рассмотрим шар $B[x_1, r].$ 
Покажем, что$ M \subset B[x_1, r]$. Пусть $x \in M$ , 
тогда $\exists$ $x_k \in S$ такой, что $\rho(x, x_k) < 1$. Далее получим
$\rho(x, x_1)	\rho(x, x_k) + \rho(x_k, x_1) < 1 +  max
2\mr k\mr n$
$\rho(x_1, x_k) = r$.
Итак, множество $M$ ограничено.

%%\noindent\dotfill

\textbf{Следствие из теоремы Хаусдорфа}

\MS. Для того чтобы множество $M \subset X$ было относительно компактным
необходимо, а в случае полноты пространства $X$ и достаточно, чтобы множество $M$
было вполне ограниченным.

%%\noindent\dotfill

\textbf{Теорема 22 (относительная компактность в $ \Rnp $) }

Множество $M \subset \Rnp$ относительно компактно \tttk это множество в $\Rnp$
ограниченно.

Доказательство. Всякое относительно компактное множество ограничено. 
Покажем, что в $r_n$ справедливо и обратное утверждение.
Пусть множество $M  \subset r_n$  и ограничено, 
то есть $M  \subset B[\theta , r]$, где $\theta  =
(0, 0, ..., 0)$ и$ r > 0$. 
Следовательно, если $x = (x_1, x_2, ..., x_n) 
\in M$ , то $\rho(x, \theta) =\sum^n_{i=1}$

$|x_i|p)1/p  \mr  r$.  Тогда  $\forall(i  =  1, n)[ |x_i|  \mr  r ]$,  
то  есть  координаты  точек

множества M в совокупности ограничены.
Берем  последовательность  $\{x(m)\}^\infty_{m=1}   \subset  M$ ,  где  в  координатах  элемент
$x(m)  =  (x_1(m), x_2(m), ..., x_n(m))$.  Из  ограниченности  последовательности
$\{x(m)\}$ следует, что $\forall(m \in N)\forall(i = 1, n)[ |x_i(m)| \mr r ]$. 
Из ограниченной числовой последовательности $\{x_1(m)\}$ 
выделим такую подпоследовательность


$\{x_1(m^1_k)\}$, что$ x_1(m^1_k) \to x_1$ при $k \to \infty$. Из числовой последовательности
$\{x_2(m^1_k)\}$ выделим подпоследовательность $\{x_2(m^2_k)\}$, что $x_2(m^2_k)  \to x_2$  при
$k \to \infty$. При этом, очевидно, $x_1(m^2_k) \to x_1$. 
Проделав эту процедуру последовательно для всех n координат, 
получим последовательность $\{x(m^n_k)\}\infty	\subset
\{x(m)\}\infty_{m=1}$,  где  $x(m^n_k)=(x_1(m^n_k), x_2(m^n_k), ..., 
x_n(m^n_k))$  
и  $x_i(m^n_k)   \to  x_i$  при
k  \to \infty для каждого $i  =  1, n$. Следовательно, последовательность $\{x(m^n_k)\}$
покоординатно сходится к элементу $x = (x_1, x_2, ..., x_n)$, что в 4 равносильно 
сходимости по метрике. Значит множество M относительно компактно.


section*{Без доков}

\textbf{Примеры полных метрических пространств  }


    1) Дискретное пространство полно.
    2) Пространство $\Rnp$, где $1\br p \br \infty $.
    3) Пространство $C[a, b]$
    4) Пространство $l_p$ $(1 \mr  p < \infty )$.



%%\noindent\dotfill

\textbf{Примеры метрических пространств, не являющихся полными}
Пространство $C_1[a, b]$  не является полным.

%%\noindent\dotfill

\textbf{Теорема 8 (об открытом множестве на прямой) }

Всякое непустое ограниченное открытое множество на прямой есть объединение
конечного или счетного числа попарно непересекающихся интервалов.

%%\noindent\dotfill

\textbf{Теорема 9 (о замкнутом множестве на прямой)}

Всякое непустое ограниченное замкнутое множество на прямой
получается из наименьшего отрезка, его содержащего, путем выбрасывания конечного или
счетного числа попарно непересекающихся интервалов.

%%\noindent\dotfill

\textbf{Теорема 13 (Бэра)}
Всякое полное метрическое пространство является множеством второй категории.
%%\noindent\dotfill

\textbf{Примеры сепарабельных пространств}

    1) Пространство $\Rnp$ для  $1\mr  p < \infty$ .
    2)Пространство  $C[a, b]$.
    3)	Пространство $s$.
    4)	Пространство $l_p$ для $1 \mr  p < \infty$ .


%%\noindent\dotfill

\textbf{Примеры пространств, не являющихся сепарабельными}


    1)     Пространство $M [a, b]$.
    2)	Пространство $m$.



%%\noindent\dotfill

\textbf{Теорема 21 (Хаусдорфа) }

\MS. Для того,
чтобы множество $M \subset X$ было относительно компактным необходимо,
а в случае
полноты пространства $X$ и достаточно, чтобы множество M
было вполне ограниченным.


%%\noindent\dotfill

\textbf{Теорема 23 (Арцела) }

Множество $M \subset C[a, b]$ относительно компактно
тогда и только тогда, когда оно ограничено и равностепенно непрерывно.
}
\end{document}