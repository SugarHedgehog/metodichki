\subsection{Хроматическое число графа}

Раскраской графа называется такое приписывание цветов его вершинам, что никакие две смежные вершины не получают одинакового цвета. Множество всех вершин одного цвета является независимым и называется одноцветным классом. В $n$-раскраске графа $G$ используется $n$ цветов, и, таким образом, эта раскраска разбивает $V$ на $n$ одноцветных классов. Хроматическое число $\chi(G)$ графа $G$ определяется как наименьшее $n$, для которого граф $G$ имеет $n$-раскраску. Граф $G$ называется $n$-раскрашиваемым, если $\chi(G) \leq n$, и $n$-хроматическим, если $\chi(G) = n$.

Поскольку граф $G$, очевидно, имеет $p$-раскраску и $\chi(G)$-раскраску, он должен иметь также $n$-раскраску для любого $n$, удовлетворяющего неравенствам $\chi(G) \leq n \leq p$. Граф на рис. 12.1 является 2-хроматическим. На этом же рисунке приведены $n$-раскраски для $n=2, 3, 4$; положительные целые числа указывают цвета.

Легко найти хроматические числа некоторых известных графов:
\[
\chi(K_p) = p, \quad \chi(K_p - x) = p - 1, \quad \chi(K'_p) = 1, \quad \chi(K_{m,n}) = 2, \quad \chi(C_{2n}) = 2, \quad \chi(C_{2n+1}) = 3 \quad \text{и} \quad \chi(T) = 2
\]
для любого нетривиального дерева $T$.

Очевидно, что граф является 1-хроматическим тогда и только тогда, когда он вполне несвязан. Описание двуцветных (2-раскрашиваемых) графов дано Кёнигом \cite{Konig}, стр. 170 и отражено в теореме 12.1 (см. также теорему 2.4).

\begin{theorem}
Граф двуцветен тогда и только тогда, когда он не содержит нечётных простых циклов.
\end{theorem}

Похоже, что проблема характеристики $n$-цветных графов для $n \geq 3$ всё ещё не решена, поскольку такой критерий даже для $n = 3$ помог бы решить гипотезу четырёх красок. Не найдены также эффективные методы определения хроматического числа произвольного графа. Однако известно несколько оценок для $\chi(G)$, в которых используются различные другие инварианты. Одна очевидная нижняя оценка — это число вершин в наибольшем полном подграфе графа $G$. Мы рассмотрим сейчас верхние оценки; первая такая оценка была получена Секерешем и Вильфом \cite{SzekeresWilf}.

\textbf{Теорема 12.2.} Для любого графа $G$
\[
\chi(G) \leq 1 + \max \delta(G'),
\]
где максимум берется по всем порожденным подграфам $G'$ графа $G$.

\textit{Доказательство.} Утверждение очевидно для вполне несвязных графов. Пусть $G$ — произвольный $n$-хроматический граф, $n \geq 2$, а $H$ — любой наименьший порожденный подграф, для которого $\chi(H) = n$. Таким образом, $\chi(H - v) = n - 1$ для всех вершин $v$ графа $H$. Следовательно, $\deg v \geq n - 1$, так что $\delta(H) \geq n - 1$, и потому
\[
n - 1 \leq \delta(H) \leq \max \delta(H') \leq \max \delta(G'),
\]
где первый максимум берется по всем порожденным подграфам $H'$ графа $H$, а второй — по всем порожденным подграфам $G'$ графа $G$. Отсюда вытекает, что
\[
\chi(G) = n \leq 1 + \max \delta(G').
\]

\textbf{Следствие 12.2 (a).} Для любого графа $G$ хроматическое число не больше чем на 1 превышает максимальную степень:
\[
\chi \leq 1 + \Delta.
\]

Брукс \cite{Brooks} показал, однако, что часто эту оценку можно улучшить.

\textbf{Теорема 12.3.} Если $\Delta(G) = n$, то граф $G$ всегда $n$-раскрашиваем, за исключением следующих двух случаев:
\begin{enumerate}
    \item $n = 2$ и $G$ имеет компоненту, являющуюся нечетным циклом;
    \item $n \geq 2$ и $K_{n+1}$ — компонента графа $G$.
\end{enumerate}

Нижняя оценка, приводимая в монографиях Бержа \cite{Berge} и Оре \cite{Ore}, и верхняя оценка, данная в статье Харари и Хедетниеми \cite{HararyHedetniemi}, содержат вершинное число независимости $\beta_0$ графа $G$.

\begin{theorem}
Для любого графа $G$
\[
\frac{p}{\beta_0} \leq \chi \leq p - \beta_0 + 1.
\]
\end{theorem}

\textit{
Если $\chi(G) = n$, то множество $V$ можно разбить на $n$ одноцветных классов $V_1, V_2, \ldots, V_n$, каждый из которых, как отмечалось выше, является независимым множеством вершин. Если $|V_i| = p_i$, то $p_i \leq \beta_0$ для всех $i$, так что $p = \sum p_i \leq n \beta_0$.
Для проверки верхней оценки рассмотрим максимальное независимое множество $S$, содержащее $\beta_0$ вершин. Ясно, что $\chi(G - S) \geq \chi(G) - 1$. Так как $G - S$ имеет $p - \beta_0$ вершин, то $\chi(G - S) \leq p - \beta_0$. Отсюда $\chi(G) \leq \chi(G - S) + 1 \leq p - \beta_0 + 1$.
}

Представленные здесь оценки не так уж хороши в том смысле, что для каждой оценки и любого положительного целого числа $n$ существует такой граф $G$, что $\chi(G)$ отличается от оценки более чем на $n$.

Исследуя приведенные выше рассуждения, легко проникнуться верой в то, что все графы с большим хроматическим числом имеют большие клики и, следовательно, содержат треугольники. И вот Дирак \cite{Dirac} поставил вопрос, существует ли граф без треугольников, но с произвольно большим хроматическим числом. Положительно на этот вопрос ответили независимо друг от друга Бланш Декарт \cite{Blanche}, Зыков \cite{Zykov} и Мышельский \cite{Myshelski}. Затем их результат обобщили Дж. и Л. Келли \cite{Kelly}, доказав, что для любого $n \geq 2$ существует $n$-хроматический граф, обхват которого превосходит 5. Они предположили, что справедливо следующее утверждение, которое первым доказал Эрдёш \cite{Erdos}, используя вероятностные соображения. Позже Ловасц \cite{Lovasz} дал конструктивное доказательство этой теоремы.

\textbf{Теорема 12.5.} Для любых двух положительных целых чисел $t$ и $n$ существует $n$-хроматический граф, обхват которого превосходит $t$.

Величина $\bar{\chi} = \chi(G) = \chi(\bar{G})$ представляет собой наименьшее число подмножеств, на которые можно разбить множество вершин графа $G$ так, чтобы каждое подмножество порождало полный подграф графа $G$. Ясно, что $\chi(G) \geq \beta_0(G)$. Оценки для суммы и произведения хроматических чисел графа и его дополнения были получены Нордхаузом и Гаддумом \cite{Nordhaus}.

\textbf{Теорема 12.6.} Для любого графа $G$ сумма и произведение чисел $\chi$ и $\bar{\chi}$ удовлетворяют неравенствам

\begin{equation}
2\sqrt{p} \leq \chi + \bar{\chi} \leq p + 1, \tag{12.4}
\end{equation}

\begin{equation}
p \leq \chi \bar{\chi} \leq \left( \frac{p + 1}{2} \right)^2. \tag{12.5}
\end{equation}

\textbf{Доказательство.} Пусть $G$ будет $n$-хроматическим графом, а $V_1, V_2, \ldots, V_n$ его одноцветными классами, $|V_i| = p_i$. Тогда, разумеется, $\sum p_i = p$ и $\max p_i \geq p/n$. Так как каждый класс $V_i$ порождает полный подграф в $G$, то $\bar{\chi} \geq \max p_i \geq p/n$, и поэтому $\chi \bar{\chi} \geq p$. Но поскольку среднее геометрическое двух положительных чисел не превосходит их среднего арифметического, то $\chi + \bar{\chi} \geq 2\sqrt{p}$. Обе нижние оценки доказаны.

Неравенство $\chi + \bar{\chi} \leq p + 1$ будем доказывать индукцией по $p$, заметив, что равенство достигается при $p = 1$. Итак, предположим, что $\chi(G) + \bar{\chi}(G) \leq p$ для всех графов $G$ с $p - 1$ вершинами. Пусть $H$ и $\bar{H}$ — взаимодополнительные графы (граф и его дополнение) с $p$ вершинами и $v$ — вершина графа $H$. Тогда $G = H - v$ и $\bar{G} = \bar{H} - v$ будут взаимодополнительными графами с $p - 1$ вершинами. Пусть степень вершины $v$ в $H$ равна $d$, так что степень в $\bar{H}$ равна $p - d - 1$.

\footnotetext[1]{Дама с этим именем есть на самом деле непустое подмножество множества \{Брукс, Смит, Стоун, Татт\}; в данном случае \{Татт\}.}

Очевидно, что
\[
\chi(H) \leq \chi(G) + 1 \quad \text{и} \quad \overline{\chi}(H) \leq \overline{\chi}(G) + 1.
\]
Если
\[
\chi(H) < \chi(G) + 1 \quad \text{или} \quad \overline{\chi}(H) < \overline{\chi}(G) + 1,
\]
то \(\chi(H) + \overline{\chi}(H) \leq p + 1\). Предположим теперь, что \(\chi(H) = \chi(G) + 1\) и \(\overline{\chi}(H) = \overline{\chi}(G) + 1\). Тогда удаление вершины \(v\) из \(H\), приводящее к образованию графа \(G\), уменьшает хроматическое число, так что \(d \geq \chi(G)\). Аналогично
\[
p - d - 1 \geq \overline{\chi}(G).
\]
Таким образом, \(\chi(G) + \overline{\chi}(G) \leq p - 1\). Следовательно, всегда
\[
\chi(H) + \overline{\chi}(H) \leq p + 1.
\]
Наконец, используя неравенство \(4\overline{\chi}\chi \leq (\chi + \overline{\chi})^2\), получаем
\[
\overline{\chi}\chi \leq \left(\frac{p + 1}{2}\right)^2.
\]