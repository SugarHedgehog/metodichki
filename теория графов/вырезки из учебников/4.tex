\subsection{Экстремальные графы}

Среди первых результатов в одном из направлений теории графов — теории экстремальных графов (см. Эрдёш) — можно отметить следующую известную теорему Турана . Как обычно, пусть \(|r|\) — наибольшее целое число, не превышающее действительного числа \(r\), а \(\{r\} = r - |r|\) есть наименьшее целое число, не меньшее \(r\).

\begin{quote}
    \textit{1) Доказательство существования чисел \(r(m, n)\) для любых натуральных \(m\) и \(n\) см., например, у М. Холла}

    \textit{2) Отметим, что по нашему определению бесконечный граф не является графом. Имеется обзорная статья о бесконечных графах: Нэш-Вильямс.}
\end{quote}

    \textbf{Теорема 2.3.} Наибольшее число рёбер у графов, имеющих \(r\) вершин и не содержащих треугольников, равно \(\lfloor r^2/4 \rfloor\).

\textbf{Доказательство.} Утверждение очевидно для малых значений \(r\). Доказательство по индукции можно дать отдельно для нечетных и для четных \(r\); здесь будет рассмотрен только случай четных значений \(r\). Предположим, что утверждение справедливо для всех четных значений \(r \leq 2n\). Докажем его для \(r = 2n + 2\). Итак, пусть \(G\) — граф с \(p = 2n + 2\) вершинами, не содержащий треугольников. Поскольку граф \(G\) не является вполне несвязным, то в нём существуют две смежные вершины \(u\) и \(v\). В подграфе \(G' = G - \{u, v\}\) имеется \(2n\) вершин и нет треугольников, так что по предположению индукции в графе \(G'\) самое большее \(\lfloor 4n^2/4 \rfloor = n^2\) рёбер. Сколько еще рёбер может быть в графе \(G\)? В графе \(G\) нет такой вершины \(w\), что вершины \(u\) и \(v\) одновременно смежны с \(w\), т. е. вершины \(u\), \(v\) и \(w\) образуют в графе \(G\) треугольник. Таким образом, если вершина \(w\) смежна с \(k\) вершинами графа \(G'\), то вершина \(v\) может быть смежна самое большее с \(2n - k\) вершинами графа \(G'\), и граф \(G\) не больше чем \[
n^2 + k + (2n - k) + 1 = n^2 + 2n + 1 = p^2/4 = \lfloor p^2/4 \rfloor
\]

рёбер.

Для завершения доказательства осталось установить, что для каждого чётного \(p\) существует \((p, p^2/4)\)-граф, не содержащий треугольников. Такой граф можно образовать следующим образом: возьмем два множества \(V_1\) и \(V_2\), каждое из которых имеет \(p/2\) вершин, и соединим каждую вершину из \(V_1\) с каждой вершиной из \(V_2\). Для \(p = 6\) соответствующий граф \(G_1\) приведен на рис. 2.5.


