\documentclass[a4paper,12pt]{article}
\usepackage[utf8]{inputenc}
\usepackage[russian]{babel}
\usepackage{geometry}
\geometry{top=10mm, bottom=10mm, left=10mm, right=10mm}
\pagestyle{empty}

\begin{document}

\begin{center}
\textbf{Муниципальное бюджетное общеобразовательное учреждение}\\
\textbf{«Средняя общеобразовательная школа № 10» г. Нижняя Салда Свердловской области}\\
\end{center}

\begin{center}
\textbf{\large ОТЗЫВ ОБ УЧЕБНО-ВОСПИТАТЕЛЬНОЙ РАБОТЕ}\\
\textbf{студентки-практикантки 5-го курса математического факультета}\\
\textbf{Суматохиной Александры Сергеевны}
\end{center}

\vspace{10mm}

\noindent
\textbf{Предмет: Математика}\\
\textbf{Класс: 5А}\\

Суматохина Александра Сергеевна проходила педагогическую практику в МБОУ «Средняя общеобразовательная школа № 10» с 30 сентября по 10 ноября 2024 года. В рамках практики она выполняла обязанности учителя математики в 5А классе и активно участвовала во внеурочной деятельности школы.

За время практики Александра Сергеевна показала себя как ответственный, компетентный и увлечённый педагог. Её уроки отличались чёткой структурой, интересными заданиями и использованием интерактивных методов обучения. Учащиеся проявляли активность на уроках, благодаря умелому подходу к подаче материала и вниманию к каждому ученику. Александра Сергеевна продемонстрировала умение создать благоприятную атмосферу на уроке, что способствовало активному вовлечению школьников в образовательный процесс.

Кроме того, Александра Сергеевна проявила себя как творческий организатор. Она занималась съёмкой школьных мероприятий и подготовкой коротких видеороликов для социальных сетей школы. Её работы помогли не только освещать важные события школьной жизни, но и повысить интерес учеников и родителей к мероприятиям. Съёмки и монтаж видеороликов были выполнены на высоком уровне, что заслужило высокую оценку со стороны педагогического коллектива.

На протяжении всего периода практики Александра Сергеевна продемонстрировала дисциплинированность, организованность и креативность в работе. Она всегда проявляла инициативу и предлагала новые идеи для улучшения образовательного процесса. Её общение с учениками и коллегами было уважительным и доброжелательным.

\vspace{5mm}

\textbf{Итоги практики:} За время педагогической практики Александра Сергеевна успешно справилась с возложенными на неё обязанностями. Её творческий подход, ответственность и высокий уровень профессионализма позволяют рекомендовать оценку за практику — \textbf{«отлично»}.

\vspace{10mm}

\noindent
Дата \underline{\hspace{5cm}}
\begin{tabbing}
\hspace{5cm} \= \kill
Классный руководитель \> \underline{\hspace{5cm}} Федорова Елена Андреевна
\end{tabbing}

\vspace{5mm}

\end{document}
